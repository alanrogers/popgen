% -*-latex-*-
\reversemarginpar
\chapter{Gene Genealogies}
\label{ch.coalescent} 

The coalescent process \citep{Kingman:SPA-13-235, Hudson:OSE-7-1}
describes the ancestry of a sample of genes.  As we trace the ancestry
of each modern gene backwards from ancestor to ancestor, we
occasionally encounter common ancestors---genes whose descendants
include more than one gene in the modern sample.  Each time this
happens, the number of ancestors shrinks in size.  Eventually, we
reach the gene that is ancestral to the entire modern sample, and the
process ends.

Since the mid-1980s, this model has revolutionized our understanding
of the effects of genetic drift and mutation.  Many of the results
obtained this way have been entirely new.  Others have merely
confirmed results that were obtained long before.  Either way, the
coalescent model provides a method of studying drift and migration
that is far easier than the methods that geneticists used to use.
We begin with a few mathematical tricks, which will be useful later.

\section{Preliminaries}

\begin{trick}
If the hazard of death is $h$ per day, then the expected life-span is
$1/h$ days.
\label{trick.1/h}
\end{trick}
For concreteness, suppose that we are talking about the life-span of a
piece of kitchen glassware.  Eventually, someone will drop it and it
will break.  Suppose that the hazard of breakage is $h$ per day
and its expected lifespan is $T$ days. Trick~\ref{trick.1/h} tells us
that $T=1/h$. 

We are envisioning time here as a continuous variable and assuming
that the glass may break at any instant. It makes no sense to talk
about the probability of breakage at a particular instant, because
that has to be zero. Instead, $h$ is a \emph{probability density}.
(See \emph{Just Enough Probability}.)  Specifically, it is the density
that the glass will break at a particular instant given that it has
not broken already. This sort of density is often called a
\emph{hazard}, and we are assuming that the hazard does not
change. This implies that the lifespan $(t)$ is a random variable
whose probability distribution is \emph{exponential}.  The mean of
this distribution is $1/h$, as shown in appendix~\ref{sec.hCalculus}.
In the exercise below, you will derive this formula for the case in
which time is discrete.

\begin{exercise}
Trick~\ref{trick.1/h} refers to the case in which time is continuous,
but the result also holds when time is discrete. For example, suppose
you are tossing a glass into the air and then catching it. On each
toss, there is a probability $h$ that you will drop the glass and
break it. How many times, on average, can you toss the glass before
dropping it? The answer is $1/h$, just as in Trick~\ref{trick.1/h}.
In this exercise, you will derive this formula. To do so, consider
what happens on the first toss. Either you drop it or you catch it. If
you drop it (probability $h$), it breaks, and its lifespan is 1
toss. The first component of $T$ is therefore $h \times 1$. If you
catch it (probability $1-h$), the expected lifespan is $1+T$. Why?
Because $h$ doesn't change. Our 1-toss-old glass can expect to survive
$T$ additional tosses, so its expected lifespan is $1+T$.  The second
component of $T$ is therefore $(1-h)(1+T)$. Write down an equation
saying that $T$ is the sum of these two components, and then solve
that equation for $T$.
\answer The equation is $T = h + (1-h)(1+T)$. Re-arranging this
expression gives $T = 1/h$.
\end{exercise}

\begin{trick}
\label{trick.pairs}
There are $k(k-1)/2$ ways to choose 2 items out of $k$.
\end{trick}
There are $k$ ways to choose the first item.  Having chosen the first,
there are $k-1$ ways to choose the second, so there are $k(k-1)$
pairs.  But this counts pair $AB$ separately from $BA$.  We are
interested in unordered pairs, so the number is $k(k-1)/2$.

\section{Coalescence time in a sample of two genes}

The genealogy of two genes, X and Y, is shown in
figure~\ref{fig.2gene}.  Genes X and Y live in the present generation,
and their common ancestor A lived $t$ generations ago.  Consequently,
as we look backward from the present into the past, the two lines of
descent remain distinct for $t$ generations, at which time they
\emph{coalesce} into a single line of descent.  In a given generation,
the lines coalesce if the two genes in that generation are copies of a
single parental gene in the generation before.  Otherwise, the two
lines remain distinct.

\begin{figure}[t]
\begin{verbatim}
   X -----------------------|
                            A------
   Y -----------------------|
    
   |--------- t ------------|

\end{verbatim}
\caption{Coalescence of a sample of two genes\label{fig.2gene}}
\end{figure}

What can we say about the length of time, $t$, that they remain distinct?  The
problem is a lot like the one above involving kitchen glassware.  If we knew
the hazard, $h$, that the lines of descent will coalesce during a generation,
then trick~\ref{trick.1/h} would tell us immediately the mean number of
generations until the two lineages coalesce.

\begin{figure}[t]
\centering
\begin{alltt}
               ________Population__________
Generation 0:  0  Z  0  0  0  0  0  0  0  0
                  |
Generation 1:  0  X  0  Y  0  0  0  0  0  0
\end{alltt}
\caption{A sample of two genes (X and Y) in a population of
size~10.  Gene Z is the parent of gene X.\label{fig.2samp}} 
\end{figure}

Consider the tiny population shown in figure~\ref{fig.2samp}.  Each
row illustrates the population in a single generation, and within each
row each character represents a gene.  The generations are numbered
backwards, so that generation~0 is the present and generation~1
contains the parents of generation~0.  The vertical line indicates
that gene Z is the parent of gene X.  What is the probability that it
is also the parent of gene Y?  If each gene in generation~1 is equally
likely to be Y's parent, then this probability is
\[
h = 1/10
\]
since there are 10 genes in generation~1.  This answer would be the
same no matter which gene in generation~1 had been X's parent.

Trick~\ref{trick.1/h} immediately tells us that the mean coalescence
time is 10 generations.  Of course, 10 is really the number of genes
in the population.  If there are $2N$ genes in the population, then
\begin{equation}
h = 1/2N
\label{eq.h}
\end{equation}
Now the answer becomes somewhat more interesting.  The average pair of
genes last shared a common ancestor \label{pg.mpd} $2N$ generations
ago.  This provides a connection between population size and the
genealogy of genes.  As we shall see in lecture~\ref{ch.addmut}, this
connection lets us use genetics to study the history of population
size.

In equation~\ref{eq.h}, the symbol $N$ is a little confusing.  If we
are talking about an autosomal locus, then there are two genes for
every person and $N$ is the number of people in the population.  The
meaning of $N$ is different, however, if we are talking about a
mitochondrial locus.  In that case, the gene is transmitted only
through women, and locus is effectively haploid.  Consequently, the
number $(2N)$ of genes is the number of females in the population, and
the symbol $N$ represents half the number of females.

\begin{example}
Suppose that we somehow knew that the average pair of mitochondrial
genes last shared a common ancestor 100,000 years ago.  What would
this imply about population size?  (Ignore the issue of statistical
error.) 
\end{example}
\begin{showanswer}
100,000 years is about 4000 generations, so the assumption
implies that
\[
2N = 4000
\]
Since we are talking about a mitochondrial gene, this is really the
number of women.  If there are as many men as women, then the
population would contain 8000 individuals.  This is about the size of
a large village or a very small town.
\end{showanswer}

\section{Coalescence times in a sample of $K$ 
genes\label{sec.ncoalescent.1}}

Now consider a sample of $K$ genes.  (Figure~\ref{fig.4gene} shows the
case in which $K=4$.)  As we move backwards in time from the present,
the first coalescent event that we encounter reduces our sample from
$K$ to $K-1$, the second from $K-1$ to $K-2$, and so on.  After $K-1$
coalescent events, only a single lineage is left and no further
coalescent events can occur.  There are thus $K-1$ intervals to
consider.  The first (i.e.\ the most recent) interval is the one in
which there are $K$ lines of descent.  This interval is $t_K$
generations.  The next interval has $K-1$ lines of descent and is
$t_{K-1}$ generations long.  The last interval is the one with two
lines of descent and is $t_2$ generations long.  Since the length of
each interval is independent of all the other lengths, we can consider
them one at a time.

\begin{figure}
\begin{verbatim}
   W --------|                               
             |------------|                  
   X --------|            |                  
                          |--------------|   
   Y ---------------------|              |---
                                         |   
   Z ------------------------------------|   
   
   |-- t_4 --|---- t_3 ---|------ t_2 ---|
\end{verbatim}
\caption{Coalescence of a sample of four genes\label{fig.4gene}}
\end{figure}

Consider a generation during which the sample has $i$ genes.  By
trick~\ref{trick.pairs} there are $i(i-1)/2$ pairs of genes.  For any given
pair, the probability that the two genes are copies of the same parental gene
is equal to $1/2N$.  Consequently, we might expect the probability of a
coalescent event to be close to
\begin{equation}
h_i = \frac{i(i-1)}{4N}
\label{eq.hi}
\end{equation}
per generation.  Although this argument is loose, it turns out that the result
is correct.  (To find out why, see section~\ref{sec.coalescent.hard}.)  The
expected length of this interval is given by trick~\ref{trick.1/h} and equals
\begin{equation}
1/h_i = \frac{4N}{i(i-1)}
\label{eq.hinv}
\end{equation}

For example, in a sample of size 5, the four coalescent intervals have 
hazards and expected lengths as follows:\\
\begin{center}
\def\mystrut{\rule{0pt}{1em}}
\begin{tabular}{ccc}
         & Coalescent & Expected\\
Interval & hazard     & length\\ \hline
5 & $h_5 = \displaystyle \frac{\mystrut 5 \times 4}{4N} = 10/2N$ & $2N/10$\\
4 & $h_4 = \displaystyle \frac{\mystrut 4 \times 3}{4N} = 6/2N$  & $2N/6$\\
3 & $h_3 = \displaystyle \frac{\mystrut 3 \times 2}{4N} = 3/2N$  & $2N/3$\\
2 & $h_2 = \displaystyle \frac{\mystrut 2 \times 1}{4N} = 1/2N$  & $2N$\\[7pt] 
\hline
\end{tabular}
\end{center}

\section{The depth of a gene tree\label{sec.treedepth}}

The \emph{depth} of a gene tree is the time (usually in generations)
since the \emph{Last Common Ancestor} (LCA) of all the genes in the
sample.  The tree's depth is simply the sum of its coalescent
intervals, and we already have a formula for the expected length of
each interval.  For example, in a sample 2 genes, the expected depth
of the gene tree is $2N$ generations.  In a sample of 3 genes it is
$2N + 2N/3 = 8N/3$ generations.  Here are a few more examples:
\begin{center}
\begin{tabular}{cr@{$\;=\;$}l}
Sample & \multicolumn{2}{c}{Mean depth} \\
size   & \multicolumn{2}{c}{of tree} \\ \hline
2 & $1/h_2$ & $2N$\\ 
3 & $1/h_3 + 1/h_2$ & $8N/3$\\
4 & $1/h_4 + 1/h_3 + 1/h_2$ & $3N$\\
5 & $1/h_5 + 1/h_4 + 1/h_3  + 1/h_2$ & $16N/5$\\ \hline
\end{tabular}
\end{center}
Notice that in each case, the mean tree depth is equal to
\begin{equation}
4N(1 - 1/K)
\label{eq.treedepth}
\end{equation}
where $K$ is the number of gene copies in the sample.  This formula
turns out to be true in general, as shown in Box~\ref{box.Edepth}
\citep[p.~132]{Tavare:TPB-26-119}. In a large sample, the $1/K$ term
is unimportant and the answer is even simpler: the average depth of a
gene tree is approximately $4N$ generations.

\begin{Box}[tbp]
The coalescent interval containing $i$ lineages has expected depth
$1/h_i = 4N/i(i-1)$, so the total expected depth in a sample of $K$
gene copies is
\[
4N\sum_{i=2}^K 1/i(i-1)
\]
This sum is easy to simplify once you notice that
\[
\frac{1}{i(i-1)} = \frac{1}{i-1} - \frac{1}{i}
\]
With this substitution, the series becomes
\[
4N\left( \frac{1}{1} - \frac{1}{2} + \frac{1}{2} - \cdots 
- \frac{1}{K-1} + \frac{1}{K-1} - \frac{1}{K} \right)
\]
Adjacent terms cancel, and we are left with
equation~\ref{eq.treedepth} \citep[p.~132]{Tavare:TPB-26-119}.
\caption{The expected depth of a gene genealogy}
\label{box.Edepth}
\end{Box}

\begin{exercise}
Box~\ref{box.Edepth} uses the fact that
\[
\frac{1}{i(i-1)} = \frac{1}{i-1} - \frac{1}{i}
\]
Verify that this is true by deriving the left side from the
right.
\answer
Beginning with the right side of the expression, we end up with the
left side:
\[
\frac{1}{i-1} - \frac{1}{i} = \frac{i}{i(i-1)} -
\frac{i-1}{i(i-1)}
= \frac{1}{i(i-1)}
\]
\end{exercise}

\begin{exercise}
Suppose that we draw a sample of size $K=10$ from a population with $2N =
5000$ genes.  What are the expected lengths (in generations) of all
the coalescent intervals? 
\answer
The intervals are:\\
\begin{tabular}{cc}
$i$ & $4N/(i(i-1))$\\\hline
10 & 111.1111111\\
 9 & 138.8888889\\
 8 & 178.5714286\\
 7 & 238.0952381\\
 6 & 333.3333333\\
 5 & 500.0000000\\
 4 & 833.3333333\\
 3 & 1666.666667\\
 2 & 5000.000000\\               
\end{tabular}
\end{exercise}

\begin{exercise}
In a sample of 10,000 genes, what is the expected age of the LCA?
What fraction of this age is accounted for by the interval during
which the tree contained only two lineages?
\answer
Since the sample is large, the mean age of the LCA is close to $4N$
generations.  Half of this period is accounted for by the interval
during which the tree contained only two lineages.
\end{exercise}

\begin{figure}
\begin{verbatim}
-|                                                          
 |--|                                                       
-|  |                                                       
    |---|                                                   
---||   |                                                   
   ||   |                                                   
---|    |---|                                               
        |   |                                               
--------|   |-------------------------------------------|   
            |                                           |   
------------|                                           |   
                                                        |   
------|                                                 |   
      |                                                 |   
---|  |-------|                                         |   
   |--|       |                                         |---
---|          |------|                                  |   
              |      |                                  |   
-|            |      |                                  |   
 |------------|      |                                  |   
-|                   |                                  |   
                     |                                  |   
-------------|       |----------------------------------|   
             |       |                                      
-|           |-|     |                                      
 |-----------| |     |                                      
-|             |     |                                      
               |     |                                      
|              |     |                                      
|---|          |-----|                                      
|   |          |                                            
    |-|        |                                            
----| |        |                                            
      |        |                                            
-|    |--------|                                            
 ||   |                                                     
-||   |                                                     
  |---|                                                     
--|                   
\end{verbatim}
\caption{Coalescence of a sample of twenty genes\label{fig.20gene}}
\end{figure}

When the sample is large, $K(K-1)/2$ is a large number.  Consequently,
initial coalescent intervals tend to be short.  In a sample of size
20, the most recent coalescent interval is (on average) 0.5 percent as
long as the interval that ends with the root.  Figure~\ref{fig.20gene}
shows an example.  Note the short terminal branches and the deep basal
branch. 

\begin{subappendices}
\section{A more detailed treatment (optional)}
\label{sec.coalescent.hard} 

\subsection{Preliminaries}

Before explaining the formula for the general case---that of a
coalescent interval during which the sample has $K$ genes---we need
one additional mathematical trick.

\begin{trick}
The sum of the numbers from 1 through $k$ is $k(k+1)/2$.
\label{trick.sum1tok}
\end{trick}
This trick was supposedly discovered by the mathematician Carl
Friedrich Gauss when he was just six years old.  According to the
story, Gauss's teacher gave the class an assignment to keep it busy
while he graded papers: Sum the numbers from 1 through 100.  Two
minutes later, Gauss walked to the front of the room with his answer.
The answer was correct, but Gauss was punished for failing to do the
work the hard way.  Here is how he did it.

First he wrote down
\[
1 + 2 + \cdots + 99 + 100
\]
Then, being bored and discouraged, he wrote it out backwards just
below:
\[
\begin{array}{ccccccccc}
1 &+& 2 &+& \cdots &+& 99 &+& 100 \\
100 &+& 99 &+& \cdots &+& 2 &+& 1
\end{array}
\]
Then the insight struck---he noticed that each of the columns added to
101: 
\[
\begin{array}{ccccccccc}
1 &+& 2 &+& \cdots &+& 99 &+& 100 \\
100 &+& 99 &+& \cdots &+& 2 &+& 1\\ \hline
101 &+& 101 &+& \cdots &+& 101 &+& 101
\end{array}
\]
Since there are 100 columns, the sum of all the numbers here is $100
\times 101$.  And this is twice the sum that he was looking for.
Thus,
\[
1 + 2 + \cdots + 99 + 100 = \frac{100 \times 101}{2}
\]
In the general case,
\[
1 + 2 + \cdots + k = \frac{k (k+1)}{2}
\]

\begin{trick}
$e^{x}$ is approximately $1+x$ when $x$ is small.
\label{trick.exp}
\end{trick}
Here $e^x$ is the exponential function, and is also written 
$\exp(x)$.  You can verify the trick with a calculator.

\subsection{Coalescence times in an interval with three genes}

Figure~\ref{fig.3gene} shows the genealogy of a sample of three genes.
It has two coalescent events, one at node $A$ (the root) and another
at node $B$.  The time between the present and the root can be broken
into two intervals of length $t_2$ and $t_3$, where $t_2$ is the
number of generations during which the genealogy has two lines of
descent and $t_3$ is the number of generations during which it had
three.  What can we say about the lengths of these intervals?

\begin{figure}
\begin{verbatim}
   X --------------------|                   
                         B---------------|   
   Y --------------------|               |   
                                         A---
   Z ------------------------------------|
   
   
   |--------- t_3 -------|------ t_2 ----|  
\end{verbatim}
\caption{Coalescence of a sample of three genes\label{fig.3gene}}
\end{figure}

The first point to notice is that the intervals are independent.  As
we ponder the length of one interval, we need not worry about the
length of the other.  And we already know the mean of $t_2$: the
preceding section showed that the mean coalescence time for a sample
of two genes is $2N$ generations.

This leaves us with only one question to answer:  What is the mean
time until the first coalescent event in a sample of three genes?  We
could answer this question using trick~\ref{trick.1/h} if we knew the
hazard of a coalescent event in a sample of that size.  Let us
therefore consider the probability that a coalescent event occurs
during some given generation.

It is easier to calculate first the probability of the event that all
three lines of descent remain distinct.  This requires that
\begin{enumerate}
\item
X and Y are copies of different parental genes.  We already know that
this event has probability $1 - 1/2N$.
\item
Z is neither a copy of X's parent nor a copy of Y's parent.  This event
has probability $(2N - 2)/2N$.  (Of the $2N$ genes that we can choose
between, 2 produce a coalescent event and $2N-2$ do not.)  This
probability can also be written as $1 - 2/2N$.
\end{enumerate}
Thus, the probability that no coalescent event occurs in some
particular generation is equal to
\[
1-h = (1 - 1/2N)(1 - 2/2N)
\]
Now it is time to invoke trick~\ref{trick.exp}.  If the population is
large, then $2N$ will be a large number and $1/2N$ and $2/2N$ will both
be small.  Trick~\ref{trick.exp} thus allows the probability above to
be re-expressed as
\[
1-h \approx e^{-1/2N} e^{-2/2N} = e^{-3/2N}
\]
Now invoke trick~\ref{trick.exp} once again to simplify the exponential:
\begin{eqnarray*}
1-h &\approx& 1 - 3/2N\\
h & \approx & 3/2N
\end{eqnarray*}
Having found the hazard of a coalescent event in a sample of three
genes, trick~\ref{trick.1/h} now gives us the mean length of the part
of the genealogy during which there were three lines of descent:
\[
\hbox{mean of}\quad t_3 = 2N/3
\]

In a sample of three genes, the hazard of a coalescent event is three
times as large as the hazard in a sample of two.  Consequently, the
mean waiting time until the first coalescent event is only 1/3 as
large.  The expected depth of the tree is the expected sum of $t_2$
and $t_3$.  It equals $2N + 2N/3$, or $8N/3$.  Three quarters of this
total is taken up by the portion of the genealogy during which there
are only two lines of descent.

\begin{example}
In a population of $10^7$, what is the mean time in years until a
sample of three mitochondrial genes coalesce to a single line of
descent. 
\end{example}
\begin{showanswer}
  If there are $10^7$ people, there will be about half that many
  females, so $2N = 5\times 10^6$.  The coalescence time is $8N/3 =
  6.67 \times 10^6$ generations.  If generations are 25 years long,
  this is $167 \times 10^6$ years.  So the Last Common Ancestor (LCA)
  should have lived during the Jurassic period. Incidentally, this
  example is far-fetched for humans, because it implies far more
  mitochondrial variation than really exists.
\end{showanswer}

\subsection{Coalescence times in an interval with $i$ genes}

As in the case of three lines of descent, it is easiest to calculate
$1-h$, the probability that no coalescent event occurs during some
particular generation.  When there are $i$ genes in the sample, this requires
\begin{center}
\begin{tabular}{lc}
\multicolumn{1}{c}{Event} & Probability\\ \hline
Gene~2 and gene~1 have different parents & $1-1/2N$\\
Gene~3's parent differs from the preceding 2 parents
   & $1 - 2/2N$\\
Gene~4's parent differs from the preceding 3 parents
   & $1 - 3/2N$\\
\multicolumn{2}{c}{\dotfill}\\
Gene~$i$'s parent differs from the preceding $i-1$ parents
   & $1 - (i-1)/2N$\\
\hline
\end{tabular}
\end{center}
The probability that a coalescent event does \emph{not} occur is
\[
\begin{array}{rcll}
1-h &=& (1-1/2N)(1-2/2N)\cdots(1-(i-1)/2N)\\
    &\approx& e^{-1/2N}e^{-2/2N}\cdots e^{-(i-1)/2N} 
                             & \hbox{(trick~\ref{trick.exp})}\\
    &=& \exp\left[-\frac{1}{2N}(1 + 2 + \cdots + (i-1))\right]\\
    &=& \exp\left[-\frac{i(i-1)}{4N}\right]
                         & \hbox{(trick~\ref{trick.sum1tok})}\\
    &\approx& 1 - \frac{i(i-1)}{4N}
                             & \hbox{(trick~\ref{trick.exp})}
\end{array}
\]
Thus,
\begin{equation}
h \approx \frac{i(i-1)}{4N}
\label{eq.coalescent.h}
\end{equation}

\section{The mean of an exponential random variable (optional)}
\label{sec.hCalculus}
If $t$ is an exponential random variable, then its density function is
$he^{-ht}$.  The mean of this distribution is:
\[
E[t] = \int_0^\infty ht e^{-ht} dt
\]
Substituting $x=ht$ turns this into
\[
E[t] = h^{-1}\int_0^\infty x e^{-x} dx
\]
On integrating by parts, the integral on the right becomes
\[
\int_0^\infty x e^{-x} dx
  = - xe^{-x}|_0^\infty - \int_0^\infty (-e^{-x}) dx
\]
The first term on the right is 0 and the second is $-e^{-x}|_0^\infty
= 1$.  Thus, $E[t] = 1/h$.
\end{subappendices}
