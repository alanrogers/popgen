% -*-latex-*-
\chapter{Genetic Drift\label{ch.drift}}

%This chapter needs to be merged with chapter~\ref{ch.homozygosity}.

\section{The four causes of evolutionary change}

\begin{enumerate}
\item mutation
\item selection
\item migration (a.k.a.\ gene flow)
\item genetic drift
\end{enumerate}

\section{What is genetic drift?}

\begin{itemize}
\item
It is everything that is left over after you account for the effects
of mutation, selection, and migration.
\item
It consists of all the stochastic (random) effects on allele
frequencies.  These include everything from Mendelian segregation to
the risk of accidentally walking in front of a bus.
\end{itemize}
How can one possibly model such an ill-defined hodgepodge?

\section{The Wright-Fisher model}

The population does not vary in size.  In each generation, it consists
of $N$ individuals, each produced by the union of two randomly chosen
gametes.   

\paragraph{Generating gametes} Each gamete is constructed by the
following algorithm: (1)~choose a parent at random from among the $N$
individuals of the previous generation.  (2)~Choose a random half of
that parent's DNA.  (Don't worry about genetic linkage; we will be
dealing here with one locus at a time.)  If there are two alleles
$A_1$ and $A_1$, segregating at some locus, what is the probability
that the gamete that we construct carries a copy of $A_1$?  If the
parent was an $A_1A_1$ homozygote, then we are bound to get $A_1$ in
the gamete.  If the parent was an $A_1A_2$ heterozygote, then the
gamete has a 50\% chance of carrying $A_1$.  Thus, the algorithm
generates an $A_1$-bearing gamete with probability $p_1 = P_{11} +
P_{12}/2$, where $P_{11}$ and $P_{12}$ are the frequencies of
genotypes $A_1A_1$ and $A_1A_2$ within the parental generation.  Note
that the formula for $p_1$ is exactly the same as the formula for the
frequency of $A_1$ among the parents.  Conclusion: The Wright-Fisher
algorithm for generating gametes is equivalent to drawing genes at
random with replacement from the parental population.  To clarify this
idea, many authors have made use of the urn metaphor.

\paragraph{The urn metaphor} In an urn full of balls, a fraction $p$
of the balls are red and a fraction $1-p$ are black.  Each ball
represents a gene.  The red balls represent copies of one allele; the
black ones copies of another.  The fraction $p$ represents the
frequency of the red allele in the population.  The urn will be used
to produce a new generation in which there are $N$ diploid
individuals, or $2N$ genes.  To produce the new generation, we perform
the following operation $2N$ times: draw a random ball from the urn,
write down its color, and then return the ball to the urn.  The number
of red balls drawn represents the number of copies of the red allele
in the new generation, and similarly for the black balls.  Both
numbers are random variables.  Their probability distribution was
taken by Wright and Fisher as a model of the process of genetic drift.

The Wright-Fisher model is undoubtedly simpler than reality, but it
has been remarkably successful at dealing with the stochastic
variation in real populations.  Let us be content with it, at least
for the moment, and ask about its properties.  In the urn, the
frequency of red balls is $p$.  Let $p'$ denote the frequency of red
balls among those drawn.  The difference between $p'$ and $p$
represents the effect of genetic drift.  How large is this difference
likely to be?

If we repeated the urn experiment over and over, the average value of
$p'$ would get closer and closer to $p$.  Another way to say this is
to say that \emph{the expected value of $p'$ equals $p$}.  In
notation,
\[
E[p'] = p
\]
where the symbol $E$ represents the ``expectation,'' or average.

But unless $N$ is extremely large, there will be some difference
between $p'$ and $p$, so we can write
\[
p' = p + \epsilon
\]
Here, $\epsilon$ (the greek letter ``epsilon'') represents the effect
of genetic drift.  Its expected value is 0, but its variance is%
\footnote{To see where this formula comes from, look up the binomial
  distribution in any text on probability and statistics.}
\[
V[\epsilon] = E[\epsilon^2] = \frac{p(1-p)}{2N}
\]
Genetic drift is important when this variance is large; unimportant
when it is small.  The formula captures two influences:
\begin{enumerate}
\item Drift is unimportant when $p(1-p)$ is near 0.  This happens when 
  $p\approx 0$ and also when $p \approx 1$.
\item Drift is unimportant when $N$ is very large.
\item Drift is most important when $p \approx 1/2$ and $N$ is small.
\end{enumerate}

Show a plot of $p$ against $t$.

% -*-latex-*-
\begin{table}
\caption{Average heterozygosity\label{tab.race}}
\footnotesize
\begin{tabular*}{\textwidth}{l@{\extracolsep{\fill}}cccccccc}
\hline
Pop.
   & Bl.\ grp.$^a$
   & Protein$^b$
   & Classical$^c$
   & RFLP$^d$
   & RSP$^e$
   & STR-4$^f$
   & STR-2$^g$
   & STR-3$^h$
\\
\hline
Africa
  & $0.164$% \pm 0.038$ %Nei's blood groups
  & $0.179$% \pm 0.021$ %Nei's proteins
  & $0.163$% \pm 0.014$ %Bowcock's classical
  & $0.297$% \pm 0.007$ %Bowcock's RFLP
  & $0.322$% \pm 0.073$ %Jorde's RSPs
  &$\bf 0.769$% \pm 0.102$ %Jorde's STR4s
  &$\bf 0.807$% \pm 0.014$ %Bowcock's STR2s
  &$\bf 0.850$% \pm 0.014$ %Watkins's STR3s
\\
Asia
  & $0.145$% \pm 0.038$ %Nei's blood groups
  & $0.164$% \pm 0.022$ %Nei's proteins
  & $0.189$% \pm 0.019$ %Bowcock's classical
  & $0.327$% \pm 0.012$ %Bowcock's RFLP
  & $0.377$% \pm 0.052$ %Jorde's RSPs
  & $0.681$% \pm 0.074$ %Jorde's STR4s
  & $0.685$% \pm 0.021$ %Bowcock's STR2s
  & $0.820$% \pm 0.022$ %Watkins's STR3s
\\
Europe
  &$\bf 0.179$% \pm 0.043$ %Nei's blood groups
  &$\bf 0.186$% \pm 0.220$ %Nei's proteins
  &$\bf 0.202$% \pm 0.017$ %Bowcock's classical
  &$\bf 0.379$% \pm 0.015$ %Bowcock's RFLP
  &$\bf 0.432$% \pm 0.032$ %Jorde's RSPs
  & $0.724$% \pm 0.086$ %Jorde's STR4s
  &$0.730$% \pm 0.016$ %Bowcock's STR2s
  & $0.807$% \pm 0.032$ %Watkins's STR3s
\\
\hline
\end{tabular*}\\
Note: Largest entry in each column is in boldface. Columns are in
order of increasing European heterozygosity.\\ 
$^a$32 blood groups \citep{Nei:GCI-93-239}.\\
$^b$80 protein polymorphisms \citep{Nei:GCI-93-239}.\\ 
$^c$110 classical polymorphisms \citep{Bowcock:N-368-455}.\\ 
$^d$79 restriction fragment length polymorphisms \citep{Bowcock:N-368-455}.\\ 
$^e$30 RFLPs consisting solely of restriction site polymorphisms
\citep{Jorde:AJH-57-523}.\\   
$^f$30 tetranucleotide STRs \citep{Jorde:AJH-57-523}.\\  
$^g$30 dinucleotide short tandem repeat polymorphisms (STRs).
Difference between Africa and Europe is significant
\citep{Bowcock:N-368-455}.\\ 
$^h$5 trinucleotide STRs \citep{Watkins:HMG-4-1485}.\\  
\end{table}


\section{Classical theory of homozygosity and heterozygosity}

Let $J_t$ represent the probability that two genes chosen at random
from some population are copies of the same allele.  If the population 
mates at random, then $J$ will also be the homozygosity.  The gene
diversity (or heterozygosity) is $H = 1-J$.  Several gene diversity
estimates are shown in table~\ref{tab.race}.  These statistics
are affected by several evolutionary forces:
\begin{enumerate}
\item Mutations reduce $J$ because they are more likely to make
  identical genes less similar than to make different genes identical.
\item Genetic drift tends to move allele frequencies toward 0 and 1.
  Consequently $2pq$ gets smaller, heterozygosity declines, and
  homozygosity increases.
\end{enumerate}
To measure the effects of these forces, we need a model.  Let us begin 
with a model dealing only with the first force.

\subsection{Drift only}

Let $J_t$ denote the probability that two genes drawn at random from
the population of generation $t$ are copies of the same allele.  There 
is a simple model that relates $J$ in one generation to its value in
the generation before:
\[
J_{t+1} = \frac{1}{2N} + \left(1 - \frac{1}{2N}\right) J_t
\]
The first term on the right accounts for the possibility that the two
genes in generation $t+1$ may be copies of the same gene in generation 
$t$.  Since there are $2N$ genes in the population, the two genes are
copies of the same gene with probability $1/(2N)$ and are copies of
distinct genes with probability $1 - 1/(2N)$.  In the latter case,
they are by definition copies of the same allele with probability
$J_t$. 

This model says that each generation's value of $J$ is a weighted
average of 1 and the previous value of $J$.  Consequently, $J$
converges toward 1.  We are eventually left with no heterozygotes at
all.

\subsection{Drift plus mutation}

To make the model interesting, we need to add in some other
evolutionary force.  Let us add in mutation.  The easiest way to do
this employs the model of ``infinite alleles,'' which assumes that
every mutation produces an allele that has never existed before.
Thus, two genes can be identical only if there has been no mutation
along the evolutionary path that connects them.  In particular, there
can have been no mutation during the past generation in the path
leading to either of our two genes.  If $u$ is the mutation rate per
generation, then $1-u$ is the probability that no mutation occurs
along a single evolutionary path during a generation, and $(1-u)^2$ is 
the probability that neither of our two genes has mutated in the past
generation.  Thus,
\[
J_{t+1} = (1-u)^2 \left(
\frac{1}{2N} + \left(1 - \frac{1}{2N}\right) J_t
\right)
\]
Population geneticists are not very accurate people and tend to ignore
whatever they can.  Consider the following:
\begin{center}
\begin{tabular}{ccc}
$u$ & $(1-u)^2$ & $1 - 2u$ \\ \hline
0.00100 & 0.9980010000 & 0.99800\\ 
0.00010 & 0.9998000100 & 0.99980\\
0.00001 & 0.9999800001 & 0.99998\\ \hline
\end{tabular}
\end{center}
The smaller the value of $u$, the less the difference between
$(1-u)^2$ and $1-2u$.  Since mutation rates are very small numbers,
population geneticists never trouble themselves about the difference
between $(1-u)^2$ and $1-2u$.  In the present case, $1-2u$ makes the
algebra simpler.  Similarly, if $u$ is small and $N$ is large,
$(1-2u)/(2N)$ is hardly different from $1/(2N)$.  
Applying both simplifications gives
\[
J_{t+1} \approx \frac{1}{2N} + \left(1-2u - \frac{1}{2N}\right) J_t
\]
This equation doesn't look very different from the one with drift
only, but it leads to a very different conclusion.  Rather than
converging toward unity, this one levels out at a different
equilibrium value, as shown in figure~\ref{fig.homoz}.

% -*-latex-*-
\begin{figure}
\begin{center}
\mbox{\beginpicture
\setcoordinatesystem units <0.003in, 50in>
\setplotarea x from 0 to 1000, y from 0 to 0.04
\axis left label {$H$}
  ticks numbered from 0.00 to 0.04 by 0.01 /
\axis right /
\axis top /
\axis bottom label {Time in generations} 
  ticks numbered from 0 to 1000 by 100 /
\plot
0 0 25 .004433284759 50 .007864214817 75 .01051942009
100 .01257429003 125 .01416455897 150 .01539527204 175 .01634772392
200 .01708482878 225 .01765527610 250 .01809674668 275 .01843840181
300 .01870280958 325 .01890743544 350 .01906579590 375 .01918835145
400 .01928319748 425 .01935659909 450 .01941340476 475 .01945736680
500 .01949138914 525 .01951771907 550 .01953809589 575 .01955386557
600 .01956606978 625 .01957551464 650 .01958282403 675 .01958848078
700 .01959285855 725 .01959624654 750 .01959886849 775 .01960089762
800 .01960246799 825 .01960368332 850 .01960462385 875 .01960535173
900 .01960591501 925 .01960635096 950 .01960668836 975 .01960694945
1000 .01960715152
/
\endpicture}
\end{center}
\caption{How heterozygosity changes over time.
Assumes: $u=0.005$, $N=2500$, H(0)=0\label{fig.homoz}}
\end{figure}


To find the equilibrium algebraically, set $J_{t+1} = J_t$ and solve
the resulting equation.  The result is
\begin{equation}
J = \frac{1}{4Nu + 1}
\label{eq.homozygosity}
\end{equation}

The gene diversity (or heterozygosity) is
\begin{equation}
H = 1-J = \frac{4Nu}{4Nu + 1}
\label{eq.H}
\end{equation}

\subsection{A simpler way: coalescent theory}

Take a random pair of genes and peer backwards down their ancestries.
So long as the two evolutionary paths remain distinct, two types of
event may happen in any given generation:
\begin{description}
\item[A mutation] A mutation may occur in either path with probability
  $u$.  The combined probability in both paths is $2u$.%
%%%%%%%%%%%%%%%%%%%%%%%%%%%%%%%%%%%%%%%%%%%%%%%%%%%%%%%%%%%%%%%%
\footnote{This is only an approximation.  If I really wanted to be
  accurate, I would say that $(1-u)^2$ was the probability of no
  mutation along either of the two paths and $1 - (1-u)^2$ the
  probability of at least one mutation.  But when $u$ is small this
  latter probability is indistinguishable from $2u$.}
%%%%%%%%%%%%%%%%%%%%%%%%%%%%%%%%%%%%%%%%%%%%%%%%%%%%%%%%%%%%%%%%
\item[A coalescent event] The two paths will coalesce when we reach
  the most recent generation in which they share a common ancestor.
  This is an event with probability $1/(2N)$.
\end{description}
The hazard of an event of either type is
\[
2u + 1/(2N)
\]
When an event does occur, it is a mutation with probability
\[
\frac{2u}{2u + 1/(2N)} = \frac{4Nu}{4Nu + 1}
\]
In this case, the two genes are copies of different alleles.  The
formula gives the probability that two random genes will be copies of
different alleles---the gene diversity.  Notice that it is identical
to equation~\ref{eq.H}.

The simplicity of this approach is remarkable.  It has led to profound
changes in population genetics during the past decade or two.  We will
return to it in the section on gene genealogies.

\begin{exercise}
For classical polymorphisms, human gene diversity is roughly
0.16.  What does this imply about the quantity $4Nu$?  (In the
literature, $4Nu$ is often denoted by $\theta$, the greek letter
``theta'').
\answer
Expected heterozygosity equals 0.16 when $\theta \approx 0.19$.
\end{exercise}

\begin{exercise}
If the mutation rate were $10^{-6}$, what value of $N$ would be
needed to account for this level of heterozygosity?
\answer
If $\theta = 0.19$ then $N = 47,500$.
\end{exercise}

\begin{exercise}
Using these same values for $N$ and $u$, suppose that $H$ were equal
to 0.01 in generation 0.  What would its value be in generation 20?
\answer
In generations 0 through 20 the heterozygosity is:\\
\begin{tabular}{rl}
Generation & Heterozygosity \\ \hline
0 & 0\\
1 & .00001052631579\\
2 & .00002105249972\\
3 & .00003157855180\\
4 & .00004210447203\\
5 & .00005263026041\\
6 & .00006315591694\\
7 & .00006315591694\\
7 & .00007368144162\\
8 & .00008420683445\\
9 & .00009473209544\\
10 & .0001052572246\\
11 & .0001157822219\\
12 & .0001263070874\\
13 & .0001368318210\\
14 & .0001473564228\\
15 & .0001578808928\\
16 & .0001684052309\\
17 & .0001789294372\\
18 & .0001894535117\\
19 & .0001999774543\\
20 & .0002105012651\\ \hline
\end{tabular}
\end{exercise}

\begin{exercise}
Using the same value for $N$, plot the variance of $\epsilon$ for
values of $p$ ranging from 0 through 1.
\answer
A few representative values are:\\
\begin{tabular}{cc}
$p$ & $p(1-p)/(2N)$\\ \hline
0 & 0\\
.25 & $0.1973684211 \times 10^{-5}$\\
.50 & $0.2631578948 \times 10^{-5}$\\
.75 & $0.1973684211 \times 10^{-5}$\\
1.00 & 0\\                               
\end{tabular}
\end{exercise}

\begin{exercise}
The square root of the variance is called the standard deviation and
(in this case) provides an estimate of the magnitude of a typical
value of $\epsilon$.  For what value of $p$ is this standard deviation 
largest?  How large is it at this value of $p$?
\answer
At $p=1/2$, $\sqrt(p(1-p)/(2N)) \approx 0.0016$ 
\end{exercise}

\begin{exercise}
Figure~\ref{fig.homoz} assumed that $u=0.005$ and $N=2,500$.  Under
these assumptions, what is the equilibrium value of $H$?  Is it
consistent with the figure?
\answer
In the figure, $u=0.005$ and $N=2500$, so $\theta = 50$, the
equilibrium value of heterozygosity is 0.98, and the equilibrium value 
of homozygosity is 0.02.  This is just what the figure shows.
\end{exercise}

%%% Local Variables: 
%%% mode: latex
%%% TeX-master: t
%%% End: 
