% -*-latex-*-
\chapter{Microsatellites\label{ch.str}}

\section{Repeat polymorphisms: Nomenclature}
\begin{description}
\item[tandem repeat polymorphisms] A particular sequence is repeated
again and again.  For example, if the repeated sequence is AT, then
one chromosome might have a sting of three repeats (ATATAT) and
another might have a string of five (ATATATATAT).  They are called
``tandem'' repeats because the repeated units are stuck together
end-to-end.  There are several categories of repeat polymorphism:
\begin{description}
\item[Minisatellite] the repeated unit is fairly large
\item[Microsatellite] the repeated unit is small, commonly 2--6
nucleotides.  
\item[Short tandem repeat (STR)] a synonym for ``microsatellite''
\end{description}
\item[dispersed repeat polymorphisms] These also consist of a sequence
that is repeated again and again, but in these polymorphisms the
repeated units are adjacent to one another.   They are not even close
together.  They are widely dispersed within the genome.  The \emph{Alu
polymorphisms} are an example of this class of polymorphism.
\end{description}

\section{Properties relevant to statistical analysis}

\begin{description}
\item[Rapid evolution] Mutation rate in the neighborhood of 1/1000 per
  generation \citep{Weber:HMG-2-1123}. 
\item[Di- and tetra-nucleotide repeats are likely to be neutral] A
  tetranucleotide repeat doesn't work very well within a reading
  frame, because each mutation produces a frame-shift mutation, which
  probably has catastrophic consequences.  Most of the di- and
  tetra-nucleotide repeats that we find are outside of reading frames.
\item[Tri-nucleotide repeats] are known within reading frames, and
  some cause genetic disease. 
\item[There are lots of microsatellite loci]
\item[Mutational process poorly understood]  Statistical analysis
  usually assumes that mutations add or subtract one repeat unit with
  equal probability.  The is almost certainly too simple.
\end{description}

\section{A sample of 60 STR loci}

Figure~\ref{fig.str} shows the frequency distributions of repeat
counts at a sample of 60 tetra-nucleotide repeat loci described by
Jorde and his colleagues \citep{Jorde:AJH-57-523, Jorde:PNA-94-3100}.
Each horizontal axis there shows the number of copies of the repeat
unit, and each vertical axis shows the number of chromosomes in the
sample that exhibited that number of copies.

% -*-latex-*-
%% Definitions:
\def\hunit{0.0341in}
%%Plots are 0.75 X 0.583333 inches
\begin{figure}
\begin{center}
\footnotesize
\mbox{%
\beginpicture
\headingtoplotskip=0.5\baselineskip
\valuestolabelleading=0.4\baselineskip
% row 0 col 0 xpoint=0.000000 ypoint=0.000000
\def\vunit{0.003038in}%
%%%%%%%%%%%% Plot figure in row 0 col 0 %%%%%%%%%%%%
\setcoordinatesystem units <\hunit,\vunit> point at 0.000000 0.000000
\setplotarea x from 0 to 22, y from 0 to 192
\axis bottom / %ticks numbered from 0 to 22 by 22 /
\axis left /
\axis top /
\axis right /
%locus   0:
\multiput {\tiny $\circ$} at 0 0 1 4 2 4 3 4 4 45
 5 116 6 192 7 91 8 18 9 2
 10 0 11 0 12 0 13 0 14 0
 15 0 16 0 17 0 18 0 19 0
 20 0 21 0
/
% row 0 col 1 xpoint=-26.400000 ypoint=0.000000
\def\vunit{0.002061in}%
%%%%%%%%%%%% Plot figure in row 0 col 1 %%%%%%%%%%%%
\setcoordinatesystem units <\hunit,\vunit> point at -26.400000 0.000000
\setplotarea x from 0 to 22, y from 0 to 283
\axis bottom / % ticks numbered from 0 to 22 by 22 /
\axis left /
\axis top /
\axis right /
%locus   1:
\multiput {\tiny $\circ$} at 0 0 1 0 2 4 3 11 4 59
 5 283 6 13 7 9 8 30 9 10
 10 25 11 6 12 0 13 0 14 0
 15 0 16 0 17 0 18 0 19 0
 20 0 21 0
/
% row 0 col 2 xpoint=-52.799999 ypoint=0.000000
\def\vunit{0.005503in}%
%%%%%%%%%%%% Plot figure in row 0 col 2 %%%%%%%%%%%%
\setcoordinatesystem units <\hunit,\vunit> point at -52.799999 0.000000
\setplotarea x from 0 to 22, y from 0 to 106
\axis bottom / % ticks numbered from 0 to 22 by 22 /
\axis left /
\axis top /
\axis right /
%locus   2:
\multiput {\tiny $\circ$} at 0 0 1 4 2 1 3 83 4 81
 5 106 6 86 7 81 8 27 9 1
 10 0 11 0 12 0 13 0 14 0
 15 0 16 0 17 0 18 0 19 0
 20 0 21 0
/
% row 0 col 3 xpoint=-79.199997 ypoint=0.000000
\def\vunit{0.003007in}%
%%%%%%%%%%%% Plot figure in row 0 col 3 %%%%%%%%%%%%
\setcoordinatesystem units <\hunit,\vunit> point at -79.199997 0.000000
\setplotarea x from 0 to 22, y from 0 to 194
\axis bottom / % ticks numbered from 0 to 22 by 22 /
\axis left /
\axis top /
\axis right /
%locus   3:
\multiput {\tiny $\circ$} at 0 0 1 1 2 1 3 25 4 194
 5 33 6 51 7 133 8 31 9 1
 10 0 11 0 12 0 13 0 14 0
 15 0 16 0 17 0 18 0 19 0
 20 0 21 0
/
% row 0 col 4 xpoint=-105.599998 ypoint=0.000000
\def\vunit{0.003838in}%
%%%%%%%%%%%% Plot figure in row 0 col 4 %%%%%%%%%%%%
\setcoordinatesystem units <\hunit,\vunit> point at -105.599998 0.000000
\setplotarea x from 0 to 22, y from 0 to 152
\axis bottom / % ticks numbered from 0 to 22 by 22 /
\axis left /
\axis top /
\axis right /
%locus   4:
\multiput {\tiny $\circ$} at 0 0 1 29 2 30 3 5 4 22
 5 105 6 152 7 99 8 29 9 2
 10 1 11 0 12 0 13 0 14 0
 15 0 16 0 17 0 18 0 19 0
 20 0 21 0
/
% row 0 col 5 xpoint=-132.000000 ypoint=0.000000
\def\vunit{0.006076in}%
%%%%%%%%%%%% Plot figure in row 0 col 5 %%%%%%%%%%%%
\setcoordinatesystem units <\hunit,\vunit> point at -132.000000 0.000000
\setplotarea x from 0 to 22, y from 0 to 96
\axis bottom / % ticks numbered from 0 to 22 by 22 /
\axis left /
\axis top /
\axis right /
%locus   5:
\multiput {\tiny $\circ$} at 0 0 1 7 2 6 3 11 4 77
 5 82 6 96 7 88 8 47 9 34
 10 23 11 3 12 0 13 0 14 0
 15 0 16 0 17 0 18 0 19 0
 20 0 21 0
/
% row 1 col 0 xpoint=0.000000 ypoint=137.800003
\def\vunit{0.006410in}%
%%%%%%%%%%%% Plot figure in row 1 col 0 %%%%%%%%%%%%
\setcoordinatesystem units <\hunit,\vunit> point at 0.000000 137.800003
\setplotarea x from 0 to 22, y from 0 to 91
\axis bottom / % ticks numbered from 0 to 22 by 22 /
\axis left /
\axis top /
\axis right /
%locus   6:
\multiput {\tiny $\circ$} at 0 0 1 0 2 0 3 0 4 5
 5 11 6 30 7 64 8 82 9 91
 10 85 11 53 12 21 13 14 14 10
 15 0 16 2 17 0 18 0 19 0
 20 0 21 0
/
% row 1 col 1 xpoint=-26.400000 ypoint=275.600006
\def\vunit{0.003205in}%
%%%%%%%%%%%% Plot figure in row 1 col 1 %%%%%%%%%%%%
\setcoordinatesystem units <\hunit,\vunit> point at -26.400000 275.600006
\setplotarea x from 0 to 22, y from 0 to 182
\axis bottom / % ticks numbered from 0 to 22 by 22 /
\axis left /
\axis top /
\axis right /
%locus   7:
\multiput {\tiny $\circ$} at 0 0 1 2 2 182 3 1 4 31
 5 5 6 8 7 5 8 32 9 24
 10 19 11 21 12 19 13 15 14 17
 15 20 16 22 17 19 18 10 19 4
 20 5 21 4
/
% row 1 col 2 xpoint=-52.799999 ypoint=371.000000
\def\vunit{0.002381in}%
%%%%%%%%%%%% Plot figure in row 1 col 2 %%%%%%%%%%%%
\setcoordinatesystem units <\hunit,\vunit> point at -52.799999 371.000000
\setplotarea x from 0 to 22, y from 0 to 245
\axis bottom / % ticks numbered from 0 to 22 by 22 /
\axis left /
\axis top /
\axis right /
%locus   8:
\multiput {\tiny $\circ$} at 0 0 1 1 2 245 3 0 4 0
 5 0 6 45 7 7 8 25 9 51
 10 17 11 21 12 17 13 15 14 7
 15 1 16 0 17 0 18 0 19 0
 20 0 21 0
/
% row 1 col 3 xpoint=-79.199997 ypoint=181.714294
\def\vunit{0.004861in}%
%%%%%%%%%%%% Plot figure in row 1 col 3 %%%%%%%%%%%%
\setcoordinatesystem units <\hunit,\vunit> point at -79.199997 181.714294
\setplotarea x from 0 to 22, y from 0 to 120
\axis bottom / % ticks numbered from 0 to 22 by 22 /
\axis left /
\axis top /
\axis right /
%locus   9:
\multiput {\tiny $\circ$} at 0 0 1 0 2 1 3 120 4 10
 5 104 6 68 7 43 8 101 9 20
 10 7 11 0 12 0 13 0 14 0
 15 0 16 0 17 0 18 0 19 0
 20 0 21 0
/
% row 1 col 4 xpoint=-105.599998 ypoint=265.000000
\def\vunit{0.003333in}%
%%%%%%%%%%%% Plot figure in row 1 col 4 %%%%%%%%%%%%
\setcoordinatesystem units <\hunit,\vunit> point at -105.599998 265.000000
\setplotarea x from 0 to 22, y from 0 to 175
\axis bottom / % ticks numbered from 0 to 22 by 22 /
\axis left /
\axis top /
\axis right /
%locus  10:
\multiput {\tiny $\circ$} at 0 0 1 2 2 15 3 15 4 44
 5 155 6 175 7 62 8 4 9 0
 10 0 11 1 12 1 13 0 14 0
 15 0 16 0 17 0 18 0 19 0
 20 0 21 0
/
% row 1 col 5 xpoint=-132.000000 ypoint=304.371429
\def\vunit{0.002902in}%
%%%%%%%%%%%% Plot figure in row 1 col 5 %%%%%%%%%%%%
\setcoordinatesystem units <\hunit,\vunit> point at -132.000000 304.371429
\setplotarea x from 0 to 22, y from 0 to 201
\axis bottom / % ticks numbered from 0 to 22 by 22 /
\axis left /
\axis top /
\axis right /
%locus  11:
\multiput {\tiny $\circ$} at 0 0 1 0 2 0 3 2 4 10
 5 42 6 201 7 181 8 38 9 6
 10 0 11 0 12 0 13 0 14 0
 15 0 16 0 17 0 18 0 19 0
 20 0 21 0
/
% row 2 col 0 xpoint=0.000000 ypoint=366.457153
\def\vunit{0.004821in}%
%%%%%%%%%%%% Plot figure in row 2 col 0 %%%%%%%%%%%%
\setcoordinatesystem units <\hunit,\vunit> point at 0.000000 366.457153
\setplotarea x from 0 to 22, y from 0 to 121
\axis bottom / % ticks numbered from 0 to 22 by 22 /
\axis left /
\axis top /
\axis right /
%locus  12:
\multiput {\tiny $\circ$} at 0 0 1 2 2 62 3 121 4 68
 5 54 6 14 7 16 8 45 9 37
 10 39 11 13 12 1 13 0 14 0
 15 1 16 1 17 0 18 0 19 0
 20 0 21 0
/
% row 2 col 1 xpoint=-26.400000 ypoint=508.800018
\def\vunit{0.003472in}%
%%%%%%%%%%%% Plot figure in row 2 col 1 %%%%%%%%%%%%
\setcoordinatesystem units <\hunit,\vunit> point at -26.400000 508.800018
\setplotarea x from 0 to 22, y from 0 to 168
\axis bottom / % ticks numbered from 0 to 22 by 22 /
\axis left /
\axis top /
\axis right /
%locus  13:
\multiput {\tiny $\circ$} at 0 0 1 3 2 11 3 33 4 131
 5 168 6 81 7 36 8 5 9 0
 10 0 11 0 12 0 13 0 14 0
 15 0 16 0 17 0 18 0 19 0
 20 0 21 0
/
% row 2 col 2 xpoint=-52.799999 ypoint=796.514282
\def\vunit{0.002218in}%
%%%%%%%%%%%% Plot figure in row 2 col 2 %%%%%%%%%%%%
\setcoordinatesystem units <\hunit,\vunit> point at -52.799999 796.514282
\setplotarea x from 0 to 22, y from 0 to 263
\axis bottom / % ticks numbered from 0 to 22 by 22 /
\axis left /
\axis top /
\axis right /
%locus  14:
\multiput {\tiny $\circ$} at 0 0 1 2 2 217 3 263 4 0
 5 0 6 0 7 0 8 0 9 0
 10 0 11 0 12 0 13 0 14 0
 15 0 16 0 17 0 18 0 19 0
 20 0 21 0
/
% row 2 col 3 xpoint=-79.199997 ypoint=354.342865
\def\vunit{0.004986in}%
%%%%%%%%%%%% Plot figure in row 2 col 3 %%%%%%%%%%%%
\setcoordinatesystem units <\hunit,\vunit> point at -79.199997 354.342865
\setplotarea x from 0 to 22, y from 0 to 117
\axis bottom / % ticks numbered from 0 to 22 by 22 /
\axis left /
\axis top /
\axis right /
%locus  15:
\multiput {\tiny $\circ$} at 0 0 1 6 2 86 3 110 4 117
 5 85 6 27 7 4 8 3 9 0
 10 0 11 0 12 0 13 0 14 0
 15 0 16 0 17 0 18 0 19 0
 20 0 21 0
/
% row 2 col 4 xpoint=-105.599998 ypoint=260.457153
\def\vunit{0.006783in}%
%%%%%%%%%%%% Plot figure in row 2 col 4 %%%%%%%%%%%%
\setcoordinatesystem units <\hunit,\vunit> point at -105.599998 260.457153
\setplotarea x from 0 to 22, y from 0 to 86
\axis bottom / % ticks numbered from 0 to 22 by 22 /
\axis left /
\axis top /
\axis right /
%locus  16:
\multiput {\tiny $\circ$} at 0 0 1 3 2 9 3 33 4 86
 5 46 6 70 7 34 8 5 9 3
 10 0 11 1 12 6 13 9 14 28
 15 24 16 43 17 30 18 29 19 9
 20 0 21 0
/
% row 2 col 5 xpoint=-132.000000 ypoint=366.457153
\def\vunit{0.004821in}%
%%%%%%%%%%%% Plot figure in row 2 col 5 %%%%%%%%%%%%
\setcoordinatesystem units <\hunit,\vunit> point at -132.000000 366.457153
\setplotarea x from 0 to 22, y from 0 to 121
\axis bottom / % ticks numbered from 0 to 22 by 22 /
\axis left /
\axis top /
\axis right /
%locus  17:
\multiput {\tiny $\circ$} at 0 0 1 4 2 33 3 92 4 121
 5 88 6 15 7 12 8 9 9 6
 10 21 11 36 12 0 13 0 14 0
 15 0 16 24 17 9 18 0 19 0
 20 0 21 0
/
% row 3 col 0 xpoint=0.000000 ypoint=1385.571411
\def\vunit{0.001913in}%
%%%%%%%%%%%% Plot figure in row 3 col 0 %%%%%%%%%%%%
\setcoordinatesystem units <\hunit,\vunit> point at 0.000000 1385.571411
\setplotarea x from 0 to 22, y from 0 to 305
\axis bottom / % ticks numbered from 0 to 22 by 22 /
\axis left /
\axis top /
\axis right /
%locus  18:
\multiput {\tiny $\circ$} at 0 0 1 0 2 100 3 39 4 305
 5 0 6 0 7 0 8 0 9 0
 10 0 11 0 12 0 13 0 14 0
 15 0 16 0 17 0 18 0 19 0
 20 0 21 0
/
% row 3 col 1 xpoint=-26.400000 ypoint=681.428589
\def\vunit{0.003889in}%
%%%%%%%%%%%% Plot figure in row 3 col 1 %%%%%%%%%%%%
\setcoordinatesystem units <\hunit,\vunit> point at -26.400000 681.428589
\setplotarea x from 0 to 22, y from 0 to 150
\axis bottom / % ticks numbered from 0 to 22 by 22 /
\axis left /
\axis top /
\axis right /
%locus  19:
\multiput {\tiny $\circ$} at 0 0 1 1 2 5 3 29 4 26
 5 145 6 150 7 45 8 62 9 5
 10 0 11 0 12 0 13 0 14 0
 15 0 16 0 17 0 18 0 19 0
 20 0 21 0
/
% row 3 col 2 xpoint=-52.799999 ypoint=417.942871
\def\vunit{0.006341in}%
%%%%%%%%%%%% Plot figure in row 3 col 2 %%%%%%%%%%%%
\setcoordinatesystem units <\hunit,\vunit> point at -52.799999 417.942871
\setplotarea x from 0 to 22, y from 0 to 92
\axis bottom / % ticks numbered from 0 to 22 by 22 /
\axis left /
\axis top /
\axis right /
%locus  20:
\multiput {\tiny $\circ$} at 0 0 1 4 2 31 3 5 4 5
 5 6 6 13 7 30 8 29 9 41
 10 49 11 92 12 86 13 55 14 20
 15 10 16 2 17 0 18 0 19 0
 20 0 21 0
/
% row 3 col 3 xpoint=-79.199997 ypoint=758.657166
\def\vunit{0.003493in}%
%%%%%%%%%%%% Plot figure in row 3 col 3 %%%%%%%%%%%%
\setcoordinatesystem units <\hunit,\vunit> point at -79.199997 758.657166
\setplotarea x from 0 to 22, y from 0 to 167
\axis bottom / % ticks numbered from 0 to 22 by 22 /
\axis left /
\axis top /
\axis right /
%locus  21:
\multiput {\tiny $\circ$} at 0 0 1 1 2 7 3 4 4 26
 5 1 6 96 7 3 8 167 9 6
 10 116 11 33 12 6 13 0 14 0
 15 0 16 0 17 0 18 0 19 0
 20 0 21 0
/
% row 3 col 4 xpoint=-105.599998 ypoint=781.371460
\def\vunit{0.003391in}%
%%%%%%%%%%%% Plot figure in row 3 col 4 %%%%%%%%%%%%
\setcoordinatesystem units <\hunit,\vunit> point at -105.599998 781.371460
\setplotarea x from 0 to 22, y from 0 to 172
\axis bottom / % ticks numbered from 0 to 22 by 22 /
\axis left /
\axis top /
\axis right /
%locus  22:
\multiput {\tiny $\circ$} at 0 0 1 1 2 2 3 8 4 52
 5 95 6 172 7 100 8 36 9 6
 10 0 11 0 12 0 13 0 14 0
 15 0 16 0 17 0 18 0 19 0
 20 0 21 0
/
% row 3 col 5 xpoint=-132.000000 ypoint=767.742859
\def\vunit{0.003452in}%
%%%%%%%%%%%% Plot figure in row 3 col 5 %%%%%%%%%%%%
\setcoordinatesystem units <\hunit,\vunit> point at -132.000000 767.742859
\setplotarea x from 0 to 22, y from 0 to 169
\axis bottom / % ticks numbered from 0 to 22 by 22 /
\axis left /
\axis top /
\axis right /
%locus  23:
\multiput {\tiny $\circ$} at 0 0 1 4 2 0 3 0 4 3
 5 1 6 9 7 85 8 79 9 169
 10 75 11 36 12 5 13 4 14 0
 15 0 16 0 17 0 18 0 19 0
 20 0 21 0
/
% row 4 col 0 xpoint=0.000000 ypoint=1611.200073
\def\vunit{0.002193in}%
%%%%%%%%%%%% Plot figure in row 4 col 0 %%%%%%%%%%%%
\setcoordinatesystem units <\hunit,\vunit> point at 0.000000 1611.200073
\setplotarea x from 0 to 22, y from 0 to 266
\axis bottom / % ticks numbered from 0 to 22 by 22 /
\axis left /
\axis top /
\axis right /
%locus  24:
\multiput {\tiny $\circ$} at 0 0 1 0 2 7 3 11 4 25
 5 58 6 266 7 76 8 17 9 2
 10 0 11 0 12 0 13 0 14 0
 15 0 16 0 17 0 18 0 19 0
 20 0 21 0
/
% row 4 col 1 xpoint=-26.400000 ypoint=1805.028564
\def\vunit{0.001957in}%
%%%%%%%%%%%% Plot figure in row 4 col 1 %%%%%%%%%%%%
\setcoordinatesystem units <\hunit,\vunit> point at -26.400000 1805.028564
\setplotarea x from 0 to 22, y from 0 to 298
\axis bottom / % ticks numbered from 0 to 22 by 22 /
\axis left /
\axis top /
\axis right /
%locus  25:
\multiput {\tiny $\circ$} at 0 0 1 0 2 2 3 2 4 33
 5 298 6 20 7 9 8 9 9 17
 10 30 11 28 12 18 13 4 14 0
 15 0 16 0 17 0 18 0 19 0
 20 0 21 0
/
% row 4 col 2 xpoint=-52.799999 ypoint=490.628571
\def\vunit{0.007202in}%
%%%%%%%%%%%% Plot figure in row 4 col 2 %%%%%%%%%%%%
\setcoordinatesystem units <\hunit,\vunit> point at -52.799999 490.628571
\setplotarea x from 0 to 22, y from 0 to 81
\axis bottom / % ticks numbered from 0 to 22 by 22 /
\axis left /
\axis top /
\axis right /
%locus  26:
\multiput {\tiny $\circ$} at 0 0 1 21 2 12 3 37 4 27
 5 39 6 80 7 56 8 55 9 81
 10 57 11 13 12 2 13 0 14 0
 15 0 16 0 17 0 18 0 19 0
 20 0 21 0
/
% row 4 col 3 xpoint=-79.199997 ypoint=1078.171387
\def\vunit{0.003277in}%
%%%%%%%%%%%% Plot figure in row 4 col 3 %%%%%%%%%%%%
\setcoordinatesystem units <\hunit,\vunit> point at -79.199997 1078.171387
\setplotarea x from 0 to 22, y from 0 to 178
\axis bottom / % ticks numbered from 0 to 22 by 22 /
\axis left /
\axis top /
\axis right /
%locus  27:
\multiput {\tiny $\circ$} at 0 0 1 15 2 4 3 5 4 77
 5 178 6 100 7 65 8 23 9 3
 10 0 11 0 12 0 13 0 14 0
 15 0 16 0 17 0 18 0 19 0
 20 0 21 0
/
% row 4 col 4 xpoint=-105.599998 ypoint=938.857178
\def\vunit{0.003763in}%
%%%%%%%%%%%% Plot figure in row 4 col 4 %%%%%%%%%%%%
\setcoordinatesystem units <\hunit,\vunit> point at -105.599998 938.857178
\setplotarea x from 0 to 22, y from 0 to 155
\axis bottom / % ticks numbered from 0 to 22 by 22 /
\axis left /
\axis top /
\axis right /
%locus  28:
\multiput {\tiny $\circ$} at 0 0 1 1 2 1 3 55 4 10
 5 15 6 114 7 155 8 90 9 16
 10 3 11 0 12 0 13 0 14 0
 15 0 16 0 17 0 18 0 19 0
 20 0 21 0
/
% row 4 col 5 xpoint=-132.000000 ypoint=1126.628540
\def\vunit{0.003136in}%
%%%%%%%%%%%% Plot figure in row 4 col 5 %%%%%%%%%%%%
\setcoordinatesystem units <\hunit,\vunit> point at -132.000000 1126.628540
\setplotarea x from 0 to 22, y from 0 to 186
\axis bottom / % ticks numbered from 0 to 22 by 22 /
\axis left /
\axis top /
\axis right /
%locus  29:
\multiput {\tiny $\circ$} at 0 0 1 5 2 72 3 124 4 186
 5 59 6 4 7 4 8 0 9 0
 10 0 11 0 12 0 13 0 14 0
 15 0 16 0 17 0 18 0 19 0
 20 0 21 0
/
% row 5 col 0 xpoint=0.000000 ypoint=1128.142822
\def\vunit{0.003915in}%
%%%%%%%%%%%% Plot figure in row 5 col 0 %%%%%%%%%%%%
\setcoordinatesystem units <\hunit,\vunit> point at 0.000000 1128.142822
\setplotarea x from 0 to 22, y from 0 to 149
\axis bottom / % ticks numbered from 0 to 22 by 22 /
\axis left /
\axis top /
\axis right /
%locus  30:
\multiput {\tiny $\circ$} at 0 0 1 14 2 82 3 71 4 140
 5 149 6 15 7 1 8 0 9 0
 10 0 11 0 12 0 13 0 14 0
 15 0 16 0 17 0 18 0 19 0
 20 0 21 0
/
% row 5 col 1 xpoint=-26.400000 ypoint=567.857178
\def\vunit{0.007778in}%
%%%%%%%%%%%% Plot figure in row 5 col 1 %%%%%%%%%%%%
\setcoordinatesystem units <\hunit,\vunit> point at -26.400000 567.857178
\setplotarea x from 0 to 22, y from 0 to 75
\axis bottom / % ticks numbered from 0 to 22 by 22 /
\axis left /
\axis top /
\axis right /
%locus  31:
\multiput {\tiny $\circ$} at 0 0 1 3 2 1 3 17 4 25
 5 46 6 44 7 61 8 58 9 75
 10 60 11 35 12 15 13 12 14 4
 15 3 16 1 17 0 18 0 19 0
 20 0 21 0
/
% row 5 col 2 xpoint=-52.799999 ypoint=507.285706
\def\vunit{0.008706in}%
%%%%%%%%%%%% Plot figure in row 5 col 2 %%%%%%%%%%%%
\setcoordinatesystem units <\hunit,\vunit> point at -52.799999 507.285706
\setplotarea x from 0 to 22, y from 0 to 67
\axis bottom / % ticks numbered from 0 to 22 by 22 /
\axis left /
\axis top /
\axis right /
%locus  32:
\multiput {\tiny $\circ$} at 0 0 1 6 2 3 3 12 4 20
 5 36 6 39 7 49 8 61 9 65
 10 67 11 59 12 33 13 8 14 4
 15 2 16 0 17 0 18 0 19 0
 20 0 21 0
/
% row 5 col 3 xpoint=-79.199997 ypoint=1877.714355
\def\vunit{0.002352in}%
%%%%%%%%%%%% Plot figure in row 5 col 3 %%%%%%%%%%%%
\setcoordinatesystem units <\hunit,\vunit> point at -79.199997 1877.714355
\setplotarea x from 0 to 22, y from 0 to 248
\axis bottom / % ticks numbered from 0 to 22 by 22 /
\axis left /
\axis top /
\axis right /
%locus  33:
\multiput {\tiny $\circ$} at 0 0 1 69 2 248 3 153 4 4
 5 0 6 0 7 0 8 0 9 0
 10 0 11 0 12 0 13 0 14 0
 15 0 16 0 17 0 18 0 19 0
 20 0 21 0
/
% row 5 col 4 xpoint=-105.599998 ypoint=1090.285767
\def\vunit{0.004051in}%
%%%%%%%%%%%% Plot figure in row 5 col 4 %%%%%%%%%%%%
\setcoordinatesystem units <\hunit,\vunit> point at -105.599998 1090.285767
\setplotarea x from 0 to 22, y from 0 to 144
\axis bottom / % ticks numbered from 0 to 22 by 22 /
\axis left /
\axis top /
\axis right /
%locus  34:
\multiput {\tiny $\circ$} at 0 0 1 2 2 105 3 36 4 98
 5 144 6 46 7 28 8 1 9 0
 10 0 11 0 12 0 13 0 14 0
 15 0 16 0 17 0 18 0 19 0
 20 0 21 0
/
% row 5 col 5 xpoint=-132.000000 ypoint=1309.857178
\def\vunit{0.003372in}%
%%%%%%%%%%%% Plot figure in row 5 col 5 %%%%%%%%%%%%
\setcoordinatesystem units <\hunit,\vunit> point at -132.000000 1309.857178
\setplotarea x from 0 to 22, y from 0 to 173
\axis bottom / % ticks numbered from 0 to 22 by 22 /
\axis left /
\axis top /
\axis right /
%locus  35:
\multiput {\tiny $\circ$} at 0 0 1 4 2 173 3 15 4 55
 5 50 6 90 7 27 8 57 9 1
 10 0 11 0 12 0 13 0 14 0
 15 0 16 0 17 0 18 0 19 0
 20 0 21 0
/
% row 6 col 0 xpoint=0.000000 ypoint=1008.514282
\def\vunit{0.005255in}%
%%%%%%%%%%%% Plot figure in row 6 col 0 %%%%%%%%%%%%
\setcoordinatesystem units <\hunit,\vunit> point at 0.000000 1008.514282
\setplotarea x from 0 to 22, y from 0 to 111
\axis bottom / % ticks numbered from 0 to 22 by 22 /
\axis left /
\axis top /
\axis right /
%locus  36:
\multiput {\tiny $\circ$} at 0 0 1 1 2 0 3 1 4 4
 5 28 6 18 7 25 8 28 9 61
 10 89 11 111 12 61 13 18 14 9
 15 0 16 0 17 0 18 0 19 0
 20 0 21 0
/
% row 6 col 1 xpoint=-26.400000 ypoint=2080.628662
\def\vunit{0.002547in}%
%%%%%%%%%%%% Plot figure in row 6 col 1 %%%%%%%%%%%%
\setcoordinatesystem units <\hunit,\vunit> point at -26.400000 2080.628662
\setplotarea x from 0 to 22, y from 0 to 229
\axis bottom / % ticks numbered from 0 to 22 by 22 /
\axis left /
\axis top /
\axis right /
%locus  37:
\multiput {\tiny $\circ$} at 0 0 1 1 2 15 3 10 4 103
 5 229 6 79 7 20 8 7 9 0
 10 0 11 0 12 0 13 0 14 0
 15 0 16 0 17 0 18 0 19 0
 20 0 21 0
/
% row 6 col 2 xpoint=-52.799999 ypoint=1181.142822
\def\vunit{0.004487in}%
%%%%%%%%%%%% Plot figure in row 6 col 2 %%%%%%%%%%%%
\setcoordinatesystem units <\hunit,\vunit> point at -52.799999 1181.142822
\setplotarea x from 0 to 22, y from 0 to 130
\axis bottom / % ticks numbered from 0 to 22 by 22 /
\axis left /
\axis top /
\axis right /
%locus  38:
\multiput {\tiny $\circ$} at 0 0 1 1 2 14 3 23 4 79
 5 130 6 128 7 61 8 20 9 0
 10 0 11 0 12 0 13 0 14 0
 15 0 16 0 17 0 18 0 19 0
 20 0 21 0
/
% row 6 col 3 xpoint=-79.199997 ypoint=808.628601
\def\vunit{0.006554in}%
%%%%%%%%%%%% Plot figure in row 6 col 3 %%%%%%%%%%%%
\setcoordinatesystem units <\hunit,\vunit> point at -79.199997 808.628601
\setplotarea x from 0 to 22, y from 0 to 89
\axis bottom / % ticks numbered from 0 to 22 by 22 /
\axis left /
\axis top /
\axis right /
%locus  39:
\multiput {\tiny $\circ$} at 0 0 1 0 2 0 3 89 4 1
 5 6 6 2 7 18 8 34 9 88
 10 81 11 33 12 7 13 4 14 12
 15 52 16 20 17 2 18 1 19 0
 20 0 21 0
/
% row 6 col 4 xpoint=-105.599998 ypoint=826.799988
\def\vunit{0.006410in}%
%%%%%%%%%%%% Plot figure in row 6 col 4 %%%%%%%%%%%%
\setcoordinatesystem units <\hunit,\vunit> point at -105.599998 826.799988
\setplotarea x from 0 to 22, y from 0 to 91
\axis bottom / % ticks numbered from 0 to 22 by 22 /
\axis left /
\axis top /
\axis right /
%locus  40:
\multiput {\tiny $\circ$} at 0 0 1 2 2 2 3 14 4 91
 5 44 6 61 7 86 8 40 9 31
 10 52 11 23 12 13 13 3 14 2
 15 0 16 0 17 0 18 0 19 0
 20 0 21 0
/
% row 6 col 5 xpoint=-132.000000 ypoint=2707.542969
\def\vunit{0.001957in}%
%%%%%%%%%%%% Plot figure in row 6 col 5 %%%%%%%%%%%%
\setcoordinatesystem units <\hunit,\vunit> point at -132.000000 2707.542969
\setplotarea x from 0 to 22, y from 0 to 298
\axis bottom / % ticks numbered from 0 to 22 by 22 /
\axis left /
\axis top /
\axis right /
%locus  41:
\multiput {\tiny $\circ$} at 0 0 1 10 2 8 3 298 4 138
 5 12 6 4 7 0 8 0 9 0
 10 0 11 0 12 0 13 0 14 0
 15 0 16 0 17 0 18 0 19 0
 20 0 21 0
/
% row 7 col 0 xpoint=0.000000 ypoint=2014.000000
\def\vunit{0.003070in}%
%%%%%%%%%%%% Plot figure in row 7 col 0 %%%%%%%%%%%%
\setcoordinatesystem units <\hunit,\vunit> point at 0.000000 2014.000000
\setplotarea x from 0 to 22, y from 0 to 190
\axis bottom / % ticks numbered from 0 to 22 by 22 /
\axis left /
\axis top /
\axis right /
%locus  42:
\multiput {\tiny $\circ$} at 0 0 1 0 2 18 3 18 4 12
 5 57 6 190 7 152 8 12 9 3
 10 0 11 0 12 0 13 0 14 0
 15 0 16 0 17 0 18 0 19 0
 20 0 21 0
/
% row 7 col 1 xpoint=-26.400000 ypoint=1346.200073
\def\vunit{0.004593in}%
%%%%%%%%%%%% Plot figure in row 7 col 1 %%%%%%%%%%%%
\setcoordinatesystem units <\hunit,\vunit> point at -26.400000 1346.200073
\setplotarea x from 0 to 22, y from 0 to 127
\axis bottom / % ticks numbered from 0 to 22 by 22 /
\axis left /
\axis top /
\axis right /
%locus  43:
\multiput {\tiny $\circ$} at 0 0 1 1 2 12 3 103 4 26
 5 127 6 24 7 29 8 72 9 40
 10 10 11 2 12 0 13 0 14 0
 15 0 16 0 17 0 18 0 19 0
 20 0 21 0
/
% row 7 col 2 xpoint=-52.799999 ypoint=1770.200073
\def\vunit{0.003493in}%
%%%%%%%%%%%% Plot figure in row 7 col 2 %%%%%%%%%%%%
\setcoordinatesystem units <\hunit,\vunit> point at -52.799999 1770.200073
\setplotarea x from 0 to 22, y from 0 to 167
\axis bottom / % ticks numbered from 0 to 22 by 22 /
\axis left /
\axis top /
\axis right /
%locus  44:
\multiput {\tiny $\circ$} at 0 0 1 40 2 75 3 167 4 117
 5 46 6 11 7 0 8 0 9 0
 10 0 11 0 12 0 13 0 14 0
 15 0 16 0 17 0 18 0 19 0
 20 0 21 0
/
% row 7 col 3 xpoint=-79.199997 ypoint=1632.400024
\def\vunit{0.003788in}%
%%%%%%%%%%%% Plot figure in row 7 col 3 %%%%%%%%%%%%
\setcoordinatesystem units <\hunit,\vunit> point at -79.199997 1632.400024
\setplotarea x from 0 to 22, y from 0 to 154
\axis bottom / % ticks numbered from 0 to 22 by 22 /
\axis left /
\axis top /
\axis right /
%locus  45:
\multiput {\tiny $\circ$} at 0 0 1 4 2 9 3 100 4 154
 5 102 6 72 7 18 8 1 9 0
 10 0 11 0 12 0 13 0 14 0
 15 0 16 0 17 0 18 0 19 0
 20 0 21 0
/
% row 7 col 4 xpoint=-105.599998 ypoint=3127.000000
\def\vunit{0.001977in}%
%%%%%%%%%%%% Plot figure in row 7 col 4 %%%%%%%%%%%%
\setcoordinatesystem units <\hunit,\vunit> point at -105.599998 3127.000000
\setplotarea x from 0 to 22, y from 0 to 295
\axis bottom / % ticks numbered from 0 to 22 by 22 /
\axis left /
\axis top /
\axis right /
%locus  46:
\multiput {\tiny $\circ$} at 0 0 1 11 2 3 3 44 4 295
 5 80 6 13 7 3 8 8 9 2
 10 1 11 0 12 0 13 0 14 0
 15 0 16 0 17 0 18 0 19 0
 20 0 21 0
/
% row 7 col 5 xpoint=-132.000000 ypoint=1060.000000
\def\vunit{0.005833in}%
%%%%%%%%%%%% Plot figure in row 7 col 5 %%%%%%%%%%%%
\setcoordinatesystem units <\hunit,\vunit> point at -132.000000 1060.000000
\setplotarea x from 0 to 22, y from 0 to 100
\axis bottom / % ticks numbered from 0 to 22 by 22 /
\axis left /
\axis top /
\axis right /
%locus  47:
\multiput {\tiny $\circ$} at 0 0 1 3 2 55 3 17 4 26
 5 7 6 100 7 71 8 88 9 45
 10 30 11 7 12 1 13 0 14 0
 15 0 16 0 17 0 18 0 19 0
 20 0 21 0
/
% row 8 col 0 xpoint=0.000000 ypoint=1526.400024
\def\vunit{0.004630in}%
%%%%%%%%%%%% Plot figure in row 8 col 0 %%%%%%%%%%%%
\setcoordinatesystem units <\hunit,\vunit> point at 0.000000 1526.400024
\setplotarea x from 0 to 22, y from 0 to 126
\axis bottom / % ticks numbered from 0 to 22 by 22 /
\axis left /
\axis top /
\axis right /
%locus  48:
\multiput {\tiny $\circ$} at 0 0 1 8 2 100 3 45 4 46
 5 126 6 103 7 33 8 1 9 0
 10 0 11 0 12 0 13 0 14 0
 15 0 16 0 17 0 18 0 19 0
 20 0 21 0
/
% row 8 col 1 xpoint=-26.400000 ypoint=2495.542969
\def\vunit{0.002832in}%
%%%%%%%%%%%% Plot figure in row 8 col 1 %%%%%%%%%%%%
\setcoordinatesystem units <\hunit,\vunit> point at -26.400000 2495.542969
\setplotarea x from 0 to 22, y from 0 to 206
\axis bottom / % ticks numbered from 0 to 22 by 22 /
\axis left /
\axis top /
\axis right /
%locus  49:
\multiput {\tiny $\circ$} at 0 0 1 6 2 3 3 33 4 206
 5 69 6 100 7 29 8 10 9 2
 10 0 11 0 12 0 13 0 14 0
 15 0 16 0 17 0 18 0 19 0
 20 0 21 0
/
% row 8 col 2 xpoint=-52.799999 ypoint=2071.542969
\def\vunit{0.003411in}%
%%%%%%%%%%%% Plot figure in row 8 col 2 %%%%%%%%%%%%
\setcoordinatesystem units <\hunit,\vunit> point at -52.799999 2071.542969
\setplotarea x from 0 to 22, y from 0 to 171
\axis bottom / % ticks numbered from 0 to 22 by 22 /
\axis left /
\axis top /
\axis right /
%locus  50:
\multiput {\tiny $\circ$} at 0 0 1 6 2 3 3 10 4 121
 5 171 6 115 7 25 8 3 9 0
 10 0 11 0 12 0 13 0 14 0
 15 0 16 0 17 0 18 0 19 0
 20 0 21 0
/
% row 8 col 3 xpoint=-79.199997 ypoint=2301.714355
\def\vunit{0.003070in}%
%%%%%%%%%%%% Plot figure in row 8 col 3 %%%%%%%%%%%%
\setcoordinatesystem units <\hunit,\vunit> point at -79.199997 2301.714355
\setplotarea x from 0 to 22, y from 0 to 190
\axis bottom / % ticks numbered from 0 to 22 by 22 /
\axis left /
\axis top /
\axis right /
%locus  51:
\multiput {\tiny $\circ$} at 0 0 1 2 2 2 3 17 4 190
 5 162 6 59 7 10 8 6 9 0
 10 0 11 0 12 0 13 0 14 0
 15 0 16 0 17 0 18 0 19 0
 20 0 21 0
/
% row 8 col 4 xpoint=-105.599998 ypoint=1356.800049
\def\vunit{0.005208in}%
%%%%%%%%%%%% Plot figure in row 8 col 4 %%%%%%%%%%%%
\setcoordinatesystem units <\hunit,\vunit> point at -105.599998 1356.800049
\setplotarea x from 0 to 22, y from 0 to 112
\axis bottom / % ticks numbered from 0 to 22 by 22 /
\axis left /
\axis top /
\axis right /
%locus  52:
\multiput {\tiny $\circ$} at 0 0 1 12 2 7 3 45 4 74
 5 111 6 112 7 77 8 21 9 8
 10 1 11 0 12 0 13 0 14 0
 15 0 16 0 17 0 18 0 19 0
 20 0 21 0
/
% row 8 col 5 xpoint=-132.000000 ypoint=2362.285645
\def\vunit{0.002991in}%
%%%%%%%%%%%% Plot figure in row 8 col 5 %%%%%%%%%%%%
\setcoordinatesystem units <\hunit,\vunit> point at -132.000000 2362.285645
\setplotarea x from 0 to 22, y from 0 to 195
\axis bottom / % ticks numbered from 0 to 22 by 22 /
\axis left /
\axis top /
\axis right /
%locus  53:
\multiput {\tiny $\circ$} at 0 0 1 11 2 57 3 7 4 11
 5 13 6 61 7 195 8 98 9 5
 10 4 11 0 12 0 13 0 14 0
 15 0 16 0 17 0 18 0 19 0
 20 0 21 0
/
% row 9 col 0 xpoint=0.000000 ypoint=4170.342773
\def\vunit{0.001906in}%
%%%%%%%%%%%% Plot figure in row 9 col 0 %%%%%%%%%%%%
\setcoordinatesystem units <\hunit,\vunit> point at 0.000000 4170.342773
\setplotarea x from 0 to 22, y from 0 to 306
\axis bottom ticks short numbered from 0 to 22 by 22 /
\axis left /
\axis top /
\axis right /
%locus  54:
\multiput {\tiny $\circ$} at 0 0 1 0 2 15 3 5 4 27
 5 306 6 84 7 9 8 0 9 0
 10 0 11 0 12 0 13 0 14 0
 15 0 16 0 17 0 18 0 19 0
 20 0 21 0
/
% row 9 col 1 xpoint=-26.400000 ypoint=1744.457153
\def\vunit{0.004557in}%
%%%%%%%%%%%% Plot figure in row 9 col 1 %%%%%%%%%%%%
\setcoordinatesystem units <\hunit,\vunit> point at -26.400000 1744.457153
\setplotarea x from 0 to 22, y from 0 to 128
\axis bottom / % ticks numbered from 0 to 22 by 22 /
\axis left /
\axis top /
\axis right /
%locus  55:
\multiput {\tiny $\circ$} at 0 0 1 1 2 63 3 10 4 76
 5 117 6 128 7 13 8 0 9 0
 10 0 11 0 12 0 13 0 14 0
 15 0 16 0 17 0 18 0 19 0
 20 0 21 0
/
% row 9 col 2 xpoint=-52.799999 ypoint=4865.399902
\def\vunit{0.001634in}%
%%%%%%%%%%%% Plot figure in row 9 col 2 %%%%%%%%%%%%
\setcoordinatesystem units <\hunit,\vunit> point at -52.799999 4865.399902
\setplotarea x from 0 to 22, y from 0 to 357
\axis bottom / % ticks numbered from 0 to 22 by 22 /
\axis left /
\axis top /
\axis right /
%locus  56:
\multiput {\tiny $\circ$} at 0 0 1 0 2 1 3 0 4 1
 5 357 6 105 7 0 8 0 9 0
 10 0 11 0 12 0 13 0 14 0
 15 0 16 0 17 0 18 0 19 0
 20 0 21 0
/
% row 9 col 3 xpoint=-79.199997 ypoint=1090.285767
\def\vunit{0.007292in}%
%%%%%%%%%%%% Plot figure in row 9 col 3 %%%%%%%%%%%%
\setcoordinatesystem units <\hunit,\vunit> point at -79.199997 1090.285767
\setplotarea x from 0 to 22, y from 0 to 80
\axis bottom / % ticks numbered from 0 to 22 by 22 /
\axis left /
\axis top /
\axis right /
%locus  57:
\multiput {\tiny $\circ$} at 0 0 1 23 2 6 3 1 4 2
 5 14 6 14 7 21 8 53 9 80
 10 79 11 73 12 53 13 21 14 1
 15 1 16 0 17 0 18 0 19 0
 20 0 21 0
/
% row 9 col 4 xpoint=-105.599998 ypoint=885.857178
\def\vunit{0.008974in}%
%%%%%%%%%%%% Plot figure in row 9 col 4 %%%%%%%%%%%%
\setcoordinatesystem units <\hunit,\vunit> point at -105.599998 885.857178
\setplotarea x from 0 to 22, y from 0 to 65
\axis bottom / % ticks numbered from 0 to 22 by 22 /
\axis left /
\axis top /
\axis right /
%locus  58:
\multiput {\tiny $\circ$} at 0 0 1 1 2 1 3 9 4 21
 5 33 6 47 7 65 8 60 9 58
 10 56 11 29 12 18 13 14 14 11
 15 1 16 0 17 0 18 0 19 0
 20 0 21 0
/
% row 9 col 5 xpoint=-132.000000 ypoint=885.857178
\def\vunit{0.008974in}%
%%%%%%%%%%%% Plot figure in row 9 col 5 %%%%%%%%%%%%
\setcoordinatesystem units <\hunit,\vunit> point at -132.000000 885.857178
\setplotarea x from 0 to 22, y from 0 to 65
\axis bottom / % ticks numbered from 0 to 22 by 22 /
\axis left /
\axis top /
\axis right /
%locus  59:
\multiput {\tiny $\circ$} at 0 0 1 1 2 0 3 29 4 1
 5 54 6 56 7 62 8 65 9 37
 10 39 11 39 12 18 13 3 14 4
 15 0 16 0 17 2 18 0 19 0
 20 0 21 0
/
\endpicture}
\end{center}
\caption{Frequency distributions of 60 STR loci\label{fig.str}}
\end{figure}

\clearpage

\section{Remark about statistical methods}

A variety of statistical methods have been introduced for analysis of
STRs.  Many of these seem useful, but none of them have yet become
widely used.  It is not obvious which should be included in an
introductory account.

\section{Descriptive statistics}

Several of the statistical methods used with STRs involve the so-called
``moments'' of the distribution of repeat counts.

\paragraph{The mean} For any given locus, the mean is the average
number of repeat counts in the sample.  For example, suppose we had
data on 4 haploid genomes with repeat counts 4, 4, 5, and 6.  The mean 
would be
\[
m = (4 + 4 + 5 + 6)/4 = 4.75
\]
The mean is also known as the ``first moment'' of the distribution.

\paragraph{The variance} is the mean squared deviation from the mean.
With the data above, this is%
%%%%%%%%%%%%%%%%%%%%%%%%%%%%%%%%%%%%%%%%%%%%%%%%%%%%%%%%%%%%%%%%
\footnote{In a statistics course, you would learn to divide by 3
  rather than 4 here in order to obtain an unbiased estimate.  I am
  ignoring such niceties.}
%%%%%%%%%%%%%%%%%%%%%%%%%%%%%%%%%%%%%%%%%%%%%%%%%%%%%%%%%%%%%%%%
\[
V = \Bigl(
(4 - 4.75)^2 + (4 - 4.75)^2 + (5 - 4.75)^2 + (6 - 4.75)^2
\Bigr)/4 = 0.6875
\]
The variance is also called the ``second moment'' of the distribution.

\paragraph{The fourth moment} is the mean of the fourth powers of
deviations from the mean.  With the example data,
\[
m_4 = \Bigl(
(4 - 4.75)^4 + (4 - 4.75)^4 + (5 - 4.75)^4 + (6 - 4.75)^4
\Bigr)/4 = 0.7695
\]

\paragraph{Gene identity} is defined as usual: it is the probability
that a random pair of genes is identical.  In the example data there
are 4 genes, so there are $(4 \times 3)/2 = 6$ pairs.  Only one of
these pairs is identical, so the gene identity is
\[
J = 1/6;
\]
This is \emph{not} a moment of the distribution.

\section{The method of Kimmel et al \citep{Kimmel:G-148-1921}}

Kimmel et al show that, at equilibrium between mutation and drift, 
\begin{eqnarray*}
  V &=& \theta\\
  J &=& 1 \Big / \sqrt{1 + 2\theta}
\end{eqnarray*}
Thus, $\theta$ can be estimated either by $V$ or by $(1/J^2 - 1)/2$.
The ratio
\begin{equation}
\beta = \frac{V}{(1/J^2 - 1)/2}
\end{equation}
ought to equal unity at migration-drift equilibrium.

\begin{wrapfigure}{r}{0.3\textwidth}
\vspace{-1.5\baselineskip}
%\kern 0.25\baselineskip
\includegraphics[width=0.3\textwidth]{kimmel-fig3.png}
\caption{From \citep[fig.~3]{Kimmel:G-148-1921}} 
\label{fig.kimmel} 
\vspace{-4\baselineskip}
\end{wrapfigure}
After a population increase, $\theta$ becomes larger so the numerator
and denominator of this ratio will both begin to increase.  The
denominator, however, increases faster.  Thus, $\beta < 1$ for some
time after a population expansion.  For the 60 tetranucleotide repeats 
shown in figure~\ref{fig.str},
\begin{center}
  \begin{tabular}{lc}
    Population & $\beta$\\ \hline
    Asia       & 1.8221\\
    Europe     & 1.3364\\
    Africa     & 1.1163\\ \hline
  \end{tabular}
\end{center}
This pattern does not match that predicted by a population expansion.
On the other hand, Kimmel et al show that these numbers do depart
significantly from unity, so the hypothesis of mutation-drift
equilibrium doesn't hold either.  What \emph{is} going on?

%\begin{figure}
%{\centering\includegraphics[width=0.3\textwidth]{kimmel-fig3.png}\\}
%\caption{Figure 3 of Kimmel et al}
%\end{figure}

When population size decreases, $\beta$ is elevated above unity and then
slowly decreases toward its equilibrium value of unity.  Taken at face
value, these data seem to imply a population collapse in the late
Pleistocene---precisely the opposite of the conclusions reached from
archaeology and mtDNA.  

In an effort to reconcile these contradictory findings, Kimmel et al
consider the effect of a ``bottleneck,'' or temporary reduction in
population size.  At the start of the bottleneck, $\theta$ decreases so
$\beta$ is elevated above unity.  After the population grows large again,
this process will reverse and $\beta$ will eventually converge to its
equilibrium value of unity.  However, the increase in $\beta$ happens
faster than the subsequent decrease.  Consequently, $\beta$ remains
elevated for a long time after the bottleneck has ended.  Kimmel et al show
that the elevation would still be apparent for several thousand generations
after the end of the bottleneck.

\section{A method-of-moments estimate of $\tau$}

\subsection{Method}

Consider a model of population history in which the population was constant
in size except for one instantaneous change, which occurred $t$ generations
ago.  The number of genes was $N_0$ before the change and $N_1$ after.  The
effect of this history on genetic data can be fully described by three
parameters:
\begin{eqnarray*}
\theta_0 &\equiv& 2uN_0\\
\theta_1 &\equiv& 2uN_1\\
\tau     &\equiv& 2ut
\end{eqnarray*}
My goal is to estimate these parameters.

% Mismatch Distributions
% -*-latex-*-
%% Definitions:
\def\hunit{0.0341in}
\def\vunit{0.583333in}
%%Plots are 0.833333 X 0.583333 inches
\begin{figure}
\begin{center}
\footnotesize
\mbox{%
\beginpicture
\headingtoplotskip=0.5\baselineskip
\valuestolabelleading=0.4\baselineskip
% row 0 col 0 xpoint=0.000000 ypoint=0.000000
%%%%%%%%%%%% Plot figure in row 0 col 0 %%%%%%%%%%%%
\setcoordinatesystem units <\hunit,\vunit> point at 0.000000 0.000000
\setplotarea x from 0 to 22, y from 0 to 1
\axis bottom / % ticks numbered from 0 to 22 by 22 /
\axis left /
\axis top /
\axis right /
%locus   0:
\multiput {\tiny $\circ$} at 0 0.000000 1 1.000000 2 0.501505 3 0.170363 4 0.066869
 5 0.098798 6 0.205152 7 0.045408 8 0.001804 9 0.000011
 10 0.000000 11 0.000000 12 0.000000 13 0.000000 14 0.000000
 15 0.000000 16 0.000000 17 0.000000 18 0.000000 19 0.000000
 20 0.000000 21 0.000000
/
% row 0 col 1 xpoint=-26.400000 ypoint=0.000000
%%%%%%%%%%%% Plot figure in row 0 col 1 %%%%%%%%%%%%
\setcoordinatesystem units <\hunit,\vunit> point at -26.400000 0.000000
\setplotarea x from 0 to 22, y from 0 to 1
\axis bottom / % ticks numbered from 0 to 22 by 22 /
\axis left /
\axis top /
\axis right /
%locus   1:
\multiput {\tiny $\circ$} at 0 0.000000 1 0.789580 2 0.283530 3 0.386967 4 0.213307
 5 1.000000 6 0.122664 7 0.024483 8 0.013667 9 0.001657
 10 0.005346 11 0.000267 12 0.000000 13 0.000000 14 0.000000
 15 0.000000 16 0.000000 17 0.000000 18 0.000000 19 0.000000
 20 0.000000 21 0.000000
/
% row 0 col 2 xpoint=-52.799999 ypoint=0.000000
%%%%%%%%%%%% Plot figure in row 0 col 2 %%%%%%%%%%%%
\setcoordinatesystem units <\hunit,\vunit> point at -52.799999 0.000000
\setplotarea x from 0 to 22, y from 0 to 1
\axis bottom / % ticks numbered from 0 to 22 by 22 /
\axis left /
\axis top /
\axis right /
%locus   2:
\multiput {\tiny $\circ$} at 0 0.000000 1 1.000000 2 0.806232 3 0.557308 4 0.330791
 5 0.164104 6 0.067117 7 0.051313 8 0.005327 9 0.000000
 10 0.000000 11 0.000000 12 0.000000 13 0.000000 14 0.000000
 15 0.000000 16 0.000000 17 0.000000 18 0.000000 19 0.000000
 20 0.000000 21 0.000000
/
% row 0 col 3 xpoint=-79.199997 ypoint=0.000000
%%%%%%%%%%%% Plot figure in row 0 col 3 %%%%%%%%%%%%
\setcoordinatesystem units <\hunit,\vunit> point at -79.199997 0.000000
\setplotarea x from 0 to 22, y from 0 to 1
\axis bottom / % ticks numbered from 0 to 22 by 22 /
\axis left /
\axis top /
\axis right /
%locus   3:
\multiput {\tiny $\circ$} at 0 0.000000 1 0.837703 2 0.597343 3 1.000000 4 0.659580
 5 0.049670 6 0.028972 7 0.154971 8 0.008185 9 0.000000
 10 0.000000 11 0.000000 12 0.000000 13 0.000000 14 0.000000
 15 0.000000 16 0.000000 17 0.000000 18 0.000000 19 0.000000
 20 0.000000 21 0.000000
/
% row 0 col 4 xpoint=-105.599998 ypoint=0.000000
%%%%%%%%%%%% Plot figure in row 0 col 4 %%%%%%%%%%%%
\setcoordinatesystem units <\hunit,\vunit> point at -105.599998 0.000000
\setplotarea x from 0 to 22, y from 0 to 1
\axis bottom / % ticks numbered from 0 to 22 by 22 /
\axis left /
\axis top /
\axis right /
%locus   4:
\multiput {\tiny $\circ$} at 0 0.000000 1 1.000000 2 0.530081 3 0.270848 4 0.245210
 5 0.276781 6 0.253073 7 0.088646 8 0.007743 9 0.000785
 10 0.000000 11 0.000000 12 0.000000 13 0.000000 14 0.000000
 15 0.000000 16 0.000000 17 0.000000 18 0.000000 19 0.000000
 20 0.000000 21 0.000000
/
% row 0 col 5 xpoint=-132.000000 ypoint=0.000000
%%%%%%%%%%%% Plot figure in row 0 col 5 %%%%%%%%%%%%
\setcoordinatesystem units <\hunit,\vunit> point at -132.000000 0.000000
\setplotarea x from 0 to 22, y from 0 to 1
\axis bottom / % ticks numbered from 0 to 22 by 22 /
\axis left /
\axis top /
\axis right /
%locus   5:
\multiput {\tiny $\circ$} at 0 0.000000 1 1.000000 2 0.819742 3 0.602081 4 0.412795
 5 0.270652 6 0.184499 7 0.097103 8 0.031457 9 0.015223
 10 0.004887 11 0.000050 12 0.000000 13 0.000000 14 0.000000
 15 0.000000 16 0.000000 17 0.000000 18 0.000000 19 0.000000
 20 0.000000 21 0.000000
/
% row 1 col 0 xpoint=0.000000 ypoint=1.300000
%%%%%%%%%%%% Plot figure in row 1 col 0 %%%%%%%%%%%%
\setcoordinatesystem units <\hunit,\vunit> point at 0.000000 1.300000
\setplotarea x from 0 to 22, y from 0 to 1
\axis bottom / % ticks numbered from 0 to 22 by 22 /
\axis left /
\axis top /
\axis right /
%locus   6:
\multiput {\tiny $\circ$} at 0 0.000000 1 1.000000 2 0.822443 3 0.604021 4 0.390459
 5 0.226217 6 0.129904 7 0.095341 8 0.082755 9 0.081783
 10 0.065794 11 0.024686 12 0.003993 13 0.001580 14 0.000781
 15 0.000000 16 0.000017 17 0.000000 18 0.000000 19 0.000000
 20 0.000000 21 0.000000
/
% row 1 col 1 xpoint=-26.400000 ypoint=1.300000
%%%%%%%%%%%% Plot figure in row 1 col 1 %%%%%%%%%%%%
\setcoordinatesystem units <\hunit,\vunit> point at -26.400000 1.300000
\setplotarea x from 0 to 22, y from 0 to 1
\axis bottom / % ticks numbered from 0 to 22 by 22 /
\axis left /
\axis top /
\axis right /
%locus   7:
\multiput {\tiny $\circ$} at 0 0.000000 1 0.277130 2 1.000000 3 0.266364 4 0.328388
 5 0.253577 6 0.502117 7 0.408790 8 0.347679 9 0.329745
 10 0.282998 11 0.230661 12 0.242534 13 0.254435 14 0.252027
 15 0.209958 16 0.119593 17 0.053555 18 0.053998 19 0.041073
 20 0.010794 21 0.000277
/
% row 1 col 2 xpoint=-52.799999 ypoint=1.300000
%%%%%%%%%%%% Plot figure in row 1 col 2 %%%%%%%%%%%%
\setcoordinatesystem units <\hunit,\vunit> point at -52.799999 1.300000
\setplotarea x from 0 to 22, y from 0 to 1
\axis bottom / % ticks numbered from 0 to 22 by 22 /
\axis left /
\axis top /
\axis right /
%locus   8:
\multiput {\tiny $\circ$} at 0 0.000000 1 0.212100 2 1.000000 3 0.226408 4 0.710948
 5 0.191469 6 0.413858 7 0.711618 8 0.251219 9 0.313193
 10 0.227962 11 0.203472 12 0.096351 13 0.016317 14 0.000616
 15 0.000000 16 0.000000 17 0.000000 18 0.000000 19 0.000000
 20 0.000000 21 0.000000
/
% row 1 col 3 xpoint=-79.199997 ypoint=1.300000
%%%%%%%%%%%% Plot figure in row 1 col 3 %%%%%%%%%%%%
\setcoordinatesystem units <\hunit,\vunit> point at -79.199997 1.300000
\setplotarea x from 0 to 22, y from 0 to 1
\axis bottom / % ticks numbered from 0 to 22 by 22 /
\axis left /
\axis top /
\axis right /
%locus   9:
\multiput {\tiny $\circ$} at 0 0.000000 1 0.723204 2 1.000000 3 0.936803 4 0.338095
 5 0.604709 6 0.142271 7 0.050293 8 0.097097 9 0.003643
 10 0.000403 11 0.000000 12 0.000000 13 0.000000 14 0.000000
 15 0.000000 16 0.000000 17 0.000000 18 0.000000 19 0.000000
 20 0.000000 21 0.000000
/
% row 1 col 4 xpoint=-105.599998 ypoint=1.300000
%%%%%%%%%%%% Plot figure in row 1 col 4 %%%%%%%%%%%%
\setcoordinatesystem units <\hunit,\vunit> point at -105.599998 1.300000
\setplotarea x from 0 to 22, y from 0 to 1
\axis bottom / % ticks numbered from 0 to 22 by 22 /
\axis left /
\axis top /
\axis right /
%locus  10:
\multiput {\tiny $\circ$} at 0 0.000000 1 1.000000 2 0.458610 3 0.183694 4 0.099653
 5 0.164155 6 0.176819 7 0.025076 8 0.001349 9 0.000653
 10 0.000370 11 0.000044 12 0.000000 13 0.000000 14 0.000000
 15 0.000000 16 0.000000 17 0.000000 18 0.000000 19 0.000000
 20 0.000000 21 0.000000
/
% row 1 col 5 xpoint=-132.000000 ypoint=1.300000
%%%%%%%%%%%% Plot figure in row 1 col 5 %%%%%%%%%%%%
\setcoordinatesystem units <\hunit,\vunit> point at -132.000000 1.300000
\setplotarea x from 0 to 22, y from 0 to 1
\axis bottom / % ticks numbered from 0 to 22 by 22 /
\axis left /
\axis top /
\axis right /
%locus  11:
\multiput {\tiny $\circ$} at 0 0.000000 1 1.000000 2 0.351735 3 0.095753 4 0.019410
 5 0.010817 6 0.192137 7 0.155531 8 0.006712 9 0.000143
 10 0.000000 11 0.000000 12 0.000000 13 0.000000 14 0.000000
 15 0.000000 16 0.000000 17 0.000000 18 0.000000 19 0.000000
 20 0.000000 21 0.000000
/
% row 2 col 0 xpoint=0.000000 ypoint=2.600000
%%%%%%%%%%%% Plot figure in row 2 col 0 %%%%%%%%%%%%
\setcoordinatesystem units <\hunit,\vunit> point at 0.000000 2.600000
\setplotarea x from 0 to 22, y from 0 to 1
\axis bottom / % ticks numbered from 0 to 22 by 22 /
\axis left /
\axis top /
\axis right /
%locus  12:
\multiput {\tiny $\circ$} at 0 0.000000 1 1.000000 2 0.694031 3 0.566917 4 0.399187
 5 0.484903 6 0.434049 7 0.329223 8 0.188650 9 0.055038
 10 0.021182 11 0.006558 12 0.007604 13 0.007363 14 0.002575
 15 0.000080 16 0.000000 17 0.000000 18 0.000000 19 0.000000
 20 0.000000 21 0.000000
/
% row 2 col 1 xpoint=-26.400000 ypoint=2.600000
%%%%%%%%%%%% Plot figure in row 2 col 1 %%%%%%%%%%%%
\setcoordinatesystem units <\hunit,\vunit> point at -26.400000 2.600000
\setplotarea x from 0 to 22, y from 0 to 1
\axis bottom / % ticks numbered from 0 to 22 by 22 /
\axis left /
\axis top /
\axis right /
%locus  13:
\multiput {\tiny $\circ$} at 0 0.000000 1 1.000000 2 0.556622 3 0.247142 4 0.172578
 5 0.180003 6 0.041052 7 0.007598 8 0.000115 9 0.000000
 10 0.000000 11 0.000000 12 0.000000 13 0.000000 14 0.000000
 15 0.000000 16 0.000000 17 0.000000 18 0.000000 19 0.000000
 20 0.000000 21 0.000000
/
% row 2 col 2 xpoint=-52.799999 ypoint=2.600000
%%%%%%%%%%%% Plot figure in row 2 col 2 %%%%%%%%%%%%
\setcoordinatesystem units <\hunit,\vunit> point at -52.799999 2.600000
\setplotarea x from 0 to 22, y from 0 to 1
\axis bottom / % ticks numbered from 0 to 22 by 22 /
\axis left /
\axis top /
\axis right /
%locus  14:
\multiput {\tiny $\circ$} at 0 0.000000 1 1.000000 2 0.212919 3 0.299563 4 0.000000
 5 0.000000 6 0.000000 7 0.000000 8 0.000000 9 0.000000
 10 0.000000 11 0.000000 12 0.000000 13 0.000000 14 0.000000
 15 0.000000 16 0.000000 17 0.000000 18 0.000000 19 0.000000
 20 0.000000 21 0.000000
/
% row 2 col 3 xpoint=-79.199997 ypoint=2.600000
%%%%%%%%%%%% Plot figure in row 2 col 3 %%%%%%%%%%%%
\setcoordinatesystem units <\hunit,\vunit> point at -79.199997 2.600000
\setplotarea x from 0 to 22, y from 0 to 1
\axis bottom / % ticks numbered from 0 to 22 by 22 /
\axis left /
\axis top /
\axis right /
%locus  15:
\multiput {\tiny $\circ$} at 0 0.000000 1 1.000000 2 0.723572 3 0.417525 4 0.199242
 5 0.074432 6 0.012992 7 0.000596 8 0.000043 9 0.000000
 10 0.000000 11 0.000000 12 0.000000 13 0.000000 14 0.000000
 15 0.000000 16 0.000000 17 0.000000 18 0.000000 19 0.000000
 20 0.000000 21 0.000000
/
% row 2 col 4 xpoint=-105.599998 ypoint=2.600000
%%%%%%%%%%%% Plot figure in row 2 col 4 %%%%%%%%%%%%
\setcoordinatesystem units <\hunit,\vunit> point at -105.599998 2.600000
\setplotarea x from 0 to 22, y from 0 to 1
\axis bottom / % ticks numbered from 0 to 22 by 22 /
\axis left /
\axis top /
\axis right /
%locus  16:
\multiput {\tiny $\circ$} at 0 0.000000 1 1.000000 2 0.823574 3 0.525618 4 0.343482
 5 0.149670 6 0.156993 7 0.149843 8 0.234524 9 0.325779
 10 0.465242 11 0.472623 12 0.489721 13 0.362195 14 0.262146
 15 0.130785 16 0.063403 17 0.022230 18 0.013263 19 0.001038
 20 0.000000 21 0.000000
/
% row 2 col 5 xpoint=-132.000000 ypoint=2.600000
%%%%%%%%%%%% Plot figure in row 2 col 5 %%%%%%%%%%%%
\setcoordinatesystem units <\hunit,\vunit> point at -132.000000 2.600000
\setplotarea x from 0 to 22, y from 0 to 1
\axis bottom / % ticks numbered from 0 to 22 by 22 /
\axis left /
\axis top /
\axis right /
%locus  17:
\multiput {\tiny $\circ$} at 0 0.000000 1 1.000000 2 0.584750 3 0.352604 4 0.286709
 5 0.258959 6 0.270163 7 0.248566 8 0.155787 9 0.059489
 10 0.025874 11 0.092454 12 0.133377 13 0.118978 14 0.058461
 15 0.014182 16 0.006279 17 0.000650 18 0.000000 19 0.000000
 20 0.000000 21 0.000000
/
% row 3 col 0 xpoint=0.000000 ypoint=3.900000
%%%%%%%%%%%% Plot figure in row 3 col 0 %%%%%%%%%%%%
\setcoordinatesystem units <\hunit,\vunit> point at 0.000000 3.900000
\setplotarea x from 0 to 22, y from 0 to 1
\axis bottom / % ticks numbered from 0 to 22 by 22 /
\axis left /
\axis top /
\axis right /
%locus  18:
\multiput {\tiny $\circ$} at 0 0.000000 1 0.478999 2 1.000000 3 0.011236 4 0.702957
 5 0.000000 6 0.000000 7 0.000000 8 0.000000 9 0.000000
 10 0.000000 11 0.000000 12 0.000000 13 0.000000 14 0.000000
 15 0.000000 16 0.000000 17 0.000000 18 0.000000 19 0.000000
 20 0.000000 21 0.000000
/
% row 3 col 1 xpoint=-26.400000 ypoint=3.900000
%%%%%%%%%%%% Plot figure in row 3 col 1 %%%%%%%%%%%%
\setcoordinatesystem units <\hunit,\vunit> point at -26.400000 3.900000
\setplotarea x from 0 to 22, y from 0 to 1
\axis bottom / % ticks numbered from 0 to 22 by 22 /
\axis left /
\axis top /
\axis right /
%locus  19:
\multiput {\tiny $\circ$} at 0 0.000000 1 1.000000 2 0.670425 3 0.446987 4 0.129556
 5 0.207394 6 0.167820 7 0.016045 8 0.026203 9 0.000138
 10 0.000000 11 0.000000 12 0.000000 13 0.000000 14 0.000000
 15 0.000000 16 0.000000 17 0.000000 18 0.000000 19 0.000000
 20 0.000000 21 0.000000
/
% row 3 col 2 xpoint=-52.799999 ypoint=3.900000
%%%%%%%%%%%% Plot figure in row 3 col 2 %%%%%%%%%%%%
\setcoordinatesystem units <\hunit,\vunit> point at -52.799999 3.900000
\setplotarea x from 0 to 22, y from 0 to 1
\axis bottom / % ticks numbered from 0 to 22 by 22 /
\axis left /
\axis top /
\axis right /
%locus  20:
\multiput {\tiny $\circ$} at 0 0.000000 1 1.000000 2 0.816797 3 0.612689 4 0.479416
 5 0.364071 6 0.253438 7 0.193313 8 0.159534 9 0.191621
 10 0.174913 11 0.184381 12 0.116844 13 0.048944 14 0.008439
 15 0.001307 16 0.000021 17 0.000000 18 0.000000 19 0.000000
 20 0.000000 21 0.000000
/
% row 3 col 3 xpoint=-79.199997 ypoint=3.900000
%%%%%%%%%%%% Plot figure in row 3 col 3 %%%%%%%%%%%%
\setcoordinatesystem units <\hunit,\vunit> point at -79.199997 3.900000
\setplotarea x from 0 to 22, y from 0 to 1
\axis bottom / % ticks numbered from 0 to 22 by 22 /
\axis left /
\axis top /
\axis right /
%locus  21:
\multiput {\tiny $\circ$} at 0 0.000000 1 0.173623 2 1.000000 3 0.182902 4 0.446810
 5 0.108752 6 0.182005 7 0.039433 8 0.205982 9 0.009701
 10 0.087401 11 0.006920 12 0.000192 13 0.000000 14 0.000000
 15 0.000000 16 0.000000 17 0.000000 18 0.000000 19 0.000000
 20 0.000000 21 0.000000
/
% row 3 col 4 xpoint=-105.599998 ypoint=3.900000
%%%%%%%%%%%% Plot figure in row 3 col 4 %%%%%%%%%%%%
\setcoordinatesystem units <\hunit,\vunit> point at -105.599998 3.900000
\setplotarea x from 0 to 22, y from 0 to 1
\axis bottom / % ticks numbered from 0 to 22 by 22 /
\axis left /
\axis top /
\axis right /
%locus  22:
\multiput {\tiny $\circ$} at 0 0.000000 1 1.000000 2 0.611011 3 0.264077 4 0.101662
 5 0.074994 6 0.177229 7 0.059045 8 0.007512 9 0.000176
 10 0.000000 11 0.000000 12 0.000000 13 0.000000 14 0.000000
 15 0.000000 16 0.000000 17 0.000000 18 0.000000 19 0.000000
 20 0.000000 21 0.000000
/
% row 3 col 5 xpoint=-132.000000 ypoint=3.900000
%%%%%%%%%%%% Plot figure in row 3 col 5 %%%%%%%%%%%%
\setcoordinatesystem units <\hunit,\vunit> point at -132.000000 3.900000
\setplotarea x from 0 to 22, y from 0 to 1
\axis bottom / % ticks numbered from 0 to 22 by 22 /
\axis left /
\axis top /
\axis right /
%locus  23:
\multiput {\tiny $\circ$} at 0 0.000000 1 1.000000 2 0.760990 3 0.335823 4 0.143255
 5 0.046210 6 0.027567 7 0.061778 8 0.061379 9 0.203454
 10 0.042050 11 0.009198 12 0.000577 13 0.000082 14 0.000000
 15 0.000000 16 0.000000 17 0.000000 18 0.000000 19 0.000000
 20 0.000000 21 0.000000
/
% row 4 col 0 xpoint=0.000000 ypoint=5.200000
%%%%%%%%%%%% Plot figure in row 4 col 0 %%%%%%%%%%%%
\setcoordinatesystem units <\hunit,\vunit> point at 0.000000 5.200000
\setplotarea x from 0 to 22, y from 0 to 1
\axis bottom / % ticks numbered from 0 to 22 by 22 /
\axis left /
\axis top /
\axis right /
%locus  24:
\multiput {\tiny $\circ$} at 0 0.000000 1 1.000000 2 0.426996 3 0.174804 4 0.087408
 5 0.041151 6 0.458153 7 0.037114 8 0.001754 9 0.000013
 10 0.000000 11 0.000000 12 0.000000 13 0.000000 14 0.000000
 15 0.000000 16 0.000000 17 0.000000 18 0.000000 19 0.000000
 20 0.000000 21 0.000000
/
% row 4 col 1 xpoint=-26.400000 ypoint=5.200000
%%%%%%%%%%%% Plot figure in row 4 col 1 %%%%%%%%%%%%
\setcoordinatesystem units <\hunit,\vunit> point at -26.400000 5.200000
\setplotarea x from 0 to 22, y from 0 to 1
\axis bottom / % ticks numbered from 0 to 22 by 22 /
\axis left /
\axis top /
\axis right /
%locus  25:
\multiput {\tiny $\circ$} at 0 0.000000 1 0.561479 2 0.176904 3 0.151242 4 0.208720
 5 1.000000 6 0.304644 7 0.199867 8 0.059220 9 0.009006
 10 0.008066 11 0.006076 12 0.002360 13 0.000093 14 0.000000
 15 0.000000 16 0.000000 17 0.000000 18 0.000000 19 0.000000
 20 0.000000 21 0.000000
/
% row 4 col 2 xpoint=-52.799999 ypoint=5.200000
%%%%%%%%%%%% Plot figure in row 4 col 2 %%%%%%%%%%%%
\setcoordinatesystem units <\hunit,\vunit> point at -52.799999 5.200000
\setplotarea x from 0 to 22, y from 0 to 1
\axis bottom / % ticks numbered from 0 to 22 by 22 /
\axis left /
\axis top /
\axis right /
%locus  26:
\multiput {\tiny $\circ$} at 0 0.000000 1 1.000000 2 0.862528 3 0.793000 4 0.601938
 5 0.441533 6 0.368775 7 0.232543 8 0.156705 9 0.130370
 10 0.046851 11 0.003466 12 0.000021 13 0.000000 14 0.000000
 15 0.000000 16 0.000000 17 0.000000 18 0.000000 19 0.000000
 20 0.000000 21 0.000000
/
% row 4 col 3 xpoint=-79.199997 ypoint=5.200000
%%%%%%%%%%%% Plot figure in row 4 col 3 %%%%%%%%%%%%
\setcoordinatesystem units <\hunit,\vunit> point at -79.199997 5.200000
\setplotarea x from 0 to 22, y from 0 to 1
\axis bottom / % ticks numbered from 0 to 22 by 22 /
\axis left /
\axis top /
\axis right /
%locus  27:
\multiput {\tiny $\circ$} at 0 0.000000 1 1.000000 2 0.574768 3 0.293633 4 0.178684
 5 0.249018 6 0.088731 7 0.034849 8 0.004278 9 0.000037
 10 0.000000 11 0.000000 12 0.000000 13 0.000000 14 0.000000
 15 0.000000 16 0.000000 17 0.000000 18 0.000000 19 0.000000
 20 0.000000 21 0.000000
/
% row 4 col 4 xpoint=-105.599998 ypoint=5.200000
%%%%%%%%%%%% Plot figure in row 4 col 4 %%%%%%%%%%%%
\setcoordinatesystem units <\hunit,\vunit> point at -105.599998 5.200000
\setplotarea x from 0 to 22, y from 0 to 1
\axis bottom / % ticks numbered from 0 to 22 by 22 /
\axis left /
\axis top /
\axis right /
%locus  28:
\multiput {\tiny $\circ$} at 0 0.000000 1 1.000000 2 0.488137 3 0.343692 4 0.285560
 5 0.153947 6 0.122997 7 0.175367 8 0.056825 9 0.001771
 10 0.000042 11 0.000000 12 0.000000 13 0.000000 14 0.000000
 15 0.000000 16 0.000000 17 0.000000 18 0.000000 19 0.000000
 20 0.000000 21 0.000000
/
% row 4 col 5 xpoint=-132.000000 ypoint=5.200000
%%%%%%%%%%%% Plot figure in row 4 col 5 %%%%%%%%%%%%
\setcoordinatesystem units <\hunit,\vunit> point at -132.000000 5.200000
\setplotarea x from 0 to 22, y from 0 to 1
\axis bottom / % ticks numbered from 0 to 22 by 22 /
\axis left /
\axis top /
\axis right /
%locus  29:
\multiput {\tiny $\circ$} at 0 0.000000 1 1.000000 2 0.541174 3 0.234747 4 0.222139
 5 0.026696 6 0.000528 7 0.000069 8 0.000000 9 0.000000
 10 0.000000 11 0.000000 12 0.000000 13 0.000000 14 0.000000
 15 0.000000 16 0.000000 17 0.000000 18 0.000000 19 0.000000
 20 0.000000 21 0.000000
/
% row 5 col 0 xpoint=0.000000 ypoint=6.500000
%%%%%%%%%%%% Plot figure in row 5 col 0 %%%%%%%%%%%%
\setcoordinatesystem units <\hunit,\vunit> point at 0.000000 6.500000
\setplotarea x from 0 to 22, y from 0 to 1
\axis bottom / % ticks numbered from 0 to 22 by 22 /
\axis left /
\axis top /
\axis right /
%locus  30:
\multiput {\tiny $\circ$} at 0 0.000000 1 1.000000 2 0.672961 3 0.414958 4 0.205963
 5 0.144888 6 0.001660 7 0.000000 8 0.000000 9 0.000000
 10 0.000000 11 0.000000 12 0.000000 13 0.000000 14 0.000000
 15 0.000000 16 0.000000 17 0.000000 18 0.000000 19 0.000000
 20 0.000000 21 0.000000
/
% row 5 col 1 xpoint=-26.400000 ypoint=6.500000
%%%%%%%%%%%% Plot figure in row 5 col 1 %%%%%%%%%%%%
\setcoordinatesystem units <\hunit,\vunit> point at -26.400000 6.500000
\setplotarea x from 0 to 22, y from 0 to 1
\axis bottom / % ticks numbered from 0 to 22 by 22 /
\axis left /
\axis top /
\axis right /
%locus  31:
\multiput {\tiny $\circ$} at 0 0.000000 1 1.000000 2 0.900536 3 0.749612 4 0.611025
 5 0.463695 6 0.325039 7 0.221238 8 0.141486 9 0.117553
 10 0.069154 11 0.025208 12 0.007815 13 0.003015 14 0.000603
 15 0.000209 16 0.000000 17 0.000000 18 0.000000 19 0.000000
 20 0.000000 21 0.000000
/
% row 5 col 2 xpoint=-52.799999 ypoint=6.500000
%%%%%%%%%%%% Plot figure in row 5 col 2 %%%%%%%%%%%%
\setcoordinatesystem units <\hunit,\vunit> point at -52.799999 6.500000
\setplotarea x from 0 to 22, y from 0 to 1
\axis bottom / % ticks numbered from 0 to 22 by 22 /
\axis left /
\axis top /
\axis right /
%locus  32:
\multiput {\tiny $\circ$} at 0 0.000000 1 1.000000 2 0.889105 3 0.746241 4 0.594934
 5 0.454930 6 0.326987 7 0.226596 8 0.156342 9 0.109705
 10 0.082679 11 0.053342 12 0.015927 13 0.002014 14 0.000687
 15 0.000023 16 0.000000 17 0.000000 18 0.000000 19 0.000000
 20 0.000000 21 0.000000
/
% row 5 col 3 xpoint=-79.199997 ypoint=6.500000
%%%%%%%%%%%% Plot figure in row 5 col 3 %%%%%%%%%%%%
\setcoordinatesystem units <\hunit,\vunit> point at -79.199997 6.500000
\setplotarea x from 0 to 22, y from 0 to 1
\axis bottom / % ticks numbered from 0 to 22 by 22 /
\axis left /
\axis top /
\axis right /
%locus  33:
\multiput {\tiny $\circ$} at 0 0.000000 1 1.000000 2 0.472599 3 0.107141 4 0.000053
 5 0.000000 6 0.000000 7 0.000000 8 0.000000 9 0.000000
 10 0.000000 11 0.000000 12 0.000000 13 0.000000 14 0.000000
 15 0.000000 16 0.000000 17 0.000000 18 0.000000 19 0.000000
 20 0.000000 21 0.000000
/
% row 5 col 4 xpoint=-105.599998 ypoint=6.500000
%%%%%%%%%%%% Plot figure in row 5 col 4 %%%%%%%%%%%%
\setcoordinatesystem units <\hunit,\vunit> point at -105.599998 6.500000
\setplotarea x from 0 to 22, y from 0 to 1
\axis bottom / % ticks numbered from 0 to 22 by 22 /
\axis left /
\axis top /
\axis right /
%locus  34:
\multiput {\tiny $\circ$} at 0 0.000000 1 1.000000 2 0.908405 3 0.682268 4 0.290847
 5 0.277844 6 0.022945 7 0.006459 8 0.000000 9 0.000000
 10 0.000000 11 0.000000 12 0.000000 13 0.000000 14 0.000000
 15 0.000000 16 0.000000 17 0.000000 18 0.000000 19 0.000000
 20 0.000000 21 0.000000
/
% row 5 col 5 xpoint=-132.000000 ypoint=6.500000
%%%%%%%%%%%% Plot figure in row 5 col 5 %%%%%%%%%%%%
\setcoordinatesystem units <\hunit,\vunit> point at -132.000000 6.500000
\setplotarea x from 0 to 22, y from 0 to 1
\axis bottom / % ticks numbered from 0 to 22 by 22 /
\axis left /
\axis top /
\axis right /
%locus  35:
\multiput {\tiny $\circ$} at 0 0.000000 1 0.526710 2 1.000000 3 0.502977 4 0.687947
 5 0.224274 6 0.410202 7 0.019729 8 0.027446 9 0.000000
 10 0.000000 11 0.000000 12 0.000000 13 0.000000 14 0.000000
 15 0.000000 16 0.000000 17 0.000000 18 0.000000 19 0.000000
 20 0.000000 21 0.000000
/
% row 6 col 0 xpoint=0.000000 ypoint=7.800000
%%%%%%%%%%%% Plot figure in row 6 col 0 %%%%%%%%%%%%
\setcoordinatesystem units <\hunit,\vunit> point at 0.000000 7.800000
\setplotarea x from 0 to 22, y from 0 to 1
\axis bottom / % ticks numbered from 0 to 22 by 22 /
\axis left /
\axis top /
\axis right /
%locus  36:
\multiput {\tiny $\circ$} at 0 0.000000 1 1.000000 2 0.748369 3 0.509341 4 0.367677
 5 0.281426 6 0.202353 7 0.110676 8 0.047395 9 0.051796
 10 0.079166 11 0.116437 12 0.034791 13 0.003188 14 0.000671
 15 0.000000 16 0.000000 17 0.000000 18 0.000000 19 0.000000
 20 0.000000 21 0.000000
/
% row 6 col 1 xpoint=-26.400000 ypoint=7.800000
%%%%%%%%%%%% Plot figure in row 6 col 1 %%%%%%%%%%%%
\setcoordinatesystem units <\hunit,\vunit> point at -26.400000 7.800000
\setplotarea x from 0 to 22, y from 0 to 1
\axis bottom / % ticks numbered from 0 to 22 by 22 /
\axis left /
\axis top /
\axis right /
%locus  37:
\multiput {\tiny $\circ$} at 0 0.000000 1 1.000000 2 0.384982 3 0.179703 4 0.111262
 5 0.302783 6 0.037349 7 0.002287 8 0.000235 9 0.000000
 10 0.000000 11 0.000000 12 0.000000 13 0.000000 14 0.000000
 15 0.000000 16 0.000000 17 0.000000 18 0.000000 19 0.000000
 20 0.000000 21 0.000000
/
% row 6 col 2 xpoint=-52.799999 ypoint=7.800000
%%%%%%%%%%%% Plot figure in row 6 col 2 %%%%%%%%%%%%
\setcoordinatesystem units <\hunit,\vunit> point at -52.799999 7.800000
\setplotarea x from 0 to 22, y from 0 to 1
\axis bottom / % ticks numbered from 0 to 22 by 22 /
\axis left /
\axis top /
\axis right /
%locus  38:
\multiput {\tiny $\circ$} at 0 0.000000 1 1.000000 2 0.650193 3 0.325234 4 0.169213
 5 0.147922 6 0.115644 7 0.024546 8 0.002494 9 0.000000
 10 0.000000 11 0.000000 12 0.000000 13 0.000000 14 0.000000
 15 0.000000 16 0.000000 17 0.000000 18 0.000000 19 0.000000
 20 0.000000 21 0.000000
/
% row 6 col 3 xpoint=-79.199997 ypoint=7.800000
%%%%%%%%%%%% Plot figure in row 6 col 3 %%%%%%%%%%%%
\setcoordinatesystem units <\hunit,\vunit> point at -79.199997 7.800000
\setplotarea x from 0 to 22, y from 0 to 1
\axis bottom / % ticks numbered from 0 to 22 by 22 /
\axis left /
\axis top /
\axis right /
%locus  39:
\multiput {\tiny $\circ$} at 0 0.000000 1 1.000000 2 0.598226 3 0.447915 4 0.408586
 5 0.635017 6 0.955883 7 0.726656 8 0.330666 9 0.207763
 10 0.154874 11 0.098066 12 0.300270 13 0.115095 14 0.013624
 15 0.048326 16 0.006105 17 0.000032 18 0.000000 19 0.000000
 20 0.000000 21 0.000000
/
% row 6 col 4 xpoint=-105.599998 ypoint=7.800000
%%%%%%%%%%%% Plot figure in row 6 col 4 %%%%%%%%%%%%
\setcoordinatesystem units <\hunit,\vunit> point at -105.599998 7.800000
\setplotarea x from 0 to 22, y from 0 to 1
\axis bottom / % ticks numbered from 0 to 22 by 22 /
\axis left /
\axis top /
\axis right /
%locus  40:
\multiput {\tiny $\circ$} at 0 0.000000 1 1.000000 2 0.893653 3 0.884731 4 0.680152
 5 0.433693 6 0.402230 7 0.271325 8 0.109859 9 0.043920
 10 0.045510 11 0.008850 12 0.002325 13 0.000261 14 0.000024
 15 0.000000 16 0.000000 17 0.000000 18 0.000000 19 0.000000
 20 0.000000 21 0.000000
/
% row 6 col 5 xpoint=-132.000000 ypoint=7.800000
%%%%%%%%%%%% Plot figure in row 6 col 5 %%%%%%%%%%%%
\setcoordinatesystem units <\hunit,\vunit> point at -132.000000 7.800000
\setplotarea x from 0 to 22, y from 0 to 1
\axis bottom / % ticks numbered from 0 to 22 by 22 /
\axis left /
\axis top /
\axis right /
%locus  41:
\multiput {\tiny $\circ$} at 0 0.000000 1 1.000000 2 0.181531 3 0.547165 4 0.107659
 5 0.001611 6 0.000066 7 0.000000 8 0.000000 9 0.000000
 10 0.000000 11 0.000000 12 0.000000 13 0.000000 14 0.000000
 15 0.000000 16 0.000000 17 0.000000 18 0.000000 19 0.000000
 20 0.000000 21 0.000000
/
% row 7 col 0 xpoint=0.000000 ypoint=9.100000
%%%%%%%%%%%% Plot figure in row 7 col 0 %%%%%%%%%%%%
\setcoordinatesystem units <\hunit,\vunit> point at 0.000000 9.100000
\setplotarea x from 0 to 22, y from 0 to 1
\axis bottom / % ticks numbered from 0 to 22 by 22 /
\axis left /
\axis top /
\axis right /
%locus  42:
\multiput {\tiny $\circ$} at 0 0.000000 1 1.000000 2 0.350481 3 0.177607 4 0.151984
 5 0.088470 6 0.216093 7 0.135346 8 0.000771 9 0.000035
 10 0.000000 11 0.000000 12 0.000000 13 0.000000 14 0.000000
 15 0.000000 16 0.000000 17 0.000000 18 0.000000 19 0.000000
 20 0.000000 21 0.000000
/
% row 7 col 1 xpoint=-26.400000 ypoint=9.100000
%%%%%%%%%%%% Plot figure in row 7 col 1 %%%%%%%%%%%%
\setcoordinatesystem units <\hunit,\vunit> point at -26.400000 9.100000
\setplotarea x from 0 to 22, y from 0 to 1
\axis bottom / % ticks numbered from 0 to 22 by 22 /
\axis left /
\axis top /
\axis right /
%locus  43:
\multiput {\tiny $\circ$} at 0 0.000000 1 0.760082 2 1.000000 3 0.833511 4 0.502439
 5 0.657243 6 0.263195 7 0.085347 8 0.076380 9 0.019699
 10 0.001138 11 0.000023 12 0.000000 13 0.000000 14 0.000000
 15 0.000000 16 0.000000 17 0.000000 18 0.000000 19 0.000000
 20 0.000000 21 0.000000
/
% row 7 col 2 xpoint=-52.799999 ypoint=9.100000
%%%%%%%%%%%% Plot figure in row 7 col 2 %%%%%%%%%%%%
\setcoordinatesystem units <\hunit,\vunit> point at -52.799999 9.100000
\setplotarea x from 0 to 22, y from 0 to 1
\axis bottom / % ticks numbered from 0 to 22 by 22 /
\axis left /
\axis top /
\axis right /
%locus  44:
\multiput {\tiny $\circ$} at 0 0.000000 1 1.000000 2 0.624341 3 0.408725 4 0.146534
 5 0.023160 6 0.000665 7 0.000000 8 0.000000 9 0.000000
 10 0.000000 11 0.000000 12 0.000000 13 0.000000 14 0.000000
 15 0.000000 16 0.000000 17 0.000000 18 0.000000 19 0.000000
 20 0.000000 21 0.000000
/
% row 7 col 3 xpoint=-79.199997 ypoint=9.100000
%%%%%%%%%%%% Plot figure in row 7 col 3 %%%%%%%%%%%%
\setcoordinatesystem units <\hunit,\vunit> point at -79.199997 9.100000
\setplotarea x from 0 to 22, y from 0 to 1
\axis bottom / % ticks numbered from 0 to 22 by 22 /
\axis left /
\axis top /
\axis right /
%locus  45:
\multiput {\tiny $\circ$} at 0 0.000000 1 1.000000 2 0.614175 3 0.345977 4 0.218659
 5 0.076784 6 0.033387 7 0.001978 8 0.000000 9 0.000000
 10 0.000000 11 0.000000 12 0.000000 13 0.000000 14 0.000000
 15 0.000000 16 0.000000 17 0.000000 18 0.000000 19 0.000000
 20 0.000000 21 0.000000
/
% row 7 col 4 xpoint=-105.599998 ypoint=9.100000
%%%%%%%%%%%% Plot figure in row 7 col 4 %%%%%%%%%%%%
\setcoordinatesystem units <\hunit,\vunit> point at -105.599998 9.100000
\setplotarea x from 0 to 22, y from 0 to 1
\axis bottom / % ticks numbered from 0 to 22 by 22 /
\axis left /
\axis top /
\axis right /
%locus  46:
\multiput {\tiny $\circ$} at 0 0.000000 1 1.000000 2 0.239711 3 0.160555 4 0.666777
 5 0.072677 6 0.012641 7 0.003681 8 0.001029 9 0.000303
 10 0.000000 11 0.000000 12 0.000000 13 0.000000 14 0.000000
 15 0.000000 16 0.000000 17 0.000000 18 0.000000 19 0.000000
 20 0.000000 21 0.000000
/
% row 7 col 5 xpoint=-132.000000 ypoint=9.100000
%%%%%%%%%%%% Plot figure in row 7 col 5 %%%%%%%%%%%%
\setcoordinatesystem units <\hunit,\vunit> point at -132.000000 9.100000
\setplotarea x from 0 to 22, y from 0 to 1
\axis bottom / % ticks numbered from 0 to 22 by 22 /
\axis left /
\axis top /
\axis right /
%locus  47:
\multiput {\tiny $\circ$} at 0 0.000000 1 1.000000 2 0.958639 3 0.562616 4 0.614000
 5 0.369123 6 0.432948 7 0.219737 8 0.180465 9 0.046337
 10 0.013779 11 0.000634 12 0.000000 13 0.000000 14 0.000000
 15 0.000000 16 0.000000 17 0.000000 18 0.000000 19 0.000000
 20 0.000000 21 0.000000
/
% row 8 col 0 xpoint=0.000000 ypoint=10.400000
%%%%%%%%%%%% Plot figure in row 8 col 0 %%%%%%%%%%%%
\setcoordinatesystem units <\hunit,\vunit> point at 0.000000 10.400000
\setplotarea x from 0 to 22, y from 0 to 1
\axis bottom / % ticks numbered from 0 to 22 by 22 /
\axis left /
\axis top /
\axis right /
%locus  48:
\multiput {\tiny $\circ$} at 0 0.000000 1 1.000000 2 0.747009 3 0.667185 4 0.451386
 5 0.273961 6 0.101065 7 0.009192 8 0.000000 9 0.000000
 10 0.000000 11 0.000000 12 0.000000 13 0.000000 14 0.000000
 15 0.000000 16 0.000000 17 0.000000 18 0.000000 19 0.000000
 20 0.000000 21 0.000000
/
% row 8 col 1 xpoint=-26.400000 ypoint=10.400000
%%%%%%%%%%%% Plot figure in row 8 col 1 %%%%%%%%%%%%
\setcoordinatesystem units <\hunit,\vunit> point at -26.400000 10.400000
\setplotarea x from 0 to 22, y from 0 to 1
\axis bottom / % ticks numbered from 0 to 22 by 22 /
\axis left /
\axis top /
\axis right /
%locus  49:
\multiput {\tiny $\circ$} at 0 0.000000 1 1.000000 2 0.856208 3 0.379915 4 0.461700
 5 0.083273 6 0.087850 7 0.008609 8 0.001104 9 0.000016
 10 0.000000 11 0.000000 12 0.000000 13 0.000000 14 0.000000
 15 0.000000 16 0.000000 17 0.000000 18 0.000000 19 0.000000
 20 0.000000 21 0.000000
/
% row 8 col 2 xpoint=-52.799999 ypoint=10.400000
%%%%%%%%%%%% Plot figure in row 8 col 2 %%%%%%%%%%%%
\setcoordinatesystem units <\hunit,\vunit> point at -52.799999 10.400000
\setplotarea x from 0 to 22, y from 0 to 1
\axis bottom / % ticks numbered from 0 to 22 by 22 /
\axis left /
\axis top /
\axis right /
%locus  50:
\multiput {\tiny $\circ$} at 0 0.000000 1 1.000000 2 0.463738 3 0.133482 4 0.125955
 5 0.180889 6 0.077101 7 0.003769 8 0.000034 9 0.000000
 10 0.000000 11 0.000000 12 0.000000 13 0.000000 14 0.000000
 15 0.000000 16 0.000000 17 0.000000 18 0.000000 19 0.000000
 20 0.000000 21 0.000000
/
% row 8 col 3 xpoint=-79.199997 ypoint=10.400000
%%%%%%%%%%%% Plot figure in row 8 col 3 %%%%%%%%%%%%
\setcoordinatesystem units <\hunit,\vunit> point at -79.199997 10.400000
\setplotarea x from 0 to 22, y from 0 to 1
\axis bottom / % ticks numbered from 0 to 22 by 22 /
\axis left /
\axis top /
\axis right /
%locus  51:
\multiput {\tiny $\circ$} at 0 0.000000 1 1.000000 2 0.369494 3 0.105002 4 0.242439
 5 0.152757 6 0.020054 7 0.000780 8 0.000169 9 0.000000
 10 0.000000 11 0.000000 12 0.000000 13 0.000000 14 0.000000
 15 0.000000 16 0.000000 17 0.000000 18 0.000000 19 0.000000
 20 0.000000 21 0.000000
/
% row 8 col 4 xpoint=-105.599998 ypoint=10.400000
%%%%%%%%%%%% Plot figure in row 8 col 4 %%%%%%%%%%%%
\setcoordinatesystem units <\hunit,\vunit> point at -105.599998 10.400000
\setplotarea x from 0 to 22, y from 0 to 1
\axis bottom / % ticks numbered from 0 to 22 by 22 /
\axis left /
\axis top /
\axis right /
%locus  52:
\multiput {\tiny $\circ$} at 0 0.000000 1 1.000000 2 0.743360 3 0.465240 4 0.272376
 5 0.189045 6 0.132462 7 0.052146 8 0.005973 9 0.000747
 10 0.000000 11 0.000000 12 0.000000 13 0.000000 14 0.000000
 15 0.000000 16 0.000000 17 0.000000 18 0.000000 19 0.000000
 20 0.000000 21 0.000000
/
% row 8 col 5 xpoint=-132.000000 ypoint=10.400000
%%%%%%%%%%%% Plot figure in row 8 col 5 %%%%%%%%%%%%
\setcoordinatesystem units <\hunit,\vunit> point at -132.000000 10.400000
\setplotarea x from 0 to 22, y from 0 to 1
\axis bottom / % ticks numbered from 0 to 22 by 22 /
\axis left /
\axis top /
\axis right /
%locus  53:
\multiput {\tiny $\circ$} at 0 0.000000 1 1.000000 2 0.361628 3 0.172818 4 0.190566
 5 0.375743 6 0.259816 7 0.323050 8 0.079195 9 0.001459
 10 0.000089 11 0.000000 12 0.000000 13 0.000000 14 0.000000
 15 0.000000 16 0.000000 17 0.000000 18 0.000000 19 0.000000
 20 0.000000 21 0.000000
/
% row 9 col 0 xpoint=0.000000 ypoint=11.700000
%%%%%%%%%%%% Plot figure in row 9 col 0 %%%%%%%%%%%%
\setcoordinatesystem units <\hunit,\vunit> point at 0.000000 11.700000
\setplotarea x from 0 to 22, y from 0 to 1
\axis bottom ticks short numbered from 0 to 22 by 22 /
\axis left /
\axis top /
\axis right /
%locus  54:
\multiput {\tiny $\circ$} at 0 0.000000 1 1.000000 2 0.200661 3 0.150521 4 0.042382
 5 0.671805 6 0.049897 7 0.000515 8 0.000000 9 0.000000
 10 0.000000 11 0.000000 12 0.000000 13 0.000000 14 0.000000
 15 0.000000 16 0.000000 17 0.000000 18 0.000000 19 0.000000
 20 0.000000 21 0.000000
/
% row 9 col 1 xpoint=-26.400000 ypoint=11.700000
%%%%%%%%%%%% Plot figure in row 9 col 1 %%%%%%%%%%%%
\setcoordinatesystem units <\hunit,\vunit> point at -26.400000 11.700000
\setplotarea x from 0 to 22, y from 0 to 1
\axis bottom / % ticks numbered from 0 to 22 by 22 /
\axis left /
\axis top /
\axis right /
%locus  55:
\multiput {\tiny $\circ$} at 0 0.000000 1 1.000000 2 0.674208 3 0.360849 4 0.360793
 5 0.160830 6 0.151084 7 0.001445 8 0.000000 9 0.000000
 10 0.000000 11 0.000000 12 0.000000 13 0.000000 14 0.000000
 15 0.000000 16 0.000000 17 0.000000 18 0.000000 19 0.000000
 20 0.000000 21 0.000000
/
% row 9 col 2 xpoint=-52.799999 ypoint=11.700000
%%%%%%%%%%%% Plot figure in row 9 col 2 %%%%%%%%%%%%
\setcoordinatesystem units <\hunit,\vunit> point at -52.799999 11.700000
\setplotarea x from 0 to 22, y from 0 to 1
\axis bottom / % ticks numbered from 0 to 22 by 22 /
\axis left /
\axis top /
\axis right /
%locus  56:
\multiput {\tiny $\circ$} at 0 0.000000 1 1.000000 2 0.002801 3 0.009434 4 0.002775
 5 0.839623 6 0.072142 7 0.000000 8 0.000000 9 0.000000
 10 0.000000 11 0.000000 12 0.000000 13 0.000000 14 0.000000
 15 0.000000 16 0.000000 17 0.000000 18 0.000000 19 0.000000
 20 0.000000 21 0.000000
/
% row 9 col 3 xpoint=-79.199997 ypoint=11.700000
%%%%%%%%%%%% Plot figure in row 9 col 3 %%%%%%%%%%%%
\setcoordinatesystem units <\hunit,\vunit> point at -79.199997 11.700000
\setplotarea x from 0 to 22, y from 0 to 1
\axis bottom / % ticks numbered from 0 to 22 by 22 /
\axis left /
\axis top /
\axis right /
%locus  57:
\multiput {\tiny $\circ$} at 0 0.000000 1 1.000000 2 0.801459 3 0.585379 4 0.384493
 5 0.224644 6 0.147303 7 0.135101 8 0.151111 9 0.170350
 10 0.153845 11 0.114571 12 0.050744 13 0.005767 14 0.000990
 15 0.000000 16 0.000000 17 0.000000 18 0.000000 19 0.000000
 20 0.000000 21 0.000000
/
% row 9 col 4 xpoint=-105.599998 ypoint=11.700000
%%%%%%%%%%%% Plot figure in row 9 col 4 %%%%%%%%%%%%
\setcoordinatesystem units <\hunit,\vunit> point at -105.599998 11.700000
\setplotarea x from 0 to 22, y from 0 to 1
\axis bottom / % ticks numbered from 0 to 22 by 22 /
\axis left /
\axis top /
\axis right /
%locus  58:
\multiput {\tiny $\circ$} at 0 0.000000 1 1.000000 2 0.894483 3 0.755043 4 0.588475
 5 0.430681 6 0.310049 7 0.229975 8 0.143507 9 0.095109
 10 0.064583 11 0.018995 12 0.005913 13 0.003077 14 0.001525
 15 0.000000 16 0.000000 17 0.000000 18 0.000000 19 0.000000
 20 0.000000 21 0.000000
/
% row 9 col 5 xpoint=-132.000000 ypoint=11.700000
%%%%%%%%%%%% Plot figure in row 9 col 5 %%%%%%%%%%%%
\setcoordinatesystem units <\hunit,\vunit> point at -132.000000 11.700000
\setplotarea x from 0 to 22, y from 0 to 1
\axis bottom / % ticks numbered from 0 to 22 by 22 /
\axis left /
\axis top /
\axis right /
%locus  59:
\multiput {\tiny $\circ$} at 0 0.000000 1 1.000000 2 0.943628 3 0.798519 4 0.594769
 5 0.506300 6 0.333333 7 0.217753 8 0.160397 9 0.074227
 10 0.037292 11 0.036815 12 0.011197 13 0.000448 14 0.003643
 15 0.000000 16 0.000119 17 0.000030 18 0.000000 19 0.000000
 20 0.000000 21 0.000000
/
\endpicture}
\end{center}
\caption{Mismatch distributions of 60 STR loci\label{fig.strmm}}
\end{figure}


Figure~\ref{fig.strmm} shows the mismatch distributions from our sample of 60
STR loci.  In each panel there, the horizontal axis shows the magnitude of the
pairwise difference, and the vertical axis shows the numbers of pairs of
chromosomes in the sample that exhibited such a difference.

\paragraph{Doing theory with samples of size 2}
Rather than attempt a statistical theory for mismatch distributions from
samples of size $n$, I work instead with theory for samples of size~2.  This
simpler theory should provide some guidance, since the expected values of the
components of a mismatch distribution do not depend on sample size
\citep[footnote~4, p.~61]{Rogers:PPG-97-55}.  Nonetheless, other statistical
properties \emph{do} depend on sample size, so caution is in order.  The
simple theory of random pairs will provide insight concerning which statistics
are likely to be useful.  Before these insights can be trusted, however, they
must be tested by computer simulation.

\paragraph{The stepwise mutation model}
I use a very simple model of mutation, which assumes that each
mutation either adds or subtracts exactly one repeat unit, the two
alternatives having equal probability.

\def\hunit{0.015in}
\begin{figure}
\begin{center}
\mbox{\beginpicture
\valuestolabelleading=.4\baselineskip
%                                   MMGEN
%                         (generate simulated data)
%                             by Alan R. Rogers
%                             Version 3.3 alpha
%                                 7 Jan 1997
%                         Type `mmgen -- ' for help
%
%     theta         mn        tau          K
% 1000.0000     0.0000    35.0000          1
%   10.0000     0.0000        Inf          1
%
%Stepwise mutation model with mutational distribution:
%     Step     Prob
%       -1  0.50000
%        0  0.00000
%        1  0.50000
%Maximum # of stepwise differences: 100
%Using histograms of size 100
%Allowing 100 mutations
%iterations=0, sampsize/subpop=100 total sampsize=100
\setcoordinatesystem units <\hunit, 0.00125in> point at 0 0
\setplotarea x from 0 to 100, y from 0 to 1000
\plotheading{Recent expansion}
\axis left 
  label {$4Nu$}
  ticks  withvalues 10 1000 / at 10 1000 / /
\axis bottom 
  label {$\tau$}
  ticks numbered from 0 to 100 by  20 /
\plot 0 1000 35 1000 35 10 100 10 /
%%%%%%%%%%%%%%%%%%%%%%%%%%%%%%%%%%%%%%
\setcoordinatesystem units <\hunit, 30.487in>    point at 0 0.0615
\setplotarea x from 0 to 100, y from 0 to 0.041
\axis bottom label {$x$}
  ticks numbered from 0 to 100 by  20 /
\axis left label {$f_x$} /
\axis right /
\axis top /
%
%Pr[random pair differs by i sites]
\plot 0 0.000999 1 0.000998 2 0.000997 3 0.000996 4 0.000995
      5 0.000994 6 0.000993 7 0.000992 8 0.000991 9 0.000990
      10 0.000989 11 0.000988 12 0.000988 13 0.000988 14 0.000989
      15 0.000994 16 0.001005 17 0.001030 18 0.001077 19 0.001165
      20 0.001319 21 0.001572 22 0.001970 23 0.002569 24 0.003428
      25 0.004612 26 0.006175 27 0.008156 28 0.010570 29 0.013398
      30 0.016584 31 0.020035 32 0.023625 33 0.027208 34 0.030627
      35 0.033729 36 0.036380 37 0.038478 38 0.039956 39 0.040787
      40 0.040983 41 0.040589 42 0.039675 43 0.038327 44 0.036639
      45 0.034705 46 0.032612 47 0.030439 48 0.028248 49 0.026092
      50 0.024009 51 0.022024 52 0.020155 53 0.018411 54 0.016794
      55 0.015303 56 0.013935 57 0.012682 58 0.011537 59 0.010494
      60 0.009543 61 0.008677 62 0.007889 63 0.007172 64 0.006520
      65 0.005928 66 0.005389 67 0.004899 68 0.004454 69 0.004049
      70 0.003681 71 0.003346 72 0.003042 73 0.002765 74 0.002514
      75 0.002286 76 0.002078 77 0.001889 78 0.001717 79 0.001561
      80 0.001419 81 0.001290 82 0.001173 83 0.001066 84 0.000969
      85 0.000881 86 0.000801 87 0.000728 88 0.000662 89 0.000602
      90 0.000547 91 0.000497 92 0.000452 93 0.000411 94 0.000374
      95 0.000340 96 0.000309 97 0.000281 98 0.000255 99 0.000232
/
%%%%%%%%%%%%%%%%%%%%%%%%%%%%%%%%%%%%%%
\setcoordinatesystem units <\hunit, 10.16in>    point at 0 0.369
\setplotarea x from 0 to 100, y from 0 to 0.123
\axis bottom label {$y$}  ticks numbered from 0 to 100 by  20 /
\axis left label {$g_y$} /
\axis right /
\axis top /
%Pr[random pair differs by |i| steps]
\plot 0 0.063010 1 0.122798 2 0.117158 3 0.109442 4 0.100004
      5 0.089430 6 0.078212 7 0.066952 8 0.056066 9 0.045986
      10 0.036923 11 0.029068 12 0.022423 13 0.016985 14 0.012623
      15 0.009229 16 0.006629 17 0.004696 18 0.003272 19 0.002255
      20 0.001530 21 0.001030 22 0.000684 23 0.000452 24 0.000294
      25 0.000192 26 0.000123 27 0.000079 28 0.000050 29 0.000032
      30 0.000020 31 0.000013 32 0.000008 33 0.000005 34 0.000003
      35 0.000002 36 0.000001 37 0.000001 38 0.000000 39 0.000000
      40 0.000000 41 0.000000 42 0.000000 43 0.000000 44 0.000000
      45 0.000000 46 0.000000 47 0.000000 48 0.000000 49 0.000000
      50 0.000000 51 0.000000 52 0.000000 53 0.000000 54 0.000000
      55 0.000000 56 0.000000 57 0.000000 58 0.000000 59 0.000000
      60 0.000000 61 0.000000 62 0.000000 63 0.000000 64 0.000000
      65 0.000000 66 0.000000 67 0.000000 68 0.000000 69 0.000000
      70 0.000000 71 0.000000 72 0.000000 73 0.000000 74 0.000000
      75 0.000000 76 0.000000 77 0.000000 78 0.000000 79 0.000000
      80 0.000000 81 0.000000 82 0.000000 83 0.000000 84 0.000000
      85 0.000000 86 0.000000 87 0.000000 88 0.000000 89 0.000000
      90 0.000000 91 0.000000 92 0.000000 93 0.000000 94 0.000000
      95 0.000000 96 0.000000 97 0.000000 98 0.000000 99 0.002321
/
\multiput {$\circ$} at
      0 0.063010 1 0.122798 2 0.117158 3 0.109442 4 0.100004
      5 0.089430 6 0.078212 7 0.066952 8 0.056066 9 0.045986
      10 0.036923 11 0.029068 12 0.022423 13 0.016985 14 0.012623
      15 0.009229 16 0.006629 17 0.004696 18 0.003272 19 0.002255
      20 0.001530 21 0.001030 22 0.000684 23 0.000452 24 0.000294
      25 0.000192 26 0.000123 27 0.000079 28 0.000050 29 0.000032
      30 0.000020 31 0.000013 32 0.000008 33 0.000005 34 0.000003
      35 0.000002 36 0.000001 37 0.000001 38 0.000000 39 0.000000
      40 0.000000 41 0.000000 42 0.000000 43 0.000000 44 0.000000
      45 0.000000 46 0.000000 47 0.000000 48 0.000000 49 0.000000
      50 0.000000 51 0.000000 52 0.000000 53 0.000000 54 0.000000
      55 0.000000 56 0.000000 57 0.000000 58 0.000000 59 0.000000
      60 0.000000 61 0.000000 62 0.000000 63 0.000000 64 0.000000
      65 0.000000 66 0.000000 67 0.000000 68 0.000000 69 0.000000
      70 0.000000 71 0.000000 72 0.000000 73 0.000000 74 0.000000
      75 0.000000 76 0.000000 77 0.000000 78 0.000000 79 0.000000
      80 0.000000 81 0.000000 82 0.000000 83 0.000000 84 0.000000
      85 0.000000 86 0.000000 87 0.000000 88 0.000000 89 0.000000
      90 0.000000 91 0.000000 92 0.000000 93 0.000000 94 0.000000
      95 0.000000 96 0.000000 97 0.000000 98 0.000000 99 0.002321
/
%%%%%%%%%%%%%%%%%%%%%%%%%%%%%%%%%%%%%%%%%%%%%%%%%%%%%%%%%%%%%%%%
%                                   MMGEN
%                         (generate simulated data)
%                             by Alan R. Rogers
%                             Version 3.3 alpha
%                                 7 Jan 1997
%                         Type `mmgen -- ' for help
%
%     theta         mn        tau          K
% 1000.0000     0.0000    70.0000          1
%   10.0000     0.0000        Inf          1
%Stepwise mutation model with mutational distribution:
%     Step     Prob
%       -1  0.50000
%        0  0.00000
%        1  0.50000
%Maximum # of stepwise differences: 100
%Using histograms of size 100
%Allowing 100 mutations
%iterations=0, sampsize/subpop=100 total sampsize=100
\setcoordinatesystem units <\hunit, 0.00125in> point at -133 0
\setplotarea x from 0 to 100, y from 0 to 1000
\plotheading{Ancient expansion}
\axis left 
  ticks  withvalues 10 1000 / at 10 1000 / /
\axis bottom 
  label {$\tau$}
  ticks numbered from 0 to 100 by  20 /
%\put{\em Population Size} [rt] at 100 1000
\plot 0 1000 70 1000 70 10 100 10 /
%%%%%%%%%%%%%%%%%%%%%%%%%%%%%%%%%%%%%%
\setcoordinatesystem units <\hunit, 37.87in>    point at -133 0.0495
\setplotarea x from 0 to 100, y from 0 to 0.033
\axis bottom label {$x$}
  ticks numbered from 0 to 100 by  20 /
\axis left /
\axis right /
\axis top /
%
%Pr[random pair differs by i sites]
\plot
      0 0.000999 1 0.000998 2 0.000997 3 0.000996 4 0.000995
      5 0.000994 6 0.000993 7 0.000992 8 0.000991 9 0.000990
      10 0.000989 11 0.000988 12 0.000987 13 0.000986 14 0.000985
      15 0.000984 16 0.000983 17 0.000982 18 0.000981 19 0.000980
      20 0.000979 21 0.000978 22 0.000977 23 0.000976 24 0.000975
      25 0.000974 26 0.000973 27 0.000972 28 0.000971 29 0.000970
      30 0.000969 31 0.000969 32 0.000968 33 0.000967 34 0.000966
      35 0.000965 36 0.000964 37 0.000964 38 0.000963 39 0.000964
      40 0.000965 41 0.000968 42 0.000974 43 0.000983 44 0.000999
      45 0.001024 46 0.001062 47 0.001119 48 0.001201 49 0.001318
      50 0.001480 51 0.001701 52 0.001995 53 0.002379 54 0.002869
      55 0.003485 56 0.004241 57 0.005155 58 0.006235 59 0.007491
      60 0.008920 61 0.010518 62 0.012270 63 0.014153 64 0.016136
      65 0.018181 66 0.020247 67 0.022285 68 0.024246 69 0.026082
      70 0.027747 71 0.029200 72 0.030408 73 0.031345 74 0.031995
      75 0.032351 76 0.032415 77 0.032199 78 0.031722 79 0.031009
      80 0.030088 81 0.028993 82 0.027757 83 0.026414 84 0.024996
      85 0.023534 86 0.022053 87 0.020579 88 0.019129 89 0.017722
      90 0.016369 91 0.015079 92 0.013859 93 0.012713 94 0.011642
      95 0.010646 96 0.009723 97 0.008872 98 0.008089 99 0.007370
/
%%%%%%%%%%%%%%%%%%%%%%%%%%%%%%%%%%%%%%
\setcoordinatesystem units <\hunit, 13.88in>    point at -133 0.27
\setplotarea x from 0 to 100, y from 0 to 0.09
\axis bottom label {$y$}  ticks numbered from 0 to 100 by  20 /
\axis left /
\axis right /
\axis top /
%Pr[random pair differs by |i| steps]
\plot
      0 0.045414 1 0.088980 2 0.085219 3 0.081807 4 0.076662
      5 0.071999 6 0.065927 7 0.060514 8 0.054093 9 0.048498
      10 0.042296 11 0.037029 12 0.031495 13 0.026922 14 0.022326
      15 0.018636 16 0.015066 17 0.012282 18 0.009679 19 0.007709
      20 0.005922 21 0.004610 22 0.003452 23 0.002627 24 0.001917
      25 0.001428 26 0.001016 27 0.000740 28 0.000513 29 0.000366
      30 0.000247 31 0.000173 32 0.000114 33 0.000078 34 0.000050
      35 0.000034 36 0.000021 37 0.000014 38 0.000008 39 0.000005
      40 0.000003 41 0.000002 42 0.000001 43 0.000001 44 0.000000
      45 0.000000 46 0.000000 47 0.000000 48 0.000000 49 0.000000
      50 0.000000 51 0.000000 52 0.000000 53 0.000000 54 0.000000
      55 0.000000 56 0.000000 57 0.000000 58 0.000000 59 0.000000
      60 0.000000 61 0.000000 62 0.000000 63 0.000000 64 0.000000
      65 0.000000 66 0.000000 67 0.000000 68 0.000000 69 0.000000
      70 0.000000 71 0.000000 72 0.000000 73 0.000000 74 0.000000
      75 0.000000 76 0.000000 77 0.000000 78 0.000000 79 0.000000
      80 0.000000 81 0.000000 82 0.000000 83 0.000000 84 0.000000
      85 0.000000 86 0.000000 87 0.000000 88 0.000000 89 0.000000
      90 0.000000 91 0.000000 92 0.000000 93 0.000000 94 0.000000
      95 0.000000 96 0.000000 97 0.000000 98 0.000000 %99 0.074100
/
\multiput {$\circ$} at
      0 0.045414 1 0.088980 2 0.085219 3 0.081807 4 0.076662
      5 0.071999 6 0.065927 7 0.060514 8 0.054093 9 0.048498
      10 0.042296 11 0.037029 12 0.031495 13 0.026922 14 0.022326
      15 0.018636 16 0.015066 17 0.012282 18 0.009679 19 0.007709
      20 0.005922 21 0.004610 22 0.003452 23 0.002627 24 0.001917
      25 0.001428 26 0.001016 27 0.000740 28 0.000513 29 0.000366
      30 0.000247 31 0.000173 32 0.000114 33 0.000078 34 0.000050
      35 0.000034 36 0.000021 37 0.000014 38 0.000008 39 0.000005
      40 0.000003 41 0.000002 42 0.000001 43 0.000001 44 0.000000
      45 0.000000 46 0.000000 47 0.000000 48 0.000000 49 0.000000
      50 0.000000 51 0.000000 52 0.000000 53 0.000000 54 0.000000
      55 0.000000 56 0.000000 57 0.000000 58 0.000000 59 0.000000
      60 0.000000 61 0.000000 62 0.000000 63 0.000000 64 0.000000
      65 0.000000 66 0.000000 67 0.000000 68 0.000000 69 0.000000
      70 0.000000 71 0.000000 72 0.000000 73 0.000000 74 0.000000
      75 0.000000 76 0.000000 77 0.000000 78 0.000000 79 0.000000
      80 0.000000 81 0.000000 82 0.000000 83 0.000000 84 0.000000
      85 0.000000 86 0.000000 87 0.000000 88 0.000000 89 0.000000
      90 0.000000 91 0.000000 92 0.000000 93 0.000000 94 0.000000
      95 0.000000 96 0.000000 97 0.000000 98 0.000000 %99 0.074100
/
\endpicture}
\end{center}
\caption{Effect of history on mutational differences and stepwise
differences.  In each column, the upper panel shows the
history of population size on a time scale in which each unit equals
$1/(2u)$ generations.  The middle panel shows the probability $f_x$
that a random pair of individuals differ by $x$ mutational changes and
the lower panel shows the probability $g_y$ that they differ by $y$
steps under the stepwise mutational process.}
\label{fig.sw}
\end{figure}


We already have a theory describing the effect of population history
on mutational differences \citep{Li:G-85-331, Rogers:MBE-9-552}, and the
stepwise mutational assumption makes it easy to calculate the
probability that two chromosomes differ by $y$ repeat units given that
they are separated by $x$ mutations.

Figure~\ref{fig.sw} shows the effect of two population histories.  The
distribution of mutational changes shows a pronounced wave whose
position reflects the time of the population expansion.  This
principle has proved useful in estimating expansion times from
mitochondrial sequence data, where most mutations can be detected in
comparisons between sequences \citep{Rogers:E-49-608}.  But the lower
panel shows that the distribution of stepwise differences is far less
sensitive to differences in population history.  Each mutation has an
even chance of erasing the effect of the preceding mutation, and
changes in the distribution are subtle.  On the other hand, our sample
of 60 STR polymorphisms provides information about 60 essentially
independent gene genealogies, whereas the mtDNA provide information
about just one.  It is not yet clear which will be more informative.

The different histories shown in figure~\ref{fig.sw} leave the mode of
the stepwise mismatch distribution unchanged but do affect its width.
Thus, it seems natural to look for look for effects on the moments of
the distribution (defined above).  One can obtain estimators of
$\theta_0$ and $\tau$ by equating the moments of the sample with
theoretical moments in a sample of size 2.  This gives
\begin{eqnarray}
\hat\theta_0 &=& \sqrt{(m_4-V)/3 - V^2}\label{eq.hattau.str}\\
\hat\tau     &=& V - \hat\theta_0 \label{eq.hatth0.str}
\end{eqnarray}
These formulas are analogous to the two-parameter method of moments
estimators introduced in lecture~\ref{ch.mismatch}.  There is no
estimator for $\theta_1$ here because the formulas assume that
$\theta_1$ is extremely large.  They are not likely to be useful
unless the population has in fact expanded.

% -*-latex-*-
% Quantiles of estimates from 10 str loci w/ initial pop history:
%     theta         mn        tau          K
% 1000.0000     0.0000     0.0000          1
%   10.0000     0.0000        Inf          1
% and with this history modified by substitution of various values of tau
% Quantiles are approximated from 1000 simulated data sets times 10 loci.
\begin{figure}[p]
\pagestyle{empty}
\begin{center}
\mbox{%
\beginpicture
\headingtoplotskip=0.5\baselineskip
\setcoordinatesystem units <.02in, .0148in> point at 0 0
\setplotarea x from 0 to 100, y from 0 to 135
\axis left label {\lines{Quantiles\cr of $\hat\tau$}} shiftedto x=-5
   ticks numbered from 0 to 125 by 25 /
\axis bottom shiftedto y=-5
  label {$\tau$} ticks numbered from 0 to 100 by 25 /
\multiput {$\bullet$} at 0 0 25 25 50 50 75 75 100 100 /
\setdots
%x=tau y=quantile 0.025 for tau hat
\plot 0 5.629322 25 26.065634 50 44.634373 75 60.677677 100 77.054039 /
\setdashes
%x=tau y=quantile 0.25 for tau hat
\plot 0 11.550186 25 35.444641 50 56.488640 75 77.832001 100 97.54421 /
\setsolid
%x=tau y=quantile 0.5 for tau hat
\plot 0 15.360001 25 39.556000 50 63.582008 75 86.358009 100 110.241989 /
\setdashes
%x=tau y=quantile 0.75 for tau hat
\plot 0 20.357555 25 45.145523 50 69.984726 75 94.590004 100 121.653992 /
\setdots
%x=tau y=quantile 0.975 for tau hat
\plot 0 33.657997 25 58.08800 50 84.214783 75 114.093994 100 143.552002 /
\endpicture%
}
\end{center}
\caption{Quantiles of $\hat\tau$ from STR simulations.  1000 10-locus
  data sets were simulated at each of several values of $\tau$, and
  each was used to estimate the model's parameters.  The bold dots
  indicate points at which $\hat\tau=\tau$.  The solid line is the
  median, the dashed lines enclose the central 50\% of the
  distribution, and the dotted lines the central 95\%.  Each simulated
  data set was generated using the coalescent algorithm with $\theta_0
  = 10$, $\theta_1 = 1000$, and $N=100$.}
\label{fig.taustr}
\end{figure}

% -*-latex-*-
\begin{figure}
\begin{center}
\mbox{%
\beginpicture
\headingtoplotskip=0.25\baselineskip
\setcoordinatesystem units <1in, .83in> point at 0 0
\setplotarea x from -1 to 1, y from -1 to 1.4
\axis left label {\lines{Quantiles\cr of $\log_{10} \hat\theta_0$}}
   shiftedto x=-1.1
   ticks numbered from -1 to 1 by 1 /
\axis bottom shiftedto y=-1.1 label {$\log_{10} \theta_0$}
   ticks numbered from -1 to 1 by 1 /
\multiput {$\bullet$} at -1 -1  0 0 1 1  /
\setdots
% x=log10 theta0 y= quantile 0.025 of log10 hat theta0_2
\setdashes
% x=log10 theta0 y= quantile 0.25 of log10 hat theta0_2
\setsolid
% x=log10 theta0 y= quantile 0.5 of log10 hat theta0_2
\setdashes
% x=log10 theta0 y= quantile 0.75 of log10 hat theta0_2
\plot -1 .9680945717 0 .9863793488 1 1.052880968 /
\setdots
% x=log10 theta0 y= quantile 0.975 of log10 hat theta0_2
\plot -1 1.241082055 0 1.261082903 1 1.405890194 /
\endpicture%
}
\end{center}
\caption{Quantiles of $\log_{10}\hat\theta_0$.  1,000 data sets were
  simulated at each of several values of $\theta_0$, and each was used
  to estimate the model's three parameters.  In each run,
  $\theta_1=10000$, $\tau=70$, and $N=100$.  97.5\% of the simulated
  estimates fell below the dotted line; 75\% fell below the dashed
  line.  The median of $\hat\theta_0$ was 0 for all three values of
  $\theta_0$.}
\label{fig.th0str}
\end{figure}

To verify that these formulas are useful as estimators, I examined
their behavior with simulated data.  At each of a series of values of
$\tau$, I generated 1000 10-locus data sets and calculated $\hat\tau$
using each data set.  The distribution of the results, shown in
figure~\ref{fig.taustr}, is gratifying.  The sampling distribution of
the estimator is relatively narrow and its central tendency increases
in response to increases in $\tau$.  There is a modest upward bias,
but since this bias can be measured, it can be corrected.  
The behavior of $\hat\theta_0$, shown in figure~\ref{fig.th0str}, is
less encouraging.  The sampling distribution is wide and its central
tendency varies little in response to changes in $\theta_0$.  Thus,
$\hat\theta_0$ does not appear to be a useful estimator of $\theta_0$.

\subsection{Results}

The present method provides a useful estimate of $\tau$ but not of
$\theta_0$.  This makes it useful for estimating the time of an expansion
but useless for establishing that an expansion has in fact occurred.
Fortunately, the study of Shriver et al.\ \citep{Shriver:GR-7-586} has shown
that the very data under study here \emph{do} imply a population expansion.
Thus, I address myself here to the task of estimating the time of this
expansion.  

\begin{table}
\centering
\caption{Estimates of $\tau$\label{tab.tau}}
\begin{tabular}{lcccccc}
\hline
           & \multicolumn{3}{c}{60 STR loci$^\dag$}
           & \multicolumn{3}{c}{mtDNA$^\ddag$}\\ \cline{2-4}\cline{5-6}
Population & L$^*$ & \rule{0pt}{12pt}$\bar{\hat\tau}$ & U$^{**}$ 
           & L$^*$ &$\hat\tau$& U$^{**}$ \\
\hline
Africa   & 3.83 & 5.75 & 8.11 & 2.5 & 12.0 & 27.5 \\
Asia     & 2.92 & 4.80 & 7.19 & 3.5 & 8.4 & 12.5 \\
Europe   & 3.04 & 5.37 & 8.51 & --- & 3.6 &  ---  \\
\hline
\multicolumn{6}{l}{$^\dag$Bootstrapped confidence intervals}\\
\multicolumn{6}{l}{$^\ddag$Simulation-based confidence intervals}\\
\multicolumn{6}{l}{$^*$Lower end of 95\% confidence interval}\\
\multicolumn{6}{l}{$^{**}$Upper end of confidence interval.}\\
\end{tabular}
\end{table}

To estimate $\tau$ from STR data, I applied
equation~\ref{eq.hattau.str} to each of the 60 loci.  The averages of
these estimates are shown in table~\ref{tab.tau} under the column
labeled $\bar{\hat\tau}$.  To measure the precision of the STR
estimates, I repeated the calculations with each of 10000 bootstrap
replicates.  Each bootstrap data set contains 60 STR loci, which were
selected by sampling with replacement from the original 60 loci.  The
STR and mitochondrial estimates agree in suggesting an earlier
expansion in Africa than in Europe or Asia.

\begin{table}
\centering
\caption{Estimates of $t$\label{tab.t}}
\begin{tabular}{lcccccc}
\hline
           & \multicolumn{3}{c}{60 STR loci$^\dag$}
           & \multicolumn{3}{c}{mtDNA$^\ddag$}\\ \cline{2-4}\cline{5-6}
Population & L$^*$ & \rule{0pt}{12pt}$\bar{\hat t}$ & U$^{**}$ 
           & L$^*$ &$\hat t$& U$^{**}$ \\
\hline
Africa   & 48,000 & 72,000 & 101,000 & 31,000 & 150,000 & 344,000\\
Asia     & 37,000 & 60,000 & 90,000  & 44,000 & 105,000 & 156,000 \\
Europe   & 38,000 & 67,000 & 106,000 &  ---   &  45,000 &   ---\\
\hline
\multicolumn{6}{l}{$^\dag$Bootstrapped confidence intervals}\\
\multicolumn{6}{l}{$^\ddag$Simulation-based confidence intervals}\\
\multicolumn{6}{l}{$^*$Lower end of 95\% confidence interval}\\
\multicolumn{6}{l}{$^{**}$Upper end of confidence interval.}\\
\end{tabular}
\end{table}

To convert $\tau$ from mutational time into years, we must divide by
$2u$ and multiply by the generation time, say 25 years.
\citet{Weber:HMG-2-1123} suggest that mutation rates at many STR loci
are near 1/1000 per generation.  This estimate is undoubtedly rough,
and it ignores variation across loci, but we set these issues aside in
order to obtain the estimates of $\hat t$ in the STR column of
table~\ref{tab.t}.  The corresponding estimates in the mitochondrial
column assume a per-nucleotide divergence rate of 14\% per million
years \citep[p.~1506]{Vigilant:S-253-1503}, or about $u=1/1000$ per
generation for 630 nucleotides.

For Africa and Asia, the mitochondrial and STR estimates differ by
roughly a factor of two---well within the range of statistical error.
For Europe, the two estimates are even closer.  Thus, the STR
estimates are in broad agreement with the mitochondrial estimates.

The confidence intervals are narrower with STR data than with
mitochondrial data.  The mitochondrial interval is six times as long
as the STR interval for Africa and twice as long for Asia.  For
Europe, the mtDNA interval is infinite, but the STR interval is 68,000
years.  Thus, STR estimates are appreciably more precise.  This
difference may reflect either the increased information available when
one studies 60 loci rather than one, or possibly a lack of efficiency
in the statistical method used for mtDNA.
