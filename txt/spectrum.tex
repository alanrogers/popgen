% -*-latex-*-
\reversemarginpar
\chapter{The Site Frequency Spectrum}
\label{ch.spectrum}

\section{The empirical site frequency spectrum\label{sec.empspec}}

In a sample of $K$ genes, a polymorphic site can divide the sample into 1
mutant and $K-1$ non-mutants, into 2 mutants and $K-2$ non-mutants, and so on.
There may be at most $K-1$ copies of the mutant if the site is to be
polymorphic.  In many cases we can't tell which allele is the mutant, so
category $i$ gets conflated with category $K-i$.  Such spectra are called
``folded.''  I will call a site a ``singleton'' if the mutant is present in a
single copy, a ``doubleton'' if it is present in two copies, and so on.

\subsection{An unfolded spectrum}

Consider the set of DNA sequence data below:
\begin{verbatim}
               123456
HumanSequence1 AATAGC
HumanSequence2 ..AC..
HumanSequence3 .TACT.
HumanSequence4 ..ACT.
---------------------
ChimpSequence1 AAAATC
\end{verbatim}
There are 4 human sequences and a chimpanzee sequence.  There are 6
sites of which 4 are polymorphic (segregating) within the human
sample.  We calculate the empirical spectrum by considering the sites
one at a time.
\begin{description}
\item[Site 1] is fixed and therefore does not contribute to the spectrum.
\item[Site 2] has both an A and a T within the human sample but has only an
A within the chimpanzee sample.  The odds are that the ancestor of humans and
chimps had an A at this site, so we can infer that T is the mutant allele.
Since there is only one copy of T in the human sample, site 2 is a singleton.
So far, our spectrum looks like this:
\begin{verbatim}
Singletons : 1
Doubletons :
Tripletons :
\end{verbatim}
\item[Site 3] is like site 2.  The human sample has a T and 3 As, and the
  chimp sample has only As.  We infer that T is the mutant allele and count
  this site as another singleton.  The spectrum now looks like this:
\begin{verbatim}
Singletons : 2
Doubletons :
Tripletons :
\end{verbatim}
\item[Site 4] has an A and 3 Cs, but it appears that A was the ancestral
  allele. We count this site as a tripleton, so the spectrum becomes
\begin{verbatim}
Singletons : 2
Doubletons :
Tripletons : 1
\end{verbatim}
\item[Site 5] has 2 Gs and 2 Ts.  It does not matter which of these is
  ancestral.  Either way, the site is a doubleton.  The spectrum becomes
\begin{verbatim}
Singletons : 2
Doubletons : 1
Tripletons : 1
\end{verbatim}
\item[Site 6] does not contribute to the spectrum.
\end{description}
We are done.  The empirical spectrum has 2 singletons, 1 doubleton, and 1
tripleton.

\subsection{A folded spectrum}

In the preceding section, the chimpanzee sequences were used at each site to
infer which nucleotide was ancestral and which was the mutant.  Let us now
pretend that we have no chimpanzee sequences and therefore cannot tell the the
ancestral allele from the mutant.  Instead of counting mutants, we will count
the rarest (sometimes called the minor) allele at each site.  This time,
however, I will omit the invariant sites (1 and 6), which do not contribute to
the spectrum.
\begin{description}
\item[Site 2] The rare allele, T, is present in a single copy, so this site
  contributes to the singleton category just as it did for the unfolded
  spectrum. 
\item[Site 3] Ditto: another singleton
\item[Site 4] The rare allele, A, is present in a single copy, so this site is
  a singleton.  Recall that it was a tripleton in the unfolded spectrum.
\item[Site 5] A doubleton
\end{description}
The folded spectrum looks like this:
\begin{verbatim}
Singletons : 3
Doubletons : 1
\end{verbatim}
The only difference is that site~4, which was a tripleton in the unfolded
spectrum, becomes a singleton in the folded spectrum.

In general, the $i$th category of the folded spectrum contains not only
category $i$ of the unfolded spectrum, but also category $K-i$, where $K$ is
the number of DNA sequences in the sample.

\section{The expected spectrum under neutrality and constant population size}

This section deals with the special case of selective neutrality and
constant population size. I will assume initially that we can tell mutants
from ancestral alleles so that our spectrum will be unfolded.

\subsection{A site's position in the spectrum depends on its position in
  the gene tree}

Consider the following gene tree:\\
\begin{minipage}{\textwidth}
\begin{verbatim}

     ----A------
                |
                |----B---
                |        |
     -----------         |
                         |-----------
                         |
                         |
     ------C-------------

\end{verbatim}
\end{minipage}
Mutations A and C are singletons, whereas B is a doubleton.  A
mutation in the most recent coalescent interval can only be a
singleton.  One in the next most recent interval can be either a
singleton or a doubleton.  One in the interval before that can be a
singleton, a doubleton, or a tripleton.  And so on.

\subsection{A tree with two leaves has nothing but singletons}

To get a sense of how the process works, it helps to start with a tree 
with just two leaves:\\
\begin{minipage}{\textwidth}
\begin{verbatim}

     -------------------
                        |
                        |
                        |-----------
                        |
                        |
     -------------------

    |- 2N generations --| 

\end{verbatim}
\end{minipage}
Since the hazard is $1/2N$, the mean depth of this tree is $2N$
generations and the total length is $4N$.  We expect $4Nu=\theta$ 
mutations, all of which will be singletons.

\subsection{A tree with three leaves has (on average) the same number
  of singletons and half that number of doubletons}

Now consider a tree with three leaves:\\
\begin{minipage}{\textwidth}
\begin{verbatim}

     ---------       
              |
              |--------------------
              |                    |
     ---------                     |
                                   |--------
                                   |
                                   |
     ------------------------------

     |--2N/3--|-------- 2N --------| 

              Generations

\end{verbatim}
\end{minipage}
If we could look at the spectrum just before the most recent coalescent event,
it would look just like that of the tree with two leaves: $\theta$ singletons
and no doubletons.  At the time of the coalescent event, half of these
mutations (the ones on the upper branch) become doubletons.  There is no
further change in the number of doubletons, so the expected number of
doubletons in a 3-leaf gene tree is $\theta/2$.  (We don't need to worry that
mutation will turn any of our doubletons back into singletons because, under
the infinite sites model, mutation never strikes the same site twice.)

This argument cheats a bit, because it assumes that the upper branch
is the one that bifurcated. In reality, the branch that bifurcates is
equally likely to be the upper or the lower. In either case, however,
the expected number of doubletons is $\theta/2$. A more rigorous
calculation would average across these two cases but reach the same
answer:
\[
\frac{1}{2}\left(\frac{\theta}{2} + \frac{\theta}{2} \right) = \frac{\theta}{2}
\]

The coalescent interval with 3 lines of descent begins with $\theta/2$
singletons, but then more singletons are added because of new
mutations.  How many new mutations should we expect to see?  The
interval's expected length is $2N/3$ generations (see
section~\ref{sec.ncoalescent.1}), and it contains 3 lines of descent,
so the sum of the branch lengths within this interval is (on average)
$2N$ generations.  We therefore expect $2Nu = \theta/2$ new singleton
mutations.

The number of singletons that is added is precisely equal to the
number that was lost.  Thus, the new spectrum has $\theta$ singletons
and $\theta/2$ doubletons.

\subsection{Number of tripletons in a tree with four leaves}
Suppose now that we add another interval, containing 4 branches, to
the tree shown above. The 4th branch arises because one of the 3
existing branches bifurcates, and that branch is chosen at random. If
the top branch bifurcates, each doubleton becomes a tripleton, so
we end up with $\theta/2$ tripletons. If the middle branch bifurcates,
we get the same result. But if the bottom branch bifurcates, we get no
tripletons at all. The expected number of tripletons is
\[
\frac{1}{3}\left(\frac{\theta}{2} + \frac{\theta}{2} + 0\right)
= \frac{\theta}{3}
\]

\subsection{The theoretical spectrum for an arbitrary number of
  leaves}
\label{sec.thspec-nleaves}

To extend this argument to larger samples, one must make it algebraic
\citep{Hudson:PLO-10-e0118087, Rogers:arXiv-2103.00335}. I will skip
those details here. The result (shown below) was first derived by
\citet{Fu:TPB-48-172}. 
\begin{center}
  \begin{tabular}{cl}
Sample & Expected spectrum\\
size   & (singletons, doubletons, $\ldots$)\\ \hline
2 & $\theta$\\
3 & $\theta, \quad \theta/2$\\
4 & $\theta, \quad \theta/2, \quad \theta/3$\\
5 & $\theta, \quad \theta/2, \quad \theta/3, \quad \theta/4$\\
\multicolumn{2}{c}{Etcetera}
\end{tabular}
\end{center}
It is remarkable that as we increase sample size, the number of
mutants in each category doesn't change.  We merely add a new category 
at the right side of the spectrum.


To use the theoretical formula with data, we need to substitute some estimate
of $\theta$.  We might use the mean pairwise difference, $\pi$, or the
estimator
\[
\hat\theta_S = \frac{S}{\sum_{i=1}^{K-1} 1/i}
\]
where $K$ is the number of DNA sequences in the sample.  Either of these
estimators might work, since both of them estimate $\theta$ under the
stationary neutral model (see the discussion of equation~\ref{eq.thetaS} on
page~\pageref{eq.thetaS}).  To choose between these estimators, we need some
additional criterion.

The sum of the observed spectrum is equal to the number $S$ of segregating
sites.  It would be useful if the theoretical spectrum summed to the same
value.  This turns out to be so only if $\hat\theta_S$ is used to estimate
$\theta$. 

\subsection{Folded theoretical spectra}
\label{sec.foldedspectrum}
When the spectrum is folded, we cannot distinguish category $i$ from
category $K-i$.  Consequently, the expected number in category $i$ in
the folded spectrum is the sum $\theta/i$ and $\theta/(K-i)$, the
expected numbers in the two corresponding categories in the unfolded
spectrum.  This works so long as $i < K/2$, for then $i$ and $K-i$ are
different numbers. If $K$ is even, there is an additional entry at
which $i=K/2$ and the expected folded spectrum is simply $\theta/i$.

\section{Human site frequency spectra}

Figure~\ref{fig.spec} shows all of the human site frequency spectra that I was
able to cull from the literature in the year 2000.  In each plot, the
empirical spectrum is shown as a histogram, and the expected values under
neutral evolution with constant population size are shown as bold dots.  The
top row shows three systems in which there is an excess of singletons,
compared with the stationary neutral model.  The middle row shows three
systems that seem to fit the neutral model, and the bottom row shows three
systems in which there is a deficit of singletons and an excess of sites at
intermediate frequency.

%-*-latex-*-
\let\put\pictexput
\mbox{%
\beginpicture
\headingtoplotskip=0mm
\valuestolabelleading=0.2\baselineskip
%%%%%%%%%%%%%%%% row 1 col 1 %%%%%%%%%%%%%%%%
%point at 2.125in 0in
\setcoordinatesystem units <2.5in, 1.51in> point at 0.85 0
\setplotarea x from 0 to 0.5, y from 0 to 0.83
\axis left label {\lines{Frac.\cr of\cr sites}}  /
\axis bottom ticks numbered from 0.0 to 0.5 by 0.5 /
\plotheading{MtDNA$^{a,b}$}
\put{$\begin{array}{rcl}
      K &=& 636\\ % number of sequences in sample
      S &=& 226\\  % number of segregating sites
       \end{array}$} [tr] at 0.5 0.83
%Expected site frequency spectrum
\multiput {\footnotesize$\bullet$} at
0.013 0.475 0.038 0.104 0.062 0.062 0.088 0.045 0.113 0.036
0.138 0.030 0.163 0.026 0.188 0.023 0.213 0.020 0.237 0.019
0.263 0.016 0.288 0.016 0.312 0.015 0.338 0.014 0.363 0.013
0.388 0.013 0.413 0.012 0.438 0.012 0.463 0.011 0.488 0.011
/
%Observed site frequency spectrum
\sethistograms
\plot
0.000 0.000 0.025 0.827 0.050 0.075 0.075 0.044 0.100 0.013
0.125 0.013 0.150 0.004 0.175 0.004 0.200 0.004 0.225 0.000
0.250 0.000 0.275 0.000 0.300 0.004 0.325 0.000 0.350 0.004
0.375 0.000 0.400 0.004 0.425 0.000 0.450 0.000 0.475 0.000
0.500 0.000
/
%%%%%%%%%%%%%%%% row 1 col 2 %%%%%%%%%%%%%%%%
% point at 0.625in 0in
\setcoordinatesystem units <1.25in, 1.47in> point at 0.5 0
\setplotarea x from 0 to 1, y from 0 to 0.85
\axis bottom ticks numbered from 0 to 1 by 1 /
\plotheading{Y$^{b,c}$}
\put{$\begin{array}{rcl}
      K &=& 718\\ % number of sequences in sample
      S &=& 20\\  % number of segregating sites
       \end{array}$} [tr] at 1 0.85
%Expected site frequency spectrum
\multiput {\footnotesize$\bullet$} at
0.033 0.349
0.100 0.161
0.167 0.100
0.233 0.071
0.300 0.055
0.367 0.045
0.433 0.038
0.500 0.032
0.567 0.029
0.633 0.026
0.700 0.024
0.767 0.022
0.833 0.020
0.900 0.017
0.967 0.010                      
/
%Observed site frequency spectrum
\sethistograms
\plot
0.000 0.000
0.067 0.850
0.133 0.050
0.200 0.050
0.267 0.000
0.333 0.000
0.400 0.000
0.467 0.000
0.533 0.000
0.600 0.000
0.667 0.000
0.733 0.000
0.800 0.000
0.867 0.000
0.933 0.000
1.000 0.000
/                                              
%%%%%%%%%%%%%%%% row 2 col 1 %%%%%%%%%%%%%%%%
% point at 2.125in 1.85in
\setcoordinatesystem units <2.5in, 2.272in> point at 0.85 0.814
\setplotarea x from 0 to 0.5, y from 0 to 0.55
\axis left label {\lines{Frac.\cr of\cr sites}} /
\axis bottom ticks numbered from 0.0 to 0.5 by 0.5 /
\plotheading{$\beta$-globin$^e$}
\put{$\begin{array}{rcl}
      K &=& 253\\ % number of sequences in sample
      S &=& 33\\  % number of segregating sites
       \end{array}$} [tr] at 0.5 0.55
%Expected site frequency spectrum
\multiput {\footnotesize$\bullet$} at
0.025 0.516
0.075 0.125
0.125 0.071
0.175 0.057
0.225 0.046
0.275 0.036
0.325 0.034
0.375 0.031
0.425 0.026
0.475 0.026           
/
%Observed site frequency spectrum
\sethistograms
\plot
0.000 0.000
0.050 0.545
0.100 0.061
0.150 0.000
0.200 0.030
0.250 0.000
0.300 0.030
0.350 0.182
0.400 0.061
0.450 0.030
0.500 0.061
/                 
%%%%%%%%%%%%%%%% row 2 col 2 %%%%%%%%%%%%%%%%
% point at 0.625in 1.85in
\setcoordinatesystem units <2.5in, 3.125in> point at 0.25 0.592
\setplotarea x from 0 to 0.5, y from 0 to 0.4
\axis bottom ticks numbered from 0.0 to 0.5 by  0.5 /
\plotheading{Misc X$^f$}
\put{$\begin{array}{rcl}
      K &=& 10\\ % number of sequences in sample
      S &=& 20\\  % number of segregating sites
       \end{array}$} [tr] at 0.5 0.4
%Expected site frequency spectrum
\multiput {\footnotesize$\bullet$} at
0.050 0.393
0.150 0.216
0.250 0.157
0.350 0.128
0.450 0.071
/
%Observed site frequency spectrum
\sethistograms
\plot
0.000 0.000
0.100 0.400
0.200 0.200
0.300 0.100
0.400 0.250
0.500 0.050
/                                       
\endpicture}
\let\put\latexput


\section{Exercises}

\begin{exercise}
  For this exercise, use the toy data set in
  section~\ref{sec.toySeqData}, on p.~\pageref{sec.toySeqData}.
  (1)~Use $S$ to estimate $\theta$, (2)~from this value, calculate the
  number of sites expected in each frequency category, (3)~fold the
  resulting theoretical spectrum by summing values for $i$ and $K-i$.
  (4)~Compare the result with the empirical spectrum that we
  calculated earlier, in section~\ref{sec.toySeqData-results}. 
  \answer That example was of a sample of $K=10$ DNA sequences, which
  had 15 segregating sites.  Thus, we can estimate $\theta$ as
\[
\hat\theta_S = \frac{15}{\sum_{i=1}^{9} \frac{1}{i}}
= 5.3
\]
I'll use the symbols $v_u$ and $v_f$ to represent the unfolded and
folded spectra respectively.  The unfolded theoretical spectrum
(assuming selective neutrality and constant population size) is
\[
v_u = [\theta, \theta/2, \theta/3, \ldots , \theta/9]
\]
The first entry in this vector is the expected number of singleton
sites, the second is the expected number of doubleton sites, and so
on. Substituting the estimated value of $\theta$ turns this into
\[
v_u = [5.30, 2.65, 1.77, 1.33, 1.06, 0.88, 0.76, 0.66, 0.59]
\]
The folded spectrum is constructed as follows:
\begin{eqnarray*}
v_f &=& [5.30 + 0.59, 2.65 + 0.66, 1.77 + 0.76, 1.33 + 0.88, 1.06]\\
&=& [5.89, 3.31, 2.53, 2.21, 1.06]
\end{eqnarray*}
Thus, we expect 5.89 sites at which the minor allele is present in 1
copy, 3.31 sites at which it is present in 2 copies, and so on.
In the real data (see section~\ref{sec.dataAnalysis},
page~\pageref{sec.dataAnalysis}), we had 
\[
\hat v_f = [6, 2, 2, 5, 0]
\]
where the ``hat'' indicates that these values refer to data rather
than from theory.  The theoretical and observed spectra are similar,
but certainly not identical.  No inference can be drawn from this
difference, because our sample is very small.
\end{exercise}
