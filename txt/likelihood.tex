% -*-latex-*-
\reversemarginpar
\chapter{The Method of Maximum Likelihood}
\label{ch.likelihood}

Before doing this exercise, please read \emph{Using Likelihood}, which
you can find at
\url{http://www.anthro.utah.edu/~rogers/pubs/index.html}.

\section{Maximum likelihood exercises with genetics problems}

\begin{exercise}
Suppose that we have data from a genetic system with two alleles, and
that we observe $N_1$ individuals of genotype $A_1A_1$, $N_2$ of
genotype $A_1A_2$, and $N_3$ of genotype $A_2A_2$.  If the (unknown)
genotype frequencies are $P_1$, $P_2$, and $P_3$, then the likelihood
function is
\[
L \propto P_1^{N_1} P_2^{N_2} (1-P_1 - P_2)^{N_3}
\]
I have used the symbol ``$\propto$'' (which stands for ``is
proportional to'') rather than the equals sign because this expression
ignores a proportional constant that will not affect the answer.  The
log likelihood is
\begin{eqnarray*}
\ln L &=& \hbox{const.} + N_1 \ln P_1 + N_2 \ln P_2 \\
&& \mbox{} + N_3 \ln (1-P_1 - P_2)
\end{eqnarray*}
Find the values of $P_1$, $P_2$, and $P_3$ that maximize the
likelihood.
\answer
The parameter values that maximize this expression are
\begin{eqnarray}
\hat P_1 &=& \frac{N_1}{N_1 + N_2 + N_3}\\
\hat P_2 &=& \frac{N_2}{N_1 + N_2 + N_3}
\end{eqnarray}
\end{exercise}

\begin{exercise}
If we assume that the population is in Hardy-Weinberg equilibrium
then the likelihood and log likelihood functions are
\begin{eqnarray*}
L &\propto& [p^2]^{N_1} [2p(1-p)]^{N_2} [(1-p)^2]^{N_3}\\
\ln L &=& \hbox{const.} + (2N_1 + N_2)\ln p \\
 &&\mbox{} + (N_2 + 2N_3)\ln(1-p)
\end{eqnarray*}
Find the value of $p$ that maximizes the likelihood.
\end{exercise}
