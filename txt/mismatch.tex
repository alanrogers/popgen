% -*-latex-*-
\reversemarginpar
\chapter{The Mismatch Distribution\label{ch.mismatch}}

\section{The observed mismatch distribution}
\label{sec.mmobs}

Count the number of site differences between each pair of sequences in 
a sample, and use the resulting counts to build a histogram.  You end
up with a ``mismatch distribution.''  The $i$th entry of the mismatch
distribution is the number of pairs of sequences that differ by $i$
sites. 

For example, consider this data set:
\begin{center}
\begin{tabular}{lc@{}c@{}c@{}c@{}c@{$\;$}c@{}c@{}c@{}c@{}c}
S01   &   A&A&A&C&T& G&T&C&A&T\\
S02   &   .&.&.&.&.& A&.&T&.&.\\
S03   &   .&.&G&.&.& A&.&.&.&.\\
S04   &   .&.&G&.&.& A&.&T&.&.\\
S05   &   .&.&.&.&.& A&.&.&.&.\\
\end{tabular}
\end{center}
To calculate the mismatch distribution, we need to count the
differences between every pair of sequences. Here are my counts:
\begin{center}
\begin{tabular}{cc}
     & Pairwise\\
Pair & differences\\
\hline
$1\times 2$ & 2\\
$1\times 3$ & 2\\
$1\times 4$ & 3\\
$1\times 5$ & 1\\
$2\times 3$ & 2\\
$2\times 4$ & 1\\
$2\times 5$ & 1\\
$3\times 4$ & 1\\
$3\times 5$ & 1\\
$4\times 5$ & 2\\
\end{tabular}
\end{center}
There are five 1s, four 2s, and one 3. Thus, the mismatch distribution
is
\begin{center}
\begin{tabular}{cc}
            & Number\\
Pairwise    &   of\\
differences & pairs\\
\hline
0 & 0\\
1 & 5\\
2 & 4\\
3 & 1
\end{tabular}
\end{center}
Here, the right column gives the number of pairs that exhibit each
level of difference. We often re-express these as fractions of the
number of pairs of sequences. Since there are 10 pairs in our data
set, this gives:
\begin{center}
\begin{tabular}{cc}
            & Fraction\\
Pairwise    &   of\\
differences & pairs\\
\hline
0 & 0.0\\
1 & 0.5\\
2 & 0.4\\
3 & 0.1
\end{tabular}
\end{center}
Now the numbers in the right column sum to 1---they are relative
frequencies. This is the \emph{observed} or \emph{empirical} mismatch
distribution. 

\section{The expected mismatch distribution under neutral 
evolution with constant population size}
\label{sec.mmconst}

The previous section concerned the observed mismatch distribution,
which we calculate from genetic data. Each entry in this distribution
is a random variable, so it is natural to wonder about its expected
value. This is easy to calculate, under a model of constant size and
selective neutrality: a random pair of sequences differs by $i$
sites with probability \citep{Watterson:TPB-7-256}
\begin{equation}
F_i = \left(\frac{1}{\theta + 1}\right)
\left(\frac{\theta}{\theta + 1}\right)^i, \qquad (i=0,1,2,\ldots)
\label{eq.eqmismatch}
\end{equation}
where $\theta = 4Nu$, $u$ is the mutation rate per generation, and
$2N$ is the number of genes in the population.  This formula is
graphed in figure~\ref{fig.poorfit} along with an empirical mismatch
distribution from human mtDNA. (The empirical distribution, shown as
open circles, is analogous to the one calculated in
section~\ref{sec.mmobs}.)

\begin{figure}
{\centering\mbox{%
\beginpicture
\renewcommand{\baselinestretch}{1}  
\setcoordinatesystem units <.12in, 10in>
\setplotarea x from 0 to 29, y from 0 to 0.12
\axis left label {$F_i$}
  ticks numbered from 0 to 0.12 by  0.04 /
\axis bottom 
  label {$i$} ticks
  numbered from 0 to 29 by  5 /
%Cann et al 1987 data
\multiput {$\circ$} at
0 0.002620373403209957 1 0.004913200131018669 2 0.01179168031444481 3
0.02292826727808712 4 0.03766786767114313 5 0.05961349492302653 6
0.08155912217490993 7 .109728136259417 8 .1162790697674418 9
.1048149361283983 10 .1015394693743858 11 0.08188666885031116 12
0.06976744186046512 13 0.05764821487061906 14 0.04192597445135932 15
0.030789387487717 16 0.01965280052407468 17 0.01375696036685227 18
0.008516213560432362 19 0.003930560104814936 20 0.003930560104814936
21 0.002620373403209957 22 0.002620373403209957 23
0.002620373403209957 24 0.002620373403209957 25 0.001310186701604979
26 0.001310186701604979 27 0.0006550933508024893 28
0.0006550933508024893 29 0.0003275466754012447
/
% Geometric fit
\plot
 0 0.106025352 1 0.094783976 2 0.084734472 3 0.075750470 4 0.067719000
5 0.060539069 6 0.054120393 7 0.048382259 8 0.043252513 9 0.038666650
10 0.034567005 11 0.030902026 12 0.027625628 13 0.024696611 14
0.022078144 15 0.019737301 16 0.017644647 17 0.015773867 18
0.014101437 19 0.012606327 20 0.011269737 21 0.010074859 22
0.009006669 23 0.008051734 24 0.007198046 25 0.006434870 26
0.005752611 27 0.005142688 28 0.004597433 29 0.004109989
/
\endpicture%
}
\\}
\caption{The poor fit of the equilibrium distribution}
\label{fig.poorfit} \small The open circles show the empirical
pairwise difference distribution of \citet{Cann:N-325-31}, based on
their figure~1.  The solid line is an equilibrium distribution with
the same mean.
\end{figure}

The poor fit between the observed and expected curves is striking.  As
usual, there are several hypotheses to consider:
\begin{description}
\item[Sampling error] Perhaps the poor fit is an artifact attributable 
  to sampling error.  This possibility is especially important here
  because the pairs of sequences in this analysis are not
  independent:  They are correlated both because each sequence
  participates in several pairs and also because of the genealogical
  relationships among sequences.
\item[Selection] More on this later
\item[Failure of infinite sites hypothesis]
\item[Non-random mating]
\item[Variation in population size] 
\end{description}
Work has been done on all of these possibilities, but we will consider
only the last.

\section{Coalescent theory in a population of varying size}
\label{sec.coalvaryn}

The principles of coalescent theory still hold in a population of
varying size.  At any given time, $t$, the hazard of a coalescent
event is 
\[
h_i(t) = \frac{i(i-1)}{4N(t)}
\]
where $i$ is the number of distinct lines of descent in the gene
genealogy at time $t$ and $2N(t)$ is the number of genes in the
population size at time $t$.

It is no longer true, however, that the mean waiting time until a
coalescent event is $1/h_i$.  That only works in populations of
constant size.  Some theoretical results are still possible, but the
more complex the model the more we are forced to base inferences on
computer simulations.

\section{The coalescent as an algorithm for computer simulations}

Fortunately, coalescent theory is just as useful in computer
simulations as in theoretical work.  The style of reasoning is the
same, except that coalescent intervals are generated from random
numbers.   As in the case of constant population size, intervals tend
to be long in those parts of the tree where there are only a few lines 
of descent.  

Now, however, there is an additional factor to consider: Coalescent
intervals also tend to be long when the population size is large.
It is not hard to understand why this should be so.  Two random genes
drawn from a large population are likely to be distantly related and
thus separated by a long genealogical path.  In a small population, a
pair of genes is likely to be separated by a shorter path.

Computer simulations are most useful if they deal with only a few
parameters at a time.  Thus, we need some economical way to describe
the history of a population that changes in size.

\section{Stepwise models of population history}

Population growth is ordinarily regulated by density-dependent
mechanisms that allow small populations to grow and cause large ones
to shrink.  Consequently, most biologists think that the sizes of most
populations are roughly constant most of the time.  Now and then,
something happens to disturb this equilibrium, and the population
either grows or shrinks until a new equilibrium is attained.  Over a
long time scale, the periods of change would look like relatively
sudden changes in the size of a population that was otherwise roughly
constant.

This is the conventional view of demographic history, and it motivates
the stepwise model of population history that is used here.  I will
assume that history can be divided into a series of epochs during
which the population does not change.  Epochs are separated by
episodes of rapid change, which I treat as instantaneous.  They are
numbered backwards from the most recent (epoch~0) to the most
ancient.\footnote{\citet{Rogers:MBE-9-552} numbered them in the
  opposite direction.}

\begin{table}
\caption{Hypothetical two-epoch population history}
\label{tab.ph1}
\centering
\begin{tabular}{ccc}
\hline
      &   $2N_i$    &   $t_i$  \\ 
Epoch &   (genes) & (generations)\\ 
\hline\hline
    0 &  $10^6$ &   2000  \\
    1 &  $10^4$ & $\infty$ \\
\hline
\end{tabular}
\end{table}

% -*-latex-*-
\let\put\pictexput
\mbox{\beginpicture
\valuestolabelleading=.4\baselineskip
\setcoordinatesystem units <0.24in, 0.0018in> point at 0 0
\setplotarea x from 0 to 14, y from 0 to 500
\axis left shiftedto x=-0.5 
  ticks  withvalues {$2N_1$} {$2N_0$} / at 1 500 / /
\axis bottom shiftedto y=-50
  label {Generations before present}
  ticks withvalues {$t$} / at 7 / /
\put{\em Population Size} [rt] at 14 500
\plot 0 500 7 500 7 1 14 1 /
\endpicture}
\let\put\latexput


Table~\ref{tab.ph1} shows the parameters of a hypothetical population
history with two epochs.  The table describes a population that was
small (10,000 people) early in time, grew suddenly 2000 generations
ago, and has been large (1,000,000 people) ever since.  The history of
this hypothetical population is graphed in figure~\ref{fig.sudden}. 

\begin{table}
\caption{Alternate parameterization of the same population history}
\label{tab.ph2}
\centering
\begin{tabular}{ccc}
\hline
Epoch &   $\theta_i$ & $\tau_i$ \\ 
\hline\hline
    0 &  2000 &   4      \\
    1 &    20 & $\infty$ \\
\hline
\end{tabular}
\end{table}

With genetic data is it not possible to estimate $N_i$ and $t_i$
directly.  We can, however, estimate the related parameters
\begin{eqnarray}
\theta_i &=& 4N_i u\label{eq.thetai}\\
\tau_i &=& 2 u t_i\label{eq.taui}
\end{eqnarray}
where $u$ is the mutation rate.  $\theta_i$ measures the size of the
population during epoch $i$ in units of $1/(2u)$ genes; $\tau_i$
measures the length of this epoch in units of $1/(2u)$ generations.
The population history of table~\ref{tab.ph1} can be
re-expressed in terms of $\theta$ and $\tau$ by multiplying each
parameter by $2u$.  For example, if the mutation rate is 0.001 (per
sequence per generation), then we obtain table~\ref{tab.ph2}.

\section{Simulations of stationary populations}

Figures~\ref{fig.eq1}--\ref{fig.eq4}, on
pages~\pageref{fig.eq1}--\pageref{fig.eq4}, each show a simulation of
a population with constant size.  In each figure, the upper panel
shows the time path of population size, which is simply a horizontal
line in these first four figures.  Proceding down each page, you will
find the gene genealogy, the mismatch distribution and the spectrum.
The expected mismatch distribution is shown as a solid line; the
expected spectrum is shown as a series of filled circles.  In both
cases, the expected values refer to a model of neutral evolution in
a population of constant size.

These figures make use of what I will call the ``mutational time
scale.''  A unit of mutational time is the amount of time it takes, on 
average, for one mutation to accumulate along the genealogical path
separating two sequences.  Thus, if two sequences have been separate
for 5 units of mutational time, we expect them to differ at about 5
nucleotide sites.  Each unit of mutational time is equal to $1/(2u)$
generations.  
\begin{example}
  In human mitochondrial D-loop sequence, it has been estimated that
  the mutation rate is $4.1\times 10^{-6}$ per nucleotide per
  generation \citep{Ward:PNA-88-8720}.  In the Jorde et al HVSI data
  there are 430 nucleotides, so $u = 430 \times 4.1\times 10^{-6} =
  0.0018$ per generation.  Each unit of mutational time is $1/(2u) =
  278$ generations, which would correspond roughly to 6950 years.
\end{example}

Notice that neither the mismatch distributions nor the spectra of
these equilibrium populations look much like the theoretical formulas
would predict.  Far from showing the smooth decline seen in the
theoretical curves, the simulated mismatch distributions tend to be
ragged, with multiple peaks.  The site frequency spectra also exhibit
pronounced departures from their expected values.  In order to get
answers that look like the theory, we would need to run the
simulations many times and average the results.  This is bad news,
since in real data analysis we cannot rewind the evolutionary process
and look at it again and again.  The situation is not hopeless, but it
is clear that merely inspecting graphs such as these is not going to
give us dependable answers.  We are going to need more sophisticated
statistical methods.

\begin{figure}
\centering
% -*-latex-*-
%
%%%%%%%%%%%%%% Tree in PicTeX Format %%%%%%%%%%%%%%
%
%Sample size   = 50
%Mutation rate = 1
%     theta         mn        tau          K
%    7.0000     0.0000        Inf          1
%maxmm=26.000000 maxtau=0.000000 maxtree=27.679483 maxx=27.679483
%\newdimen\offsety
%\newdimen\yunit
%\newdimen\xunit
%\newdimen\thusfar
%\newdimen\plotht
%\newdimen\plotwd
%\newdimen\plotsp
\thusfar=0.000000in  % Keeps track of what's above
\plotht=0.2500000in   % Height of each plot
\plotwd=0.7\textwidth   % Width of each plot
\plotsp=0.500000in   % Spacing between plots
\let\put\pictexput
\mbox{\beginpicture
\def\mutation{\tiny$\bullet$}
\small
\valuestolabelleading=.4\baselineskip
\headingtoplotskip=0.4\baselineskip
%
%%%%%%%%%%%%%% Population size %%%%%%%%%%%%%%
%
\yunit=\plotht
\xunit=\plotwd
\Divide <\xunit> by <28.000000pt> forming <\xunit>
\Divide <\yunit> by <8.400001pt> forming <\yunit>
\advance\thusfar by \plotht
\advance\thusfar by \plotsp
\Divide <\thusfar> by <\yunit> forming <\offsety>
\placevalueinpts of <\offsety> in {\mktree_tmp}
\setcoordinatesystem units <\xunit, \yunit> point at 0 {\mktree_tmp}
\setplotarea x from 0.000000 to 28.000000, y from 0.000000 to 8.400001
\advance\thusfar by 0.030000\plotht
\axis left label {\lines{Population\cr size}} /
\axis bottom shiftedto y=-0.252000
    label {Mutational time before present}
    ticks numbered from 0 to 27 by 9 /
\putrule from 0.000000 7.000000 to 28.000000 7.000000
%
%%%%%%%%%%%%%% Gene Genealogy %%%%%%%%%%%%%%
%
\yunit=\plotht
\xunit=\plotwd
\Divide <\xunit> by <28.000000pt> forming <\xunit>
\Divide <\yunit> by <49.000000pt> forming <\yunit>
\advance\thusfar by \plotht
\advance\thusfar by \plotsp
\Divide <\thusfar> by <\yunit> forming <\offsety>
\placevalueinpts of <\offsety> in {\mktree_tmp}
\setcoordinatesystem units <\xunit, \yunit> point at 0 {\mktree_tmp}
\setplotarea x from 0.000000 to 28.000000, y from 0.000000 to 49.000000
\axis left invisible label {\lines{Gene\cr genealogy}} /
\axis bottom invisible
     label {Mutational time before present} /
\putrule from 0.000000 0.000000 to 0.163105 0.000000
\putrule from 0.000000 1.000000 to 0.163105 1.000000
\putrule from 0.163105 1.000000 to 0.163105 0.000000
\putrule from 0.163105 0.500000 to 1.664524 0.500000
\putrule from 0.000000 2.000000 to 0.030882 2.000000
\putrule from 0.000000 3.000000 to 0.030882 3.000000
\putrule from 0.030882 3.000000 to 0.030882 2.000000
\putrule from 0.030882 2.500000 to 0.170518 2.500000
\put {\mutation} at 0.099945 2.500000
\putrule from 0.000000 4.000000 to 0.170518 4.000000
\putrule from 0.170518 4.000000 to 0.170518 2.500000
\putrule from 0.170518 3.250000 to 0.533060 3.250000
\put {\mutation} at 0.322730 3.250000
\putrule from 0.000000 5.000000 to 0.213589 5.000000
\put {\mutation} at 0.200248 5.000000
\putrule from 0.000000 6.000000 to 0.073629 6.000000
\putrule from 0.000000 7.000000 to 0.073629 7.000000
\putrule from 0.073629 7.000000 to 0.073629 6.000000
\putrule from 0.073629 6.500000 to 0.213589 6.500000
\putrule from 0.213589 6.500000 to 0.213589 5.000000
\putrule from 0.213589 5.750000 to 0.273359 5.750000
\putrule from 0.000000 8.000000 to 0.273359 8.000000
\put {\mutation} at 0.123819 8.000000
\putrule from 0.273359 8.000000 to 0.273359 5.750000
\putrule from 0.273359 6.875000 to 0.533060 6.875000
\putrule from 0.533060 6.875000 to 0.533060 3.250000
\putrule from 0.533060 5.062500 to 1.097055 5.062500
\putrule from 0.000000 9.000000 to 0.025113 9.000000
\putrule from 0.000000 10.000000 to 0.025113 10.000000
\putrule from 0.025113 10.000000 to 0.025113 9.000000
\putrule from 0.025113 9.500000 to 1.097055 9.500000
\putrule from 1.097055 9.500000 to 1.097055 5.062500
\putrule from 1.097055 7.281250 to 1.629606 7.281250
\putrule from 0.000000 11.000000 to 0.623208 11.000000
\putrule from 0.000000 12.000000 to 0.623208 12.000000
\putrule from 0.623208 12.000000 to 0.623208 11.000000
\putrule from 0.623208 11.500000 to 0.672018 11.500000
\putrule from 0.000000 13.000000 to 0.137701 13.000000
\put {\mutation} at 0.033537 13.000000
\putrule from 0.000000 14.000000 to 0.044849 14.000000
\putrule from 0.000000 15.000000 to 0.044849 15.000000
\putrule from 0.044849 15.000000 to 0.044849 14.000000
\putrule from 0.044849 14.500000 to 0.137701 14.500000
\putrule from 0.137701 14.500000 to 0.137701 13.000000
\putrule from 0.137701 13.750000 to 0.624356 13.750000
\putrule from 0.000000 16.000000 to 0.624356 16.000000
\putrule from 0.624356 16.000000 to 0.624356 13.750000
\putrule from 0.624356 14.875000 to 0.635488 14.875000
\putrule from 0.000000 17.000000 to 0.057782 17.000000
\putrule from 0.000000 18.000000 to 0.057782 18.000000
\putrule from 0.057782 18.000000 to 0.057782 17.000000
\putrule from 0.057782 17.500000 to 0.635488 17.500000
\putrule from 0.635488 17.500000 to 0.635488 14.875000
\putrule from 0.635488 16.187500 to 0.672018 16.187500
\putrule from 0.672018 16.187500 to 0.672018 11.500000
\putrule from 0.672018 13.843750 to 1.629606 13.843750
\putrule from 1.629606 13.843750 to 1.629606 7.281250
\putrule from 1.629606 10.562500 to 1.664523 10.562500
\putrule from 1.664523 10.562500 to 1.664523 0.500000
\putrule from 1.664523 5.531250 to 1.743762 5.531250
\putrule from 0.000000 19.000000 to 0.024676 19.000000
\putrule from 0.000000 20.000000 to 0.024676 20.000000
\putrule from 0.024676 20.000000 to 0.024676 19.000000
\putrule from 0.024676 19.500000 to 1.743762 19.500000
\putrule from 1.743762 19.500000 to 1.743762 5.531250
\putrule from 1.743762 12.515625 to 4.822913 12.515625
\put {\mutation} at 3.987646 12.515625
\put {\mutation} at 4.640998 12.515625
\put {\mutation} at 2.394315 12.515625
\putrule from 0.000000 21.000000 to 0.343685 21.000000
\putrule from 0.000000 22.000000 to 0.093079 22.000000
\put {\mutation} at 0.087093 22.000000
\putrule from 0.000000 23.000000 to 0.093079 23.000000
\putrule from 0.093079 23.000000 to 0.093079 22.000000
\putrule from 0.093079 22.500000 to 0.343685 22.500000
\putrule from 0.343685 22.500000 to 0.343685 21.000000
\putrule from 0.343685 21.750000 to 0.382976 21.750000
\putrule from 0.000000 24.000000 to 0.382976 24.000000
\putrule from 0.382976 24.000000 to 0.382976 21.750000
\putrule from 0.382976 22.875000 to 0.836458 22.875000
\putrule from 0.000000 25.000000 to 0.836458 25.000000
\putrule from 0.836458 25.000000 to 0.836458 22.875000
\putrule from 0.836458 23.937500 to 2.914227 23.937500
\put {\mutation} at 1.590106 23.937500
\put {\mutation} at 1.241755 23.937500
\putrule from 0.000000 26.000000 to 0.166674 26.000000
\putrule from 0.000000 27.000000 to 0.166674 27.000000
\putrule from 0.166674 27.000000 to 0.166674 26.000000
\putrule from 0.166674 26.500000 to 0.239347 26.500000
\putrule from 0.000000 28.000000 to 0.211987 28.000000
\putrule from 0.000000 29.000000 to 0.211987 29.000000
\putrule from 0.211987 29.000000 to 0.211987 28.000000
\putrule from 0.211987 28.500000 to 0.239347 28.500000
\putrule from 0.239347 28.500000 to 0.239347 26.500000
\putrule from 0.239347 27.500000 to 2.914227 27.500000
\putrule from 2.914227 27.500000 to 2.914227 23.937500
\putrule from 2.914227 25.718750 to 4.822913 25.718750
\put {\mutation} at 3.754659 25.718750
\putrule from 4.822913 25.718750 to 4.822913 12.515625
\putrule from 4.822913 19.117188 to 25.163166 19.117188
\put {\mutation} at 19.398636 19.117188
\put {\mutation} at 7.721166 19.117188
\put {\mutation} at 5.676535 19.117188
\put {\mutation} at 5.254761 19.117188
\put {\mutation} at 16.644871 19.117188
\put {\mutation} at 14.109597 19.117188
\put {\mutation} at 13.206370 19.117188
\putrule from 0.000000 30.000000 to 0.814964 30.000000
\putrule from 0.000000 31.000000 to 0.010703 31.000000
\putrule from 0.000000 32.000000 to 0.010703 32.000000
\putrule from 0.010703 32.000000 to 0.010703 31.000000
\putrule from 0.010703 31.500000 to 0.177288 31.500000
\putrule from 0.000000 33.000000 to 0.000728 33.000000
\putrule from 0.000000 34.000000 to 0.000728 34.000000
\putrule from 0.000728 34.000000 to 0.000728 33.000000
\putrule from 0.000728 33.500000 to 0.072904 33.500000
\putrule from 0.000000 35.000000 to 0.072904 35.000000
\putrule from 0.072904 35.000000 to 0.072904 33.500000
\putrule from 0.072904 34.250000 to 0.177288 34.250000
\putrule from 0.177288 34.250000 to 0.177288 31.500000
\putrule from 0.177288 32.875000 to 0.814964 32.875000
\putrule from 0.814964 32.875000 to 0.814964 30.000000
\putrule from 0.814964 31.437500 to 4.416076 31.437500
\put {\mutation} at 2.567282 31.437500
\put {\mutation} at 2.539873 31.437500
\putrule from 0.000000 36.000000 to 0.431432 36.000000
\putrule from 0.000000 37.000000 to 0.068241 37.000000
\putrule from 0.000000 38.000000 to 0.068241 38.000000
\putrule from 0.068241 38.000000 to 0.068241 37.000000
\putrule from 0.068241 37.500000 to 0.251193 37.500000
\putrule from 0.000000 39.000000 to 0.251193 39.000000
\put {\mutation} at 0.041658 39.000000
\putrule from 0.251193 39.000000 to 0.251193 37.500000
\putrule from 0.251193 38.250000 to 0.431432 38.250000
\putrule from 0.431432 38.250000 to 0.431432 36.000000
\putrule from 0.431432 37.125000 to 0.990780 37.125000
\putrule from 0.000000 40.000000 to 0.169943 40.000000
\putrule from 0.000000 41.000000 to 0.169943 41.000000
\putrule from 0.169943 41.000000 to 0.169943 40.000000
\putrule from 0.169943 40.500000 to 0.683963 40.500000
\put {\mutation} at 0.289147 40.500000
\putrule from 0.000000 42.000000 to 0.061294 42.000000
\putrule from 0.000000 43.000000 to 0.061294 43.000000
\putrule from 0.061294 43.000000 to 0.061294 42.000000
\putrule from 0.061294 42.500000 to 0.116969 42.500000
\putrule from 0.000000 44.000000 to 0.045075 44.000000
\putrule from 0.000000 45.000000 to 0.005711 45.000000
\putrule from 0.000000 46.000000 to 0.005711 46.000000
\putrule from 0.005711 46.000000 to 0.005711 45.000000
\putrule from 0.005711 45.500000 to 0.045075 45.500000
\putrule from 0.045075 45.500000 to 0.045075 44.000000
\putrule from 0.045075 44.750000 to 0.083422 44.750000
\putrule from 0.000000 47.000000 to 0.083422 47.000000
\putrule from 0.083422 47.000000 to 0.083422 44.750000
\putrule from 0.083422 45.875000 to 0.116969 45.875000
\putrule from 0.116969 45.875000 to 0.116969 42.500000
\putrule from 0.116969 44.187500 to 0.475756 44.187500
\putrule from 0.000000 48.000000 to 0.311578 48.000000
\putrule from 0.000000 49.000000 to 0.311578 49.000000
\putrule from 0.311578 49.000000 to 0.311578 48.000000
\putrule from 0.311578 48.500000 to 0.475756 48.500000
\putrule from 0.475756 48.500000 to 0.475756 44.187500
\putrule from 0.475756 46.343750 to 0.683963 46.343750
\putrule from 0.683963 46.343750 to 0.683963 40.500000
\putrule from 0.683963 43.421875 to 0.990780 43.421875
\put {\mutation} at 0.973821 43.421875
\putrule from 0.990780 43.421875 to 0.990780 37.125000
\putrule from 0.990780 40.273438 to 4.416076 40.273438
\put {\mutation} at 3.035250 40.273438
\put {\mutation} at 3.680078 40.273438
\putrule from 4.416076 40.273438 to 4.416076 31.437500
\putrule from 4.416076 35.855469 to 25.163166 35.855469
\put {\mutation} at 24.780346 35.855469
\put {\mutation} at 8.649749 35.855469
\put {\mutation} at 8.414587 35.855469
\put {\mutation} at 10.081877 35.855469
\put {\mutation} at 24.477703 35.855469
\put {\mutation} at 7.035466 35.855469
\put {\mutation} at 11.454272 35.855469
\put {\mutation} at 12.025785 35.855469
\put {\mutation} at 17.238018 35.855469
\put {\mutation} at 11.386416 35.855469
\putrule from 25.163166 35.855469 to 25.163166 19.117188
\putrule from 25.163166 27.486328 to 28.000000 27.486328
%
%%%%%%%%%%%%%% Mismatch Distribution %%%%%%%%%%%%%%
%
\yunit=\plotht
\xunit=\plotwd
\Divide <\xunit> by <28.000000pt> forming <\xunit>
\Divide <\yunit> by <0.176327pt> forming <\yunit>
\advance\thusfar by \plotht
\advance\thusfar by \plotsp
\Divide <\thusfar> by <\yunit> forming <\offsety>
\placevalueinpts of <\offsety> in {\mktree_tmp}
\setcoordinatesystem units <\xunit, \yunit> point at 0 {\mktree_tmp}
\setplotarea x from 0.000000 to 28.000000, y from 0.000000 to 0.176327
\axis left label {\lines{Mismatch\cr distribution}} /
\axis bottom
   label {Pairwise differences}
   ticks numbered from 0 to 27 by 9 /
\multiput {$\circ$} at
 0 0.134694 1 0.088980 2 0.053878 3 0.009796 4 0.063673
 5 0.057143 6 0.065306 7 0.025306 8 0.009796 9 0.001633
 10 0.000000 11 0.000000 12 0.000000 13 0.000000 14 0.000000
 15 0.000000 16 0.000000 17 0.000000 18 0.000000 19 0.000000
 20 0.029388 21 0.029388 22 0.146122 23 0.176327 24 0.082449
 25 0.022857 26 0.003265
/
%Pr[random pair differs by i sites]
\plot 0 0.125000 1 0.109375 2 0.095703 3 0.083740 4 0.073273
      5 0.064114 6 0.056099 7 0.049087 8 0.042951 9 0.037582
      10 0.032884 11 0.028774 12 0.025177 13 0.022030 14 0.019276
      15 0.016867 16 0.014758 17 0.012914 18 0.011299 19 0.009887
      20 0.008651 21 0.007570 22 0.006623 23 0.005796 24 0.005071
      25 0.004437 26 0.003883
/                                         
\plotht=1.100000\plotht
%
%%%%%%%%%%%%%% Simulated site frequency spectrum %%%%%%%%%%%%%%
%
\yunit=\plotht
\xunit=\plotwd
\Divide <\xunit> by <25.000000pt> forming <\xunit>
\Divide <\yunit> by <0.519444pt> forming <\yunit>
\advance\thusfar by \plotht
\advance\thusfar by \plotsp
\Divide <\thusfar> by <\yunit> forming <\offsety>
\placevalueinpts of <\offsety> in {\mktree_tmp}
\setcoordinatesystem units <\xunit, \yunit> point at 0 {\mktree_tmp}
\setplotarea x from 0.000000 to 25.000000, y from 0.000000 to 0.519444
\axis left invisible label {\lines{Site\cr frequency\cr spectrum}} /
\axis bottom
   label {Frequency of minor allele}
   ticks withvalues 0 {1/2} / at 0 25 / /
\linethickness 1.2pt
\axis top /
\linethickness 0.4pt
\sethistograms
\plot 0 0 1 0.138889 2 0.055556 3 0.027778 4 0.000000
 5 0.055556 6 0.055556 7 0.000000 8 0.000000 9 0.027778
 10 0.027778 11 0.000000 12 0.000000 13 0.000000 14 0.055556
 15 0.000000 16 0.000000 17 0.000000 18 0.000000 19 0.000000
 20 0.472222 21 0.083333 22 0.000000 23 0.000000 24 0.000000
 25 0.000000
/
\setlinear
%
%%%%%%%%%%%%%% Neutral expectation of site freq spectrum %%%%%%%%%%%%%%
%
\multiput {$\bullet$} at 
     0.500000 0.22781  1.500000 0.116278  2.500000 0.079168  3.500000 0.0606668  4.500000 0.049612
     5.500000 0.0422829  6.500000 0.0370854  7.500000 0.0332223  8.500000 0.0302512  9.500000 0.0279067
     10.500000 0.0260203  11.500000 0.0244796  12.500000 0.0232073  13.500000 0.0221482  14.500000 0.0212623
     15.500000 0.0205197  16.500000 0.0198978  17.500000 0.0193797  18.500000 0.0189519  19.500000 0.0186045
     20.500000 0.0183295  21.500000 0.0181213  22.500000 0.0179754  23.500000 0.0178889  24.500000 0.0178603
/
\endpicture}
\let\put\latexput

\caption{An equilibrium population with $\theta=7$\label{fig.eq1}}
\end{figure}

\begin{figure}
\centering
% -*-latex-*-
%
%%%%%%%%%%%%%% Tree in PicTeX Format %%%%%%%%%%%%%%
%
%Sample size   = 50
%Mutation rate = 1
%     theta         mn        tau          K
%    7.0000     0.0000        Inf          1
%maxmm=8.000000 maxtau=0.000000 maxtree=4.709246 maxx=8.000000
%\newdimen\offsety
%\newdimen\yunit
%\newdimen\xunit
%\newdimen\thusfar
%\newdimen\plotht
%\newdimen\plotwd
%\newdimen\plotsp
\thusfar=0.000000in  % Keeps track of what's above
\plotht=1.400000in   % Height of each plot
\plotwd=4.000000in   % Width of each plot
\plotsp=0.500000in   % Spacing between plots
\mbox{\beginpicture
\def\mutation{\tiny$\bullet$}
\small
\valuestolabelleading=.4\baselineskip
\headingtoplotskip=0.4\baselineskip
%
%%%%%%%%%%%%%% Population size %%%%%%%%%%%%%%
%
\yunit=\plotht
\xunit=\plotwd
\Divide <\xunit> by <8.000000pt> forming <\xunit>
\Divide <\yunit> by <8.400001pt> forming <\yunit>
\advance\thusfar by \plotht
\advance\thusfar by \plotsp
\Divide <\thusfar> by <\yunit> forming <\offsety>
\placevalueinpts of <\offsety> in {\mktree_tmp}
\setcoordinatesystem units <\xunit, \yunit> point at 0 {\mktree_tmp}
\setplotarea x from 0.000000 to 8.000000, y from 0.000000 to 8.400001
\advance\thusfar by 0.030000\plotht
\axis left label {\lines{Population\cr size}} /
\axis bottom shiftedto y=-0.252000
    label {Mutational time before present}
    ticks numbered from 0 to 8 by 4 /
\putrule from 0.000000 7.000000 to 8.000000 7.000000
%
%%%%%%%%%%%%%% Gene Genealogy %%%%%%%%%%%%%%
%
\yunit=\plotht
\xunit=\plotwd
\Divide <\xunit> by <8.000000pt> forming <\xunit>
\Divide <\yunit> by <49.000000pt> forming <\yunit>
\advance\thusfar by \plotht
\advance\thusfar by \plotsp
\Divide <\thusfar> by <\yunit> forming <\offsety>
\placevalueinpts of <\offsety> in {\mktree_tmp}
\setcoordinatesystem units <\xunit, \yunit> point at 0 {\mktree_tmp}
\setplotarea x from 0.000000 to 8.000000, y from 0.000000 to 49.000000
\axis left invisible label {\lines{Gene\cr genealogy}} /
\axis bottom invisible
     label {Mutational time before present} /
\putrule from 0.000000 0.000000 to 0.634129 0.000000
\putrule from 0.000000 1.000000 to 0.634129 1.000000
\putrule from 0.634129 1.000000 to 0.634129 0.000000
\putrule from 0.634129 0.500000 to 0.847749 0.500000
\putrule from 0.000000 2.000000 to 0.847749 2.000000
\putrule from 0.847749 2.000000 to 0.847749 0.500000
\putrule from 0.847749 1.250000 to 2.783293 1.250000
\put {\mutation} at 2.119152 1.250000
\put {\mutation} at 1.042484 1.250000
\putrule from 0.000000 3.000000 to 2.149696 3.000000
\put {\mutation} at 0.890352 3.000000
\put {\mutation} at 2.064633 3.000000
\put {\mutation} at 1.482429 3.000000
\putrule from 0.000000 4.000000 to 0.062847 4.000000
\putrule from 0.000000 5.000000 to 0.062847 5.000000
\putrule from 0.062847 5.000000 to 0.062847 4.000000
\putrule from 0.062847 4.500000 to 0.466567 4.500000
\put {\mutation} at 0.113029 4.500000
\putrule from 0.000000 6.000000 to 0.466567 6.000000
\putrule from 0.466567 6.000000 to 0.466567 4.500000
\putrule from 0.466567 5.250000 to 0.581282 5.250000
\putrule from 0.000000 7.000000 to 0.126153 7.000000
\put {\mutation} at 0.067100 7.000000
\putrule from 0.000000 8.000000 to 0.126153 8.000000
\putrule from 0.126153 8.000000 to 0.126153 7.000000
\putrule from 0.126153 7.500000 to 0.581282 7.500000
\putrule from 0.581282 7.500000 to 0.581282 5.250000
\putrule from 0.581282 6.375000 to 0.898697 6.375000
\putrule from 0.000000 9.000000 to 0.715297 9.000000
\put {\mutation} at 0.407041 9.000000
\put {\mutation} at 0.303064 9.000000
\putrule from 0.000000 10.000000 to 0.715297 10.000000
\putrule from 0.715297 10.000000 to 0.715297 9.000000
\putrule from 0.715297 9.500000 to 0.766300 9.500000
\putrule from 0.000000 11.000000 to 0.032979 11.000000
\putrule from 0.000000 12.000000 to 0.032979 12.000000
\putrule from 0.032979 12.000000 to 0.032979 11.000000
\putrule from 0.032979 11.500000 to 0.766300 11.500000
\putrule from 0.766300 11.500000 to 0.766300 9.500000
\putrule from 0.766300 10.500000 to 0.898697 10.500000
\putrule from 0.898697 10.500000 to 0.898697 6.375000
\putrule from 0.898697 8.437500 to 1.020260 8.437500
\putrule from 0.000000 13.000000 to 1.020260 13.000000
\put {\mutation} at 0.411434 13.000000
\putrule from 1.020260 13.000000 to 1.020260 8.437500
\putrule from 1.020260 10.718750 to 2.149696 10.718750
\putrule from 2.149696 10.718750 to 2.149696 3.000000
\putrule from 2.149696 6.859375 to 2.783293 6.859375
\putrule from 2.783293 6.859375 to 2.783293 1.250000
\putrule from 2.783293 4.054688 to 4.281133 4.054688
\put {\mutation} at 3.997842 4.054688
\put {\mutation} at 2.817667 4.054688
\put {\mutation} at 4.165835 4.054688
\putrule from 0.000000 14.000000 to 0.793256 14.000000
\put {\mutation} at 0.014041 14.000000
\putrule from 0.000000 15.000000 to 0.002451 15.000000
\putrule from 0.000000 16.000000 to 0.002451 16.000000
\putrule from 0.002451 16.000000 to 0.002451 15.000000
\putrule from 0.002451 15.500000 to 0.793256 15.500000
\putrule from 0.793256 15.500000 to 0.793256 14.000000
\putrule from 0.793256 14.750000 to 0.854818 14.750000
\putrule from 0.000000 17.000000 to 0.575900 17.000000
\putrule from 0.000000 18.000000 to 0.178235 18.000000
\putrule from 0.000000 19.000000 to 0.178235 19.000000
\putrule from 0.178235 19.000000 to 0.178235 18.000000
\putrule from 0.178235 18.500000 to 0.575900 18.500000
\putrule from 0.575900 18.500000 to 0.575900 17.000000
\putrule from 0.575900 17.750000 to 0.854818 17.750000
\putrule from 0.854818 17.750000 to 0.854818 14.750000
\putrule from 0.854818 16.250000 to 1.297729 16.250000
\put {\mutation} at 1.255169 16.250000
\putrule from 0.000000 20.000000 to 0.080789 20.000000
\putrule from 0.000000 21.000000 to 0.080789 21.000000
\putrule from 0.080789 21.000000 to 0.080789 20.000000
\putrule from 0.080789 20.500000 to 0.540423 20.500000
\putrule from 0.000000 22.000000 to 0.221121 22.000000
\putrule from 0.000000 23.000000 to 0.221121 23.000000
\putrule from 0.221121 23.000000 to 0.221121 22.000000
\putrule from 0.221121 22.500000 to 0.474956 22.500000
\putrule from 0.000000 24.000000 to 0.053115 24.000000
\putrule from 0.000000 25.000000 to 0.053115 25.000000
\putrule from 0.053115 25.000000 to 0.053115 24.000000
\putrule from 0.053115 24.500000 to 0.474956 24.500000
\putrule from 0.474956 24.500000 to 0.474956 22.500000
\putrule from 0.474956 23.500000 to 0.540423 23.500000
\putrule from 0.540423 23.500000 to 0.540423 20.500000
\putrule from 0.540423 22.000000 to 0.747836 22.000000
\put {\mutation} at 0.729805 22.000000
\putrule from 0.000000 26.000000 to 0.272483 26.000000
\putrule from 0.000000 27.000000 to 0.062339 27.000000
\putrule from 0.000000 28.000000 to 0.062339 28.000000
\putrule from 0.062339 28.000000 to 0.062339 27.000000
\putrule from 0.062339 27.500000 to 0.272483 27.500000
\putrule from 0.272483 27.500000 to 0.272483 26.000000
\putrule from 0.272483 26.750000 to 0.590653 26.750000
\putrule from 0.000000 29.000000 to 0.590653 29.000000
\putrule from 0.590653 29.000000 to 0.590653 26.750000
\putrule from 0.590653 27.875000 to 0.747836 27.875000
\putrule from 0.747836 27.875000 to 0.747836 22.000000
\putrule from 0.747836 24.937500 to 1.174846 24.937500
\putrule from 0.000000 30.000000 to 0.100851 30.000000
\put {\mutation} at 0.028760 30.000000
\putrule from 0.000000 31.000000 to 0.100851 31.000000
\putrule from 0.100851 31.000000 to 0.100851 30.000000
\putrule from 0.100851 30.500000 to 0.101641 30.500000
\putrule from 0.000000 32.000000 to 0.101641 32.000000
\putrule from 0.101641 32.000000 to 0.101641 30.500000
\putrule from 0.101641 31.250000 to 0.926572 31.250000
\put {\mutation} at 0.194599 31.250000
\putrule from 0.000000 33.000000 to 0.926572 33.000000
\put {\mutation} at 0.850577 33.000000
\putrule from 0.926572 33.000000 to 0.926572 31.250000
\putrule from 0.926572 32.125000 to 1.174846 32.125000
\putrule from 1.174846 32.125000 to 1.174846 24.937500
\putrule from 1.174846 28.531250 to 1.297729 28.531250
\putrule from 1.297729 28.531250 to 1.297729 16.250000
\putrule from 1.297729 22.390625 to 3.813482 22.390625
\putrule from 0.000000 34.000000 to 2.317688 34.000000
\putrule from 0.000000 35.000000 to 0.061661 35.000000
\putrule from 0.000000 36.000000 to 0.036405 36.000000
\putrule from 0.000000 37.000000 to 0.036405 37.000000
\putrule from 0.036405 37.000000 to 0.036405 36.000000
\putrule from 0.036405 36.500000 to 0.061661 36.500000
\putrule from 0.061661 36.500000 to 0.061661 35.000000
\putrule from 0.061661 35.750000 to 0.225388 35.750000
\putrule from 0.000000 38.000000 to 0.225388 38.000000
\putrule from 0.225388 38.000000 to 0.225388 35.750000
\putrule from 0.225388 36.875000 to 0.333999 36.875000
\putrule from 0.000000 39.000000 to 0.017096 39.000000
\putrule from 0.000000 40.000000 to 0.017096 40.000000
\putrule from 0.017096 40.000000 to 0.017096 39.000000
\putrule from 0.017096 39.500000 to 0.199996 39.500000
\putrule from 0.000000 41.000000 to 0.029638 41.000000
\putrule from 0.000000 42.000000 to 0.029638 42.000000
\putrule from 0.029638 42.000000 to 0.029638 41.000000
\putrule from 0.029638 41.500000 to 0.087442 41.500000
\putrule from 0.000000 43.000000 to 0.087442 43.000000
\putrule from 0.087442 43.000000 to 0.087442 41.500000
\putrule from 0.087442 42.250000 to 0.199995 42.250000
\putrule from 0.199995 42.250000 to 0.199995 39.500000
\putrule from 0.199995 40.875000 to 0.233606 40.875000
\putrule from 0.000000 44.000000 to 0.233606 44.000000
\putrule from 0.233606 44.000000 to 0.233606 40.875000
\putrule from 0.233606 42.437500 to 0.333432 42.437500
\putrule from 0.000000 45.000000 to 0.063403 45.000000
\putrule from 0.000000 46.000000 to 0.063403 46.000000
\putrule from 0.063403 46.000000 to 0.063403 45.000000
\putrule from 0.063403 45.500000 to 0.333432 45.500000
\putrule from 0.333432 45.500000 to 0.333432 42.437500
\putrule from 0.333432 43.968750 to 0.333999 43.968750
\putrule from 0.333999 43.968750 to 0.333999 36.875000
\putrule from 0.333999 40.421875 to 0.815478 40.421875
\putrule from 0.000000 47.000000 to 0.000132 47.000000
\putrule from 0.000000 48.000000 to 0.000132 48.000000
\putrule from 0.000132 48.000000 to 0.000132 47.000000
\putrule from 0.000132 47.500000 to 0.011762 47.500000
\putrule from 0.000000 49.000000 to 0.011762 49.000000
\putrule from 0.011762 49.000000 to 0.011762 47.500000
\putrule from 0.011762 48.250000 to 0.815478 48.250000
\putrule from 0.815478 48.250000 to 0.815478 40.421875
\putrule from 0.815478 44.335938 to 2.317688 44.335938
\putrule from 2.317688 44.335938 to 2.317688 34.000000
\putrule from 2.317688 39.167969 to 3.813482 39.167969
\put {\mutation} at 3.516678 39.167969
\putrule from 3.813482 39.167969 to 3.813482 22.390625
\putrule from 3.813482 30.779297 to 4.281133 30.779297
\putrule from 4.281133 30.779297 to 4.281133 4.054688
\putrule from 4.281133 17.416992 to 8.000000 17.416992
%
%%%%%%%%%%%%%% Mismatch Distribution %%%%%%%%%%%%%%
%
\yunit=\plotht
\xunit=\plotwd
\Divide <\xunit> by <8.000000pt> forming <\xunit>
\Divide <\yunit> by <0.262857pt> forming <\yunit>
\advance\thusfar by \plotht
\advance\thusfar by \plotsp
\Divide <\thusfar> by <\yunit> forming <\offsety>
\placevalueinpts of <\offsety> in {\mktree_tmp}
\setcoordinatesystem units <\xunit, \yunit> point at 0 {\mktree_tmp}
\setplotarea x from 0.000000 to 8.000000, y from 0.000000 to 0.262857
\axis left label {\lines{Mismatch\cr distribution}} /
\axis bottom
   label {Pairwise differences}
   ticks numbered from 0 to 8 by 4 /
\multiput {$\circ$} at
 0 0.135510 1 0.120000 2 0.262857 3 0.076735 4 0.142041
 5 0.122449 6 0.107755 7 0.031020 8 0.001633
/
%Pr[random pair differs by i sites]
\plot 0 0.125000 1 0.109375 2 0.095703 3 0.083740 4 0.073273
      5 0.064114 6 0.056099 7 0.049087 8 0.042951 /
\plotht=1.100000\plotht
%
%%%%%%%%%%%%%% Simulated site frequency spectrum %%%%%%%%%%%%%%
%
\yunit=\plotht
\xunit=\plotwd
\Divide <\xunit> by <25.000000pt> forming <\xunit>
\Divide <\yunit> by <0.550000pt> forming <\yunit>
\advance\thusfar by \plotht
\advance\thusfar by \plotsp
\Divide <\thusfar> by <\yunit> forming <\offsety>
\placevalueinpts of <\offsety> in {\mktree_tmp}
\setcoordinatesystem units <\xunit, \yunit> point at 0 {\mktree_tmp}
\setplotarea x from 0.000000 to 25.000000, y from 0.000000 to 0.550000
\axis left invisible label {\lines{Site\cr frequency\cr spectrum}} /
\axis bottom
   label {Frequency of minor allele}
   ticks withvalues 0 {1/2} / at 0 25 / /
\linethickness 1.2pt
\axis top /
\linethickness 0.4pt
\sethistograms
\plot 0 0 1 0.500000 2 0.050000 3 0.150000 4 0.000000
 5 0.000000 6 0.100000 7 0.000000 8 0.000000 9 0.000000
 10 0.000000 11 0.000000 12 0.000000 13 0.000000 14 0.150000
 15 0.000000 16 0.050000 17 0.000000 18 0.000000 19 0.000000
 20 0.000000 21 0.000000 22 0.000000 23 0.000000 24 0.000000
 25 0.000000
/
\setlinear
%
%%%%%%%%%%%%%% Neutral expectation of site freq spectrum %%%%%%%%%%%%%%
%
\multiput {$\bullet$} at 
     0.500000 0.22781  1.500000 0.116278  2.500000 0.079168  3.500000 0.0606668  4.500000 0.049612
     5.500000 0.0422829  6.500000 0.0370854  7.500000 0.0332223  8.500000 0.0302512  9.500000 0.0279067
     10.500000 0.0260203  11.500000 0.0244796  12.500000 0.0232073  13.500000 0.0221482  14.500000 0.0212623
     15.500000 0.0205197  16.500000 0.0198978  17.500000 0.0193797  18.500000 0.0189519  19.500000 0.0186045
     20.500000 0.0183295  21.500000 0.0181213  22.500000 0.0179754  23.500000 0.0178889  24.500000 0.0178603
/
\endpicture}

\caption{An equilibrium population with $\theta=7$\label{fig.eq2}}
\end{figure}

\begin{figure}
\centering
% -*-latex-*-
%
%%%%%%%%%%%%%% Tree in PicTeX Format %%%%%%%%%%%%%%
%
%Sample size   = 50
%Mutation rate = 1
%     theta         mn        tau          K
%    7.0000     0.0000        Inf          1
%maxmm=11.000000 maxtau=0.000000 maxtree=14.368049 maxx=14.368049
%\newdimen\offsety
%\newdimen\yunit
%\newdimen\xunit
%\newdimen\thusfar
%\newdimen\plotht
%\newdimen\plotwd
%\newdimen\plotsp
\thusfar=0.000000in  % Keeps track of what's above
\plotht=1.400000in   % Height of each plot
\plotwd=4.000000in   % Width of each plot
\plotsp=0.500000in   % Spacing between plots
\mbox{\beginpicture
\def\mutation{\tiny$\bullet$}
\small
\valuestolabelleading=.4\baselineskip
\headingtoplotskip=0.4\baselineskip
%
%%%%%%%%%%%%%% Population size %%%%%%%%%%%%%%
%
\yunit=\plotht
\xunit=\plotwd
\Divide <\xunit> by <15.000000pt> forming <\xunit>
\Divide <\yunit> by <8.400001pt> forming <\yunit>
\advance\thusfar by \plotht
\advance\thusfar by \plotsp
\Divide <\thusfar> by <\yunit> forming <\offsety>
\placevalueinpts of <\offsety> in {\mktree_tmp}
\setcoordinatesystem units <\xunit, \yunit> point at 0 {\mktree_tmp}
\setplotarea x from 0.000000 to 15.000000, y from 0.000000 to 8.400001
\advance\thusfar by 0.030000\plotht
\axis left label {\lines{Population\cr size}} /
\axis bottom shiftedto y=-0.252000
    label {Mutational time before present}
    ticks numbered from 0 to 15 by 5 /
\putrule from 0.000000 7.000000 to 15.000000 7.000000
%
%%%%%%%%%%%%%% Gene Genealogy %%%%%%%%%%%%%%
%
\yunit=\plotht
\xunit=\plotwd
\Divide <\xunit> by <15.000000pt> forming <\xunit>
\Divide <\yunit> by <49.000000pt> forming <\yunit>
\advance\thusfar by \plotht
\advance\thusfar by \plotsp
\Divide <\thusfar> by <\yunit> forming <\offsety>
\placevalueinpts of <\offsety> in {\mktree_tmp}
\setcoordinatesystem units <\xunit, \yunit> point at 0 {\mktree_tmp}
\setplotarea x from 0.000000 to 15.000000, y from 0.000000 to 49.000000
\axis left invisible label {\lines{Gene\cr genealogy}} /
\axis bottom invisible
     label {Mutational time before present} /
\putrule from 0.000000 0.000000 to 0.204082 0.000000
\putrule from 0.000000 1.000000 to 0.204082 1.000000
\putrule from 0.204082 1.000000 to 0.204082 0.000000
\putrule from 0.204082 0.500000 to 0.285741 0.500000
\putrule from 0.000000 2.000000 to 0.285741 2.000000
\putrule from 0.285741 2.000000 to 0.285741 0.500000
\putrule from 0.285741 1.250000 to 0.576969 1.250000
\putrule from 0.000000 3.000000 to 0.000021 3.000000
\putrule from 0.000000 4.000000 to 0.000021 4.000000
\putrule from 0.000021 4.000000 to 0.000021 3.000000
\putrule from 0.000021 3.500000 to 0.446480 3.500000
\putrule from 0.000000 5.000000 to 0.215462 5.000000
\putrule from 0.000000 6.000000 to 0.215462 6.000000
\putrule from 0.215462 6.000000 to 0.215462 5.000000
\putrule from 0.215462 5.500000 to 0.387124 5.500000
\putrule from 0.000000 7.000000 to 0.276382 7.000000
\put {\mutation} at 0.005203 7.000000
\putrule from 0.000000 8.000000 to 0.276382 8.000000
\putrule from 0.276382 8.000000 to 0.276382 7.000000
\putrule from 0.276382 7.500000 to 0.387124 7.500000
\putrule from 0.387124 7.500000 to 0.387124 5.500000
\putrule from 0.387124 6.500000 to 0.446480 6.500000
\putrule from 0.446480 6.500000 to 0.446480 3.500000
\putrule from 0.446480 5.000000 to 0.576969 5.000000
\putrule from 0.576969 5.000000 to 0.576969 1.250000
\putrule from 0.576969 3.125000 to 1.504534 3.125000
\put {\mutation} at 0.767805 3.125000
\putrule from 0.000000 9.000000 to 1.504535 9.000000
\put {\mutation} at 0.107289 9.000000
\putrule from 1.504535 9.000000 to 1.504535 3.125000
\putrule from 1.504535 6.062500 to 3.263711 6.062500
\put {\mutation} at 2.406863 6.062500
\putrule from 0.000000 10.000000 to 0.037491 10.000000
\putrule from 0.000000 11.000000 to 0.037491 11.000000
\putrule from 0.037491 11.000000 to 0.037491 10.000000
\putrule from 0.037491 10.500000 to 0.117599 10.500000
\putrule from 0.000000 12.000000 to 0.117599 12.000000
\putrule from 0.117599 12.000000 to 0.117599 10.500000
\putrule from 0.117599 11.250000 to 0.160065 11.250000
\putrule from 0.000000 13.000000 to 0.039853 13.000000
\putrule from 0.000000 14.000000 to 0.039853 14.000000
\putrule from 0.039853 14.000000 to 0.039853 13.000000
\putrule from 0.039853 13.500000 to 0.160065 13.500000
\putrule from 0.160065 13.500000 to 0.160065 11.250000
\putrule from 0.160065 12.375000 to 0.704034 12.375000
\putrule from 0.000000 15.000000 to 0.704034 15.000000
\putrule from 0.704034 15.000000 to 0.704034 12.375000
\putrule from 0.704034 13.687500 to 3.263711 13.687500
\put {\mutation} at 1.909646 13.687500
\putrule from 3.263711 13.687500 to 3.263711 6.062500
\putrule from 3.263711 9.875000 to 13.061862 9.875000
\put {\mutation} at 8.206156 9.875000
\put {\mutation} at 3.650843 9.875000
\put {\mutation} at 12.237598 9.875000
\put {\mutation} at 5.010610 9.875000
\putrule from 0.000000 16.000000 to 0.255625 16.000000
\putrule from 0.000000 17.000000 to 0.255625 17.000000
\putrule from 0.255625 17.000000 to 0.255625 16.000000
\putrule from 0.255625 16.500000 to 0.386801 16.500000
\putrule from 0.000000 18.000000 to 0.143697 18.000000
\putrule from 0.000000 19.000000 to 0.047126 19.000000
\putrule from 0.000000 20.000000 to 0.047126 20.000000
\putrule from 0.047126 20.000000 to 0.047126 19.000000
\putrule from 0.047126 19.500000 to 0.143697 19.500000
\putrule from 0.143697 19.500000 to 0.143697 18.000000
\putrule from 0.143697 18.750000 to 0.380298 18.750000
\putrule from 0.000000 21.000000 to 0.206475 21.000000
\putrule from 0.000000 22.000000 to 0.206475 22.000000
\putrule from 0.206475 22.000000 to 0.206475 21.000000
\putrule from 0.206475 21.500000 to 0.380298 21.500000
\putrule from 0.380298 21.500000 to 0.380298 18.750000
\putrule from 0.380298 20.125000 to 0.386801 20.125000
\putrule from 0.386801 20.125000 to 0.386801 16.500000
\putrule from 0.386801 18.312500 to 4.110594 18.312500
\put {\mutation} at 2.671021 18.312500
\putrule from 0.000000 23.000000 to 0.295246 23.000000
\putrule from 0.000000 24.000000 to 0.156666 24.000000
\putrule from 0.000000 25.000000 to 0.150465 25.000000
\putrule from 0.000000 26.000000 to 0.150465 26.000000
\putrule from 0.150465 26.000000 to 0.150465 25.000000
\putrule from 0.150465 25.500000 to 0.156666 25.500000
\putrule from 0.156666 25.500000 to 0.156666 24.000000
\putrule from 0.156666 24.750000 to 0.295246 24.750000
\putrule from 0.295246 24.750000 to 0.295246 23.000000
\putrule from 0.295246 23.875000 to 3.108447 23.875000
\putrule from 0.000000 27.000000 to 0.057488 27.000000
\putrule from 0.000000 28.000000 to 0.057488 28.000000
\putrule from 0.057488 28.000000 to 0.057488 27.000000
\putrule from 0.057488 27.500000 to 0.593975 27.500000
\putrule from 0.000000 29.000000 to 0.525947 29.000000
\putrule from 0.000000 30.000000 to 0.001903 30.000000
\putrule from 0.000000 31.000000 to 0.001903 31.000000
\putrule from 0.001903 31.000000 to 0.001903 30.000000
\putrule from 0.001903 30.500000 to 0.525947 30.500000
\putrule from 0.525947 30.500000 to 0.525947 29.000000
\putrule from 0.525947 29.750000 to 0.593975 29.750000
\putrule from 0.593975 29.750000 to 0.593975 27.500000
\putrule from 0.593975 28.625000 to 0.705019 28.625000
\putrule from 0.000000 32.000000 to 0.050849 32.000000
\putrule from 0.000000 33.000000 to 0.050849 33.000000
\putrule from 0.050849 33.000000 to 0.050849 32.000000
\putrule from 0.050849 32.500000 to 0.074109 32.500000
\putrule from 0.000000 34.000000 to 0.074109 34.000000
\putrule from 0.074109 34.000000 to 0.074109 32.500000
\putrule from 0.074109 33.250000 to 0.432739 33.250000
\putrule from 0.000000 35.000000 to 0.380454 35.000000
\putrule from 0.000000 36.000000 to 0.087920 36.000000
\putrule from 0.000000 37.000000 to 0.057195 37.000000
\putrule from 0.000000 38.000000 to 0.057195 38.000000
\putrule from 0.057195 38.000000 to 0.057195 37.000000
\putrule from 0.057195 37.500000 to 0.087920 37.500000
\putrule from 0.087920 37.500000 to 0.087920 36.000000
\putrule from 0.087920 36.750000 to 0.334290 36.750000
\putrule from 0.000000 39.000000 to 0.210681 39.000000
\putrule from 0.000000 40.000000 to 0.210681 40.000000
\putrule from 0.210681 40.000000 to 0.210681 39.000000
\putrule from 0.210681 39.500000 to 0.334290 39.500000
\putrule from 0.334290 39.500000 to 0.334290 36.750000
\putrule from 0.334290 38.125000 to 0.338654 38.125000
\putrule from 0.000000 41.000000 to 0.250006 41.000000
\put {\mutation} at 0.083567 41.000000
\putrule from 0.000000 42.000000 to 0.047636 42.000000
\putrule from 0.000000 43.000000 to 0.047636 43.000000
\putrule from 0.047636 43.000000 to 0.047636 42.000000
\putrule from 0.047636 42.500000 to 0.250006 42.500000
\putrule from 0.250006 42.500000 to 0.250006 41.000000
\putrule from 0.250006 41.750000 to 0.338654 41.750000
\putrule from 0.338654 41.750000 to 0.338654 38.125000
\putrule from 0.338654 39.937500 to 0.380455 39.937500
\putrule from 0.380455 39.937500 to 0.380455 35.000000
\putrule from 0.380455 37.468750 to 0.432740 37.468750
\putrule from 0.432740 37.468750 to 0.432740 33.250000
\putrule from 0.432740 35.359375 to 0.621198 35.359375
\putrule from 0.000000 44.000000 to 0.621198 44.000000
\putrule from 0.621198 44.000000 to 0.621198 35.359375
\putrule from 0.621198 39.679688 to 0.705019 39.679688
\putrule from 0.705019 39.679688 to 0.705019 28.625000
\putrule from 0.705019 34.152344 to 0.720571 34.152344
\putrule from 0.000000 45.000000 to 0.154577 45.000000
\put {\mutation} at 0.139029 45.000000
\putrule from 0.000000 46.000000 to 0.024650 46.000000
\putrule from 0.000000 47.000000 to 0.024650 47.000000
\putrule from 0.024650 47.000000 to 0.024650 46.000000
\putrule from 0.024650 46.500000 to 0.154577 46.500000
\putrule from 0.154577 46.500000 to 0.154577 45.000000
\putrule from 0.154577 45.750000 to 0.720571 45.750000
\putrule from 0.720571 45.750000 to 0.720571 34.152344
\putrule from 0.720571 39.951172 to 3.108447 39.951172
\putrule from 3.108447 39.951172 to 3.108447 23.875000
\putrule from 3.108447 31.913086 to 4.110594 31.913086
\putrule from 4.110594 31.913086 to 4.110594 18.312500
\putrule from 4.110594 25.112793 to 4.141377 25.112793
\putrule from 0.000000 48.000000 to 0.216789 48.000000
\putrule from 0.000000 49.000000 to 0.216789 49.000000
\putrule from 0.216789 49.000000 to 0.216789 48.000000
\putrule from 0.216789 48.500000 to 4.141377 48.500000
\put {\mutation} at 0.624563 48.500000
\put {\mutation} at 1.745603 48.500000
\putrule from 4.141377 48.500000 to 4.141377 25.112793
\putrule from 4.141377 36.806396 to 13.061862 36.806396
\put {\mutation} at 9.482232 36.806396
\put {\mutation} at 5.187711 36.806396
\putrule from 13.061862 36.806396 to 13.061862 9.875000
\putrule from 13.061862 23.340698 to 15.000000 23.340698
%
%%%%%%%%%%%%%% Mismatch Distribution %%%%%%%%%%%%%%
%
\yunit=\plotht
\xunit=\plotwd
\Divide <\xunit> by <15.000000pt> forming <\xunit>
\Divide <\yunit> by <0.259592pt> forming <\yunit>
\advance\thusfar by \plotht
\advance\thusfar by \plotsp
\Divide <\thusfar> by <\yunit> forming <\offsety>
\placevalueinpts of <\offsety> in {\mktree_tmp}
\setcoordinatesystem units <\xunit, \yunit> point at 0 {\mktree_tmp}
\setplotarea x from 0.000000 to 15.000000, y from 0.000000 to 0.259592
\axis left label {\lines{Mismatch\cr distribution}} /
\axis bottom
   label {Pairwise differences}
   ticks numbered from 0 to 15 by 5 /
\multiput {$\circ$} at
 0 0.259592 1 0.175510 2 0.056327 3 0.059592 4 0.004898
 5 0.000000 6 0.000000 7 0.112653 8 0.213061 9 0.094694
 10 0.022041 11 0.001633
/
%Pr[random pair differs by i sites]
\plot 0 0.125000 1 0.109375 2 0.095703 3 0.083740 4 0.073273
      5 0.064114 6 0.056099 7 0.049087 8 0.042951 9 0.037582
      10 0.032884 11 0.028774 12 0.025177 13 0.022030 14 0.019276
      15 0.016867
/
\plotht=1.100000\plotht
%
%%%%%%%%%%%%%% Simulated site frequency spectrum %%%%%%%%%%%%%%
%
\yunit=\plotht
\xunit=\plotwd
\Divide <\xunit> by <25.000000pt> forming <\xunit>
\Divide <\yunit> by <0.412500pt> forming <\yunit>
\advance\thusfar by \plotht
\advance\thusfar by \plotsp
\Divide <\thusfar> by <\yunit> forming <\offsety>
\placevalueinpts of <\offsety> in {\mktree_tmp}
\setcoordinatesystem units <\xunit, \yunit> point at 0 {\mktree_tmp}
\setplotarea x from 0.000000 to 25.000000, y from 0.000000 to 0.412500
\axis left invisible label {\lines{Site\cr frequency\cr spectrum}} /
\axis bottom
   label {Frequency of minor allele}
   ticks withvalues 0 {1/2} / at 0 25 / /
\linethickness 1.2pt
\axis top /
\linethickness 0.4pt
\sethistograms
\plot 0 0 1 0.250000 2 0.125000 3 0.000000 4 0.000000
 5 0.000000 6 0.062500 7 0.062500 8 0.000000 9 0.062500
 10 0.062500 11 0.000000 12 0.000000 13 0.000000 14 0.000000
 15 0.000000 16 0.375000 17 0.000000 18 0.000000 19 0.000000
 20 0.000000 21 0.000000 22 0.000000 23 0.000000 24 0.000000
 25 0.000000
/
\setlinear
%
%%%%%%%%%%%%%% Neutral expectation of site freq spectrum %%%%%%%%%%%%%%
%
\multiput {$\bullet$} at 
     0.500000 0.22781  1.500000 0.116278  2.500000 0.079168  3.500000 0.0606668  4.500000 0.049612
     5.500000 0.0422829  6.500000 0.0370854  7.500000 0.0332223  8.500000 0.0302512  9.500000 0.0279067
     10.500000 0.0260203  11.500000 0.0244796  12.500000 0.0232073  13.500000 0.0221482  14.500000 0.0212623
     15.500000 0.0205197  16.500000 0.0198978  17.500000 0.0193797  18.500000 0.0189519  19.500000 0.0186045
     20.500000 0.0183295  21.500000 0.0181213  22.500000 0.0179754  23.500000 0.0178889  24.500000 0.0178603
/
\endpicture}

\caption{An equilibrium population with $\theta=7$\label{fig.eq3}}
\end{figure}

\begin{figure}
\centering
% -*-latex-*-
%
%%%%%%%%%%%%%% Tree in PicTeX Format %%%%%%%%%%%%%%
%
%Sample size   = 50
%Mutation rate = 1
%     theta         mn        tau          K
%    7.0000     0.0000        Inf          1
%maxmm=14.000000 maxtau=0.000000 maxtree=18.241398 maxx=18.241398
%\newdimen\offsety
%\newdimen\yunit
%\newdimen\xunit
%\newdimen\thusfar
%\newdimen\plotht
%\newdimen\plotwd
%\newdimen\plotsp
\thusfar=0.000000in  % Keeps track of what's above
\plotht=0.250000in   % Height of each plot
\plotwd=0.7\textwidth   % Width of each plot
\plotsp=0.500000in   % Spacing between plots
\begin{center}
\mbox{\beginpicture
\def\mutation{\tiny$\bullet$}
\small
\valuestolabelleading=.4\baselineskip
\headingtoplotskip=0.4\baselineskip
%
%%%%%%%%%%%%%% Population size %%%%%%%%%%%%%%
%
\yunit=\plotht
\xunit=\plotwd
\Divide <\xunit> by <19.000000pt> forming <\xunit>
\Divide <\yunit> by <8.400001pt> forming <\yunit>
\advance\thusfar by \plotht
\advance\thusfar by \plotsp
\Divide <\thusfar> by <\yunit> forming <\offsety>
\placevalueinpts of <\offsety> in {\mktree_tmp}
\setcoordinatesystem units <\xunit, \yunit> point at 0 {\mktree_tmp}
\setplotarea x from 0.000000 to 19.000000, y from 0.000000 to 8.400001
\advance\thusfar by 0.030000\plotht
\axis left label {\lines{Population\cr size}} /
\axis bottom shiftedto y=-0.252000
    label {Mutational time before present}
    ticks numbered from 0 to 18 by 9 /
\putrule from 0.000000 7.000000 to 19.000000 7.000000
%
%%%%%%%%%%%%%% Gene Genealogy %%%%%%%%%%%%%%
%
\yunit=\plotht
\xunit=\plotwd
\Divide <\xunit> by <19.000000pt> forming <\xunit>
\Divide <\yunit> by <49.000000pt> forming <\yunit>
\advance\thusfar by \plotht
\advance\thusfar by \plotsp
\Divide <\thusfar> by <\yunit> forming <\offsety>
\placevalueinpts of <\offsety> in {\mktree_tmp}
\setcoordinatesystem units <\xunit, \yunit> point at 0 {\mktree_tmp}
\setplotarea x from 0.000000 to 19.000000, y from 0.000000 to 49.000000
\axis left invisible label {\lines{Gene\cr genealogy}} /
\axis bottom invisible
     label {Mutational time before present} /
\putrule from 0.000000 0.000000 to 0.999090 0.000000
\putrule from 0.000000 1.000000 to 0.502273 1.000000
\putrule from 0.000000 2.000000 to 0.432293 2.000000
\putrule from 0.000000 3.000000 to 0.275172 3.000000
\putrule from 0.000000 4.000000 to 0.061897 4.000000
\putrule from 0.000000 5.000000 to 0.061897 5.000000
\putrule from 0.061897 5.000000 to 0.061897 4.000000
\putrule from 0.061897 4.500000 to 0.275171 4.500000
\putrule from 0.275171 4.500000 to 0.275171 3.000000
\putrule from 0.275171 3.750000 to 0.432293 3.750000
\putrule from 0.432293 3.750000 to 0.432293 2.000000
\putrule from 0.432293 2.875000 to 0.502273 2.875000
\putrule from 0.502273 2.875000 to 0.502273 1.000000
\putrule from 0.502273 1.937500 to 0.999090 1.937500
\putrule from 0.999090 1.937500 to 0.999090 0.000000
\putrule from 0.999090 0.968750 to 16.583090 0.968750
\put {\mutation} at 1.222834 0.968750
\put {\mutation} at 3.882470 0.968750
\put {\mutation} at 3.840911 0.968750
\put {\mutation} at 11.680396 0.968750
\put {\mutation} at 3.112742 0.968750
\put {\mutation} at 9.442352 0.968750
\put {\mutation} at 9.824396 0.968750
\put {\mutation} at 12.536413 0.968750
\putrule from 0.000000 6.000000 to 0.389336 6.000000
\putrule from 0.000000 7.000000 to 0.090103 7.000000
\putrule from 0.000000 8.000000 to 0.090103 8.000000
\putrule from 0.090103 8.000000 to 0.090103 7.000000
\putrule from 0.090103 7.500000 to 0.389336 7.500000
\putrule from 0.389336 7.500000 to 0.389336 6.000000
\putrule from 0.389336 6.750000 to 3.798044 6.750000
\put {\mutation} at 1.774263 6.750000
\putrule from 0.000000 9.000000 to 0.000206 9.000000
\putrule from 0.000000 10.000000 to 0.000206 10.000000
\putrule from 0.000206 10.000000 to 0.000206 9.000000
\putrule from 0.000206 9.500000 to 0.125942 9.500000
\putrule from 0.000000 11.000000 to 0.125942 11.000000
\putrule from 0.125942 11.000000 to 0.125942 9.500000
\putrule from 0.125942 10.250000 to 0.347003 10.250000
\putrule from 0.000000 12.000000 to 0.341617 12.000000
\putrule from 0.000000 13.000000 to 0.211500 13.000000
\putrule from 0.000000 14.000000 to 0.093828 14.000000
\putrule from 0.000000 15.000000 to 0.093828 15.000000
\putrule from 0.093828 15.000000 to 0.093828 14.000000
\putrule from 0.093828 14.500000 to 0.143514 14.500000
\putrule from 0.000000 16.000000 to 0.143514 16.000000
\putrule from 0.143514 16.000000 to 0.143514 14.500000
\putrule from 0.143514 15.250000 to 0.163295 15.250000
\putrule from 0.000000 17.000000 to 0.012244 17.000000
\putrule from 0.000000 18.000000 to 0.012244 18.000000
\putrule from 0.012244 18.000000 to 0.012244 17.000000
\putrule from 0.012244 17.500000 to 0.163295 17.500000
\putrule from 0.163295 17.500000 to 0.163295 15.250000
\putrule from 0.163295 16.375000 to 0.211500 16.375000
\putrule from 0.211500 16.375000 to 0.211500 13.000000
\putrule from 0.211500 14.687500 to 0.240437 14.687500
\putrule from 0.000000 19.000000 to 0.240437 19.000000
\put {\mutation} at 0.208058 19.000000
\putrule from 0.240437 19.000000 to 0.240437 14.687500
\putrule from 0.240437 16.843750 to 0.341617 16.843750
\putrule from 0.341617 16.843750 to 0.341617 12.000000
\putrule from 0.341617 14.421875 to 0.347003 14.421875
\putrule from 0.347003 14.421875 to 0.347003 10.250000
\putrule from 0.347003 12.335938 to 0.794500 12.335938
\put {\mutation} at 0.534861 12.335938
\putrule from 0.000000 20.000000 to 0.794500 20.000000
\putrule from 0.794500 20.000000 to 0.794500 12.335938
\putrule from 0.794500 16.167969 to 1.000484 16.167969
\putrule from 0.000000 21.000000 to 1.000484 21.000000
\putrule from 1.000484 21.000000 to 1.000484 16.167969
\putrule from 1.000484 18.583984 to 2.140445 18.583984
\put {\mutation} at 1.201656 18.583984
\putrule from 0.000000 22.000000 to 0.080740 22.000000
\put {\mutation} at 0.065228 22.000000
\putrule from 0.000000 23.000000 to 0.080740 23.000000
\putrule from 0.080740 23.000000 to 0.080740 22.000000
\putrule from 0.080740 22.500000 to 0.203982 22.500000
\putrule from 0.000000 24.000000 to 0.203982 24.000000
\putrule from 0.203982 24.000000 to 0.203982 22.500000
\putrule from 0.203982 23.250000 to 0.715645 23.250000
\putrule from 0.000000 25.000000 to 0.715645 25.000000
\putrule from 0.715645 25.000000 to 0.715645 23.250000
\putrule from 0.715645 24.125000 to 2.140445 24.125000
\put {\mutation} at 1.283084 24.125000
\putrule from 2.140445 24.125000 to 2.140445 18.583984
\putrule from 2.140445 21.354492 to 3.798045 21.354492
\put {\mutation} at 3.301491 21.354492
\putrule from 3.798045 21.354492 to 3.798045 6.750000
\putrule from 3.798045 14.052246 to 6.988003 14.052246
\putrule from 0.000000 26.000000 to 0.533399 26.000000
\putrule from 0.000000 27.000000 to 0.082279 27.000000
\putrule from 0.000000 28.000000 to 0.082279 28.000000
\putrule from 0.082279 28.000000 to 0.082279 27.000000
\putrule from 0.082279 27.500000 to 0.533399 27.500000
\putrule from 0.533399 27.500000 to 0.533399 26.000000
\putrule from 0.533399 26.750000 to 0.539262 26.750000
\putrule from 0.000000 29.000000 to 0.539262 29.000000
\putrule from 0.539262 29.000000 to 0.539262 26.750000
\putrule from 0.539262 27.875000 to 1.283239 27.875000
\putrule from 0.000000 30.000000 to 0.189412 30.000000
\putrule from 0.000000 31.000000 to 0.141156 31.000000
\putrule from 0.000000 32.000000 to 0.141156 32.000000
\putrule from 0.141156 32.000000 to 0.141156 31.000000
\putrule from 0.141156 31.500000 to 0.189412 31.500000
\putrule from 0.189412 31.500000 to 0.189412 30.000000
\putrule from 0.189412 30.750000 to 0.405091 30.750000
\putrule from 0.000000 33.000000 to 0.262861 33.000000
\putrule from 0.000000 34.000000 to 0.194704 34.000000
\putrule from 0.000000 35.000000 to 0.060466 35.000000
\putrule from 0.000000 36.000000 to 0.060466 36.000000
\putrule from 0.060466 36.000000 to 0.060466 35.000000
\putrule from 0.060466 35.500000 to 0.194704 35.500000
\putrule from 0.194704 35.500000 to 0.194704 34.000000
\putrule from 0.194704 34.750000 to 0.262861 34.750000
\putrule from 0.262861 34.750000 to 0.262861 33.000000
\putrule from 0.262861 33.875000 to 0.352252 33.875000
\put {\mutation} at 0.349213 33.875000
\putrule from 0.000000 37.000000 to 0.275630 37.000000
\putrule from 0.000000 38.000000 to 0.236910 38.000000
\putrule from 0.000000 39.000000 to 0.236910 39.000000
\putrule from 0.236910 39.000000 to 0.236910 38.000000
\putrule from 0.236910 38.500000 to 0.275630 38.500000
\putrule from 0.275630 38.500000 to 0.275630 37.000000
\putrule from 0.275630 37.750000 to 0.352252 37.750000
\putrule from 0.352252 37.750000 to 0.352252 33.875000
\putrule from 0.352252 35.812500 to 0.405091 35.812500
\putrule from 0.405091 35.812500 to 0.405091 30.750000
\putrule from 0.405091 33.281250 to 1.283239 33.281250
\putrule from 1.283239 33.281250 to 1.283239 27.875000
\putrule from 1.283239 30.578125 to 3.217756 30.578125
\put {\mutation} at 1.957520 30.578125
\putrule from 0.000000 40.000000 to 2.264585 40.000000
\putrule from 0.000000 41.000000 to 0.605036 41.000000
\putrule from 0.000000 42.000000 to 0.282794 42.000000
\put {\mutation} at 0.116015 42.000000
\putrule from 0.000000 43.000000 to 0.282794 43.000000
\putrule from 0.282794 43.000000 to 0.282794 42.000000
\putrule from 0.282794 42.500000 to 0.499285 42.500000
\putrule from 0.000000 44.000000 to 0.035450 44.000000
\putrule from 0.000000 45.000000 to 0.009211 45.000000
\putrule from 0.000000 46.000000 to 0.009211 46.000000
\putrule from 0.009211 46.000000 to 0.009211 45.000000
\putrule from 0.009211 45.500000 to 0.035450 45.500000
\put {\mutation} at 0.020934 45.500000
\putrule from 0.035450 45.500000 to 0.035450 44.000000
\putrule from 0.035450 44.750000 to 0.499285 44.750000
\putrule from 0.499285 44.750000 to 0.499285 42.500000
\putrule from 0.499285 43.625000 to 0.605036 43.625000
\putrule from 0.605036 43.625000 to 0.605036 41.000000
\putrule from 0.605036 42.312500 to 0.938469 42.312500
\putrule from 0.000000 47.000000 to 0.238608 47.000000
\putrule from 0.000000 48.000000 to 0.106546 48.000000
\putrule from 0.000000 49.000000 to 0.106546 49.000000
\putrule from 0.106546 49.000000 to 0.106546 48.000000
\putrule from 0.106546 48.500000 to 0.238608 48.500000
\putrule from 0.238608 48.500000 to 0.238608 47.000000
\putrule from 0.238608 47.750000 to 0.938469 47.750000
\putrule from 0.938469 47.750000 to 0.938469 42.312500
\putrule from 0.938469 45.031250 to 2.264585 45.031250
\putrule from 2.264585 45.031250 to 2.264585 40.000000
\putrule from 2.264585 42.515625 to 3.217756 42.515625
\put {\mutation} at 2.834729 42.515625
\putrule from 3.217756 42.515625 to 3.217756 30.578125
\putrule from 3.217756 36.546875 to 6.988003 36.546875
\put {\mutation} at 4.318192 36.546875
\put {\mutation} at 6.293453 36.546875
\put {\mutation} at 5.542046 36.546875
\putrule from 6.988003 36.546875 to 6.988003 14.052246
\putrule from 6.988003 25.299561 to 16.583088 25.299561
\put {\mutation} at 7.570506 25.299561
\putrule from 16.583088 25.299561 to 16.583088 0.968750
\putrule from 16.583088 13.134155 to 19.000000 13.134155
%
%%%%%%%%%%%%%% Mismatch Distribution %%%%%%%%%%%%%%
%
\yunit=\plotht
\xunit=\plotwd
\Divide <\xunit> by <19.000000pt> forming <\xunit>
\Divide <\yunit> by <0.181224pt> forming <\yunit>
\advance\thusfar by \plotht
\advance\thusfar by \plotsp
\Divide <\thusfar> by <\yunit> forming <\offsety>
\placevalueinpts of <\offsety> in {\mktree_tmp}
\setcoordinatesystem units <\xunit, \yunit> point at 0 {\mktree_tmp}
\setplotarea x from 0.000000 to 19.000000, y from 0.000000 to 0.181224
\axis left label {\lines{Mismatch\cr distribution}} /
\axis bottom
   label {Pairwise differences}
   ticks numbered from 0 to 18 by 9 /
\multiput {$\circ$} at
 0 0.114286 1 0.076735 2 0.065306 3 0.085714 4 0.047347
 5 0.044898 6 0.086531 7 0.181224 8 0.076735 9 0.005714
 10 0.014694 11 0.024490 12 0.053878 13 0.088163 14 0.034286
/
%Pr[random pair differs by i sites]
\plot 0 0.125000 1 0.109375 2 0.095703 3 0.083740 4 0.073273
      5 0.064114 6 0.056099 7 0.049087 8 0.042951 9 0.037582
      10 0.032884 11 0.028774 12 0.025177 13 0.022030 14 0.019276
      15 0.016867 16 0.014758 17 0.012914 18 0.011299 19 0.009887 
/
\plotht=1.100000\plotht
%
%%%%%%%%%%%%%% Simulated site frequency spectrum %%%%%%%%%%%%%%
%
\yunit=\plotht
\xunit=\plotwd
\Divide <\xunit> by <25.000000pt> forming <\xunit>
\Divide <\yunit> by <0.412500pt> forming <\yunit>
\advance\thusfar by \plotht
\advance\thusfar by \plotsp
\Divide <\thusfar> by <\yunit> forming <\offsety>
\placevalueinpts of <\offsety> in {\mktree_tmp}
\setcoordinatesystem units <\xunit, \yunit> point at 0 {\mktree_tmp}
\setplotarea x from 0.000000 to 25.000000, y from 0.000000 to 0.412500
\axis left invisible label {\lines{Site\cr frequency\cr spectrum}} /
\axis bottom
   label {Frequency of minor allele}
   ticks withvalues 0 {1/2} / at 0 25 / /
\linethickness 1.2pt
\axis top /
\linethickness 0.4pt
\sethistograms
\plot 0 0 1 0.125000 2 0.041667 3 0.041667 4 0.083333
 5 0.000000 6 0.375000 7 0.000000 8 0.000000 9 0.000000
 10 0.041667 11 0.041667 12 0.000000 13 0.041667 14 0.041667
 15 0.000000 16 0.000000 17 0.041667 18 0.000000 19 0.000000
 20 0.000000 21 0.000000 22 0.000000 23 0.000000 24 0.125000
 25 0.000000
/
\setlinear
%
%%%%%%%%%%%%%% Neutral expectation of site freq spectrum %%%%%%%%%%%%%%
%
\multiput {$\bullet$} at 
     0.500000 0.22781  1.500000 0.116278  2.500000 0.079168  3.500000 0.0606668  4.500000 0.049612
     5.500000 0.0422829  6.500000 0.0370854  7.500000 0.0332223  8.500000 0.0302512  9.500000 0.0279067
     10.500000 0.0260203  11.500000 0.0244796  12.500000 0.0232073  13.500000 0.0221482  14.500000 0.0212623
     15.500000 0.0205197  16.500000 0.0198978  17.500000 0.0193797  18.500000 0.0189519  19.500000 0.0186045
     20.500000 0.0183295  21.500000 0.0181213  22.500000 0.0179754  23.500000 0.0178889  24.500000 0.0178603
/
\endpicture}\\
An equilibrium population with $\theta=7$
\end{center}

%%% Local Variables: 
%%% mode: latex
%%% TeX-master: t
%%% End: 

\caption{An equilibrium population with $\theta=7$\label{fig.eq4}}
\end{figure}
\clearpage

\section{Simulations of expanded populations}
\label{sec.popgrow}

Figures~\ref{fig.gro1}--\ref{fig.gro4}, on
pages~\pageref{fig.gro1}--\pageref{fig.gro4}, show simulations of
populations that grew suddenly by 100-fold at 7 units of mutational
time before the present.  In these graphs, the solid lines drawn for
mismatch distributions refer to the expectation under the model of
population history that was used in the simulations.  On the other
hand, the filled circles shown with the site frequency spectra refer
as before to a model of constant population size.

\paragraph{The gene genealogies} of these expanded populations look
different.  Coalescent events occur only rarely during the period when 
the population was large, but occur rapidly in the earlier period when 
the population was small.  This gives the gene genealogies a comb-like 
shape, which seldom appears in the genealogies of stationary
populations.  Many pairs of individuals differ by just over 7 units of 
mutational time.

\paragraph{The mismatch distributions} in these simulations are all
unimodal, with peaks just a little before 7.  This reflects the fact
that many pairs of individuals differ by just over 7 units of
mutational time.

\paragraph{The spectra} in these simulations exhibit an excess of
singletons.  This is because the terminal branches in the gene
genealogies are long and attract a disproportionate number of the
mutations.  The mutations that fall on these long terminal branches
are all singletons.  Methods for calculating the expected spectrum are
introduced by \citet{Harpending:PNA-95-1961} and by
\citet{Wooding:G-161-1641}.

\begin{figure}
\centering
% -*-latex-*-
%
%%%%%%%%%%%%%% Tree in PicTeX Format %%%%%%%%%%%%%%
%
%Sample size   = 50
%Mutation rate = 1
%     theta         mn        tau          K
%  100.0000     0.0000     7.0000          1
%    1.0000     0.0000        Inf          1
%maxmm=20.000000 maxtau=7.700000 maxtree=8.748507 maxx=20.000000
%\newdimen\offsety
%\newdimen\yunit
%\newdimen\xunit
%\newdimen\thusfar
%\newdimen\plotht
%\newdimen\plotwd
%\newdimen\plotsp
\thusfar=0.000000in  % Keeps track of what's above
\plotht=0.250000in   % Height of each plot
\plotwd=0.7\textwidth   % Width of each plot
\plotsp=0.500000in   % Spacing between plots
\let\put\pictexput
\mbox{\beginpicture
\def\mutation{\tiny$\bullet$}
\small
\valuestolabelleading=.4\baselineskip
\headingtoplotskip=0.4\baselineskip
%
%%%%%%%%%%%%%% Population size %%%%%%%%%%%%%%
%
\yunit=\plotht
\xunit=\plotwd
\Divide <\xunit> by <20.000000pt> forming <\xunit>
\Divide <\yunit> by <120.000008pt> forming <\yunit>
\advance\thusfar by \plotht
\advance\thusfar by \plotsp
\Divide <\thusfar> by <\yunit> forming <\offsety>
\placevalueinpts of <\offsety> in {\mktree_tmp}
\setcoordinatesystem units <\xunit, \yunit> point at 0 {\mktree_tmp}
\setplotarea x from 0.000000 to 20.000000, y from 0.000000 to 120.000008
\advance\thusfar by 0.030000\plotht
\axis left label {\lines{Population\cr size}} /
\axis bottom shiftedto y=-3.600000
    label {Mutational time before present}
    ticks numbered from 0 to 20 by 10 /
\putrule from 0.000000 100.000000 to 7.000000 100.000000
\putrule from 7.000000 100.000000 to 7.000000 1.000000
\putrule from 7.000000 1.000000 to 20.000000 1.000000
%
%%%%%%%%%%%%%% Gene Genealogy %%%%%%%%%%%%%%
%
\yunit=\plotht
\xunit=\plotwd
\Divide <\xunit> by <20.000000pt> forming <\xunit>
\Divide <\yunit> by <49.000000pt> forming <\yunit>
\advance\thusfar by \plotht
\advance\thusfar by \plotsp
\Divide <\thusfar> by <\yunit> forming <\offsety>
\placevalueinpts of <\offsety> in {\mktree_tmp}
\setcoordinatesystem units <\xunit, \yunit> point at 0 {\mktree_tmp}
\setplotarea x from 0.000000 to 20.000000, y from 0.000000 to 49.000000
\axis left invisible label {\lines{Gene\cr genealogy}} /
\axis bottom invisible
     label {Mutational time before present} /
\putrule from 0.000000 0.000000 to 7.627243 0.000000
\put {\mutation} at 7.511514 0.000000
\put {\mutation} at 3.887524 0.000000
\put {\mutation} at 5.560976 0.000000
\putrule from 0.000000 1.000000 to 0.136694 1.000000
\putrule from 0.000000 2.000000 to 0.136694 2.000000
\putrule from 0.136694 2.000000 to 0.136694 1.000000
\putrule from 0.136694 1.500000 to 7.006217 1.500000
\put {\mutation} at 1.224997 1.500000
\put {\mutation} at 1.389629 1.500000
\put {\mutation} at 2.431011 1.500000
\put {\mutation} at 5.308657 1.500000
\put {\mutation} at 3.022621 1.500000
\putrule from 0.000000 3.000000 to 2.133670 3.000000
\put {\mutation} at 1.563863 3.000000
\putrule from 0.000000 4.000000 to 2.133670 4.000000
\put {\mutation} at 0.411467 4.000000
\putrule from 2.133670 4.000000 to 2.133670 3.000000
\putrule from 2.133670 3.500000 to 2.843236 3.500000
\putrule from 0.000000 5.000000 to 0.251554 5.000000
\putrule from 0.000000 6.000000 to 0.251554 6.000000
\putrule from 0.251554 6.000000 to 0.251554 5.000000
\putrule from 0.251554 5.500000 to 0.845390 5.500000
\putrule from 0.000000 7.000000 to 0.845390 7.000000
\put {\mutation} at 0.452105 7.000000
\putrule from 0.845390 7.000000 to 0.845390 5.500000
\putrule from 0.845390 6.250000 to 2.843236 6.250000
\put {\mutation} at 0.847289 6.250000
\putrule from 2.843236 6.250000 to 2.843236 3.500000
\putrule from 2.843236 4.875000 to 7.006217 4.875000
\put {\mutation} at 4.334228 4.875000
\put {\mutation} at 4.408728 4.875000
\putrule from 7.006217 4.875000 to 7.006217 1.500000
\putrule from 7.006217 3.187500 to 7.340049 3.187500
\put {\mutation} at 7.020493 3.187500
\putrule from 0.000000 8.000000 to 7.125215 8.000000
\put {\mutation} at 6.773659 8.000000
\put {\mutation} at 3.924935 8.000000
\put {\mutation} at 4.348257 8.000000
\putrule from 0.000000 9.000000 to 7.125215 9.000000
\put {\mutation} at 4.086046 9.000000
\put {\mutation} at 4.149639 9.000000
\put {\mutation} at 3.990512 9.000000
\put {\mutation} at 3.135542 9.000000
\put {\mutation} at 7.043122 9.000000
\put {\mutation} at 6.415633 9.000000
\putrule from 7.125215 9.000000 to 7.125215 8.000000
\putrule from 7.125215 8.500000 to 7.133833 8.500000
\putrule from 0.000000 10.000000 to 7.133833 10.000000
\put {\mutation} at 0.106589 10.000000
\put {\mutation} at 5.752342 10.000000
\put {\mutation} at 6.010053 10.000000
\putrule from 7.133833 10.000000 to 7.133833 8.500000
\putrule from 7.133833 9.250000 to 7.340049 9.250000
\putrule from 7.340049 9.250000 to 7.340049 3.187500
\putrule from 7.340049 6.218750 to 7.627243 6.218750
\putrule from 7.627243 6.218750 to 7.627243 0.000000
\putrule from 7.627243 3.109375 to 7.953188 3.109375
\putrule from 0.000000 11.000000 to 6.726050 11.000000
\put {\mutation} at 0.620364 11.000000
\put {\mutation} at 0.690726 11.000000
\put {\mutation} at 5.391620 11.000000
\put {\mutation} at 3.257604 11.000000
\put {\mutation} at 3.841564 11.000000
\put {\mutation} at 4.610816 11.000000
\put {\mutation} at 4.919965 11.000000
\putrule from 0.000000 12.000000 to 1.519580 12.000000
\putrule from 0.000000 13.000000 to 1.519580 13.000000
\put {\mutation} at 0.833057 13.000000
\put {\mutation} at 0.432966 13.000000
\put {\mutation} at 1.007880 13.000000
\putrule from 1.519580 13.000000 to 1.519580 12.000000
\putrule from 1.519580 12.500000 to 4.231307 12.500000
\put {\mutation} at 4.184648 12.500000
\put {\mutation} at 2.260075 12.500000
\put {\mutation} at 2.769642 12.500000
\putrule from 0.000000 14.000000 to 0.716411 14.000000
\put {\mutation} at 0.066226 14.000000
\putrule from 0.000000 15.000000 to 0.716411 15.000000
\putrule from 0.716411 15.000000 to 0.716411 14.000000
\putrule from 0.716411 14.500000 to 4.231307 14.500000
\put {\mutation} at 1.717686 14.500000
\put {\mutation} at 1.464990 14.500000
\put {\mutation} at 2.175282 14.500000
\putrule from 4.231307 14.500000 to 4.231307 12.500000
\putrule from 4.231307 13.500000 to 6.726050 13.500000
\put {\mutation} at 6.559241 13.500000
\putrule from 6.726050 13.500000 to 6.726050 11.000000
\putrule from 6.726050 12.250000 to 7.154037 12.250000
\putrule from 0.000000 16.000000 to 1.567791 16.000000
\putrule from 0.000000 17.000000 to 0.628280 17.000000
\putrule from 0.000000 18.000000 to 0.628280 18.000000
\put {\mutation} at 0.482753 18.000000
\putrule from 0.628280 18.000000 to 0.628280 17.000000
\putrule from 0.628280 17.500000 to 1.567791 17.500000
\put {\mutation} at 1.315146 17.500000
\putrule from 1.567791 17.500000 to 1.567791 16.000000
\putrule from 1.567791 16.750000 to 7.154036 16.750000
\putrule from 7.154036 16.750000 to 7.154036 12.250000
\putrule from 7.154036 14.500000 to 7.338073 14.500000
\put {\mutation} at 7.327387 14.500000
\putrule from 0.000000 19.000000 to 1.153078 19.000000
\put {\mutation} at 0.326220 19.000000
\putrule from 0.000000 20.000000 to 1.153078 20.000000
\put {\mutation} at 0.386781 20.000000
\put {\mutation} at 0.002759 20.000000
\put {\mutation} at 0.360504 20.000000
\put {\mutation} at 0.339855 20.000000
\put {\mutation} at 0.909012 20.000000
\put {\mutation} at 0.589890 20.000000
\putrule from 1.153078 20.000000 to 1.153078 19.000000
\putrule from 1.153078 19.500000 to 7.070724 19.500000
\put {\mutation} at 4.651176 19.500000
\put {\mutation} at 6.627040 19.500000
\put {\mutation} at 1.334290 19.500000
\putrule from 0.000000 21.000000 to 0.057312 21.000000
\putrule from 0.000000 22.000000 to 0.057312 22.000000
\putrule from 0.057312 22.000000 to 0.057312 21.000000
\putrule from 0.057312 21.500000 to 5.097942 21.500000
\put {\mutation} at 0.993268 21.500000
\put {\mutation} at 4.325611 21.500000
\put {\mutation} at 2.202307 21.500000
\put {\mutation} at 0.389805 21.500000
\put {\mutation} at 1.326711 21.500000
\put {\mutation} at 2.656885 21.500000
\put {\mutation} at 2.687060 21.500000
\put {\mutation} at 3.788157 21.500000
\put {\mutation} at 3.124467 21.500000
\putrule from 0.000000 23.000000 to 2.999591 23.000000
\put {\mutation} at 1.637203 23.000000
\putrule from 0.000000 24.000000 to 2.999591 24.000000
\put {\mutation} at 0.145192 24.000000
\put {\mutation} at 1.442032 24.000000
\putrule from 2.999591 24.000000 to 2.999591 23.000000
\putrule from 2.999591 23.500000 to 3.423567 23.500000
\putrule from 0.000000 25.000000 to 1.382004 25.000000
\put {\mutation} at 0.293163 25.000000
\putrule from 0.000000 26.000000 to 1.382004 26.000000
\put {\mutation} at 0.723168 26.000000
\putrule from 1.382004 26.000000 to 1.382004 25.000000
\putrule from 1.382004 25.500000 to 3.423567 25.500000
\putrule from 3.423567 25.500000 to 3.423567 23.500000
\putrule from 3.423567 24.500000 to 5.097942 24.500000
\putrule from 5.097942 24.500000 to 5.097942 21.500000
\putrule from 5.097942 23.000000 to 7.027488 23.000000
\put {\mutation} at 5.332838 23.000000
\put {\mutation} at 5.887644 23.000000
\putrule from 0.000000 27.000000 to 7.027488 27.000000
\put {\mutation} at 3.786136 27.000000
\put {\mutation} at 2.942184 27.000000
\put {\mutation} at 1.580147 27.000000
\put {\mutation} at 4.647880 27.000000
\put {\mutation} at 3.306496 27.000000
\putrule from 7.027488 27.000000 to 7.027488 23.000000
\putrule from 7.027488 25.000000 to 7.070724 25.000000
\putrule from 7.070724 25.000000 to 7.070724 19.500000
\putrule from 7.070724 22.250000 to 7.309804 22.250000
\putrule from 0.000000 28.000000 to 1.193126 28.000000
\put {\mutation} at 1.172986 28.000000
\putrule from 0.000000 29.000000 to 1.193126 29.000000
\put {\mutation} at 0.982344 29.000000
\putrule from 1.193126 29.000000 to 1.193126 28.000000
\putrule from 1.193126 28.500000 to 7.037893 28.500000
\put {\mutation} at 3.994942 28.500000
\putrule from 0.000000 30.000000 to 2.179040 30.000000
\putrule from 0.000000 31.000000 to 0.396843 31.000000
\putrule from 0.000000 32.000000 to 0.396843 32.000000
\putrule from 0.396843 32.000000 to 0.396843 31.000000
\putrule from 0.396843 31.500000 to 2.179041 31.500000
\putrule from 2.179041 31.500000 to 2.179041 30.000000
\putrule from 2.179041 30.750000 to 3.315289 30.750000
\put {\mutation} at 2.994316 30.750000
\putrule from 0.000000 33.000000 to 3.315288 33.000000
\put {\mutation} at 0.772321 33.000000
\putrule from 3.315288 33.000000 to 3.315288 30.750000
\putrule from 3.315288 31.875000 to 3.393521 31.875000
\putrule from 0.000000 34.000000 to 0.665897 34.000000
\putrule from 0.000000 35.000000 to 0.665897 35.000000
\putrule from 0.665897 35.000000 to 0.665897 34.000000
\putrule from 0.665897 34.500000 to 2.361441 34.500000
\put {\mutation} at 0.859023 34.500000
\putrule from 0.000000 36.000000 to 0.135048 36.000000
\putrule from 0.000000 37.000000 to 0.135048 37.000000
\put {\mutation} at 0.031401 37.000000
\putrule from 0.135048 37.000000 to 0.135048 36.000000
\putrule from 0.135048 36.500000 to 1.303917 36.500000
\putrule from 0.000000 38.000000 to 0.875026 38.000000
\putrule from 0.000000 39.000000 to 0.875026 39.000000
\putrule from 0.875026 39.000000 to 0.875026 38.000000
\putrule from 0.875026 38.500000 to 1.303917 38.500000
\putrule from 1.303917 38.500000 to 1.303917 36.500000
\putrule from 1.303917 37.500000 to 2.361441 37.500000
\putrule from 2.361441 37.500000 to 2.361441 34.500000
\putrule from 2.361441 36.000000 to 3.393521 36.000000
\putrule from 3.393521 36.000000 to 3.393521 31.875000
\putrule from 3.393521 33.937500 to 7.037893 33.937500
\put {\mutation} at 5.088339 33.937500
\put {\mutation} at 4.904209 33.937500
\putrule from 7.037893 33.937500 to 7.037893 28.500000
\putrule from 7.037893 31.218750 to 7.039517 31.218750
\putrule from 0.000000 40.000000 to 3.453931 40.000000
\putrule from 0.000000 41.000000 to 3.453931 41.000000
\put {\mutation} at 2.301917 41.000000
\putrule from 3.453931 41.000000 to 3.453931 40.000000
\putrule from 3.453931 40.500000 to 7.002421 40.500000
\put {\mutation} at 6.435420 40.500000
\putrule from 0.000000 42.000000 to 2.783494 42.000000
\put {\mutation} at 1.295345 42.000000
\put {\mutation} at 2.278445 42.000000
\put {\mutation} at 0.571229 42.000000
\putrule from 0.000000 43.000000 to 2.783494 43.000000
\put {\mutation} at 0.600403 43.000000
\put {\mutation} at 2.601781 43.000000
\put {\mutation} at 1.600289 43.000000
\putrule from 2.783494 43.000000 to 2.783494 42.000000
\putrule from 2.783494 42.500000 to 6.639799 42.500000
\put {\mutation} at 3.518412 42.500000
\put {\mutation} at 3.170515 42.500000
\put {\mutation} at 3.563039 42.500000
\put {\mutation} at 3.468022 42.500000
\putrule from 0.000000 44.000000 to 5.296117 44.000000
\put {\mutation} at 2.695622 44.000000
\put {\mutation} at 1.220775 44.000000
\putrule from 0.000000 45.000000 to 0.551007 45.000000
\putrule from 0.000000 46.000000 to 0.551007 46.000000
\putrule from 0.551007 46.000000 to 0.551007 45.000000
\putrule from 0.551007 45.500000 to 3.608453 45.500000
\put {\mutation} at 3.289145 45.500000
\putrule from 0.000000 47.000000 to 0.356757 47.000000
\put {\mutation} at 0.172152 47.000000
\putrule from 0.000000 48.000000 to 0.356757 48.000000
\putrule from 0.356757 48.000000 to 0.356757 47.000000
\putrule from 0.356757 47.500000 to 3.608452 47.500000
\put {\mutation} at 3.325210 47.500000
\putrule from 3.608452 47.500000 to 3.608452 45.500000
\putrule from 3.608452 46.500000 to 5.296117 46.500000
\put {\mutation} at 4.097186 46.500000
\putrule from 5.296117 46.500000 to 5.296117 44.000000
\putrule from 5.296117 45.250000 to 6.639799 45.250000
\put {\mutation} at 5.562236 45.250000
\put {\mutation} at 6.121646 45.250000
\putrule from 6.639799 45.250000 to 6.639799 42.500000
\putrule from 6.639799 43.875000 to 7.002421 43.875000
\putrule from 7.002421 43.875000 to 7.002421 40.500000
\putrule from 7.002421 42.187500 to 7.031114 42.187500
\putrule from 0.000000 49.000000 to 7.031114 49.000000
\putrule from 7.031114 49.000000 to 7.031114 42.187500
\putrule from 7.031114 45.593750 to 7.039518 45.593750
\putrule from 7.039518 45.593750 to 7.039518 31.218750
\putrule from 7.039518 38.406250 to 7.309804 38.406250
\put {\mutation} at 7.265698 38.406250
\putrule from 7.309804 38.406250 to 7.309804 22.250000
\putrule from 7.309804 30.328125 to 7.338074 30.328125
\putrule from 7.338074 30.328125 to 7.338074 14.500000
\putrule from 7.338074 22.414062 to 7.953188 22.414062
\putrule from 7.953188 22.414062 to 7.953188 3.109375
\putrule from 7.953188 12.761719 to 20.000000 12.761719
%
%%%%%%%%%%%%%% Mismatch Distribution %%%%%%%%%%%%%%
%
\yunit=\plotht
\xunit=\plotwd
\Divide <\xunit> by <20.000000pt> forming <\xunit>
\Divide <\yunit> by <0.149388pt> forming <\yunit>
\advance\thusfar by \plotht
\advance\thusfar by \plotsp
\Divide <\thusfar> by <\yunit> forming <\offsety>
\placevalueinpts of <\offsety> in {\mktree_tmp}
\setcoordinatesystem units <\xunit, \yunit> point at 0 {\mktree_tmp}
\setplotarea x from 0.000000 to 20.000000, y from 0.000000 to 0.149388
\axis left label {\lines{Mismatch\cr distribution}} /
\axis bottom
   label {Pairwise differences}
   ticks numbered from 0 to 20 by 10 /
\multiput {$\circ$} at
 0 0.008980 1 0.023673 2 0.029388 3 0.017959 4 0.038367
 5 0.055510 6 0.097143 7 0.149388 8 0.127347 9 0.123265
 10 0.080000 11 0.068571 12 0.054694 13 0.029388 14 0.036735
 15 0.023673 16 0.016327 17 0.011429 18 0.000000 19 0.006531
 20 0.001633
/
%Pr[random pair differs by i sites]
\plot 0 0.010318 1 0.012924 2 0.021443 3 0.039166 4 0.065612
      5 0.095054 6 0.118885 7 0.130095 8 0.126489 9 0.110838
      10 0.088640 11 0.065416 12 0.044995 13 0.029106 14 0.017854
      15 0.010466 16 0.005906 17 0.003230 18 0.001723 19 0.000901
      20 0.000464 /                                                                           
\plotht=1.100000\plotht
%
%%%%%%%%%%%%%% Simulated site frequency spectrum %%%%%%%%%%%%%%
%
\yunit=\plotht
\xunit=\plotwd
\Divide <\xunit> by <25.000000pt> forming <\xunit>
\Divide <\yunit> by <0.615596pt> forming <\yunit>
\advance\thusfar by \plotht
\advance\thusfar by \plotsp
\Divide <\thusfar> by <\yunit> forming <\offsety>
\placevalueinpts of <\offsety> in {\mktree_tmp}
\setcoordinatesystem units <\xunit, \yunit> point at 0 {\mktree_tmp}
\setplotarea x from 0.000000 to 25.000000, y from 0.000000 to 0.615596
\axis left invisible label {\lines{Site\cr frequency\cr spectrum}} /
\axis bottom
   label {Frequency of minor allele}
   ticks withvalues 0 {1/2} / at 0 25 / /
\linethickness 1.2pt
\axis top /
\linethickness 0.4pt
\sethistograms
\plot 0 0 1 0.559633 2 0.302752 3 0.018349 4 0.018349
 5 0.036697 6 0.018349 7 0.009174 8 0.009174 9 0.000000
 10 0.018349 11 0.000000 12 0.000000 13 0.000000 14 0.000000
 15 0.000000 16 0.000000 17 0.000000 18 0.000000 19 0.000000
 20 0.000000 21 0.000000 22 0.009174 23 0.000000 24 0.000000
 25 0.000000
/
\setlinear
%
%%%%%%%%%%%%%% Neutral expectation of site freq spectrum %%%%%%%%%%%%%%
%
\multiput {$\bullet$} at 
     0.500000 0.22781  1.500000 0.116278  2.500000 0.079168  3.500000 0.0606668  4.500000 0.049612
     5.500000 0.0422829  6.500000 0.0370854  7.500000 0.0332223  8.500000 0.0302512  9.500000 0.0279067
     10.500000 0.0260203  11.500000 0.0244796  12.500000 0.0232073  13.500000 0.0221482  14.500000 0.0212623
     15.500000 0.0205197  16.500000 0.0198978  17.500000 0.0193797  18.500000 0.0189519  19.500000 0.0186045
     20.500000 0.0183295  21.500000 0.0181213  22.500000 0.0179754  23.500000 0.0178889  24.500000 0.0178603
/
\endpicture}
\let\put\latexput

\caption{A coalescent simulation with growth: $\theta_0=1$, $\theta_1=100$,
  $\tau=7$. $\star$: expected; $\bullet$: expected
if population size is constant.}\label{fig.gro1}
\end{figure}

\begin{figure}
\centering
% -*-latex-*-
%
%%%%%%%%%%%%%% Tree in PicTeX Format %%%%%%%%%%%%%%
%
%Sample size   = 50
%Mutation rate = 1
%     theta         mn        tau          K
%  100.0000     0.0000     7.0000          1
%    1.0000     0.0000        Inf          1
%maxmm=16.000000 maxtau=7.700000 maxtree=9.723701 maxx=16.000000
%\newdimen\offsety
%\newdimen\yunit
%\newdimen\xunit
%\newdimen\thusfar
%\newdimen\plotht
%\newdimen\plotwd
%\newdimen\plotsp
\thusfar=0.000000in  % Keeps track of what's above
\plotht=0.250000in   % Height of each plot
\plotwd=0.7\textwidth   % Width of each plot
\plotsp=0.500000in   % Spacing between plots
\begin{center}
\mbox{\beginpicture
\def\mutation{\tiny$\bullet$}
\small
\valuestolabelleading=.4\baselineskip
\headingtoplotskip=0.4\baselineskip
%
%%%%%%%%%%%%%% Population size %%%%%%%%%%%%%%
%
\yunit=\plotht
\xunit=\plotwd
\Divide <\xunit> by <16.000000pt> forming <\xunit>
\Divide <\yunit> by <120.000008pt> forming <\yunit>
\advance\thusfar by \plotht
\advance\thusfar by \plotsp
\Divide <\thusfar> by <\yunit> forming <\offsety>
\placevalueinpts of <\offsety> in {\mktree_tmp}
\setcoordinatesystem units <\xunit, \yunit> point at 0 {\mktree_tmp}
\setplotarea x from 0.000000 to 16.000000, y from 0.000000 to 120.000008
\advance\thusfar by 0.030000\plotht
\axis left label {\lines{Population\cr size}} /
\axis bottom shiftedto y=-3.600000
    label {Mutational time before present}
    ticks numbered from 0 to 16 by 8 /
\putrule from 0.000000 100.000000 to 7.000000 100.000000
\putrule from 7.000000 100.000000 to 7.000000 1.000000
\putrule from 7.000000 1.000000 to 16.000000 1.000000
%
%%%%%%%%%%%%%% Gene Genealogy %%%%%%%%%%%%%%
%
\yunit=\plotht
\xunit=\plotwd
\Divide <\xunit> by <16.000000pt> forming <\xunit>
\Divide <\yunit> by <49.000000pt> forming <\yunit>
\advance\thusfar by \plotht
\advance\thusfar by \plotsp
\Divide <\thusfar> by <\yunit> forming <\offsety>
\placevalueinpts of <\offsety> in {\mktree_tmp}
\setcoordinatesystem units <\xunit, \yunit> point at 0 {\mktree_tmp}
\setplotarea x from 0.000000 to 16.000000, y from 0.000000 to 49.000000
\axis left invisible label {\lines{Gene\cr genealogy}} /
\axis bottom invisible
     label {Mutational time before present} /
\putrule from 0.000000 0.000000 to 0.512433 0.000000
\putrule from 0.000000 1.000000 to 0.512433 1.000000
\putrule from 0.512433 1.000000 to 0.512433 0.000000
\putrule from 0.512433 0.500000 to 8.839728 0.500000
\put {\mutation} at 5.186596 0.500000
\put {\mutation} at 3.618633 0.500000
\put {\mutation} at 2.851720 0.500000
\putrule from 0.000000 2.000000 to 7.004996 2.000000
\put {\mutation} at 1.317947 2.000000
\put {\mutation} at 3.960779 2.000000
\put {\mutation} at 0.709163 2.000000
\put {\mutation} at 4.313396 2.000000
\put {\mutation} at 3.350458 2.000000
\put {\mutation} at 6.785172 2.000000
\putrule from 0.000000 3.000000 to 0.447611 3.000000
\put {\mutation} at 0.017953 3.000000
\putrule from 0.000000 4.000000 to 0.447611 4.000000
\put {\mutation} at 0.255883 4.000000
\put {\mutation} at 0.418558 4.000000
\putrule from 0.447611 4.000000 to 0.447611 3.000000
\putrule from 0.447611 3.500000 to 7.004996 3.500000
\put {\mutation} at 5.425284 3.500000
\putrule from 7.004996 3.500000 to 7.004996 2.000000
\putrule from 7.004996 2.750000 to 7.065513 2.750000
\putrule from 0.000000 5.000000 to 0.995178 5.000000
\put {\mutation} at 0.443013 5.000000
\putrule from 0.000000 6.000000 to 0.073466 6.000000
\putrule from 0.000000 7.000000 to 0.073466 7.000000
\put {\mutation} at 0.012728 7.000000
\putrule from 0.073466 7.000000 to 0.073466 6.000000
\putrule from 0.073466 6.500000 to 0.576184 6.500000
\putrule from 0.000000 8.000000 to 0.576184 8.000000
\putrule from 0.576184 8.000000 to 0.576184 6.500000
\putrule from 0.576184 7.250000 to 0.611012 7.250000
\putrule from 0.000000 9.000000 to 0.611012 9.000000
\putrule from 0.611012 9.000000 to 0.611012 7.250000
\putrule from 0.611012 8.125000 to 0.995178 8.125000
\putrule from 0.995178 8.125000 to 0.995178 5.000000
\putrule from 0.995178 6.562500 to 1.678274 6.562500
\put {\mutation} at 1.182069 6.562500
\putrule from 0.000000 10.000000 to 1.678274 10.000000
\put {\mutation} at 1.085228 10.000000
\put {\mutation} at 1.318794 10.000000
\putrule from 1.678274 10.000000 to 1.678274 6.562500
\putrule from 1.678274 8.281250 to 7.065513 8.281250
\put {\mutation} at 3.659049 8.281250
\put {\mutation} at 1.698335 8.281250
\putrule from 7.065513 8.281250 to 7.065513 2.750000
\putrule from 7.065513 5.515625 to 7.066610 5.515625
\putrule from 0.000000 11.000000 to 7.066610 11.000000
\put {\mutation} at 1.885652 11.000000
\put {\mutation} at 5.139637 11.000000
\put {\mutation} at 4.099978 11.000000
\put {\mutation} at 6.848048 11.000000
\putrule from 7.066610 11.000000 to 7.066610 5.515625
\putrule from 7.066610 8.257812 to 7.087685 8.257812
\putrule from 0.000000 12.000000 to 7.087685 12.000000
\put {\mutation} at 4.619718 12.000000
\put {\mutation} at 2.376045 12.000000
\put {\mutation} at 2.852695 12.000000
\put {\mutation} at 3.808904 12.000000
\put {\mutation} at 6.862144 12.000000
\put {\mutation} at 4.009418 12.000000
\putrule from 7.087685 12.000000 to 7.087685 8.257812
\putrule from 7.087685 10.128906 to 7.466074 10.128906
\putrule from 0.000000 13.000000 to 7.125145 13.000000
\put {\mutation} at 0.174895 13.000000
\put {\mutation} at 1.009070 13.000000
\put {\mutation} at 4.013494 13.000000
\put {\mutation} at 6.852795 13.000000
\putrule from 0.000000 14.000000 to 1.086378 14.000000
\putrule from 0.000000 15.000000 to 1.086378 15.000000
\putrule from 1.086378 15.000000 to 1.086378 14.000000
\putrule from 1.086378 14.500000 to 3.826067 14.500000
\put {\mutation} at 3.715576 14.500000
\put {\mutation} at 2.562854 14.500000
\putrule from 0.000000 16.000000 to 3.826067 16.000000
\put {\mutation} at 1.967073 16.000000
\putrule from 3.826067 16.000000 to 3.826067 14.500000
\putrule from 3.826067 15.250000 to 5.696297 15.250000
\putrule from 0.000000 17.000000 to 5.696297 17.000000
\put {\mutation} at 4.624300 17.000000
\put {\mutation} at 1.879917 17.000000
\put {\mutation} at 3.652698 17.000000
\put {\mutation} at 5.167134 17.000000
\putrule from 5.696297 17.000000 to 5.696297 15.250000
\putrule from 5.696297 16.125000 to 7.125144 16.125000
\put {\mutation} at 5.797542 16.125000
\putrule from 7.125144 16.125000 to 7.125144 13.000000
\putrule from 7.125144 14.562500 to 7.466073 14.562500
\putrule from 7.466073 14.562500 to 7.466073 10.128906
\putrule from 7.466073 12.345703 to 7.510207 12.345703
\putrule from 0.000000 18.000000 to 7.050459 18.000000
\put {\mutation} at 6.435995 18.000000
\put {\mutation} at 6.473135 18.000000
\put {\mutation} at 1.396295 18.000000
\put {\mutation} at 5.775696 18.000000
\putrule from 0.000000 19.000000 to 7.046778 19.000000
\put {\mutation} at 4.997455 19.000000
\put {\mutation} at 5.512586 19.000000
\put {\mutation} at 2.468662 19.000000
\put {\mutation} at 6.682427 19.000000
\put {\mutation} at 1.281582 19.000000
\putrule from 0.000000 20.000000 to 1.129384 20.000000
\putrule from 0.000000 21.000000 to 1.129384 21.000000
\putrule from 1.129384 21.000000 to 1.129384 20.000000
\putrule from 1.129384 20.500000 to 7.015024 20.500000
\put {\mutation} at 5.520108 20.500000
\put {\mutation} at 5.298704 20.500000
\putrule from 0.000000 22.000000 to 1.040648 22.000000
\put {\mutation} at 0.595026 22.000000
\put {\mutation} at 0.757312 22.000000
\putrule from 0.000000 23.000000 to 1.040648 23.000000
\putrule from 1.040648 23.000000 to 1.040648 22.000000
\putrule from 1.040648 22.500000 to 7.015025 22.500000
\putrule from 7.015025 22.500000 to 7.015025 20.500000
\putrule from 7.015025 21.500000 to 7.026169 21.500000
\putrule from 0.000000 24.000000 to 5.095344 24.000000
\put {\mutation} at 2.387719 24.000000
\put {\mutation} at 3.423661 24.000000
\put {\mutation} at 2.199321 24.000000
\put {\mutation} at 0.959070 24.000000
\putrule from 0.000000 25.000000 to 5.095344 25.000000
\put {\mutation} at 1.684014 25.000000
\put {\mutation} at 0.260498 25.000000
\put {\mutation} at 2.975549 25.000000
\put {\mutation} at 4.594189 25.000000
\put {\mutation} at 2.503215 25.000000
\putrule from 5.095344 25.000000 to 5.095344 24.000000
\putrule from 5.095344 24.500000 to 7.026169 24.500000
\put {\mutation} at 6.791253 24.500000
\put {\mutation} at 5.446722 24.500000
\putrule from 7.026169 24.500000 to 7.026169 21.500000
\putrule from 7.026169 23.000000 to 7.036411 23.000000
\putrule from 0.000000 26.000000 to 1.543690 26.000000
\put {\mutation} at 1.085735 26.000000
\putrule from 0.000000 27.000000 to 1.543690 27.000000
\put {\mutation} at 0.116507 27.000000
\putrule from 1.543690 27.000000 to 1.543690 26.000000
\putrule from 1.543690 26.500000 to 7.036411 26.500000
\put {\mutation} at 4.324181 26.500000
\put {\mutation} at 4.497512 26.500000
\putrule from 7.036411 26.500000 to 7.036411 23.000000
\putrule from 7.036411 24.750000 to 7.046778 24.750000
\putrule from 7.046778 24.750000 to 7.046778 19.000000
\putrule from 7.046778 21.875000 to 7.050459 21.875000
\putrule from 7.050459 21.875000 to 7.050459 18.000000
\putrule from 7.050459 19.937500 to 7.064291 19.937500
\putrule from 0.000000 28.000000 to 1.295636 28.000000
\put {\mutation} at 0.717458 28.000000
\putrule from 0.000000 29.000000 to 0.717845 29.000000
\put {\mutation} at 0.462706 29.000000
\put {\mutation} at 0.337065 29.000000
\putrule from 0.000000 30.000000 to 0.717845 30.000000
\put {\mutation} at 0.501310 30.000000
\putrule from 0.717845 30.000000 to 0.717845 29.000000
\putrule from 0.717845 29.500000 to 1.295636 29.500000
\put {\mutation} at 1.227887 29.500000
\putrule from 1.295636 29.500000 to 1.295636 28.000000
\putrule from 1.295636 28.750000 to 7.000079 28.750000
\put {\mutation} at 1.749650 28.750000
\put {\mutation} at 4.645698 28.750000
\put {\mutation} at 6.652993 28.750000
\putrule from 0.000000 31.000000 to 1.319274 31.000000
\put {\mutation} at 0.922745 31.000000
\putrule from 0.000000 32.000000 to 1.319274 32.000000
\put {\mutation} at 0.381200 32.000000
\put {\mutation} at 0.182015 32.000000
\putrule from 1.319274 32.000000 to 1.319274 31.000000
\putrule from 1.319274 31.500000 to 7.000078 31.500000
\put {\mutation} at 4.009562 31.500000
\put {\mutation} at 6.327330 31.500000
\putrule from 7.000078 31.500000 to 7.000078 28.750000
\putrule from 7.000078 30.125000 to 7.064291 30.125000
\putrule from 7.064291 30.125000 to 7.064291 19.937500
\putrule from 7.064291 25.031250 to 7.366339 25.031250
\putrule from 0.000000 33.000000 to 5.468971 33.000000
\put {\mutation} at 0.540789 33.000000
\put {\mutation} at 1.847085 33.000000
\putrule from 0.000000 34.000000 to 5.468971 34.000000
\put {\mutation} at 2.054661 34.000000
\putrule from 5.468971 34.000000 to 5.468971 33.000000
\putrule from 5.468971 33.500000 to 7.366339 33.500000
\putrule from 7.366339 33.500000 to 7.366339 25.031250
\putrule from 7.366339 29.265625 to 7.510207 29.265625
\putrule from 7.510207 29.265625 to 7.510207 12.345703
\putrule from 7.510207 20.805664 to 8.632509 20.805664
\put {\mutation} at 8.102641 20.805664
\putrule from 0.000000 35.000000 to 6.676580 35.000000
\put {\mutation} at 3.953121 35.000000
\putrule from 0.000000 36.000000 to 0.782837 36.000000
\putrule from 0.000000 37.000000 to 0.119050 37.000000
\putrule from 0.000000 38.000000 to 0.119050 38.000000
\putrule from 0.119050 38.000000 to 0.119050 37.000000
\putrule from 0.119050 37.500000 to 0.782837 37.500000
\putrule from 0.782837 37.500000 to 0.782837 36.000000
\putrule from 0.782837 36.750000 to 6.676580 36.750000
\put {\mutation} at 6.215918 36.750000
\put {\mutation} at 6.465530 36.750000
\put {\mutation} at 4.385518 36.750000
\putrule from 6.676580 36.750000 to 6.676580 35.000000
\putrule from 6.676580 35.875000 to 7.023606 35.875000
\putrule from 0.000000 39.000000 to 7.023607 39.000000
\put {\mutation} at 0.700363 39.000000
\put {\mutation} at 4.836319 39.000000
\putrule from 7.023607 39.000000 to 7.023607 35.875000
\putrule from 7.023607 37.437500 to 7.062778 37.437500
\putrule from 0.000000 40.000000 to 7.014595 40.000000
\put {\mutation} at 2.263929 40.000000
\put {\mutation} at 0.620857 40.000000
\put {\mutation} at 4.615184 40.000000
\put {\mutation} at 1.509842 40.000000
\put {\mutation} at 6.535007 40.000000
\putrule from 0.000000 41.000000 to 7.014595 41.000000
\putrule from 7.014595 41.000000 to 7.014595 40.000000
\putrule from 7.014595 40.500000 to 7.062778 40.500000
\putrule from 7.062778 40.500000 to 7.062778 37.437500
\putrule from 7.062778 38.968750 to 7.149688 38.968750
\putrule from 0.000000 42.000000 to 0.541228 42.000000
\put {\mutation} at 0.048336 42.000000
\put {\mutation} at 0.101630 42.000000
\putrule from 0.000000 43.000000 to 0.211446 43.000000
\putrule from 0.000000 44.000000 to 0.211446 44.000000
\putrule from 0.211446 44.000000 to 0.211446 43.000000
\putrule from 0.211446 43.500000 to 0.541228 43.500000
\putrule from 0.541228 43.500000 to 0.541228 42.000000
\putrule from 0.541228 42.750000 to 1.074174 42.750000
\putrule from 0.000000 45.000000 to 1.074174 45.000000
\putrule from 1.074174 45.000000 to 1.074174 42.750000
\putrule from 1.074174 43.875000 to 2.953797 43.875000
\putrule from 0.000000 46.000000 to 2.953797 46.000000
\put {\mutation} at 0.862175 46.000000
\putrule from 2.953797 46.000000 to 2.953797 43.875000
\putrule from 2.953797 44.937500 to 5.797513 44.937500
\putrule from 0.000000 47.000000 to 1.497720 47.000000
\put {\mutation} at 0.643530 47.000000
\putrule from 0.000000 48.000000 to 1.497720 48.000000
\put {\mutation} at 0.310739 48.000000
\putrule from 1.497720 48.000000 to 1.497720 47.000000
\putrule from 1.497720 47.500000 to 3.396708 47.500000
\put {\mutation} at 2.747654 47.500000
\putrule from 0.000000 49.000000 to 3.396708 49.000000
\putrule from 3.396708 49.000000 to 3.396708 47.500000
\putrule from 3.396708 48.250000 to 5.797513 48.250000
\putrule from 5.797513 48.250000 to 5.797513 44.937500
\putrule from 5.797513 46.593750 to 7.149687 46.593750
\put {\mutation} at 5.905941 46.593750
\put {\mutation} at 6.827953 46.593750
\putrule from 7.149687 46.593750 to 7.149687 38.968750
\putrule from 7.149687 42.781250 to 8.632509 42.781250
\put {\mutation} at 7.331868 42.781250
\put {\mutation} at 7.814545 42.781250
\put {\mutation} at 8.546970 42.781250
\putrule from 8.632509 42.781250 to 8.632509 20.805664
\putrule from 8.632509 31.793457 to 8.839727 31.793457
\putrule from 8.839727 31.793457 to 8.839727 0.500000
\putrule from 8.839727 16.146729 to 16.000000 16.146729
%
%%%%%%%%%%%%%% Mismatch Distribution %%%%%%%%%%%%%%
%
\yunit=\plotht
\xunit=\plotwd
\Divide <\xunit> by <16.000000pt> forming <\xunit>
\Divide <\yunit> by <0.136327pt> forming <\yunit>
\advance\thusfar by \plotht
\advance\thusfar by \plotsp
\Divide <\thusfar> by <\yunit> forming <\offsety>
\placevalueinpts of <\offsety> in {\mktree_tmp}
\setcoordinatesystem units <\xunit, \yunit> point at 0 {\mktree_tmp}
\setplotarea x from 0.000000 to 16.000000, y from 0.000000 to 0.136327
\axis left label {\lines{Mismatch\cr distribution}} /
\axis bottom
   label {Pairwise differences}
   ticks numbered from 0 to 16 by 8 /
\multiput {$\circ$} at
 0 0.012245 1 0.009796 2 0.021224 3 0.029388 4 0.040816
 5 0.071020 6 0.098776 7 0.115918 8 0.129796 9 0.136327
 10 0.129796 11 0.088163 12 0.061224 13 0.033469 14 0.015510
 15 0.005714 16 0.000816
/
%Pr[random pair differs by i sites]
\plot 0 0.010318 1 0.012924 2 0.021443 3 0.039166 4 0.065612
      5 0.095054 6 0.118885 7 0.130095 8 0.126489 9 0.110838
      10 0.088640 11 0.065416 12 0.044995 13 0.029106 14 0.017854
      15 0.010466 16 0.005906 /
\plotht=1.100000\plotht
%
%%%%%%%%%%%%%% Simulated site frequency spectrum %%%%%%%%%%%%%%
%
\yunit=\plotht
\xunit=\plotwd
\Divide <\xunit> by <25.000000pt> forming <\xunit>
\Divide <\yunit> by <0.777064pt> forming <\yunit>
\advance\thusfar by \plotht
\advance\thusfar by \plotsp
\Divide <\thusfar> by <\yunit> forming <\offsety>
\placevalueinpts of <\offsety> in {\mktree_tmp}
\setcoordinatesystem units <\xunit, \yunit> point at 0 {\mktree_tmp}
\setplotarea x from 0.000000 to 25.000000, y from 0.000000 to 0.777064
\axis left invisible label {\lines{Site\cr frequency\cr spectrum}} /
\axis bottom
   label {Frequency of minor allele}
   ticks withvalues 0 {1/2} / at 0 25 / /
\linethickness 1.2pt
\axis top /
\linethickness 0.4pt
\sethistograms
\plot 0 0 1 0.706422 2 0.146789 3 0.055046 4 0.009174
 5 0.009174 6 0.018349 7 0.000000 8 0.018349 9 0.000000
 10 0.000000 11 0.000000 12 0.000000 13 0.000000 14 0.000000
 15 0.027523 16 0.000000 17 0.009174 18 0.000000 19 0.000000
 20 0.000000 21 0.000000 22 0.000000 23 0.000000 24 0.000000
 25 0.000000
/
\setlinear
%
%%%%%%%%%%%%%% Neutral expectation of site freq spectrum %%%%%%%%%%%%%%
%
\multiput {$\bullet$} at 
     0.500000 0.22781  1.500000 0.116278  2.500000 0.079168  3.500000 0.0606668  4.500000 0.049612
     5.500000 0.0422829  6.500000 0.0370854  7.500000 0.0332223  8.500000 0.0302512  9.500000 0.0279067
     10.500000 0.0260203  11.500000 0.0244796  12.500000 0.0232073  13.500000 0.0221482  14.500000 0.0212623
     15.500000 0.0205197  16.500000 0.0198978  17.500000 0.0193797  18.500000 0.0189519  19.500000 0.0186045
     20.500000 0.0183295  21.500000 0.0181213  22.500000 0.0179754  23.500000 0.0178889  24.500000 0.0178603
/
\endpicture}\\
A coalescent simulation with growth: $\theta_0=1$, $\theta_1=100$,
  $\tau=7$
\end{center}

%%% Local Variables: 
%%% mode: latex
%%% TeX-master: t
%%% End: 

\caption{A coalescent simulation with growth: $\theta_0=1$, $\theta_1=100$,
  $\tau=7$. $\star$: expected; $\bullet$: expected
if population size is constant.}\label{fig.gro2}
\end{figure}

\begin{figure}
\centering
% -*-latex-*-
%
%%%%%%%%%%%%%% Tree in PicTeX Format %%%%%%%%%%%%%%
%
%Sample size   = 50
%Mutation rate = 1
%     theta         mn        tau          K
%  100.0000     0.0000     7.0000          1
%    1.0000     0.0000        Inf          1
%maxmm=19.000000 maxtau=7.700000 maxtree=12.026645 maxx=19.000000
%\newdimen\offsety
%\newdimen\yunit
%\newdimen\xunit
%\newdimen\thusfar
%\newdimen\plotht
%\newdimen\plotwd
%\newdimen\plotsp
\thusfar=0.000000in  % Keeps track of what's above
\plotht=1.400000in   % Height of each plot
\plotwd=4.000000in   % Width of each plot
\plotsp=0.500000in   % Spacing between plots
\mbox{\beginpicture
\def\mutation{\tiny$\bullet$}
\small
\valuestolabelleading=.4\baselineskip
\headingtoplotskip=0.4\baselineskip
%
%%%%%%%%%%%%%% Population size %%%%%%%%%%%%%%
%
\yunit=\plotht
\xunit=\plotwd
\Divide <\xunit> by <19.000000pt> forming <\xunit>
\Divide <\yunit> by <120.000008pt> forming <\yunit>
\advance\thusfar by \plotht
\advance\thusfar by \plotsp
\Divide <\thusfar> by <\yunit> forming <\offsety>
\placevalueinpts of <\offsety> in {\mktree_tmp}
\setcoordinatesystem units <\xunit, \yunit> point at 0 {\mktree_tmp}
\setplotarea x from 0.000000 to 19.000000, y from 0.000000 to 120.000008
\advance\thusfar by 0.030000\plotht
\axis left label {\lines{Population\cr size}} /
\axis bottom shiftedto y=-3.600000
    label {Mutational time before present}
    ticks numbered from 0 to 18 by 9 /
\putrule from 0.000000 100.000000 to 7.000000 100.000000
\putrule from 7.000000 100.000000 to 7.000000 1.000000
\putrule from 7.000000 1.000000 to 19.000000 1.000000
%
%%%%%%%%%%%%%% Gene Genealogy %%%%%%%%%%%%%%
%
\yunit=\plotht
\xunit=\plotwd
\Divide <\xunit> by <19.000000pt> forming <\xunit>
\Divide <\yunit> by <49.000000pt> forming <\yunit>
\advance\thusfar by \plotht
\advance\thusfar by \plotsp
\Divide <\thusfar> by <\yunit> forming <\offsety>
\placevalueinpts of <\offsety> in {\mktree_tmp}
\setcoordinatesystem units <\xunit, \yunit> point at 0 {\mktree_tmp}
\setplotarea x from 0.000000 to 19.000000, y from 0.000000 to 49.000000
\axis left invisible label {\lines{Gene\cr genealogy}} /
\axis bottom invisible
     label {Mutational time before present} /
\putrule from 0.000000 0.000000 to 3.112880 0.000000
\putrule from 0.000000 1.000000 to 3.112880 1.000000
\put {\mutation} at 1.487036 1.000000
\put {\mutation} at 2.474092 1.000000
\put {\mutation} at 1.169243 1.000000
\putrule from 3.112880 1.000000 to 3.112880 0.000000
\putrule from 3.112880 0.500000 to 7.150349 0.500000
\put {\mutation} at 4.104473 0.500000
\putrule from 0.000000 2.000000 to 3.758807 2.000000
\putrule from 0.000000 3.000000 to 0.985063 3.000000
\putrule from 0.000000 4.000000 to 0.147765 4.000000
\putrule from 0.000000 5.000000 to 0.147765 5.000000
\putrule from 0.147765 5.000000 to 0.147765 4.000000
\putrule from 0.147765 4.500000 to 0.985063 4.500000
\put {\mutation} at 0.715304 4.500000
\putrule from 0.985063 4.500000 to 0.985063 3.000000
\putrule from 0.985063 3.750000 to 1.832072 3.750000
\putrule from 0.000000 6.000000 to 1.832072 6.000000
\putrule from 1.832072 6.000000 to 1.832072 3.750000
\putrule from 1.832072 4.875000 to 3.758807 4.875000
\putrule from 3.758807 4.875000 to 3.758807 2.000000
\putrule from 3.758807 3.437500 to 7.150349 3.437500
\putrule from 7.150349 3.437500 to 7.150349 0.500000
\putrule from 7.150349 1.968750 to 7.205728 1.968750
\putrule from 0.000000 7.000000 to 2.695403 7.000000
\putrule from 0.000000 8.000000 to 2.571097 8.000000
\putrule from 0.000000 9.000000 to 0.748095 9.000000
\put {\mutation} at 0.139237 9.000000
\putrule from 0.000000 10.000000 to 0.748095 10.000000
\putrule from 0.748095 10.000000 to 0.748095 9.000000
\putrule from 0.748095 9.500000 to 2.571097 9.500000
\put {\mutation} at 0.758373 9.500000
\putrule from 2.571097 9.500000 to 2.571097 8.000000
\putrule from 2.571097 8.750000 to 2.695403 8.750000
\putrule from 2.695403 8.750000 to 2.695403 7.000000
\putrule from 2.695403 7.875000 to 7.019698 7.875000
\put {\mutation} at 4.976461 7.875000
\put {\mutation} at 5.028620 7.875000
\putrule from 0.000000 11.000000 to 7.019698 11.000000
\put {\mutation} at 5.796228 11.000000
\put {\mutation} at 1.744785 11.000000
\put {\mutation} at 4.344889 11.000000
\put {\mutation} at 4.489740 11.000000
\putrule from 7.019698 11.000000 to 7.019698 7.875000
\putrule from 7.019698 9.437500 to 7.020987 9.437500
\putrule from 0.000000 12.000000 to 1.357071 12.000000
\putrule from 0.000000 13.000000 to 1.357071 13.000000
\put {\mutation} at 0.352714 13.000000
\putrule from 1.357071 13.000000 to 1.357071 12.000000
\putrule from 1.357071 12.500000 to 7.020988 12.500000
\put {\mutation} at 6.681878 12.500000
\put {\mutation} at 2.365661 12.500000
\put {\mutation} at 1.808513 12.500000
\putrule from 7.020988 12.500000 to 7.020988 9.437500
\putrule from 7.020988 10.968750 to 7.073905 10.968750
\putrule from 0.000000 14.000000 to 5.531047 14.000000
\putrule from 0.000000 15.000000 to 5.531047 15.000000
\put {\mutation} at 3.371211 15.000000
\put {\mutation} at 4.217485 15.000000
\put {\mutation} at 0.660114 15.000000
\putrule from 5.531047 15.000000 to 5.531047 14.000000
\putrule from 5.531047 14.500000 to 6.955988 14.500000
\putrule from 0.000000 16.000000 to 3.026353 16.000000
\put {\mutation} at 0.628327 16.000000
\putrule from 0.000000 17.000000 to 0.838351 17.000000
\putrule from 0.000000 18.000000 to 0.596925 18.000000
\putrule from 0.000000 19.000000 to 0.596925 19.000000
\putrule from 0.596925 19.000000 to 0.596925 18.000000
\putrule from 0.596925 18.500000 to 0.838351 18.500000
\putrule from 0.838351 18.500000 to 0.838351 17.000000
\putrule from 0.838351 17.750000 to 3.026353 17.750000
\put {\mutation} at 1.263795 17.750000
\putrule from 3.026353 17.750000 to 3.026353 16.000000
\putrule from 3.026353 16.875000 to 6.955989 16.875000
\put {\mutation} at 5.498782 16.875000
\putrule from 6.955989 16.875000 to 6.955989 14.500000
\putrule from 6.955989 15.687500 to 7.051027 15.687500
\putrule from 0.000000 20.000000 to 7.051026 20.000000
\put {\mutation} at 5.301806 20.000000
\put {\mutation} at 2.631285 20.000000
\putrule from 7.051026 20.000000 to 7.051026 15.687500
\putrule from 7.051026 17.843750 to 7.073905 17.843750
\putrule from 7.073905 17.843750 to 7.073905 10.968750
\putrule from 7.073905 14.406250 to 7.205729 14.406250
\putrule from 7.205729 14.406250 to 7.205729 1.968750
\putrule from 7.205729 8.187500 to 8.217843 8.187500
\put {\mutation} at 7.968054 8.187500
\putrule from 0.000000 21.000000 to 7.019394 21.000000
\put {\mutation} at 3.571173 21.000000
\put {\mutation} at 6.690228 21.000000
\put {\mutation} at 4.611714 21.000000
\put {\mutation} at 3.598184 21.000000
\put {\mutation} at 1.467985 21.000000
\put {\mutation} at 4.141644 21.000000
\put {\mutation} at 5.895297 21.000000
\putrule from 0.000000 22.000000 to 1.493711 22.000000
\put {\mutation} at 0.737645 22.000000
\putrule from 0.000000 23.000000 to 1.493711 23.000000
\putrule from 1.493711 23.000000 to 1.493711 22.000000
\putrule from 1.493711 22.500000 to 7.019395 22.500000
\put {\mutation} at 4.943167 22.500000
\put {\mutation} at 4.460835 22.500000
\put {\mutation} at 6.848886 22.500000
\put {\mutation} at 2.951485 22.500000
\putrule from 7.019395 22.500000 to 7.019395 21.000000
\putrule from 7.019395 21.750000 to 7.184114 21.750000
\putrule from 0.000000 24.000000 to 7.026964 24.000000
\put {\mutation} at 0.371731 24.000000
\put {\mutation} at 5.716016 24.000000
\put {\mutation} at 3.370884 24.000000
\putrule from 0.000000 25.000000 to 7.026964 25.000000
\put {\mutation} at 0.878547 25.000000
\put {\mutation} at 4.684217 25.000000
\putrule from 7.026964 25.000000 to 7.026964 24.000000
\putrule from 7.026964 24.500000 to 7.042114 24.500000
\putrule from 0.000000 26.000000 to 1.991573 26.000000
\put {\mutation} at 0.177875 26.000000
\putrule from 0.000000 27.000000 to 0.738697 27.000000
\putrule from 0.000000 28.000000 to 0.738697 28.000000
\putrule from 0.738697 28.000000 to 0.738697 27.000000
\putrule from 0.738697 27.500000 to 1.991574 27.500000
\put {\mutation} at 1.026051 27.500000
\putrule from 1.991574 27.500000 to 1.991574 26.000000
\putrule from 1.991574 26.750000 to 3.696810 26.750000
\putrule from 0.000000 29.000000 to 0.898137 29.000000
\put {\mutation} at 0.649543 29.000000
\putrule from 0.000000 30.000000 to 0.395461 30.000000
\putrule from 0.000000 31.000000 to 0.395461 31.000000
\putrule from 0.395461 31.000000 to 0.395461 30.000000
\putrule from 0.395461 30.500000 to 0.898137 30.500000
\putrule from 0.898137 30.500000 to 0.898137 29.000000
\putrule from 0.898137 29.750000 to 3.696811 29.750000
\putrule from 3.696811 29.750000 to 3.696811 26.750000
\putrule from 3.696811 28.250000 to 7.042115 28.250000
\put {\mutation} at 5.000599 28.250000
\put {\mutation} at 3.699873 28.250000
\putrule from 7.042115 28.250000 to 7.042115 24.500000
\putrule from 7.042115 26.375000 to 7.184115 26.375000
\putrule from 7.184115 26.375000 to 7.184115 21.750000
\putrule from 7.184115 24.062500 to 7.379159 24.062500
\putrule from 0.000000 32.000000 to 1.350910 32.000000
\put {\mutation} at 0.561083 32.000000
\put {\mutation} at 1.325726 32.000000
\putrule from 0.000000 33.000000 to 0.148479 33.000000
\putrule from 0.000000 34.000000 to 0.148479 34.000000
\put {\mutation} at 0.088747 34.000000
\putrule from 0.148479 34.000000 to 0.148479 33.000000
\putrule from 0.148479 33.500000 to 1.350910 33.500000
\putrule from 1.350910 33.500000 to 1.350910 32.000000
\putrule from 1.350910 32.750000 to 7.111775 32.750000
\putrule from 0.000000 35.000000 to 7.111775 35.000000
\put {\mutation} at 3.682785 35.000000
\put {\mutation} at 1.960070 35.000000
\put {\mutation} at 5.330941 35.000000
\putrule from 7.111775 35.000000 to 7.111775 32.750000
\putrule from 7.111775 33.875000 to 7.262045 33.875000
\putrule from 0.000000 36.000000 to 7.262045 36.000000
\put {\mutation} at 6.786242 36.000000
\put {\mutation} at 4.091760 36.000000
\put {\mutation} at 6.245974 36.000000
\put {\mutation} at 1.632644 36.000000
\put {\mutation} at 6.190063 36.000000
\put {\mutation} at 4.831530 36.000000
\put {\mutation} at 6.646873 36.000000
\put {\mutation} at 1.560789 36.000000
\putrule from 7.262045 36.000000 to 7.262045 33.875000
\putrule from 7.262045 34.937500 to 7.379158 34.937500
\putrule from 7.379158 34.937500 to 7.379158 24.062500
\putrule from 7.379158 29.500000 to 8.217843 29.500000
\put {\mutation} at 8.050341 29.500000
\putrule from 8.217843 29.500000 to 8.217843 8.187500
\putrule from 8.217843 18.843750 to 10.933313 18.843750
\put {\mutation} at 9.020187 18.843750
\putrule from 0.000000 37.000000 to 1.642399 37.000000
\putrule from 0.000000 38.000000 to 1.642399 38.000000
\put {\mutation} at 0.903247 38.000000
\putrule from 1.642399 38.000000 to 1.642399 37.000000
\putrule from 1.642399 37.500000 to 3.390726 37.500000
\put {\mutation} at 1.940723 37.500000
\putrule from 0.000000 39.000000 to 3.390725 39.000000
\put {\mutation} at 3.028777 39.000000
\putrule from 3.390725 39.000000 to 3.390725 37.500000
\putrule from 3.390725 38.250000 to 7.192924 38.250000
\put {\mutation} at 4.837399 38.250000
\putrule from 0.000000 40.000000 to 1.191824 40.000000
\put {\mutation} at 0.311850 40.000000
\putrule from 0.000000 41.000000 to 1.191824 41.000000
\put {\mutation} at 0.263598 41.000000
\putrule from 1.191824 41.000000 to 1.191824 40.000000
\putrule from 1.191824 40.500000 to 7.033977 40.500000
\put {\mutation} at 4.292322 40.500000
\put {\mutation} at 3.816191 40.500000
\put {\mutation} at 1.467453 40.500000
\putrule from 0.000000 42.000000 to 7.033976 42.000000
\put {\mutation} at 6.463368 42.000000
\putrule from 7.033976 42.000000 to 7.033976 40.500000
\putrule from 7.033976 41.250000 to 7.080195 41.250000
\put {\mutation} at 7.070042 41.250000
\putrule from 0.000000 43.000000 to 5.645717 43.000000
\put {\mutation} at 3.131067 43.000000
\put {\mutation} at 5.110791 43.000000
\put {\mutation} at 0.076184 43.000000
\put {\mutation} at 5.067952 43.000000
\putrule from 0.000000 44.000000 to 2.208411 44.000000
\putrule from 0.000000 45.000000 to 0.735596 45.000000
\putrule from 0.000000 46.000000 to 0.735596 46.000000
\putrule from 0.735596 46.000000 to 0.735596 45.000000
\putrule from 0.735596 45.500000 to 2.208411 45.500000
\put {\mutation} at 0.850178 45.500000
\putrule from 2.208411 45.500000 to 2.208411 44.000000
\putrule from 2.208411 44.750000 to 3.299272 44.750000
\putrule from 0.000000 47.000000 to 0.847355 47.000000
\put {\mutation} at 0.216045 47.000000
\putrule from 0.000000 48.000000 to 0.847355 48.000000
\putrule from 0.847355 48.000000 to 0.847355 47.000000
\putrule from 0.847355 47.500000 to 3.299272 47.500000
\putrule from 3.299272 47.500000 to 3.299272 44.750000
\putrule from 3.299272 46.125000 to 5.645718 46.125000
\put {\mutation} at 4.742387 46.125000
\putrule from 5.645718 46.125000 to 5.645718 43.000000
\putrule from 5.645718 44.562500 to 7.075065 44.562500
\putrule from 0.000000 49.000000 to 7.075065 49.000000
\put {\mutation} at 4.129642 49.000000
\put {\mutation} at 6.214550 49.000000
\put {\mutation} at 4.540078 49.000000
\put {\mutation} at 5.673804 49.000000
\put {\mutation} at 6.067805 49.000000
\put {\mutation} at 4.632312 49.000000
\putrule from 7.075065 49.000000 to 7.075065 44.562500
\putrule from 7.075065 46.781250 to 7.080195 46.781250
\putrule from 7.080195 46.781250 to 7.080195 41.250000
\putrule from 7.080195 44.015625 to 7.192924 44.015625
\putrule from 7.192924 44.015625 to 7.192924 38.250000
\putrule from 7.192924 41.132812 to 10.933313 41.132812
\put {\mutation} at 9.941317 41.132812
\put {\mutation} at 7.288903 41.132812
\put {\mutation} at 7.916128 41.132812
\putrule from 10.933313 41.132812 to 10.933313 18.843750
\putrule from 10.933313 29.988281 to 19.000000 29.988281
%
%%%%%%%%%%%%%% Mismatch Distribution %%%%%%%%%%%%%%
%
\yunit=\plotht
\xunit=\plotwd
\Divide <\xunit> by <19.000000pt> forming <\xunit>
\Divide <\yunit> by <0.113469pt> forming <\yunit>
\advance\thusfar by \plotht
\advance\thusfar by \plotsp
\Divide <\thusfar> by <\yunit> forming <\offsety>
\placevalueinpts of <\offsety> in {\mktree_tmp}
\setcoordinatesystem units <\xunit, \yunit> point at 0 {\mktree_tmp}
\setplotarea x from 0.000000 to 19.000000, y from 0.000000 to 0.113469
\axis left label {\lines{Mismatch\cr distribution}} /
\axis bottom
   label {Pairwise differences}
   ticks numbered from 0 to 18 by 9 /
\multiput {$\circ$} at
 0 0.012245 1 0.026939 2 0.042449 3 0.047347 4 0.073469
 5 0.083265 6 0.097959 7 0.111837 8 0.095510 9 0.113469
 10 0.093061 11 0.071837 12 0.047347 13 0.036735 14 0.026122
 15 0.012245 16 0.002449 17 0.002449 18 0.002449 19 0.000816
/
%Pr[random pair differs by i sites]
\plot 0 0.010318 1 0.012924 2 0.021443 3 0.039166 4 0.065612
      5 0.095054 6 0.118885 7 0.130095 8 0.126489 9 0.110838
      10 0.088640 11 0.065416 12 0.044995 13 0.029106 14 0.017854
      15 0.010466 16 0.005906 17 0.003230 18 0.001723 19 0.000901
      /
\plotht=1.100000\plotht
%
%%%%%%%%%%%%%% Simulated site frequency spectrum %%%%%%%%%%%%%%
%
\yunit=\plotht
\xunit=\plotwd
\Divide <\xunit> by <25.000000pt> forming <\xunit>
%\Divide <\yunit> by <0.725275pt> forming <\yunit>
\Divide <\yunit> by <0.693pt> forming <\yunit>
\advance\thusfar by \plotht
\advance\thusfar by \plotsp
\Divide <\thusfar> by <\yunit> forming <\offsety>
\placevalueinpts of <\offsety> in {\mktree_tmp}
\setcoordinatesystem units <\xunit, \yunit> point at 0 {\mktree_tmp}
%\setplotarea x from 0.000000 to 25.000000, y from 0.000000 to 0.725275
\setplotarea x from 0.000000 to 25.000000, y from 0.000000 to 0.693
\axis left invisible label {\lines{Site\cr frequency\cr spectrum}} /
\axis bottom
   label {Frequency of minor allele}
   ticks withvalues 0 {1/2} / at 0 25 / /
\linethickness 1.2pt
\axis top /
\linethickness 0.4pt
\sethistograms
\plot 0 0 1 0.659341 2 0.175824 3 0.032967 4 0.032967
 5 0.010989 6 0.021978 7 0.000000 8 0.000000 9 0.000000
 10 0.000000 11 0.000000 12 0.000000 13 0.043956 14 0.000000
 15 0.000000 16 0.010989 17 0.000000 18 0.000000 19 0.000000
 20 0.000000 21 0.010989 22 0.000000 23 0.000000 24 0.000000
 25 0.000000
/
\setlinear
%
%%%%%% stationary Neutral expectation of folded spectrum %%%%%%%%%%%%%%
%
\multiput {$\bullet$} at 
     0.500000 0.22781  1.500000 0.116278  2.500000 0.079168  3.500000 0.0606668  4.500000 0.049612
     5.500000 0.0422829  6.500000 0.0370854  7.500000 0.0332223  8.500000 0.0302512  9.500000 0.0279067
     10.500000 0.0260203  11.500000 0.0244796  12.500000 0.0232073  13.500000 0.0221482  14.500000 0.0212623
     15.500000 0.0205197  16.500000 0.0198978  17.500000 0.0193797  18.500000 0.0189519  19.500000 0.0186045
     20.500000 0.0183295  21.500000 0.0181213  22.500000 0.0179754  23.500000 0.0178889  24.500000 0.0178603
/
%%%%%%%% NONstationary neutral expectation of folded spectrum %%%%%%%%%%%%%
\multiput {$\star$} at
   0.5    0.62975
   1.5    0.20128
   2.5   0.085283
   3.5   0.040411
   4.5   0.020302
   5.5   0.010559
   6.5  0.0056144
   7.5  0.0030288
   8.5  0.0016502
   9.5 0.00090581
  10.5 0.00050037
  11.5 0.00027842
  12.5 0.00015658
  13.5 0.00008963
  14.5 0.000052836
  15.5 0.000032635
  16.5 0.000021557
  17.5 0.000015486
  18.5 0.00001216
  19.5 0.000010334
  20.5 0.0000093317
  21.5 0.0000087815
  22.5 0.0000084858
  23.5 0.0000083409
  24.5 0.0000041487
/
\endpicture}

\caption{A coalescent simulation with growth: $\theta_0=1$,
  $\theta_1=100$, $\tau=7$. $\star$: expected; $\bullet$: expected if
  population size is constant.}\label{fig.gro3}
\end{figure}

\begin{figure}
\centering
% -*-latex-*-
%
%%%%%%%%%%%%%% Tree in PicTeX Format %%%%%%%%%%%%%%
%
%Sample size   = 50
%Mutation rate = 1
%     theta         mn        tau          K
%  100.0000     0.0000     7.0000          1
%    1.0000     0.0000        Inf          1
%maxmm=15.000000 maxtau=7.700000 maxtree=9.124707 maxx=15.000000
%\newdimen\offsety
%\newdimen\yunit
%\newdimen\xunit
%\newdimen\thusfar
%\newdimen\plotht
%\newdimen\plotwd
%\newdimen\plotsp
\thusfar=0.000000in  % Keeps track of what's above
\plotht=1.400000in   % Height of each plot
\plotwd=4.000000in   % Width of each plot
\plotsp=0.500000in   % Spacing between plots
\mbox{\beginpicture
\def\mutation{\tiny$\bullet$}
\small
\valuestolabelleading=.4\baselineskip
\headingtoplotskip=0.4\baselineskip
%
%%%%%%%%%%%%%% Population size %%%%%%%%%%%%%%
%
\yunit=\plotht
\xunit=\plotwd
\Divide <\xunit> by <15.000000pt> forming <\xunit>
\Divide <\yunit> by <120.000008pt> forming <\yunit>
\advance\thusfar by \plotht
\advance\thusfar by \plotsp
\Divide <\thusfar> by <\yunit> forming <\offsety>
\placevalueinpts of <\offsety> in {\mktree_tmp}
\setcoordinatesystem units <\xunit, \yunit> point at 0 {\mktree_tmp}
\setplotarea x from 0.000000 to 15.000000, y from 0.000000 to 120.000008
\advance\thusfar by 0.030000\plotht
\axis left label {\lines{Population\cr size}} /
\axis bottom shiftedto y=-3.600000
    label {Mutational time before present}
    ticks numbered from 0 to 15 by 5 /
\putrule from 0.000000 100.000000 to 7.000000 100.000000
\putrule from 7.000000 100.000000 to 7.000000 1.000000
\putrule from 7.000000 1.000000 to 15.000000 1.000000
%
%%%%%%%%%%%%%% Gene Genealogy %%%%%%%%%%%%%%
%
\yunit=\plotht
\xunit=\plotwd
\Divide <\xunit> by <15.000000pt> forming <\xunit>
\Divide <\yunit> by <49.000000pt> forming <\yunit>
\advance\thusfar by \plotht
\advance\thusfar by \plotsp
\Divide <\thusfar> by <\yunit> forming <\offsety>
\placevalueinpts of <\offsety> in {\mktree_tmp}
\setcoordinatesystem units <\xunit, \yunit> point at 0 {\mktree_tmp}
\setplotarea x from 0.000000 to 15.000000, y from 0.000000 to 49.000000
\axis left invisible label {\lines{Gene\cr genealogy}} /
\axis bottom invisible
     label {Mutational time before present} /
\putrule from 0.000000 0.000000 to 4.089862 0.000000
\put {\mutation} at 3.273713 0.000000
\putrule from 0.000000 1.000000 to 1.753935 1.000000
\put {\mutation} at 1.693327 1.000000
\putrule from 0.000000 2.000000 to 1.753935 2.000000
\put {\mutation} at 1.229874 2.000000
\putrule from 1.753935 2.000000 to 1.753935 1.000000
\putrule from 1.753935 1.500000 to 3.024858 1.500000
\putrule from 0.000000 3.000000 to 3.024858 3.000000
\put {\mutation} at 0.668111 3.000000
\putrule from 3.024858 3.000000 to 3.024858 1.500000
\putrule from 3.024858 2.250000 to 4.089862 2.250000
\put {\mutation} at 3.495496 2.250000
\putrule from 4.089862 2.250000 to 4.089862 0.000000
\putrule from 4.089862 1.125000 to 7.253007 1.125000
\put {\mutation} at 5.511385 1.125000
\put {\mutation} at 6.521324 1.125000
\putrule from 0.000000 4.000000 to 7.253007 4.000000
\put {\mutation} at 5.438126 4.000000
\put {\mutation} at 3.279545 4.000000
\put {\mutation} at 1.689530 4.000000
\put {\mutation} at 6.880178 4.000000
\putrule from 7.253007 4.000000 to 7.253007 1.125000
\putrule from 7.253007 2.562500 to 7.472447 2.562500
\putrule from 0.000000 5.000000 to 7.036518 5.000000
\put {\mutation} at 2.031423 5.000000
\put {\mutation} at 2.020653 5.000000
\put {\mutation} at 6.815755 5.000000
\put {\mutation} at 1.062389 5.000000
\put {\mutation} at 3.213229 5.000000
\putrule from 0.000000 6.000000 to 7.025216 6.000000
\put {\mutation} at 2.624675 6.000000
\putrule from 0.000000 7.000000 to 0.214853 7.000000
\putrule from 0.000000 8.000000 to 0.214853 8.000000
\putrule from 0.214853 8.000000 to 0.214853 7.000000
\putrule from 0.214853 7.500000 to 7.025215 7.500000
\put {\mutation} at 2.876151 7.500000
\put {\mutation} at 2.465980 7.500000
\put {\mutation} at 2.531638 7.500000
\put {\mutation} at 0.285783 7.500000
\putrule from 7.025215 7.500000 to 7.025215 6.000000
\putrule from 7.025215 6.750000 to 7.036518 6.750000
\putrule from 7.036518 6.750000 to 7.036518 5.000000
\putrule from 7.036518 5.875000 to 7.120313 5.875000
\putrule from 0.000000 9.000000 to 6.381786 9.000000
\put {\mutation} at 4.918630 9.000000
\putrule from 0.000000 10.000000 to 5.523664 10.000000
\put {\mutation} at 1.200229 10.000000
\putrule from 0.000000 11.000000 to 0.836723 11.000000
\put {\mutation} at 0.044373 11.000000
\putrule from 0.000000 12.000000 to 0.836723 12.000000
\putrule from 0.836723 12.000000 to 0.836723 11.000000
\putrule from 0.836723 11.500000 to 2.973038 11.500000
\putrule from 0.000000 13.000000 to 2.973038 13.000000
\putrule from 2.973038 13.000000 to 2.973038 11.500000
\putrule from 2.973038 12.250000 to 5.523664 12.250000
\put {\mutation} at 4.199162 12.250000
\put {\mutation} at 5.444866 12.250000
\put {\mutation} at 4.110117 12.250000
\putrule from 5.523664 12.250000 to 5.523664 10.000000
\putrule from 5.523664 11.125000 to 6.381785 11.125000
\put {\mutation} at 5.670516 11.125000
\putrule from 6.381785 11.125000 to 6.381785 9.000000
\putrule from 6.381785 10.062500 to 7.120312 10.062500
\putrule from 7.120312 10.062500 to 7.120312 5.875000
\putrule from 7.120312 7.968750 to 7.472447 7.968750
\putrule from 7.472447 7.968750 to 7.472447 2.562500
\putrule from 7.472447 5.265625 to 8.295189 5.265625
\putrule from 0.000000 14.000000 to 7.794590 14.000000
\put {\mutation} at 1.026723 14.000000
\put {\mutation} at 1.237329 14.000000
\put {\mutation} at 6.298368 14.000000
\put {\mutation} at 7.560247 14.000000
\putrule from 0.000000 15.000000 to 7.001818 15.000000
\put {\mutation} at 0.045569 15.000000
\put {\mutation} at 2.934522 15.000000
\putrule from 0.000000 16.000000 to 3.993508 16.000000
\put {\mutation} at 3.456326 16.000000
\put {\mutation} at 2.444788 16.000000
\put {\mutation} at 3.658743 16.000000
\put {\mutation} at 0.024212 16.000000
\put {\mutation} at 3.234780 16.000000
\putrule from 0.000000 17.000000 to 3.917101 17.000000
\putrule from 0.000000 18.000000 to 1.330436 18.000000
\putrule from 0.000000 19.000000 to 1.330436 19.000000
\put {\mutation} at 0.354461 19.000000
\putrule from 1.330436 19.000000 to 1.330436 18.000000
\putrule from 1.330436 18.500000 to 2.304349 18.500000
\putrule from 0.000000 20.000000 to 2.304349 20.000000
\put {\mutation} at 1.870433 20.000000
\putrule from 2.304349 20.000000 to 2.304349 18.500000
\putrule from 2.304349 19.250000 to 3.917101 19.250000
\putrule from 3.917101 19.250000 to 3.917101 17.000000
\putrule from 3.917101 18.125000 to 3.993508 18.125000
\putrule from 3.993508 18.125000 to 3.993508 16.000000
\putrule from 3.993508 17.062500 to 7.001818 17.062500
\putrule from 7.001818 17.062500 to 7.001818 15.000000
\putrule from 7.001818 16.031250 to 7.018748 16.031250
\putrule from 0.000000 21.000000 to 7.018748 21.000000
\put {\mutation} at 2.748759 21.000000
\put {\mutation} at 1.729176 21.000000
\put {\mutation} at 2.634480 21.000000
\putrule from 7.018748 21.000000 to 7.018748 16.031250
\putrule from 7.018748 18.515625 to 7.036377 18.515625
\putrule from 0.000000 22.000000 to 7.033746 22.000000
\put {\mutation} at 0.706862 22.000000
\put {\mutation} at 6.837131 22.000000
\put {\mutation} at 6.157099 22.000000
\put {\mutation} at 0.887419 22.000000
\putrule from 0.000000 23.000000 to 7.016287 23.000000
\put {\mutation} at 4.317172 23.000000
\put {\mutation} at 2.687002 23.000000
\put {\mutation} at 2.667936 23.000000
\putrule from 0.000000 24.000000 to 4.713354 24.000000
\put {\mutation} at 1.788911 24.000000
\put {\mutation} at 1.563316 24.000000
\put {\mutation} at 2.840984 24.000000
\put {\mutation} at 1.796808 24.000000
\put {\mutation} at 3.035937 24.000000
\putrule from 0.000000 25.000000 to 1.399768 25.000000
\put {\mutation} at 0.780328 25.000000
\putrule from 0.000000 26.000000 to 1.399768 26.000000
\putrule from 1.399768 26.000000 to 1.399768 25.000000
\putrule from 1.399768 25.500000 to 1.469811 25.500000
\putrule from 0.000000 27.000000 to 1.262489 27.000000
\put {\mutation} at 0.766127 27.000000
\putrule from 0.000000 28.000000 to 1.262489 28.000000
\putrule from 1.262489 28.000000 to 1.262489 27.000000
\putrule from 1.262489 27.500000 to 1.469811 27.500000
\putrule from 1.469811 27.500000 to 1.469811 25.500000
\putrule from 1.469811 26.500000 to 4.713353 26.500000
\put {\mutation} at 2.676186 26.500000
\putrule from 4.713353 26.500000 to 4.713353 24.000000
\putrule from 4.713353 25.250000 to 7.016286 25.250000
\put {\mutation} at 6.796206 25.250000
\putrule from 7.016286 25.250000 to 7.016286 23.000000
\putrule from 7.016286 24.125000 to 7.033745 24.125000
\putrule from 7.033745 24.125000 to 7.033745 22.000000
\putrule from 7.033745 23.062500 to 7.036376 23.062500
\putrule from 7.036376 23.062500 to 7.036376 18.515625
\putrule from 7.036376 20.789062 to 7.038386 20.789062
\putrule from 0.000000 29.000000 to 2.152663 29.000000
\putrule from 0.000000 30.000000 to 1.051499 30.000000
\putrule from 0.000000 31.000000 to 0.090138 31.000000
\putrule from 0.000000 32.000000 to 0.090138 32.000000
\putrule from 0.090138 32.000000 to 0.090138 31.000000
\putrule from 0.090138 31.500000 to 1.051499 31.500000
\putrule from 1.051499 31.500000 to 1.051499 30.000000
\putrule from 1.051499 30.750000 to 2.152663 30.750000
\put {\mutation} at 1.301192 30.750000
\putrule from 2.152663 30.750000 to 2.152663 29.000000
\putrule from 2.152663 29.875000 to 4.790973 29.875000
\put {\mutation} at 3.458041 29.875000
\put {\mutation} at 4.192587 29.875000
\putrule from 0.000000 33.000000 to 4.790973 33.000000
\put {\mutation} at 1.761414 33.000000
\put {\mutation} at 2.597583 33.000000
\put {\mutation} at 0.791650 33.000000
\putrule from 4.790973 33.000000 to 4.790973 29.875000
\putrule from 4.790973 31.437500 to 7.038386 31.437500
\put {\mutation} at 5.063148 31.437500
\putrule from 7.038386 31.437500 to 7.038386 20.789062
\putrule from 7.038386 26.113281 to 7.099000 26.113281
\putrule from 0.000000 34.000000 to 2.845707 34.000000
\put {\mutation} at 0.842916 34.000000
\putrule from 0.000000 35.000000 to 2.845707 35.000000
\put {\mutation} at 0.447551 35.000000
\put {\mutation} at 1.112992 35.000000
\put {\mutation} at 1.409663 35.000000
\putrule from 2.845707 35.000000 to 2.845707 34.000000
\putrule from 2.845707 34.500000 to 7.062455 34.500000
\put {\mutation} at 6.110289 34.500000
\put {\mutation} at 4.187581 34.500000
\putrule from 0.000000 36.000000 to 2.762441 36.000000
\put {\mutation} at 2.283840 36.000000
\putrule from 0.000000 37.000000 to 2.762441 37.000000
\put {\mutation} at 0.272124 37.000000
\put {\mutation} at 0.595534 37.000000
\putrule from 2.762441 37.000000 to 2.762441 36.000000
\putrule from 2.762441 36.500000 to 7.062455 36.500000
\put {\mutation} at 6.742062 36.500000
\putrule from 7.062455 36.500000 to 7.062455 34.500000
\putrule from 7.062455 35.500000 to 7.099000 35.500000
\putrule from 7.099000 35.500000 to 7.099000 26.113281
\putrule from 7.099000 30.806641 to 7.654825 30.806641
\putrule from 0.000000 38.000000 to 0.810135 38.000000
\put {\mutation} at 0.129631 38.000000
\putrule from 0.000000 39.000000 to 0.810135 39.000000
\putrule from 0.810135 39.000000 to 0.810135 38.000000
\putrule from 0.810135 38.500000 to 2.270491 38.500000
\put {\mutation} at 1.358859 38.500000
\putrule from 0.000000 40.000000 to 2.270491 40.000000
\putrule from 2.270491 40.000000 to 2.270491 38.500000
\putrule from 2.270491 39.250000 to 7.115453 39.250000
\put {\mutation} at 6.333641 39.250000
\putrule from 0.000000 41.000000 to 5.330717 41.000000
\putrule from 0.000000 42.000000 to 0.684761 42.000000
\putrule from 0.000000 43.000000 to 0.684761 43.000000
\put {\mutation} at 0.528808 43.000000
\putrule from 0.684761 43.000000 to 0.684761 42.000000
\putrule from 0.684761 42.500000 to 5.330717 42.500000
\put {\mutation} at 3.904910 42.500000
\put {\mutation} at 3.183702 42.500000
\putrule from 5.330717 42.500000 to 5.330717 41.000000
\putrule from 5.330717 41.750000 to 5.344400 41.750000
\putrule from 0.000000 44.000000 to 0.713531 44.000000
\putrule from 0.000000 45.000000 to 0.555886 45.000000
\putrule from 0.000000 46.000000 to 0.555886 46.000000
\putrule from 0.555886 46.000000 to 0.555886 45.000000
\putrule from 0.555886 45.500000 to 0.713531 45.500000
\putrule from 0.713531 45.500000 to 0.713531 44.000000
\putrule from 0.713531 44.750000 to 5.344400 44.750000
\put {\mutation} at 4.113705 44.750000
\put {\mutation} at 4.893755 44.750000
\put {\mutation} at 2.039665 44.750000
\put {\mutation} at 0.970940 44.750000
\put {\mutation} at 4.285342 44.750000
\putrule from 5.344400 44.750000 to 5.344400 41.750000
\putrule from 5.344400 43.250000 to 7.115453 43.250000
\put {\mutation} at 5.893427 43.250000
\put {\mutation} at 5.595569 43.250000
\put {\mutation} at 6.624681 43.250000
\putrule from 7.115453 43.250000 to 7.115453 39.250000
\putrule from 7.115453 41.250000 to 7.378718 41.250000
\putrule from 0.000000 47.000000 to 3.533234 47.000000
\put {\mutation} at 2.408500 47.000000
\put {\mutation} at 1.905325 47.000000
\put {\mutation} at 0.589001 47.000000
\put {\mutation} at 0.090591 47.000000
\putrule from 0.000000 48.000000 to 1.444135 48.000000
\putrule from 0.000000 49.000000 to 1.444135 49.000000
\putrule from 1.444135 49.000000 to 1.444135 48.000000
\putrule from 1.444135 48.500000 to 3.533234 48.500000
\put {\mutation} at 1.687585 48.500000
\putrule from 3.533234 48.500000 to 3.533234 47.000000
\putrule from 3.533234 47.750000 to 7.378718 47.750000
\put {\mutation} at 3.828863 47.750000
\put {\mutation} at 5.875672 47.750000
\put {\mutation} at 4.912933 47.750000
\putrule from 7.378718 47.750000 to 7.378718 41.250000
\putrule from 7.378718 44.500000 to 7.654826 44.500000
\put {\mutation} at 7.599838 44.500000
\putrule from 7.654826 44.500000 to 7.654826 30.806641
\putrule from 7.654826 37.653320 to 7.794589 37.653320
\putrule from 7.794589 37.653320 to 7.794589 14.000000
\putrule from 7.794589 25.826660 to 8.295188 25.826660
\putrule from 8.295188 25.826660 to 8.295188 5.265625
\putrule from 8.295188 15.546143 to 15.000000 15.546143
%
%%%%%%%%%%%%%% Mismatch Distribution %%%%%%%%%%%%%%
%
\yunit=\plotht
\xunit=\plotwd
\Divide <\xunit> by <15.000000pt> forming <\xunit>
\Divide <\yunit> by <0.154286pt> forming <\yunit>
\advance\thusfar by \plotht
\advance\thusfar by \plotsp
\Divide <\thusfar> by <\yunit> forming <\offsety>
\placevalueinpts of <\offsety> in {\mktree_tmp}
\setcoordinatesystem units <\xunit, \yunit> point at 0 {\mktree_tmp}
\setplotarea x from 0.000000 to 15.000000, y from 0.000000 to 0.154286
\axis left label {\lines{Mismatch\cr distribution}} /
\axis bottom
   label {Pairwise differences}
   ticks numbered from 0 to 15 by 5 /
\multiput {$\circ$} at
 0 0.008980 1 0.016327 2 0.019592 3 0.038367 4 0.070204
 5 0.109388 6 0.133878 7 0.154286 8 0.146939 9 0.103673
 10 0.062041 11 0.040816 12 0.040816 13 0.039184 14 0.010612
 15 0.004898
/
%\plot
% 0 0.008980 1 0.016327 2 0.019592 3 0.038367 4 0.070204
% 5 0.109388 6 0.133878 7 0.154286 8 0.146939 9 0.103673
% 10 0.062041 11 0.040816 12 0.040816 13 0.039184 14 0.010612
% 15 0.004898
%/
%Pr[random pair differs by i sites]
\plot 0 0.010318 1 0.012924 2 0.021443 3 0.039166 4 0.065612
      5 0.095054 6 0.118885 7 0.130095 8 0.126489 9 0.110838
      10 0.088640 11 0.065416 12 0.044995 13 0.029106 14 0.017854
      15 0.010466 /
\plotht=1.100000\plotht
%
%%%%%%%%%%%%%% Simulated site frequency spectrum %%%%%%%%%%%%%%
%
\yunit=\plotht
\xunit=\plotwd
\Divide <\xunit> by <25.000000pt> forming <\xunit>
\Divide <\yunit> by <0.693000pt> forming <\yunit>
\advance\thusfar by \plotht
\advance\thusfar by \plotsp
\Divide <\thusfar> by <\yunit> forming <\offsety>
\placevalueinpts of <\offsety> in {\mktree_tmp}
\setcoordinatesystem units <\xunit, \yunit> point at 0 {\mktree_tmp}
\setplotarea x from 0.000000 to 25.000000, y from 0.000000 to 0.693000
\axis left invisible label {\lines{Site\cr frequency\cr spectrum}} /
\axis bottom
   label {Frequency of minor allele}
   ticks withvalues 0 {1/2} / at 0 25 / /
\linethickness 1.2pt
\axis top /
\linethickness 0.4pt
\sethistograms
\plot 0 0 1 0.630000 2 0.110000 3 0.140000 4 0.060000
 5 0.020000 6 0.030000 7 0.000000 8 0.000000 9 0.000000
 10 0.000000 11 0.000000 12 0.010000 13 0.000000 14 0.000000
 15 0.000000 16 0.000000 17 0.000000 18 0.000000 19 0.000000
 20 0.000000 21 0.000000 22 0.000000 23 0.000000 24 0.000000
 25 0.000000
/
\setlinear
%
%%%%% stationary Neutral expectation of folded spectrum %%%%%%%%%%%%%%
%
\multiput {$\bullet$} at 
     0.500000 0.22781  1.500000 0.116278  2.500000 0.079168  3.500000 0.0606668  4.500000 0.049612
     5.500000 0.0422829  6.500000 0.0370854  7.500000 0.0332223  8.500000 0.0302512  9.500000 0.0279067
     10.500000 0.0260203  11.500000 0.0244796  12.500000 0.0232073  13.500000 0.0221482  14.500000 0.0212623
     15.500000 0.0205197  16.500000 0.0198978  17.500000 0.0193797  18.500000 0.0189519  19.500000 0.0186045
     20.500000 0.0183295  21.500000 0.0181213  22.500000 0.0179754  23.500000 0.0178889  24.500000 0.0178603
/
%%%%%%%% NONstationary neutral expectation of folded spectrum %%%%%%%%%%%%%
\multiput {$\star$} at
   0.5    0.62975
   1.5    0.20128
   2.5   0.085283
   3.5   0.040411
   4.5   0.020302
   5.5   0.010559
   6.5  0.0056144
   7.5  0.0030288
   8.5  0.0016502
   9.5 0.00090581
  10.5 0.00050037
  11.5 0.00027842
  12.5 0.00015658
  13.5 0.00008963
  14.5 0.000052836
  15.5 0.000032635
  16.5 0.000021557
  17.5 0.000015486
  18.5 0.00001216
  19.5 0.000010334
  20.5 0.0000093317
  21.5 0.0000087815
  22.5 0.0000084858
  23.5 0.0000083409
  24.5 0.0000041487
/
\endpicture}

\caption{A coalescent simulation with growth: $\theta_0=1$,
  $\theta_1=100$, $\tau=7$. $\star$: expected; $\bullet$: expected if
  population size is constant.}\label{fig.gro4}
\end{figure}
\clearpage

\begin{subappendices}

\section{Point estimators for expanded populations (optional)}

If the expansion has been dramatic, the mismatch distribution will be
a smooth wave with a single mode.  In that case, the time of the
expansion and the size of the pre-expansion population can be
estimated using the following statistics \citep{Rogers:E-49-608}:
\begin{eqnarray}
\hat\theta_1 &=& \sqrt{v - \pi}\label{eq.hat.theta1}\\
\hat\tau_0 &=& m - \hat\theta_1\label{eq.hat.tau0}
\end{eqnarray}
where $\pi$ is the mean pairwise difference per sequence within the
sample, $m$ is the mean of pairwise differences, and $v$ is the
variance.

\section{Statistical properties of point estimates (optional)}

To determine the statistical properties of $\hat\theta_1$ and
$\hat\tau_0$, I used the coalescent algorithm \citep{Hudson:OSE-7-1}
to generate 1,000 simulated data sets at each of a wide variety of
parameter values.  In order to allow for changes in population size, I
used a modified version of the coalescent algorithm, which is
described elsewhere \citep{Rogers:PPG-97-55}.  I estimated $\theta_1$
and $\tau_0$ from each simulated data set, thus obtaining an estimate
of the sampling distribution of the estimators for each set of
parameter values.

% -*-latex-*-
\begin{figure}[p]
\pagestyle{empty}
\begin{center}
\mbox{%
\beginpicture
\headingtoplotskip=0.5\baselineskip
\setcoordinatesystem units <.18in, .18in> point at 0 0
\setplotarea x from 0 to 12, y from -0.4 to 12
\axis left label {\lines{Quantiles\cr of $\hat\tau_0$}} shiftedto x=-1
   ticks numbered from 0 to 12 by 3 /
\axis bottom shiftedto y=-1.4
  label {$\tau_0$} ticks numbered from 0 to 12 by 2 /
\multiput {$\bullet$} at 0 0 2 2 4 4 6 6 8 8 10 10 12 12 /
\setdots
%x=tau y=quantile 0.025 for tau hat
\plot 0 0 2 1.397624 4 3.179512 6 5.007843 8 6.524790 10
8.207713 12 9.851202 /
\setdashes
%x=tau y=quantile 0.25 for tau hat
\plot 0 0.265772 2 2.174634 4 4.000555 6 5.849672 8 7.650327 10 9.418620
 12 11.097290 /
\setsolid
%x=tau y=quantile 0.5 for tau hat
\plot 0 0.594539 2 2.567178  4 4.502656  6 6.384325  8 8.229242
10 10.089647  12 11.879292  /
\setdashes
%x=tau y=quantile 0.75 for tau hat
\plot 0 0.941873 2 2.932317  4 4.973155  6 6.848570  8 8.817445 
10 10.757429 12 12.574410 /
\setdots
%x=tau y=quantile 0.975 for tau hat
\plot 0 1.609915 2 3.702730 4 5.724909 6 7.776349 8 9.942038
10 12.014911 12 13.931237 /
\endpicture%
}
\end{center}
\caption{Quantiles of $\hat\tau_0$. 
1000 data sets were simulated at each of several values of
$\tau_0$, and each was used to estimate the model's parameters.  The
bold dots indicate points at which $\hat\tau_0=\tau_0$.  The solid line is
the median, the dashed lines enclose the central 50\% of the
distribution, and the dotted lines the central 95\%.  Each simulated
data set was generated using the coalescent algorithm with $\theta_0 =
500$, $\theta_1 = 1$, and $N=147$.}
\label{fig.tau}
\end{figure}
%%% Local Variables: 
%%% mode: latex
%%% TeX-master: t
%%% End: 

%
Figure~\ref{fig.tau} shows how the sampling distribution of $\hat\tau_0$
changes in response to variation in the underlying parameter $\tau_0$.
If $\hat\tau_0$ is in fact an estimator of $\tau_0$, we would expect the
median of $\hat\tau_0$ (shown as a solid line in the figure) to increase
in response to increases in $\tau_0$.  This is indeed the case.  An
ideal estimator should also have a relatively narrow distribution at
each value of $\tau_0$.  The dashed and dotted lines show that
$\hat\tau_0$ also satisfies this test.  The dashed lines enclose the
central 50\% of the distribution, and the dotted lines the central
95\%.  Both sets of lines enclose a relatively narrow interval about
the median.  In all of these respects, $\hat\tau_0$ behaves as an
estimator of $\tau_0$.

\input{figtheta1}
%
Figure~\ref{fig.theta1} performs a similar analysis on $\hat\theta_1$,
and shows it to perform well as an estimator when $\theta_1>1$.  The
distribution is tightly centered about the bold dots, showing that
$\hat\theta$ is rich in information and nearly unbiased when
$\hat\theta_1>1$.  But when $\theta_1<1$, the upper quantiles of
$\log_{10}\hat\theta_1$ are horizontal, while the median and lower
quantiles of $\hat\theta_1$ equal 0.  Thus an estimate of
$\hat\theta_1\approx 1$ is equally consistent with the hypotheses that
$\theta_1=1$ and that $\theta_1=0$.  Although $\hat\theta_1$ will
always allow us to place an upper bound on $\theta_1$, it can provide
no lower bound unless $\hat\theta_1$ is much greater than 1.  This is
no serious problem; it means only that when the estimate is near
unity, the confidence interval will reach all the way to 0.

% -*-latex-*-
\begin{figure}
\begin{center}
\mbox{%
\beginpicture
\renewcommand{\baselinestretch}{1}  
\setcoordinatesystem units <.069in, 12.5in> point at 36.25 0
\setplotarea x from 0 to 29, y from 0 to 0.16
\axis left label {$F_i$}
  ticks numbered from 0.0 to 0.1 by  0.1 /
\axis bottom 
  label {$i$} ticks
  numbered from 0 to 25 by  5 /
%Africa
\setbox0=\hbox{$\swarrow$}%
\put {$\swarrow$ \raise \ht0 \hbox{Africa}}
     [bl] at 26 0.007433
\plot
  0 0.000782 1 0.007042 2 0.008607 3 0.011346
 4 0.008998 5 0.012128 6 0.011346 7 0.017997 8 0.028169
 9 0.033646 10 0.043427 11 0.040297 12 0.043427 13 0.046557
 14 0.060250 15 0.078638 16 0.078638 17 0.102113 18 0.094288
 19 0.070423 20 0.059859 21 0.050078 22 0.034820 23 0.020344
 24 0.014476 25 0.010955 26 0.007433 27 0.001565 28 0.001174
 29 0.001174
/
%Asia
\setbox0=\hbox{$\swarrow$}%
\put {$\swarrow$ \raise \ht0 \hbox{Asia}}
     [bl] at 10 0.129528
\plot
  0 0.001709 1 0.001709 2 0.003759 3 0.010253
 4 0.044771 5 0.068353 6 0.096719 7 0.133288 8 0.139781
 9 0.160971 10 0.129528 11 0.091593 12 0.063568 13 0.032468
 14 0.014696 15 0.005810 16 0.000684 17 0.000342
/
\setplotsymbol ({\scriptsize $\bullet$})
\setdots <10pt>
%Africa X Asia
\plot
  0 0.000000 1 0.000000 2 0.000000 3 0.001263
 4 0.004329 5 0.006133 6 0.012987 7 0.028860 8 0.028139
 9 0.037879 10 0.041847 11 0.035534 12 0.052850 13 0.055195
 14 0.067821 15 0.078102 16 0.087482 17 0.091450 18 0.085678
 19 0.082612 20 0.073413 21 0.061328 22 0.038240 23 0.019120
 24 0.006494 25 0.002345 26 0.000722 27 0.000180
/
\setplotsymbol ({\fiverm .})
\setsolid
%%%%%%%%%%%%%%%%%%%%%%%%%%%%%%%%%%%%%%%%%%%%%%%%%%%%%%%%%%%%%%%%
\setcoordinatesystem units <.069in, 12.5in> point at -7.25 0
\setplotarea x from 0 to 29, y from 0 to 0.16
\axis left label {$F_i$}
  ticks numbered from 0.0 to 0.1 by  0.1 /
\axis bottom 
  label {$i$} ticks
  numbered from 0 to 25 by  5 /
%Africa
\setbox0=\hbox{$\swarrow$}%
\put {$\swarrow$ \raise \ht0 \hbox{Africa}}
     [bl] at 26 0.007433
\plot
  0 0.000782 1 0.007042 2 0.008607 3 0.011346
 4 0.008998 5 0.012128 6 0.011346 7 0.017997 8 0.028169
 9 0.033646 10 0.043427 11 0.040297 12 0.043427 13 0.046557
 14 0.060250 15 0.078638 16 0.078638 17 0.102113 18 0.094288
 19 0.070423 20 0.059859 21 0.050078 22 0.034820 23 0.020344
 24 0.014476 25 0.010955 26 0.007433 27 0.001565 28 0.001174
 29 0.001174
/
%Europe
\setbox0=\hbox{$\swarrow$}%
\put {$\swarrow$ \raise \ht0 \hbox{Europe}}
     [bl] at 6 0.118233 
\plot
  0 0.020174 1 0.063841 2 0.092186 3 0.124362
 4 0.127426 5 0.118488 6 0.118233 7 0.105975 8 0.089377
 9 0.059244 10 0.037794 11 0.019663 12 0.011747 13 0.005873
 14 0.004341 15 0.000511 16 0.000766
/
\setplotsymbol ({\scriptsize $\bullet$})
\setdots <10pt>
%Africa X Europe
\plot
  0 0.000000 1 0.000000 2 0.000312 3 0.002185
 4 0.008427 5 0.018883 6 0.024969 7 0.026529 8 0.026373
 9 0.028558 10 0.030899 11 0.043071 12 0.045256 13 0.061486
 14 0.076467 15 0.082865 16 0.098471 17 0.100187 18 0.097690
 19 0.079432 20 0.061017 21 0.042135 22 0.022316 23 0.011704
 24 0.007179 25 0.002653 26 0.000780 27 0.000156
/
\setsolid
\setplotsymbol ({\fiverm .})
%%%%%%%%%%%%%%%%%%%%%%%%%%%%%%%%%%%%%%%%%%%%%%%%%%%%%%%%%%%%%%%%
\setcoordinatesystem units <.069in, 12.5in> point at 14.5 0.2
\setplotarea x from 0 to 29, y from 0 to 0.16
\axis left label {$F_i$}
  ticks numbered from 0.0 to 0.1 by  0.1 /
\axis bottom 
  label {$i$} ticks
  numbered from 0 to 25 by  5 /
%Asia
\setbox0=\hbox{$\swarrow$}%
\put {$\swarrow$ \raise \ht0 \hbox{Asia}}
     [bl] at 10 0.129528
\plot
  0 0.001709 1 0.001709 2 0.003759 3 0.010253
 4 0.044771 5 0.068353 6 0.096719 7 0.133288 8 0.139781
 9 0.160971 10 0.129528 11 0.091593 12 0.063568 13 0.032468
 14 0.014696 15 0.005810 16 0.000684 17 0.000342
/
%Europe
\setbox0=\hbox{$\searrow$}%
\put {\raise \ht0 \hbox{Eur.}$\searrow$}
     [br] at 4 0.127426 
\plot
  0 0.020174 1 0.063841 2 0.092186 3 0.124362
 4 0.127426 5 0.118488 6 0.118233 7 0.105975 8 0.089377
 9 0.059244 10 0.037794 11 0.019663 12 0.011747 13 0.005873
 14 0.004341 15 0.000511 16 0.000766
/
\setplotsymbol ({\scriptsize $\bullet$})
\setdots <10pt>
%Asia X Europe
\plot
  0 0.000000 1 0.001021 2 0.010069 3 0.027287
 4 0.057931 5 0.098351 6 0.138334 7 0.155115 8 0.149278
 9 0.124909 10 0.095141 11 0.066394 12 0.037356 13 0.023785
 14 0.009485 15 0.003794 16 0.001313 17 0.000438
/
\endpicture}
\end{center}
\caption{Mitochondrial Mismatch Distributions\label{fig.ALLmm}} 
%
\small In each panel, the solid lines show mismatch distributions for
within-population comparisons, and the dotted lines show the analogous
between-population comparisons.  The data comprise 72 Africans, 77
Asians, and 89 Europeans \citep{Jorde:AJH-57-523}.
\end{figure}

%%% Local Variables: 
%%% mode: latex
%%% TeX-master: t
%%% End: 


(Need some prose to go with figure~\ref{fig.ALLmm}.)

%\section{Confidence intervals (optional)}

\def\hunit{0.05in}
\def\vunit{0.18in}
\def\accept{{$\bullet$}}
\def\99pct{{$\cdot$}}
\def\reject{{\footnotesize$\circ$}}
%% Use the following to hide the 1% region
\def\99pct{\reject}

\begin{figure}
{\centering% -*-latex-*-
%InputFile = tau4.ci
%RangeLog10Theta0: -1.000000 1.301030 0.333333 
%RangeGrowth: 0.000000 7.000000 1.000000 
%RangeTau: 1.000000 9.000000 1.000000 
%% Confidence Interval: PicTeX output %%
%% Definitions:
\def\accept{{$\bullet$}}
\def\99pct{{$\cdot$}}
\def\reject{{\footnotesize$\circ$}}
%% Use the following to hide the 1% region
\def\99pct{\reject}
%%Plots are 1.000000 inch wide and 0.500000 inches high
\mbox{\beginpicture
\headingtoplotskip=0.5\baselineskip
\valuestolabelleading=0.4\baselineskip
%%%%%%%%%%%% Plot figure in row 0 col 0 %%%%%%%%%%%%
\setcoordinatesystem units <0.125000in,0.217294in> point at 17.000000 1.000000
\setplotarea x from 1.000000 to 9.000000, y from -1.000000 to 1.301030
\axis bottom shiftedto y=-1.230103  /
\axis left shiftedto x=0.600000 label {$\theta_1$}
   ticks withvalues 0.1 1 10 /
     at -1.000000 0.000000 1.000000 / /
\plotheading{\small$10^{0}$-fold growth}
%                        tau log10[theta0]
\put   {\reject} at        5            -1   %p-val=0
\put   {\reject} at        5      -0.66667   %p-val=0
\put   {\reject} at        5      -0.33333   %p-val=0
\put   {\reject} at        5             0   %p-val=0
\put   {\reject} at        5       0.33333   %p-val=0
\put   {\reject} at        5       0.66667   %p-val=0
\put   {\reject} at        5             1   %p-val=0
\put   {\reject} at        5        1.3333   %p-val=0
%%%%%%%%%%%% Plot figure in row 0 col 1 %%%%%%%%%%%%
\setcoordinatesystem units <0.125000in,0.217294in> point at 1.000000 1.000000
\setplotarea x from 1.000000 to 9.000000, y from -1.000000 to 1.301030
\axis bottom shiftedto y=-1.230103 label {$\tau_0$}
   ticks withvalues 2.5 5 7.5 /
     at 2.500000 5.000000 7.500000 / /
\axis left shiftedto x=0.600000 label {$\theta_1$}
   ticks withvalues 0.1 1 10 /
     at -1.000000 0.000000 1.000000 / /
\plotheading{\small$10^{1}$-fold growth}
%                        tau log10[theta0]
\put   {\reject} at        1            -1   %p-val=0
\put   {\reject} at        1      -0.66667   %p-val=0
\put   {\reject} at        1      -0.33333   %p-val=0
\put   {\reject} at        1             0   %p-val=0
\put   {\reject} at        1       0.33333   %p-val=0
\put   {\reject} at        1       0.66667   %p-val=0
\put   {\reject} at        1             1   %p-val=0
\put   {\reject} at        1        1.3333   %p-val=0
\put   {\reject} at        2            -1   %p-val=0
\put   {\reject} at        2      -0.66667   %p-val=0
\put   {\reject} at        2      -0.33333   %p-val=0
\put   {\reject} at        2             0   %p-val=0
\put   {\reject} at        2       0.33333   %p-val=0
\put   {\reject} at        2       0.66667   %p-val=0
\put   {\reject} at        2             1   %p-val=0
\put   {\reject} at        2        1.3333   %p-val=0
\put   {\reject} at        3            -1   %p-val=0
\put   {\reject} at        3      -0.66667   %p-val=0
\put   {\reject} at        3      -0.33333   %p-val=0
\put   {\reject} at        3             0   %p-val=0
\put   {\reject} at        3       0.33333   %p-val=0
\put   {\reject} at        3       0.66667   %p-val=0
%           ERROR          3             1    p-val=-2
\put   {\reject} at        3        1.3333   %p-val=0
\put   {\reject} at        4            -1   %p-val=0
\put   {\reject} at        4      -0.66667   %p-val=0
\put   {\reject} at        4      -0.33333   %p-val=0
\put   {\reject} at        4             0   %p-val=0
\put   {\reject} at        4       0.33333   %p-val=0
\put   {\reject} at        4       0.66667   %p-val=0
\put   {\reject} at        4             1   %p-val=0
\put   {\reject} at        4        1.3333   %p-val=0
\put   {\reject} at        5            -1   %p-val=0
\put   {\reject} at        5      -0.66667   %p-val=0
\put   {\reject} at        5      -0.33333   %p-val=0
\put   {\reject} at        5             0   %p-val=0
\put   {\reject} at        5       0.33333   %p-val=0
\put   {\reject} at        5       0.66667   %p-val=0
\put   {\reject} at        5             1   %p-val=0
\put   {\reject} at        5        1.3333   %p-val=0
\put   {\reject} at        6            -1   %p-val=0
\put   {\reject} at        6      -0.66667   %p-val=0
\put   {\reject} at        6      -0.33333   %p-val=0
\put   {\reject} at        6             0   %p-val=0
\put   {\reject} at        6       0.33333   %p-val=0
\put   {\reject} at        6       0.66667   %p-val=0
%           ERROR          6             1    p-val=-2
%           ERROR          6        1.3333    p-val=-2
\put   {\reject} at        7            -1   %p-val=0
\put   {\reject} at        7      -0.66667   %p-val=0
\put   {\reject} at        7      -0.33333   %p-val=0
\put   {\reject} at        7             0   %p-val=0
\put   {\reject} at        7       0.33333   %p-val=0
\put   {\reject} at        7       0.66667   %p-val=0
%           ERROR          7             1    p-val=-2
\put   {\reject} at        7        1.3333   %p-val=0
\put   {\reject} at        8            -1   %p-val=0
\put   {\reject} at        8      -0.66667   %p-val=0
\put   {\reject} at        8      -0.33333   %p-val=0
\put   {\reject} at        8             0   %p-val=0
%           ERROR          8       0.33333    p-val=-2
%           ERROR          8       0.66667    p-val=-2
%           ERROR          8             1    p-val=-2
\put   {\reject} at        8        1.3333   %p-val=0
\put   {\reject} at        9            -1   %p-val=0
\put   {\reject} at        9      -0.66667   %p-val=0
\put   {\reject} at        9      -0.33333   %p-val=0
\put   {\reject} at        9             0   %p-val=0
%           ERROR          9       0.33333    p-val=-2
%           ERROR          9       0.66667    p-val=-2
%           ERROR          9             1    p-val=-2
\put   {\reject} at        9        1.3333   %p-val=0
%%%%%%%%%%%% Plot figure in row 1 col 0 %%%%%%%%%%%%
\setcoordinatesystem units <0.125000in,0.217294in> point at 17.000000 6.752575
\setplotarea x from 1.000000 to 9.000000, y from -1.000000 to 1.301030
\axis bottom shiftedto y=-1.230103 label {$\tau_0$}
   ticks withvalues 2.5 5 7.5 /
     at 2.500000 5.000000 7.500000 / /
\axis left shiftedto x=0.600000 label {$\theta_1$}
   ticks withvalues 0.1 1 10 /
     at -1.000000 0.000000 1.000000 / /
\plotheading{\small$10^{2}$-fold growth}
%                        tau log10[theta0]
\put   {\reject} at        1            -1   %p-val=0
\put   {\reject} at        1      -0.66667   %p-val=0
\put   {\reject} at        1      -0.33333   %p-val=0
\put   {\reject} at        1             0   %p-val=0
\put   {\reject} at        1       0.33333   %p-val=0
\put   {\reject} at        1       0.66667   %p-val=0
\put   {\reject} at        1             1   %p-val=0
\put   {\reject} at        1        1.3333   %p-val=0
\put   {\reject} at        2            -1   %p-val=0
\put   {\reject} at        2      -0.66667   %p-val=0
\put   {\reject} at        2      -0.33333   %p-val=0
\put   {\reject} at        2             0   %p-val=0
\put   {\reject} at        2       0.33333   %p-val=0
\put   {\reject} at        2       0.66667   %p-val=0
\put   {\reject} at        2             1   %p-val=0.001
\put   {\reject} at        2        1.3333   %p-val=0
\put   {\reject} at        3            -1   %p-val=0
\put   {\reject} at        3      -0.66667   %p-val=0
\put   {\reject} at        3      -0.33333   %p-val=0
\put   {\reject} at        3             0   %p-val=0.008
\put    {\99pct} at        3       0.33333   %p-val=0.011
\put    {\99pct} at        3       0.66667   %p-val=0.011
\put   {\reject} at        3             1   %p-val=0
\put   {\reject} at        3        1.3333   %p-val=0
\put   {\reject} at        4            -1   %p-val=0
\put   {\reject} at        4      -0.66667   %p-val=0
\put   {\reject} at        4      -0.33333   %p-val=0.001
\put    {\99pct} at        4             0   %p-val=0.041
\put    {\99pct} at        4       0.33333   %p-val=0.035
\put   {\reject} at        4       0.66667   %p-val=0.001
\put   {\reject} at        4             1   %p-val=0
\put   {\reject} at        4        1.3333   %p-val=0
\put   {\reject} at        5            -1   %p-val=0
\put   {\reject} at        5      -0.66667   %p-val=0
\put   {\reject} at        5      -0.33333   %p-val=0.003
\put    {\99pct} at        5             0   %p-val=0.016
\put    {\99pct} at        5       0.33333   %p-val=0.011
\put   {\reject} at        5       0.66667   %p-val=0
\put   {\reject} at        5             1   %p-val=0
\put   {\reject} at        5        1.3333   %p-val=0
\put   {\reject} at        6            -1   %p-val=0
\put   {\reject} at        6      -0.66667   %p-val=0
\put   {\reject} at        6      -0.33333   %p-val=0
\put   {\reject} at        6             0   %p-val=0
\put   {\reject} at        6       0.33333   %p-val=0
\put   {\reject} at        6       0.66667   %p-val=0
\put   {\reject} at        6             1   %p-val=0
\put   {\reject} at        6        1.3333   %p-val=0
\put   {\reject} at        7            -1   %p-val=0
\put   {\reject} at        7      -0.66667   %p-val=0
\put   {\reject} at        7      -0.33333   %p-val=0
\put   {\reject} at        7             0   %p-val=0
\put   {\reject} at        7       0.33333   %p-val=0
\put   {\reject} at        7       0.66667   %p-val=0
%           ERROR          7             1    p-val=-2
\put   {\reject} at        7        1.3333   %p-val=0
\put   {\reject} at        8            -1   %p-val=0
\put   {\reject} at        8      -0.66667   %p-val=0
\put   {\reject} at        8      -0.33333   %p-val=0
\put   {\reject} at        8             0   %p-val=0
\put   {\reject} at        8       0.33333   %p-val=0
\put   {\reject} at        8       0.66667   %p-val=0
%           ERROR          8             1    p-val=-2
\put   {\reject} at        8        1.3333   %p-val=0
\put   {\reject} at        9            -1   %p-val=0
\put   {\reject} at        9      -0.66667   %p-val=0
\put   {\reject} at        9      -0.33333   %p-val=0
\put   {\reject} at        9             0   %p-val=0
\put   {\reject} at        9       0.33333   %p-val=0
\put   {\reject} at        9       0.66667   %p-val=0
\put   {\reject} at        9             1   %p-val=0
\put   {\reject} at        9        1.3333   %p-val=0
%%%%%%%%%%%% Plot figure in row 1 col 1 %%%%%%%%%%%%
\setcoordinatesystem units <0.125000in,0.217294in> point at 1.000000 6.752575
\setplotarea x from 1.000000 to 9.000000, y from -1.000000 to 1.301030
\axis bottom shiftedto y=-1.230103 label {$\tau_0$}
   ticks withvalues 2.5 5 7.5 /
     at 2.500000 5.000000 7.500000 / /
\axis left shiftedto x=0.600000 label {$\theta_1$}
   ticks withvalues 0.1 1 10 /
     at -1.000000 0.000000 1.000000 / /
\plotheading{\small$10^{3}$-fold growth}
%                        tau log10[theta0]
\put   {\reject} at        1            -1   %p-val=0
\put   {\reject} at        1      -0.66667   %p-val=0
\put   {\reject} at        1      -0.33333   %p-val=0
\put   {\reject} at        1             0   %p-val=0
\put   {\reject} at        1       0.33333   %p-val=0
\put   {\reject} at        1       0.66667   %p-val=0
\put   {\reject} at        1             1   %p-val=0
\put   {\reject} at        1        1.3333   %p-val=0
\put   {\reject} at        2            -1   %p-val=0
\put   {\reject} at        2      -0.66667   %p-val=0
\put   {\reject} at        2      -0.33333   %p-val=0
\put   {\reject} at        2             0   %p-val=0
\put   {\reject} at        2       0.33333   %p-val=0
\put   {\reject} at        2       0.66667   %p-val=0.001
\put   {\reject} at        2             1   %p-val=0
\put   {\reject} at        2        1.3333   %p-val=0
\put   {\reject} at        3            -1   %p-val=0
\put   {\reject} at        3      -0.66667   %p-val=0.001
\put   {\reject} at        3      -0.33333   %p-val=0.005
\put    {\99pct} at        3             0   %p-val=0.016
\put    {\99pct} at        3       0.33333   %p-val=0.04
\put    {\99pct} at        3       0.66667   %p-val=0.019
\put   {\reject} at        3             1   %p-val=0
\put   {\reject} at        3        1.3333   %p-val=0
\put    {\99pct} at        4            -1   %p-val=0.03
\put   {\accept} at        4      -0.66667   %p-val=0.082
\put   {\accept} at        4      -0.33333   %p-val=0.289
\put   {\accept} at        4             0   %p-val=0.426
\put   {\accept} at        4       0.33333   %p-val=0.062
\put   {\reject} at        4       0.66667   %p-val=0.002
\put   {\reject} at        4             1   %p-val=0
\put   {\reject} at        4        1.3333   %p-val=0
\put   {\accept} at        5            -1   %p-val=0.119
\put   {\accept} at        5      -0.66667   %p-val=0.227
\put   {\accept} at        5      -0.33333   %p-val=0.166
\put   {\accept} at        5             0   %p-val=0.091
\put   {\reject} at        5       0.33333   %p-val=0.007
\put   {\reject} at        5       0.66667   %p-val=0
\put   {\reject} at        5             1   %p-val=0
\put   {\reject} at        5        1.3333   %p-val=0
\put    {\99pct} at        6            -1   %p-val=0.012
\put    {\99pct} at        6      -0.66667   %p-val=0.012
\put   {\reject} at        6      -0.33333   %p-val=0.005
\put   {\reject} at        6             0   %p-val=0.002
\put   {\reject} at        6       0.33333   %p-val=0
\put   {\reject} at        6       0.66667   %p-val=0
\put   {\reject} at        6             1   %p-val=0
%           ERROR          6        1.3333    p-val=-2
\put   {\reject} at        7            -1   %p-val=0
\put   {\reject} at        7      -0.66667   %p-val=0.001
\put   {\reject} at        7      -0.33333   %p-val=0.001
\put   {\reject} at        7             0   %p-val=0
\put   {\reject} at        7       0.33333   %p-val=0
\put   {\reject} at        7       0.66667   %p-val=0
\put   {\reject} at        7             1   %p-val=0
\put   {\reject} at        7        1.3333   %p-val=0
\put   {\reject} at        8            -1   %p-val=0
\put   {\reject} at        8      -0.66667   %p-val=0
\put   {\reject} at        8      -0.33333   %p-val=0
\put   {\reject} at        8             0   %p-val=0
\put   {\reject} at        8       0.33333   %p-val=0
\put   {\reject} at        8       0.66667   %p-val=0
\put   {\reject} at        8             1   %p-val=0
%           ERROR          8        1.3333    p-val=-2
\put   {\reject} at        9            -1   %p-val=0
\put   {\reject} at        9      -0.66667   %p-val=0
\put   {\reject} at        9      -0.33333   %p-val=0
\put   {\reject} at        9             0   %p-val=0
\put   {\reject} at        9       0.33333   %p-val=0
\put   {\reject} at        9       0.66667   %p-val=0
\put   {\reject} at        9             1   %p-val=0
%           ERROR          9        1.3333    p-val=-2
%%%%%%%%%%%% Plot figure in row 2 col 0 %%%%%%%%%%%%
\setcoordinatesystem units <0.125000in,0.217294in> point at 17.000000 12.505151
\setplotarea x from 1.000000 to 9.000000, y from -1.000000 to 1.301030
\axis bottom shiftedto y=-1.230103 label {$\tau_0$}
   ticks withvalues 2.5 5 7.5 /
     at 2.500000 5.000000 7.500000 / /
\axis left shiftedto x=0.600000 label {$\theta_1$}
   ticks withvalues 0.1 1 10 /
     at -1.000000 0.000000 1.000000 / /
\plotheading{\small$10^{4}$-fold growth}
%                        tau log10[theta0]
\put   {\reject} at        1            -1   %p-val=0
\put   {\reject} at        1      -0.66667   %p-val=0
\put   {\reject} at        1      -0.33333   %p-val=0
\put   {\reject} at        1             0   %p-val=0
\put   {\reject} at        1       0.33333   %p-val=0
\put   {\reject} at        1       0.66667   %p-val=0
\put   {\reject} at        1             1   %p-val=0
\put   {\reject} at        1        1.3333   %p-val=0
\put   {\reject} at        2            -1   %p-val=0
\put   {\reject} at        2      -0.66667   %p-val=0
\put   {\reject} at        2      -0.33333   %p-val=0
\put   {\reject} at        2             0   %p-val=0
\put   {\reject} at        2       0.33333   %p-val=0
\put   {\reject} at        2       0.66667   %p-val=0
\put   {\reject} at        2             1   %p-val=0
\put   {\reject} at        2        1.3333   %p-val=0
\put   {\reject} at        3            -1   %p-val=0
\put   {\reject} at        3      -0.66667   %p-val=0.002
\put   {\reject} at        3      -0.33333   %p-val=0.003
\put    {\99pct} at        3             0   %p-val=0.017
\put    {\99pct} at        3       0.33333   %p-val=0.034
\put    {\99pct} at        3       0.66667   %p-val=0.028
\put   {\reject} at        3             1   %p-val=0
\put   {\reject} at        3        1.3333   %p-val=0
\put    {\99pct} at        4            -1   %p-val=0.046
\put   {\accept} at        4      -0.66667   %p-val=0.107
\put   {\accept} at        4      -0.33333   %p-val=0.406
\put   {\accept} at        4             0   %p-val=0.47
\put   {\accept} at        4       0.33333   %p-val=0.088
\put   {\reject} at        4       0.66667   %p-val=0.002
\put   {\reject} at        4             1   %p-val=0
\put   {\reject} at        4        1.3333   %p-val=0
\put   {\accept} at        5            -1   %p-val=0.273
\put   {\accept} at        5      -0.66667   %p-val=0.236
\put   {\accept} at        5      -0.33333   %p-val=0.141
\put   {\accept} at        5             0   %p-val=0.06
\put   {\reject} at        5       0.33333   %p-val=0.002
\put   {\reject} at        5       0.66667   %p-val=0
\put   {\reject} at        5             1   %p-val=0
\put   {\reject} at        5        1.3333   %p-val=0
\put   {\reject} at        6            -1   %p-val=0.003
\put   {\reject} at        6      -0.66667   %p-val=0.006
\put   {\reject} at        6      -0.33333   %p-val=0.004
\put   {\reject} at        6             0   %p-val=0
\put   {\reject} at        6       0.33333   %p-val=0
\put   {\reject} at        6       0.66667   %p-val=0
\put   {\reject} at        6             1   %p-val=0
\put   {\reject} at        6        1.3333   %p-val=0
\put   {\reject} at        7            -1   %p-val=0
\put   {\reject} at        7      -0.66667   %p-val=0
\put   {\reject} at        7      -0.33333   %p-val=0
\put   {\reject} at        7             0   %p-val=0
\put   {\reject} at        7       0.33333   %p-val=0
\put   {\reject} at        7       0.66667   %p-val=0
\put   {\reject} at        7             1   %p-val=0
\put   {\reject} at        7        1.3333   %p-val=0
\put   {\reject} at        8            -1   %p-val=0
\put   {\reject} at        8      -0.66667   %p-val=0
\put   {\reject} at        8      -0.33333   %p-val=0
\put   {\reject} at        8             0   %p-val=0
\put   {\reject} at        8       0.33333   %p-val=0
\put   {\reject} at        8       0.66667   %p-val=0
\put   {\reject} at        8             1   %p-val=0
%           ERROR          8        1.3333    p-val=-2
\put   {\reject} at        9            -1   %p-val=0
\put   {\reject} at        9      -0.66667   %p-val=0
\put   {\reject} at        9      -0.33333   %p-val=0
\put   {\reject} at        9             0   %p-val=0
\put   {\reject} at        9       0.33333   %p-val=0
\put   {\reject} at        9       0.66667   %p-val=0
\put   {\reject} at        9             1   %p-val=0
\put   {\reject} at        9        1.3333   %p-val=0
%%%%%%%%%%%% Plot figure in row 2 col 1 %%%%%%%%%%%%
\setcoordinatesystem units <0.125000in,0.217294in> point at 1.000000 12.505151
\setplotarea x from 1.000000 to 9.000000, y from -1.000000 to 1.301030
\axis bottom shiftedto y=-1.230103 label {$\tau_0$}
   ticks withvalues 2.5 5 7.5 /
     at 2.500000 5.000000 7.500000 / /
\axis left shiftedto x=0.600000 label {$\theta_1$}
   ticks withvalues 0.1 1 10 /
     at -1.000000 0.000000 1.000000 / /
\plotheading{\small$10^{5}$-fold growth}
%                        tau log10[theta0]
\put   {\reject} at        1            -1   %p-val=0
\put   {\reject} at        1      -0.66667   %p-val=0
\put   {\reject} at        1      -0.33333   %p-val=0
\put   {\reject} at        1             0   %p-val=0
\put   {\reject} at        1       0.33333   %p-val=0
\put   {\reject} at        1       0.66667   %p-val=0
\put   {\reject} at        1             1   %p-val=0
\put   {\reject} at        1        1.3333   %p-val=0
\put   {\reject} at        2            -1   %p-val=0
\put   {\reject} at        2      -0.66667   %p-val=0
\put   {\reject} at        2      -0.33333   %p-val=0
\put   {\reject} at        2             0   %p-val=0
\put   {\reject} at        2       0.33333   %p-val=0
\put   {\reject} at        2       0.66667   %p-val=0.001
\put   {\reject} at        2             1   %p-val=0
\put   {\reject} at        2        1.3333   %p-val=0
\put   {\reject} at        3            -1   %p-val=0
\put   {\reject} at        3      -0.66667   %p-val=0
\put   {\reject} at        3      -0.33333   %p-val=0.001
\put    {\99pct} at        3             0   %p-val=0.016
\put    {\99pct} at        3       0.33333   %p-val=0.023
\put    {\99pct} at        3       0.66667   %p-val=0.015
\put   {\reject} at        3             1   %p-val=0
\put   {\reject} at        3        1.3333   %p-val=0
\put   {\accept} at        4            -1   %p-val=0.055
\put   {\accept} at        4      -0.66667   %p-val=0.122
\put   {\accept} at        4      -0.33333   %p-val=0.398
\put   {\accept} at        4             0   %p-val=0.551
\put   {\accept} at        4       0.33333   %p-val=0.076
\put   {\reject} at        4       0.66667   %p-val=0.003
\put   {\reject} at        4             1   %p-val=0
\put   {\reject} at        4        1.3333   %p-val=0
\put   {\accept} at        5            -1   %p-val=0.29
\put   {\accept} at        5      -0.66667   %p-val=0.172
\put   {\accept} at        5      -0.33333   %p-val=0.149
\put   {\accept} at        5             0   %p-val=0.067
\put   {\reject} at        5       0.33333   %p-val=0.004
\put   {\reject} at        5       0.66667   %p-val=0
\put   {\reject} at        5             1   %p-val=0
\put   {\reject} at        5        1.3333   %p-val=0
\put   {\reject} at        6            -1   %p-val=0.001
\put   {\reject} at        6      -0.66667   %p-val=0.005
\put   {\reject} at        6      -0.33333   %p-val=0.005
\put   {\reject} at        6             0   %p-val=0
\put   {\reject} at        6       0.33333   %p-val=0
\put   {\reject} at        6       0.66667   %p-val=0
%           ERROR          6             1    p-val=-2
\put   {\reject} at        6        1.3333   %p-val=0
\put   {\reject} at        7            -1   %p-val=0
\put   {\reject} at        7      -0.66667   %p-val=0
\put   {\reject} at        7      -0.33333   %p-val=0
\put   {\reject} at        7             0   %p-val=0
\put   {\reject} at        7       0.33333   %p-val=0
\put   {\reject} at        7       0.66667   %p-val=0
\put   {\reject} at        7             1   %p-val=0
\put   {\reject} at        7        1.3333   %p-val=0
\put   {\reject} at        8            -1   %p-val=0
\put   {\reject} at        8      -0.66667   %p-val=0
\put   {\reject} at        8      -0.33333   %p-val=0
\put   {\reject} at        8             0   %p-val=0
\put   {\reject} at        8       0.33333   %p-val=0
\put   {\reject} at        8       0.66667   %p-val=0
\put   {\reject} at        8             1   %p-val=0
\put   {\reject} at        8        1.3333   %p-val=0
\put   {\reject} at        9            -1   %p-val=0
\put   {\reject} at        9      -0.66667   %p-val=0
\put   {\reject} at        9      -0.33333   %p-val=0
\put   {\reject} at        9             0   %p-val=0
\put   {\reject} at        9       0.33333   %p-val=0
\put   {\reject} at        9       0.66667   %p-val=0
\put   {\reject} at        9             1   %p-val=0
\put   {\reject} at        9        1.3333   %p-val=0
%%%%%%%%%%%% Plot figure in row 3 col 0 %%%%%%%%%%%%
\setcoordinatesystem units <0.125000in,0.217294in> point at 17.000000 18.257727
\setplotarea x from 1.000000 to 9.000000, y from -1.000000 to 1.301030
\axis bottom shiftedto y=-1.230103 label {$\tau_0$}
   ticks withvalues 2.5 5 7.5 /
     at 2.500000 5.000000 7.500000 / /
\axis left shiftedto x=0.600000 label {$\theta_1$}
   ticks withvalues 0.1 1 10 /
     at -1.000000 0.000000 1.000000 / /
\plotheading{\small$10^{6}$-fold growth}
%                        tau log10[theta0]
\put   {\reject} at        1            -1   %p-val=0
\put   {\reject} at        1      -0.66667   %p-val=0
\put   {\reject} at        1      -0.33333   %p-val=0
\put   {\reject} at        1             0   %p-val=0
\put   {\reject} at        1       0.33333   %p-val=0
\put   {\reject} at        1       0.66667   %p-val=0
\put   {\reject} at        1             1   %p-val=0
\put   {\reject} at        1        1.3333   %p-val=0
\put   {\reject} at        2            -1   %p-val=0
\put   {\reject} at        2      -0.66667   %p-val=0
\put   {\reject} at        2      -0.33333   %p-val=0
\put   {\reject} at        2             0   %p-val=0
\put   {\reject} at        2       0.33333   %p-val=0
\put   {\reject} at        2       0.66667   %p-val=0
\put   {\reject} at        2             1   %p-val=0
\put   {\reject} at        2        1.3333   %p-val=0
\put   {\reject} at        3            -1   %p-val=0
\put   {\reject} at        3      -0.66667   %p-val=0.002
\put   {\reject} at        3      -0.33333   %p-val=0.003
\put    {\99pct} at        3             0   %p-val=0.024
\put    {\99pct} at        3       0.33333   %p-val=0.033
\put    {\99pct} at        3       0.66667   %p-val=0.019
\put   {\reject} at        3             1   %p-val=0
\put   {\reject} at        3        1.3333   %p-val=0
\put   {\accept} at        4            -1   %p-val=0.056
\put   {\accept} at        4      -0.66667   %p-val=0.104
\put   {\accept} at        4      -0.33333   %p-val=0.389
\put   {\accept} at        4             0   %p-val=0.452
\put   {\accept} at        4       0.33333   %p-val=0.077
\put   {\reject} at        4       0.66667   %p-val=0.001
\put   {\reject} at        4             1   %p-val=0
\put   {\reject} at        4        1.3333   %p-val=0
\put   {\accept} at        5            -1   %p-val=0.302
\put   {\accept} at        5      -0.66667   %p-val=0.259
\put   {\accept} at        5      -0.33333   %p-val=0.186
\put   {\accept} at        5             0   %p-val=0.078
\put    {\99pct} at        5       0.33333   %p-val=0.01
\put   {\reject} at        5       0.66667   %p-val=0
\put   {\reject} at        5             1   %p-val=0
%           ERROR          5        1.3333    p-val=-2
\put   {\reject} at        6            -1   %p-val=0
\put   {\reject} at        6      -0.66667   %p-val=0.003
\put   {\reject} at        6      -0.33333   %p-val=0.007
\put   {\reject} at        6             0   %p-val=0
\put   {\reject} at        6       0.33333   %p-val=0
\put   {\reject} at        6       0.66667   %p-val=0
\put   {\reject} at        6             1   %p-val=0
%           ERROR          6        1.3333    p-val=-2
\put   {\reject} at        7            -1   %p-val=0
\put   {\reject} at        7      -0.66667   %p-val=0
\put   {\reject} at        7      -0.33333   %p-val=0
\put   {\reject} at        7             0   %p-val=0
\put   {\reject} at        7       0.33333   %p-val=0
\put   {\reject} at        7       0.66667   %p-val=0
\put   {\reject} at        7             1   %p-val=0
%           ERROR          7        1.3333    p-val=-2
\put   {\reject} at        8            -1   %p-val=0
\put   {\reject} at        8      -0.66667   %p-val=0
\put   {\reject} at        8      -0.33333   %p-val=0
\put   {\reject} at        8             0   %p-val=0
\put   {\reject} at        8       0.33333   %p-val=0
\put   {\reject} at        8       0.66667   %p-val=0
%           ERROR          8             1    p-val=-2
\put   {\reject} at        8        1.3333   %p-val=0
\put   {\reject} at        9            -1   %p-val=0
\put   {\reject} at        9      -0.66667   %p-val=0
\put   {\reject} at        9      -0.33333   %p-val=0
\put   {\reject} at        9             0   %p-val=0
\put   {\reject} at        9       0.33333   %p-val=0
\put   {\reject} at        9       0.66667   %p-val=0
\put   {\reject} at        9             1   %p-val=0
%           ERROR          9        1.3333    p-val=-2
%%%%%%%%%%%% Plot figure in row 3 col 1 %%%%%%%%%%%%
\setcoordinatesystem units <0.125000in,0.217294in> point at 1.000000 18.257727
\setplotarea x from 1.000000 to 9.000000, y from -1.000000 to 1.301030
\axis bottom shiftedto y=-1.230103 label {$\tau_0$}
   ticks withvalues 2.5 5 7.5 /
     at 2.500000 5.000000 7.500000 / /
\axis left shiftedto x=0.600000 label {$\theta_1$}
   ticks withvalues 0.1 1 10 /
     at -1.000000 0.000000 1.000000 / /
\plotheading{\small$10^{7}$-fold growth}
%                        tau log10[theta0]
\put   {\reject} at        1            -1   %p-val=0
\put   {\reject} at        1      -0.66667   %p-val=0
\put   {\reject} at        1      -0.33333   %p-val=0
\put   {\reject} at        1             0   %p-val=0
\put   {\reject} at        1       0.33333   %p-val=0
\put   {\reject} at        1       0.66667   %p-val=0
\put   {\reject} at        1             1   %p-val=0
\put   {\reject} at        1        1.3333   %p-val=0
\put   {\reject} at        2            -1   %p-val=0
\put   {\reject} at        2      -0.66667   %p-val=0
\put   {\reject} at        2      -0.33333   %p-val=0
\put   {\reject} at        2             0   %p-val=0
\put   {\reject} at        2       0.33333   %p-val=0
\put   {\reject} at        2       0.66667   %p-val=0
\put   {\reject} at        2             1   %p-val=0
\put   {\reject} at        2        1.3333   %p-val=0
\put   {\reject} at        3            -1   %p-val=0
\put   {\reject} at        3      -0.66667   %p-val=0.001
\put   {\reject} at        3      -0.33333   %p-val=0.006
\put    {\99pct} at        3             0   %p-val=0.019
\put    {\99pct} at        3       0.33333   %p-val=0.043
\put    {\99pct} at        3       0.66667   %p-val=0.017
\put   {\reject} at        3             1   %p-val=0
\put   {\reject} at        3        1.3333   %p-val=0
\put    {\99pct} at        4            -1   %p-val=0.036
\put   {\accept} at        4      -0.66667   %p-val=0.115
\put   {\accept} at        4      -0.33333   %p-val=0.368
\put   {\accept} at        4             0   %p-val=0.568
\put   {\accept} at        4       0.33333   %p-val=0.086
\put   {\reject} at        4       0.66667   %p-val=0.002
\put   {\reject} at        4             1   %p-val=0
\put   {\reject} at        4        1.3333   %p-val=0
\put   {\accept} at        5            -1   %p-val=0.253
\put   {\accept} at        5      -0.66667   %p-val=0.218
\put   {\accept} at        5      -0.33333   %p-val=0.139
\put   {\accept} at        5             0   %p-val=0.055
\put   {\reject} at        5       0.33333   %p-val=0.008
\put   {\reject} at        5       0.66667   %p-val=0
\put   {\reject} at        5             1   %p-val=0
\put   {\reject} at        5        1.3333   %p-val=0
\put   {\reject} at        6            -1   %p-val=0
\put   {\reject} at        6      -0.66667   %p-val=0
\put   {\reject} at        6      -0.33333   %p-val=0.004
\put   {\reject} at        6             0   %p-val=0.003
\put   {\reject} at        6       0.33333   %p-val=0.001
\put   {\reject} at        6       0.66667   %p-val=0
\put   {\reject} at        6             1   %p-val=0
\put   {\reject} at        6        1.3333   %p-val=0
\put   {\reject} at        7            -1   %p-val=0
\put   {\reject} at        7      -0.66667   %p-val=0
\put   {\reject} at        7      -0.33333   %p-val=0
\put   {\reject} at        7             0   %p-val=0
\put   {\reject} at        7       0.33333   %p-val=0
\put   {\reject} at        7       0.66667   %p-val=0
\put   {\reject} at        7             1   %p-val=0
\put   {\reject} at        7        1.3333   %p-val=0
\put   {\reject} at        8            -1   %p-val=0
\put   {\reject} at        8      -0.66667   %p-val=0
\put   {\reject} at        8      -0.33333   %p-val=0
\put   {\reject} at        8             0   %p-val=0
\put   {\reject} at        8       0.33333   %p-val=0
\put   {\reject} at        8       0.66667   %p-val=0
\put   {\reject} at        8             1   %p-val=0
\put   {\reject} at        8        1.3333   %p-val=0
\put   {\reject} at        9            -1   %p-val=0
\put   {\reject} at        9      -0.66667   %p-val=0
\put   {\reject} at        9      -0.33333   %p-val=0
\put   {\reject} at        9             0   %p-val=0
\put   {\reject} at        9       0.33333   %p-val=0
\put   {\reject} at        9       0.66667   %p-val=0
\put   {\reject} at        9             1   %p-val=0
\put   {\reject} at        9        1.3333   %p-val=0
\endpicture}
\\}
\caption{An experiment designed to test the method for generating
  confidence regions.}
\label{fig.ci4} 
\small Rather than beginning with a real data set, this experiment
began with a simulated data set that was generated under the
assumptions that $N=147$, and $(\tau,\theta_0, \theta_1) = (4, 1,
500)$.  A confidence region was then generated as described in the text.
Open circles $(\circ)$ represent points outside the 95\% confidence
region; closed circles $(\bullet)$ represent points within.  Note that
the correct hypothesis $\{(\tau,\theta_0, \theta_1) = (4, 1,
500)\}$ falls inside the confidence region.
\end{figure}

\begin{figure}
{\centering
% -*-latex-*-
%InputFile = csw.ci
%RangeLog10Theta0: -1.000000 1.301030 0.333333 
%RangeGrowth: 0.000000 7.000000 1.000000 
%RangeTau: 4.000000 11.000000 1.000000 
%% Confidence Interval: PicTeX output %%
%% Definitions:
\def\hunit{0.05in}
\def\vunit{0.18in}
\def\accept{{$\bullet$}}
\def\99pct{{$\cdot$}}
\def\reject{{\footnotesize$\circ$}}
%% Use the following to hide the 1% region
\def\99pct{\reject}
%%Plots are 1.000000 inch wide and 0.500000 inches high
\mbox{\beginpicture
\headingtoplotskip=0.5\baselineskip
\valuestolabelleading=0.4\baselineskip
%%%%%%%%%%%% Plot figure in row 0 col 0 %%%%%%%%%%%%
\setcoordinatesystem units <\hunit,\vunit> point at 20.000000 0
\setplotarea x from 4.000000 to 11.000000, y from -1.000000 to 1.301030
\axis bottom shiftedto y=-1.230103 /
\axis left shiftedto x=3.650000 label {$\theta_0$}
   ticks withvalues 0.1 1 10 /
     at -1.000000 0.000000 1.000000 / /
\plotheading{\small$10^{0}\times$ growth}
%                        tau log10[theta0]
\put   {\reject} at      7.5            -1   %p-val=0
\put   {\reject} at      7.5      -0.66667   %p-val=0
\put   {\reject} at      7.5      -0.33333   %p-val=0
\put   {\reject} at      7.5             0   %p-val=0
\put   {\reject} at      7.5       0.33333   %p-val=0
\put   {\reject} at      7.5       0.66667   %p-val=0
\put   {\reject} at      7.5             1   %p-val=0
\put   {\reject} at      7.5        1.3333   %p-val=0
%%%%%%%%%%%% Plot figure in row 0 col 1 %%%%%%%%%%%%
\setcoordinatesystem units <\hunit,\vunit> point at 0.000000 0
\setplotarea x from 4.000000 to 11.000000, y from -1.000000 to 1.301030
\axis bottom shiftedto y=-1.230103 
   ticks numbered from 5 to 10 by 5 /
\axis right shiftedto x=11.350000 
   ticks withvalues 0.1 1 10 /
     at -1.000000 0.000000 1.000000 / /
\plotheading{\small$10^{1}\times$ growth}
%                        tau log10[theta0]
\put   {\reject} at        4            -1   %p-val=0
\put   {\reject} at        4      -0.66667   %p-val=0
\put   {\reject} at        4      -0.33333   %p-val=0
\put   {\reject} at        4             0   %p-val=0
\put   {\reject} at        4       0.33333   %p-val=0
\put   {\reject} at        4       0.66667   %p-val=0
%           ERROR          4             1    p-val=-2
%           ERROR          4        1.3333    p-val=-2
\put   {\reject} at        5            -1   %p-val=0
\put   {\reject} at        5      -0.66667   %p-val=0
\put   {\reject} at        5      -0.33333   %p-val=0
\put   {\reject} at        5             0   %p-val=0
\put   {\reject} at        5       0.33333   %p-val=0
\put   {\reject} at        5       0.66667   %p-val=0
\put   {\reject} at        5             1   %p-val=0
\put   {\reject} at        5        1.3333   %p-val=0
\put   {\reject} at        6            -1   %p-val=0
\put   {\reject} at        6      -0.66667   %p-val=0
\put   {\reject} at        6      -0.33333   %p-val=0
\put   {\reject} at        6             0   %p-val=0
\put   {\reject} at        6       0.33333   %p-val=0
\put   {\reject} at        6       0.66667   %p-val=0
%           ERROR          6             1    p-val=-2
\put   {\reject} at        6        1.3333   %p-val=0
\put   {\reject} at        7            -1   %p-val=0
\put   {\reject} at        7      -0.66667   %p-val=0
\put   {\reject} at        7      -0.33333   %p-val=0
\put   {\reject} at        7             0   %p-val=0
\put   {\reject} at        7       0.33333   %p-val=0
\put   {\reject} at        7       0.66667   %p-val=0
%           ERROR          7             1    p-val=-2
\put   {\reject} at        7        1.3333   %p-val=0
\put   {\reject} at        8            -1   %p-val=0
\put   {\reject} at        8      -0.66667   %p-val=0
\put   {\reject} at        8      -0.33333   %p-val=0
\put   {\reject} at        8             0   %p-val=0
\put   {\reject} at        8       0.33333   %p-val=0
%           ERROR          8       0.66667    p-val=-2
\put   {\reject} at        8             1   %p-val=0
\put   {\reject} at        8        1.3333   %p-val=0
\put   {\reject} at        9            -1   %p-val=0
\put   {\reject} at        9      -0.66667   %p-val=0
\put   {\reject} at        9      -0.33333   %p-val=0
\put   {\reject} at        9             0   %p-val=0
%           ERROR          9       0.33333    p-val=-2
%           ERROR          9       0.66667    p-val=-2
\put   {\reject} at        9             1   %p-val=0
\put   {\reject} at        9        1.3333   %p-val=0
\put   {\reject} at       10            -1   %p-val=0
\put   {\reject} at       10      -0.66667   %p-val=0
\put   {\reject} at       10      -0.33333   %p-val=0
\put   {\reject} at       10             0   %p-val=0
%           ERROR         10       0.33333    p-val=-2
\put   {\reject} at       10       0.66667   %p-val=0
\put   {\reject} at       10             1   %p-val=0
\put   {\reject} at       10        1.3333   %p-val=0
\put   {\reject} at       11            -1   %p-val=0
\put   {\reject} at       11      -0.66667   %p-val=0
\put   {\reject} at       11      -0.33333   %p-val=0
\put   {\reject} at       11             0   %p-val=0
\put   {\reject} at       11       0.33333   %p-val=0
\put   {\reject} at       11       0.66667   %p-val=0
\put   {\reject} at       11             1   %p-val=0
\put   {\reject} at       11        1.3333   %p-val=0
%%%%%%%%%%%% Plot figure in row 1 col 0 %%%%%%%%%%%%
\setcoordinatesystem units <\hunit,\vunit> point at 20.000000 6
\setplotarea x from 4.000000 to 11.000000, y from -1.000000 to 1.301030
\axis bottom shiftedto y=-1.230103 
   ticks numbered from 5 to 10 by 5 /
\axis left shiftedto x=3.650000 label {$\theta_0$}
   ticks withvalues 0.1 1 10 /
     at -1.000000 0.000000 1.000000 / /
\plotheading{\small$10^{2}\times$ growth}
%                        tau log10[theta0]
\put   {\reject} at        4            -1   %p-val=0
\put   {\reject} at        4      -0.66667   %p-val=0
\put   {\reject} at        4      -0.33333   %p-val=0
\put   {\reject} at        4             0   %p-val=0
\put   {\reject} at        4       0.33333   %p-val=0
\put   {\reject} at        4       0.66667   %p-val=0.004
\put   {\reject} at        4             1   %p-val=0.005
\put   {\reject} at        4        1.3333   %p-val=0
\put   {\reject} at        5            -1   %p-val=0
\put   {\reject} at        5      -0.66667   %p-val=0
\put   {\reject} at        5      -0.33333   %p-val=0
\put   {\reject} at        5             0   %p-val=0
\put   {\reject} at        5       0.33333   %p-val=0
\put    {\99pct} at        5       0.66667   %p-val=0.021
\put   {\reject} at        5             1   %p-val=0.002
\put   {\reject} at        5        1.3333   %p-val=0.001
\put   {\reject} at        6            -1   %p-val=0
\put   {\reject} at        6      -0.66667   %p-val=0
\put   {\reject} at        6      -0.33333   %p-val=0
\put   {\reject} at        6             0   %p-val=0
\put    {\99pct} at        6       0.33333   %p-val=0.016
\put    {\99pct} at        6       0.66667   %p-val=0.026
\put   {\reject} at        6             1   %p-val=0
\put   {\reject} at        6        1.3333   %p-val=0
\put   {\reject} at        7            -1   %p-val=0
\put   {\reject} at        7      -0.66667   %p-val=0
\put   {\reject} at        7      -0.33333   %p-val=0
\put   {\reject} at        7             0   %p-val=0
\put    {\99pct} at        7       0.33333   %p-val=0.012
\put    {\99pct} at        7       0.66667   %p-val=0.014
\put   {\reject} at        7             1   %p-val=0.001
\put   {\reject} at        7        1.3333   %p-val=0
\put   {\reject} at        8            -1   %p-val=0
\put   {\reject} at        8      -0.66667   %p-val=0
\put   {\reject} at        8      -0.33333   %p-val=0
\put   {\reject} at        8             0   %p-val=0
\put   {\reject} at        8       0.33333   %p-val=0.003
\put   {\reject} at        8       0.66667   %p-val=0.004
%           ERROR          8             1    p-val=-2
%           ERROR          8        1.3333    p-val=-2
\put   {\reject} at        9            -1   %p-val=0
\put   {\reject} at        9      -0.66667   %p-val=0
\put   {\reject} at        9      -0.33333   %p-val=0
\put   {\reject} at        9             0   %p-val=0
\put   {\reject} at        9       0.33333   %p-val=0.001
\put   {\reject} at        9       0.66667   %p-val=0.002
%           ERROR          9             1    p-val=-2
\put   {\reject} at        9        1.3333   %p-val=0
\put   {\reject} at       10            -1   %p-val=0
%           ERROR         10      -0.66667    p-val=-2
\put   {\reject} at       10      -0.33333   %p-val=0
\put   {\reject} at       10             0   %p-val=0
\put   {\reject} at       10       0.33333   %p-val=0
\put   {\reject} at       10       0.66667   %p-val=0.002
\put   {\reject} at       10             1   %p-val=0.001
\put   {\reject} at       10        1.3333   %p-val=0
\put   {\reject} at       11            -1   %p-val=0
%           ERROR         11      -0.66667    p-val=-2
%           ERROR         11      -0.33333    p-val=-2
\put   {\reject} at       11             0   %p-val=0
\put   {\reject} at       11       0.33333   %p-val=0
\put   {\reject} at       11       0.66667   %p-val=0
%           ERROR         11             1    p-val=-2
\put   {\reject} at       11        1.3333   %p-val=0
%%%%%%%%%%%% Plot figure in row 1 col 1 %%%%%%%%%%%%
\setcoordinatesystem units <\hunit,\vunit> point at 0.000000 6
\setplotarea x from 4.000000 to 11.000000, y from -1.000000 to 1.301030
\axis bottom shiftedto y=-1.230103 
   ticks numbered from 5 to 10 by 5 /
\axis right shiftedto x=11.350000 
   ticks withvalues 0.1 1 10 /
     at -1.000000 0.000000 1.000000 / /
\plotheading{\small$10^{3}\times$ growth}
%                        tau log10[theta0]
\put   {\reject} at        4            -1   %p-val=0
\put   {\reject} at        4      -0.66667   %p-val=0
\put   {\reject} at        4      -0.33333   %p-val=0
\put   {\reject} at        4             0   %p-val=0
\put   {\reject} at        4       0.33333   %p-val=0
\put    {\99pct} at        4       0.66667   %p-val=0.011
\put   {\reject} at        4             1   %p-val=0.005
\put   {\reject} at        4        1.3333   %p-val=0
\put   {\reject} at        5            -1   %p-val=0
\put   {\reject} at        5      -0.66667   %p-val=0
\put   {\reject} at        5      -0.33333   %p-val=0
\put   {\reject} at        5             0   %p-val=0.001
\put    {\99pct} at        5       0.33333   %p-val=0.048
\put   {\accept} at        5       0.66667   %p-val=0.056
\put   {\reject} at        5             1   %p-val=0.008
\put   {\reject} at        5        1.3333   %p-val=0
\put   {\reject} at        6            -1   %p-val=0
\put   {\reject} at        6      -0.66667   %p-val=0
\put   {\reject} at        6      -0.33333   %p-val=0.004
\put   {\accept} at        6             0   %p-val=0.058
\put   {\accept} at        6       0.33333   %p-val=0.215
\put   {\accept} at        6       0.66667   %p-val=0.06
\put   {\reject} at        6             1   %p-val=0.002
\put   {\reject} at        6        1.3333   %p-val=0
\put   {\reject} at        7            -1   %p-val=0
\put   {\reject} at        7      -0.66667   %p-val=0.006
\put   {\accept} at        7      -0.33333   %p-val=0.111
\put   {\accept} at        7             0   %p-val=0.267
\put   {\accept} at        7       0.33333   %p-val=0.132
\put    {\99pct} at        7       0.66667   %p-val=0.041
\put   {\reject} at        7             1   %p-val=0
\put   {\reject} at        7        1.3333   %p-val=0
\put   {\reject} at        8            -1   %p-val=0
\put    {\99pct} at        8      -0.66667   %p-val=0.035
\put   {\accept} at        8      -0.33333   %p-val=0.102
\put   {\accept} at        8             0   %p-val=0.059
\put    {\99pct} at        8       0.33333   %p-val=0.015
\put    {\99pct} at        8       0.66667   %p-val=0.041
\put   {\reject} at        8             1   %p-val=0.007
\put   {\reject} at        8        1.3333   %p-val=0.001
\put   {\reject} at        9            -1   %p-val=0
\put   {\reject} at        9      -0.66667   %p-val=0.006
\put    {\99pct} at        9      -0.33333   %p-val=0.014
\put   {\reject} at        9             0   %p-val=0.002
\put   {\reject} at        9       0.33333   %p-val=0
\put    {\99pct} at        9       0.66667   %p-val=0.046
%           ERROR          9             1    p-val=-2
%           ERROR          9        1.3333    p-val=-2
\put   {\reject} at       10            -1   %p-val=0
\put   {\reject} at       10      -0.66667   %p-val=0.001
\put   {\reject} at       10      -0.33333   %p-val=0.001
\put   {\reject} at       10             0   %p-val=0
\put   {\reject} at       10       0.33333   %p-val=0
\put   {\reject} at       10       0.66667   %p-val=0
\put   {\reject} at       10             1   %p-val=0.008
\put   {\reject} at       10        1.3333   %p-val=0
\put   {\reject} at       11            -1   %p-val=0
\put   {\reject} at       11      -0.66667   %p-val=0
\put   {\reject} at       11      -0.33333   %p-val=0
\put   {\reject} at       11             0   %p-val=0
\put   {\reject} at       11       0.33333   %p-val=0
\put   {\reject} at       11       0.66667   %p-val=0
\put    {\99pct} at       11             1   %p-val=0.013
\put   {\reject} at       11        1.3333   %p-val=0
%%%%%%%%%%%% Plot figure in row 2 col 0 %%%%%%%%%%%%
\setcoordinatesystem units <\hunit,\vunit> point at 20.000000 12
\setplotarea x from 4.000000 to 11.000000, y from -1.000000 to 1.301030
\axis bottom shiftedto y=-1.230103 label {$\tau$}
   ticks numbered from 5 to 10 by 5 /
\axis left shiftedto x=3.650000 label {$\theta_0$}
   ticks withvalues 0.1 1 10 /
     at -1.000000 0.000000 1.000000 / /
\plotheading{\small$10^{4}\times$ growth}
%                        tau log10[theta0]
\put   {\reject} at        4            -1   %p-val=0
\put   {\reject} at        4      -0.66667   %p-val=0
\put   {\reject} at        4      -0.33333   %p-val=0
\put   {\reject} at        4             0   %p-val=0
\put   {\reject} at        4       0.33333   %p-val=0.001
\put    {\99pct} at        4       0.66667   %p-val=0.011
\put   {\reject} at        4             1   %p-val=0.003
\put   {\reject} at        4        1.3333   %p-val=0
\put   {\reject} at        5            -1   %p-val=0
\put   {\reject} at        5      -0.66667   %p-val=0
\put   {\reject} at        5      -0.33333   %p-val=0
\put   {\reject} at        5             0   %p-val=0.001
\put   {\accept} at        5       0.33333   %p-val=0.08
\put   {\accept} at        5       0.66667   %p-val=0.068
%           ERROR          5             1    p-val=-2
\put   {\reject} at        5        1.3333   %p-val=0
\put   {\reject} at        6            -1   %p-val=0
\put   {\reject} at        6      -0.66667   %p-val=0.002
\put    {\99pct} at        6      -0.33333   %p-val=0.017
\put   {\accept} at        6             0   %p-val=0.076
\put   {\accept} at        6       0.33333   %p-val=0.23
\put    {\99pct} at        6       0.66667   %p-val=0.049
%           ERROR          6             1    p-val=-2
\put   {\reject} at        6        1.3333   %p-val=0
\put    {\99pct} at        7            -1   %p-val=0.04
\put   {\accept} at        7      -0.66667   %p-val=0.093
\put   {\accept} at        7      -0.33333   %p-val=0.213
\put   {\accept} at        7             0   %p-val=0.282
\put   {\accept} at        7       0.33333   %p-val=0.192
\put    {\99pct} at        7       0.66667   %p-val=0.044
\put   {\reject} at        7             1   %p-val=0.002
\put   {\reject} at        7        1.3333   %p-val=0.001
\put   {\accept} at        8            -1   %p-val=0.199
\put   {\accept} at        8      -0.66667   %p-val=0.127
\put   {\accept} at        8      -0.33333   %p-val=0.096
\put    {\99pct} at        8             0   %p-val=0.044
\put   {\reject} at        8       0.33333   %p-val=0.007
\put    {\99pct} at        8       0.66667   %p-val=0.048
\put   {\reject} at        8             1   %p-val=0.004
\put   {\reject} at        8        1.3333   %p-val=0.001
\put    {\99pct} at        9            -1   %p-val=0.013
\put    {\99pct} at        9      -0.66667   %p-val=0.013
\put   {\reject} at        9      -0.33333   %p-val=0.005
\put   {\reject} at        9             0   %p-val=0.002
\put   {\reject} at        9       0.33333   %p-val=0
\put   {\reject} at        9       0.66667   %p-val=0.001
\put    {\99pct} at        9             1   %p-val=0.013
\put   {\reject} at        9        1.3333   %p-val=0
\put   {\reject} at       10            -1   %p-val=0
\put   {\reject} at       10      -0.66667   %p-val=0.001
\put   {\reject} at       10      -0.33333   %p-val=0.002
\put   {\reject} at       10             0   %p-val=0
\put   {\reject} at       10       0.33333   %p-val=0
\put   {\reject} at       10       0.66667   %p-val=0
\put   {\reject} at       10             1   %p-val=0.006
%           ERROR         10        1.3333    p-val=-2
\put   {\reject} at       11            -1   %p-val=0
\put   {\reject} at       11      -0.66667   %p-val=0
\put   {\reject} at       11      -0.33333   %p-val=0
\put   {\reject} at       11             0   %p-val=0
\put   {\reject} at       11       0.33333   %p-val=0
\put   {\reject} at       11       0.66667   %p-val=0
\put    {\99pct} at       11             1   %p-val=0.01
\put   {\reject} at       11        1.3333   %p-val=0
%%%%%%%%%%%% Plot figure in row 2 col 1 %%%%%%%%%%%%
\setcoordinatesystem units <\hunit,\vunit> point at 0.000000 12
\setplotarea x from 4.000000 to 11.000000, y from -1.000000 to 1.301030
\axis bottom shiftedto y=-1.230103 label {$\tau$}
   ticks numbered from 5 to 10 by 5 /
\axis right shiftedto x=11.350000 
   ticks withvalues 0.1 1 10 /
     at -1.000000 0.000000 1.000000 / /
\plotheading{\small$10^{5}\times$ growth}
%                        tau log10[theta0]
\put   {\reject} at        4            -1   %p-val=0
\put   {\reject} at        4      -0.66667   %p-val=0
\put   {\reject} at        4      -0.33333   %p-val=0
\put   {\reject} at        4             0   %p-val=0
\put   {\reject} at        4       0.33333   %p-val=0
\put    {\99pct} at        4       0.66667   %p-val=0.016
\put   {\reject} at        4             1   %p-val=0.005
\put   {\reject} at        4        1.3333   %p-val=0
\put   {\reject} at        5            -1   %p-val=0
\put   {\reject} at        5      -0.66667   %p-val=0
\put   {\reject} at        5      -0.33333   %p-val=0
\put   {\reject} at        5             0   %p-val=0.003
\put   {\accept} at        5       0.33333   %p-val=0.064
\put   {\accept} at        5       0.66667   %p-val=0.061
%           ERROR          5             1    p-val=-2
\put   {\reject} at        5        1.3333   %p-val=0.001
\put   {\reject} at        6            -1   %p-val=0
\put   {\reject} at        6      -0.66667   %p-val=0.004
\put   {\reject} at        6      -0.33333   %p-val=0.009
\put   {\accept} at        6             0   %p-val=0.094
\put   {\accept} at        6       0.33333   %p-val=0.244
\put   {\accept} at        6       0.66667   %p-val=0.069
\put   {\reject} at        6             1   %p-val=0.004
\put   {\reject} at        6        1.3333   %p-val=0
\put    {\99pct} at        7            -1   %p-val=0.041
\put   {\accept} at        7      -0.66667   %p-val=0.126
\put   {\accept} at        7      -0.33333   %p-val=0.257
\put   {\accept} at        7             0   %p-val=0.278
\put   {\accept} at        7       0.33333   %p-val=0.16
\put   {\accept} at        7       0.66667   %p-val=0.055
\put   {\reject} at        7             1   %p-val=0.005
\put   {\reject} at        7        1.3333   %p-val=0
\put   {\accept} at        8            -1   %p-val=0.152
\put   {\accept} at        8      -0.66667   %p-val=0.141
\put   {\accept} at        8      -0.33333   %p-val=0.067
\put    {\99pct} at        8             0   %p-val=0.034
\put   {\reject} at        8       0.33333   %p-val=0.009
\put   {\accept} at        8       0.66667   %p-val=0.051
\put   {\reject} at        8             1   %p-val=0.004
\put   {\reject} at        8        1.3333   %p-val=0
\put   {\reject} at        9            -1   %p-val=0.004
\put   {\reject} at        9      -0.66667   %p-val=0.003
\put   {\reject} at        9      -0.33333   %p-val=0.007
\put   {\reject} at        9             0   %p-val=0.001
\put   {\reject} at        9       0.33333   %p-val=0
\put   {\reject} at        9       0.66667   %p-val=0.005
\put   {\reject} at        9             1   %p-val=0.006
\put   {\reject} at        9        1.3333   %p-val=0
\put   {\reject} at       10            -1   %p-val=0
\put   {\reject} at       10      -0.66667   %p-val=0.001
\put   {\reject} at       10      -0.33333   %p-val=0
\put   {\reject} at       10             0   %p-val=0
\put   {\reject} at       10       0.33333   %p-val=0
\put   {\reject} at       10       0.66667   %p-val=0
%           ERROR         10             1    p-val=-2
\put   {\reject} at       10        1.3333   %p-val=0.001
\put   {\reject} at       11            -1   %p-val=0
\put   {\reject} at       11      -0.66667   %p-val=0
\put   {\reject} at       11      -0.33333   %p-val=0
\put   {\reject} at       11             0   %p-val=0
\put   {\reject} at       11       0.33333   %p-val=0
\put   {\reject} at       11       0.66667   %p-val=0
\put    {\99pct} at       11             1   %p-val=0.01
\put   {\reject} at       11        1.3333   %p-val=0
\endpicture}
\\}
\caption{95\% confidence region for the Cann-Stoneking-Wilson data}
\label{fig.cswci} 
\small Open circles $(\circ)$ represent points outside the 95\%
confidence region; closed circles $(\bullet)$ represent points within.
The marginal confidence regions are $\tau: [5,8]$, $\theta_0: [0,
4.6]$, and $\theta_1/\theta_0: (100, \infty]$.  If the mutation rate
is $u = 0.001$ per sequence per generation, and generations are 25
years, these intervals become $t: [62, 100]$ thousand years, and $N_0:
[0, 2320]$ females.
\end{figure}

\end{subappendices}
