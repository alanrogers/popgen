%-*-latex-*-
\pagenumbering{arabic}
\chapter{Allele Frequencies in Subdivided Populations}
\label{ch.genefrq}

This project will give you a little hands on experience in dealing
with genetic data.  You will get practice in (1)~computing gene
frequencies from genotype frequencies, (2)~using the Hardy-Weinberg
principle to predict what genotype frequencies would be under random
mating, (3)~comparing observed with predicted genotype frequencies,
and (4)~figuring out what such comparisons tell us about the
populations involved.  Before starting, you should read Hartl's
chapter~1.  Before explaining the assignment, I will go over some
things that you will have to understand in order to do it.

\begin{table}
\centering
\caption{Transferrin Data from 100 Baboons \citep[p.~56]{Jolly:PAA-??}}
\label{tab.PollyJog}
\begin{tabular}{|rrrr|}
\multicolumn{4}{c}{Genotype counts}\\
\hline
        &\multicolumn{3}{c|}{Number of}\\ \cline{2-4}
Genotype& individuals &$C$ genes&$D$ genes\\ \hline\hline
$CC$    &          80 &      160&       0\\
$CD$    &          15 &       15&      15\\
$DD$    &           5 &        0&      10\\ \hline
Total   &         100 &      175&      25\\ \hline
\end{tabular}\\[12pt]
%%%%%%%%%%%%%%%%%%%%%%%%%%%%%%%%%%%%%%%%%%%%%%%%%%%%%%%%%%%%%%%%
\begin{tabular}{|rr@{$\;=\;$}c@{$\;=\;$}l|}
\multicolumn{4}{c}{Genotype Frequencies}\\
\hline
Genotype & \multicolumn{3}{c|}{Frequency}\\ 
\hline\hline 
$CC$ &$P$&80/100&0.80\\
$CD$ &$H$&15/100&0.15\\ 
$DD$ &$Q$& 5/100&0.05\\ 
\hline
\hline
\multicolumn{4}{|c|}{\small Note: $P+H+Q=1$}\\
\hline
\end{tabular}\\[12pt]
%%%%%%%%%%%%%%%%%%%%%%%%%%%%%%%%%%%%%%%%%%%%%%%%%%%%%%%%%%%%%%%%
\begin{tabular}{|rr@{$\;=\;$}c@{$\;=\;$}l|}
\multicolumn{4}{c}{Allele Frequencies}\\
\hline
Allele & \multicolumn{3}{c|}{Frequency}\\ 
\hline\hline 
$C$ &$p$&175/200&0.875\\
$D$ &$q$&25/200&0.125\\
\hline
\hline
\multicolumn{4}{|c|}{\small Notes}\\
\multicolumn{4}{|c|}{\small $p+q=1$}\\
\multicolumn{4}{|c|}{\small $p=P+H/2$}\\
\multicolumn{4}{|c|}{\small $q=Q+H/2$}\\
\hline
\end{tabular}\\[12pt]
\end{table}

\section{Genotype frequencies}

The table~\ref{tab.PollyJog} shows the number of individuals in a
baboon troop, broken down by genotype at the transferrin locus.
This table indicates that 15 of the baboons are heterozygotes: they have
the CD genotype.  It also shows that the troop contains 100 individuals in
all.  The frequency of the genotype CD is 15/100 -- simply the fraction of
the troop composed of CD heterozygotes.  Similarly, the frequencies of the
other genotypes are 80/100 and 5/100.  This is what I mean by the term
\emph{genotype frequency}.

\section{Heterozygosity}

The frequency of heterozygotes in a population is called its
\emph{heterozygosity}.   The heterozygosity of the baboon troop is 
15/100 (or 0.15) at the transferrin locus.  We could also compute
heterozygosity using other loci, if we had the data.  Geneticists
often do this and average the results to obtain an overall measure of
heterozygosity.

\section{Allele frequencies}

From the table above it is clear that there are 175 (=160+15) copies
of the C allele in this baboon troop, and that there are 200 genes in
all.  Thus, the frequency of C is 175/200 = 0.875.  This is an
\emph{gene frequency}. It would be better to call it an \emph{allele
frequency,} but the other term is firmly entrenched in the literature.
If you want, you can always construct a table like the one above when
you want to calculate a gene frequency, but there is an easier way.
Here is a formula that you may find helpful:
\[
p = N_{AA}/N + N_{Aa}/2N = P + H/2
\]
Here $p$ stands for the frequency of the A allele, $N_{AA}$ is the number of 
AA individuals, $N_{Aa}$ is the number of Aa individuals, and $N$ is the 
population size.   If you plug in the values for the baboon troop, this is
\begin{eqnarray*}
p &=& \frac{2\times 80+15}{2\times 100}\\
  &=& 175/200
\end{eqnarray*}
This is the same calculation we did before, and I find this method easier than
always making tables.  If you don't, don't feel obliged to use it.  You can
always make tables.

\section{The Hardy-Weinberg principle}

When the individuals in a population mate at random, the genotype
frequencies among zygotes can be predicted from gene frequencies alone.
If, at some locus, there are two alleles (A \& a) with frequencies $p$ and
$q=1-p$, then the Hardy-Weinberg principle predicts that the genotype
frequencies will be
\begin{center}
\begin{tabular}{ccc}
AA   &      Aa   &         aa\\

$p^2$  &  $2pq$   &        $q^2$\\
\end{tabular}
\end{center}

\section{The data}

You will be analyzing genetic data for the MN locus in the human population
of Dehi, India. They were taken from a paper by Das et.\ al.\ in the
\emph{Annals of Human Genetics} (Vol. 5, pp. 25-31). As many of you
probably know, Indian society is highly stratified.  It is divided
into numerous groups, called castes, between which there is little
intermarriage.  Das et. al. sampled about 100 individuals in each of
five castes.  Their sample is broken down by caste and genotype in the
table below.  As we shall see, these data have a story to tell about
the population of Delhi.
\begin{center}
\begin{tabular}{lrrrr}
\hline
Caste       &    MM  &    MN  &    NN &     Total\\
\hline\hline
Rajput      &    29  &    16  &    55 &     100\\
Chamar      &    32  &    13  &    56 &     101\\
Guijjar     &    40  &    13  &    49 &     102\\
Ahir        &    45  &     4  &    52 &     101\\
Jat         &    40  &    13  &    52 &     105\\
\hline
Total Pop   &   186  &   59   &  264  &    509\\
\hline
\end{tabular}
\end{center}

\section{The exercise}

\subsection{Step 1}

Convert these counts into genotype frequencies and gene frequencies. You
should do this separately for each caste.  Then use the last row to 
compute gene and genotype frequencies for the population as a
whole.  Make a table like the one above with 
three columns for genotype frequencies and one for the frequency
of the M allele. Leave plenty of room for additional columns.

\subsection{Step 2}

What would the genotype frequencies be at Hardy-Weinberg equilibrium?
Calculate expected frequencies for each genotype and caste, and also
for the total population.  Add this information to your table.

Recall that the word ``heterozygosity'' refers to the frequency of
heterozygotes.  You have now computed both the observed heterozygosity in
each group and that expected at Hardy-Weinberg equilibrium.

\section{Step 3}

In this step we will see how heterozygosity is affected by subdividing
the population into castes.  In your table you should now have a
column of Hardy-Weinberg predictions of heterozygosity.  The
prediction for the whole population is the heterozygosity we would
expect if the entire population were mating at random.  Let us call
this number $H_t$ (the $t$ stands for "total").  The actual
heterozygosity of the population will differ from $H_t$ for two
reasons: (1)~the population is subdivided into castes, and (2)~the
castes themselves may not be at HW equilibrium.  To investigate the
first of these factors, find the average of the predicted
heterozygosities of the five castes.  This is the heterozygosity we
would expect if each caste were an independent, random-mating
population, and I will call it $H_s$ (the $s$ stands for
"subdivision").  You will find that $H_s$ is smaller than $H_t$,
unless there is a mistake in your arithmetic.  This is \emph{always}
true in subdivided populations, and is an example of what is called
the Wahlund principle: subdivision always reduces heterozygosity.
When the gene frequencies in the various groups differ a lot,
subdivision causes a large reduction in heterozygosity.  In this
example the reduction is small because the castes have similar gene
frequencies.

The observed heterozygosity of the population is called $H_o$, and you
have already computed it: it should be the last item in the second column
of your table.  It is reasonable to expect that, since heterozygosity has
been reduced by subdivision, $H_o$ will be smaller than $H_t$.  Is 
this so with the data used here?

Inbreeding produces an excess of homozygotes and a deficit of
heterozygotes.  What factors could produce an excess of heterozygotes?
Here are a few possibilities.
\begin{description}
\item[Negative assortative mating]
If MM genotypes prefer to mate with NN genotypes, an excess of
heterozygotes would result.  But few people even know their spouses
genotype at the MN locus, so this explanation seems farfetched.

\item[Selective mortality]
If MM and NN genotypes tend to die early, we would find an excess of MN's.
But this pattern has not shown up in other studies, so this is unlikely to
be important with the data studied here.

\item[Outbreeding]
In most societies, people avoid marrying close kin, which tends to inflate
heterozygosity.  On the other hand, they also tend to marry neighbors, which
has the opposite effect.

\item[Migration]
When people marry immigrants from populations with different gene
frequencies, their children are likely to be heterozygotes.   We are told,
however, that there is almost no intermarriage between castes.

\item[Sampling error]
It is quite likely that the excess of heterozygotes is an artifact of
sampling error.  It is difficult to say much on the basis of
information from one locus.  If we got the same results with a sample
of 20 loci, I would be pretty curious about the marriage patterns in
these castes---perhaps they aren't as endogamous as everyone
thinks---and this might point the way to further research.
\end{description}

\section{Awk Tips}

We turned the data above into a file called das.dat that looks like
this:
\begin{verbatim}
#Caste          MM      MN      NN      Total
Rajput          29      16      55      100
Chamar          32      13      56      101
Guijjar         40      13      49      102
Ahir            45       4      52      101
Jat             40      13      52      105
\end{verbatim}
Then we created another file called play.awk that looks likt this:
\begin{verbatim}
# Omit comments
$1 ~ "#" {next};
# record names and accumulate sums of cols 2-4
{
  name[i++]=$1;
  h1  += $2; 
  h2 += $3;
  h3 += $4;
}
# executed at end of input
END{
# print names  
  for (i in name ) print i " " name[i];
# print sums
  print "Sums: "h1 " " h2 " " h3;
  
}
\end{verbatim}
Now typing ``awk -f play.awk das.dat'' produces
\begin{verbatim}
4 Jat
0 Rajput
1 Chamar
2 Guijjar
3 Ahir
Sums: 186 59 264
\end{verbatim}
