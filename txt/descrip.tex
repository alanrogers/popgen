%-*-latex-*-
\chapter{Descriptive Statistics for DNA Sequences}
\label{ch.descrip}

\section{DNA sequence data}
\label{sec.toySeqData}
Not until the 1980s did population geneticists begin the study of
DNA sequence data.  Until then, our measures of genetic
variation were incomplete.  We worked only with a small fraction of
the genetic variation in our samples.  With DNA sequence data we were
finally able to study it all.

But this opportunity posed an immediate challenge.  How should we
\emph{measure} that variation?  Population geneticists were used to
summarizing variation with statistics such as the sample
heterozygosity: the probability that two random gene copies are copies
of different alleles.  But if the DNA sequences are long enough, it is
unlikely that any two of them will be identical.  The heterozygosity,
in other words, is always~1.  Clearly, new measures of variation are
needed.

\begin{table}
  \caption{Ten DNA sequences, each consisting of 40 sites.
    The sites are numbered across the top.  The dots represent sites
    that are identical to the \emph{reference sequence} at the top.}
\label{tab.hyposeq}
\begin{verbatim}
             0000000001 1111111112 2222222223 3333333334
             1234567890 1234567890 1234567890 1234567890

Sequence01   AATATGGCAC CTCCCAACCC TCTAGCATAT ACCACTTACA
Sequence02   .......T.. .C......TG C......C.. ..........
Sequence03   ..C....... .......... .......... ..........
Sequence04   .......T.. .C......TG C......... G.........
Sequence05   .......... .......... .......... ..........
Sequence06   .....A.... ........T. C......... G....C....
Sequence07   ..C....T.. .C......TG C......... G.........
Sequence08   .....A.T.. TC......TG C......... G.........
Sequence09   .......... .......... C......... ..........
Sequence10   .G...A.... ........T. C......C.. .T....C..G
Segregating:  ^^  ^ ^   ^^      ^^ ^      ^   ^^   ^^  ^
\end{verbatim}
\end{table}

Table~\ref{tab.hyposeq} presents 10 DNA sequences from some
hypothetical species.  Take a minute to study them.  How many ways can
you think of to summarize the variation in these data?  This is
precisely the problem that confronted population geneticists during
the 1980s.  The lecture that follows will summarize some of the ideas
they came up with.

\section{Statistics}
\begin{description}
\item[Gene diversity (a.k.a. heterozygosity)] is the probability that
  two random sequences are different.  To calculate it, the
  straightforward approach is to examine all pairs and count the
  fraction of the pairs in which the two sequences are different from
  each other.  It is often faster, however, to start by counting the
  number of copies of each type in the data.  Let $k_i$ denote the
  number of copies of type $i$, and $K = \sum k_i$ the number of gene
  copies in the sample.  The the heterozygosity is estimated by
\[
H = 1 - \sum_i \left(\frac{k_i}{K}\right)
 \left(\frac{k_i - 1}{K - 1}\right)
\]
In the past, we have expressed heterozygosity as $2p(1-p)$ (for
bi-allelic loci) or as $1 - \sum_{i} p_i^2$ (for loci with multiple
alleles).  These formulas are correct when $p$ is the population
allele frequency of the parents but contain a subtle bias when $p$ is
the allele frequency within a sample.  The new formula corrects this
bias.\footnote{Imagine drawing two gene copies without replacement
  from a sample of size $K$.  The first is a copy of allele $A_i$ with
  probability $k_i/K$.  Given this, the second is a copy of $A_i$ with
  probability $(k_i-1)/(K-1)$.  Thus, the sum of these quantities is
  the homozygosity and 1 minus this sum is the heterozygosity.}
\item[Number of segregating sites] A ``segregating site'' is a site
that is polymorphic in the data.  The number of such sites is usually
denoted by $S$.  
\item[Mean pairwise difference] The average number, $\Pi$, of
  nucleotide site differences between pairs of sequences.
\item[Mean pairwise difference per nucleotide] If the sequences are
$L$ bases long, it is often useful to standardadize $\Pi$ by dividing
it by $L$.  The resulting statistic is
\[
\pi = \Pi / L
\]
\item[Mismatch distribution] A histogram whose $i$th entry is the
number of pairs of sequences that differ by $i$ sites.  Here, $i$
ranges from 0 through the maximal difference between pairs in the
sample. 
\item[Site frequency spectrum] A histogram whose $i$th entry is the
number of polymorphic sites at which the mutant allele is present in
$i$ copies within the sample.  Here, $i$ ranges from 1 to $K-1$.
\item[Folded site frequency spectrum] It is often impossible to tell
which allele is the mutant and which is ancestral.  In that case, we
combine the entries for $i$ and $K-i$, so the new $i$ ranges from 1
through $K/2$.
\end{description}

\section{Data analysis}
\label{sec.dataAnalysis}

\subsection{The number $(S)$ of segregating sites}
On the last line of Table~\ref{tab.hyposeq}, segregating
(i.e.\ polymorphic) sites are indicated with a caret (\verb|^|).
There are 15 such sites.  Thus, the number of segregating sites is
$S=15$.

\subsection{The mean pairwise difference $(\Pi)$} We want the average
number of differences between pairs of individuals.  There are two
ways to do this calculation, the direct way and the easy way.

\subsubsection{The direct way} Count the number of differences between
each pair of sequences.  For example, sequences~1 and~2 differ at~6
sites.  Compare every pair of sequences, and write down the number of
differences between each pair.  If you do this (and I don't recommend
it), you should end up with 45 numbers that sum to~248.  The average
is $\Pi = 248/45 = 5.511111$.

\subsubsection{The easy way} The direct calculation involved two steps.
Step~1 calculated the number (248) of pairwise differences, and then
step~2 divided by the number (45) of pairs.  The first of these
numbers can be thought of as a sum over sites: the number of pairwise
differences at site~1 plus that at site~2 and so on.  The monomorphic
sites make no contribution to this sum, so we need consider only the
15 polymorphic sites.  And each site makes a contribution that is easy
to calculate.

Suppose that at some site the sample contains only two nucleotides:
$x$ As and $y$ Gs.  Among pairs of sequences there will be some AA
pairs, some AG pairs, and some GG pairs, but only the AG pairs will
contribute a difference.  The number of such pairs is $x\times y$, so
this is the value that this particular site makes to the sum of
pairwise differences.

For example, consider site~6 in the data above.  There are 3~As and
7~Gs, so there are $3 \times 7 = 21$ AG pairs, and site~6 contributes
21 to the sum of pairwise differences.  At site~2, on the other hand,
there are 1~G and 9~As, so the site contributes $1 \times 9 = 9$ to
the sum.  Summing across the~15 polymorphic sites gives~248 as before.

There is also an easy way to find the number of pairs.  In a sample of
$K$ sequences, there are $K(K-1)/2$ pairs.  There are 10 sequences in
the data above, so the formula gives $(10 \times 9) / 2 = 45$ pairs.

\subsection{Computer output}
\label{sec.toySeqData-results}
Here is the output of my seqstat program, which calculates descriptive
statistics  for DNA sequences:
\begin{verbatim}
%                                  seqstat
%                (descriptive statistics from sequence data)
%                             by Alan R. Rogers
%                                Version 5-1
%                                30 Jan 2000
%                        Type `seqstat -- ' for help

% Cmd line: seqstat af10.seq

% Population 0
meanPairwiseDiff = 5.511111 ;
nsequences = 10 ; 
nsites =  40 ;
mismatch = 1 5 3 2 2 6 8 8 5 2 2 1 ;
segregating = 15 ;
spectrum =  6 2 2 5 0 ;
% Count of minor allele at each polymorphic site:
%psite  site  count | psite  site  count
     1     2      1 |     9    21      3
     2     3      2 |    10    28      2
     3     6      3 |    11    31      4
     4     8      4 |    12    32      1
     5    11      1 |    13    36      1
     6    12      4 |    14    37      1
     7    19      4 |    15    40      1
     8    20      4 |
\end{verbatim}

\newpage
\begin{exercise}
Here is a set of 10 made-up DNA sequences, each with 10 nucleotide
sites.\\[4pt]
\begin{minipage}{\textwidth}
\begin{verbatim}
S01       AAACT GTCAT
S02       ..... A....
S03       ..G.. A....
S04       ..G.. A....
S05       ..... A....
S06       ..... AC...
S07       ..... A....
S08       ..... A....
S09       ..... A...C
S10       ..... A....
\end{verbatim}
\end{minipage}\\[6pt]
Calculate the mean pairwise difference, the number of segregating
sites, the mismatch distribution and the site frequency spectrum.
\answer
\begin{verbatim}
% Cmd line: seqstat eu10.dat -c
% 1st line of input specifies 10 subjects and 10 sites
% Results will include the reference sequence (line 1).
 
% Population 0
    sequences       : 10
    sites           : 10
    mismatch = 11 25 9 ;
    meanPairwiseDiff
        per sequence: 0.955556
        per site    : 0.0955556
   segregating sites: 4
   theta estimated from segregating sites
        per sequence: 1.41394
        per site    : 0.141394
    spectrum =  3 1 ;
    % Count of minor allele at each polymorphic site:
    %psite  site  count
         1     3      2
         2     6      1
         3     7      1
         4    10      1    
\end{verbatim}
\end{exercise}
