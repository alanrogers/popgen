% -*-latex-*-
\reversemarginpar
\chapter{Homozygosity\label{ch.homozygosity}} 

This chapter needs to be merged with chapter~\ref{ch.drift}.

\section{Probability primer}

\subsection{What is probability?}

Probability is a common-sense notion.  If you get a pair of aces once
in every 300 poker hands, then we say that a pair of aces occurs with
probability 1/300.  If, on average, twin births occur once in every
1000 pregnancies, then the probability that a pregnancy will produce
twins is 1/1000.

\subsection{The probability of $A$ and $B$}

What is the probability that it will rain both today and tomorrow?  If
these events were independent,\footnote{These two events are unlikely
to be independent.  But bear with me---this is only an example.} then
\begin{displaymath}
\Pr[\hbox{rain today and rain tomorrow}] = 
\Pr[\hbox{rain today}] \times \Pr[\hbox{rain tomorrow}]
\end{displaymath}
This is an example of the ``multiplication rule'' of probabilities:
The probabilities of independent events multiply.  In general,
In general,
\begin{displaymath}
\Pr[\hbox{Event $A$ and event $B$}]
 = \Pr[\hbox{Event $A$}] \times \Pr[\hbox{Event $B$}]
\end{displaymath}
if events $A$ and $B$ are independent.

\subsection{The probability of $A$ or $B$}

What is the probability that it will either rain or snow?  If we
assume that it can't do both, then
\begin{displaymath}
\Pr[\hbox{rain or snow}] = \Pr[\hbox{rain}] + \Pr[\hbox{snow}]
\end{displaymath}
This is an example of the ``addition law'' of probabilities: The
probabilities of mutually exclusive events add.  In general,
\begin{displaymath}
\Pr[\hbox{$A$ or $B$}] = \Pr[A] + \Pr[B]
\end{displaymath}
if $A$ and $B$ are mutually exclusive.

\subsection{Example}

In 1850, the Swiss astronomer Wolf recorded the results of 20,000
tosses of a pair of dice.  The results are shown in
table~\ref{tab.dice}.

\begin{table}
\caption{Results of 20,000 tosses of two dice\label{tab.dice}}
\centering
\begin{tabular}{ccccccccc}
Red/White & 1 & 2 & 3 & 4 & 5 & 6 & Total & Proportion\\
\hline
  1 & 547 & 587 & 500 & 462 & 621 & 690 & 3407 & 0.170\\
  2 & 609 & 655 & 497 & 535 & 651 & 684 & 3631 & 0.182\\
  3 & 514 & 540 & 468 & 438 & 587 & 629 & 3176 & 0.159\\    
\hline
\end{tabular}
\end{table}

\section{What is homozygosity?}

Homozygosity is the probability that a random pair of genes will
be identical.  For example, suppose that a population consisted of
just the three genes whose DNA sequences are shown below:
\begin{center}
\begin{minipage}{16em}
\begin{verbatim}
Gene 1: AATAGGACCC
Gene 2: AACAAGACCC
Gene 3: AACAAGACCC
\end{verbatim}
\end{minipage}
\end{center}
Genes 2 and 3 are identical, but they differ from gene~1 at sites~3
and~5.  What is the probability that a random pair of genes drawn from
this population would be identical?  To find out, we count the pairs
that are identical:
\begin{center}
\begin{tabular}{cc}
Pair & Comparison\\
\hline
1 versus 2 & differ\\
1 versus 3 & differ\\
2 versus 3 & identical\\
\hline
\end{tabular}
\end{center}
One third of the pairs are identical, so if we drew pairs at random we
would get an identical pair 1/3 of the time.  The heterozygosity is
0.333.

\section{Real data}

We collected DNA from a sample of 77 Asian humans, and sequenced 630
sites from the mitochondrial genome of each subject.  Since our sample
had 77 subjects, we had 2926 pairs to consider. Of these, 5 were
identical, so our homozygosity is
\begin{displaymath}
H = 5/2926 = 0.0017
\end{displaymath}
But so what?  This number is of no interest whatever unless we have
some way of relating it to things that \emph{are} interesting.  What
we need is a theory, so let us build one.

\section{Theory}

\subsection{How heterozygosity changes with time}

I will write $G(t)$ for the homozygosity during generation $t$.  In
other words, if we draw a pair of genes from the population at random
during generation $t$, $G(t)$ is the probability that they are
identical.  The probability that two random genes are identical can be
decomposed into two independent events:
\begin{enumerate}
\item neither of the two genes has mutated in the past generation, and
\label{event.nomutation}
\item the ancestors of the two two genes were identical in generation
$t-1$. 
\label{event.id}
\end{enumerate}
We want to know the probability of event~1 \emph{and} event~2, and
these events are (presumably) independent.  Thus, the product rule
applies.  We could calculate $G(t)$ if only we knew the probabilities
of events~1 and~2.  We have the first of these probabilities already.
(It is $1-2u$.)  Event~2, is more complicated, but we can decompose it
into two simpler events.  The two ancestral genes may have been
identical in either of two ways:
\begin{enumerate}
\setcounter{enumi}{2}
\item They could have been genes that were separate but identical in
generation $t-1$.
\label{event.iis}
\item They could have been the same gene (if our original two genes
are copies of the same parental gene in generation $t-1$).
\label{event.ibd}
\end{enumerate}
Events \ref{event.ibd} and \ref{event.iis} are mutually exclusive, and
we want to know the probability of either event~\ref{event.ibd}
\emph{or} event~\ref{event.iis}.  Thus, the addition rule applies.
We need to calculate the probabilities of three simple
events, which we will then combine using the addition and
multiplication rules.

\subsection{The probability of event~\ref{event.ibd}}

Let $N$ represent the number of genes in the population from
which our random pair of genes was drawn.  The two genes are copies of
the same parental gene with probability $1/N$.  Why?  Consider the
following diagram:
\begin{verbatim}
             N parental genes
            _________^_______
Generation /                 \
   t-1     O  O  O  O ... O  O
           |
    t      A  B
\end{verbatim}
A and B represent the two genes that we drew at random from
generation $t$, and the $N$ O's represent the genes of the parental
generation (generation $t-1$).  The vertical bar shows you that the
first gene of generation $t-1$ was parental to A.  Which gene was
parental to B?  We don't know.  Any of the $N$ parental genes is
equally likely.  Of these, only the 1st was also parental to A.  Thus,
the probability that the gene parental to B was also parental to A
equals $1/N$.

\subsection{The probability of event~\ref{event.iis}}

The two genes are copies of distinct parental genes with probability
$1-1/N$.  These distinct genes are identical with probability
$G(t-1)$, for this is how we have defined $G$.  These two events are
independent, so we use the product rule to combine them.  The two
genes that we drew are copies of parental genes that are distinct and
identical with probability
\begin{displaymath}
(1 - 1/N)G(t-1)
\end{displaymath}

\subsection{The probability of event~\ref{event.id}}

Event~\ref{event.id} says that the ancestors of our two chosen genes
were identical in generation~\ref{event.id}.  By the addition rule,
the probability of event~\ref{event.id} is the sum of the
probabilities of events~\ref{event.ibd} and~\ref{event.iis}:
\begin{displaymath}
\Pr[\hbox{parental genes identical}] = 1/N + (1 - 1/N)G(t-1)
\end{displaymath}

\subsection{The probability of event~\ref{event.nomutation}} 

Write $u$ for the probability that one gene mutation mutates during a
single generation.  It is called the \emph{mutation rate}.  If $u$ is
the probability that a gene mutates, then $1-u$ must be the
probability that it does not mutate.  The probability that neither
gene mutates is (using the multiplication rule)
\begin{eqnarray*}
\lefteqn{\Pr[\hbox{1st doesn't mutate}] 
  \times \Pr[\hbox{2nd doesn't mutate}]}\hspace{2in}\\
 &=&(1-u)^2 \\
 &=& 1 - 2u + u^2
\end{eqnarray*}
This formula is exact, but we are often more interested in
approximations than in exact answers.  In this case, $u$ is
approximately equal to 0.005, so the equation above can be written as
\begin{displaymath}
(1-u)^2 = 1 - \overbrace{0.01}^{2u} + \overbrace{0.000025}^{u^2}
\end{displaymath}
The $u^2$ term is so much smaller than the others that we will not
lose much by ignoring it.  Thus, the probability that neither gene
mutates is approximately $1-2u$.  

\subsection{Homozygosity}
Putting this all together,
\begin{eqnarray*}
G(t) &=& (1-2u)(1/N + (1 - 1/N)G(t-1))\nonumber\\
  &=& \frac{1}{N} - \frac{2u}{N}
 + \left(1 - 2u - \frac{1}{N} + \frac{2u}{N} \right) G(t-1)\nonumber\\
 &\approx& \frac{1}{N} 
 + \left(1 - 2u - \frac{1}{N} \right) G(t-1)\label{eq.G(t)}
\end{eqnarray*}
In the last line here I have dropped the term $2u/N$ since it is
extremely small compared with $1/N$.  We won't lose much accuracy by
ignoring it.

\input{fighomoz1} 
%
Just to see what this formula looks like, I put in some numeric
values.  I set $u=0.005$, $N=5000$, and $H(0) = 0$.  The formula above
was then used to calculate $H(1)$, then used again to calculate
$H(2)$, and so on.  (The computer made this process a lot less
laborious than it sounds.)  The result is shown in
figure~\ref{fig.homoz1}.  There, you can see that homozygosity
increases gradually and then levels out at a value near 0.02.

\subsection{Heterozygosity}
We are often interested in the probability that two genes differ. This
is called the \emph{heterozygosity}. Eqn.~\ref{eq.G(t)} shows that
\begin{eqnarray*}
H(t) &=& 1 - G(t)\\
    &\approx& 1 - \frac{1}{N} 
 - \left(1 - 2u - \frac{1}{N} \right) (1-H(t-1))\\
&=&  2u + \left(1 - 2u - \frac{1}{N} \right) H(t-1)
\label{eq.H(t)}
\end{eqnarray*}

\subsection{Equilibria}

The value at which a system stops changing is called its
\emph{equilibrium}.  Equilibria are particularly interesting, because
they are what we tend to see in the world around us.  At equilibrium
$G(t-1) = G(t)$.  We can find the equilibrium value by fixing the
value of $G$ in equation~\ref{eq.G(t)}.  The resulting equation is
\begin{displaymath}
G = \frac{1}{N}  + \left(1 - 2u - \frac{1}{N} \right) H
\end{displaymath}

\begin{exercise}
What is the value of $G$ that satisfies this equation?
\end{exercise}
\begin{exercise}
What is the corresponding formula for $H$?
\end{exercise}
\begin{exercise}
Figure~\ref{fig.homoz1} assumed that $u=0.005$ and $N=5000$.  Under
these assumptions, what is the equilibrium value of $G$?  Is it
consistent with the figure?
\end{exercise}
\begin{exercise}
With our data, $H=0.0017$.  What does this imply about the values of
$u$ and $N$?
\end{exercise}
\begin{exercise}
If (as we think) $u = 0.005$, what is implied about the value of $N$?
\end{exercise}

%%% Local Variables: 
%%% mode: latex
%%% TeX-master: t
%%% End: 
