% -*-latex-*-
\chapter{Relating Gene Genealogies to Genetics}
\label{ch.addmut}

This course began with a section on probability theory.  Then came
material on genetic variation and genetic drift.  In the last lecture,
we discussed gene genealogies.  These may have seemed like
disconnected threads.  This lecture will tie them all together.

We will make extensive use of the theory introduced last time about
genealogical relationships among genes.  Although that theory is
elegant, it is also limited, for gene genealogies cannot be observed.
They describe obscure events that happened many thousands of years
ago.  We can never \emph{know} the genealogy of a sample of genes.  We
can \emph{estimate} it from genetic data, but that requires a theory
that relates gene genealogies to observable genetic data.

This lecture begins by adding mutations to gene genealogies, and then
relates these to two genetic statistics: the number $S$ of segregating
sites, and the mean pairwise difference $\pi$ between nucleotide
sequences.  Finally, it will consider two ways to estimate the
parameter $\theta$.

\section{The number of mutations on a gene genealogy}
\label{sec.nMutations}

Consider the gene genealogy below:\\
\begin{minipage}{\textwidth}
\begin{verbatim}

     --x----x---
                |
                |------x-
                |        |
     ------xx---         |
                         |-----x----
                         |
                         |
     ---x-x----------x---

     |---t -----|---t ---|
          3          2
\end{verbatim}
\end{minipage}\\[1ex]
Each ``x'' represents a different mutation, and I'll assume that each
mutation is at a different nucleotide site. Although there are 9
mutations, the variation within the sample results only from the 8
that are ``downstream'' of the genealogy's root.  The 9th mutation
would produce an identical effect on all members of the sample and is
therefore of no interest to us.  For our purposes, this is a genealogy
with 8 mutations.

How many mutations would we expect to see in a sample of 3 genes?  The
answer will depend in part on the number of nucleotide sites being
examined. We expect more mutations per generation on an entire
chromosome than at a single nucleotide site. Although this effect is
large, we can avoid dealing with it directly by using a
flexible\label{pg.flexu} definition of the mutation rate, $u$. If we
are studying a single nucleotide site, then $u$ will represent the
expected number of mutations per generation per site. If we are
studying a larger region---say an entire gene or chromosome---then $u$
will represent the expected number of mutations per generation in this
entire region.

Unless we are studying a large genomic region, $u$ will be very small
and can be interpreted not only as the expected number of mutations
per generation, but also as the probability of a single mutation. This
is because we don't lose much, when mutations are very rare, by
ignoring the remote possibility that several of them happen at once.

The expected number of mutations depends not only on the mutation
rate, $u$, but also on the total length of the gene genealogy. In a
sample of 3 genes, this length is
\begin{equation}
L = 3t_3 + 2t_2
\label{eq.L3}
\end{equation}
where $t_3$ is the length of the coalescent interval during which the
genealogy had 3 lines of descent, and $t_2$ is the length of the other interval.
If there are $u$ mutations per generation, then we would expect this
tree to have $uL$ mutations---if we knew the value of $L$.

But since the value of $L$ is ordinarily unknown,
\[
E[\hbox{\# of mutations}] = E[uL] = uE[L]
\]
To calculate this expected value, we need the expectation of $L$.  The
direct approach would involve inspecting thousands of gene
genealogies, each generated by the coalescent process described in the
last lecture.  You cannot, of course, look at even one real gene
genealogy, let alone thousands of them.  You could write a computer
program to simulate the process, but we can do the same job more
easily using the theory from the previous lecture.

The $E$ stands for ``expectation,'' and in the present context it
refers to an average over genealogies.  For example, $E[t_2]$ is the
expectation (that is, average) of $t_2$ over a very large number of
genealogies.  We learned in the last lecture that $E[t_2] = 2N$ and
$E[t_3] = 2N/3$. Thus, equation~\ref{eq.L3} implies that the
expectation of $L$ is
\[
E[L] = 3 E[t_3] + 2 E[t_2] = 2N + 4N
\]

In general, the expected length of the coalescent interval during
which there are $i$ lines of descent is (see equation~\ref{eq.hinv})
\[
E[t_i] = \frac{4N}{i(i-1)}
\]
and the contribution of this interval to the expected length
(including all branches) of the tree is
\[
iE[t_i] = \frac{4N}{i-1}
\]
The total expected length of the tree is
\begin{equation}
E[L] = \sum_{i=2}^K iE[t_i] = 4N \sum_{i=1}^{K-1} \frac{1}{i}
\label{eq.EL}
\end{equation}
where $K$ is the number of genes in the sample.  The expected number
of mutations on the gene genealogy is thus
\begin{eqnarray}
E[\hbox{\# of mutations}] &=& u E[L]\nonumber\\
  &=& 4Nu \sum_{i=1}^{K-1} \frac{1}{i}\nonumber\\
  &=& \theta \sum_{i=1}^{K-1} \frac{1}{i}
\label{eq.nMutations}
\end{eqnarray}
where $u$ is the mutation rate per generation, and $\theta = 4Nu$, a
quantity that appears often in the formulas of population genetics.
It equals twice the number of mutations that occur each generation in
the population as a whole.  Like $u$ its magnitude depends on the size
of the genomic region under study (see p.~\pageref{pg.flexu}). Below,
we will consider the problem of estimating $\theta$ from genetic data.

\section{The model of infinite sites}
\label{sec.infsite}

We can now calculate the expected number of mutations in a gene
genealogy of any size.  The next challenge is to connect this result
to data---to genetic differences between individuals.  The easiest
approach involves what is known as the ``model of infinite sites.''
This model assumes mutation never strikes the same nucleotide site
twice.  This model is never really correct, but it is often a good
approximation, especially in intra-specific data sets where the
genetic differences between individuals are small.  It makes sense to
use it when the mutation rate per nucleotide site is low enough that
only a small fraction of nucleotide sites will mutate twice in any
given gene genealogy.

This model, of course, is only an approximation. In the real world,
nucleotide sites may mutate more than once. Appendix
section~\ref{sec.errinfsite} (p.~\pageref{sec.errinfsite}) shows that
these violations occur, on average, at a fraction $(ut)^2/2$ of sites,
where $u$ is the mutation rate per site per generation and $t$ is the
number of generations.

For example, suppose that some branch of the gene genealogy is
$t=10^4$ generations long and that $u=10^{-8}$. Along this branch, the
expected number of mutations at a single nucleotide site is
$ut=10^{-4}$. In an entire human genome, the number of sites is about
$3 \times 10^{9}$. The expected number of sites that violate the
infinite sites model is therefore
\[
3 \times 10^{9} \times 10^{-8} /2 = 15
\]
The model of infinite sites is expected to fail only at 15 sites out
of 3 billion.  This analysis should not be taken too literally,
because the mutation rate is not really constant across the genome,
and we may be interested in much longer time intervals. Nonetheless,
it does show that the model of infinite sites works well when the
product $ut$ is small.

\section{The number of segregating sites}
\label{sec.segsite}

If mutation never strikes the same site twice, then the number $S$ of
segregating (i.e.\ polymorphic) sites in a data set is the same as the
number of mutations in its gene genealogy, as given in
equation~\ref{eq.nMutations}.  The expected number of segregating
sites is \citep{Watterson:TPB-7-256}
\begin{equation}
E[S] = \theta \{1 + 1/2 + 1/3 + \cdots + 1/(K-1)\}
\label{eq.ES}
\end{equation}
Finally, we have arrived at a statistic that can be calculated from
data.  The expected number of segregating sites is equal to $\theta$
times some number that increases with sample size.  Thus, we expect
more segregating sites in a large sample.  But the effect of sample
size is not pronounced, because the sum in the expression above
doesn't increase very fast.  Here are a few example values:
\begin{center}
\begin{tabular}{rr}
$K$ & $\sum_{i=1}^{K-1} 1/i$\\ \hline
2 & 1.00\\
3 & 1.50\\
5 & 2.08\\
10 & 2.82\\
100 & 5.17\\
1000 & 7.48\\
\hline
\end{tabular}
\end{center}
In a sample of 100, we expect only about 5 times as many segregating
sites as in a sample of 2.

The effect of the population size, on the other hand, is pronounced
since $E[S]$ is proportional to $\theta$, and $\theta$ is proportional
to the population's size.  In a population twice as large, we expect
twice as many segregating sites.

\section{The mean pairwise difference}

Given any pair of DNA sequences, it is a simple matter to count the
number of nucleotide positions at which they differ.  Given a sample
of size $K$, there are $K(K-1)/2$ pairwise comparisons that can be
made and we can count the number of nucleotide differences between
each pair.  Averaging these numbers gives a statistic that is called
the ``mean pairwise nucleotide difference'' and is generally denoted
by the symbol $\pi$.%
%%%%%%%%%%%%%%%%%%%%%%%%%%%%%%%%%%%%%%%%%%%%%%%%%%%%%%%%%%%%%%%%
\footnote{Some authors \citep{Li:ME-97} use the capital letter $(\Pi)$
  to denote the mean pairwise differences per sequence and the
  lower-case letter $(\pi)$ to refer to the mean pairwise difference
  per site.  I use the lower-case letter for both purposes.}
%%%%%%%%%%%%%%%%%%%%%%%%%%%%%%%%%%%%%%%%%%%%%%%%%%%%%%%%%%%%%%%%

What is the expected value of $\pi$?  The number of nucleotide site
differences between a pair of sequences is the same as the number of
segregating sites in a sample of size 2.  Thus, equation~\ref{eq.ES}
tells us that the average pair of sequences differs at $\theta$ sites.
Averaging over all the pairs in a sample doesn't change this
expectation, so
\begin{equation}
E[\pi] = \theta
\label{eq.Epi}
\end{equation}

This gives us the expected value of a second statistic that can be
estimated from genetic data, and this time the formula is especially
simple.  As in the case of $S$, we can expect the value of $\pi$ to be
large if the population is large, small if the population is small.  
\begin{exercise}
Just above, I said that if the expected difference between each pair
of sequences is $\theta$, then the expectation of $\pi$ is also
$\theta$.  Prove that this is so.
\answer
%
Let $x_{ij}$ denote the difference between the $i$th and $j$th
sequences, and let $M$ denote the number of pairs of sequences in the
sample.  Then $\pi = \sum x_{ij}/M$, where the sum runs over all $i$
and $j$ such that $i < j$.  The expected value of $\pi$ is
\begin{eqnarray*}
E[\pi] &=& E\left[M^{-1}\sum x_{ij} \right]
     \qquad(\hbox{definition of $\pi$})\\
       &=& M^{-1} E\left[\sum x_{ij} \right]
     \qquad(\hbox{JEPr eqn.~\ref{JEPr-eq.E[aX]}})\\
       &=& M^{-1} \sum[ E x_{ij} ]
     \qquad(\hbox{JEPr eqn.~\ref{JEPr-eq.E[X+Y]}})\\
       &=& M^{-1} M \theta
     \qquad(\hbox{eqn.~\ref{eq.ES} with $K=2$})\\
       &=& \theta
\end{eqnarray*}
Here, JEPr eqn.~\ref{JEPr-eq.E[aX]} tells us that $E[aX] = aE[X]$ for
constant $a$ and variable $X$, and JEPr eqn.~\ref{JEPr-eq.E[X+Y]}
tells us that the expectation of a sum equals the corresponding sum of
expectations. Eqn.~\ref{eq.ES} (with $K=2$) tells us that the expected
number of segregating sites in a sample of size~2 is equal to
$\theta$. For such samples, the number of segregating sites is equal
to the number of differences between the two sequences.
\end{exercise}

\begin{exercise}
For the following questions, assume that the population mates at
random, has constant size $2N=1000$, that there is no selection, and
that the mutation rate is $u=1/2000$ per sequence per generation.
Assume that you are working with a sample of $K=5$ DNA sequences, and
that mutations obey the model of infinite sites.
\begin{enumerate}
\item 
What is the expected depth of the gene tree?  (In other words, the
expected number of generations since the last common ancestor.)
\item
What is the expected length of the tree?  (In other words, the
expected sum of the lengths of all branches in the tree.)
\item
What is the expected number of mutations on the tree?
\item
What is the expected number of mutational differences between each
pair of sequences?
\end{enumerate}
\end{exercise}

\section{Theta and Two Ways to Estimate It\label{sec.theta}} 

In this lecture, we have twice run into the parameter $\theta$, which
is proportional to the product of mutation rate and population size.
This parameter appears often in population genetics, and it is useful
to have a way to estimate it.  The results above suggest two ways.
Equation~\ref{eq.ES} suggests
\begin{equation}
\hat\theta_S = \frac{S}{\sum_{i=1}^{K-1} \frac{1}{i}}
\label{eq.thetaS}
\end{equation}
and equation~\ref{eq.Epi} suggests.
\begin{equation}
\hat\theta_{\pi} = \pi
\label{eq.thetaPi}
\end{equation}
Here $\hat\theta$ is read ``theta hat.''  The ``hat'' indicates that
these formulas are intended to estimate the parameter $\theta$.

To make sure that these formulas estimate the same parameter, it is
important to be consistent. $S$ usually refers to the number of
segregating sites within some larger DNA sequence. To make $\pi$ 
comparable, we interpret it here as the mean pairwise difference
\emph{per sequence} rather than that \emph{per site}. We also need to
interpret $u$ as the mutation rate per sequence when we define $\theta
= 4Nu$. 

With these consistent definitions, $\hat\theta_S$ and
$\hat\theta_{\pi}$ estimate the same parameter. It seems natural to
suppose that their values would be similar in real data.  Let's have a
look at some human mitochondrial DNA sequence data.

\section{Example}

Jorde et al (ref) published sequence data from the control region of
human mitochondrial DNA.  The example described here uses 430
nucleotide positions from HVS1 (the first hypervariable region).
Jorde et al sequenced DNAs from all three major human racial groups,
but this example will deal only with the 77 Asian and 72 African
sequences.  In these  data:
\begin{center}
\begin{tabular}{lcc}
                            & Asian & African \\ \hline
$S$                         &  82    &   63\\
$\sum_{i=1}^{K-1} 1/i$      & 4.915  & 4.847\\
$\hat\theta_S$ (per sequence) & 16.685 & 12.998\\
$\pi$ (per sequence)        & 6.231  & 9.208\\
\hline
\end{tabular}
\end{center}

The theory above says that $\pi$ and $\hat\theta_S$ are both estimates 
of the parameter $\theta$, so we have every reason to expect their
values to be similar.  Yet $\hat\theta_S$ is half again as large as
$\pi$ in the African data and nearly three times as large in the
Asian.  Why are these numbers so different?

There are at least four possibilities worth considering:
\begin{description}
\item[Sampling error] To figure out whether these
  discrepancies are large enough to worry about, we need a theory of
  errors.

\item[Natural selection] The theory we have used assumes neutral
  evolution.  If selection has been at work, then we have no reason to
  think that $\pi$ and $\hat\theta_S$ will be equal.  In fact, the
  difference between $\pi$ and $\hat\theta_S$ is often used to test
  the hypothesis of selective neutrality.  (Look up Tajima's D in any
  textbook on population genetics.) 

\item[Variation in population size] Our theory also assumes that the
  population has been constant in size.  We need to investigate how
  $\pi$ and $\hat\theta_S$ respond to changes in population size.

\item[Failure of the infinite sites model] Our theory assumes that
  mutation never strikes the same site twice.
\end{description}


\begin{subappendices}

\section{The probability that a nucleotide site is polymorphic within
  a sample}

In comparisons between pairs of haploid human genomes, about one
nucleotide site in a thousand is polymorphic. In larger samples, of
course, the polymorphic fraction is larger. What is the fraction
$(Q_K)$ that is expected to be polymorphic in a sample of size $K$?
It is easier to work with the monomorphic fraction, $1-Q_K$.  The gene
genealogy is monomorphic only if no mutation occur in any coalescent
interval. Let us consider first the coalescent interval during which
there were $k$ ancestors. As we trace time backwards across this
interval, we might encounter either of two types of event: a mutation
or a coalescent event. We encounter a coalescent event first if (and
only if) the interval is free of mutations.

During this interval, coalescent events happen with hazard
$\lambda_2=k(k-1)/4N$ per generation, and mutations happen with hazard
$\lambda_1 = ku$, where $u$ is the mutation rate per site per
generation.  Once an event does occur, it is a coalescent event with
probability
\[
z_k = \frac{\lambda_2}{\lambda_1 + \lambda_2}
    = 1 - \frac{\theta}{\theta + k-1}
\]
where $\theta = 4Nu$  \citep[see reference][pp.~48--49]{Wakeley:CTI-09}.  This is the
probability that the $k$th coalescent interval is free of
mutations. When $\theta$ is small, $z_k$ is approximately
\[
z_k \approx 1 - \frac{\theta}{k-1}
    \approx e^{-\theta/(k-1)}
\]
Because the mutations that occur in different coalescent
intervals are independent, these probabilities multiply. 
The entire
gene genealogy is free of mutations with probability 
\begin{eqnarray}
1-Q_K &=& z_2 z_3 z_4 \cdots z_{K}\nonumber\\ 
      &\approx& \exp[-\theta\{1 + 1/2 + 1/3 + ... + 1/(K-1)\}]
\label{eq.fracmono}
\end{eqnarray}
The expected fraction of polymorphic sites is $Q_K$. For example, if
$\theta = 1/1000$, the fraction of polymorphic sites should be 0.001
in a sample of size 2, 0.003 in a sample of 10, and 0.005 in a sample
of 100. The polymorphic fraction increases with $K$, but not very
fast.

\section{When you assume the model of infinite sites, how wrong are
  you likely to be? (optional)}
\label{sec.errinfsite}

The model of infinite sites assumes that mutation never strikes the
same site twice. Clearly, this is only an approximation, and when we
use this model we are bound to introduce errors. The question is, how
large are these errors likely to be? What fraction of the sites in our
data can be expected to mutate more than once?

To find out, let us consider the mutations that occur at some
nucleotide site along a single branch of a gene genealogy. If the
branch is $t$ generations long, then the number, $X$, of mutations is
a Poisson-distributed random variable with mean $\lambda = ut$, where
$u$ is the mutation rate per generation.

Consider the probability, $P$, that $X<2$. This is the probability
that our site conforms to the model of infinite sites. Because $X$ is
Poisson, 
\[
P = e^{-\lambda} + \lambda e^{-\lambda} 
\]
If $\lambda$ is small, $e^{-\lambda} \approx 1 - \lambda +
\lambda^2/2$, ignoring terms of order $\lambda^3$.  (This is from the
series expansion of the exponential function.)  To this standard of
approximation,
\begin{eqnarray*}
P &\approx& 1-\lambda + \lambda^2/2\\
 &&\mbox{} + \lambda-\lambda^2 + \lambda^3/2\\
 &\approx&1 - \lambda^2/2
\end{eqnarray*}
The fraction of sites that \emph{violate} the infinite sites model is
approximately $1-P = \lambda^2/2$---a very small number.
\end{subappendices}
