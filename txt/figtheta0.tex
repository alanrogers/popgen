% -*-latex-*-
\begin{figure}
\begin{center}
\mbox{%
\beginpicture
\headingtoplotskip=0.25\baselineskip
\setcoordinatesystem units <0.4in, 0.4in> point at 0 0
\setplotarea x from -2 to 3, y from -2 to 3
\axis left label {\lines{Quantiles\cr of $\log_{10}\hat\theta_0$}}
   shiftedto x=-2.25
   ticks numbered from -2 to 3 by 1 /
\axis bottom shiftedto y=-2.25 label {$\log_{10} \theta_0$}
   ticks numbered from -2 to 3 by 1 /
\multiput {$\bullet$} at -2 -2 -1 -1 0 0 1 1 2 2 3 3 /
\setdots
% x=log10 theta0 y= quantile 0.025 of log10 hat theta0_2
\plot  1 0.066045 2 1.314945 3 2.315672 /
\setdashes
% x=log10 theta0 y= quantile 0.25 of log10 hat theta0_2
\plot 0.5 -0.116484 1 0.580528 2 1.619688 3 2.636972 /
\setsolid
% x=log10 theta0 y= quantile 0.5 of log10 hat theta0_2
\plot 0 -0.315373 0.5 0.178825 1 0.763900 2 1.794424 3 2.798383 /
\setdashes
% x=log10 theta0 y= quantile 0.75 of log10 hat theta0_2
\plot -2 -0.122112 -1 -0.144424 0 0.035116 0.5 0.433008
    1 0.950616 2 1.981298 3 2.967431 /
\setdots
% x=log10 theta0 y= quantile 0.975 of log10 hat theta0_2
\plot -2 0.142102 -1 0.156814 0 0.390789 0.5 0.818314
    1 1.299401 2 2.301716 3 3.304085 /
\endpicture%
}
\end{center}
\caption{Quantiles of $\hat\theta_$\label{fig.theta0}}
%
1,000 data sets were simulated at each of several values of
$\theta_0$, and each was used to estimate the model's three
parameters.  In each run, $\theta_1=1000$, $\tau=7$, and $N=147$. The
lines and bold dots are interpreted as in figure~\ref{fig.tau}.
\end{figure}
