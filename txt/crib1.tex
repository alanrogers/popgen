%-*-latex-*-
\chapter{Study Guide for Quiz 1}
\section{Polymorphisms}
A \emph{polymorphism} is a genetic system in which there is variation
among individuals.  \emph{Classical polymorphisms} are those we knew
about before roughly 1980.  \emph{Molecular polymorphisms} are those
we've found out about since then.  Classical polymorphisms include
blood groups and enzyme polymorphisms.  Molecular polymorphisms
include 
\begin{description}
\item[restriction fragment length polymorphisms (RFLPs)]  These are
obtained using an enzyme that cuts the DNA wherever it finds a
particular pattern, such as ATAGGC.  If two chromosomes have this
pattern in exactly the same places, then the enzyme will cut the
two chromosomes into pieces of the same size.  Otherwise, the pieces
obtained from one chromosome will differ from the pieces obtained from
the other.  This generates data that are fairly hard to analyze, and
the method is now obsolete.
\item[DNA sequences] This kind of data is like examining the DNA
directly.  You have the complete sequence of nucleotides for some
specific segment of each chromosome that you study.
\item[tandem repeat polymorphisms] A particular sequence is repeated
again and again.  For example, if the repeated sequence is AT, then
one chromosome might have a sting of three repeats (ATATAT) and
another might have a string of five (ATATATATAT).  They are called
``tandem'' repeats because the repeated units are stuck together
end-to-end.  There are several categories of repeat polymorphism:
\begin{description}
\item[Minisatellite] the repeated unit is fairly large
\item[Microsatellite] the repeated unit is small, commonly 2, 3, or 4
nucleotides.  Microsatellites are also called \emph{Short Tandem
Repeats} (STRs).
\end{description}
\item[dispersed repeat polymorphisms] These also consist of a sequence
that is repeated again and again, but in these polymorphisms the
repeated units are adjacent to one another.   They are not even close
together.  They are widely dispersed within the genome.  The \emph{Alu
polymorphisms} are an example of this class of polymorphism.
\end{description}

Know the characteristics of 
\begin{enumerate}
\item Classical polymorphisms, which include blood groups, enz
\end{enumerate}

\section{Random Mating}

A population evolves as its allele frequencies change.  To understand
evolution, then, we must understand the forces that cause allele
frequencies to change.  As a baseline, however, it is useful first to
consider a population that is not changing.

There are two principles to understand here: (1)~the Hardy-Weinberg
principle, and (2)~the Wahlund principle.  

\subsection{The Hardy-Weinberg Principle}

The question addressed here is, How do the frequencies of the
genotypes at some locus relate to the frequencies of the alleles?
When the population is large, mating at random, and unaffected by
selection or migration from the outside, the answer is very simple.
(See Hartl's p.~25).

When a population's genotype frequencies are consistent with the
predictions of the Hardy-Weinberg formula, the population is said to
be at \emph{Hardy-Weinberg equilibrium}.

You need to understand (1)~the HW formula, and (2)~the distinction
between the observed genotype frequencies and the genotype frequency
predicted at HW equilibrium.

\subsection{The Wahlund Principle}

The question addressed here is, How do the genotype frequencies within
each group of a subdivided population relate to the genotype
frequencies within the population as a whole.  Hartl discusses this on
pages 106--107.  To focus on the effect of subdivision per se, it is
useful to deal with a hypothetical population in which each
subdivision mates at random.  It turns out that the more the groups
differ in their gene frequencies, the fewer heterozygotes there will
be in the group as a whole.  If groups all have the same allele
frequencies, then the heterozygosity is unaffected.  But if the groups
differ at all, the heterozygosity of will be diminished.  The
heterozygosity in the population as a whole is
\[
H = 2 \bar p (1- \bar p) - 2 V(p)
\]
where $\bar p$ is the allele frequency within the population as a
whole, and $V(p)$ is the variance among subpopulations in allele
frequencies. 

Your first lab assignment provided an illustration of this principle,
although I didn't give you this formula then.  You found that that the
expected (i.e. Hardy-Weinberg) heterozygosity within the group as a
whole was less than the average expected heterozygosity with the
subpopulations. 

\begin{example}
If there are two groups of equal size with allele frequencies 0.2 and
0.3, then what (a)~heterozygosity would be expected in the group as a
whole.  (b)~If the groups were mixed together and allowed to mate at
random for one generation, what would happen to the heterozygosity?
(For part b, assume that the population is so large that genetic drift
does not change $\bar p$.
\end{example}
\begin{showanswer}
The the allele frequency within the population as whole is
\[
\bar p = (0.2 + 0.3)/2 = 0.25
\]
The variance in allele frequency is
\begin{eqnarray*}
V(p) &=& 1/2 \times (0.2 - 0.25)^2 + 1/2 (0.3 - 0.25)^2\\
     &=& 0.0025
\end{eqnarray*}
(a)~The Wahlund formula gives the heterozygosity as
\[
H = 2 \times 0.25 \times 0.75 - 2 \times 0.0025 = 0.370
\]
(b)~If the population were mixed together and allowed to mate at
random, the overall allele frequency would not change, and the
heterozygosity would become
\[
H = 2 \times 0.25 \times 0.75  = 0.375
\]
\end{showanswer}

\section{Inbreeding}

Inbreeding occurs whenever relatives mate, and it changes genotype
frequencies.  Instead of the Hardy-Weinberg formula, the genotype
frequencies at a locus with two alleles become
\begin{center}\begin{tabular}{crcl}
Unordered\\
Genotype & \multicolumn{3}{c}{Frequency}\\ 
\hline\hline
$AA$     & $p^2$     &$+$& $p(1-p)F$\\
$Aa$     & $2p(1-p)$ &$-$& $2p(1-p)F$\\
$aa$     & $(1-p)^2$ &$+$& $p(1-p)F$\\
\hline
\end{tabular}\end{center}
Can you see that this is equivalent to the formula that Hartl gives on
p.~55? 

Here, $F$ is called the \emph{inbreeding coefficient} or the
\emph{fixation index}.  

An inbreeding coefficient is a probability---the probability that two
random genes, each drawn from a different individual, are copies of
the same gene in some ancestor.  To calculate $F$, we must first
specify which generation the ancestor lived in.  Then the $p$ in the
formula is the allele frequency in that generation.  In other words,
$p$ is the allele frequency in the \emph{reference population}.


\subsection{Pedigree inbreeding} 

Pedigree inbreeding is calculated as Hartl describes on p.~61.  The
base population is the generation at the top of the pedigree.  Suppose
we have genealogical data that goes back 4 generations in one
population and 12 generations in another.  Then the reference
population comprises the ancestors who lived 4 generations back in the
first population, and comprises those 12 generations back in the
other.  Other things being equal, the $F$s of the first population
will on average be smaller than those in the second.

\subsection{Random inbreeding} 

You had two parents, each of them had two parents, and so on back into
the past.  Ten generations back, you had $2^{10}$ (about a thousand)
ancestors.  Twenty generations back, you had about a million.  Thirty
generations back, about a billion.  But thirty generations ago was
about the time of the Norman conquest of England, and there were
surely fewer than a billion people on the planet at that time.  The
conclusion is inevitable: some of those ancestors must have been the
same people, and you are inbred.  All of us are.  This inbreeding
occurs even in a population that mates at random.  The larger the
population, the more slowly inbreeding accumulates.

There is a simple formula for calculating random inbreeding, which
Hartl gives on p.~83.  In a population of size 1 million, the
inbreeding coefficient after 30 generations is
\[
F_{30} = 1 - (1 - 1/(2\times 1,000,000))^30 = 0.000015
\]
The expected heterozygosity after this much time is
\[
H_{30} = 2p(1-p) (1- 0.000015)
\]
even if the population is mating at random.  But wait a minute---if
the population is mating at random, then the Hardy-Weinberg principle
says that the expected heterozygosity is
\[
H_{0} = 2p(1-p)
\]
How can these formulas both be right?  The trick is that the $p$s in
the two formulas are different.  The $p$ in $H_{30}$ is the allele
frequency at the time of William the Conqueror, 30 generations ago.
The $p$ in $H_0$ is the current allele frequency.

Other topics:

\begin{enumerate}
\item drift
\item mutation
\item mutation in microsatellites
\item migration
\item selection
\item probability
\end{enumerate}
