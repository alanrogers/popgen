\section{Genetic systems with dominance}

When there is dominance, two or more genotypes may be
indistinguishable.  For example, suppose that phenotype~1 comprises
genotypes $A_1A_1$ and $A_1A_2$, and that phenotype~2 consists only of
genotype $A_2A_2$.   If our sample contains $N_1$ subjects of
genotype~1 and $N_2$ of genotype~2, then
\begin{eqnarray*}
N_1 &=& n_{11} + n_{12}\\
N_2 &=& n_{22}
\end{eqnarray*}
The easiest ways to estimate allele frequencies in such systems rely
on the assumption that the sample is drawn from a population in
Hardy-Weinberg equilibrium.

Since $N_2$ is the number of $A_2A_2$ homozygotes, the Hardy-Weinberg
formula says that it should equal
\[
N_2 = N q^2
\]
where $N \equiv N_1 + N_2$ is the number of subjects and $q$ is the
frequency of $A_2$.  Thus,
\begin{equation}
q = \sqrt{N_2/N}
\label{eq.phat.dominance}
\end{equation}
\begin{example}
Estimate $p$ from a sample in which $N_1=75$ and $N_2=25$.
\end{example}
\begin{showanswer}
\[
\hat p = 1-\hat q = 1 - \sqrt{25/100} = 1/2.
\]
\end{showanswer}

There is every reason to be skeptical of this answer, since we have
justified equation~\ref{eq.phat.dominance} only by an heuristic
argument.  It turns out, however, that this is also the maximum
likelihood estimator of $p$.

\hardstuff\textsf{\small The likelihood and log likelihood are 
\begin{eqnarray*}
L &=& (1 - q^2)^{N_1} q^{2N_2}\\
\ln L &=& N_1\ln(1 - q^2) + 2N_2\ln q
\end{eqnarray*}
The derivative of $\ln L$ is
\[
\frac{d\ln L}{dq} = \frac{-2qN_1}{1 - q^2} + \frac{2N_2}{q}
\]
The maximum value of $\ln L$ occurs at a value of $q$ at which this
derivative equals zero.  To find this value, set $d\ln L/dq = 0$ and
solve the resulting equation.  This leads to
\[
q^2 = \frac{N_2}{N_1 + N_2}
\]
Taking the square root gives equation~\ref{eq.phat.dominance}.  This
shows that our simply-derived formula is in fact a maximum likelihood
estimator. 
}
