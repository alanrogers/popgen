%-*-latex-*-
\chapter{\emph{Alu} Insertions\label{ch.alu}}

\section{What is an \emph{Alu} insertion?}

\begin{description}
\item[Transposable elements] are DNA sequences that can copy
themselves and insert the copies into different parts of the genome.
\begin{itemize}
\item constitute large fractions of eukaryotic genomes
\item ``selfish DNA'': they don't have any function that we know of 
\end{itemize}
\item[Short Interspersed Elements (SINEs)] are short retroposons
(typically 75--500bp) that lack the machinery needed to transpose
themselves.  They must be produced by active sites somewhere in the
genome, but the details of this process are mysterious.  They are
present in huge numbers in eukaryotic genomes.
\item[\emph{Alu} insertions] are SINEs that contain roughly 300bp and are
present in huge numbers within the human genome.
\item[How many?] \emph{Alu}s constitute 10.6\% of the genome, or just
over 290~Mb \citep{Lander:N-409-860}.
\item[Probably neutral]
(Substitution rate is comparable to that of other non-coding
genomic regions.)
\end{description}

\section{The ``master gene'' model}

It is thought that only a few \emph{Alu}s are capable of transposing.
These give rise to all the others.  Most of the \emph{Alu} elements in
our genomes are inert.

\paragraph{Families of \emph{Alu}s} are defined on the basis of mutations that
(presumably) accumulated in the master copies.  Many members of
younger families are polymorphic in the human gene pool.  This makes
them especially interesting in intra-specific population genetics.

\section{Properties that make \emph{Alu}s interesting}
\begin{enumerate}
\item
Insertions are seldom if ever lost cleanly
\begin{itemize}
\item They could be deleted, but this would ordinarily remove not only
the \emph{Alu} but also some largish region around it.  Such a loss would not
go undetected.
\item They are degraded by point mutations, but this process is very
slow.  If the mutation rate is $10^{-9}$ per nucleotide per year, then half
the the sites should still be unmutated after 700 million years.
\end{itemize}
\item
We know the ancestral state: \emph{Alu}-absent
\item
At any given site in the genome, the probability per generation that
an insertion will occur is very low.
\item
We don't have to worry about insertion occurring twice at the same
place in the genome.
\item
There are many \emph{Alu}s, and we don't have to estimate a mutation rate
separately for each locus.

(Maybe we should: \emph{Alu}s seem slightly more likely to insert in AT-rich
regions.) 
\item
Since there are so many of them, we can ignore the ones that are
closely linked.  Unlinked loci have nearly independent gene
genealogies. 
\item
Consequences:  Theory is simple, and it is easy to aggregate over
loci.  Genealogical trees are never plagued by shared-derived
characters. 
\end{enumerate}

\section{Average \emph{Alu} frequencies}

Table \ref{tab.alumeans} shows the mean frequency of \emph{Alu} insertions in
several populations.  These data raise two questions:
\begin{itemize}
\item
Why is the mean \emph{Alu} frequency roughly 0.5?
\item
Why is it lower in Africa than elsewhere?
\end{itemize}
\citet{Bulayeva:HB-75-837} attempt to answer these questions.
Their analysis is summarized in the remainder of this section.


\begin{table}
\caption{Sample sizes and allele frequencies}
\label{tab.alumeans}
\centering
\begin{tabular}{lrr}
\hline
           & Haploid & Mean freq.\\\
           & sample  & of ``+''\\ 
Population & size    & allele \\
\hline
\hline
Africa    & 302 & 0.46251\\
Asia      & 154 & 0.55671\\
Europe    & 236 & 0.55919\\
India     & 728 & 0.54373\\
\hline
\multicolumn{3}{l}{Data from Watkins et al.{} \cite{Watkins:GR-13-1607}}
\end{tabular}
\end{table}

The first of these questions has a simple answer.  Recall from the
discussion of the site frequency spectrum that, at mutation-drift
equilibrium, the frequency of sites that divide a sample into $i$
mutants and $K-i$ non-mutants is proportional to $1/i$.  But we
ascertain \emph{Alu} loci by looking at a single haploid genome.  This
is equivalent to saying that alleles with frequency $p$ should occur
with probability proportional to $1/p$.  Meanwhile, we ascertain
\emph{Alu} loci by looking at a single haploid genome.  If a locus has
allele frequency $p$, we ascertain it with probability $p$.  Thus, the
probability that a locus in our sample will have allele frequency $p$
is proportional to $1/p \times p = 1$.  In other words, our sample
should contain equal numbers of loci in every frequency category.  The
average of this uniform distribution is 1/2, in good agreement with
the data.

\subsection{Tests of two hypotheses}

We are already in a position to test two hypotheses from the literature.

\paragraph{Homo erectus diaspora (HED) hypothesis} is an otherwise-plausible
hypothesis \cite{Hawks:MBE-17-2} that is falsified by these data:
\begin{itemize}
\item
\emph{Homo erectus} dispersed out of Africa 1.8 mya, or 72,000 generations
ago. 
\item
Consequently, many nuclear gene trees are roughly this deep.
\item
Neutral theory says that expected time to LCA is $4N_e$ generations.
\item
Setting $4N_e = 72,000$ gives $N_e = 18,000$.
\item
This is in fair agreement with $N_e$ as estimated from a variety of nuclear
loci. 
\item
Simulations show that, under this hypothesis, the world mean \emph{Alu} frequency
would be smaller than 0.19.  This did not happen.
\end{itemize}

\paragraph{Pleistocene population explosion} 
\begin{itemize}
\item Human mitochondrial mismatch distributions suggest that population
  expanded by several-hundred-fold between 30~kya and 130~kya.
\item This would lead to an excess of low-frequency \emph{Alu}s.
\item Simulation shows that a history in which
human numbers expanded from 3,000 to 300,000 at 5,000 generations ago would
lead to a world mean \emph{Alu} frequency between 0.30 and 0.40 in samples of 100
\emph{Alu}s.  
\item This is also excluded by the data.
\end{itemize}

\subsection{Why are \emph{Alu} frequencies higher outside Africa?}

To answer this question, we need a model.

\paragraph{Model}
\begin{itemize}
\item 
An ancestral pop, in which \emph{Alu} has frequency $\pi'$.
\item
Daughter pops $A$ and $B$, in which \emph{Alu} frequencies are $x'_A$ and $x'_B$. 
\item
Since diaspora, $A$ and $B$ have regularly exchanged migrants.
\item
Time is short: no \emph{Alu}s have inserted since diaspora.
\item 
$x'_A$ and $x'_B$ differ from $\pi'$ because of genetic drift.
\item
We can only estimate $\pi'$, $x'_A$, and $x'_B$, we can't observe them
directly. 
\item
Our estimates are biased upward.
\item Why? Because we discover \emph{Alu} loci by observing a single haploid
  genome drawn from population $A$.  We are more likely to discover loci whose
  ``+'' allele is common in $A$.
\item
The biased versions of the allele frequencies are $\pi$, $x_A$, and $x_B$.
\end{itemize}

\begin{figure}
\setlength{\unitlength}{1.3mm}
\centering
\begin{picture}(100,68)
\thicklines
\put(27.5,50){\framebox(48,18){\shortstack{\normalsize Ancestral Population\\
\small Unascertained \emph{Alu} frequencies $\pi '$\\
\small Ascertained \emph{Alu} frequencies $\pi$}}}

\put(0,0){\framebox(48,18){\shortstack{\normalsize Daughter Population A\\
\small Unascertained \emph{Alu} frequencies $x_A'$\\
\small Ascertained \emph{Alu} frequencies $x_A$}}}

\put(55,0){\framebox(48,18){\shortstack{\normalsize Daughter Population B\\
\small Unascertained \emph{Alu} frequencies $x_B'$\\
\small Ascertained \emph{Alu} frequencies $x_B$}}}

\put(50,49.75){\line(-1,-1){31.5}}
\put(50,49.75){\line(1,-1){31.5}}

\put(50,29){\makebox(0,0)[b]{Gene flow}}
\put(50,28){\vector(-1,0){21}}
\put(50,28){\vector(1,0){21}}
\end{picture}
\caption{Model of diaspora of contemporary populations from ancient source}
\label{fig:model}
\end{figure}

\paragraph{$R$-statistics}  Define
\begin{eqnarray*}
  R_{AA} &=& \frac{E[(x'_A - \pi')^2]}{\pi'(1 - \pi')}\\
  R_{AB} &=& \frac{E[(x'_A - \pi')(x'_B - \pi')]}{\pi'(1 - \pi')}
\end{eqnarray*}
$R_{AA}$ is analogous to Wright's $F_{ST}$ and measures the amount of drift
that $A$ has experienced since the diaspora.  $R_{AB}$ is a normalized
covariance.  It is positive if $A$ and $B$ exchange migrants and zero
otherwise. 

\paragraph{Results}
Harpending (unpublished) shows that
\begin{eqnarray}
E[x_A] &=& E[\pi] + R_{AA}E[1 - \pi]\label{eq.xA}\\
E[x_B] &=& E[\pi] + R_{AB}E[1 - \pi]\label{eq.xB}
\end{eqnarray}
These results are interesting because they express $R$-statistics in terms of
means.  Ordinarily, one needs variances, which are harder to estimate than
means.  We could estimate the expectations by averaging over loci.

To use these formulas with data, Bulayeva et al.{} assume
\begin{itemize}
\item The ancestral population was African.
\item The African population has remained large, so Africa has experienced
  little genetic drift, and $\pi$ is approximately equal to the African allele
  frequency in the data.
\item The \emph{Alu} polymorphisms were ascertained in a European, so that $x_{A}$
  equals the European allele frequency in the data.
\end{itemize}
The last assumption is needed because \emph{Alu} loci were discovered in the
published sequence of the Human Genome Project, and the source of this
sequence is not publically known.

Now, equations~\ref{eq.xA} and~\ref{eq.xB} indicate that \emph{Alu} allele
frequencies should be higher in non-African populations than in Africa.
This is just what the data show.  Furthermore, the difference provides an
estimate of $R_{AA}$:
\begin{eqnarray*}
\hat{R}_{AA} &=& \frac{E[x_A - \pi]}{E[1 - \pi]}\\
   &=&  \frac {0.56 - 0.46} {0.55} = 0.18
\end{eqnarray*}
where the expectations have been approximated by averages over loci.

For an isolated population, $R_{AA}$ follows
\[
R_{AA}(t) = 1-e^{-t/2N_e}
\] 
where $t$ is time in generations and $N_e$ is the effective size of the
population \cite{Crow:IPG-70}.  If $R=0.18$, then $t/2N$ = .22.  If the
diaspora occurred 40,000 years (or 1600 generations) ago, the European
effective size would be $N_e = 4000$, and the census size perhaps 10,000.

\paragraph{Comments}
\begin{itemize}
\item The estimate of $N_e$ is in fair agreement with others, even though the
  method of analysis is very different.
\item The method requires knowledge of ancestral state.
\item Bias is ordinarily bad, but here it is crucial to the analysis.
\item Thus, this analysis makes use of the unusual features of \emph{Alu}
  elements. 
\end{itemize}

\section{The study of Sherry et al}

This section will not be used in the 2004 version of the course.

Sherry et al.{} \cite{Sherry:G-147-1977} report a series of \emph{Alu}
polymorphisms that were ascertained in HeLa cells.  (This means that an
element did not appear in the study unless it was present in HeLa.)  They also
ignored all elements that were present in chimpanzees.  (I don't know how many
chimpanzees they looked at.)  They then screened a panel of 122 humans for
each of the 57 elements that they had identified.  Of the 57 elements
identified, 13 turned out to be polymorphic in humans.  The remaining 44 were
fixed in humans and absent in chimps.

\begin{figure}
\rule{0pt}{5in}
\caption{Figure 1 from Sherry et al\label{fig.sherry}}
\end{figure}

\subsection{Consequences of insertions in different parts of the gene
tree} 
\begin{enumerate}
\item
A locus that is polymorphic in humans inserted inside the human gene
tree.  This is the portion labeled ``a'' in figure~\ref{fig.sherry}.
\item
A locus that is fixed in humans and absent in chimps inserted along
the branch leading from the chimp/human MRCA to the human MRCA.  In
the figure, this is the sum of parts~b and~c.
\end{enumerate}

\subsubsection{Part~a of the tree}

The expected length of part~a of the tree (including all branches) is
\[
L_a = 4N \sum_{i=1}^{K-1} \frac{1}{i}
\]
as shown in section~\ref{sec.nMutations} and it's depth is
\[
T_a = 4N (1 - 1/K)
\]
as shown in section~\ref{sec.treedepth}.  Here, $K$ is the number of
haploid genomes in the sample and equals 246.  Furthermore, these DNA
polymorphisms are diploid, so $N$ can be interpreted as the population
size.  Thus,
\begin{eqnarray*}
L_a &=& 4N \times 6.08 \approx 24 N \quad \hbox{generations}\\
T_a &=& 4N (1 - 1/246) \approx 4 N \quad \hbox{generations}
\end{eqnarray*}

\subsubsection{Part~b of the tree}

If the human/chimp speciation event occurred 4.5 myr (perhaps 225,000
generations) ago, then the length of part~b is
\[
T_b = 225,000 - 4N \quad \hbox{generations}
\]

\subsubsection{Part~c of the tree}

This part of the tree represents the coalescence time of a pair of
genes within the species ancestral to humans and chimps.  On average,
it should equal
\[
T_c = 2N_c \quad \hbox{generations}
\]
where $N_c$ is the size of the ancestral species.  Takahata et al.{}
\cite{Takahata:TPB-48-198} estimate that $N_c \approx 100,000$.  If
this is right, then
\[
T_c = 200,000 \quad \hbox{generations}
\]

\subsection{Data analysis}

The total branch length is
\[
L_a + T_b + T_c = 20 N + 425000
\]
The sample includes insertions that occur anywhere within these
branches.  Only a fraction of these mutations are polymorphic,
however: those occuring in region~a, which has length $24 N$.  The
fraction of polymorphic insertions ought to equal (on average) the
fraction of the total branch length that holds polymorphic
insertions.  In other words,
\[
\frac{13}{57} = \frac{L_a}{L_a + T_b + T_c}
= \frac{24 N}{20 N + 425000}
\]
This gives
\[
N = 4986
\]
as an estimate of the effective human population size.  This is very
close to the estimate obtained from mtDNA.

\section{Ascertainment bias}

Unfortunately, the estimate just obtained is biased.  The problem is
that we only included \emph{Alu}s that were present in HeLa cells.

% -*-latex-*-
%     theta         mn        tau          K
%    5.0000     0.0000        Inf          1
%Sample size   = 10
%Mutation rate = 0
%% Tree in PicTeX Format %%
\begin{figure}
\begin{center}
\mbox{\beginpicture
\def\normal{\setsolid \linethickness 0.4pt}
\def\haploid{\setdashes}
\def\line2{\setdots <2pt> }
\small
\valuestolabelleading=.4\baselineskip
\setcoordinatesystem units <0.568035in, 0.2222in> point at 0 0
\setplotarea x from 0.000000 to 7.041821, y from 0.000000 to 9.000000
\plotheading{Locus A}
\put {$A_4$} [l] <2pt,0pt> at 7.041821 5.656250
\put {$\bullet$} at 7.041821 5.656250
%%%%%% LOCUS A %%%%%%%%
\putrule from 0.000000 0.000000 to 0.021850 0.000000
\putrule from 0.000000 1.000000 to 0.021850 1.000000
\putrule from 0.021850 1.000000 to 0.021850 0.000000
\putrule from 0.021850 0.500000 to 0.620190 0.500000
\putrule from 0.000000 2.000000 to 0.620190 2.000000
\putrule from 0.620190 2.000000 to 0.620190 0.500000
\putrule from 0.620190 1.250000 to 2.860713 1.250000
\haploid
\putrule from 0.000000 3.000000 to 0.612722 3.000000
\normal
\putrule from 0.000000 4.000000 to 0.320973 4.000000
\putrule from 0.000000 5.000000 to 0.320973 5.000000
\putrule from 0.320973 5.000000 to 0.320973 4.000000
\putrule from 0.320973 4.500000 to 0.612722 4.500000
\haploid
\putrule from 0.612722 3.000000 to 0.612722 3.750000
\normal
\putrule from 0.612722 3.750000 to 0.612722 4.500000
\haploid
\putrule from 0.612722 3.750000 to 2.382189 3.750000
\normal
\line2
\putrule from 0.000000 6.000000 to 2.382189 6.000000
\putrule from 2.382189 4.875000 to 2.382189 6.000000 
\normal
\put {$\bullet$} at 2.382189 4.875000 
\put {$A_3$} [r] <-4pt,0pt> at 2.382189 4.875000 
\haploid
\putrule from 2.382189 3.750000 to 2.382189 4.875000
\putrule from 2.382189 4.875000 to 2.860714 4.875000
\putrule from 2.860714 3.062500 to 2.860714 4.875000 
\normal
\putrule from 2.860714 1.250000 to 2.860714 3.062500 
\haploid
\putrule from 2.860714 3.062500 to 5.633456 3.062500
\normal
\putrule from 0.000000 7.000000 to 1.909067 7.000000
\putrule from 0.000000 8.000000 to 1.909067 8.000000
\putrule from 1.909067 8.000000 to 1.909067 7.000000
\putrule from 1.909067 7.500000 to 5.440119 7.500000
\putrule from 0.000000 9.000000 to 5.440119 9.000000
\putrule from 5.440119 9.000000 to 5.440119 7.500000
\putrule from 5.440119 8.250000 to 5.633456 8.250000
\haploid
\putrule from 5.633456 3.062500 to 5.633456 5.656250
\normal
\putrule from 5.633456 5.656250 to 5.633456 8.250000 
\haploid
\putrule from 5.633456 5.656250 to 7.041821 5.656250
\normal
\put {$A_1$} [r] <-4pt,0pt> at 0 3
\put {$A_2$} [r] <-4pt,0pt> at 0 6
%%%%%%% LOCUS B %%%%%%%%%%%%%%%%%%%
\valuestolabelleading=.4\baselineskip
\setcoordinatesystem units <1.136in, 0.22222in> point at 0 12
\setplotarea x from 0 to 3.5209, y from 0 to 9
\put {$B_4$} [l] <2pt,0pt> at 3.5209 2.921875
\put {$\bullet$} at 3.5209 2.921875
\plotheading{Locus B}
\putrule from 0.000000 0.000000 to 0.276052 0.000000
\putrule from 0.000000 1.000000 to 0.276052 1.000000
\putrule from 0.276052 1.000000 to 0.276052 0.000000
\putrule from 0.276052 0.500000 to 1.317919 0.500000
\putrule from 0.000000 2.000000 to 0.018119 2.000000
\haploid
\putrule from 0.000000 3.000000 to 0.018119 3.000000
\normal
\haploid
\putrule from 0.018119 2.500000 to 0.018119 3.000000 
\normal
\putrule from 0.018119 2.000000 to 0.018119 2.500000
\haploid
\putrule from 0.018119 2.500000 to 0.099805 2.500000
\normal
\putrule from 0.000000 4.000000 to 0.099805 4.000000
\haploid
\putrule from 0.099805 2.500000 to 0.099805 3.250000
\normal
\putrule from 0.099805 3.250000 to 0.099805 4.000000
\haploid
\putrule from 0.099805 3.250000 to 0.593166 3.250000
\normal
\putrule from 0.000000 5.000000 to 0.040129 5.000000
\line2
\putrule from 0.000000 6.000000 to 0.010551 6.000000
\normal
\putrule from 0.000000 7.000000 to 0.002854 7.000000
\putrule from 0.000000 8.000000 to 0.002854 8.000000
\putrule from 0.002854 8.000000 to 0.002854 7.000000
\putrule from 0.002854 7.500000 to 0.010551 7.500000
\line2
\putrule from 0.010551 6.000000 to 0.010551 6.750000
\normal
\putrule from 0.010551 6.750000 to 0.010551 7.500000 
\line2
\putrule from 0.010551 6.750000 to 0.040129 6.750000
\normal
\line2
\putrule from 0.040129 5.875000 to 0.040129 6.750000 
\normal
\putrule from 0.040129 5.000000 to 0.040129 5.875000
\line2
\putrule from 0.040129 5.875000 to 0.089246 5.875000
\normal
\putrule from 0.000000 9.000000 to 0.089246 9.000000
\line2
\putrule from 0.089246 5.875000 to 0.089246 7.437500
\normal
\putrule from 0.089246 7.437500 to 0.089246 9.000000
\line2
\putrule from 0.089246 7.437500 to 0.593166 7.437500
\normal
\haploid
\putrule from 0.593166  3.250000 to 0.593166 5.343750 
\normal
\put {$B_3$} [r] <-4pt,0pt> at  0.593166 5.343750 
\put {$\bullet$} at  0.593166 5.343750 
\line2
\putrule from 0.593166  5.343750 to 0.593166 7.437500
\normal
\haploid
\putrule from 0.593166 5.343750 to 1.317919 5.343750
\normal
\haploid
\putrule from 1.317919 2.921875 to 1.317919 5.343750 
\normal
\putrule from 1.317919 0.500000 to 1.317919 2.921875 
\haploid
\putrule from 1.317919 2.921875 to 3.5209 2.921875
\normal
\put {$B_1$} [r] <-4pt,0pt> at 0 3
\put {$\bullet$} at 0 3
\put {$B_2$} [r] <-4pt,0pt> at 0 6
\put {$\bullet$} at 0 6
\endpicture}
\end{center}
\caption{Haploid (and diploid) ascertainment on a large and a small 
genealogy\label{fig.hapdip}}
\end{figure}


\section{Exercises}

\begin{exercise}
How would the estimate of $N$ have changed if 26 of the 57 \emph{Alu}s has
been polymorphic?
\end{exercise}


%%% Local Variables: 
%%% mode: latex
%%% TeX-master: "main"
%%% End: 
