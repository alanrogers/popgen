Beginning in the 1960s, it became possible to compare the amino-acid
sequences of the same protein in different organisms. These sequences
are so similar that one can align the proteins of very different
organisms. One can even align many plant proteins with those of
animals. It quickly became clear that some proteins (such as
fibrinopeptides) vary more among species than do others (such as
histones). Because these differences presumably accumulated over
evolutionary time, proteins that vary a lot are said to ``evolve
rapidly.'' But we are seldom able to watch proteins change over
time. When we say that one evolves faster than another, we really mean
that it varies more across species. This variation in evolutionary
rate is one of the observations that the neutral theory was designed
to explain.

This variation in rate is apparent in comparisons not only between
proteins but also within them. In many proteins, some portions of the
amino-acid sequence evolve faster than others. The neutral theory also
sheds light on these differences.

In the 1970s and 1980s, it became possible to compare the DNA
sequences of different organisms, and these comparisons confirmed the
pattern that had been observed in amino-acid sequences. In addition,
other comparisons became possible: it turned out that portions of the
genome that encode protein evolve more slowly than those (such as
introns and pseudogenes) that do not. Furthermore, the 3rd position in
each codon evolves more rapidly than the 1st and 2nd positions. The
neutral theory also made sense of these observations.

In addition to these differences in rate of evolution, there are also
other aspects to the pattern in molecular data. First, differences
seem to accumulate at a relatively constant rate. In a given
protein, the number of amino-acid differences between a human and
another animal is roughly proportional to the time since we last
shared a common ancestor with that animal. The neutral theory provides
an explanation for this fact too.

Finally, there is the hierarchical structure in molecular data. For
example, in the protein cytochrome C, all mammals differ from all
birds at about 10 amino acids, mammals and birds differ from all fish
at about 15 amino acids, all vertebrates differ from all insects at
20--30, and so on. The neutral theory accounts for this too.

Here are some questions that might appear on quizzes:
\begin{cenum}
  \item Within the amino-acid sequence of a typical protein, some parts
    of the sequence evolve rapidly, and other parts evolve slowly. Why
    (according to the neutral theory) does this happen? Is it because
    selection is favoring change in the rapidly-changing parts or
    opposing it in the slowly-changing parts?
  \item Some proteins evolve faster than others. How does the
    neutral theory account for this? Is selection favoring change in
    rapidly-evolving proteins or opposing it in the slowly evolving
    ones?
  \item Protein-coding genomic regions evolve more slowly than introns
    and the non-coding regions between genes. Pseudogenes evolve very
    rapidly. How does the neutral theory account for this?
  \item According to the neutral theory, the rate at which one
    nucleotide replaces another is approximately equal to (a)~the
    mutation rate time population size, (b)~the mutation rate,
    (c)~population size, (d)~the ratio of mutation rate to population
    size. 
\end{cenum}

\section{The molecular clock}
The ``molecular clock'' uses known dates to estimate unknown ones by
capitalizing on the approximately constant rate of molecular
evolution. In doing so it confronts several difficulties, one of which
is called \emph{saturation}. This refers to the fact that once a
particular site has mutated, additional mutations at this site would
not increase the count of differences. In what follows, we illustrate
the clock using a concrete example: the date of the last common
ancestor of chimpanzee and human. To estimate this date, we'll use
genetic data from human, chimpanzee, and orangutan. We'll also need an
externally-derived date in order to ``calibrate the clock''---to
estimate the rate of evolutionary change.

Molecular clocks are ordinarily calibrated using fossils, and this
process introduces a considerable undertainty into molecular
dates. Here, we simply assume that the orangutan--human common
ancestor lived 14~My ago.

The genetic data in this example consists of 929 bases of
mitochondrial DNA, sequenced in a human, a chimpanzee, and an
orangutan \citep[Table~1]{Hasegawa:JME-22-160}.  In these data, human
differs from chimpanzee and orangutan at 79 and 149 nucleotide sites,
respectively. Re-expressed as fractions of the total sites, these
numbers become $p_{ch} = 79/929 = 0.085$, and $p_{oh} = 149/929 =
0.1539$.

As discussed above, some of these nucleotide sites have probably
changed more than once, so $p_{oh}$ and $p_{ch}$ underestimate the
number of substitutions (i.e. evolutionary changes) per site. To
correct for this, we use the simplest of several available
methods---that of \citet{Jukes:MPM-69-21}. This method assumes that
each nucleotide state (A, T, G, or C) mutates to each other state at
the same rate.  Although this is an oversimplification, it is good
enough for a numerical example.

Under the Jukes-Cantor model, there is a simple relationship between
the fraction $p$ of sites that differ and the average number $d$ of
substitutions per site:
\[
d = -(3/4)\ln\bigl(1 - (4/3)p\bigr)
\]
By plugging $p_{oh}$ and $p_{ch}$ into the right side of this
equation, we obtain estimates of the corresponding numbers, $d_{oh}$
and $d_{ch}$, of substitutions per site. This gives $d_{oh} = 0.1723$
and $d_{ch} = 0.0903$. These numbers are only slightly larger than
$p_{oh}$ and $p_{ch}$, suggesting that saturation has had only a
modest effect.

We estimate the rate of evolution by dividing $d_{oh}$ by the combined
lengths of the two branches leading from orangutan and human back to
their common ancestor. As this ancestor (by assumption) lived 14~My
ago, the combined branch length is 28~My, and the rate of evolution is
$r = 0.0062$ changes per site per million years.  Given this rate, we
estimate the age of the chimpanzee--human common ancestor as $d_{ch}/2r
= 7.33$~My.

This analysis is not adequate by modern standards. In the first place,
it ignores the facts that different parts of this DNA sequence evolve
at different rates and that, at any given nucleotide site, some
changes are more likely than others. Furthermore, it assumes that the
rate of change has been constant and takes no account of
uncertainties.  Nonetheless, the answer is consistent with current
knowledge about the age of the chimpanzee--human common ancestor.

\begin{cenum}
  \item Fossils imply that species A and C last shared a common
    ancestor 20 million years ago, and we assume that species A and B
    are more closely related to each other than either is to C. After
    correcting for saturation, we estimate the number of substitutions
    per site at $d_{AC} = 0.2$ (for A and C) and $d_{AB} = 0.1$ (for A
    and B). Use these data to estimate the separation time of species
    A and B.

    \bigskip

    \emph{Answer:} $r = 0.2/40 = 0.005$ substitutions per
    myr. Separation time of A and B is $0.1/2r = 10$ myr.
\end{cenum}
