\begin{cenum}
\item Discuss the urn model as a metaphor for genetic drift.  What
  biology is it meant to model?  What are its strengths and
  weaknesses?  How can we tell whether it is useful?

\item What does Gillespie mean by the ``boundary process,'' and why does
 he feel it is worth distinguishing from genetic drift?

\item Contrast the model of infinite alleles to that of infinite sites.
 What biology is each meant to model?  When would you use one rather
 than another?

\item Fitnesses are said to exhibit overdominance when the
  heterozygote has higher fitness than either homozygote.  How does
  this affect the graph of $\bar w$ against $p$?  What about the graph
  of $\Delta p$ against $p$?  Where is the equilibrium?

\item What is genetic draft?  What problem is it meant to solve?  What
  are the strengths and weaknesses of the model?  How can we tell
  whether the model is useful?

\item In class, Henry Harpending described the results of Cochran and
  Ewald, who argue that many chronic diseases are caused by
  undiscovered infectious agents.  Summarize and their argument and any
  weaknesses you see in it.

\item What is linkage disequilibrium, and why is is produced by
  directional selection?

\item Many people have used genetic variation as a basis for
  inferences about population size.  Why does this work?  Why might it
  fail?

\item How do the effects of genetic drift differ from those of genetic
  draft?  How are they similar?

\item List all of the ways in which population size affects the rate
  of substitution of advantageous alleles.  Would you expect adaptive
  evolution to be faster in a large population or a small one?

\item
While inbreeding is ``bad'' for individuals, it may be ``good'' for
populations in the sense that it exposes deleterious mutations to selection.

\item
In what sense is a highly inbred population better off than a
population with low levels of inbreeding?  Or is it not better off?
Why?

\item
There was some discussion in class about consequences of sudden breakup
of isolates and population-wide reduction in level of inbreeding.  What was
this dicussion all about?  Back up your answer with algebra.

\item
Consider the opposite situation: a sudden increase in level of
inbreeding?  What would the consequences be?  

\item
Archaeological data indicate that, all over the world, people got more
frail and sickly after their populations turned to agriculture.  What,
if anything, might this have to do with inbreeding, and why?
(\emph{Hint:} Individual property is more important among
agriculturalists than among foragers.)

\item
Genetic data so far all agree that the effective size of our species is
10,000.  It may be 5,000 or it may be 25,000 but it certainly is not 1000
and it is certainly not 100,000.  What does this observation mean?  What
could account for this observation given that there are about 6 billion
humans on earth.

\item 
 This should be answered with a long paragraph or two in the space below.  If
 necessary, continue your answer on the back of this page.  State in words the
 hypothesis of Cochran et al.\ about infectious causation of chronic and
 degenerative diseases.  How does their hypothesis conflict with current
 medical and public health orthodoxy?
\end{cenum}

\section{Post-midterm}
\begin{cenum}

\item In class, I argued that genetic drift is a form of inbreeding.
 What did I mean?

\item What is wrong with inbreeding?  If it is so bad, why do
 some species breed with siblings, or with themselves?

\end{cenum}
