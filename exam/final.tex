% -*-latex-*-
\documentclass[11pt,twocolumn]{article}
\usepackage{etex,pictex,paralist,alltt,float,amsmath}
\usepackage[letterpaper,margin=1in]{geometry}
\usepackage{ifthen}
\usepackage{answers}
\Opensolutionfile{answers}
\Newassociation{answer}{Answr}{answers}
\newcounter{iskey}
\setcounter{iskey}{0}  % 1 for answer key; 0 for exam
\newtheorem{exercise}{}
\newcommand{\G}{\mathcal{G}}
\newcommand{\Het}{\mathcal{H}}
\newcommand{\monthname}[1]{\ifcase#1\or January\or February\or
  March\or April\or May\or June\or July\or August\or
  September\or October\or November\or December\fi}
\renewcommand{\today}{\monthname{\month}\space\number\year}
\setcounter{secnumdepth}{0}
\begin{document}
\begin{center}
\Large
\ifthenelse{\theiskey=1}{\textbf{Answer Key}}{%
Name:\hspace{2.25in}\hfill\mbox{}}\\[1ex]
\textbf{Final Exam Anth/Biol 5221}\\
\today
\end{center}

\paragraph{Instructions}
Write your name at the top of page~1 and your answers on the attached
blank sheets. Number your answers.

Before doing any calculations, ask yourself whether calculations are
really needed. You may be able to answer the question faster without
any calculation at all.

Each question is worth 5 points.  You have 2~hours.

\subsubsection*{Python, probability, heterozygosity and selection}

\begin{exercise}
In a toss of two fair dice, one red and one black, what is the
probability of observing a red~4 \emph{and} a black~5?
\begin{answer}
$1/6 \times 1/6 = 1/36 \approx 0.028$
\end{answer}
\end{exercise}

\begin{exercise}
An urn contains 40 black balls and 60 red balls.  You choose a
ball at random, put it back, and choose another at random.  What are
the possible outcomes and their probabilities?
\begin{answer}
  There are two right answers. First, the student might choose to
  think in terms of ordered outcomes:
  \begin{center}
    \begin{tabular}{rrr}
      Ball~1 & Ball~2 & Probability\\ \hline
      red    & red    & $\frac{6}{10} \times \frac{6}{10} = 0.36$\\
      red    & black  & $\frac{6}{10} \times \frac{4}{10} = 0.24$\\
      black  & red    & $\frac{6}{10} \times \frac{4}{10} = 0.24$\\
      black  & black  & $\frac{4}{10} \times \frac{4}{10} = 0.16$\\
    \end{tabular}
  \end{center}
  Second, in terms of unordered outcomes:
  \begin{center}
    \begin{tabular}{rr}
      Pair  & Probability\\ \hline
      red/red    & $\frac{6}{10} \times \frac{6}{10} = 0.36$\\
      red/black  & $2 \times \frac{6}{10} \times \frac{4}{10} = 0.48$\\
      black/black  & $\frac{4}{10} \times \frac{4}{10} = 0.16$\\
    \end{tabular}
  \end{center}

\end{answer}
\end{exercise}

\begin{exercise}
  In the urn model of genetic drift, there are $2N$ balls in the urn
  of which a fraction $p$ are black.  Suppose we draw $2N$ balls from
  the urn with replacement.  Of the balls that we draw, a random
  number $X$ are black.  What are the \emph{mean} and \emph{variance}
  of $X$?  (Your answers should be in terms of $2N$ and $p$.)
\begin{answer}
$m=2Np$, $V=2Np(1-p)$
\end{answer}
\end{exercise}

\subsubsection*{Gene genealogies}

\begin{exercise}
  What is the hazard of a coalescent event within an interval
  containing 4 lines of descent (i.e. 4 ancestors of sampled gene
  copies) in a randomly-mating population of $N$ diploid individuals?
\begin{answer}
  \[
  h = \frac{4 \times 3}{4N} = 3/N
  \]
\end{answer}
\end{exercise}

\begin{exercise}
  What is the expected depth of a gene genealogy (i.e.\ the expected
  age of the last common ancestor) for a sample of $K$ gene copies
  drawn at random from a randomly-mating population of $N$ diploid
  individuals?
  \begin{answer}
The expected depth of the gene genealogy is $4N(1-1/K)$ generations.
  \end{answer}
\end{exercise}

\begin{exercise}
  What is the expected number of segregating sites in a sample of $K$
  gene copies? (Assume selective neutrality, constant population size,
  diploidy, and the model of infinite sites. Your answer should be in
  terms of two parameters, $\theta = 4Nu$ and $K$.)
\begin{answer}
\[
E[S] = \theta \{1 + 1/2 + 1/3 + \cdots + 1/(K-1)\}
\]
\end{answer}
\end{exercise}

\subsubsection*{Molecular evolution}

\begin{exercise}
Do we expect more adaptive evolution (fixations of advantageous
mutations) in large or in small populations, other things being equal?
Why?
\begin{answer}
We expect more in large populations, because the number if new
advantageous mutants is proportional to population size, but the
fixation probability for such mutants is essentially independent of
population size. The rate of adaptive evolution is thus proportional
to population size.

Some students may want to answer this question more formally, so here
is the formal version: let $u$ represent the mutation rate for
advantageous alleles whose fitness in heterozygotes is $1+s/2$,
relative to that of the wild-type allele. The number of such mutations
in the population as a whole is $2Nu$, and each of them has
probability $s$ of ultimate fixation. For this category of adaptive
mutations, the expected number fixed per generation is therefore
$2Nus$, an increasing function of population size.
\end{answer}
\end{exercise}

\begin{exercise}
  Why do large populations tend to have more genetic variation than
  small ones?  (Answer in words, not in formulas.)
\begin{answer}
There are several ways to answer this question, including
\begin{inparaenum}[(a)]
\item drift is weaker in large populations;
\item in large populations, random pairs of gene copies are distantly
  related and therefore have deep genealogies, which allow lots of
  time for mutation;
\item there is more random inbreeding in small populations.
\end{inparaenum}
Accept anything that makes sense.
\end{answer}
\end{exercise}

\begin{exercise}
Suppose a rare mutation increases the fitness of heterozygous
carriers---their fitness is 1.01 times that of wild-type
homozygotes. The fitness of homozygous mutants is 1.02 times that of
the wild type. What is the probability that this mutation will spread
to fixation? (Feel free to use the approximate formula rather than
Kimura's detailed version. If you use Kimura's version, then assume
$N$ is large.)
\begin{answer}
The mutation will fix with probability of roughly 0.02.
\end{answer}
\end{exercise}

\begin{exercise}
Consider the model of selection in Fig.~\ref{fig.overdom}. (a)~Which
values of $p$ are equilibria? (b)~Which of these equilibria are
stable?
\begin{answer}
There are three equilibria, at $p=0$, 0.4, and 1. The intermediate
equilibrium is stable, and the other two are unstable.
\end{answer}
\end{exercise}

\begin{figure}
{\centering\mbox{\beginpicture
\setcoordinatesystem units <4cm, 79.4cm>
\setplotarea x from 0 to 1, y from -0.0265 to 0.0136
\axis left label {$\Delta_s p$}
  ticks numbered from -0.02 to 0.01 by 0.01 /
\axis bottom label {$p$}
  ticks withvalues 0 {0.4} 1 / at 0 0.4 1 / /
\setdots
\putrule from 0 0 to 1 0
\setsolid
%       p       dp
\plot
   0.0000   0.0000
   0.0417   0.0060
   0.0833   0.0101
   0.1250   0.0126
   0.1667   0.0136
   0.2083   0.0133
   0.2500   0.0118
   0.2917   0.0094
   0.3333   0.0062
   0.3750   0.0025
   0.4      0
   0.4167  -0.0017
   0.4583  -0.0061
   0.5000  -0.0104
   0.5417  -0.0146
   0.5833  -0.0184
   0.6250  -0.0217
   0.6667  -0.0242
   0.7083  -0.0259
   0.7500  -0.0265
   0.7917  -0.0259
   0.8333  -0.0240
   0.8750  -0.0205
   0.9167  -0.0155
   0.9583  -0.0087
   1.0000  -0.0000
/
\endpicture}\\}
\caption{A model of selection. $\Delta_s p$ is the per-generation
  change in the frequency, $p$, of allele $A_1$. In other words, $p' =
p + \Delta_s p$, where $p$ is the allele frequency in one generation,
and $p'$ is that in the next. The allele is
increasing in frequency when $\Delta_s p > 0$ and decreasing when
$\Delta_s p < 0$.}
\label{fig.overdom}
\end{figure}

\subsubsection*{Multiple loci}

\begin{exercise}
\label{prob.2loc-xD}
In Gillespie's standard setup, there are four gamete types (or
haplotypes) $A_1B_1$, $A_1B_2$, $A_2B_1$, and $A_2B_2$, with
frequencies $x_1$, $x_2$, $x_3$, and $x_4$.  Suppose that the gamete
types have frequencies $(x_1, x_2, x_3, x_4) = (1/8, 3/8, 3/8, 1/8)$.
What is the value of $D$, the coefficient of linkage disequilibrium?
\begin{answer}
  \[
  D = x_1 x_4 - x_2 x_3 = 1/64 - 9/64 = -8/64 =  -1/8
  \]
\end{answer}
\end{exercise}

\begin{exercise}
  In another population, suppose the recombination rate is $c=1/11$
  and that $D=1/10$ in generation~0.  What would $D$ be after 1
  generation? (Assume selective neutrality, random mating, and an
  infinite population.)
\begin{answer}
After 1 generation, $D = 1/11$.
\end{answer}
\end{exercise}

\begin{exercise}
Soon after a selective sweep has finished, \begin{inparaenum}[\it (a)]
\item what would you expect of nucleotide diversity at nearby loci?
  More diversity, less diversity, or about the same amount as before
  the sweep?
\item Is this effect larger or smaller in regions of low recombination?
\end{inparaenum}
\begin{answer}
(a)~Diversity should be lower at nearby loci. (b)~This effect is
larger in regions of low recombination.
\end{answer}
\end{exercise}

\subsubsection*{Inbreeding and population structure}

\begin{exercise}
What is inbreeding depression and why is it thought to
occur?
\begin{answer}
  Inbred offspring are often smaller, less healthy, and lower in
  Darwinian fitness than their outbred counterparts. This is
  \emph{inbreeding depression}. It happens either because of fitness
  overdominance (heterozygote advantage) or because deleterious
  alleles tend to be recessive. Accept either answer.
\end{answer}
\end{exercise}

\begin{figure}
{\centering% Darwin-Wedgewood Genealogy
\mbox{\beginpicture
\setcoordinatesystem units <0.85cm,1cm>
\setplotarea x from -1 to 7, y from 0 to 6
%\setdots
%\axis invisible bottom ticks andacross numbered from -1 to 7 by 1 /
%\axis invisible left ticks andacross numbered from 0 to 6 by 1 /
%\setsolid
\setplotsymbol ({\footnotesize .})
\arrow <10pt> [.2,.67] from 1 5.5 to 1 4.5
\arrow <10pt> [.2,.67] from 1 5.5 to 6 4.5
\arrow <10pt> [.2,.67] from 6 5.5 to 1 4.5
\arrow <10pt> [.2,.67] from 6 5.5 to 6 4.5
%
\arrow <10pt> [.2,.67] from 1 3.5 to 1 2.5
\arrow <10pt> [.2,.67] from 6 3.5 to 6 2.5
%
\arrow <10pt> [.2,.67] from 1 1.5 to 3.5 0.5
\arrow <10pt> [.2,.67] from 6 1.5 to 3.5 0.5
%
\put {\frame <1ex> {\sf Josiah Wedgewood}} at 1 6
\put {\frame <1ex> {\sf Sarah Wedgewood}}  at 6 6
\put {\frame <1ex> {\sf Susannah Wedgewood}} at 1 4
\put {\frame <1ex> {\sf Josiah Wedgewood II}} at 6 4
\put {\frame <1ex> {\sf Charles Darwin}} at 1 2
\put {\frame <1ex> {\sf Emma Wedgewood}} at 6 2
\put {\frame <1ex> {\sf George Darwin}} at 3.5 0
\endpicture}
\\}
\caption{Genealogy of George Darwin}
\label{fig.darwedge}
\end{figure}

\begin{exercise}
Figure~\ref{fig.darwedge} shows the genealogy of George Darwin, one of
Charles and Emma Darwin's sons. What is his inbreeding coefficient?
\begin{answer}
$1/16$
\end{answer}
\end{exercise}

\begin{exercise}
Suppose that a population consists of several subdivisions (or groups)
of equal size, within each of which mating is at random.  The mean of
group allele frequencies is 0.25 and their variance is 0.1875.  What is
the heterozygosity in the population as a whole? (Hint: use Wahlund's
formula.)
\begin{answer}
  The average heterozygosity is zero. The heterozygosity of a
  random-mating population with the same allele frequency would be
  $H_0 = 2 \times 0.25 \times (1 - 0.25) = 0.375$. By Wahlund's
  formula, the heterozygosity of the subdivided population is $H_0 -
  2V = 0$, where $V = 0.1875$ is the variance of group allele
  frequencies.
\end{answer}  
\end{exercise}  

\subsubsection*{Quantitative traits}

A selection experiment was undertaken using a population of outbred
plants.  Among unselected parents (i.e.\ the whole population), the
mean leaf length was 14~cm.  The mean was 18~cm among selected parents
and 17~cm among offspring.
\begin{exercise}
What was the selection differential ($S$)?
\begin{answer}
$S = 18 - 14 = 4$~cm.
\end{answer}
\end{exercise}

\begin{exercise}
What was the response ($R$) to selection?
\begin{answer}
$R = 17 - 14 = 3$~cm.
\end{answer}
\end{exercise}

\begin{exercise}
What was the narrow-sense heritability ($h^2$)?
\begin{answer}
$h^2 = R/S = 3/4$.
\end{answer}
\end{exercise}

\begin{exercise}
The average bill depth of Geospiza fortis (Darwin's medium ground
finch) increased by 0.5~mm in one generation (1976 to 1978) in the
population on Isla Daphne Major that has been studied by Peter and
Rosemary Grant and their students and colleagues since the early
1970s. The phenotypic standard deviation of this trait is around 1~mm,
and its heritability has been estimated (from the correlations among
relatives) to be around $h^2 = 0.8$.

Given these numbers, what was the selection gradient $(\beta)$ during
the drought of 1977? (The drought was caused by a severe El Ni{\~ n}o
event that forced the birds to feed on large, hard seeds that they
otherwise wouldn't eat.)
\begin{answer}
The problem says that $h^2 = 0.8$ and that the phenotypic
standard deviation is $\sqrt{V_P} = 1$. This implies that $V_P=1$. The
problem also says that $h^2=0.8$. This implies that $V_A = h^2 V_P =
0.8$. Finally, the response to selection is $R = V_A \beta$, and we
know that $R=0.5$. Thus, $\beta = 0.5/0.8 = 0.625$.
\end{answer}
\end{exercise}

\begin{exercise}
  Suppose that a trait is highly heritable within each of two
  populations and that the mean of the trait is different in the two
  populations. Does this imply that genetic factors account for a
  large fraction of the between-group difference? Why or why not?
\begin{answer}
  It does not. The difference between populations could be entirely
  environmental even if the within-group variance is entirely
  genetic. Or there could be gene-by-environment interaction.  (Accept
  either explanation.)
\end{answer}
\end{exercise}

\Closesolutionfile{answers}
\ifthenelse{\theiskey=1}{%
  \clearpage
  \section{Answer key}
  \mbox{}\\
  \IfFileExists{answers.tex}{\begin{Answr}{1.1}
$1/4$
\end{Answr}
\begin{Answr}{1.2}
$1/4$
\end{Answr}
\begin{Answr}{1.3}
$1/4$
\end{Answr}
\begin{Answr}{1.4}
$1/2$
\end{Answr}
\begin{Answr}{1.5}
3: THH, HTH, and HHT.
\end{Answr}
\begin{Answr}{1.6}
There are 6 ways to choose 2 out of 4: ${4 \choose 2} = \frac{4!}{2!
  \times 2!} = 24/[2 \times 2] = 6$.  They are: HHTT, HTHT, HTTH,
THHT, THTH, and TTHH.
\end{Answr}
\begin{Answr}{1.7}
$3 \times 1/2^3 = 3/8$
\end{Answr}
\begin{Answr}{1.8}
$6 \times 1/2^4 = 6/16 = 3/8$
\end{Answr}
\begin{Answr}{1.9}
$1/6 + 1/6 - 1/36 = 11/36$
\end{Answr}
\begin{Answr}{1.10}
$1/6 + 1/6 = 1/3$.
\end{Answr}
\begin{Answr}{1.11}
$1/6 \times 1/6 = 1/36$
\end{Answr}
\begin{Answr}{1.12}
$1/2 \times 2/5 = 1/5$
\end{Answr}
\begin{Answr}{1.13}
$1/2 \times 3/5 = 3/10$
\end{Answr}
\begin{Answr}{1.14}
There are two ways to answer this question. The first is algebraic:
point out that $X$ is a binomial distribution with parameters $N=3$
and $p=1/2$. This implies that $X$ can take values 0, 1, 2, and 3; it
takes value $x$ with probability $P_x = {3 \choose x} /8$. The second
way to answer the question is to write the answer in tabular form:
\[
\begin{array}{cc}
x & P_x\\ \hline
0   & 1/8\\
1   & 3/8\\
2   & 3/8\\
3   & 1/8\\
\end{array}
\]
\end{Answr}
\begin{Answr}{1.15}
The hard way to answer this question is to evaluate $E[X]$ directly,
using the numerical values of $\Pr[0]$, $\Pr[1]$, $\Pr[2]$, and
$\Pr[3]$. The easy way is to point out that, because $X$ is binomial
with $N=3$ and $p=1/2$, the mean is $E[X]=Np = 3/2$.
\end{Answr}
\begin{Answr}{1.16}
The hard way to answer this question is to evaluate $V[X] = E[X^2] -
E[X]^2$, or some similar expression. The easy way is to point out
that, because $X$ is binomial with $N=3$ and $p=1/2$, the the variance
is $V[X]=Np(1-p) = 3/4$.
\end{Answr}
\begin{Answr}{1.17}
Freq. of sad: $80/200 = 2/5$
\end{Answr}
\begin{Answr}{1.18}
Freq. of rainy: $100/200 = 1/2$
\end{Answr}
\begin{Answr}{1.19}
Freq. of sad given rainy: $70/100 = 7/10$
\end{Answr}
\begin{Answr}{1.20}
Freq. of rainy given sad: $70/80 = 7/8$
\end{Answr}
\begin{Answr}{1.21}
    The quick way to answer this question is to notice that question
    implies that $X$ is a Bernoulli random variable with parameter $p$
    and that therefore $E[X] = p$ and $V[X] = p(1-p)$. It's only
    slightly harder work from first principles. The probability
    distribution of $X$, in tabular form, is
    \[
    \begin{array}{cc}
      X & \Pr[X]\\ \hline
      0 & 1-p\\
      1 & p
    \end{array}
    \]
    This implies that $E[X] = 0 \times (1-p) + 1 \times p = p$. For a
    random variable that takes only the values 0 and 1, it is always
    true that $E[X] = E[X^2]$, because $0^2=0$ and
    $1^2=1$. Consequently, $E[X^2]$ also equals $p$. Recall that, for
    any random variable, the variance equals $V = E[X^2] - E[X]^2$. In
    the current case, this becomes $V = p - p^2 = p(1-p)$.
\end{Answr}
\begin{Answr}{1.22}
$E[Y] = 2p$ and $V[Y] = 2p(1-p)$. There are two simple ways to get
these results:
\begin{enumerate}
\item The question implies that $Y$ is binomial with $N=2$ and
  probability parameter $p$. Consequently, $E[Y]=2p$ and $V[Y] =
  2p(1-p)$.
\item The question says that $Y$ is a sum of two independent values,
  each of which is a Bernoulli random variable with mean $p$ and
  variance $p(1-p)$. Therefore, $E[Y] = 2p$ and $V[Y] = 2p(1-p)$.
\end{enumerate}
There are also harder ways, which involve deriving the properties of
the Binomial or Bernoulli distributions.
\end{Answr}
\begin{Answr}{1.23}
Easy answer:~The distribution is binomial with $N=2$ and probability
parameter $p=0.8$.
Hard answer:~The distribution takes values 0, 1, and 2, with
probabilities 0.04, 0.32, and 0.64.
\end{Answr}
\begin{Answr}{1.24}
Because $X$ is binomial with $N=2$ and probability parameter $p=0.8$,
its mean is $E[X] = 2p = 1.6$.
\end{Answr}
\begin{Answr}{1.25}
Because $X$ is binomial with $N=2$ and probability parameter $p=0.8$,
its, variance is $V[X] = 2p(1-p) = 0.32$ and its standard deviation is
$\sqrt{(V[X])} = 0.566$.
\end{Answr}
\begin{Answr}{1.26}
Answer not provided.
\end{Answr}
\begin{Answr}{1.27}
Pr no deaths: $e^{-0.61} = 0.54$
\end{Answr}
\begin{Answr}{1.28}
Pr one death: $0.61 e^{-0.61}$
\end{Answr}
\begin{Answr}{1.29}
Pr at least one death: $1- e^{-0.61} = 0.46$
\end{Answr}
\begin{Answr}{1.30}
\hbox{Pr 2 deaths}: $0.61^2 e^{-0.61}/2! = 0.101$
\end{Answr}
\begin{Answr}{1.31}
Pr no child: $e^{-2} = 0.135$
\end{Answr}
\begin{Answr}{1.32}
Pr 1 child: $2 e^{-2} = 0.27$.
\end{Answr}
\begin{Answr}{1.33}
Pr at least 1 child: $1-e^{-2} = 0.865$.
\end{Answr}
\begin{Answr}{1.34}
 This is exactly like Fig.~\ref{JEPr-fig.tree} of
    JEPr, which presents the following probability distribution:
\begin{center}
\begin{tabular}{cc}
Event & Prob\\ \hline
RR & 1/6\\
RG & 1/3\\
GR & 1/3\\
GG & 1/6\\
\end{tabular}
\end{center}
where ``R'' and ``G'' stand for ``red'' and ``green'', ``RG'' means
``1st ball red and 2nd green,'' and so on. For the current question,
we translate ``R'' and ``G'' into into ``1'' and ``0,'' and we define
$X$ and $Y$ to equal the values of ball~1 and ball~2. The probability
distribution becomes
\begin{center}
\begin{tabular}{ccc}
$X$ & $Y$ & $\Pr(X,Y)$\\ \hline
1 & 1 & 1/6\\
1 & 0 & 1/3\\
0 & 1 & 1/3\\
0 & 0 & 1/6\\
\end{tabular}
\end{center}
\end{Answr}
\begin{Answr}{1.35}
$E[X] = 1/6 + 1/3 = 1/2$; $E[Y] = 1/6 + 1/3 = 1/2$.
\end{Answr}
\begin{Answr}{1.36}
$E[X^2] = E[Y^2] = 1/2$, so $V[X] = E[X^2]- E[X]^2 = 1/2
- 1/4 = 1/4$. $V[Y]$ is also =1/4.
\end{Answr}
\begin{Answr}{1.37}
$E[XY] = 1/6$, so $C(X,Y) = E[XY] - E[X]E[Y] = 1/6 -
1/2^2 = -1/12 \approx -0.0833$. It makes sense that the covariance
should be negative, because if the first ball is red, then the second
ball is less likey to be red.
\end{Answr}
\begin{Answr}{1.38}
\begin{flushleft}
%-*-latex-*-
\let\put\pictexput
\framebox{\beginpicture
\setcoordinatesystem units <0.85cm, 0.6cm>
\setplotarea x from 0 to 7.5, y from -3 to 4
\plot 0 0 2 1.7 /
\plot 0 0 2 -1.7 /
\put {$1/N$} [br] <-1pt,1pt> at 1 0.85
\put {$(N-1)/N$} [tr] <-1pt,-1pt> at 1 -0.85
\put {R}   [c] at 2.25 2
\put {G} [c] at 2.25 -2
%
\plot 2.5 2 4.5 3 /
\plot 2.5 2 4.5 1 /
\put {0} [br] <-1pt,1pt> at 4 2.75
\put {1} [tr] <-1pt,-1pt> at 4 1.25
\put {R}   [l] <3pt,0pt> at 4.5 3
\put {G} [l] <3pt,0pt> at 4.5 1
%
\plot 2.5 -2  4.5 -1 /
\plot 2.5 -2  4.5 -3 /
\put {$1/(N-1)$} [br] <-1pt,1pt> at 4 -1.25
\put {$(N-2)/(N-1)$} [tr] <-1pt,-1pt> at 4 -2.75
\put {R}   [l] <3pt,0pt> at 4.5 -1
\put {G} [l] <3pt,0pt> at 4.5 -3
%
\put {\lines{First\cr ball}}  [b] at 2.25 3.7
\put {\lines{Second\cr ball}} [b] at 4.75 3.7
\put {$XY$} [b] at 5.75 3.7
\put {$11$} <0pt,-1pt> at 5.75 3
\put {$10$} <0pt,0pt> at 5.75 1
\put {$01$} <0pt,-1pt> at 5.75 -1
\put {$00$} <0pt,0pt> at 5.75 -3
%
\put {Prob} [b] at 7 3.7
\put {0} <0pt,-1pt> at 7 3
\put {$1/N$} <0pt,0pt> at 7 1
\put {$1/N$} <0pt,-1pt> at 7 -1
\put {$(N-2)/N$} <0pt,0pt> at 7 -3
\endpicture}
\let\put\latexput

\end{flushleft}
\end{Answr}
\begin{Answr}{1.39}
  The probability distribution of $X$ and $Y$ is shown on the right
  side of the figure in the answer to exercise~\ref{ex.Ntree}. For
  reference, it looks like this:
    \[
    \begin{array}{ccc}
      X & Y & \Pr\\ \hline
      1 & 1 & 0\\
      1 & 0 & 1/N\\
      0 & 1 & 1/N\\
      0 & 0 & (N-2)/N
      \end{array}
    \]
    Using this distribution,
    \begin{eqnarray*}
      E[X] &=& 1\times 0\\
      && \mbox{} + 1 \times 1/N \\
      && \mbox{} + 0\times 1/N\\
      && \mbox{} + 0\times (N-2)/N\\
      &=& 1/N
    \end{eqnarray*}
    This is also $E[X^2]$, because $X$ takes only the values 0 and 1,
    and $X^2 = X$. Thus, $V[X] = E[X^2] - E[X]^2 = 1/N - 1/N^2 = (1/N)
    (1-1/N)$. An alternative answer: these results follow because $X$
  is a Bernoulli random variable with parameter $p=1/N$.
\end{Answr}
\begin{Answr}{1.40}
  The probability distribution of $X$ and $Y$ is shown on the right
  side of the figure in the answer to exercise~\ref{ex.Ntree}. For
  reference, it looks like this:
    \[
    \begin{array}{ccc}
      X & Y & \Pr\\ \hline
      1 & 1 & 0\\
      1 & 0 & 1/N\\
      0 & 1 & 1/N\\
      0 & 0 & (N-2)/N
      \end{array}
    \]
    Using this distribution,
    \begin{eqnarray*}
      E[Y] &=& 1\times 0\\
      && \mbox{} + 0 \times 1/N \\
      && \mbox{} + 1\times 1/N\\
      && \mbox{} + 0\times (N-2)/N\\
      &=& 1/N
    \end{eqnarray*}
    This is also $E[Y^2]$, because $Y$ takes only the values 0 and 1,
    and $Y^2 = Y$. Thus, $V[Y] = E[Y^2] - E[Y]^2 = 1/N - 1/N^2 = (1/N)
    (1-1/N)$. An alternative answer: these results follow because $Y$
  is a Bernoulli random variable with parameter $p=1/N$.
\end{Answr}
\begin{Answr}{1.41}
After adding a column for the product of $X$ and $Y$, the probability
distribution becomes
\[
\begin{array}{cccc}
  X & Y & XY & \Pr\\ \hline
  1 & 1 & 1 & 0\\
  1 & 0 & 0 & 1/N\\
  0 & 1 & 0 & 1/N\\
  0 & 0 & 0 & (N-2)/N
\end{array}
\]
The expected product of $X$ and $Y$ is
\begin{eqnarray*}
  E[XY] &=& 1\times 0\\
  && \mbox{} + 0 \times (\hbox{\small the rest})\\
  &=& 0
\end{eqnarray*}
Thus, $C[X,Y] = E[XY] - E[X] E[Y] = -1/N^2$
\end{Answr}
\begin{Answr}{1.42}
$2/5$ or 0.4
\end{Answr}
\begin{Answr}{1.43}
$(20/50) \times (19/49)$, or $38/245$, or 0.155.
\end{Answr}
\begin{Answr}{1.44}
normal
\end{Answr}
\begin{Answr}{1.45}
Poisson
\end{Answr}
\begin{Answr}{1.46}
binomial
\end{Answr}
\begin{Answr}{2.1}
$P_{CC} = 80/100 = 0.8$
\end{Answr}
\begin{Answr}{2.2}
$P_{CD} = 15/100 = 0.15$
\end{Answr}
\begin{Answr}{2.3}
$P_{DD} = 5/100 = 0.05$
\end{Answr}
\begin{Answr}{2.4}
By gene counting, $p_C = 175/200 = 0.875$. By the formula, $p_C = 0.8
+ 0.15/2 = 0.875$.
\end{Answr}
\begin{Answr}{2.5}
By gene counting, $p_D = 25/200 = 0.125$. By the formula, $p_D = 0.05
+ 0.15/2 = 0.125$.
\end{Answr}
\begin{Answr}{2.6}
There is more than one correct answer. One approach begins with the
observation that the frequency of $A_1$ is the same as the probability
that a random gene copy chosen from a random individual is allele
$A_1$.  Let us calculate this probability.

With probability $P_{11}$, we choose genotype $A_1A_1$. In this case,
we get allele $A_1$ with probability~1. With probability $P_{12}$, our
individual is $A_1A_2$, and we then get $A_1$ with probability
$1/2$. If we choose $A_2A_2$, we cannot possibly get $A_1$. Thus, the
probability of $A_1$ is
\[
(P_{11} \times 1) + (P_{12} \times 1/2) + (P_{22} \times 0)
 = P_{11} + P_{12}/2
\]
\end{Answr}
\begin{Answr}{2.7}
200
\end{Answr}
\begin{Answr}{2.8}
For allele C, $p = 0.15 + 0.5/2 = 0.4$. For T, $q = 1-p = 0.6$
\end{Answr}
\begin{Answr}{2.9}
$H = 2pq = 2\cdot 0.4 \cdot 0.6 \approx 0.48$.
\end{Answr}
\begin{Answr}{2.10}
$H_{obs} = 0.5$.
\end{Answr}
\begin{Answr}{2.11}
The observed genotype frequencies are $P_{11} = P_{12} = P_{22} =
1/3$. The frequency of $A_1$ is $p = 1/2$.  Given $p$ the three
expected genotype frequencies are $E[P_{11}] = p^2 = 1/4$, $E[P_{12}]
= 2p(1-p) = 1/2$, and $E[P_{22}] = (1-p)^2 = 1/4$.
\end{Answr}
\begin{Answr}{2.12}
Using (\ref{eq.F11}), $F = (P_{11} - p^2)/pq
 = (1/3 - 1/4)/(1/4) = 1/3$. Using (\ref{eq.F12}), $F = -(P_{12} -
 2pq)/2pq =  -(1/3 - 1/2)/(1/2) = 1/3$. Using (\ref{eq.F22}), $F =
 (P_{22} - q^2)/pq = (1/3 - 1/4)/(1/4) = 1/3$.
\end{Answr}
\begin{Answr}{2.13}
$P_{11} = 0$, $P_{12} = 0.069444$, $P_{22} = 0.930556$, $p_1 =
0.034722$, and $F = -0.035971$.
\end{Answr}
\begin{Answr}{2.14}
$P_{11} = 0.726027$, $P_{12} = 0.260274$, $P_{22} = 0.013699$, $p_1 =
0.856164$, and $F = -0.056762$.
\end{Answr}
\begin{Answr}{3.1}
$m=2Np$, $V=2Np(1-p)$
\end{Answr}
\begin{Answr}{3.2}
\[
\frac{1}{2} \times \left(1 - \frac{1}{10}\right) = \frac{9}{20} = 0.45
\]
\end{Answr}
\begin{Answr}{3.3}
\[
\frac{1}{2} \times \left(1 - \frac{1}{1000}\right) = \frac{999}{2000} = 0.4995
\]
\end{Answr}
\begin{Answr}{3.4}
$2N=17.93$.
\end{Answr}
\begin{Answr}{3.5}
$H_0 (1-1/26)^9 (1-1/10)^9 = 0.14$, smaller than the observed value of
$H_t$.  Variation in population size accelerates the decay of
heterozygosity.
\end{Answr}
\begin{Answr}{3.6}
Same as above.  We start with formulas like this:
\begin{eqnarray*}
H_9 &=& H_0 (1-1/26)^9\\
H_{18} &=& H_9 (1-1/10)^9\\
\end{eqnarray*}
Substituting $H_9$ into the second equation gives
\[
H_{18} =  H_0 (1-1/26)^9(1-1/10)^9
\]
You can multiply in any order, so the answer to the second problem is
the same as that of the first.
\end{Answr}
\begin{Answr}{3.7}
$\log(0.1/H_0)/\log(1-1/18) = 28.64$
\end{Answr}
\begin{Answr}{3.8}
\[ t = \frac{\ln(H_t/H_0)}{\ln(1 - 1/2N)} \]
\end{Answr}
\begin{Answr}{3.9}
Here is the solution in Python:
\begin{verbatim}
>>> n = 2.0  # current population size
>>> h = 1.0  # relative heterozygosity
>>> while n < 1024.0:
...     h *= 1-1/(2.0*n) # drift
...     n *= 2.0         # pop growth
...     print "H/H0 = %f" %h
...
H/H0 = 0.750000
H/H0 = 0.656250
H/H0 = 0.615234
H/H0 = 0.596008
H/H0 = 0.586696
H/H0 = 0.582112
H/H0 = 0.579838
H/H0 = 0.578706
H/H0 = 0.578141
\end{verbatim}
Note that most of the drift happens in the first few generations.
Even a very severe bottleneck removes only a fraction of the variance,
provided that recovery is rapid.
\end{Answr}
\begin{Answr}{3.10}
  $4Nu = 36 \times 10^{-6}$, and the expected heterozygosity is
  $4Nu/(1+4Nu)$, or 0.000036.  The expected number of heterozygous
  flies is 16 times this number, or 0.00058, which rounds to 0. You
  would probably see no heterozygous flies.
\end{Answr}
\begin{Answr}{3.11}
$\theta = \hat H/(1-\hat H)$
\end{Answr}
\begin{Answr}{3.12}
Heterozygosity is maximal when $\theta\rightarrow\infty$. For the
model of infinite sites, this gives $\max\hat H = 1$.
\end{Answr}
\begin{Answr}{3.13}
Heterozygosity is maximal when $\theta\rightarrow\infty$. For the
symmetric biallelic model, $\max\hat H = 1/2$.
\end{Answr}
\begin{Answr}{3.14}
  The calculation is based on Eqn.~\ref{eq.eqhetk}, because we are
  using the symmetric biallelic model of mutation. The question tells
  us that $k=2$ and $\hat H = 0.005$. Eqn~\ref{eq.eqhetk} becomes
  $0.005 = \theta/(1 + 2\theta)$, or $\theta=0.00505$. And since
  $\theta = 4Nu$, where $u = 2\times 10^{-7}$, we have $2N = 12626$.
\end{Answr}
\begin{Answr}{3.15}
  The calculation is based on Eqn.~\ref{eq.eqhet}, because we are
  using the model of infinite alleles.  The question tells us that
  $\hat H = 1/3$, so Eqn~\ref{eq.eqhet} becomes $1/3 = \theta/(1 +
  \theta)$, or $\theta=1/2$. And since $\theta = 4Nu$, where $u =
  10^{-6}$, we have $N = 125000$.
\end{Answr}
\begin{Answr}{4.1}
(a)~5, (b)~1 and 2, (c)~4, 5, and 6.
\end{Answr}
\begin{Answr}{4.2}
For an interval with $i$ lines of descent, the expected duration in
generations is $4N/[i(i-1)]$. For our problem, $2N=5000$. The expected
duration is therefore.
\begin{inparaenum}[(a)]
\item $5000$ for 2 lines of descent,
\item $1000/9 \approx 111$ for 10, and
\item $10000/(1000 \times 999) = 10/999 \approx 0.01$ for 1000.
\end{inparaenum}
\end{Answr}
\begin{Answr}{4.3}
The duration of a coalescent interval is an exponential random
variable. As there are only 2 lines of descent, the hazard is
$h=1/2N$. As explained in JEPr, the mean of this random variable is
$1/h = 2N$, the variance is $1/h^2 = 4N^2$, and the standard
deviation (the square root of the variance) is $2N$. If $2N=5000$, the
mean, variance, and standard deviation are 5000, 25,000,000, and 5000.
\end{Answr}
\begin{Answr}{4.4}
We have 2 coalescent intervals, one with 3 lines of descent and the
other with 2. The expected duration of the one with 3 lines of descent
is $5000/3$ and the expected duration of the other is 5000
generations. The expected depth of the tree is the sum of these, $5000
+ 5000/3 = 20000/3$. The easy way to get this answer is with the
formula $4N(1-1/K)$, where $K=3$ is the number of gene copies in the
modern sample. This gives the same answer, $10000 \times 2/3 =
20000/3$.
\end{Answr}
\begin{Answr}{4.5}
In a tree with sample of size 3, the expected total branch length is
$4N(1 + 1/2) = 6N$. In our problem, $2N=5000$, so the answer is
15,000.
\end{Answr}
\begin{Answr}{4.6}
For an interval with $i$ lines of descent, the expected length in
generations is $4N/[i(i-1)]$, and the total branch length within the
interval is $i$ times this value, or $4N/(i-1)$. The expected number
of mutations is $u$ times the total branch length, or $\theta/(i-1)$,
where $\theta = 4Nu$. For our problem, $\theta=10$. The expected
numbers of mutations are therefore
\begin{inparaenum}[(a)]
\item $10/1 = 10$ for 2 lines of descent,
\item $10/9 = 1.11$ for 10, and
\item $10/999 = 0.01$ for 1000.
\end{inparaenum}
\end{Answr}
\begin{Answr}{4.7}
\begin{minipage}{0.8\columnwidth}
\begin{verbatim}
----------
          |----
----------     |
               |--
------         |
      |--------
------
\end{verbatim}
\end{minipage}
Lengths are $4N/i(i-1)$, so the lengths are $2N/6$, $2N/3$, and $2N$,
for intervals 4, 3, and 2.  The expected depth of the tree is
$4N(1-1/4) = 3N$.  The expected total branch length is $4N(1 + 1/2 +
1/3) = 22N/3 = 7.33N$
\end{Answr}
\begin{Answr}{4.8}
18.3
\end{Answr}
\begin{Answr}{4.9}
25.93
\end{Answr}
\begin{Answr}{4.10}
0.797
\end{Answr}
\begin{Answr}{4.11}
Mean pairwise diff: $\pi = 5.51$ per sequence, and $\pi = 0.1378$ per
site.
\end{Answr}
\begin{Answr}{4.12}
Mean pairwise diff: $\pi=11.69$ per sequence, and $\pi = 0.29 $ per
site.
\end{Answr}
\begin{Answr}{4.13}
$S = 15$ per sequence or 0.375 per site
\end{Answr}
\begin{Answr}{4.14}
$S = 30$ per sequence or 0.75 per site
\end{Answr}
\begin{Answr}{4.15}
$\hat\theta_{\pi} = 5.51$ per sequence or 0.1378 per site.
\end{Answr}
\begin{Answr}{4.16}
$\hat\theta_{\pi} = 11.69$ per sequence or 0.29 per site.
\end{Answr}
\begin{Answr}{4.17}
$\hat\theta_S = 15/2.83 = 5.3$ per sequence or $0.375/2.83 = 0.133$
  per site.
\end{Answr}
\begin{Answr}{4.18}
$\hat\theta_S = 30/2.83 = 10.6$ per sequence or $0.75/2.83 = 0.265$
  per site.
\end{Answr}
\begin{Answr}{4.19}
 In per-site units, we need to compare $\hat\theta_{\pi}
  = 0.1378$ and $\hat\theta_S = 0.133$.  These numbers don't differ too
  much, so there is no obvious reason to reject the model
  of neutral DNA in a randomly mating population of constant size.
\end{Answr}
\begin{Answr}{4.20}
 In per-site units, we need to compare $\hat\theta_{\pi}
  = 0.29$ and $\hat\theta_S = 0.265$.  These numbers don't differ too
  much, so there is no obvious reason to reject the model
  of neutral DNA in a randomly mating population of constant size.
\end{Answr}
\begin{Answr}{4.21}
Table~\ref{tab.Aanswer} presents data set~A again, with an extra row
at the top showing the minor allele counts:
\begin{table*}
\caption{Data set~A with counts of minor allele at top}
\label{tab.Aanswer}
\begin{verbatim}
       12  3 4   14      44 3      2   41   11  1 <-counts
seq01 AATATGGCAC CTCCCAACCC TCTAGCATAT ACCACTTACA
seq02 .......T.. .C......TG C......C.. ..........
seq03 ..C....... .......... .......... ..........
seq04 .......T.. .C......TG C......... G.........
seq05 .......... .......... .......... ..........
seq06 .....A.... ........T. C......... G....C....
seq07 ..C....T.. .C......TG C......... G.........
seq08 .....A.T.. TC......TG C......... G.........
seq09 .......... .......... C......... ..........
seq10 .G...A.... ........T. C......C.. .T....C..G
\end{verbatim}
\end{table*}
Tabulating the counts gives
\begin{center}
\begin{tabular}{cc}
Minor  & Number\\
allele & of\\
count  & sites\\
\hline
1 & 6\\
2 & 2\\
3 & 2\\
4 & 5
\end{tabular}
\end{center}
\end{Answr}
\begin{Answr}{4.22}
Table~\ref{tab.Banswer} presents data set~B again, with an extra row
at the top showing the minor allele counts:
\begin{table*}
\caption{Data set~B with counts of minor allele at top}
\label{tab.Banswer}
\begin{verbatim}
      35 43  22  23112123 2 42 4311142 14 213  1 <-counts
seq01 TGCCACTCCA ATCTCTCGCC AGATGGCATG CCTTATCGCG
seq02 .......... G......... .A.C...GCA T.........
seq03 .......... G......A.. .A.C...GC. T....C....
seq04 C..TG..T.. .C.....A.. G......C.. TT.C......
seq05 CA.TG..T.. .C....TA.. G......CC. TT.C......
seq06 C...G..... CC..T..A.A ....AA.C.. TT..G.....
seq07 CA.TG..... CC..TC.A.A ...CA..CC. TT...C....
seq08 CA.TG...T. GCT....A.. G..C..TC.. T.......T.
seq09 CA.TG..... GC.C..T... ...CA..C.A T.........
seq10 CA.TG...T. .C.....A.. G..C...C.. T....C....
\end{verbatim}
\end{table*}
Tabulating the counts gives
\begin{center}
\begin{tabular}{cc}
Minor  & Number\\
allele & of\\
count  & sites\\
\hline
1 & 9\\
2 & 9\\
3 & 6\\
4 & 5\\
5 & 1
\end{tabular}
\end{center}
\end{Answr}
\begin{Answr}{4.23}
  The expected numbers are $\hat\theta(1 + 1/9) = 5.89$ for singletons,
  $\hat\theta(1/2 + 1/8) = 3.31$ for doubletons,
  $\hat\theta(1/3 + 1/7) = 2.52$ for tripletons,
  $\hat\theta(1/4 + 1/6) = 2.21$ for quadrupletons, and
  $\hat\theta\times 1/5 = 1.1$ for quintupletons.
\end{Answr}
\begin{Answr}{4.24}
  The expected numbers are $\hat\theta(1 + 1/9) = 11.78$ for singletons,
  $\hat\theta(1/2 + 1/8) = 6.63$ for doubletons,
  $\hat\theta(1/3 + 1/7) = 5.05$ for tripletons,
  $\hat\theta(1/4 + 1/6) = 4.41$ for quadrupletons, and
  $\hat\theta\times 1/5 = 2.12$ for quintupletons.
\end{Answr}
\begin{Answr}{4.25}
  The genealogy below has all the lineages coalescing at the same
  time, 6~units of mutational time ago. That is fine, but it would
  also be fine to draw a genealogy in which the coalescent events
  happened more gradually. Many of them should however be
  clustered in a relatively brief interval ending 6 units of time ago.\\
\begin{minipage}[t]{0.8\columnwidth}
\begin{verbatim}
-----------------------|
-----------------------|
-----------------------|
-----------------------|
-----------------------|---------
-----------------------|
-----------------------|
-----------------------|
-----------------------|
-----------------------|

|-------6 time units---|
\end{verbatim}
\end{minipage}
\end{Answr}
\begin{Answr}{4.26}
Poisson with mean 3.
\end{Answr}
\begin{Answr}{4.27}
Accept anything vaguely wave-shaped with a peak near 6 differences.\\
\begin{minipage}[t]{0.8\columnwidth}
\begin{verbatim}
                       *
                  *        *
                  *            *
               *
       *   *                       *
   *
*                                      *

0  1   2   3   4   5   6   7   8   9  10
            Pairwise Differences
\end{verbatim}
\end{minipage}\\[0.5\baselineskip]
\end{Answr}
\begin{Answr}{4.28}
 $\theta = 0.01$ for Europe and Asia; 0.0204 for Africa.
\end{Answr}
\begin{Answr}{4.29}
There are $K(K-1)/2$ pairs of gene copies. Of these pairs, the number
that differ is $1 \times (K-1) = K-1$. The contribution to $\pi$ is
the fraction of pairs that differ, $2/K$.
\end{Answr}
\begin{Answr}{4.30}
  Each segregating site (whether a singleton or not) increments the
  value of $S$ by 1. Consequently, it increments $\hat\theta_S$ by
$\left(\sum_{k=1}^{K-1} 1/k\right)^{-1}$.
\end{Answr}
\begin{Answr}{4.31}
In the table below, $\Delta \hat\theta_S$ represents the contribution
of a singleton site to $\hat\theta_S$ and $\Delta\hat\theta_{\pi}$ is
the corresponding contribution to $\hat\theta_{\pi}$.
\begin{center}
\begin{tabular}{rcc}
$K$ & $\Delta \hat\theta_S$ & $\Delta\hat\theta_{\pi}$\\ \hline
2   & 1.000        & 1.000\\
3   & 0.667        & 0.667\\
4   & 0.545        & 0.500\\
10  & 0.353        & 0.200\\
100 & 0.193        & 0.020\\
\end{tabular}
\end{center}
When $K>3$, each singleton site has a larger effect on $\Delta
\hat\theta_S$ than on $\Delta\hat\theta_{\pi}$.
\end{Answr}
\begin{Answr}{4.32}
  Many segregating sites will be singletons, which add more to
  $\hat\theta_S$ than to $\hat\theta_{\pi}$.
\end{Answr}
\begin{Answr}{5.1}
The mismatch distribution is likely to be unimodal, with a mode near
8.
\end{Answr}
\begin{Answr}{5.2}
The expanded population should have an excess of sites in which the
derived allele is present in only one copy: an excess of singletons.
\end{Answr}
\begin{Answr}{5.3}
The mismatch distribution is
\begin{center}
\begin{tabular}{rr}
Differences & Pairs\\
\hline
0 & 0\\
1 & 4\\
2 & 7\\
3 & 10\\
4 & 9\\
5 & 8\\
6 & 3\\
7 & 2\\
8 & 2\\
\hline
\end{tabular}
\end{center}
\end{Answr}
\begin{Answr}{5.4}
The sum of the mismatch distribution is 45, which is also equal to
  $10\times 9 / 2$.
\end{Answr}
\begin{Answr}{5.5}
The mean pairwise difference is calculated from the mismatch
distribution as
\begin{eqnarray*}
\pi &=& \sum_i i F_i \bigg / \left(\frac{K(K-1)}{2}\right)\\
    &=& 172 / 45 = 3.82
\end{eqnarray*}
where $F_i$ is the number of pairs of sequences that differ by $i$ nucleotide
sites and $K$ is the number of sequences.
\end{Answr}
\begin{Answr}{5.6}
\begin{eqnarray*}
\pi &=&
\biggl(\overbrace{2\times 8}^{\hbox{\tiny site 0}}
+ \overbrace{4\times 6}^{\hbox{\tiny site 1}}
+ \overbrace{2\times 8}^{\hbox{\tiny site 2}}
+ \overbrace{2\times 8}^{\hbox{\tiny site 3}}
+ \overbrace{3\times 7}^{\hbox{\tiny site 4}}\\
&&\mbox{}
+ \overbrace{2\times 8}^{\hbox{\tiny site 5}}
+ \overbrace{1\times 9}^{\hbox{\tiny site 6}}
+ \overbrace{1\times 9}^{\hbox{\tiny site 7}}
+ \overbrace{4\times 6}^{\hbox{\tiny site 8}}\\
&&\mbox{}
+ \overbrace{3\times 7}^{\hbox{\tiny site 9}}\biggr) / 45
= 172/45 = 3.82
\end{eqnarray*}
\end{Answr}
\begin{Answr}{5.7}
\begin{center}
\mbox{\beginpicture
\setcoordinatesystem units <.2in, 5in>
\setplotarea x from 0 to 11, y from 0 to 0.17
\axis bottom label {Pairwise differences $(i)$}
  ticks numbered from 0 to 10 by  5 /
\axis left label {$F_i$}
  ticks numbered from 0 to 0.15 by  0.05 /
\plotheading{Data set A}
%dataset A
% normalized mismatch distribution inPicTeX format:
\multiput {$\circ$} at  0 0.005263 1 0.021053 2 0.057895 3 0.142105
   4 0.173684 5 0.173684 6 0.147368 7 0.136842 8 0.063158
   9 0.031579 10 0.031579 11 0.015789
/
\plot  0 0.005263 1 0.021053 2 0.057895 3 0.142105
   4 0.173684 5 0.173684 6 0.147368 7 0.136842 8 0.063158
   9 0.031579 10 0.031579 11 0.015789
/
\endpicture}

\end{center}
The vertical axis $(F_i)$ is the fraction of pairs of DNA sequences
that differ by the number of sites shown on the horizontal axis.
\end{Answr}
\begin{Answr}{5.8}
\begin{center}
\mbox{\beginpicture
\setcoordinatesystem units <.17in, 5in>
\setplotarea x from 0 to 14, y from 0 to 0.17
\axis bottom label {Pairwise differences $(i)$}
  ticks numbered from 0 to 10 by  5 /
\axis left label {$F_i$}
  ticks numbered from 0 to 0.15 by  0.05 /
\plotheading{Data set B}
%dataset B
% normalized mismatch distribution inPicTeX format:
\multiput {$\circ$} at  0 0.131579 1 0.063158 2 0.015789 3 0.131579
   4 0.142105 5 0.157895 6 0.078947 7 0.063158 8 0.047368
   9 0.015789 10 0.047368 11 0.094737 12 0.000000 13 0.000000
   14 0.010526
/
\plot 0 0.131579 1 0.063158 2 0.015789 3 0.131579
   4 0.142105 5 0.157895 6 0.078947 7 0.063158 8 0.047368
   9 0.015789 10 0.047368 11 0.094737 12 0.000000 13 0.000000
   14 0.010526
/
\endpicture}

\end{center}
The vertical axis $(F_i)$ is the fraction of pairs of DNA sequences
that differ by the number of sites shown on the horizontal axis.
\end{Answr}
\begin{Answr}{5.9}
  For the 31 polymorphic sites, the frequencies of the ``minor
  allele'' (the rarer of the two at each site) are 2, 1, 1, 1, 1, 1,
  5, 1, 6, 1, 1, 6, 1, 1, 1, 3, 2, 1, 1, 1, 1, 1, 1, 1, 2, 1, 5, 3, 2,
  2, and 2. The folded observed spectrum is
  \begin{center}
    \begin{tabular}{rr}
      Minor  & \\
      allele & Observed\\
      count  & spectrum\\ \hline
      1 & 19\\
      2 & 6\\
      3 & 2\\
      4 & 0\\
      5 & 2\\
      6 & 2\\
      7 & 0\\
      8 & 0\\
      9 & 0\\
     10 & 0
    \end{tabular}
\end{center}
\end{Answr}
\begin{Answr}{5.10}
  For the 20 polymorphic sites, the frequencies of the ``minor
  allele'' (the rarer of the two at each site) are 2, 7, 2, 6, 2, 2,
  7, 7, 2, 5, 1, 1, 2, 1, 1, 7, 3, 1, 2, and 1.  The folded observed
  spectrum is
  \begin{center}
    \begin{tabular}{rr}
      Minor  & \\
      allele & Observed\\
      count  & spectrum\\ \hline
      1 & 6\\
      2 & 7\\
      3 & 1\\
      4 & 0\\
      5 & 1\\
      6 & 1\\
      7 & 4\\
      8 & 0\\
      9 & 0\\
      10 & 0
    \end{tabular}
  \end{center}
\end{Answr}
\begin{Answr}{5.11}
  To estimate $\theta$, note that $S=31$, so $\hat\theta =
  31/\sum_{i=1}^{K-1} 1/i = 8.73796$. The expected folded spectrum is
\begin{center}
\begin{tabular}{rr}
\multicolumn{2}{c}{Data set~A}\\
 Minor &\\
allele &Expected\\
 count &spectrum\\
 \hline
 1 & 9.19785\\
 2 & 4.85442\\
 3 & 3.42665\\
 4 & 2.73061\\
 5 & 2.33012\\
 6 & 2.08047\\
 7 & 1.92043\\
 8 & 1.82041\\
 9 & 1.76524\\
10 & 0.87380\\
\hline
\end{tabular}
\end{center}
\end{Answr}
\begin{Answr}{5.12}
  To estimate $\theta$, note that $S=20$, so $\hat\theta =
  20/\sum_{i=1}^{K-1} 1/i = 5.63739$. The expected folded spectrum is
  \begin{center}
\begin{tabular}{rr}
\multicolumn{2}{c}{Data set~B}\\
 Minor &\\
allele &Expected\\
 count &spectrum\\
 \hline
 1 & 5.93410\\
 2 & 3.13188\\
 3 & 2.21074\\
 4 & 1.76169\\
 5 & 1.50330\\
 6 & 1.34224\\
 7 & 1.23899\\
 8 & 1.17446\\
 9 & 1.13887\\
10 & 0.56374\\
\hline
\end{tabular}
\end{center}
\end{Answr}
\begin{Answr}{5.13}
\begin{center}
\mbox{\beginpicture
\setcoordinatesystem units <0.2in,0.05in>
\setplotarea x from 0 to 10, y from 0 to 19
\axis left label {\lines{No.\cr of\cr sites}}  /
\axis bottom label {Minor allele count}
  ticks withvalues 1 5 10 / at 1 5 10 / /
\plotheading{Spectrum A}
% Observed folded spectrum
\multiput {$\circ$} at
 1 19
 2  6
 3  2
 4  0
 5  2
 6  2
 7  0
 8  0
 9  0
10  0
/
% Theoretical folded spectrum
\plot
 1 9.19785
 2 4.85442
 3 3.42665
 4 2.73061
 5 2.33012
 6 2.08047
 7 1.92043
 8 1.82041
 9 1.76524
10 0.87380
/
\endpicture}

\end{center}
The graph shows the observed spectrum as open circles and the expected
one as a solid line.
\end{Answr}
\begin{Answr}{5.14}
\begin{center}
\mbox{\beginpicture
\setcoordinatesystem units <0.2in,0.16in>
\setplotarea x from 0 to 10, y from 0 to 6
\axis left label {\lines{No.\cr of\cr sites}}  /
\axis bottom label {Minor allele count}
  ticks withvalues 1 5 10 / at 1 5 10 / /
\plotheading{Spectrum B}
% Observed folded spectrum
\multiput {$\circ$} at
 1 6
 2 7
 3 1
 4 0
 5 1
 6 1
 7 4
 8 0
 9 0
10 0
/
% Theoretical folded spectrum
\plot
 1 5.93410
 2 3.13188
 3 2.21074
 4 1.76169
 5 1.50330
 6 1.34224
 7 1.23899
 8 1.17446
 9 1.13887
10 0.56374
/
\endpicture}

\end{center}
The graph shows the observed spectrum as open circles and the expected
one as a solid line.
\end{Answr}
\begin{Answr}{5.15}
Data set~A came from an expanded population, and data set~B from a
stationary one.
\end{Answr}
\begin{Answr}{6.1}
Because mutations may occur on either of the two paths separating the
two species from their common ancestor.
\end{Answr}
\begin{Answr}{6.2}
The values below indicate that relative error is small when $p$ is
less than about 0.1.\\
{\centering
\begin{tabular}{cc}
 $p$ & relerr\\ \hline
0.30 & 0.22\\
0.20 & 0.14\\
0.10 & 0.07\\
0.08 & 0.05\\
0.06 & 0.04\\
0.05 & 0.03\\
\hline
\end{tabular}\\}
\end{Answr}
\begin{Answr}{6.3}
For the chimpanzee-human data,
\[
u = K/2t = 0.01177/(12\times 10^{6}) = 0.98\times 10^{-9}
\]
That's a bit lower than the rates estimated for typical mammals.  But
we hominins have had longer generation times. Suppose our ancestors
had 20-year generations. Then the rate per generation would be
\[
u = 20 \times 0.98\times 10^{-9} \approx 2 \times 10^{-8}
\]
which is in good aggrement with other mammals.
\end{Answr}
\begin{Answr}{6.4}
As with the Jukes-Cantor model, we begin with
\begin{eqnarray*}
q &=& e^{-2\lambda t} + (1 - e^{-2\lambda t})/2\\
&=& \frac{1}{2} + \frac{1}{2}\,e^{-2\lambda t}
\end{eqnarray*}
When at least one perturbation has occurred, the sites are identical
with probability 1/2. This accounts for the value ``1/2'' that appears
above. Now substitute $2u=\lambda$ (because only half of perturbations
are mutations), $p=1-q$, and $K = 2ut$:
\[
p = \frac{1}{2} - \frac{1}{2}\,e^{-2K}
\]
Solving for $K$ gives
\[
K = -\frac{1}{2}\,\log_e(1-2p)
\]
which is analogous to Eqn~\ref{eq.jukescantor}. This was introduced in
1919 and is called ``Haldane's mapping function''
\cite{Haldane:Distance}. It has often been used to make linkage maps
of chromosomes.
\end{Answr}
\begin{Answr}{6.5}
For the chimpanzee-human comparison, $p = 0.01167$. Plugging this into
the 2-state formula gives $K=0.01180$.
\end{Answr}
\begin{Answr}{6.6}
(a)~For the three species, Jukes-Cantor yields the following estimates
  of $K$:\\
\begin{tabular}{lrrrrr}
                 &\multicolumn{3}{c}{$K$ estimates}\\
                 & BN Rat& Mouse&Human\\
\hline
Brown Norway Rat &  ---&0.38  & 0.20\\
Mouse            &     & ---  & 0.40\\
Human            &     &      &  ---\\
\hline
\end{tabular}

(b)~The mouse-human distance is \emph{much} larger than the rat-human
distance. (We hope students will contemplate the causes of this
difference, but grades will not be based on this issue.)
\end{Answr}
\begin{Answr}{7.1}
\begin{inparaenum}[(a)]
\item
The mean relative fitness is $\bar w = (p^2\times 1) + (2pq\times 1.02) +
(q^2\times 1.03) = 1.0279$.
\item
The marginal fitness of $A_1$ is $w_1 = p w_{11} + q w_{12} =
1.018$. That of $A_2$ is $w_2 = p w_{12} + q w_{22} = 1.029$.
\item
  In the following generation, the expected frequency of $A_1$ is $p'
  = (p^{2} w_{11} + pqw_{12})/\bar w$, as shown on Gillespie's
  p.~62. This gives 0.099.
\end{inparaenum}
\end{Answr}
\begin{Answr}{7.2}
\begin{inparaenum}[(a)]
\item
The mean relative fitness is $\bar w = 1 - 2pqhs - q^2 s = 0.966$.
\item
The marginal fitness of $A_1$ is $w_1 = p + q(1-hs) = 0.996$. That of
$A_2$ is $w_2 = p(1-hs) + q(1-s) = 0.959$.
\item
In the following generation, the expected frequency $A_1$ is $p' =
(p^{2} w_{11} + pqw_{12})/\bar w$, as shown on Gillespie's p.~62. This
gives 0.2061.
\end{inparaenum}
\end{Answr}
\begin{Answr}{7.3}
\begin{inparaenum}[(a)]
\item
The mean relative fitness is $\bar w = 1 - 2pqhs - q^2 s = 0.999$.
\item
The marginal fitness of $A_1$ is $w_1 = p + q(1-hs) = 1.0015$. That of
$A_2$ is $w_2 = p(1-hs) + q(1-s) = 0.9965$.
\item
  In the following generation, the expected frequency $A_1$ is $p' =
  (p^{2} w_{11} + pqw_{12})/\bar w$, as shown on Gillespie's
  p.~62. This gives 0.501.
\end{inparaenum}
\end{Answr}
\begin{Answr}{7.4}
\begin{inparaenum}[(a)]
\item Gillespie makes his fitnesses relative to that of genotype
$A_1A_1$. To obtain these relative fitnesses, divide each absolute
fitness by $W_{11}$. This gives relative fitnesses $w_{11}=1$, $w_{12} =
0.769$, and $w_{22} = 0.692$. In Gillespie's fitness scheme, $w_{12} =
1-hs$, and $w_{22} = 1-s$. This implies that $s = 1 - w_{22} = 0.308$,
and $h = (1-w_{12})/s = 0.75$.

\item Gillespie's Eqns.~3.2--3.3 express $\Delta_s p$ in terms of $p$,
  $q=1-p$, $s$, $h$, and $\bar w = 1 - pqhs - q^2s$. Note that $q$ and
  $\bar w$ depend on $p$ and must therefore be recalculated for each
  value of $p$ that you graph. I did these calculations using a Python
  script. Here are a few of the results in tabular form:

{\centering\begin{tabular}{rr} $p$ & $\Delta_s p$\\ \hline
      0.0000 &   0.0000\\
      0.0417 &   0.0047\\
      0.0833 &   0.0095\\
      $\cdots$ & $\cdots$\\
      0.9167 &   0.0170\\
      0.9583 &   0.0090\\
      1.0000 &   0.0000\\
\end{tabular}\\}

The graph of these results looks like this:

{\centering\mbox{\beginpicture
\setcoordinatesystem units <4cm, 70cm>
\setplotarea x from 0 to 1, y from 0.0000 to 0.0455
\axis left label {$\Delta_s p$}
  ticks numbered from 0.00 to 0.04 by 0.01 /
\axis bottom label {$p$}
  ticks numbered from 0.00 to 1.00 by 0.25 /
%       p       dp
\plot
   0.0000   0.0000
   0.0417   0.0047
   0.0833   0.0095
   0.1250   0.0143
   0.1667   0.0189
   0.2083   0.0234
   0.2500   0.0277
   0.2917   0.0316
   0.3333   0.0352
   0.3750   0.0383
   0.4167   0.0409
   0.4583   0.0430
   0.5000   0.0445
   0.5417   0.0454
   0.5833   0.0455
   0.6250   0.0450
   0.6667   0.0436
   0.7083   0.0415
   0.7500   0.0385
   0.7917   0.0346
   0.8333   0.0297
   0.8750   0.0239
   0.9167   0.0170
   0.9583   0.0090
   1.0000   0.0000
/
\endpicture}\\}

\item
Equilibria occur where $\Delta_s p = 0$. As the graph shows, the only
such points are at $p=0$ and $p=1$. Only the second of these is
stable. This is because $\Delta_s p>0$ whenever $0<p<1$. Consequently,
$p$ always moves \emph{toward} the equilibrium at $p=1$ but \emph{away
  from} the one at $p=0$.
\end{inparaenum}
\end{Answr}
\begin{Answr}{7.5}
\begin{inparaenum}[(a)]
\item Gillespie makes his fitnesses relative to that of genotype
$A_1A_1$. To obtain these relative fitnesses, divide each absolute
fitness by $W_{11}$. This gives relative fitnesses $w_{11}=1$, $w_{12} =
0.769$, and $w_{22} = 0.923$. In Gillespie's fitness scheme, $w_{12} =
1-hs$, and $w_{22} = 1-s$. This implies that $s = 1 - w_{22} = 0.077$,
and $h = (1-w_{12})/s = 3$.

\item Gillespie's Eqns.~3.2--3.3 express $\Delta_s p$ in terms of $p$,
  $q=1-p$, $s$, $h$, and $\bar w = 1 - pqhs - q^2s$. Note that $q$ and
  $\bar w$ depend on $p$ and must therefore be recalculated for each
  value of $p$ that you graph. I did these calculations using a Python
  script. Here are a few of the results in tabular form:

{\centering\begin{tabular}{rr} $p$ & $\Delta_s p$\\ \hline
   0.0000  & --0.0000\\
   0.0417  & --0.0060\\
   0.0833  & --0.0101\\
  $\cdots$ & $\cdots$\\
   0.9167  & 0.0155\\
   0.9583  & 0.0087\\
   1.0000  & 0.0000\\
\end{tabular}\\}

The graph of these results looks like this:

{\centering\mbox{\beginpicture
\setcoordinatesystem units <4cm, 79.4cm>
\setplotarea x from 0 to 1, y from -0.0136 to 0.0265
\axis left label {$\Delta_s p$}
  ticks numbered from -0.01 to 0.02 by 0.01 /
\axis bottom label {$p$}
  ticks numbered from 0.00 to 1.00 by 0.25 /
\setdots
\putrule from 0 0 to 1 0
\setsolid
%       p       dp
\plot
   0.0000  -0.0000
   0.0417  -0.0060
   0.0833  -0.0101
   0.1250  -0.0126
   0.1667  -0.0136
   0.2083  -0.0133
   0.2500  -0.0118
   0.2917  -0.0094
   0.3333  -0.0062
   0.3750  -0.0025
   0.4167   0.0017
   0.4583   0.0061
   0.5000   0.0104
   0.5417   0.0146
   0.5833   0.0184
   0.6250   0.0217
   0.6667   0.0242
   0.7083   0.0259
   0.7500   0.0265
   0.7917   0.0259
   0.8333   0.0240
   0.8750   0.0205
   0.9167   0.0155
   0.9583   0.0087
   1.0000   0.0000
/
\endpicture}\\}

\item Equilibria occur where $\Delta_s p = 0$. There are three such
  points: at $p=0$, at $p=1$, and at $p=0.4$.  $\Delta_s p$ is
  positive to the right of the intermediate equilibrium but negative
  to the left. Consequently, the intermediate equilibrium is unstable
  and the two extreme equilibria are stable.
\end{inparaenum}
\end{Answr}
\begin{Answr}{7.6}
\begin{inparaenum}[(a)]
\item Gillespie makes his fitnesses relative to that of genotype
$A_1A_1$. To obtain these relative fitnesses, divide each absolute
fitness by $W_{11}$. This gives relative fitnesses $w_{11}=1$, $w_{12} =
1.111$, and $w_{22} = 0.889$. In Gillespie's fitness scheme, $w_{12} =
1-hs$, and $w_{22} = 1-s$. This implies that $s = 1 - w_{22} = 0.111$,
and $h = (1-w_{12})/s = -1$.

\item Gillespie's Eqns.~3.2--3.3 express $\Delta_s p$ in terms of $p$,
  $q=1-p$, $s$, $h$, and $\bar w = 1 - pqhs - q^2s$. Note that $q$ and
  $\bar w$ depend on $p$ and must therefore be recalculated for each
  value of $p$ that you graph. I did these calculations using a Python
  script. Here are a few of the results in tabular form:

{\centering\begin{tabular}{rr} $p$ & $\Delta_s p$\\ \hline
   0.0000 & 0.0000\\
   0.0417 & 0.0092\\
   0.0833 & 0.0162\\
  $\cdots$ & $\cdots$\\
   0.9167 &--0.0063\\
   0.9583 &--0.0039\\
   1.0000 &--0.0000\\
\end{tabular}\\}

The graph of these results looks like this:

{\centering\mbox{\beginpicture
\setcoordinatesystem units <4cm, 85.6cm>
\setplotarea x from 0 to 1, y from -0.01 to 0.0272
\axis left label {$\Delta_s p$}
  ticks numbered from -0.01 to 0.02 by 0.01 /
\axis bottom label {$p$}
  ticks numbered from 0.00 to 1.00 by 0.25 /
\setdots
\putrule from 0 0 to 1 0
\setsolid
%       p       dp
\plot
   0.0000   0.0000
   0.0417   0.0092
   0.0833   0.0162
   0.1250   0.0213
   0.1667   0.0247
   0.2083   0.0266
   0.2500   0.0272
   0.2917   0.0267
   0.3333   0.0253
   0.3750   0.0232
   0.4167   0.0205
   0.4583   0.0173
   0.5000   0.0139
   0.5417   0.0103
   0.5833   0.0067
   0.6250   0.0032
   0.6667  -0.0000
   0.7083  -0.0028
   0.7500  -0.0051
   0.7917  -0.0068
   0.8333  -0.0076
   0.8750  -0.0075
   0.9167  -0.0063
   0.9583  -0.0039
   1.0000  -0.0000
/
\endpicture}\\}

\item
Equilibria occur where $\Delta_s p = 0$. There are three such points:
at $p=0$, at $p=1$, and at $p=2/3$.
$\Delta_s p$ is positive to the left of the intermediate equilibrium
but negative to the right. Consequently, the intermediate equilibrium
is stable and the two extreme equilibria are unstable.
\end{inparaenum}
\end{Answr}
\begin{Answr}{7.7}
  No calculation is required. Because $A_2A_2$ has higher fitness than
  $A_1A_1$ and the heterozygote is intermediate, $A_2$ will go to
  fixation and the stable equilibrium frequency of $A_1$ is 0.
\end{Answr}
\begin{Answr}{7.8}
  No calculation is required. Because the two homozygotes have equal
  fitness and the heterozygote is superior, the stable equilibrium
  frequency of $A_1$ is $1/2$.
\end{Answr}
\begin{Answr}{7.9}
  No calculation is required. Because the heterozygote has lower
  fitness than either homozygote, there are two stable
  equilibria---one at which the frequency of $A_1$ is~0 and another at
  which it is~1.
\end{Answr}
\begin{Answr}{7.10}
$\hat p$, the equilibrium value of $p$, is equal to $(h-1)/(2h-1)$
(see Gillespie's Eqn.~3.4). Here are a few values and then the graph:

{\centering\begin{tabular}{rr} $h$ & $\hat p$\\ \hline
 --1.0000 &  0.6667\\
 --0.8421 &  0.6863\\
 --0.6842 &  0.7111\\
  $\cdots$ & $\cdots$\\
  1.6842 &  0.2889\\
  1.8421 &  0.3137\\
  2.0000 &  0.3333\\
\end{tabular}\\}

{\centering\mbox{\beginpicture
\setcoordinatesystem units <1.33cm, 4cm>
\setplotarea x from -1 to 2, y from 0 to 1
\axis left label {$\hat p$}
  ticks numbered from 0.00 to 1.00 by 0.25 /
\axis bottom label {$h$}
  ticks numbered from -1.0 to 2.0 by 1 /
\setdots
\putrule from 0 0 to 1 0
\setsolid
%       h       phat
\multiput {$\circ$} at
 -1.0000   0.6667
 -0.8421   0.6863
 -0.6842   0.7111
 -0.5263   0.7436
 -0.3684   0.7879
 -0.2105   0.8519
 -0.0526   0.9524
%  0.1053   1.1333
%  0.2632   1.5556
%  0.4211   3.6667
%  0.5789  -2.6667
%  0.7368  -0.5556
%  0.8947  -0.1333
  1.0526   0.0476
  1.2105   0.1481
  1.3684   0.2121
  1.5263   0.2564
  1.6842   0.2889
  1.8421   0.3137
  2.0000   0.3333
/
\endpicture}\\}

(a)~Overdominance occurs when $h<0$ and generates a stable equilibrium
within the interval $(0,1)$. (b)~Incomplete dominance occurs when
$0<h<1$. In this case, selection is directional, and there are no
internal equilibria. (c)~Underdominance occurs when $h>1$ and
generates an unstable internal equilibrium. The only stable
equilibria are at $p=0$ and $p=1$.
\end{Answr}
\begin{Answr}{7.11}
The change per generation is $\Delta p \approx pq^2s - up$. At
equilibrium between mutation and drift, this quantity equals
zero. This implies that the equilibrium occurs at $\hat q =
\sqrt{u/s}$.
\end{Answr}
\begin{Answr}{7.12}
\begin{inparaenum}[\it (a)]
\item In the first generation, the frequency of white flowers is
  $P_{rr} = 200/1000 = 0.2$. At Hardy-Weinberg equilibrium, this
  frequency is $q^2$, where $q$ is the frequency of allele $r$. This
  implies that $q = \sqrt{0.2} = 0.447$. The frequency of allele $R$
  is therefore $p = 1-q = 0.553$.
\item In the 2nd generation, $P_{rr} = 250/1000 = 0.25$, so $q' =
  \sqrt{0.25} = 0.5$, and $p'=1-q'=0.5$.
\item Because $w_{RR} = w_{Rr} = 1$, Eqn.~\ref{eq.pprime} simplifies
  to $p' = p/(1-q^2 s)$, or $s = (1-p/p')/q^2$. With our data, this is
  $s = (1-0.553/0.5)/0.2 = -0.53$.
\end{inparaenum}
\end{Answr}
\begin{Answr}{7.13}
\begin{inparaenum}[\it (a)]
\item In the first generation, the frequency of white flowers is
  $P_{rr} = 100/1000 = 0.1$. At Hardy-Weinberg equilibrium, this
  frequency is $q^2$, where $q$ is the frequency of allele $r$. This
  implies that $q = \sqrt{0.1} = 0.316$. The frequency of allele $R$
  is therefore $p = 1-q = 0.684$.
\item In the 2nd generation, $P_{rr} = 95/1000 = 0.095$, so $q' =
  \sqrt{0.095} = 0.308$, and $p'=1-q'=0.692$.
\item Because $w_{RR} = w_{Rr} = 1$, Eqn.~\ref{eq.pprime} simplifies
  to $p' = p/(1-q^2 s)$, or $s = (1-p/p')/q^2$. With our data, this is
  $s = (1-0.684/0.692)/0.1 = 0.116$.
\end{inparaenum}
\end{Answr}
\begin{Answr}{7.14}
Because the number of new advantageous mutants is proportional to
population size, but the fixation probability for such mutants is
essentially independent of population size. The rate of adaptive
evolution is thus proportional to population size.

Some students may want to answer this question more formally, so here
is the formal version: let $u$ represent the mutation rate for
advantageous alleles whose fitness in heterozygotes is $1+s/2$,
relative to that of the wild-type allele. The number of such mutations
in the population as a whole is $2Nu$, and each of them has
probability $s$ of ultimate fixation. For this category of adaptive
mutations, the expected number fixed per generation is therefore
$2Nus$, an increasing function of population size.
\end{Answr}
\begin{Answr}{7.15}
With $s=-0.001$, the fixation probability is
\begin{inparaenum}[(a)]
\item $2.1\times 10^{-12}$ if $N=10000$;  or
\item $1.57\times 10^{-4}$ if $N=1000$; or
\item $4.5\times 10^{-3}$ if $N=100$.
\end{inparaenum}
\end{Answr}
\begin{Answr}{7.16}
The substitution rate per million years is $10^6 \times 2Nu $ times the
answers from the preceding question.
\begin{inparaenum}[(a)]
\item $9.07\times 10^{-11}$ if $N=10000$;  or
\item $6.89\times 10^{-4}$ if $N=1000$; or
\item $1.99\times 10^{-3}$ if $N=100$.
\end{inparaenum}
\end{Answr}
\begin{Answr}{7.17}
With $s=0.001$, the fixation probability is
\begin{inparaenum}[(a)]
\item $1\times 10^{-3}$ if $N=10000$;  or
\item $1.16\times 10^{-3}$ if $N=1000$; or
\item $5.51\times 10^{-3}$ if $N=100$.
\end{inparaenum}
\end{Answr}
\begin{Answr}{7.18}
  The substitution rate per million years is $10^6 \times 2Nu $ times
  the results from the \verb|pfix| function (Gillespie's Eqn.~3.22)
  defined above.
\begin{inparaenum}[(a)]
\item 0.044 if $N=10000$;  or
\item 0.0051 if $N=1000$; or
\item 0.0024 if $N=100$.
\end{inparaenum}
\end{Answr}
\begin{Answr}{7.19}
\begin{inparaenum}[(a)]
\item For $N=10000$: $\pi_1=0.0009995$, $s=0.001$,
  $\mbox{relerr}=0.000500$;
\item for $N= 1000$: $\pi_1=0.0011559$, $s=0.001$,
  $\mbox{relerr}=0.134903$;
\item for $N=  100$: $\pi_1=0.0055139$, $s=0.001$,
  $\mbox{relerr}=0.818640$.
\end{inparaenum}
The approximation works well for $N=10000$, fairly well for $N=1000$,
but poorly for $N=100$.
\end{Answr}
\begin{Answr}{8.1}
\begin{inparaenum}[(a)]
\item The gamete frequencies are $x_1=30/170 = 0.176$, $x_2= 70/170 = 0.412$,
  $x_3=50/170 = 0.294$, and $x_4= 20/170=0.118$.
\item The allele frequencies at the two loci are $p_A=x_1+x_2 = 0.588$, and
  $p_B=x_1 + x_3 = 0.471$.
\item $D = x_1x_4 - x_2x_3 = -0.1003$.
\item $r^2 = D^2/[p_A(1-p_A)p_G(1-p_G)] = 0.167$.
\end{inparaenum}
\end{Answr}
\begin{Answr}{8.2}
\begin{inparaenum}[(a)]
\item The gamete frequencies are $x_1=80/165=0.485$, $x_2= 30/165 = 0.182$,
  $x_3=10/165 = 0.061$, and $x_4= 45/165 = 0.273$.
\item The allele frequencies at the two loci are $p_A=x_1+x_2 = 0.667$, and
  $p_B=x_1 + x_3 = 0.545$.
\item $D=x_1x_4 - x_2x_3 = 0.121$.
\item $r^2 = D^2/[p_A(1-p_A)p_G(1-p_G)] = 0.267$.
\end{inparaenum}
\end{Answr}
\begin{Answr}{8.3}
\begin{inparaenum}[(a)]
\item $x_1=0.2$, $x_2=0.1$, $x_3=0.5$, and $x_4=0.2$.
\item $p_A=x_1 + x_2 = 0.3$, and $p_G=x_1 + x_3 = 0.7$.
\item $D = x_1x_4 - x_2x_3 = -0.01$.
\item $r^2=D^2/[p_A(1-p_A)p_G(1-p_G)] = 0.0023$.
\end{inparaenum}
\end{Answr}
\begin{Answr}{8.4}
\begin{inparaenum}[(a)]
\item $x_1=0.1$, $x_2=0.3$, $x_3=0.5$, and $x_4=0.1$.
\item $p_A=x_1 + x_2 = 0.4$, and $p_G=x_1 + x_3 = 0.6$.
\item $D = x_1x_4 - x_2x_3 = -0.14$.
\item $r^2=D^2/[p_A(1-p_A)p_G(1-p_G)] = 0.3403$.
\end{inparaenum}
\end{Answr}
\begin{Answr}{8.5}
(a)~After 1 generation, $D = 0.2 \times 0.5 = 0.1$.
(b)~After 3 generations, $D = 0.2 \times 0.5^3 =0.025$.
\end{Answr}
\begin{Answr}{8.6}
(a)~After 1 generation, $D = 0.2997$.
(b)~After 100 generations, $D = 0.2714$.
\end{Answr}
\begin{Answr}{8.7}
\begin{eqnarray*}
D &=& x_1 - (x_1+x_2)(x_1+x_3)\\
  &=& x_1 - x_1^2 - x_1x_3-x_1x_2-x_2x_3
\end{eqnarray*}
At this point, all the numbers are in pairs except the lonely $x_1$ at
the beginning.  Something has to be done about this lonely $x_1$,
because we're trying to derive a formula in which \emph{all} the
numbers are in pairs. To fix things, multiply the lonely $x_1$ by
$x_1+x_2+x_3+x_4$.  This does not change anything because these
numbers sum to 1.  We get
\begin{eqnarray*}
D &=& x_1^2+x_1x_2+x_1x_3+x_1x_4 - x_1^2 - x_1x_3\\
  &&\mbox{} -x_1x_2-x_2x_3\\
  &=& x_1x_4 -x_2x_3
\end{eqnarray*}
\end{Answr}
\begin{Answr}{8.8}
One approach is to add the two expressions on the right side. If both
equations are correct, then this sum should equal $D-D=0$. Summing the
two equations gives
\begin{eqnarray*}
D - D &=& x_1 - p_A p_B + x_2 - p_A (1-p_B)\\
  &=& x_1 + x_2 - p_A(p_B + 1 - p_B)\\
  &=& x_1 + x_2 - p_A\\
  &=& p_A - p_A\\
  &=& 0
\end{eqnarray*}
The two expressions sum to zero, so the two equations are equivalent.
\end{Answr}
\begin{Answr}{8.9}
  (a)~$(x_1,x_2,x_3,x_4) = (0,1/2,1/2,0)$;
  (b)~$D = x_1 x_4 - x_2 x_3 = -1/4$.
\end{Answr}
\begin{Answr}{8.10}
The gametic fitnesses are $(w_1, w_2, w_3, w_4) = (1+s, 1+s, 1, 1)$,
and mean fitness is $\bar w = \sum_i w_i x_i = 1+s/2$.
\end{Answr}
\begin{Answr}{8.11}
\begin{eqnarray*}
  x'_1 &=& (1+s) (0 +1/16)/(1+s/2)\\
       &=& \left(\frac{1}{16}\right)
           \left(\frac{1+s}{1+s/2}\right)\\
x'_2 &=& (1+s) (1/2 -1/16)/(1+s/2)\\
     &=& \left(\frac{7}{16}\right)
         \left(\frac{1+s}{1+s/2}\right)\\
x'_3 &=& (1/2 -1/16)/(1+s/2)\\
     &=& \left(\frac{7}{16}\right)
         \left(\frac{1}{1+s/2}\right)\\
x'_4 &=& (0 +1/16)/(1+s/2)\\
     &=& \left(\frac{1}{16}\right)
         \left(\frac{1}{1+s/2}\right)
\end{eqnarray*}
To simplify this a little further, some students might assume that
$s\ll 1$, so that $(1+s)/(1+s/2) \approx 1+s/2$ and $1/(1+s/2) \approx
1-s/2$, ignoring terms of order $s^2$. That is OK too. With that
approximation, the answers would be
\begin{eqnarray*}
x'_1 &\approx& (1/16)(1+s/2)\\
x'_2   &\approx& (7/16)(1+s/2)\\
x'_3   &\approx& (7/16)(1-s/2)\\
x'_4   &\approx& (1/16)(1-s/2)
\end{eqnarray*}
\end{Answr}
\begin{Answr}{8.12}
(a)~Diversity should be lower at nearby loci. (b)~This effect is
larger in regions of low recombination.
\end{Answr}
\begin{Answr}{8.13}
$\lambda = 1.0101\times 10^{-8}$, which is within about 1\% of the
  approximate value. The approximate formula is quite accurate.
\end{Answr}
\begin{Answr}{8.14}
\begin{inparaenum}[(a)]
\item The approximate formula gives $c = 10^{-8} \times 5\times 10^7
  =0.5$.
\item Haldane's mapping function gives $c = (1 - \exp(2 \times 10^{-8}
  \times 5\times 10^7))/2 = 0.316$.
\item The approximation is poor when the distance between loci is
  large.
\end{inparaenum}
\end{Answr}
\begin{Answr}{8.15}
\begin{center}
\mbox{\beginpicture
\setcoordinatesystem units <1.2\columnwidth, 1.2\columnwidth>
\setplotarea x from 0.0 to 0.5, y from 0.0 to 0.5
\axis left label {$c$}
  ticks numbered from 0.0 to 0.5 by  0.1 /
\axis bottom label {$\lambda k$}
  ticks numbered from 0.0 to 0.5 by 0.1 /
\setdashes
% Approximation: c = l*k
\plot
0 0
0.5 0.5
/
\setsolid
% Haldane's mapping function
\plot
0.000 0.000
0.026 0.026
0.053 0.050
0.079 0.073
0.105 0.095
0.132 0.116
0.158 0.135
0.184 0.154
0.211 0.172
0.237 0.189
0.263 0.205
0.289 0.220
0.316 0.234
0.342 0.248
0.368 0.261
0.395 0.273
0.421 0.285
0.447 0.296
0.474 0.306
0.500 0.316
/
\endpicture}\\
\end{center}
This figure shows both
\begin{inparaenum}[(a)]
\item the approximation (dashes), and
\item Haldane's mapping formula (solid line).
\item The approximation looks good for $\lambda k <
0.1$.  On a genome-wide average, $\lambda\approx10^{-8}$ for
humans. Consequently,
\item the approximation should work well when $k < 10^7$. In other
  words, it should work for sites separated by no more than 10
  megabases.
\end{inparaenum}
\end{Answr}
\begin{Answr}{8.16}
  We are looking for sites such that $c/s < 0.1$, or $c < 0.1s$, where
  $c$ is the recombination rate between the site in question and the
  selected site. For this problem, $s=0.001$, so $c < 10^{-4}$. We
  established above that when $c$ less than 0.1 or so, it is
  approximately equal to $\lambda k$, where $\lambda = 10^{-8}$, and
  $k$ is the distance between the sites in base pairs. Our largest $c$
  value is much smaller than 0.1, so we can use this approximate
  formula. Using this formula, $c = 10^{-8} k < 10^{-4}$, so $k <
  10,000$. Variance is removed from a region twice this size, because
  the region extends for 10,000 bases each way from the selected
  site. The size of the affected region is therefore 20,000 bases, or
  20~kb.
\end{Answr}
\begin{Answr}{8.17}
The affected region covers 2~Mb
\end{Answr}
\begin{Answr}{8.18}
 $s=0.05$.
\end{Answr}
\begin{Answr}{8.19}
There are many possible answers. Here are a few examples.
\begin{inparaenum}[(a)]
\item Sites 1--12 in sequences 41--42.
\item Sites 1--15 in sequence 31.
\item Sites 38-42 in sequences 40--49, excluding sequence 46.
\end{inparaenum}
\end{Answr}
\begin{Answr}{8.20}
There are many possible answers. Here are a few examples.
\begin{inparaenum}[(a)]
\item Sites 1--12 in sequences 41--42.
\item Sites 1--15 in sequence 31.
\item Sites 38-42 in sequences 40--49, excluding sequence 46.
\end{inparaenum}
\end{Answr}
\begin{Answr}{9.1}
The data tell us that
$P_{11} = 100/500 = 0.2$, and $P_{12} = 200/500 = 0.4$, so the
frequencies of alleles $A_1$ and $A_2$ are $p = P_{11} + P_{12}/2 =
0.4$, and $q = 1-p = 0.6$. To calculate $F$, we might work with the
formula for any of the three genotype frequencies. Let's use $P_{12} =
2pq(1-F)$. Rearranging this gives $F = 1 - P_{12}/2pq$, which works
out to equal 0.167.
\end{Answr}
\begin{Answr}{9.2}
The data tell us that
$P_{11} = 200/500 = 0.4$, and $P_{12} = 100/500 = 0.2$, so the
frequencies of alleles $A_1$ and $A_2$ are $p = P_{11} + P_{12}/2 =
0.5$, and $q = 1-p = 0.5$. To calculate $F$, we might work with the
formula for any of the three genotype frequencies. Let's use $P_{12} =
2pq(1-F)$. Rearranging this gives $F = 1 - P_{12}/2pq$, which works
out to equal 0.6.
\end{Answr}
\begin{Answr}{9.3}
$P_{11} = 0.156$
\end{Answr}
\begin{Answr}{9.4}
$P_{11} = 0.313$
\end{Answr}
\begin{Answr}{9.5}
1/8
\end{Answr}
\begin{Answr}{9.6}
1/16
\end{Answr}
\begin{Answr}{9.7}
$1/16$
\end{Answr}
\begin{Answr}{9.8}
$P_{11} = 0.19$,
$P_{12} = 0.42$, and
$P_{22} = 0.39$.
\end{Answr}
\begin{Answr}{9.9}
$P_{11} = 0.0156$,
$P_{12} = 0.169$, and
$P_{22} = 0.816$.
\end{Answr}
\begin{Answr}{9.10}
$P_{11} = p^2 + pqf$,
$P_{12} = 2pq(1-f)$, and
$P_{22} = q^2 + pqf$.
\end{Answr}
\begin{Answr}{10.1}
$G_S = 1-H_S$, and $G_T = 1-H_T$. Substituting these values into
  equation~\ref{eq.Fst.G} gives
\[
F_{ST} = \frac{1-H_S - 1 + H_T}{1 - 1 + H_T} = \frac{H_T - H_S}{H_T},
\]
which is the same as equation~\ref{eq.Fst.H}.
\end{Answr}
\begin{Answr}{10.2}
We will work with equation~\ref{eq.Fst.H}, which is a function of
$H_S$ and $H_T$. Let us begin with the formula for $H_S$:
\begin{eqnarray*}
H_S &=& \sum c_i 2p_i(1-p_i)\\
    &=& 2\sum c_i p_i - 2\sum c_i p_i^2\\
    &=& 2p - 2\sum c_i p_i^2
\end{eqnarray*}
The sum here is a weighted average over subpopulations, so think of it
as an expectation, and use the hint mentioned in the text of this
question: $\sum c_i p_i^2 = V + p^2$, where $V$ is the variance of the
$p_i$. This gives
\begin{equation}
H_S = 2p - 2p^2 - 2V = 2p(1-p) - 2V
\label{eq.wahlund}
\end{equation}
By definition, $H_T= 2p(1-p)$. Substitute this, along with
equation~\ref{eq.wahlund}, into equation~\ref{eq.Fst.H}, and you will
get
\begin{eqnarray*}
F_{ST} &=& (H_T - H_S)/H_T\\
      &=& \frac{2p(1-p) - 2p(1-p) + 2V}{2p(1-p)}\\
      &=& V/p(1-p)
\end{eqnarray*}
\end{Answr}
\begin{Answr}{10.3}
The question only asks for $F_{ST}$, but I'm providing more detail:
$H_S = 0.429123$,  $H_T=0.495905$, and $F_{ST}=0.134666$.
\end{Answr}
\begin{Answr}{10.4}
The question only asks for $F_{ST}$, but I'm providing more detail:
$H_S = 0.185367$,  $H_T=0.482516$, and $F_{ST}=0.615832$.
\end{Answr}
\begin{Answr}{10.5}
The average is 0.375246. For these two loci, $F_{ST}$ is appreciably
larger than it is for the average locus in Wright's (or anyone else's)
data.
\end{Answr}
\begin{Answr}{10.6}
$P_{AA} = 0.125$, $P_{Aa} = 0.25$, and $P_{aa} = 0.625$,
\end{Answr}
\begin{Answr}{10.7}
$P_{AA} = 0.275$, $P_{Aa} = 0.45$, and $P_{aa} = 0.275$,
\end{Answr}
\begin{Answr}{10.8}
It implies that $Nm=2$.
\end{Answr}
\begin{Answr}{11.1}
In the absence of gene flow, patterns $yn$ and $xn$ arise only
when the European, African, and Neanderthal lineages all coalesce
within the ancestral population---a process called ``incomplete
lineage sorting.'' In that case, the three lineages
are equally likely to coalesce in any order, so all three site
patterns are equally likely.
\end{Answr}
\begin{Answr}{12.1}
Line~1: $\bar X = 3.94$; $V = 0.865$.
\end{Answr}
\begin{Answr}{12.2}
Line~2: $\bar X = 4.48$; $V = 0.959$.
\end{Answr}
\begin{Answr}{12.3}
Line~3: $\bar X = 3.99$; $V = 0.899$.
\end{Answr}
\begin{Answr}{12.4}
Line~4: $\bar X = 3.55$; $V = 0.876$.
\end{Answr}
\begin{Answr}{12.5}
Pooled: $\bar X = 3.99$; $V_P = 1.002$.
\end{Answr}
\begin{Answr}{12.6}
Variance between lines: 0.1454.
\end{Answr}
\begin{Answr}{12.7}
Variance w/i lines: 0.8998.
\end{Answr}
\begin{Answr}{12.8}
  Between plus within: $0.1454 + 0.8998 = 1.04520$. In
  question~\ref{it.pooledvar}, I got $V_P = 1.002$, which is pretty
  close.
\end{Answr}
\begin{Answr}{12.9}
According to exercise~\ref{it.VG}, $V_G = 0.1454$. According to
exercise~\ref{it.pooledvar}, $V_P = 1.002$. The ratio of these is $H^2
= V_G/V_P = 0.1451$.
\end{Answr}
\begin{Answr}{12.10}
Broad-sense heritability.
\end{Answr}
\begin{Answr}{12.11}
The realized heritability is $h^2 = 0.2/4 = 0.05$, because $0.2/4$ is
the average response \emph{per generation} to selection. The
broad-sense heritability estimated in exercise~\ref{ex.H} was much
larger---0.1451---suggesting that there is a lot of non-additive
genetic variance.
\end{Answr}
\begin{Answr}{12.12}
$V_A = h^2 V_P = 0.05 \times 1.002 = 0.0501$. The dominance variance
is $V_D = V_G - V_A = 0.1454 - 0.0501 = 0.0953$. Most of the genetic
variance seems to be dominance variance. This is disappointing because
selection responds only to additive variance.
\end{Answr}
\begin{Answr}{12.13}
  The regression of offspring on mid-parent has slope $h^2$. (See
  Gillespie's table~6.2, p.~149.) From exercise~\ref{it.h}, we know
  that $h^2 = 0.05$. The regression line should have approximately
  this slope.
\end{Answr}
\begin{Answr}{12.14}
  The response to selection is $R = h^2 S$, as shown in Gillespie's
  Eqn.~6.11. If you made $h^2$ larger while keeping $S$ the same, the
  response would be larger. But this might be hard to do, because if
  each plant is represented by its average seed weight, differences
  among plants will probably be small. For this reason, it might be
  necessary to reduce $S$ in order to make selection feasible.
\end{Answr}
\begin{Answr}{12.15}
  The correlation between parent and offspring is $h^2/2$. Thus, we
  estimate $h^2$ as twice the observed correlation. For stature, this
  gives $h^2 = 1.014$. This cannot be literally correct, because $h^2$
  cannot exceed~1. But all estimates contain error, and this error can
  move the estimate outside the legal range. This estimate implies
  that stature is highly heritable.
\end{Answr}
\begin{Answr}{12.16}
  The correlation between parent and offspring is $h^2/2$. Thus, we
  estimate $h^2$ as twice the observed correlation. For span, this
  gives $h^2 = 0.904$.
\end{Answr}
\begin{Answr}{12.17}
  The correlation between parent and offspring is $h^2/2$. Thus, we
  estimate $h^2$ as twice the observed correlation. For cubit, this
  gives $h^2 = 0.842$.
\end{Answr}
\begin{Answr}{12.18}
The change per generation is $0.157 \times 28/10 = 0.4396$~in. If this
represents a response to selection, then it should equal $V_A \beta$. The next
step is to convert our estimate that $h^2 \approx 1$ into an estimate of $V_A$.
We know that $h^2 = V_A/V_P$ (Gillespie's Eqn.~6.4), and we know that $V_P =
6.26$ (table~\ref{tab.pearson} of this homework). This implies that $V_A
\approx 6.26$ and that the selection gradient is $\beta \approx 0.4396 / 6.26 =
0.07$. In words, this says that an extra inch of stature implies nearly a 10\%
increase in fitness---\emph{very} strong selection. It seems implausible that
small differences in stature could have such large effects on fitness. It is
more likely that the observed trend reflects changes in the environment.
\end{Answr}
\begin{Answr}{12.19}
The problem says that $h^2 = 0.8$ and that the phenotypic
standard deviation is $\sqrt{V_P} = 1$. This implies that $V_P=1$. The
problem also says that $h^2=0.8$. This implies that $V_A = h^2 V_P =
0.8$. Finally, the response to selection is $R = V_A \beta$, and we
know that $R=0.5$. Thus, $\beta = 0.5/0.8 = 0.625$.
\end{Answr}
}{}
}{%
  \newpage\mbox{}
  \newpage\mbox{}
  \newpage\mbox{}
  \newpage\mbox{}
  \newpage\mbox{}
  \newpage\mbox{}
  \newpage\mbox{}
  \newpage\mbox{}
  \newpage\mbox{}
}
\end{document}

\begin{exercise}
Use the data in the table below to estimate $F_{ST}$.  (Hint: you only
need the last line.)
\begin{center}
\begin{tabular}{lcc}
Population & $p$ & $p^2 + q^2$\\
\hline
African    & 0.734 & 0.610\\
European   & 0.496 & 0.500\\
Australian & 0.330 & 0.558\\
East Asian & 0.259 & 0.616\\
\hline
Average    & 0.455 & 0.571\\
\end{tabular}
\end{center}
\begin{answer}
\begin{verbatim}
>>> p = 0.455
>>> gt = 1 - 2*p*(1-p)
>>> gs = 0.571
>>> fst = (gs-gt)/(1-gt)
0.13499344692005213
\end{verbatim}
\end{answer}
\end{exercise}

\begin{exercise}
Archaeologists and geneticists have argued for decades about what
happened during the Neolithic, as farming moved into Europe. Some
argued that the Neolithic was an invasion of farmers into regions
previously occupied by hunter-gatherers. Others argued that the
idea of farming spread, but not of the people themselves. According to
this view, the Neolithic spread as hunter-gathers learned from their
neighbors how to farm. In the past few years, DNA from ancient
skeletons has settled this debate. Summarize this evidence in a couple
of sentences.
\begin{answer}
Human remains from early Neolithic sites in Europe have DNA unlike
that of the hunter-gatherers who previously lived in the same
areas. The early farmers had DNA similar to that found today in
Sardinia and Corsica. This evidence shows that the Neolithic was an
invasion of farmers, not merely a movement of the idea of farming.
\end{answer}
\end{exercise}

\section{Human evolutionary history}

\begin{exercise}
  Summarize (in a few sentences) the evidence for gene flow from
  archaic human populations into the modern populations of Eurasia.
\begin{answer}
As modern humans expanded out of Africa and across Eurasia, where was
some admixture with pre-existing archaic populations. It appears that
around 2\% of the DNA of modern Europeans derives from
Neanderthals. Similar levels are found throughout Eurasia. In
Melanesia, there is also appreciable admixture from a distinct archaic
population, the Denisovans.
\end{answer}
\end{exercise}
