%-*-latex-*-
\documentclass[11pt]{article}
\newcommand{\blank}{\rule{2in}{0.4pt}}
\usepackage[letterpaper,margin=1in]{geometry}
\usepackage{hyperref}
\usepackage{longtable}
\usepackage[backend=biber,
  natbib=true,
  style=ext-numeric-comp,
  sorting=nyt]{biblatex}
\DeclareFieldFormat % Titles in sentence case
  [article,inbook,incollection,inproceedings,patent,thesis,unpublished]
  {titlecase:title}{\MakeSentenceCase*{#1}}
\renewbibmacro{in:}{ % Don't use "In:" before journal names.  
  \ifentrytype{article}{}{\printtext{\bibstring{in}\intitlepunct}}}  
\addbibresource{defs.bib}
\addbibresource{arr.bib}
\addbibresource{molrec.bib}
\addbibresource{arrpubs.bib}
\addbibresource{arrunpub.bib}
\addbibresource{qchar.bib}
%\usepackage{datetime}
%\renewcommand{\dateseparator}{-}
%\yyyymmdddate
%\usepackage{jeep}
%\cfoot{\thepage}
\newcommand{\heading}[1]{\bigskip\noindent\textbf{#1}~}
\usepackage{xr-hyper}
\externaldocument[H-]{../homework/prob}
\externaldocument[H-]{../homework/drift}
\externaldocument[H-]{../homework/randommate}
\externaldocument[H-]{../homework/theta}
\externaldocument[H-]{../homework/neutraltheo}
\externaldocument[H-]{../homework/tree-b}
\externaldocument[H-]{../homework/seln}
\externaldocument[H-]{../homework/twoloc}
\externaldocument[H-]{../homework/inbreed}
\externaldocument[H-]{../homework/pstruc}
\externaldocument[H-]{../homework/admix}
\externaldocument[H-]{../homework/qchar}
\externaldocument[H-]{../homework/mmspec}
\begin{document}
\begin{flushleft}
Biol 5221     \hfill  Generated \today\\
Profs: Angela Hancock, Alan Rogers, Jon Seger\\
Lecture: Tue,Thu 10:45AM--12:05PM \hfill Lab: Wed 1--3PM\\
\end{flushleft}
\begin{center}
  \url{https://content.csbs.utah.edu/~rogers/tch/ant5221/index.php}
\end{center}

\section*{\centering Human Evolutionary Genetics}

\heading{Description} This course covers theories and methods of
molecular population genetics, with emphasis on human examples. Using
these tools, genetic data can inform us about population history and
adaptive evolution. Laboratory exercises with the Python programming
language connect theory to data. Satisfies Quantitative Intensive
Requirement.

We will meet twice a week for lecture and once a week for
lab. Lectures are in person. For labs, students will log onto Zoom
sessions via Canvas.

\heading{Prerequisites} You should be comfortable with algebra and
first-semester calculus.  No prior knowledge of Python is
needed.

\heading{Grading} is based on labs, homework assignments, and
quizzes. Each lab, homework assignment, and quiz contributes equally
toward the final grade.

\heading{Extra credit} A 2-point extra credit assignment is available
on the course website. It provides practice in algebra and is
available only during the first half of the semester. The due date is
listed in the schedule below.

\heading{Quizzes} are administered via Canvas and are open book.  Most
questions will be multiple choice, but you may need to do calculations
to choose the right answer.

\heading{Weekly computer lab} In this lab, students do projects using
the Python programming language. The lab assignments are short enough
to complete during the two-hour lab. The lab syllabus is available on
the class web site. The projects themselves are described in JEPy and
in the \emph{Lab Manual for Biol 5221}, which is also available on the
website.

\heading{Homework} There are also paper-and-pencil homework
assignments, which are due at roughly weekly intervals as indicated in
the schedule below. The homework assignments are available on the
class website. Answers to even-numbered problems are in the back of
the book of assignments. Only odd-numbered problems will be graded.

Write your homework assignments on paper, with your name and the title
of the assignment at the top. Number each page. If there are 3 pages,
the numbers should look like ``1/3,'' ``2/3,'' and ``3/3,'' so that we
can tell that we have all the pages. Finally, scan the assignment or
take a cell-phone photograph, and upload to Canvas. There is a Canvas
cell-phone app, will allow you to photograph several pages, one
at a time, and upload them all together.

\heading{Late assignments} If assignments are late, but not more than
one week late, we'll deduct 10\% of the grade. We intended not to
accept assignments more than one week, but that policy was not
reflected in the syllabus. To keep the promise made in the original
syllabus, we're modifying the policy as follows. For assignments due
on or before 9 March 2024, we will accept submissions until the last
week of class, with a 50\% penalty if they are more than one week
late. For assignments due after 9 March, we will not accept
assignments more than one week late. No late assignments will be
accepted during the last week of the course. If the assignment is
delayed because of a medical problem or other crisis, there is no
penalty.

\heading{Required readings} are listed numerically in the outline
below. The full reference corresponding to each number is in the list
of references at the end. The main text,
\begin{quote}
Gillespie, John. 2004.
\emph{Population Genetics, a Concise Guide}, 2nd edition
\end{quote}
is available at the campus bookstore (also in paper and electronic
versions on the web). All other readings are on the class website. In
addition, we will occasionally assign other published papers and notes
of our own.  When we do, they will be available either on paper or on
the course web site.

There are also two excellent texts, which are freely available online:
\begin{enumerate}
  \item
    \href{https://web.stanford.edu/group/pritchardlab/HGbook.html}{Pritchard,
      Jonathan. 2024} \emph{An Owner's Guide to the Human Genome:
      An Introduction to Human Population Genetics, Variation and
      Disease}.
  \item
    \href{https://github.com/cooplab/popgen-notes/releases/tag/v1.1}{Coop,
      Graham. 2024.} \emph{Population and Quantitative Genetics}.
\end{enumerate}
These are not required, but you may find them helpful. We suggest
portions of them as optional readings on the course website.

\heading{Contact} Feel free to contact us either through Canvas or by
email. \emph{Hancock} \url{hancock@mpipz.mpg.de}; \emph{Rogers}:
\url{rogers@anthro.utah.edu}; \emph{Seger}:
\url{seger@biology.utah.edu}.

\heading{Learning outcomes}
\begin{description}
\item[Evolution] Students will be able to use the principles of evolutionary
theory to interpret patterns of variation within species, and of change over
time.

\item[Scientific reasoning] Students will be able to apply the method
  of critical scientific reasoning to identify knowledge gaps,
  formulate hypotheses, and test them against experimental and
  observational data to advance an understanding of the natural world.

\item[Quantitative reasoning] Students will be able to use
  mathematical and computational methods and tools to describe living
  systems and be able to apply quantitative approaches, such as
  statistics, quantitative analysis of dynamic systems, or
  mathematical modeling.
\end{description}

\heading{Americans with Disabilities Act} The University of Utah seeks
to provide equal access to its programs, services, and activities for
people with disabilities. If you will need accommodations in this
class, reasonable prior notice should be given to the Center for
Disability and Access, 162 Olpin Union Building, (801) 581-5020.

\heading{Attendance} Given the nature and goals of this course,
attendance is required and adjustments cannot be granted to allow
non-attendance. However, if you need to seek an ADA accommodation to
request an exception to this attendance policy due to a disability,
please contact the Center for Disability and Access (CDA). CDA will
work with us to determine what, if any, ADA accommodations are
reasonable and appropriate

\heading{University Safety Statement} The University of Utah values
the safety of all campus community members. To report suspicious
activity or to request a courtesy escort, call campus police at
801-585-COPS (801-585-2677). You will receive important emergency
alerts and safety messages regarding campus safety via text
message. For more information regarding safety and to view available
training resources, including helpful videos, visit
\url{https://safeu.utah.edu}.

\heading{Sexual Misconduct} Title~IX makes clear that violence and
harassment based on sex and gender (including sexual orientation and
gender identity/expression) is a civil rights offense subject to the
same kinds of accountability and support applied to offenses against
other protected categories such as race, national origin, color,
religion, age, status as a person with a disability, veteran’s status
or genetic information. If you or someone you know has been harassed
or assaulted, you are encouraged to report it to the Title~IX
Coordinator in the Office of Equal Opportunity and Affirmative Action,
383 South University Street, 801--581--8365, or the Dean of Students,
270 Union Building, 801--581--7066. For support and confidential
consultation contact the Center for Student Wellness, 426~SSB,
801--581--7776. To report to the police, contact the Department of
Public Safety, 801-585-2677 (COPS).

\heading{Academic Misconduct} It is expected that students adhere to
University of Utah policies regarding academic honesty, including but
not limited to refraining from cheating, plagiarizing, misrepresenting
one's work, and/or inappropriately collaborating. This includes the
use of generative artificial intelligence (AI) tools without citation,
documentation, or authorization. Students are expected to adhere to
the prescribed professional and ethical standards of the
profession/discipline for which they are preparing. Any student who
engages in academic dishonesty or who violates the professional and
ethical standards for their profession/discipline may be subject to
academic sanctions as per the University of Utah’s 
\href{https://regulations.utah.edu/academics/6-410.php}{Student Code}. 

\setlength\LTleft{0pt}
\setlength\LTright{0pt}
\setlength\tabcolsep{0.25em}

\bigskip

\raggedright
\begin{longtable}{l@{\,}lp{0.55\textwidth}lp{0.3\textwidth}}
  \multicolumn{5}{c}{\Large\bf Schedule}\\
\multicolumn{2}{l}{\textbf{Date}}&\textbf{Lecture}&&\textbf{Reading}\\ \hline
Jan&10 W&2nd lecture on probability, plus a brief introduction to Python. No lab report required.&\\
   &17 W&Python: introduction&JEPy: Chs~0--1\\
   &    &Lab~1: interacting with the Python shell&\\
   &24 W&Python: loops and lists&JEPy: Ch~2\\
   &    &Lab~2: Were Wolf's dice fair?&\\
   &31 W&Python: list magic and functions&JEPy: Ch~3--4\\
   &    &Lab~3: Using variance to study Wolf's dice.&\\
Feb&07 W&Lab~\ref{L-ch.drift}: Simulating drift and mutation&LabMan: Ch~\ref{L-ch.drift}\\
   &14 W&Lab~\ref{L-ch.coal1}: Simulating gene genealogies&LabMan: Ch~\ref{L-ch.coal1}\\
   &21 W&Lab~\ref{L-ch.stattest}: Using simulation to test a statistical hypothesis&LabMan: Ch~\ref{L-ch.stattest}\\
   &28 W&Lab~\ref{L-ch.seldrft}: Simulating selection and drift&LabMan: Ch~\ref{L-ch.seldrft}\\
Mar&06 W&\verb|***| NO CLASS&\\
   &13 W&Lab~\ref{L-ch.hapintro}: Using HapMap and catching up&LabMan: Ch~\ref{L-ch.hapintro}\\
   &20 W&Lab~\ref{L-ch.haphet}: HapMap heterozygosity&LabMan: Ch~\ref{L-ch.haphet}\\
   &27 W&Lab~\ref{L-ch.twolocsim}: Simulating selection and drift at two loci&LabMan: Ch~\ref{L-ch.twolocsim}\\
Apr&03 W&Lab~\ref{L-ch.rsq}: LD in the human genome&LabMan: Ch~\ref{L-ch.rsq}\\
   &10 W&Lab~\ref{L-ch.lactase}: LD near human lactase gene&LabMan: Ch~\ref{L-ch.lactase}\\
   &17 W&Nothing planned&\\

\end{longtable}  

\printbibliography
\end{document}

\end{document}


