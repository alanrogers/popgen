\chapter{Genetic Drift}
\label{hw.drift}

Each question is worth 10 points.

\section{Drift acting alone}
\label{sec.driftonly}
In the first few problems, we assume that the only force at work is
genetic drift.  There is no selection, no mutation, and no migration.
The first few problems all involve the equation
\begin{equation}
H_t = H_0(1-1/2N)^t
\label{eq.Ht}
\end{equation}
which describes the decay of heterozygosity in a randomly-mating
population. 

\begin{exercise}
In the urn model of genetic drift, there are $2N$ balls in the urn of
which a fraction $p$ are black.  The remaining fraction, $1-p$, are
white. Suppose we draw $2N$ balls from the urn with replacement.  Of
the balls that we draw, a random number $X$ are black, and the
remaining $2N-X$ are white.  What are the \emph{mean} and
\emph{variance} of $X$?  (Your answers should be in terms of $2N$ and
$p$.)
\begin{answer}
$m=2Np$, $V=2Np(1-p)$
\end{answer}
\end{exercise}

\begin{exercise}
Consider a very small population in which $2N=10$, mating is at
random, and genetic drift is the only evolutionary force at work. If
heterozygosity equals $1/2$ in one generation, what is its expected value
in the following generation?
\begin{answer}
\[
\frac{1}{2} \times \left(1 - \frac{1}{10}\right) = \frac{9}{20} = 0.45
\]
\end{answer}
\end{exercise}

\begin{exercise}
Consider a larger population in which $2N=1000$, mating is at
random, and genetic drift is the only evolutionary force at work. If
heterozygosity equals $1/2$ in one generation, what is its expected value
in the following generation?
\begin{answer}
\[
\frac{1}{2} \times \left(1 - \frac{1}{1000}\right) = \frac{999}{2000} = 0.4995
\]
\end{answer}
\end{exercise}

\begin{exercise}
In Series~I of Buri's experiment, the initial heterozygosity was
$H_0=0.514$ and in the 18th generation it was $H_{18} = 0.183$.
Take these values as given and solve equation~\ref{eq.Ht} for $2N$.  In
your answer, include two digits to the right of the decimal point.
How does your estimate compare to the one $(2N=18)$ that Buri got?
\begin{answer}
$2N=17.93$.
\end{answer}
\end{exercise}

\begin{exercise}
Suppose that $2N=18$ on average, but that the value was not
constant.  Specifically, suppose that $2N=26$ for the first 9
generations and $2N=10$ for the last 9.  (Note that these numbers
average to 18.)  What value would you predict for $H_{18}$?  Compare
this to the answer you get when $2N=18$ in every generation.  Which
answer is higher?  How does variation in population size affect the
decay of heterozygosity?
\begin{answer}
$H_0 (1-1/26)^9 (1-1/10)^9 = 0.14$, smaller than the observed value of
$H_t$.  Variation in population size accelerates the decay of
heterozygosity.
\end{answer}
\end{exercise}

\begin{exercise}
Suppose now that the population size alternated between $2N=26$
and $2N=10$.  It was 26 in the 1st generation, 10 in the 2nd, 26
again in the 3rd, and so on.  What is the expected heterozygosity in
generation~18? (Hint: Think carefully about equation~\ref{eq.Ht}.  There
is a way to answer this without doing any additional calculations.)
\begin{answer}
Same as above.  We start with formulas like this:
\begin{eqnarray*}
H_9 &=& H_0 (1-1/26)^9\\
H_{18} &=& H_9 (1-1/10)^9\\
\end{eqnarray*}
Substituting $H_9$ into the second equation gives
\[
H_{18} =  H_0 (1-1/26)^9(1-1/10)^9
\]
You can multiply in any order, so the answer to the second problem is
the same as that of the first.
\end{answer}
\end{exercise}

\begin{exercise}
In Buri's experiment, how many generations would it take before
  $H_t=0.1$?  (Assume $2N=18$.)
\begin{answer}
$\log(0.1/H_0)/\log(1-1/18) = 28.64$
\end{answer}
\end{exercise}

\begin{exercise}
Now rearrange equation~\ref{eq.Ht} so that $t$ is on the left side,
and everything else is on the right.  (Hint: The right side will be an
algebraic expression involving the symbols $H_0$, $H_t$, and $2N$.
There will be logarithms.)  Use your answer to this question to check
your answer to the preceding one.
\begin{answer} 
\[ t = \frac{\ln(H_t/H_0)}{\ln(1 - 1/2N)} \]
\end{answer}
\end{exercise}

\begin{exercise}
Suppose that a catastrophe reduces diploid population size from 1024 to 2. 
Thereafter, the population doubles every generation until it reaches
its original size. What fraction of the original heterozygosity still
remains? To do this precisely, you would want to ask whether the
species had separate sexes. We haven't taught you how to do this, so
please assume that these organisms have monoecious sexual reproduction
and mate at random. Then the theory we have taught in class works all
the way down to $N=2$. Feel free to do this either by hand or in
Python.
\begin{answer}
Here is the solution in Python:
\begin{verbatim}
>>> n = 2.0  # current population size
>>> h = 1.0  # relative heterozygosity
>>> while n < 1024.0:
...     h *= 1-1/(2.0*n) # drift
...     n *= 2.0         # pop growth
...     print "H/H0 = %f" %h
... 
H/H0 = 0.750000
H/H0 = 0.656250
H/H0 = 0.615234
H/H0 = 0.596008
H/H0 = 0.586696
H/H0 = 0.582112
H/H0 = 0.579838
H/H0 = 0.578706
H/H0 = 0.578141
\end{verbatim}
Note that most of the drift happens in the first few generations.
Even a very severe bottleneck removes only a fraction of the variance,
provided that recovery is rapid.
\end{answer}
\end{exercise}

\section{Mutation and drift}
\label{sec.mutdrift}
Gillespie shows that, under the ``infinite alleles'' model of
mutation, heterozygosity $(H)$ evolves under the combined effects of
mutation and genetic drift toward an equilbrium value,
\begin{equation}
\hat H = \frac{\theta}{1+\theta}
\label{eq.eqhet}
\end{equation}
where $\theta = 4Nu$.
\begin{exercise}
At autosomal loci, the mutation rate per gene is often around
$u=10^{-6}$ per generation.  Let us suppose this is so for Buri's
flies.  Assuming once again that the effective population size is
$2N_e=18$, calculate the equilibrium heterozygosity in Buri's
experiment. This value represents the expected heterozygosity within
each bottle in the long run---after many generations of simulaneous
mutation and drift.  At this equilibrium, how many of the 16 flies in
each bottle would be likely to be heterozygous? Round your answer to
the nearest integer.
\begin{answer}
  $4Nu = 36 \times 10^{-6}$, and the expected heterozygosity is
  $4Nu/(1+4Nu)$, or 0.000036.  The expected number of heterozygous
  flies is 16 times this number, or 0.00058, which rounds to 0. You
  would probably see no heterozygous flies.
\end{answer}
\end{exercise}

\begin{exercise}
  Solve Eqn.~\ref{eq.eqhet} algebraically for $\theta$ as a function
  of $\hat H$.
\begin{answer}
$\theta = \hat H/(1-\hat H)$
\end{answer}
\end{exercise}

The model of infinite alleles, which underlies Eqn.~\ref{eq.eqhet}, is
plausible model for protein-coding loci, which may contain thousands
of nucleotide sites. The number of possible alleles is so large that
we get a good approximation by taking that number to be infinite. This
approximation works poorly, however, for individual nucleotide
sites. These have at most 4 alleles---A, T, G, and C. If there are $k$
possible alleles, and each allele mutates with equal probability to
each of the other alleles, then Eqn.~\ref{eq.eqhet} becomes
\begin{equation}
\hat H = \frac{\theta}{1+\theta k/(k-1)}
\label{eq.eqhetk}
\end{equation}
This model is also unrealistic, however, because it is seldom true
that each allele mutates with equal probability to each other allele.
Typically, the most common type of mutation is the ``transition,''
which toggles the nucleotide back and forth either between A and G or
between C and T. In human mitochondrial DNA, for example, the vast
majority of mutations are of this type. When this is so, we get a
reasonable approximation by assuming that $k=2$ in
Eqn.~\ref{eq.eqhetk}. Let us call this the ``symmetric biallelic''
model of mutation, because it implies that each polymorphic site has 2
alleles, each of which mutates to the other at the same rate. For this
model, 
\[
\hat H = \frac{\theta}{1+2\theta}
\]
\begin{exercise}
What is the largest value that $\hat H$ can take in the model of
infinite sites?
\begin{answer}
Heterozygosity is maximal when $\theta\rightarrow\infty$. For the
model of infinite sites, this gives $\max\hat H = 1$.
\end{answer}
\end{exercise}

\begin{exercise}
What is the largest value that $\hat H$ can take in the symmetric
biallelic model? 
\begin{answer}
Heterozygosity is maximal when $\theta\rightarrow\infty$. For the
symmetric biallelic model, $\max\hat H = 1/2$.
\end{answer}
\end{exercise}

\begin{exercise}
  Suppose that random pairs of mitochondria differ on average at 0.5\%
  of nucleotide sites, that the population is at mutation-drift
  equilibrium, that the mutation rate is $u=2 \times 10^{-7}$ per
  nucleotide per generation, and that the symmetric biallelic model of
  mutation applies. What does this imply about the effective number of
  females? (Hint, we can use Eqn.~\ref{eq.eqhetk}, provided that we
  interpret the number, $2N$, of gene copies in the population as the
  effective number of females.)
\begin{answer}
  The calculation is based on Eqn.~\ref{eq.eqhetk}, because we are
  using the symmetric biallelic model of mutation. The question tells
  us that $k=2$ and $\hat H = 0.005$. Eqn~\ref{eq.eqhetk} becomes
  $0.005 = \theta/(1 + 2\theta)$, or $\theta=0.00505$. And since
  $\theta = 4Nu$, where $u = 2\times 10^{-7}$, we have $2N = 12626$.
\end{answer}
\end{exercise}

\begin{exercise}
  Suppose that the average individual is heterozygous at $1/3$ of
  protein-coding loci, that the population is at mutation-drift
  equilibrium, that the mutation rate is $u=10^{-6}$ per locus per
  generation, and that the infinite alleles model of mutation
  applies. What does this imply about $N$?
\begin{answer}
  The calculation is based on Eqn.~\ref{eq.eqhet}, because we are
  using the model of infinite alleles.  The question tells us that
  $\hat H = 1/3$, so Eqn~\ref{eq.eqhet} becomes $1/3 = \theta/(1 +
  \theta)$, or $\theta=1/2$. And since $\theta = 4Nu$, where $u =
  10^{-6}$, we have $N = 125000$.
\end{answer}
\end{exercise}

