\chapter{Gene Genealogies}
\label{hw.theta}

In the following exercises, assume that the population size is
constant and that the genetic variation under study is selectively
neutral. Each exercise is worth 10 points.

The exercises below are all based on material in \emph{Lecture Notes
  on Gene Genealogies} (LNGG). Those in section~\ref{sec.genealogy} 
are based on chapter~\ref{G-ch.coalescent} of LNGG, whereas those in 
section~\ref{sec.theta} are based on chapter~\ref{G-ch.addmut}. There
is one exception to this: exercise~\ref{it.totalbranch} is based on 
chapter~\ref{G-ch.addmut}.

\section{Gene genealogies without mutation}
\label{sec.genealogy}

\begin{exercise}
  Figure~\ref{fig.segertree} (on page~\pageref{fig.segertree}) shows a
  made-up gene genealogy of 6 DNA sequences. The branches are labeled
  with capital letters. (a)~Which sequences would carry the derived
  allele if a mutation occurred on branch H? (b)~On branch B? (c) On
  branch G?
\begin{answer}
(a)~5, (b)~1 and 2, (c)~4, 5, and 6.
\end{answer}
\end{exercise}

\begin{figure*}
\begin{verbatim}
          1 --A-
                |-B--
          2 --C-     |
                     |-----D-----
          3 ---E-----            |
                                 |-----------
          4 ----F-----           |
                      |----G-----
          5 --H-      |
                |--I--
          6 --J-
\end{verbatim}
\caption{A genealogy with labeled branches. The samples
  are numbered (1--6) and the branches are labeled (A--J).}
\label{fig.segertree}
\end{figure*}

For the exercises in section~\ref{sec.genealogy}, assume that
$2N=5000$.  

\begin{exercise}
What is the expected duration in generations of coalescent intervals with
\begin{inparaenum}[(a)]
\item 2 lines of descent,
\item 10 lines of descent, and
\item 1000 lines of descent?
\end{inparaenum}
\begin{answer}
For an interval with $i$ lines of descent, the expected duration in
generations is $4N/[i(i-1)]$. For our problem, $2N=5000$. The expected
duration is therefore.
\begin{inparaenum}[(a)]
\item $5000$ for 2 lines of descent,
\item $1000/9 \approx 111$ for 10, and
\item $10000/(1000 \times 999) = 10/999 \approx 0.01$ for 1000.
\end{inparaenum}
\end{answer}
\end{exercise}

\begin{exercise}
In an interval with 2 lines of descent, what are the \emph{mean}, the
\emph{variance} and the \emph{standard deviation} of the interval's
duration? Hint: the duration of the interval is an
exponentially-distributed random variable. This distribution is
discussed in JEPr. 
\begin{answer}
The duration of a coalescent interval is an exponential random
variable. As there are only 2 lines of descent, the hazard is
$h=1/2N$. As explained in JEPr, the mean of this random variable is
$1/h = 2N$, the variance is $1/h^2 = 4N^2$, and the standard
deviation (the square root of the variance) is $2N$. If $2N=5000$, the
mean, variance, and standard deviation are 5000, 25,000,000, and 5000.
\end{answer}
\end{exercise}

For exercises~\ref{it.treedepth} and~\ref{it.totalbranch}, assume that
you have a sample consisting of $K=3$ gene copies chosen at random from a
population of size $2N=5000$.

\begin{exercise}
\label{it.treedepth}
What is the expected depth of the gene genealogy? In other words, the
expected age in generations of the last common ancestor?
\begin{answer}
We have 2 coalescent intervals, one with 3 lines of descent and the
other with 2. The expected duration of the one with 3 lines of descent
is $5000/3$ and the expected duration of the other is 5000
generations. The expected depth of the tree is the sum of these, $5000
+ 5000/3 = 20000/3$. The easy way to get this answer is with the
formula $4N(1-1/K)$, where $K=3$ is the number of gene copies in the
modern sample. This gives the same answer, $10000 \times 2/3 =
20000/3$. 
\end{answer}
\end{exercise}

\begin{exercise}
\label{it.totalbranch}
What is the expected total branch length the gene genealogy? In other
words, the expected sum of the lengths of all branches throughout the
genealogy? (This problem is based on material in
section~\ref{G-sec.nMutations} of \emph{Lecture Notes on Gene
  Genealogies}.)
\begin{answer}
In a tree with sample of size 3, the expected total branch length is
$4N(1 + 1/2) = 6N$. In our problem, $2N=5000$, so the answer is
15,000.
\end{answer}
\end{exercise}

\section{Gene genealogies with mutation}
\label{sec.theta}
\begin{exercise}
Suppose that $\theta=10$, indicating either that the population is
very large or the mutation rate is high. How many mutations should
occur on average in coalescent intervals with 
\begin{inparaenum}[(a)]
\item 2 lines of descent,
\item 10 lines of descent, and
\item 1000 lines of descent?
\end{inparaenum}
\begin{answer}
For an interval with $i$ lines of descent, the expected length in
generations is $4N/[i(i-1)]$, and the total branch length within the
interval is $i$ times this value, or $4N/(i-1)$. The expected number
of mutations is $u$ times the total branch length, or $\theta/(i-1)$,
where $\theta = 4Nu$. For our problem, $\theta=10$. The expected
numbers of mutations are therefore
\begin{inparaenum}[(a)]
\item $10/1 = 10$ for 2 lines of descent,
\item $10/9 = 1.11$ for 10, and
\item $10/999 = 0.01$ for 1000.
\end{inparaenum}
\end{answer}
\end{exercise}

\begin{exercise}
\begin{inparaenum}[(a)]
\item Draw a gene tree with 4 tips.  What are 
\item the expected lengths of each interval (assuming the population
  consists of $N$ diploid individuals),
\item the expected depth of the tree, and 
\item the expected total tree length (i.e.\ the sum of all branch lengths)?
\end{inparaenum}
(Hint: your answers will be in terms of the unknown quantity $N$.)
\begin{answer}
\begin{minipage}{0.8\columnwidth}
\begin{verbatim}
----------
          |----
----------     |
               |--
------         |
      |--------
------
\end{verbatim}
\end{minipage}
Lengths are $4N/i(i-1)$, so the lengths are $2N/6$, $2N/3$, and $2N$,
for intervals 4, 3, and 2.  The expected depth of the tree is
$4N(1-1/4) = 3N$.  The expected total branch length is $4N(1 + 1/2 +
1/3) = 22N/3 = 7.33N$
\end{answer}
\end{exercise}

\begin{exercise}
  How many mutations would you expect to see on such a tree,
  assuming a mutation rate of $u=1/1000$ and a haploid population size
  of $2N=5000$?
\begin{answer}
18.3
\end{answer}
\end{exercise}

\begin{exercise}
  Now suppose that you doubled the sample size from 4 to 8.
  Don't draw the tree.  Just calculate the expected number of
  mutations.  How much did the number of mutations increase?
\begin{answer}
25.93
\end{answer}
\end{exercise}

\begin{exercise}
  What is the ratio between the expected value of $S$ in a sample
  of 10 DNA sequences and the expected value in a sample of 20?
\begin{answer}
0.797
\end{answer}
\end{exercise}

\begin{table}
{\centering
\begin{minipage}{\columnwidth}
\footnotesize
\begin{alltt}
\begin{tabbing}
seq01 \=AATATGGCAC CTCCCAACCC TCTAGCATAT ACCACTTACA\kill
seq01 \>AATATGGCAC CTCCCAACCC TCTAGCATAT ACCACTTACA\\
seq02 \>.......T.. .C......TG C......C.. ..........\\
seq03 \>..C....... .......... .......... ..........\\
seq04 \>.......T.. .C......TG C......... G.........\\
seq05 \>.......... .......... .......... ..........\\
seq06 \>.....A.... ........T. C......... G....C....\\
seq07 \>..C....T.. .C......TG C......... G.........\\
seq08 \>.....A.T.. TC......TG C......... G.........\\
seq09 \>.......... .......... C......... ..........\\
seq10 \>.G...A.... ........T. C......C.. .T....C..G
\end{tabbing}
\end{alltt}
\end{minipage}\\}
\caption{Data set A. Periods indicate sites that are identical to
  \texttt{seq01}.} 
\label{tab.datA}
\end{table}

\begin{table}
{\centering
\begin{minipage}{\columnwidth}
\footnotesize
\begin{alltt}
\begin{tabbing}
seq01 \=TGCCACTCCA ATCTCTCGCC AGATGGCATG CCTTATCGCG\kill
seq01 \>TGCCACTCCA ATCTCTCGCC AGATGGCATG CCTTATCGCG\\
seq02 \>.......... G......... .A.C...GCA T.........\\
seq03 \>.......... G......A.. .A.C...GC. T....C....\\
seq04 \>C..TG..T.. .C.....A.. G......C.. TT.C......\\
seq05 \>CA.TG..T.. .C....TA.. G......CC. TT.C......\\
seq06 \>C...G..... CC..T..A.A ....AA.C.. TT..G.....\\
seq07 \>CA.TG..... CC..TC.A.A ...CA..CC. TT...C....\\
seq08 \>CA.TG...T. GCT....A.. G..C..TC.. T.......T.\\
seq09 \>CA.TG..... GC.C..T... ...CA..C.A T.........\\
seq10 \>CA.TG...T. .C.....A.. G..C...C.. T....C....
\end{tabbing}
\end{alltt}
\end{minipage}\\}
\caption{Data set B}
\label{tab.datB}
\end{table}

\begin{exercise}
  Use data set~A to calculate $\pi$, the mean number of nucleotide
  site differences per sequence \emph{and} per site between pairs of
  sequences. 
\begin{answer}
Mean pairwise diff: $\pi = 5.51$ per sequence, and $\pi = 0.1378$ per
site. 
\end{answer}
\end{exercise}

In calculating $\pi$ (the number of pairwise differences), data set B
poses a new problem: two of the sites (11 and 27) have more than two
nucleotides. This will not cause a problem if you do the calculation
the laborious way, by counting the differences between each pair of
sequences. But if you use the easier site-by-site method described in
the text, you need to know how to deal with such sites.

Consider site 11, which has 4 $A$s, 4 $G$s, and 2 $C$s. We are
interested only in the pairs that have different nucleotides. In other
words, we are only interested in pairs of type $AG$, $AC$, or
$GC$. The number of $AG$ pairs is $4 \times 4=16$. This follows
because each of the 4 As can combine with each of the 4
$G$s. Similarly the number of $AC$ pairs is $4 \times 2=8$, and the
number of $GC$ pairs is $4 \times 2=8$. This site therefore
contributes $16+8+8=32$ to our count of differences.

At site 27, we have 1 $A$, 2 $G$s, and 7 $C$s. This gives $1 \times
2=2$ $AG$s, $1 \times 7=7$ $AC$s, and $2 \times 7=14$ $GC$s, so the
total contribution from site~27 is $2+7+14=23$.

\begin{exercise}
  Use data set~B to calculate $\pi$, the mean number of nucleotide
  site differences per sequence \emph{and} per site between pairs of
  sequences.
\begin{answer}
Mean pairwise diff: $\pi=11.69$ per sequence, and $\pi = 0.29 $ per
site. 
\end{answer}
\end{exercise}

\begin{table}
\caption{Values of $\sum_{k=1}^{K-1} 1/k$ for various values of $K$}
\label{tab.Ssum}

\medskip

\centering
\fbox{\begin{tabular}{rr@{\hspace{3em}}rr}
$K$& $\sum_{k=1}^{K-1} 1/k$ & $K$& $\sum_{k=1}^{K-1} 1/k$\\[2pt] 
\hline
 2 & 1.0000 & 12 & 3.0199\\
 3 & 1.5000 & 13 & 3.1032\\
 4 & 1.8333 & 14 & 3.1801\\
 5 & 2.0833 & 15 & 3.2516\\
 6 & 2.2833 & 16 & 3.3182\\
 7 & 2.4500 & 17 & 3.3807\\
 8 & 2.5929 & 18 & 3.4396\\
 9 & 2.7179 & 19 & 3.4951\\
10 & 2.8290 & 20 & 3.5477\\
11 & 2.9290 & 21 & 3.5977\\
\end{tabular}}
\end{table}

\begin{exercise}
Calculate the number $S$ of segregating sites from data set~A.
\begin{answer}
$S = 15$ per sequence or 0.375 per site
\end{answer}
\end{exercise}

\begin{exercise}
Calculate the number $S$ of segregating sites from data set~B.
\begin{answer}
$S = 30$ per sequence or 0.75 per site
\end{answer}
\end{exercise}

\begin{exercise}
Estimate $\theta$ (per sequence and per site) from the value of $\pi$
you got from data set~A. 
\begin{answer}
$\hat\theta_{\pi} = 5.51$ per sequence or 0.1378 per site.
\end{answer}
\end{exercise}

\begin{exercise}
Estimate $\theta$ (per sequence and per site) from the value of $\pi$
you got from data set~B. 
\begin{answer}
$\hat\theta_{\pi} = 11.69$ per sequence or 0.29 per site.
\end{answer}
\end{exercise}

\begin{exercise}
Estimate $\theta$ from the value of $S$ you got from data set~A.
(Hint: Use table~\ref{tab.Ssum}.)
\begin{answer}
$\hat\theta_S = 15/2.83 = 5.3$ per sequence or $0.375/2.83 = 0.133$
  per site.  
\end{answer}
\end{exercise}

\begin{exercise}
Estimate $\theta$ from the value of $S$ you got from data set~B.
\begin{answer}
$\hat\theta_S = 30/2.83 = 10.6$ per sequence or $0.75/2.83 = 0.265$
  per site. 
\end{answer}
\end{exercise}

\begin{exercise}
  What can you infer from the similarity (or the dissimilarity) of the
  two estimates of $\theta$ you got from data set~A.  (Hint: Remember
  that it is only reasonable to compare values in the same
  units. Don't compare differences per site with differences per
  sequence.)
\begin{answer} In per-site units, we need to compare $\hat\theta_{\pi}
  = 0.1378$ and $\hat\theta_S = 0.133$.  These numbers don't differ too
  much, so there is no obvious reason to reject the model
  of neutral DNA in a randomly mating population of constant size.
\end{answer}
\end{exercise}

\begin{exercise}
  What can you infer from the similarity (or the dissimilarity) of the
  two estimates of $\theta$ you got from data set~B.
\begin{answer} In per-site units, we need to compare $\hat\theta_{\pi}
  = 0.29$ and $\hat\theta_S = 0.265$.  These numbers don't differ too
  much, so there is no obvious reason to reject the model
  of neutral DNA in a randomly mating population of constant size.
\end{answer}
\end{exercise}

\begin{exercise}
  Calculate the folded site frequency spectrum for data set~A.
\begin{answer}
Table~\ref{tab.Aanswer} presents data set~A again, with an extra row
at the top showing the minor allele counts:
\begin{table*}
\caption{Data set~A with counts of minor allele at top}
\label{tab.Aanswer}
\begin{verbatim}
       12  3 4   14      44 3      2   41   11  1 <-counts
seq01 AATATGGCAC CTCCCAACCC TCTAGCATAT ACCACTTACA
seq02 .......T.. .C......TG C......C.. ..........
seq03 ..C....... .......... .......... ..........
seq04 .......T.. .C......TG C......... G.........
seq05 .......... .......... .......... ..........
seq06 .....A.... ........T. C......... G....C....
seq07 ..C....T.. .C......TG C......... G.........
seq08 .....A.T.. TC......TG C......... G.........
seq09 .......... .......... C......... ..........
seq10 .G...A.... ........T. C......C.. .T....C..G
\end{verbatim}
\end{table*}
Tabulating the counts gives
\begin{center}
\begin{tabular}{cc}
Minor  & Number\\
allele & of\\
count  & sites\\
\hline
1 & 6\\
2 & 2\\
3 & 2\\
4 & 5
\end{tabular}
\end{center}
\end{answer}
\end{exercise}

\begin{exercise}
  Calculate the folded site frequency spectrum for data set~B.
\begin{answer}
Table~\ref{tab.Banswer} presents data set~B again, with an extra row
at the top showing the minor allele counts:
\begin{table*}
\caption{Data set~B with counts of minor allele at top}
\label{tab.Banswer}
\begin{verbatim}
      35 43  22  23112123 2 42 4311142 14 213  1 <-counts
seq01 TGCCACTCCA ATCTCTCGCC AGATGGCATG CCTTATCGCG
seq02 .......... G......... .A.C...GCA T.........
seq03 .......... G......A.. .A.C...GC. T....C....
seq04 C..TG..T.. .C.....A.. G......C.. TT.C......
seq05 CA.TG..T.. .C....TA.. G......CC. TT.C......
seq06 C...G..... CC..T..A.A ....AA.C.. TT..G.....
seq07 CA.TG..... CC..TC.A.A ...CA..CC. TT...C....
seq08 CA.TG...T. GCT....A.. G..C..TC.. T.......T.
seq09 CA.TG..... GC.C..T... ...CA..C.A T.........
seq10 CA.TG...T. .C.....A.. G..C...C.. T....C....
\end{verbatim}
\end{table*}
Tabulating the counts gives
\begin{center}
\begin{tabular}{cc}
Minor  & Number\\
allele & of\\
count  & sites\\
\hline
1 & 9\\
2 & 9\\
3 & 6\\
4 & 5\\
5 & 1
\end{tabular}
\end{center}
\end{answer}
\end{exercise}

\begin{exercise}
  Use the $\hat\theta_S$ for data set~A to calculate the theoretical
  folded spectrum for neutral loci in a population of constant
  size. (You will want the value of $\hat\theta_S$ \emph{per
    sequence}, not per site.) How well do the observed and theoretical
  spectra match?
\begin{answer}
  The expected numbers are $\hat\theta(1 + 1/9) = 5.89$ for singletons,
  $\hat\theta(1/2 + 1/8) = 3.31$ for doubletons,
  $\hat\theta(1/3 + 1/7) = 2.52$ for tripletons, 
  $\hat\theta(1/4 + 1/6) = 2.21$ for quadrupletons, and
  $\hat\theta\times 1/5 = 1.1$ for quintupletons.
\end{answer}
\end{exercise}

\begin{exercise}
  Use the $\hat\theta_S$ for data set~B to calculate the theoretical
  folded spectrum. (You will want the value of $\hat\theta_S$
  \emph{per sequence}, not per site.) How well do the observed and
  theoretical spectra match?
\begin{answer}
  The expected numbers are $\hat\theta(1 + 1/9) = 11.78$ for singletons,
  $\hat\theta(1/2 + 1/8) = 6.63$ for doubletons,
  $\hat\theta(1/3 + 1/7) = 5.05$ for tripletons, 
  $\hat\theta(1/4 + 1/6) = 4.41$ for quadrupletons, and
  $\hat\theta\times 1/5 = 2.12$ for quintupletons.
\end{answer}
\end{exercise}

\begin{exercise}
Sketch a gene genealogy for a sample of 10 sequences taken from
a population that grew suddenly from a small size to a very large
size 6 units of mutational time ago.
\begin{answer}
  The genealogy below has all the lineages coalescing at the same
  time, 6~units of mutational time ago. That is fine, but it would
  also be fine to draw a genealogy in which the coalescent events
  happened more gradually. Many of them should however be
  clustered in a relatively brief interval ending 6 units of time ago.\\
\begin{minipage}[t]{0.8\columnwidth}
\begin{verbatim}
-----------------------|
-----------------------|
-----------------------|
-----------------------|
-----------------------|---------
-----------------------|
-----------------------|
-----------------------|
-----------------------|
-----------------------|

|-------6 time units---|
\end{verbatim}
\end{minipage}
\end{answer}
\end{exercise}

\begin{exercise}
  Looking back from the present to the time of the population
  expansion, the number of new mutations on a single line of descent
  will be 3 on average, but not in every case. What probability
  distribution would best describe the variation in the number of
  mutations among independent lines of descent?
\begin{answer}
Poisson with mean 3.
\end{answer}
\end{exercise}

\begin{exercise}
  Sketch the distribution of pairwise differences (i.e.\ the mismatch
  distribution) that you would expect to see in this sample. Assume
  that the infinite sites model is a good approximation here. In other
  words, don't worry about multiple hits. Be sure to label the X axis
  in a way that indicates the mode of the
  distribution.
\begin{answer}
Accept anything vaguely wave-shaped with a peak near 6 differences.\\
\begin{minipage}[t]{0.8\columnwidth}
\begin{verbatim}
                       *
                  *        *
                  *            *
               *
       *   *                       *
   *
*                                      *

0  1   2   3   4   5   6   7   8   9  10
            Pairwise Differences
\end{verbatim}
\end{minipage}\\[0.5\baselineskip]
\end{answer}
\end{exercise}

\begin{exercise}
  In sequence data from the control region of human mitochondrial
  DNA, gene diversity $(\pi)$ is often around 0.01 for European and
  Asian samples and around 0.02 for African ones.  What does this
  imply about the numerical values of $\theta = 4Nu$ in Europe and
  Africa? 
\begin{answer}
 $\theta = 0.01$ for Europe and Asia; 0.0204 for Africa.
\end{answer}
\end{exercise}

The parameter $\theta$ can be estimated either from
\[
\hat\theta_{\pi} = \pi\qquad\hbox{or from}\qquad
\hat\theta_S = S \Big/ \sum_{k=1}^{K-1} 1/k,
\]
where $\pi$ is the mean pairwise difference per sequence, $S$ is the
number of segregating sites and $K$ is the number of gene copies in
the sample. In a population of constant size, these statistics are
often similar in size. In a population whose size has changed, they
can be dramatically different.

The reason involves the effect of population growth on the site
frequency spectrum. As explained in section~\ref{G-sec.popgrow}
of \citet{Rogers:LNG-13}, we see an excess of singleton sites in
populations that have expanded in size at some time since the last
common ancestor of the gene genealogy. To understand how population
growth affects $\hat\theta_{\pi}$ and $\hat\theta_S$, we need to think
about how these statistics are affected by singleton sites.

\begin{exercise}
  What does a singleton site contribute to the value of
  $\hat\theta_{\pi}$ in a sample of $K$ gene copies?
\begin{answer}
There are $K(K-1)/2$ pairs of gene copies. Of these pairs, the number
that differ is $1 \times (K-1) = K-1$. The contribution to $\pi$ is
the fraction of pairs that differ, $2/K$.
\end{answer}
\end{exercise}

\begin{exercise}
  What does a singleton site contribute to the value of $\hat\theta_S$
  in a sample of $K$ gene copies? 
\begin{answer}
  Each segregating site (whether a singleton or not) increments the
  value of $S$ by 1. Consequently, it increments $\hat\theta_S$ by 
$\left(\sum_{k=1}^{K-1} 1/k\right)^{-1}$.
\end{answer}
\end{exercise}

\begin{exercise}
  Make a table with three columns: the first for the number, $K$ of gene
  copies in the sample, the second for the effect of a singleton site on
  $\hat\theta_S$ and the third for the effect on
  $\hat\theta_{\pi}$. Consider $K$ values of 2, 3, 4, 10, and 100.
  For what values of $K$ does a singleton site have a larger effect on
  $\hat\theta_S$ than on $\hat\theta_{\pi}$?
\begin{answer}
In the table below, $\Delta \hat\theta_S$ represents the contribution
of a singleton site to $\hat\theta_S$ and $\Delta\hat\theta_{\pi}$ is
the corresponding contribution to $\hat\theta_{\pi}$.
\begin{center}
\begin{tabular}{rcc}
$K$ & $\Delta \hat\theta_S$ & $\Delta\hat\theta_{\pi}$\\ \hline
2   & 1.000        & 1.000\\
3   & 0.667        & 0.667\\
4   & 0.545        & 0.500\\
10  & 0.353        & 0.200\\
100 & 0.193        & 0.020\\
\end{tabular}
\end{center}
When $K>3$, each singleton site has a larger effect on $\Delta
\hat\theta_S$ than on $\Delta\hat\theta_{\pi}$.
\end{answer}
\end{exercise}

\begin{exercise}
  In a population that has expanded dramatically in size,
  it is usually true that $\hat\theta_S > \hat\theta_{\pi}$. Why?
\begin{answer}
  Many segregating sites will be singletons, which add more to
  $\hat\theta_S$ than to $\hat\theta_{\pi}$.
\end{answer}
\end{exercise}
