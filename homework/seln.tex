\chapter{Selection}
\label{hw.seln}

Many of the exercises in this homework have several components, which
are given letters: \emph{a}, \emph{b}, and so on. Each component of
each exercise is worth 10 points.  Several of the exercises ask you to
make graphs. Feel free to use any method you please. There is nothing
wrong with a pencil and graph paper. And if you don't have graph
paper, try this:
\begin{quote}
For emergency graph paper, take out one sheet of ruled paper, turn it
on its side, and place it beneath another sheet of ruled paper. If
these two sheets have a light-colored backing---often provided by the
rest of the pad or notebook---the vertical lines on the lower sheet
are almost certain to show through well enough, combining with the
horizontal lines on the top sheet to form a grid on which plotting is
reasonably easy.\\
\mbox{}\hfill \citet[p.~43]{Tukey:EDA-77}:
\end{quote}

\section{How selection changes allele frequencies}
\begin{exercise}
At a biallelic locus, suppose that genotypes $A_1A_1$, $A_1A_2$, and
$A_2A_2$ have relative fitnesses 1, 1.02, and 1.03 and that the
frequency of $A_1$ is $p=0.1$.
\begin{inparaenum}[(a)]
\item
What is the population's mean relative fitness?
\item
What are the ``marginal'' or allele-specific relative fitnesses of $A_1$ and
$A_2$?
\item
What is the expected frequency of $A_1$ in the following generation?
\end{inparaenum}
\begin{answer}
\begin{inparaenum}[(a)]
\item
The mean relative fitness is $\bar w = (p^2\times 1) + (2pq\times 1.02) +
(q^2\times 1.03) = 1.0279$.
\item
The marginal fitness of $A_1$ is $w_1 = p w_{11} + q w_{12} =
1.018$. That of $A_2$ is $w_2 = p w_{12} + q w_{22} = 1.029$.
\item
  In the following generation, the expected frequency of $A_1$ is $p'
  = (p^{2} w_{11} + pqw_{12})/\bar w$, as shown on Gillespie's
  p.~62. This gives 0.099.
\end{inparaenum}
\end{answer}
\end{exercise}

\begin{exercise}
\label{it.1gen}
At a biallelic locus, suppose that $p = 0.2$, $s = 0.05$, and $h=
0.1$. Here, $p$ is the relative frequency of allele $A_1$, and the
relative fitnesses of genotypes $A_1A_1$, $A_1A_2$, and $A_2A_2$ are
1, $1-hs$, and $1-s$. Assume that the population is at Hardy-Weinberg
equilibrium.
\begin{inparaenum}[(a)]
\item
What is the population's mean relative fitness?
\item
What are the ``marginal'' or allele-specific relative fitnesses of $A_1$ and
$A_2$?
\item
What is the expected frequency of $A_1$ in the following generation?
\end{inparaenum}
\begin{answer}
\begin{inparaenum}[(a)]
\item
The mean relative fitness is $\bar w = 1 - 2pqhs - q^2 s = 0.966$.
\item
The marginal fitness of $A_1$ is $w_1 = p + q(1-hs) = 0.996$. That of
$A_2$ is $w_2 = p(1-hs) + q(1-s) = 0.959$.
\item
In the following generation, the expected frequency $A_1$ is $p' =
(p^{2} w_{11} + pqw_{12})/\bar w$, as shown on Gillespie's p.~62. This
gives 0.2061.
\end{inparaenum}
\end{answer}
\end{exercise}

\begin{exercise}
  Repeat exercise~\ref{it.1gen}, assuming that $p = 0.5$, $s = 0.01$,
  and $h = -0.3$.
\begin{answer}
\begin{inparaenum}[(a)]
\item
The mean relative fitness is $\bar w = 1 - 2pqhs - q^2 s = 0.999$.
\item
The marginal fitness of $A_1$ is $w_1 = p + q(1-hs) = 1.0015$. That of
$A_2$ is $w_2 = p(1-hs) + q(1-s) = 0.9965$.
\item
  In the following generation, the expected frequency $A_1$ is $p' =
  (p^{2} w_{11} + pqw_{12})/\bar w$, as shown on Gillespie's
  p.~62. This gives 0.501.
\end{inparaenum}
\end{answer}
\end{exercise}

\begin{exercise} 
\label{it.A}
Suppose that genotypes $A_1A_1$, $A_1A_2$, and $A_2A_2$ have relative
fitnesses $1$, $1-hs$, and $1-s$, and that their absolute fitnesses
are 1.3, 1, and 0.9.
\begin{inparaenum}[(a)]
\item What are $s$ and $h$? 
\item Plot $\Delta_s p$ against $p$.
\item Where are the equilibria? Which are stable? Which are unstable?
\end{inparaenum}
Hints: use Gillespie's Eqns.~3.2 and~3.3. 
\begin{answer}
\begin{inparaenum}[(a)]
\item Gillespie makes his fitnesses relative to that of genotype
$A_1A_1$. To obtain these relative fitnesses, divide each absolute
fitness by $W_{11}$. This gives relative fitnesses $w_{11}=1$, $w_{12} =
0.769$, and $w_{22} = 0.692$. In Gillespie's fitness scheme, $w_{12} =
1-hs$, and $w_{22} = 1-s$. This implies that $s = 1 - w_{22} = 0.308$,
and $h = (1-w_{12})/s = 0.75$.

\item Gillespie's Eqns.~3.2--3.3 express $\Delta_s p$ in terms of $p$,
  $q=1-p$, $s$, $h$, and $\bar w = 1 - pqhs - q^2s$. Note that $q$ and
  $\bar w$ depend on $p$ and must therefore be recalculated for each
  value of $p$ that you graph. I did these calculations using a Python
  script. Here are a few of the results in tabular form:

{\centering\begin{tabular}{rr} $p$ & $\Delta_s p$\\ \hline
      0.0000 &   0.0000\\
      0.0417 &   0.0047\\
      0.0833 &   0.0095\\
      $\cdots$ & $\cdots$\\
      0.9167 &   0.0170\\
      0.9583 &   0.0090\\
      1.0000 &   0.0000\\
\end{tabular}\\}

The graph of these results looks like this:

{\centering\mbox{\beginpicture
\setcoordinatesystem units <4cm, 70cm>
\setplotarea x from 0 to 1, y from 0.0000 to 0.0455
\axis left label {$\Delta_s p$} 
  ticks numbered from 0.00 to 0.04 by 0.01 /
\axis bottom label {$p$} 
  ticks numbered from 0.00 to 1.00 by 0.25 /
%       p       dp
\plot
   0.0000   0.0000
   0.0417   0.0047
   0.0833   0.0095
   0.1250   0.0143
   0.1667   0.0189
   0.2083   0.0234
   0.2500   0.0277
   0.2917   0.0316
   0.3333   0.0352
   0.3750   0.0383
   0.4167   0.0409
   0.4583   0.0430
   0.5000   0.0445
   0.5417   0.0454
   0.5833   0.0455
   0.6250   0.0450
   0.6667   0.0436
   0.7083   0.0415
   0.7500   0.0385
   0.7917   0.0346
   0.8333   0.0297
   0.8750   0.0239
   0.9167   0.0170
   0.9583   0.0090
   1.0000   0.0000
/
\endpicture}\\}

\item
Equilibria occur where $\Delta_s p = 0$. As the graph shows, the only
such points are at $p=0$ and $p=1$. Only the second of these is
stable. This is because $\Delta_s p>0$ whenever $0<p<1$. Consequently,
$p$ always moves \emph{toward} the equilibrium at $p=1$ but \emph{away
  from} the one at $p=0$.
\end{inparaenum}
\end{answer}
\end{exercise}

\begin{exercise} 
\label{it.B}
Repeat exercise~\ref{it.A}, assuming that genotypes $A_1A_1$,
$A_1A_2$, and $A_2A_2$ have absolute fitnesses 1.3, 1, and 1.2.  
\begin{answer}
\begin{inparaenum}[(a)]
\item Gillespie makes his fitnesses relative to that of genotype
$A_1A_1$. To obtain these relative fitnesses, divide each absolute
fitness by $W_{11}$. This gives relative fitnesses $w_{11}=1$, $w_{12} =
0.769$, and $w_{22} = 0.923$. In Gillespie's fitness scheme, $w_{12} =
1-hs$, and $w_{22} = 1-s$. This implies that $s = 1 - w_{22} = 0.077$,
and $h = (1-w_{12})/s = 3$.

\item Gillespie's Eqns.~3.2--3.3 express $\Delta_s p$ in terms of $p$,
  $q=1-p$, $s$, $h$, and $\bar w = 1 - pqhs - q^2s$. Note that $q$ and
  $\bar w$ depend on $p$ and must therefore be recalculated for each
  value of $p$ that you graph. I did these calculations using a Python
  script. Here are a few of the results in tabular form:

{\centering\begin{tabular}{rr} $p$ & $\Delta_s p$\\ \hline
   0.0000  & --0.0000\\
   0.0417  & --0.0060\\
   0.0833  & --0.0101\\
  $\cdots$ & $\cdots$\\
   0.9167  & 0.0155\\
   0.9583  & 0.0087\\
   1.0000  & 0.0000\\
\end{tabular}\\}

The graph of these results looks like this:

{\centering\mbox{\beginpicture
\setcoordinatesystem units <4cm, 79.4cm>
\setplotarea x from 0 to 1, y from -0.0136 to 0.0265
\axis left label {$\Delta_s p$} 
  ticks numbered from -0.01 to 0.02 by 0.01 /
\axis bottom label {$p$} 
  ticks numbered from 0.00 to 1.00 by 0.25 /
\setdots
\putrule from 0 0 to 1 0
\setsolid
%       p       dp
\plot
   0.0000  -0.0000
   0.0417  -0.0060
   0.0833  -0.0101
   0.1250  -0.0126
   0.1667  -0.0136
   0.2083  -0.0133
   0.2500  -0.0118
   0.2917  -0.0094
   0.3333  -0.0062
   0.3750  -0.0025
   0.4167   0.0017
   0.4583   0.0061
   0.5000   0.0104
   0.5417   0.0146
   0.5833   0.0184
   0.6250   0.0217
   0.6667   0.0242
   0.7083   0.0259
   0.7500   0.0265
   0.7917   0.0259
   0.8333   0.0240
   0.8750   0.0205
   0.9167   0.0155
   0.9583   0.0087
   1.0000   0.0000
/
\endpicture}\\}

\item Equilibria occur where $\Delta_s p = 0$. There are three such
  points: at $p=0$, at $p=1$, and at $p=0.4$.  $\Delta_s p$ is
  positive to the right of the intermediate equilibrium but negative
  to the left. Consequently, the intermediate equilibrium is unstable
  and the two extreme equilibria are stable.
\end{inparaenum}
\end{answer}
\end{exercise}

\begin{exercise} 
\label{it.C}
Repeat exercise~\ref{it.A}, assuming that genotypes $A_1A_1$,
$A_1A_2$, and $A_2A_2$ have absolute fitnesses 0.9, 1, and 0.8.
\begin{answer}
\begin{inparaenum}[(a)]
\item Gillespie makes his fitnesses relative to that of genotype
$A_1A_1$. To obtain these relative fitnesses, divide each absolute
fitness by $W_{11}$. This gives relative fitnesses $w_{11}=1$, $w_{12} =
1.111$, and $w_{22} = 0.889$. In Gillespie's fitness scheme, $w_{12} =
1-hs$, and $w_{22} = 1-s$. This implies that $s = 1 - w_{22} = 0.111$,
and $h = (1-w_{12})/s = -1$.

\item Gillespie's Eqns.~3.2--3.3 express $\Delta_s p$ in terms of $p$,
  $q=1-p$, $s$, $h$, and $\bar w = 1 - pqhs - q^2s$. Note that $q$ and
  $\bar w$ depend on $p$ and must therefore be recalculated for each
  value of $p$ that you graph. I did these calculations using a Python
  script. Here are a few of the results in tabular form:

{\centering\begin{tabular}{rr} $p$ & $\Delta_s p$\\ \hline
   0.0000 & 0.0000\\
   0.0417 & 0.0092\\
   0.0833 & 0.0162\\
  $\cdots$ & $\cdots$\\
   0.9167 &--0.0063\\
   0.9583 &--0.0039\\
   1.0000 &--0.0000\\
\end{tabular}\\}

The graph of these results looks like this:

{\centering\mbox{\beginpicture
\setcoordinatesystem units <4cm, 85.6cm>
\setplotarea x from 0 to 1, y from -0.01 to 0.0272
\axis left label {$\Delta_s p$} 
  ticks numbered from -0.01 to 0.02 by 0.01 /
\axis bottom label {$p$} 
  ticks numbered from 0.00 to 1.00 by 0.25 /
\setdots
\putrule from 0 0 to 1 0
\setsolid
%       p       dp
\plot
   0.0000   0.0000
   0.0417   0.0092
   0.0833   0.0162
   0.1250   0.0213
   0.1667   0.0247
   0.2083   0.0266
   0.2500   0.0272
   0.2917   0.0267
   0.3333   0.0253
   0.3750   0.0232
   0.4167   0.0205
   0.4583   0.0173
   0.5000   0.0139
   0.5417   0.0103
   0.5833   0.0067
   0.6250   0.0032
   0.6667  -0.0000
   0.7083  -0.0028
   0.7500  -0.0051
   0.7917  -0.0068
   0.8333  -0.0076
   0.8750  -0.0075
   0.9167  -0.0063
   0.9583  -0.0039
   1.0000  -0.0000
/
\endpicture}\\}

\item
Equilibria occur where $\Delta_s p = 0$. There are three such points:
at $p=0$, at $p=1$, and at $p=2/3$.
$\Delta_s p$ is positive to the left of the intermediate equilibrium
but negative to the right. Consequently, the intermediate equilibrium
is stable and the two extreme equilibria are unstable.
\end{inparaenum}
\end{answer}
\end{exercise}

\begin{exercise} 
  Suppose that genotypes $A_1A_1$, $A_1A_2$, and $A_2A_2$ have
  fitnesses 1, 1.15, and 1.2.  The population is infinite and mates at
  random. What is the frequency of allele~$A_1$ at the stable
  equilibrium? (No calculation is needed.)
\begin{answer}
  No calculation is required. Because $A_2A_2$ has higher fitness than
  $A_1A_1$ and the heterozygote is intermediate, $A_2$ will go to
  fixation and the stable equilibrium frequency of $A_1$ is 0.
\end{answer}
\end{exercise}

\begin{exercise} 
  Suppose that genotypes $A_1A_1$, $A_1A_2$, and $A_2A_2$ have
  fitnesses 1, 1.5, and 1.  The population is infinite and mates at
  random. What is the frequency of allele~$A_1$ at the stable
  equilibrium?  (No calculation is needed.)
\begin{answer}
  No calculation is required. Because the two homozygotes have equal
  fitness and the heterozygote is superior, the stable equilibrium
  frequency of $A_1$ is $1/2$.
\end{answer}
\end{exercise}

\begin{exercise} 
  Suppose that genotypes $A_1A_1$, $A_1A_2$, and $A_2A_2$ have
  fitnesses 1, 0.5, and 1.5.  The population is infinite and mates at
  random. What is the frequency of allele~$A_1$ at the two stable
  equilibria?  (No calculation is needed.)
\begin{answer}
  No calculation is required. Because the heterozygote has lower
  fitness than either homozygote, there are two stable
  equilibria---one at which the frequency of $A_1$ is~0 and another at
  which it is~1.
\end{answer}
\end{exercise}

\begin{exercise} 
  Use Gillespie's Eqn.~3.4 to graph $\hat p$ as a function of $h$ for
  $-1 < h < 2$. Your vertical axis should extend only from~0 to~1,
  because values outside this range are not legitimate allele
  frequencies. Locate the regions of the graph that correspond to
  (a)~overdominance, (b)~incomplete dominance, and
  (c)~underdominance. In making these determinations, assume that
  $s>0$.  For each of these regions, discuss the outcome of natural
  selection.
\begin{answer}
$\hat p$, the equilibrium value of $p$, is equal to $(h-1)/(2h-1)$
(see Gillespie's Eqn.~3.4). Here are a few values and then the graph:

{\centering\begin{tabular}{rr} $h$ & $\hat p$\\ \hline
 --1.0000 &  0.6667\\
 --0.8421 &  0.6863\\
 --0.6842 &  0.7111\\
  $\cdots$ & $\cdots$\\
  1.6842 &  0.2889\\
  1.8421 &  0.3137\\
  2.0000 &  0.3333\\
\end{tabular}\\}

{\centering\mbox{\beginpicture
\setcoordinatesystem units <1.33cm, 4cm>
\setplotarea x from -1 to 2, y from 0 to 1
\axis left label {$\hat p$} 
  ticks numbered from 0.00 to 1.00 by 0.25 /
\axis bottom label {$h$} 
  ticks numbered from -1.0 to 2.0 by 1 /
\setdots
\putrule from 0 0 to 1 0
\setsolid
%       h       phat
\multiput {$\circ$} at
 -1.0000   0.6667
 -0.8421   0.6863
 -0.6842   0.7111
 -0.5263   0.7436
 -0.3684   0.7879
 -0.2105   0.8519
 -0.0526   0.9524
%  0.1053   1.1333
%  0.2632   1.5556
%  0.4211   3.6667
%  0.5789  -2.6667
%  0.7368  -0.5556
%  0.8947  -0.1333
  1.0526   0.0476
  1.2105   0.1481
  1.3684   0.2121
  1.5263   0.2564
  1.6842   0.2889
  1.8421   0.3137
  2.0000   0.3333
/
\endpicture}\\}

(a)~Overdominance occurs when $h<0$ and generates a stable equilibrium
within the interval $(0,1)$. (b)~Incomplete dominance occurs when
$0<h<1$. In this case, selection is directional, and there are no
internal equilibria. (c)~Underdominance occurs when $h>1$ and
generates an unstable internal equilibrium. The only stable
equilibria are at $p=0$ and $p=1$.
\end{answer}
\end{exercise}

\begin{exercise} 
Gillespie's equation~3.9 doesn't work if $h=0$, i.e.\ if allele $A_2$
is completely recessive. Derive a formula for this case. Hints:
\begin{inparaenum}[\it (i)]
\item equation~3.6 still works;
\item equation~3.8 simplifies to $\Delta_s p = pq^2s/(1 - q^2 s)
  \approx pq^2s$;
\item it is still true that $\Delta p = \Delta_s p + \Delta_u p$.
\end{inparaenum}
\begin{answer}
The change per generation is $\Delta p \approx pq^2s - up$. At
equilibrium between mutation and drift, this quantity equals
zero. This implies that the equilibrium occurs at $\hat q =
\sqrt{u/s}$.
\end{answer}
\end{exercise}

\begin{exercise}
\label{it.flowers1}
White flowers (genotype $rr$) are recessive to red ($RR$ and $Rr$) in
an outbreeding plant species. In a large random sample, you count 200
white-flowered plants and 800 red-flowered plants.  One generation
later, you count 250 white and 750 red plants. To study these data,
use the following notation for fitnesses and genotypic frequencies:
\begin{center}
\begin{tabular}{lccc}
G'type  & $RR$ & $Rr$ & $rr$\\
Fitness & $w_{RR} = 1$   & $w_{Rr} = 1$   & $w_{rr} = 1-s$\\
Freq    & $P_{RR} = p^2$ & $P_{Rr} = 2pq$ & $P_{rr} = q^2$\\
\end{tabular}
\end{center}
This says that $p$ is the frequency of allele $R$, $q=1-p$ is that of
allele $r$, and the genotype frequencies are at Hardy-Weinberg
equilibrium. Furthermore, relative fitness is 1 for both of the
genotypes ($RR$ and $Rr$) that produce red flowers. It equals $1-s$
for the genotype ($rr$) that produces white flowers. To study the
change between generations, we'll use the displayed equation at the
top of Gillespie's p.~62. In the current notation, this equation
becomes
\begin{equation}
p' = \frac{p^2 w_{RR} + pq w_{Rr}}{\bar w}
\label{eq.pprime}
\end{equation}
where $\bar w = p^2 w_{RR} + 2pq w_{Rr} + q^2 w_{rr}$ is the mean
allele frequency. Answer the following questions:
\begin{inparaenum}[\it (a)]
\item What's the frequency, $p$, of allele $R$ in the first
  generation?  
\item What's the frequency, $p'$, of allele $R$ in the second
  generation?  
\item If this change was caused by selection, then what was the
  coefficient $(s)$ of selection?
\end{inparaenum}
This calculation is sensitive to numerical error: use at least 3
digits of precision throughout.
\begin{answer}
\begin{inparaenum}[\it (a)]
\item In the first generation, the frequency of white flowers is
  $P_{rr} = 200/1000 = 0.2$. At Hardy-Weinberg equilibrium, this
  frequency is $q^2$, where $q$ is the frequency of allele $r$. This
  implies that $q = \sqrt{0.2} = 0.447$. The frequency of allele $R$
  is therefore $p = 1-q = 0.553$.
\item In the 2nd generation, $P_{rr} = 250/1000 = 0.25$, so $q' =
  \sqrt{0.25} = 0.5$, and $p'=1-q'=0.5$.
\item Because $w_{RR} = w_{Rr} = 1$, Eqn.~\ref{eq.pprime} simplifies
  to $p' = p/(1-q^2 s)$, or $s = (1-p/p')/q^2$. With our data, this is 
  $s = (1-0.553/0.5)/0.2 = -0.53$.
\end{inparaenum}
\end{answer}
\end{exercise}

\begin{exercise}
\label{it.flowers2}
Repeat exercise~\ref{it.flowers1}, this time assuming that the
fraction of white flowers was $100/1000$ in the first generation and
$95/1000$ in the second.

This calculation is sensitive to numerical error: use at least 3
digits of precision throughout.
\begin{answer}
\begin{inparaenum}[\it (a)]
\item In the first generation, the frequency of white flowers is
  $P_{rr} = 100/1000 = 0.1$. At Hardy-Weinberg equilibrium, this
  frequency is $q^2$, where $q$ is the frequency of allele $r$. This
  implies that $q = \sqrt{0.1} = 0.316$. The frequency of allele $R$
  is therefore $p = 1-q = 0.684$.
\item In the 2nd generation, $P_{rr} = 95/1000 = 0.095$, so $q' =
  \sqrt{0.095} = 0.308$, and $p'=1-q'=0.692$.
\item Because $w_{RR} = w_{Rr} = 1$, Eqn.~\ref{eq.pprime} simplifies
  to $p' = p/(1-q^2 s)$, or $s = (1-p/p')/q^2$. With our data, this is 
  $s = (1-0.684/0.692)/0.1 = 0.116$.
\end{inparaenum}
\end{answer}
\end{exercise}

\section{Selection and drift}
In these problems, we'll follow the notation in Gillespie's
section~3.9, which assumes that $A_1$ is the ``wild type'' allele,
$A_2$ is the mutant allele, and genotypes $A_1A_1$, $A_1A_2$, and
$A_2A_2$ have relative fitnesses 1, $1+s/2$, and $1+s$. (This notation
differs from Kimura's, which Jon presented in lecture.)
\begin{exercise}
Do we expect more adaptive evolution (fixations of advantageous
mutations) in large or in small populations?  Why?
\begin{answer}
Because the number of new advantageous mutants is proportional to
population size, but the fixation probability for such mutants is
essentially independent of population size. The rate of adaptive
evolution is thus proportional to population size.

Some students may want to answer this question more formally, so here
is the formal version: let $u$ represent the mutation rate for
advantageous alleles whose fitness in heterozygotes is $1+s/2$,
relative to that of the wild-type allele. The number of such mutations
in the population as a whole is $2Nu$, and each of them has
probability $s$ of ultimate fixation. For this category of adaptive
mutations, the expected number fixed per generation is therefore
$2Nus$, an increasing function of population size.
\end{answer}
\end{exercise}

\begin{exercise}
\label{ex.fixbad}
For a deleterious mutation with $s = -0.001$, what is the probability
of fixation in a population of size
\begin{inparaenum}[(a)]
\item $N_e = 10,000$,
\item $N_e = 1000$, and
\item $N_e = 100$.
\end{inparaenum}
Hint: In Python, Gillespie's equation~3.22 is
\begin{verbatim}
# Prob of fixation of a new mutation.
def pfix(N, s):
    return (1-exp(-s))/(1-exp(-2*N*s))
\end{verbatim}
\begin{answer}
With $s=-0.001$, the fixation probability is
\begin{inparaenum}[(a)]
\item $2.1\times 10^{-12}$ if $N=10000$;  or
\item $1.57\times 10^{-4}$ if $N=1000$; or
\item $4.5\times 10^{-3}$ if $N=100$.
\end{inparaenum}
\end{answer}
\end{exercise}

\begin{exercise}
\label{ex.ratebad}
Suppose that for the deleterious alleles of the preceding problem, the
mutation rate is $u = 2.2\times 10^{-9}$ per year. What is the rate of
substitution of such alleles per million years?
\begin{answer}
The substitution rate per million years is $10^6 \times 2Nu $ times the
answers from the preceding question.
\begin{inparaenum}[(a)]
\item $9.07\times 10^{-11}$ if $N=10000$;  or
\item $6.89\times 10^{-4}$ if $N=1000$; or
\item $1.99\times 10^{-3}$ if $N=100$.
\end{inparaenum}
\end{answer}
\end{exercise}

\begin{exercise}
\label{ex.fixgood}
Problem~\ref{ex.fixbad} was about deleterious mutations. Repeat it for
advantageous mutations with $s=0.001$. (Hint: Use the \verb|pfix|
Python function defined above.)
\begin{answer}
With $s=0.001$, the fixation probability is
\begin{inparaenum}[(a)]
\item $1\times 10^{-3}$ if $N=10000$;  or
\item $1.16\times 10^{-3}$ if $N=1000$; or
\item $5.51\times 10^{-3}$ if $N=100$.
\end{inparaenum}
\end{answer}
\end{exercise}

\begin{exercise}
\label{ex.rategood}
Repeat problem~\ref{ex.ratebad} for advantageous mutations with
$s=0.001$, still assuming that $u = 2.2\times 10^{-9}$. 
\begin{answer}
  The substitution rate per million years is $10^6 \times 2Nu $ times
  the results from the \verb|pfix| function (Gillespie's Eqn.~3.22)
  defined above.
\begin{inparaenum}[(a)]
\item 0.044 if $N=10000$;  or
\item 0.0051 if $N=1000$; or
\item 0.0024 if $N=100$.
\end{inparaenum}
\end{answer}
\end{exercise}

On p.~93, Gillespie shows that if $s\ll 1 \ll 2N_e s$, and $N_e
\approx N$, then the probability of fixation of a newly-arisen
advantageous mutant allele is approximately 
\begin{equation}
\pi_1(1/2N) \approx s
\label{eq.fixmut-approx}
\end{equation}
In other words, the probability of fixation for a new adaptive
mutation is approximately twice the selective advantage of a
heterozygote. 

\begin{exercise}
Repeat exercise~\ref{ex.fixgood}, this time using the approximation in
Eqn.~\ref{eq.fixmut-approx}. For each population size, calculate the
relative error of the approximation,
\[
\hbox{\rm relerr} = \left|\frac{s - \pi_1(1/2N)}{\pi_1(1/2N)}\right|
\]
where $\pi_1(1/2N)$ is the value given by Kimura's formula, and the
vertical bars indicate the absolute value.\footnote{For example, the
  absolute value of $-3$ is written $|{-}3|$ and equals 3.} Based on
these relative errors, in which cases does the approximation work
well, and in which does it work poorly?
\begin{answer}
\begin{inparaenum}[(a)]
\item For $N=10000$: $\pi_1=0.0009995$, $s=0.001$,
  $\mbox{relerr}=0.000500$;  
\item for $N= 1000$: $\pi_1=0.0011559$, $s=0.001$,
  $\mbox{relerr}=0.134903$;  
\item for $N=  100$: $\pi_1=0.0055139$, $s=0.001$,
  $\mbox{relerr}=0.818640$.    
\end{inparaenum}
The approximation works well for $N=10000$, fairly well for $N=1000$,
but poorly for $N=100$.
\end{answer}
\end{exercise}




