\chapter*{Introduction}
\label{ch.intro}
This document contains homework assignments for \emph{Human
  Evolutionary Genetics} (Anth/Biol~5221, at the University of
Utah). The due dates of the assignments are given on the course
syllabus. In some cases, more than one assignment may be due on the
same day. 

\ifthenelse{\theshowodd=1}{%
  \ifthenelse{\theshoweven=1}{%
This version of the document is the key (provided only to teaching
assistants) and contains answers to all problems, beginning on
p.~\pageref{ch.answers}.}{%
Answers to odd-numbered problems are at the back of this document,
beginning on p.~\pageref{ch.answers}.}}{%
  \ifthenelse{\theshoweven=1}{%
Answers to even-numbered problems are at the back of this document,
beginning on p.~\pageref{ch.answers}.}{}
}

\section*{Notation for logarithms}
Logarithms appear in various places in the following homework
assignments. Usually, we are interested in the ``natural''
logarithm---the logarithm to base $e$, where $e \approx 2.72$.  Our
notation is as follows:
\begin{center}
\begin{tabular}{rl}
Base & Notation\\
\hline
$e$  & $\log_e x$ or $\ln x$ or $\log x$\\
10   &$\log_{10} x$\\
 2   &$\log_{2} x$
\end{tabular}
\end{center}
Note that for us, $\log x$ means $\log_{e} x$. Some disciplines and
some calculators use a different convention, in which $\log x$ means
$\log_{10} x$, so be sure to understand what your own calculator does.

Our convention is also used within the Python programming
language. For example,
\begin{verbatim}
>>> from math import *
>>> log(e**3)
3.0
>>> log10(10**2)
2.0
\end{verbatim}
In this example, Python's \texttt{log} function inverts (reverses the
effect of) raising $e$ to a power. This implies that Python's
\texttt{log(x)} means $\log_e x$.  Similarly, Python's \texttt{log10}
function inverts the operation of raising 10 to a power, so
\texttt{log10(x)} means $\log_{10} x$.
