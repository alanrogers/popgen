\chapter{Admixture}
\label{hw.admix}

\begin{exercise}
The table below summarizes a comparison among four haploid genomes:
one European, one African, one Neanderthal, and one chimpanzee. ``0''
represents the ancestral allele and ``1'' the derived (i.e.\ mutant)
allele. Each column refers to a different pattern in which these
alleles may occur at an individual nucleotide site. The bottom row
shows the number of nucleotide sites that have each pattern.
\begin{center}
  \em
\begin{tabular}{l@{\hspace{0.4em}}c@{\hspace{0.4em}}c@{\hspace{0.4em}}c}
&\multicolumn{3}{c}{Nucleotide}\\
&\multicolumn{3}{c}{Site Pattern}\\
 &\emph{xy}&\emph{yn}&\emph{xn}\\
\cline{2-4}
African (X)& 1 & 0 & 1\\
European (Y) & 1 & 1 & 0\\
Neanderthal (N) & 0 & 1 & 1\\
Chimpanzee (C) & 0 & 0 & 0\\
\hline
  &303,340 & 103,612 & 95,347
\end{tabular}
\end{center}
Explain why, in the absence of gene flow from the Neanderthal
population into modern humans, we expect site patterns $yn$ and $xn$
to be approximately equal in frequency.
\begin{answer}
In the absence of gene flow, patterns $yn$ and $xn$ arise only
when the European, African, and Neanderthal lineages all coalesce
within the ancestral population---a process called ``incomplete
lineage sorting.'' In that case, the three lineages
are equally likely to coalesce in any order, so all three site
patterns are equally likely.
\end{answer}
\end{exercise}

