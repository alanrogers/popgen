\chapter{Neutral Theory}
\label{hw.neutraltheo}

\noindent
Each exercise is worth 10 points.

To estimate separation times from genetic data, we rely on the
``molecular clock.'' If, at some locus, two species differ by 0.01
(presumably neutral) nucleotide substitutions per site, and if neutral
substitutions occur at a rate of $\rho$ per generation, then we
estimate the separation time as $0.01/2\rho$. 

\begin{exercise}
Why is there a ``2'' in the denominator of this expression?
\begin{answer}
Because mutations may occur on either of the two paths separating the
two species from their common ancestor.
\end{answer}
\end{exercise}

To make such estimates, we must first estimate the rate, $\rho$, of
neutral evolution.  Such rates are estimated from the number of
nucleotide substitutions that accumulate between DNA sequences
separated for a known amount of time. Unfortunately, neither the
number of substitutions nor the separation time is often known
exactly, and this can add both uncertainty and bias to the molecular
clock.

In the calculation just described, the number of nucleotide
substitutions cannot be observed directly. What we do observe is the
number of nucleotide differences. The problem is that when a
substitution occurs at a nucleotide site that has previously mutated,
it does not increase the number of site differences.  The uncorrected
number of site differences therefore underestimates the number of
substitutions, a phenomenon known as \emph{saturation}.  This effect
is insignificant in very recent comparisons but increases with age. It
is exacerbated when nucleotide sites vary in rate, because rates at
rapidly-evolving sites may be underestimated.  In modern phylogenetic
studies, these problems are addressed by fitting models of the
substitution process \citep[sec.~3.2--3.4]{Nei:MEP-00}. If the model
is appropriate, saturation adds noise but not bias to the molecular
clock. Fortunately, estimates of dates are relatively insensitive to
this component of the model: we get approximately the same answer from
many different models \citep[p.~143]{Yang:CME-06}.

\section{The Jukes-Cantor model of nucleotide substitution}
Neutral theory predicts that substitutional changes will accumulate at
a constant rate. Yet because of saturation, we cannot measure
substitutional changes directly. What we do measure is the fraction,
$p$, of sites that differ between each pair of sequences. We need a
way to convert this into an estimate of the mean number, $K$, of
substitutional changes per site. This requires some model of the
substitutional process. A variety of such models have been introduced,
but we will use only the simplest, which was introduced by 
\citet{Jukes:MPM-69-21} in 1969.

Their model assumes that when a site mutates, it is equally likely to
end up in any of the other three states. Under this assumption, 
\citet{Jukes:MPM-69-21} show that
\begin{equation}
K = -\frac{3}{4} \ln \left(1 - \frac{4}{3}\,p\right)
\label{eq.jukescantor}
\end{equation}
This formula allows us to estimate $K$ (the number of subsitutional
changes per site) from $p$ (the fraction of sites that differ).  If
the neutral theory is correct, then $E[K]=2ut$, where $u$ is the
neutral mutation rate, and $t$ is the time since the last common
ancestor of the two DNA sequences.

For example, consider the difference between humans and chimpanzees.
We differ at about $35\times 10^6$ of the $3\times 10^9$ nucleotide
sites in our (haploid) genomes---a fraction $p = 0.01167$. Plugging
this into Eqn.~\ref{eq.jukescantor} gives $K=0.01177$ substitutions per
site. There is hardly any difference between our estimates of $K$ and
$p$, indicating that not much saturation has occurred. In cases such
as this, there is no compelling reason to correct for saturation at
all.

When $p$ is small, geneticists often use it as an approximation for
$K$. This makes sense, provided that the error involved in the
approximation is small. How small is it in the chimpanzee-human
comparison? Consider the relative error, which is defined as
\[
\hbox{relerr} = \frac{|p-K|}{K}
\]
where $||$ indicates the absolute value. For the chimpanzee--human
data, this gives $\hbox{relerr} = |0.01167 - 0.01177|/0.01177 \approx
0.008$. This means that when we use $p$ instead of $K$, we make an
error that is less than 1\% as large as the correct answer, $K$.

\begin{exercise}
Calculate the relative error for a variety of values of $p$ in order
to decide when it is and is not necessary to correct for saturation.
\begin{answer}
The values below indicate that relative error is small when $p$ is
less than about 0.1.\\
{\centering
\begin{tabular}{cc}
 $p$ & relerr\\ \hline
0.30 & 0.22\\
0.20 & 0.14\\
0.10 & 0.07\\
0.08 & 0.05\\
0.06 & 0.04\\
0.05 & 0.03\\
\hline
\end{tabular}\\}
\end{answer}
\end{exercise}

\begin{exercise}
Fossils suggest that humans and chimpanzees last shared
ancestor around $t = 6\times 10^{6}$ years ago. The neutral theory
implies that $K = 2ut$, where $u$ is the neutral substitution rate. 
Use the chimpanzee-human data to estimate $u$. How does the estimate
compare to the typical mammalian rate, approximately $10^{-8}$
per site per year? What might account for the difference?
\begin{answer}
For the chimpanzee-human data,
\[
u = K/2t = 0.01177/(12\times 10^{6}) = 0.98\times 10^{-9}
\]
That's a bit lower than the rates estimated for typical mammals.  But
we hominins have had longer generation times. Suppose our ancestors
had 20-year generations. Then the rate per generation would be
\[
u = 20 \times 0.98\times 10^{-9} \approx 2 \times 10^{-8}
\]
which is in good aggrement with other mammals.
\end{answer}
\end{exercise}

\subsection{Deriving the Jukes-Cantor formula}
\label{sec.jcderive}

Although we are interested in the process of mutation, we will focus
here on an imaginary process of ``perturbation.''  In each time unit,
a site is perturbed with probability $\lambda$. After a perturbation
the site is equally likely to be in any of the four states (A, T, G,
and C). If it is perturbed, it ends up in a different state with
probability 3/4. Consequently, mutations occur with probability $u =
(3/4)\lambda$.

This formulation of the problem provides an enormous simplification:
once a site has been perturbed, there is no further change in the
probabilities of states A, T, G, and C. These probabilities are the
same, no matter whether the site has been perturbed once, twice, or a
thousand times. This makes the process far easier to study.  Yet once
we have derived a result, we can easily re-express it in terms of
mutations rather than perturbations. All we'll need to do is replace
$\lambda$ in our formula with the equivalent value $(4/3)u$.

Suppose that we compare homologous sites in two DNA sequences that
have been separated for $t$ units of time. Because the rate of
perturbation is constant, the number of perturbations along the the
two branches that separate our DNA sequences is a Poisson random
variable with mean $2\lambda t$. The probability that neither sequence
has been perturbed is given by the zero term of the Poisson
distribution: $e^{-2\lambda t}$. In this case, they are obviously
identical. If there has been at least one perturbation, then the
perturbed sequences are equally likely to be in any of the four
states, and the two sites are identical with probability 1/4. The
probability, $q$, of identity is therefore
\begin{eqnarray*}
q &=& e^{-2\lambda t} + (1 - e^{-2\lambda t})/4\\
&=& \frac{1}{4} + \frac{3}{4}\,e^{-2\lambda t}
\end{eqnarray*}

Let us now re-express this result in terms of things we can
measure. First substitute $(4/3)u = \lambda$, so that our formula is
in terms of mutations rather than perturbations:
\[
q = \frac{1}{4} + \frac{3}{4}e^{-(8/3)u t}
\]
Next, re-express the formula in terms of the probability, $p = 1-q$,
that two homologous sites will differ:
\[
p = \frac{3}{4} - \frac{3}{4}\,e^{-(8/3)u t}
\]
Finally, substitute $K = 2ut$, the expected number of nucleotide site changes
along the path that separates the two sites:
\[
p = \frac{3}{4} - \frac{3}{4}\,e^{-(4/3)K}
\]
After solving this equation for $K$, we end up with
Eqn.~\ref{eq.jukescantor}.

\section{Other models of substitution}

Some nucleotide site changes (A$\leftrightarrow$G and
C$\leftrightarrow$T) are especially common.  These common changes are
called ``transitions,'' and the others are called ``transversions.''
In mitochondrial DNA, transitions seem to be at least 30 times as
common as transvertions. With such data, the Jukes-Cantor model is
clearly inappropriate, and various alternatives are often used.

One alternative is especially simple. If transitions are sufficiently
common, relative to transversions, it may be reasonable to assume that
\emph{all} mutations are transitions. In such a model, some sites
toggle back and forth between A and T, while others toggle between C
and G. Either way, we can use a variant of the Jukes-Cantor model in
which the number of states is two rather than four.

\begin{exercise}
Derive a formula, analogous to Eqn.~\ref{eq.jukescantor}, in which the
number of states is two rather than four. (Hint: repeat the steps in
section~\ref{sec.jcderive}. Your derivation should assume that, after
a perturbation, the site is equally likely to be in either of the
\emph{two} states.) 
\begin{answer}
As with the Jukes-Cantor model, we begin with
\begin{eqnarray*}
q &=& e^{-2\lambda t} + (1 - e^{-2\lambda t})/2\\
&=& \frac{1}{2} + \frac{1}{2}\,e^{-2\lambda t}
\end{eqnarray*}
When at least one perturbation has occurred, the sites are identical
with probability 1/2. This accounts for the value ``1/2'' that appears
above. Now substitute $2u=\lambda$ (because only half of perturbations
are mutations), $p=1-q$, and $K = 2ut$:
\[
p = \frac{1}{2} - \frac{1}{2}\,e^{-2K}
\]
Solving for $K$ gives
\[
K = -\frac{1}{2}\,\log_e(1-2p)
\]
which is analogous to Eqn~\ref{eq.jukescantor}. This was introduced in
1919 and is called ``Haldane's mapping function''
\cite{Haldane:Distance}. It has often been used to make linkage maps
of chromosomes.
\end{answer}
\end{exercise}

\begin{exercise}
Use the 2-state model to estimate $K$ from the chimpanzee-human data.
\begin{answer}
For the chimpanzee-human comparison, $p = 0.01167$. Plugging this into
the 2-state formula gives $K=0.01180$.
\end{answer}
\end{exercise}

\begin{table}
\caption{Nucleotide sequence differences between complete
  mitochondrial genomes. Source: \citep{Gibbs:N-428-493}.}
\label{tab.mtdiffs}

\begin{tabular}{lrrrrr}
                 & BN Rat& Mouse&Human\\
\hline     
Brown Norway Rat &  ---&4897  & 2897\\
Mouse            &     & ---  & 5050\\
Human            &     &      &  ---\\
\hline
\end{tabular}
\end{table}

\section{Rodent mitochondrial DNA sequences}

Table~\ref{tab.mtdiffs} shows the number of nucleotide site
differences between the complete mitochondrial genomes of mouse, rat,
and humans. These genomes are of slightly different sizes: about
16,300~bp for mouse and rat, and 16,569~bp for human. 

\begin{exercise}
Use the data in table~\ref{tab.mtdiffs} to estimate the fraction $(p)$
of site differences between each pair of species. To do this you will
need to know the total number of sites in the mitochondrial genome.
These numbers are about 16,300~bp for mouse and rat, and 16,569~bp for
human.  We can take the \emph{smaller} of these two numbers as the
effective size in our comparisons, because this is the largest number
of sites we could conceivably align. Then 
\begin{inparaenum}[(a)]
\item
use the Jukes-Cantor formula to estimate $K$ for each pair of
species. 
\item Because rats and mice are close relatives, we expect the
  rat-human number to equal the mouse-human number. How well do your
  results conform to this expectation?
\end{inparaenum}
\begin{answer}
(a)~For the three species, Jukes-Cantor yields the following estimates
  of $K$:\\
\begin{tabular}{lrrrrr}
                 &\multicolumn{3}{c}{$K$ estimates}\\
                 & BN Rat& Mouse&Human\\
\hline     
Brown Norway Rat &  ---&0.38  & 0.20\\
Mouse            &     & ---  & 0.40\\
Human            &     &      &  ---\\
\hline
\end{tabular}

(b)~The mouse-human distance is \emph{much} larger than the rat-human
distance. (We hope students will contemplate the causes of this
difference, but grades will not be based on this issue.)
\end{answer}
\end{exercise}


%\begin{table}
%\caption{Nucleotide sequence differences between complete
%  mitochondrial genomes. Differences over the whole genome are shown
%  above the diagonal; those between 16S rRNA sequences are shown in
%  parentheses below. In mammals, the mitochondrial genome has
%  15--17 thousand nucleotides. In humans, it has 16,569~bp.
%rat:16,313~bp 
%mouse:16,300~bp
%source: \citep{Gibbs:N-428-493}
%}
%\label{tab.mtdiffs}
%
%\centering
%\begin{tabular}{lrrrrr}
%            &RGSP & Wild&Wistar& Mouse& Human\\
%            & Rat &  Rat&   Rat&      &      \\
%\hline     
%RGSP BN Rat &  ---&   95&   357&4891  &  2890\\
%Wild BN Rat &  (7)&  ---&   359&4897  &  2897\\
%Wistar Rat  &(144)&(140)&   ---&4988  &  2990\\
%Mouse       &(385)&(380)& (455)& ---  &  5050\\
%Human       &(242)&(240)& (310)&(421) &  ---\\
%\end{tabular}
%\end{table}
