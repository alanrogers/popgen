\chapter{Random mating}
\label{hw.randommate}

\section{Frequencies of alleles and genotypes}

\emph{Transferrin} is a protein involved in iron transport. Table
\ref{tab.transferrin} shows the number of individuals in a baboon
troop, grouped by transferrin genotype.  The relative frequency of
each genotype is simply the number of individuals of that genotype
divided by the total number of individuals.

\begin{table}
\centering
\caption{Transferrin genotype frequencies in a baboon troop
  \citep[p.~56]{Jolly:PAA-86}.}
\label{tab.transferrin}
\begin{tabular}{rrrr}
\hline
        &\multicolumn{3}{c}{Number of}\\ \cline{2-4}
Genotype& individuals &allele $C$&allele $D$\\ \hline\hline
$CC$    &          80 &      160&       0\\
$CD$    &          15 &       15&      15\\
$DD$    &           5 &        0&      10\\ \hline
Total   &         100 &      175&      25\\ \hline
\end{tabular}
\end{table}

\begin{exercise}
What is the relative frequency, $P_{CC}$, of the $CC$ genotype?
\begin{answer}
$P_{CC} = 80/100 = 0.8$
\end{answer}
\end{exercise}

\begin{exercise}
What is the relative frequency of the $CD$ genotype?
\begin{answer}
$P_{CD} = 15/100 = 0.15$
\end{answer}
\end{exercise}

\begin{exercise}
What is the relative frequency of the $DD$ genotype?
\begin{answer}
$P_{DD} = 5/100 = 0.05$
\end{answer}
\end{exercise}

There are two ways to calculate the allele frequency within a
sample:
\begin{enumerate}
\item Divide the number of copies of one allele by the total number of
  gene copies in the sample.
\item If the locus has just two alleles, we can use the formula 
\begin{equation}
p_1 = P_{11} + P_{12}/2
\label{eq.allelefrq}
\end{equation}
where $P_{11}$ and $P_{12}$ are the frequencies of genotypes $A_1A_1$
and $A_1A_2$, and $p_1$ is the frequency of allele $A_1$.
\end{enumerate}

\begin{exercise}
Use both methods to calculate the frequency, $p_C$ of the $C$ allele.
\begin{answer}
By gene counting, $p_C = 175/200 = 0.875$. By the formula, $p_C = 0.8
+ 0.15/2 = 0.875$.
\end{answer}
\end{exercise}

\begin{exercise}
Use both methods to calculate the frequency, $p_D$ of the $D$ allele.
\begin{answer}
By gene counting, $p_D = 25/200 = 0.125$. By the formula, $p_D = 0.05
+ 0.15/2 = 0.125$.
\end{answer}
\end{exercise}

\begin{exercise}
Prove that the two methods are equivalent.
\begin{answer}
There is more than one correct answer. One approach begins with the
observation that the frequency of $A_1$ is the same as the probability
that a random gene copy chosen from a random individual is allele
$A_1$.  Let us calculate this probability.

With probability $P_{11}$, we choose genotype $A_1A_1$. In this case,
we get allele $A_1$ with probability~1. With probability $P_{12}$, our
individual is $A_1A_2$, and we then get $A_1$ with probability
$1/2$. If we choose $A_2A_2$, we cannot possibly get $A_1$. Thus, the
probability of $A_1$ is
\[
(P_{11} \times 1) + (P_{12} \times 1/2) + (P_{22} \times 0)
 = P_{11} + P_{12}/2
\]
\end{answer}
\end{exercise}

\bigskip\noindent
At some SNP locus, a sample of 100 individuals includes 15 copies of
genotype ``CC,'' 50 of ``CT,'' and 35 of ``TT.''  Please use these
data in answering the following:
\begin{exercise}
What is the number of gene copies in this sample?  
\begin{answer}
200
\end{answer}
\end{exercise}

\begin{exercise}
What are the allele frequencies $p$ and $q$ of the two nucleotides (C
and T)? 
\begin{answer}
For allele C, $p = 0.15 + 0.5/2 = 0.4$. For T, $q = 1-p = 0.6$
\end{answer}
\end{exercise}

\begin{exercise}
What is the expected heterozygosity under random mating?
\begin{answer}
$H = 2pq = 2\cdot 0.4 \cdot 0.6 \approx 0.48$.
\end{answer}
\end{exercise}

\begin{exercise}
What is the observed heterozygosity?
\begin{answer}
$H_{obs} = 0.5$.
\end{answer}
\end{exercise}

\section{Using $F$-statistics}

The observed and HW genotype frequencies are
\begin{center}
  \begin{tabular}{ccc}
    & \multicolumn{2}{c}{Genotype Frequencies}\\ \cline{2-3}
Genotype & Observed & Hardy-Weinberg\\ \hline
$A_1A_1$ & $P_{11}$ & $p^2$ \\
$A_1A_2$ & $P_{12}$ & $2pq$ \\
$A_2A_2$ & $P_{22}$ & $q^2$ \\
  \end{tabular}
\end{center}
In the HW formulas, $p$ and $q$ are the allele frequencies of the
\emph{population} and are unknown.  We can, however, estimate them from our
data by $\hat p = P_{11} + P_{12}/2$ and $\hat q=1-\hat p$.  Plugging these
estimates into the HW formulas will provide estimates of the HW genotype
frequencies, which can then be compared with $P_{11}$, $P_{12}$, and
$P_{22}$. 

To measure the deviation between observed and expected genotype frequencies,
it is convenient to define a variable,
$F$, which satisfies
\begin{eqnarray*}
  P_{11} &=& p^2 + pqF\\
  P_{12} &=& 2pq - 2pqF\\
  P_{22} &=& q^2 + pqF
\end{eqnarray*}
The three equations above provide three different ways to estimate $F$:
\begin{eqnarray}
  F &=& \frac{P_{11} - p^2}{pq}\label{eq.F11}\\
  F &=& -\frac{P_{12} - 2pq}{2pq}\label{eq.F12}\\
  F &=& \frac{P_{22} - q^2}{pq}\label{eq.F22}
\end{eqnarray}
All three formulas produce the same number.

Consider the following data
\begin{center}
  \begin{tabular}{cc}
             & Number\\
    Genotype & of copies\\ \hline
$A_1A_1$ & $n_{11} = 100$\\
$A_1A_2$ & $n_{12} = 100$\\
$A_2A_2$ & $n_{22} = 100$\\
\end{tabular}
\end{center}
In other words, there are 100 copies of each of the three genotypes.  
\begin{exercise}
Use these data to calculate both the observed and the expected
genotype frequencies.
\begin{answer}
The observed genotype frequencies are $P_{11} = P_{12} = P_{22} =
1/3$. The frequency of $A_1$ is $p = 1/2$.  Given $p$ the three
expected genotype frequencies are $E[P_{11}] = p^2 = 1/4$, $E[P_{12}]
= 2p(1-p) = 1/2$, and $E[P_{22}] = (1-p)^2 = 1/4$.
\end{answer}
\end{exercise}

\begin{exercise}
Next, calculate $F$ using each of formulas~\ref{eq.F11}, \ref{eq.F12},
  and~\ref{eq.F22}.
\begin{answer}
Using (\ref{eq.F11}), $F = (P_{11} - p^2)/pq 
 = (1/3 - 1/4)/(1/4) = 1/3$. Using (\ref{eq.F12}), $F = -(P_{12} -
 2pq)/2pq =  -(1/3 - 1/2)/(1/2) = 1/3$. Using (\ref{eq.F22}), $F =
 (P_{22} - q^2)/pq = (1/3 - 1/4)/(1/4) = 1/3$.
\end{answer}
\end{exercise}

\begin{exercise}
In a sample of 72 individuals, we have the following genotype counts:
$n_{11}= 0$,  $n_{12}= 5$, and $n_{22}=67$. What are $P_{11}$,
$P_{12}$, $P_{22}$, $p_1$, and $F$?
\begin{answer}
$P_{11} = 0$, $P_{12} = 0.069444$, $P_{22} = 0.930556$, $p_1 =
0.034722$, and $F = -0.035971$.
\end{answer}
\end{exercise}

\begin{exercise}
In another sample, we have the following genotype counts:
$n_{11}= 53$,  $n_{12}= 19$, and $n_{22}=1$. What are $P_{11}$,
$P_{12}$, $P_{22}$, $p_1$, and $F$?
\begin{answer}
$P_{11} = 0.726027$, $P_{12} = 0.260274$, $P_{22} = 0.013699$, $p_1 =
0.856164$, and $F = -0.056762$.
\end{answer}
\end{exercise}

\section{Final words}

We have glossed over various technicalities in this homework
assignment.  For example, the HW formulas $p^2$, $2p(1-p)$, and
$(1-p)^2$ are correct when $p$ is the allele frequency \emph{in the
  population}, but we have encouraged you to substitute the estimate
of $p$ obtained from the data.  This introduces a small bias, which
can be corrected by using suitably modified versions of the HW
formulas.  For details, see \citet[p.~509]{Morton:TPB-2-507}.  Unless
your sample is very small, however, the effect of this correction is
small.
