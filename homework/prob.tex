\chapter{Probability}
\label{hw.prob}

Each exercise is worth 10 points.

\begin{exercise}
You toss a fair coin twice.  What is the probability of
  observing two heads?
\begin{answer}
$1/4$
\end{answer}
\end{exercise}

\begin{exercise}
With the same coin experiment, what is the probability of a head
  and a tail (in that order)?
\begin{answer}
$1/4$
\end{answer}
\end{exercise}

\begin{exercise}
What is the probability of a tail and then a head?
\begin{answer}
$1/4$
\end{answer}
\end{exercise}

\begin{exercise}
What about a head and a tail in either order?
\begin{answer}
$1/2$
\end{answer}
\end{exercise}

\begin{exercise}
You toss a coin three times and observe 2 heads and a tail in
  some order.  In how many ways can this happen?  (In other words, how
  many sequences like ``HHT'' contains 2 ``H''s and one ``T''?)
\begin{answer}
3: THH, HTH, and HHT.
\end{answer}
\end{exercise}

\begin{exercise}
You toss a coin four times and observe 2 heads and 2 tails in
  some order.  In how many ways can this happen?  (In other words, how
  many sequences like ``HHTT'' contains 2 ``H''s and 2 ``T''s?)
\begin{answer}
There are 6 ways to choose 2 out of 4: ${4 \choose 2} = \frac{4!}{2!
  \times 2!} = 24/[2 \times 2] = 6$.  They are: HHTT, HTHT, HTTH,
THHT, THTH, and TTHH.
\end{answer}
\end{exercise}

\begin{exercise}
What is the probability of observing 2 heads and a tail (order
  unspecified) in 3 tosses.
\begin{answer}
$3 \times 1/2^3 = 3/8$
\end{answer}
\end{exercise}

\begin{exercise}
What is the probability of observing 2 heads and 2 tails (order
  unspecified) in 4 tosses.
\begin{answer}
$6 \times 1/2^4 = 6/16 = 3/8$
\end{answer}
\end{exercise}

\begin{exercise}
You toss two fair dice, one red and one black.  What is the
probability that you observe either a red 4 or a black 6 (or
both)?
\begin{answer}
$1/6 + 1/6 - 1/36 = 11/36$
\end{answer}
\end{exercise}

\begin{exercise}
You toss one fair die.  What is the probability that you observe
either a 4 or a 6? 
\begin{answer}
$1/6 + 1/6 = 1/3$.
\end{answer}
\end{exercise}

\begin{exercise}
  You toss two fair dice, one red and one black.  What is the
  probability that you observe both a red 4 and a black 6?
\begin{answer}
$1/6 \times 1/6 = 1/36$
\end{answer}
\end{exercise}

Imagine a modified version of Kerrich's urn experiment in which each
trial begins with 3 balls of each color (red and black). 
\begin{exercise}
What is the probability that, in a single trial, both of the balls
drawn are red?
\begin{answer}
$1/2 \times 2/5 = 1/5$
\end{answer}
\end{exercise}

\begin{exercise}
What is the probability that, in a single trial, the first ball is red
and the second black?
\begin{answer}
$1/2 \times 3/5 = 3/10$
\end{answer}
\end{exercise}

You toss a fair coin 3 times.  You receive \$1 for each head and
nothing for tails.  Let $X$ represent the number of dollars you
receive.  
\begin{exercise}
For this random variable, what are the possible values
and the probability of each value? In other words, what is the
probability distribution of $X$?
\begin{answer}
There are two ways to answer this question. The first is algebraic:
point out that $X$ is a binomial distribution with parameters $N=3$
and $p=1/2$. This implies that $X$ can take values 0, 1, 2, and 3; it
takes value $x$ with probability $P_x = {3 \choose x} /8$. The second
way to answer the question is to write the answer in tabular form:
\[
\begin{array}{cc}
x & P_x\\ \hline
0   & 1/8\\
1   & 3/8\\
2   & 3/8\\
3   & 1/8\\
\end{array}
\]
\end{answer}
\end{exercise}

\begin{exercise}
What is the mean?
\begin{answer}
The hard way to answer this question is to evaluate $E[X]$ directly,
using the numerical values of $\Pr[0]$, $\Pr[1]$, $\Pr[2]$, and
$\Pr[3]$. The easy way is to point out that, because $X$ is binomial
with $N=3$ and $p=1/2$, the mean is $E[X]=Np = 3/2$.
\end{answer}
\end{exercise}

\begin{exercise}
What is the variance?
\begin{answer}
The hard way to answer this question is to evaluate $V[X] = E[X^2] -
E[X]^2$, or some similar expression. The easy way is to point out
that, because $X$ is binomial with $N=3$ and $p=1/2$, the the variance
is $V[X]=Np(1-p) = 3/4$.
\end{answer}
\end{exercise}

In JEPr, you saw several tables of counts.  Here is another:
\begin{center}
\begin{tabular}{lccc}
       & \multicolumn{2}{c}{My mood}\\ \cline{2-3}
Weather& Happy & Sad& Sum\\ 
\hline 
Rain   &   30  & 70 & 100\\
Sun    &   90  & 10 & 100\\
\hline
Sum    &  120  & 80 & 200\\
\end{tabular}
\end{center}
In this table, each cell counts the number of days during which
\begin{inparaenum}[(a)]
\item it rained and I was happy,
\item it rained and I was sad,
\end{inparaenum}
and so on.  The table tells us that it rained on 100 days, was sunny
on another 100, and that I was happy 90\% of the time on the sunny
days but only 30\% of the time on rainy ones. Use this table in the
following exercises.

\begin{exercise}
What is the \emph{unconditional} relative frequency of sad?
\begin{answer}
Freq. of sad: $80/200 = 2/5$
\end{answer}
\end{exercise}

\begin{exercise}
What is the \emph{unconditional} relative frequency of rainy?
\begin{answer}
Freq. of rainy: $100/200 = 1/2$
\end{answer}
\end{exercise}

\begin{exercise}
What is the \emph{conditional} relative frequency of sad given rainy?
\begin{answer}
Freq. of sad given rainy: $70/100 = 7/10$
\end{answer}
\end{exercise}

\begin{exercise}
What is the \emph{conditional} relative frequency of rainy given sad?
\begin{answer}
Freq. of rainy given sad: $70/80 = 7/8$
\end{answer}
\end{exercise}

It should turn out that $f(\hbox{rain}|\hbox{sad}) \neq
f(\hbox{rain})$. In a statistical sense, therefore, the weather
depends on my mood. Yet this does \emph{not} imply that my mood has
any causal effect. Statistical dependence should not be confused with
causation. 

\begin{exercise}\label{q.randomgene}
Consider a locus with two alleles, $A$ and $a$, whose
frequencies are $p$ and $q=1-p$.  Suppose that you draw one gene
copy from this population at random.  Let $X=1$ if that gene is a
copy of $A$ and let $X=0$ otherwise.  What are (a)~the mean and
(b)~the variance of $X$, as a function of $p$?
  \begin{answer}
    The quick way to answer this question is to notice that question
    implies that $X$ is a Bernoulli random variable with parameter $p$
    and that therefore $E[X] = p$ and $V[X] = p(1-p)$. It's only
    slightly harder work from first principles. The probability
    distribution of $X$, in tabular form, is
    \[
    \begin{array}{cc}
      X & \Pr[X]\\ \hline
      0 & 1-p\\
      1 & p
    \end{array}
    \]
    This implies that $E[X] = 0 \times (1-p) + 1 \times p = p$. For a
    random variable that takes only the values 0 and 1, it is always
    true that $E[X] = E[X^2]$, because $0^2=0$ and
    $1^2=1$. Consequently, $E[X^2]$ also equals $p$. Recall that, for
    any random variable, the variance equals $V = E[X^2] - E[X]^2$. In
    the current case, this becomes $V = p - p^2 = p(1-p)$.
\end{answer}
\end{exercise}

\begin{exercise}
  Suppose we form diploid offspring by sampling with replacement from
  the parental gene pool.  Then in any random offspring, the number,
  $Y$, of copies of allele $A$ is the sum of two independent random
  variables, each of which equals~1 with probability $p$ and 0 with
  probability $1-p$. What does this imply about the mean and variance
  of $Y$?
\begin{answer}
$E[Y] = 2p$ and $V[Y] = 2p(1-p)$. There are two simple ways to get
these results:
\begin{enumerate}
\item The question implies that $Y$ is binomial with $N=2$ and
  probability parameter $p$. Consequently, $E[Y]=2p$ and $V[Y] =
  2p(1-p)$. 
\item The question says that $Y$ is a sum of two independent values,
  each of which is a Bernoulli random variable with mean $p$ and
  variance $p(1-p)$. Therefore, $E[Y] = 2p$ and $V[Y] = 2p(1-p)$.
\end{enumerate}
There are also harder ways, which involve deriving the properties of
the Binomial or Bernoulli distributions.
\end{answer}
\end{exercise}

Consider a particular nucleotide site in the DNA of some hypothetical
population.  In this population, 80\% of chromosomes have the
nucleotide adenine (A) while 20\% have guanine (G).  In other words,
the relative frequency of allele A is 0.80.  We take samples from this
population, each of which consists of two chromosomes drawn
independently and at random.  In each sample, we observe either AA,
AG, or GG.  The number of A's in this sample is a random variable,
which I will call $X$.  This variable can take values 0, 1, or
2.
\begin{exercise}
What is the distribution of $X$?  
\begin{answer}
Easy answer:~The distribution is binomial with $N=2$ and probability
parameter $p=0.8$. 
Hard answer:~The distribution takes values 0, 1, and 2, with
probabilities 0.04, 0.32, and 0.64.
\end{answer}
\end{exercise}

\begin{exercise}
What is the mean?
\begin{answer}
Because $X$ is binomial with $N=2$ and probability parameter $p=0.8$,
its mean is $E[X] = 2p = 1.6$.
\end{answer}
\end{exercise}

\begin{exercise}
What are the variance and standard deviation?
\begin{answer}
Because $X$ is binomial with $N=2$ and probability parameter $p=0.8$,
its, variance is $V[X] = 2p(1-p) = 0.32$ and its standard deviation is
$\sqrt{(V[X])} = 0.566$.
\end{answer}
\end{exercise}

\begin{exercise}
Sketch a histogram of the distribution of $X$.
\begin{answer}
Answer not provided.
\end{answer}
\end{exercise}

JEPr describes Bortkiewicz's data on the frequency of deaths caused by
mule kicks in the Prussian army.  As you will recall, those data are
approximately Poisson, and the mean number of deaths per corps-year
was $\lambda = 0.61$. Use the Poisson distribution formula to answer
the following questions.
\begin{exercise}
What is the probability that there will be zero such deaths during a
given year for a given corps?
\begin{answer}
Pr no deaths: $e^{-0.61} = 0.54$
\end{answer}
\end{exercise}

\begin{exercise}
What is the probability that there will be one such death during a
given year for a given corps?
\begin{answer}
Pr one death: $0.61 e^{-0.61}$
\end{answer}
\end{exercise}

\begin{exercise}
What is the probability that there will be \emph{AT LEAST} one such
death during a given year for a given corps?
\begin{answer}
Pr at least one death: $1- e^{-0.61} = 0.46$
\end{answer}
\end{exercise}

\begin{exercise}
What is the probability that there will be 2 such deaths during a
given year for a given corps?
\begin{answer}
\hbox{Pr 2 deaths}: $0.61^2 e^{-0.61}/2! = 0.101$
\end{answer}
\end{exercise}

Suppose that the number of offspring per female is Poisson, and that
the average female has two offspring. (This keeps the population from
growing or shrinking.) We are assuming, in other words, that the
Poisson distribution has mean $\lambda=2$.
\begin{exercise}
What is the probability that a random female will have no children
at all?
\begin{answer}
Pr no child: $e^{-2} = 0.135$
\end{answer}
\end{exercise}

\begin{exercise}
What is the probability that a random female will have one
child? 
\begin{answer}
Pr 1 child: $2 e^{-2} = 0.27$.
\end{answer}
\end{exercise}

\begin{exercise}
What is the probability that a random female will have \emph{AT LEAST} one
child? 
\begin{answer}
Pr at least 1 child: $1-e^{-2} = 0.865$.
\end{answer}
\end{exercise}

In Kerrich's urn experiment, suppose you get \$1 for each red ball and
\$0 for each green one, and let $X$ and $Y$ represent the dollars you
receive on the two draws within a single trial of the experiment.
\begin{exercise}
Write down the probability distribution of $X$ and $Y$ in
  tabular form. Your table should have columns for $X$, for $Y$, and
  for the joint probability of $X$ and $Y$, i.e. $\Pr[X,Y]$.
  \begin{answer} This is exactly like Fig.~\ref{JEPr-fig.tree} of
    JEPr, which presents the following probability distribution:
\begin{center}
\begin{tabular}{cc}
Event & Prob\\ \hline
RR & 1/6\\
RG & 1/3\\
GR & 1/3\\
GG & 1/6\\
\end{tabular}
\end{center}
where ``R'' and ``G'' stand for ``red'' and ``green'', ``RG'' means
``1st ball red and 2nd green,'' and so on. For the current question,
we translate ``R'' and ``G'' into into ``1'' and ``0,'' and we define
$X$ and $Y$ to equal the values of ball~1 and ball~2. The probability
distribution becomes
\begin{center}
\begin{tabular}{ccc}
$X$ & $Y$ & $\Pr(X,Y)$\\ \hline
1 & 1 & 1/6\\
1 & 0 & 1/3\\
0 & 1 & 1/3\\
0 & 0 & 1/6\\
\end{tabular}
\end{center}
\end{answer}
\end{exercise}

\begin{exercise}
What are the expected values of $X$ and of $Y$?
\begin{answer}
$E[X] = 1/6 + 1/3 = 1/2$; $E[Y] = 1/6 + 1/3 = 1/2$.
\end{answer}
\end{exercise}

\begin{exercise}
What are the variances of $X$ and of $Y$?
\begin{answer}
$E[X^2] = E[Y^2] = 1/2$, so $V[X] = E[X^2]- E[X]^2 = 1/2
- 1/4 = 1/4$. $V[Y]$ is also =1/4.
\end{answer}
\end{exercise}

\begin{exercise}
What is the covariance of $X$ and $Y$?
\begin{answer}
$E[XY] = 1/6$, so $C(X,Y) = E[XY] - E[X]E[Y] = 1/6 -
1/2^2 = -1/12 \approx -0.0833$. It makes sense that the covariance
should be negative, because if the first ball is red, then the second
ball is less likey to be red.
\end{answer}
\end{exercise}

Imagine an urn with $N$ balls, of which 1 is red and the rest are
green. You draw 2 balls from the urn at random \emph{without}
replacement.  Let $X=1$ if the first ball is red and $X=0$
otherwise. Define $Y$ similarly for the second ball.

\begin{exercise}
Draw a tree to represent the probabilities in this experiment. Use
Fig.~\ref{JEPr-fig.tree} of JEPr as a model.
\begin{answer}
\begin{flushleft}
%-*-latex-*-
\let\put\pictexput
\framebox{\beginpicture
\setcoordinatesystem units <0.85cm, 0.6cm>
\setplotarea x from 0 to 7.5, y from -3 to 4
\plot 0 0 2 1.7 /
\plot 0 0 2 -1.7 /
\put {$1/N$} [br] <-1pt,1pt> at 1 0.85
\put {$(N-1)/N$} [tr] <-1pt,-1pt> at 1 -0.85
\put {R}   [c] at 2.25 2
\put {G} [c] at 2.25 -2
%
\plot 2.5 2 4.5 3 /
\plot 2.5 2 4.5 1 /
\put {0} [br] <-1pt,1pt> at 4 2.75
\put {1} [tr] <-1pt,-1pt> at 4 1.25
\put {R}   [l] <3pt,0pt> at 4.5 3
\put {G} [l] <3pt,0pt> at 4.5 1
%
\plot 2.5 -2  4.5 -1 /
\plot 2.5 -2  4.5 -3 /
\put {$1/(N-1)$} [br] <-1pt,1pt> at 4 -1.25
\put {$(N-2)/(N-1)$} [tr] <-1pt,-1pt> at 4 -2.75
\put {R}   [l] <3pt,0pt> at 4.5 -1
\put {G} [l] <3pt,0pt> at 4.5 -3
%
\put {\lines{First\cr ball}}  [b] at 2.25 3.7
\put {\lines{Second\cr ball}} [b] at 4.75 3.7
\put {$XY$} [b] at 5.75 3.7
\put {$11$} <0pt,-1pt> at 5.75 3
\put {$10$} <0pt,0pt> at 5.75 1
\put {$01$} <0pt,-1pt> at 5.75 -1
\put {$00$} <0pt,0pt> at 5.75 -3
%
\put {Prob} [b] at 7 3.7
\put {0} <0pt,-1pt> at 7 3
\put {$1/N$} <0pt,0pt> at 7 1
\put {$1/N$} <0pt,-1pt> at 7 -1
\put {$(N-2)/N$} <0pt,0pt> at 7 -3
\endpicture}
\let\put\latexput

\end{flushleft}
\end{answer}
\label{ex.Ntree}
\end{exercise}

\begin{exercise}
What are the mean and variance of $X$?
\begin{answer}
  The probability distribution of $X$ and $Y$ is shown on the right
  side of the figure in the answer to exercise~\ref{ex.Ntree}. For
  reference, it looks like this:
    \[
    \begin{array}{ccc}
      X & Y & \Pr\\ \hline
      1 & 1 & 0\\
      1 & 0 & 1/N\\
      0 & 1 & 1/N\\
      0 & 0 & (N-2)/N
      \end{array}
    \]
    Using this distribution,
    \begin{eqnarray*}
      E[X] &=& 1\times 0\\
      && \mbox{} + 1 \times 1/N \\
      && \mbox{} + 0\times 1/N\\
      && \mbox{} + 0\times (N-2)/N\\
      &=& 1/N
    \end{eqnarray*}
    This is also $E[X^2]$, because $X$ takes only the values 0 and 1,
    and $X^2 = X$. Thus, $V[X] = E[X^2] - E[X]^2 = 1/N - 1/N^2 = (1/N)
    (1-1/N)$. An alternative answer: these results follow because $X$ 
  is a Bernoulli random variable with parameter $p=1/N$.
\end{answer}
\end{exercise}

\begin{exercise}
What are the mean and variance of $Y$?
\begin{answer}
  The probability distribution of $X$ and $Y$ is shown on the right
  side of the figure in the answer to exercise~\ref{ex.Ntree}. For
  reference, it looks like this:
    \[
    \begin{array}{ccc}
      X & Y & \Pr\\ \hline
      1 & 1 & 0\\
      1 & 0 & 1/N\\
      0 & 1 & 1/N\\
      0 & 0 & (N-2)/N
      \end{array}
    \]
    Using this distribution,
    \begin{eqnarray*}
      E[Y] &=& 1\times 0\\
      && \mbox{} + 0 \times 1/N \\
      && \mbox{} + 1\times 1/N\\
      && \mbox{} + 0\times (N-2)/N\\
      &=& 1/N
    \end{eqnarray*}
    This is also $E[Y^2]$, because $Y$ takes only the values 0 and 1,
    and $Y^2 = Y$. Thus, $V[Y] = E[Y^2] - E[Y]^2 = 1/N - 1/N^2 = (1/N)
    (1-1/N)$. An alternative answer: these results follow because $Y$ 
  is a Bernoulli random variable with parameter $p=1/N$.
\end{answer}
\end{exercise}

\begin{exercise}
What is the covariance of $X$ and $Y$? (Hint: the previous two
questions used a table to represent the probability distribution of
$(X, Y)$. Add a column to this table to represent the product, $XY$.)
\begin{answer}
After adding a column for the product of $X$ and $Y$, the probability
distribution becomes
\[
\begin{array}{cccc}
  X & Y & XY & \Pr\\ \hline
  1 & 1 & 1 & 0\\
  1 & 0 & 0 & 1/N\\
  0 & 1 & 0 & 1/N\\
  0 & 0 & 0 & (N-2)/N
\end{array}
\]
The expected product of $X$ and $Y$ is
\begin{eqnarray*}
  E[XY] &=& 1\times 0\\
  && \mbox{} + 0 \times (\hbox{\small the rest})\\
  &=& 0
\end{eqnarray*}
Thus, $C[X,Y] = E[XY] - E[X] E[Y] = -1/N^2$
\end{answer}
\end{exercise}

\bigskip

\noindent
Suppose that, in a class of 50 students, 20 are women. 
\begin{exercise}
If we choose a student at random from the class, what is the
  probability that this student is a woman?
\begin{answer}
$2/5$ or 0.4
\end{answer}
\end{exercise}

\begin{exercise}
If we choose 2 students from this class at random \emph{without}
replacement, what is the probability that both are women?
\begin{answer}
$(20/50) \times (19/49)$, or $38/245$, or 0.155.
\end{answer}
\end{exercise}

\bigskip\noindent
JEPr discussed the following probability distributions:
(1)~binomial, (2)~Bernoulli, (3)~Poisson, (4)~uniform,
(5)~exponential, and (6)~normal.  Which of these choices would be
most appropriate in each of the following contexts?  (Just write
down the name of the appropriate distribution.)
\begin{exercise}
The weight of a mouse, selected at random from those that live in this
building. 
\begin{answer}
normal
\end{answer}
\end{exercise}

\begin{exercise}
The number of neutral mutations on a gene genealogy of known length.
\begin{answer}
Poisson
\end{answer}
\end{exercise}

\begin{exercise}
The number of copies of $A_1$ (a neutral allele) on some small island
in the South Pacific, assuming that we know the size of this
population and the allele frequency among the parents.
\begin{answer}
binomial
\end{answer}
\end{exercise}

%\begin{exercise}
%Consider a variant of Kerrich's experiment in which each trial begins
%with 6 balls (3 red and 3 green) and you draw 3 balls from the box
%rather than 2. As in Kerrich's experiment, you shake the box before
%each draw, and you sample without replacement.
%\begin{inparaenum}[\it (a)]
%\item Build a model of this experiment by drawing a tree like the one
%  in JEPr. Your tree should end with 8 branches, corresponding to the
%  events RRR, RRG, RGR, RGG, GRR, GRG, GGR, and GGG.
%\item
%  Assume that you get \$1 for each red ball and \$0 for each green
%  one, and let $X$, $Y$, and $Z$ represent the dollars you receive on
%  the 1st, 2nd, and 3rd draws within a single trial of the
%  experiment. What are the means (expected values) of $X$, $Y$, and
%  $Z$?
%\item
%  What is the covariance of $X$ and $Y$?
%\item
%  What is the covariance of $X$ and $Z$?
%\item
%  What is the covariance of $Y$ and $Z$?
%\end{inparaenum}
%\end{answer}
%\end{exercise}

%\begin{exercise}
%  Congenital adrenal hypoplasia (CAH) is a genetic disease that can
%  distort the external genetalia of females and can produce adrenal crises in
%  victims of either sex.  There is a test for the disease, but it is not
%  perfect.  5\% of affected individuals get negative test results, and 1\% of
%  unaffected individuals get positive results.  Among the Yupik eskimo, 1
%  individual in every 400 individuals has the disease.  
%\begin{enumerate}
%\item What is the probability that a Yupik eskimo \emph{has} the disease given
%  that   he got a \emph{positive} test result?
%\item What is the probability that a Yupik eskimo \emph{doesn't have} the
%  disease given that he got a \emph{negative} test result?
%\end{enumerate}
%\end{answer}
%\end{exercise}
