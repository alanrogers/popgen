\chapter{Inbreeding}
\label{hw.inbreed}

Each exercise is worth 10 points.

\begin{exercise}
Find $p$ and $F$ for a sample with genotype frequencies 100,
  200, and 200 of genotypes $A_1A_1$, $A_1A_2$, and $A_2A_2$,
  respectively. 
\begin{answer}
The data tell us that
$P_{11} = 100/500 = 0.2$, and $P_{12} = 200/500 = 0.4$, so the
frequencies of alleles $A_1$ and $A_2$ are $p = P_{11} + P_{12}/2 =
0.4$, and $q = 1-p = 0.6$. To calculate $F$, we might work with the
formula for any of the three genotype frequencies. Let's use $P_{12} =
2pq(1-F)$. Rearranging this gives $F = 1 - P_{12}/2pq$, which works
out to equal 0.167.
\end{answer}
\end{exercise}

\begin{exercise}
Find $p$ and $F$ for a sample with genotype frequencies 200,
  100, and 200 of genotypes $A_1A_1$, $A_1A_2$, and $A_2A_2$,
  respectively. 
\begin{answer}
The data tell us that
$P_{11} = 200/500 = 0.4$, and $P_{12} = 100/500 = 0.2$, so the
frequencies of alleles $A_1$ and $A_2$ are $p = P_{11} + P_{12}/2 =
0.5$, and $q = 1-p = 0.5$. To calculate $F$, we might work with the
formula for any of the three genotype frequencies. Let's use $P_{12} =
2pq(1-F)$. Rearranging this gives $F = 1 - P_{12}/2pq$, which works
out to equal 0.6.
\end{answer}
\end{exercise}

\begin{exercise}
At a bi-allelic locus, alleles $A_1$ and $A_2$ have frequencies
$p=1/4$ and and $q=3/4$. Relative to the current generation, $F$ is
$1/2$. What is the frequency of genotype $A_1A_1$?
\begin{answer}
$P_{11} = 0.156$
\end{answer}
\end{exercise}

\begin{exercise}
At a bi-allelic locus, alleles $A_1$ and $A_2$ have frequencies
$p=q=1/2$. Relative to the current generation, $F$ is
$1/4$. What is the frequency of genotype $A_1A_1$?
\begin{answer}
$P_{11} = 0.313$
\end{answer}
\end{exercise}

\begin{exercise}
What is the coefficient of kinship of half-siblings?
\begin{answer}
1/8
\end{answer}
\end{exercise}

\begin{exercise}
What is the coefficient of kinship of full cousins?
\begin{answer}
1/16
\end{answer}
\end{exercise}

\begin{figure}
{\centering
% Darwin-Wedgewood Genealogy
\mbox{\beginpicture
\setcoordinatesystem units <0.85cm,1cm>
\setplotarea x from -1 to 7, y from 0 to 6
%\setdots
%\axis invisible bottom ticks andacross numbered from -1 to 7 by 1 /
%\axis invisible left ticks andacross numbered from 0 to 6 by 1 /
%\setsolid
\setplotsymbol ({\footnotesize .})
\arrow <10pt> [.2,.67] from 1 5.5 to 1 4.5
\arrow <10pt> [.2,.67] from 1 5.5 to 6 4.5
\arrow <10pt> [.2,.67] from 6 5.5 to 1 4.5
\arrow <10pt> [.2,.67] from 6 5.5 to 6 4.5
%
\arrow <10pt> [.2,.67] from 1 3.5 to 1 2.5
\arrow <10pt> [.2,.67] from 6 3.5 to 6 2.5
%
\arrow <10pt> [.2,.67] from 1 1.5 to 3.5 0.5
\arrow <10pt> [.2,.67] from 6 1.5 to 3.5 0.5
%
\put {\frame <1ex> {\sf Josiah Wedgewood}} at 1 6
\put {\frame <1ex> {\sf Sarah Wedgewood}}  at 6 6
\put {\frame <1ex> {\sf Susannah Wedgewood}} at 1 4
\put {\frame <1ex> {\sf Josiah Wedgewood II}} at 6 4
\put {\frame <1ex> {\sf Charles Darwin}} at 1 2
\put {\frame <1ex> {\sf Emma Wedgewood}} at 6 2
\put {\frame <1ex> {\sf George Darwin}} at 3.5 0
\endpicture}
\\}
\caption{Genealogy of George Darwin}
\label{fig.darwedge}
\end{figure}

\begin{exercise}
Figure~\ref{fig.darwedge} shows the genealogy of George
Darwin, one of Charles Darwin's sons. What is his inbreeding
  coefficient? 
\begin{answer}
$1/16$
\end{answer}
\end{exercise}

\begin{exercise}
  Suppose that (a)~we are studying a locus with two alleles, $A_1$ and
  $A_2$, (b)~the frequency of allele $A_1$ is $p=0.4$, and (c)~the coefficient
  of kinship between two individuals is 1/8.  What are the
  probabilities that their offspring will be
\begin{inparaenum}[(1)]
\item an $A_1A_1$ homozygote,
\item an $A_1A_2$ heterozygote, and
\item an $A_2A_2$ homozygote?
\end{inparaenum}
\begin{answer}
$P_{11} = 0.19$,
$P_{12} = 0.42$, and
$P_{22} = 0.39$.
\end{answer}
\end{exercise}

\begin{exercise}
  Suppose that (a)~we are studying a locus with two alleles, $A_1$ and
  $A_2$, (b)~the frequency of allele $A_1$ is $p=0.1$, and (c)~the coefficient
  of kinship between two individuals is 1/16.  What are the
  probabilities that their offspring will be
\begin{inparaenum}[(1)]
\item an $A_1A_1$ homozygote,
\item an $A_1A_2$ heterozygote, and
\item an $A_2A_2$ homozygote?
\end{inparaenum}
\begin{answer}
$P_{11} = 0.0156$,
$P_{12} = 0.169$, and
$P_{22} = 0.816$.
\end{answer}
\end{exercise}

\begin{exercise}
  This exercise is the same as the last one, except that this time you are
  not given numerical values for the various parameters.  This time, you know
  only that (a)~the frequency of allele $A_1$ is $p$, and (b)~the coefficient
  of kinship between two individuals is $f$.  What are the
  probabilities that their offspring will be
\begin{inparaenum}[(1)]
\item an $A_1A_1$ homozygote?
\item an $A_1A_2$ heterozygote?
\item an $A_2A_2$ homozygote?
\end{inparaenum}
Your answers should be formulas, not numbers.
\begin{answer}
$P_{11} = p^2 + pqf$,
$P_{12} = 2pq(1-f)$, and
$P_{22} = q^2 + pqf$.
\end{answer}
\end{exercise}
