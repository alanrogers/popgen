\chapter{Two Loci}
\label{hw.twoloc}

\begin{exercise}
Consider the following data set:
\begin{center}
\begin{tabular}{lcccc}
Haplotype: & $AB$ & $Ab$ & $aB$ & $ab$\\
Count:     &   30     &   70     & 50       & 20
\end{tabular}
\end{center}
Use these data to answer the following questions:
\begin{inparaenum}[(a)]
\item
  What are the relative frequencies $(x_1, x_2, x_3, x_4)$ of the four
  gamete types $(AB, Ab, aB, ab)$?
\item
 What are the frequencies $p_A$ and $p_B$ of alleles $A$ and $B$?
\item
 What is $D$?
\item
 What is the squared correlation, $r^2$, between loci? (Don't confuse
 this $r$ with the recombination rate, which is also often called $r$.  See
 Gillespie's p.~105.) 
\end{inparaenum}
\begin{answer}
\begin{inparaenum}[(a)]
\item The gamete frequencies are $x_1=30/170 = 0.176$, $x_2= 70/170 = 0.412$,
  $x_3=50/170 = 0.294$, and $x_4= 20/170=0.118$.
\item The allele frequencies at the two loci are $p_A=x_1+x_2 = 0.588$, and
  $p_B=x_1 + x_3 = 0.471$. 
\item $D = x_1x_4 - x_2x_3 = -0.1003$.
\item $r^2 = D^2/[p_A(1-p_A)p_G(1-p_G)] = 0.167$.
\end{inparaenum}
\end{answer}
\end{exercise}

\begin{exercise}
Answer the same questions with these data:
\begin{center}
\begin{tabular}{lcccc}
Haplotype: & $AB$ & $Ab$ & $aB$ & $ab$\\
Count:     &   80     &   30     & 10       & 45
\end{tabular}
\end{center}
\begin{answer}
\begin{inparaenum}[(a)]
\item The gamete frequencies are $x_1=80/165=0.485$, $x_2= 30/165 = 0.182$,
  $x_3=10/165 = 0.061$, and $x_4= 45/165 = 0.273$.
\item The allele frequencies at the two loci are $p_A=x_1+x_2 = 0.667$, and
  $p_B=x_1 + x_3 = 0.545$. 
\item $D=x_1x_4 - x_2x_3 = 0.121$.
\item $r^2 = D^2/[p_A(1-p_A)p_G(1-p_G)] = 0.267$.
\end{inparaenum}
\end{answer}
\end{exercise}

\begin{exercise}
Answer the same questions as before, using the data below, and
defining $x_1$, $x_2$, $x_3$, and $x_4$ to represent the relative
frequencies of $AG$, $AC$, $TG$, and $TC$.
\begin{center}
\begin{tabular}{cc}
Locus & Locus\\
  1   &  2 \\
\hline
A & G\\
A & G\\
A & C\\
T & G\\
T & G\\
T & G\\
T & G\\
T & G\\
T & C\\
T & C\\
\end{tabular}
\end{center}
\begin{answer}
\begin{inparaenum}[(a)]
\item $x_1=0.2$, $x_2=0.1$, $x_3=0.5$, and $x_4=0.2$.
\item $p_A=x_1 + x_2 = 0.3$, and $p_G=x_1 + x_3 = 0.7$.
\item $D = x_1x_4 - x_2x_3 = -0.01$.
\item $r^2=D^2/[p_A(1-p_A)p_G(1-p_G)] = 0.0023$.
\end{inparaenum}
\end{answer}
\end{exercise}

\begin{exercise}
Answer the same questions as before, using the data below, and
defining $x_1$, $x_2$, $x_3$, and $x_4$ to represent the relative
frequencies of $AG$, $AC$, $TG$, and $TC$.
\begin{center}
\begin{tabular}{cc}
Locus & Locus\\
  1   &  2 \\
\hline
A & G\\
A & C\\
A & C\\
A & C\\
T & G\\
T & G\\
T & G\\
T & G\\
T & G\\
T & C\\
\end{tabular}
\end{center}
\begin{answer}
\begin{inparaenum}[(a)]
\item $x_1=0.1$, $x_2=0.3$, $x_3=0.5$, and $x_4=0.1$.
\item $p_A=x_1 + x_2 = 0.4$, and $p_G=x_1 + x_3 = 0.6$.
\item $D = x_1x_4 - x_2x_3 = -0.14$.
\item $r^2=D^2/[p_A(1-p_A)p_G(1-p_G)] = 0.3403$.
\end{inparaenum}
\end{answer}
\end{exercise}

\begin{exercise}
Suppose the recombination rate is $c=1/2$ and that $D=0.2$ in
generation~0.  What would $D$ be after (a)~1 generation, and (b)~3
generations? (Hint: in last displayed equation on p.~104, Gillespie's
$r$ is my $c$.)
\begin{answer}
(a)~After 1 generation, $D = 0.2 \times 0.5 = 0.1$.
(b)~After 3 generations, $D = 0.2 \times 0.5^3 =0.025$.
\end{answer}
\end{exercise}

\begin{exercise}
Suppose the recombination rate is $c=1/1000$ and that $D=0.3$ in
generation~0.  What would $D$ be after (a)~1 generation, and (b)~100
generations? 
\begin{answer}
(a)~After 1 generation, $D = 0.2997$.
(b)~After 100 generations, $D = 0.2714$.
\end{answer}
\end{exercise}

\begin{exercise}
$D$ can be defined in several ways, including
\begin{eqnarray*}
D &=& x_1 - p_A p_B\\
  &=& x_1 x_4 - x_2 x_3
\end{eqnarray*}
Prove that these are equivalent.  (Hint: Start by writing $p_A$ and
$p_B$ in terms of the $x_i$s.  Don't forget that $\sum x_i = 1$.)
\begin{answer} 
\begin{eqnarray*}
D &=& x_1 - (x_1+x_2)(x_1+x_3)\\
  &=& x_1 - x_1^2 - x_1x_3-x_1x_2-x_2x_3
\end{eqnarray*}
At this point, all the numbers are in pairs except the lonely $x_1$ at
the beginning.  Something has to be done about this lonely $x_1$,
because we're trying to derive a formula in which \emph{all} the
numbers are in pairs. To fix things, multiply the lonely $x_1$ by
$x_1+x_2+x_3+x_4$.  This does not change anything because these
numbers sum to 1.  We get
\begin{eqnarray*}
D &=& x_1^2+x_1x_2+x_1x_3+x_1x_4 - x_1^2 - x_1x_3\\
  &&\mbox{} -x_1x_2-x_2x_3\\
  &=& x_1x_4 -x_2x_3
\end{eqnarray*}
\end{answer}
\end{exercise}

\begin{exercise}
$D$ can be defined in several ways, including
\begin{eqnarray*}
D &=& x_1 - p_A p_B\\
-D &=& x_2 - p_A (1-p_B)\\
\end{eqnarray*}
Prove that these are equivalent. (Hint: remember that $x_1$, $x_2$,
$x_3$, and $x_4$ are the frequencies of gametes $AB$, $Ab$, $aB$, and
$ab$.) 
\begin{answer}
One approach is to add the two expressions on the right side. If both
equations are correct, then this sum should equal $D-D=0$. Summing the
two equations gives
\begin{eqnarray*}
D - D &=& x_1 - p_A p_B + x_2 - p_A (1-p_B)\\
  &=& x_1 + x_2 - p_A(p_B + 1 - p_B)\\
  &=& x_1 + x_2 - p_A\\
  &=& p_A - p_A\\
  &=& 0
\end{eqnarray*}
The two expressions sum to zero, so the two equations are equivalent.
\end{answer}
\end{exercise}

\begin{exercise}
\label{prob.2loc-xD}
Suppose that, in generation~1, half the gametes are $A_1B_2$ and the
other half are $A_2B_1$.  What are the values of (a)~the four gamete
frequencies $(x_1,x_2,x_3,x_4)$ and (b)~the coefficient, $D$, of
linkage disequilibrium?
\begin{answer}
  (a)~$(x_1,x_2,x_3,x_4) = (0,1/2,1/2,0)$; 
  (b)~$D = x_1 x_4 - x_2 x_3 = -1/4$.
\end{answer}
\end{exercise}

\begin{exercise}
\label{prob.2loc-w}
Now assume that selection operates at the gamete stage, that alleles
$B_1$ and $B_2$ are neutral, that allele $A_1$ has fitness $1+s$
relative to $A_2$. What are (a)~the four gametic fitnesses, $(w_1,
w_2, w_3, w_4)$, and (b) the mean gametic fitness, $\bar w$. In
calculating the mean, use the gametic relative frequencies
$(x_1,x_2,x_3,x_4)$ that you calculated in
problem~\ref{prob.2loc-xD}. Your answers will be functions of a single
unknown value, $s$.
\begin{answer}
The gametic fitnesses are $(w_1, w_2, w_3, w_4) = (1+s, 1+s, 1, 1)$,
and mean fitness is $\bar w = \sum_i w_i x_i = 1+s/2$. 
\end{answer}
\end{exercise}

We have assumed that selection operates in the gamete stage. As
explained in lecture, this model implies that the expected gamete
frequencies in the following generation are
\begin{eqnarray*}
x'_1 &=& w_1 (x_1 -cD)/\bar w\\
x'_2 &=& w_2 (x_2 +cD)/\bar w\\
x'_3 &=& w_3 (x_3 +cD)/\bar w\\
x'_4 &=& w_4 (x_4 -cD)/\bar w
\end{eqnarray*}
where $\bar w = \sum_i x_i w_i$ is the mean fitness.

\begin{exercise}
\label{prob.2loc-nextgen}
Assume that the recombination rate is $c=1/4$, and take the other
values $(w_i, \bar w, x_i, D)$ from your answers to
problems~\ref{prob.2loc-xD} and ~\ref{prob.2loc-w}.  What are the
expected values of the four gamete frequencies in the following
generation? Your answers will be functions of a single unknown value,
$s$.
  \begin{answer}
\begin{eqnarray*}
  x'_1 &=& (1+s) (0 +1/16)/(1+s/2)\\
       &=& \left(\frac{1}{16}\right)
           \left(\frac{1+s}{1+s/2}\right)\\
x'_2 &=& (1+s) (1/2 -1/16)/(1+s/2)\\
     &=& \left(\frac{7}{16}\right)
         \left(\frac{1+s}{1+s/2}\right)\\
x'_3 &=& (1/2 -1/16)/(1+s/2)\\
     &=& \left(\frac{7}{16}\right)
         \left(\frac{1}{1+s/2}\right)\\
x'_4 &=& (0 +1/16)/(1+s/2)\\
     &=& \left(\frac{1}{16}\right)
         \left(\frac{1}{1+s/2}\right)
\end{eqnarray*}
To simplify this a little further, some students might assume that
$s\ll 1$, so that $(1+s)/(1+s/2) \approx 1+s/2$ and $1/(1+s/2) \approx
1-s/2$, ignoring terms of order $s^2$. That is OK too. With that
approximation, the answers would be
\begin{eqnarray*}
x'_1 &\approx& (1/16)(1+s/2)\\
x'_2   &\approx& (7/16)(1+s/2)\\
x'_3   &\approx& (7/16)(1-s/2)\\
x'_4   &\approx& (1/16)(1-s/2)
\end{eqnarray*}
\end{answer}
\end{exercise}

\begin{exercise}
Soon after a selective sweep has finished, \begin{inparaenum}[\it (a)]
\item what would you expect of nucleotide diversity at nearby loci?
  More diversity, less diversity, or about the same amount as before
  the sweep?
\item Is this effect larger or smaller in regions of low recombination?
\end{inparaenum}
\begin{answer}
(a)~Diversity should be lower at nearby loci. (b)~This effect is
larger in regions of low recombination.
\end{answer}
\end{exercise}

If two sites are 1~centimorgan (cM) apart on a chromosome, then the
rate of recombination between them is $c = 1/100$.  (This is the
definition.)  How does this relate to physical distance along the
chromosome?  In general, 1~cM is about a million base pairs---a
``megabase''---for humans.  In the exercise below, you will use this
generalization to calculate the probability $\lambda$ of recombination
between two adjacent nucleotides.

As a first approximation, you might suppose that
\[
c = \lambda k
\]
where $k$ is the physical distance between the two sites---the number of
nucleotides between them.  Our rule of thumb says that $c=0.01$ when
$k=10^{6}$.  Thus,
\[
\lambda = 0.01 / 10^{6} = 10^{-8}
\]
The rate of recombination between adjacent nucleotides is about
$10^{-8}$. 

But this calculation is only approximate.  To see why, consider sites A
and B below.
\begin{verbatim}
               Before         
     --A-------------------B
                              
     --a-------------------b
\end{verbatim}

\begin{verbatim}
                After          
     --A---\ /-----\ /-----B 
            X       X         
     --a---/ \-----/ \-----b
\end{verbatim}
In the ``Before'' picture, we see two chromosomes, one with the $AB$
haplotype, the other with $ab$.  In the ``After'' picture, two
cross-overs have happened, yet we still have the same two haplotypes.
There were cross-overs but no recombination.  This happens whenever
the number of cross-overs between two loci is even.  The probability
$(c)$ of recombination is the probability that the number of
cross-overs is odd.

In 1919, JBS Haldane calculated the probability of an odd number of
crossovers.  His answer, now known as ``Haldane's mapping function,''
is 
\[
c = (1 - e^{-2\lambda k})/2
\]
provided that cross-overs occur at a constant rate along the
chromosome.
\begin{exercise}
 Our rule of thumb (a centimorgan is a megabase) says that
  $c=1/100$ when $k=10^6$.  Substitute these values into Haldane's
  formula above, and solve for $\lambda$.  How does your result
  compare to the approximate one $(\lambda=10^{-8})$ that we got
  above? 
\begin{answer} 
$\lambda = 1.0101\times 10^{-8}$, which is within about 1\% of the
  approximate value. The approximate formula is quite accurate.
\end{answer}
\end{exercise}

\begin{exercise}
Let's try extrapolating in the other direction. Suppose that two sites
are separated by 50 megabases (\/$5\times 10^7$ bases). 
Estimate $c$, assuming that $\lambda=10^{-8}$, and using 
\begin{inparaenum}[(a)]
\item the approximate formula, and
\item Haldane's mapping function. Finally,
\item comment of the accuracy of the approximation in this case.
\end{inparaenum}
\begin{answer} 
\begin{inparaenum}[(a)]
\item The approximate formula gives $c = 10^{-8} \times 5\times 10^7
  =0.5$.
\item Haldane's mapping function gives $c = (1 - \exp(2 \times 10^{-8}
  \times 5\times 10^7))/2 = 0.316$.
\item The approximation is poor when the distance between loci is
  large. 
\end{inparaenum}
\end{answer}
\end{exercise}

\begin{exercise}
 Make a graph.  The horizontal axis represents $\lambda k$ and
 should run from 0 to 0.5.  On this graph, plot 
\begin{inparaenum}[(a)]
\item the approximate formula $c \approx \lambda k$, and 
\item Haldane's mapping function, as defined above.  The vertical
  difference between the two lines shows the error involved in the
  approximate formula.  
\item Over what range of $\lambda k$ is the approximation
  satisfactory?  
\item Translate this range into units of base pairs (i.e.\ $k$),
  assuming that $\lambda = 10^{-8}$. 
\end{inparaenum}
\begin{answer} 
\begin{center}
\mbox{\beginpicture
\setcoordinatesystem units <1.2\columnwidth, 1.2\columnwidth>
\setplotarea x from 0.0 to 0.5, y from 0.0 to 0.5
\axis left label {$c$}
  ticks numbered from 0.0 to 0.5 by  0.1 /
\axis bottom label {$\lambda k$}
  ticks numbered from 0.0 to 0.5 by 0.1 /
\setdashes
% Approximation: c = l*k
\plot
0 0
0.5 0.5
/
\setsolid
% Haldane's mapping function
\plot
0.000 0.000
0.026 0.026
0.053 0.050
0.079 0.073
0.105 0.095
0.132 0.116
0.158 0.135
0.184 0.154
0.211 0.172
0.237 0.189
0.263 0.205
0.289 0.220
0.316 0.234
0.342 0.248
0.368 0.261
0.395 0.273
0.421 0.285
0.447 0.296
0.474 0.306
0.500 0.316
/
\endpicture}\\
\end{center}
This figure shows both 
\begin{inparaenum}[(a)]
\item the approximation (dashes), and
\item Haldane's mapping formula (solid line).
\item The approximation looks good for $\lambda k <
0.1$.  On a genome-wide average, $\lambda\approx10^{-8}$ for
humans. Consequently, 
\item the approximation should work well when $k < 10^7$. In other
  words, it should work for sites separated by no more than 10
  megabases.
\end{inparaenum}
\end{answer}
\end{exercise}

On p.~110, Gillespie shows that a selective sweep removes variation
at all sites such that (roughly) that $c/s < 0.1$, where $c$ is the
recombination rate with the selected site.  (This applies both
upstream and downstream from the selected site.)  
\begin{exercise}
How large is the affected region of the chromosome if $\lambda =
10^{-8}$ and $s=0.001$?  
\begin{answer}
  We are looking for sites such that $c/s < 0.1$, or $c < 0.1s$, where
  $c$ is the recombination rate between the site in question and the
  selected site. For this problem, $s=0.001$, so $c < 10^{-4}$. We
  established above that when $c$ less than 0.1 or so, it is
  approximately equal to $\lambda k$, where $\lambda = 10^{-8}$, and
  $k$ is the distance between the sites in base pairs. Our largest $c$
  value is much smaller than 0.1, so we can use this approximate
  formula. Using this formula, $c = 10^{-8} k < 10^{-4}$, so $k <
  10,000$. Variance is removed from a region twice this size, because
  the region extends for 10,000 bases each way from the selected
  site. The size of the affected region is therefore 20,000 bases, or
  20~kb.
\end{answer}
\end{exercise}

\begin{exercise}
How large is the affected region of the chromosome if $\lambda =
10^{-8}$ and $s=0.1$?
  \begin{answer} 
The affected region covers 2~Mb
\end{answer}
\end{exercise}

\begin{exercise}
 In Europeans, the allele for lactase persistence sits on a
  region of LD that extends for a megabase.  The persistence allele is
  still polymorphic, but let us suppose that it sweeps to fixation,
  removing most of the variation from this entire megabase-sized
  region.  Based on your answer to the previous question, how strong
  would this selection need to be?
\begin{answer}
 $s=0.05$.
\end{answer}
\end{exercise}

%DNA sequences near human lactase gene, typed in a
%European sample.  Columns are
%nucleotide sites.  Top row (the
%reference sequence) shows common state at each site. Capital
%\texttt{A} is lactase persistance
%allele\cite{Bersaglieri:AJH-74-1111}.  Numbered rows represent
%individual chromosomes. Dots indicate identity with top row.  60
%chromosomes identical to reference sequence were omitted to save
%space. From HapMap release r23.}\label{tab.lctseq}
%
\tiny
\centering
\ttfamily
\begin{tabular}{cc}
  &cgcttcaggcattcctatctaaacagaccaacgtaAgggtacaatgcctaacccagacgtttcaactct\\
20&.....................................................................\\
21&.....................................................................\\
22&.....................................................................\\
23&.....................................................................\\
24&.....................................................................\\
25&.....................................................................\\
26&.....................................................................\\
27&..................t..................................................\\
28&..................t..................................................\\
29&..................................................c..................\\
37&...................................G..a.gt.....t.........gac.c.tgtct.\\
38&...ccgga....gat..at..gg..c.....tc.gGaaa.g..ccttt...tg......c...t.t...\\
39&...ccgga....gat..at..gg..c.....tc.gGaaa.g..ccttt...tg......c...t.t...\\
40&..tcc...agtag.t.cat..g.....t..ttccgG..a.gt.....t.........gac.c.tgtct.\\
41&..tcc...agtag.t.cat..g.....t.gttccgG..a.gt.....t.........gac.c.tgtct.\\
42&..tcc...agtag.t.cat..g.....t.gttccgG..a.gt.....t.........gac.c.tgtct.\\
43&..tcc...agtag.t.cat..g.....t.g.tc.gG..a.gt.....t.........gac.c.tgtct.\\
44&..tcc...agtag.t.cat..g.....t..ttc.gG..acgt.....t.........gac.c.tgtct.\\
45&..tcc...agtag.t.cat..g.....t.gttc.gG..a.gt.....t.........gac.c.tgtct.\\
46&...ccgga....gat..at..gg..c.....tc.gGaaa.g..ccttt...tg......cg.gt.t..c\\
47&..tcc...agtag.t.cat..g.....t.gttccgG..a.gt.....t.........gac.c.tgtct.\\
48&..tcc...agtag.t.cat..g.....t.gttccgG..a.gt.....t.........gac.c.tgtct.\\
49&..tcc...agtag.t.cat..g.....t.gttccgG..a.gt.....t.........gac.c.tgtct.\\
50&tatccgga....g.tc.atcgg.tc.g.tg.tc.gG..a.g.g....tg....ggt...cg.gt.t..c\\
51&ta.ccgga....g.t..atcgg.tc.g.tg.tc.gG..a.g.g....tg....ggt...cg.gt.t..c\\
52&ta.ccgga....g.t..atc.g.tc.g.tg.tc.gG..a.g.g....tg....ggt...cg.gt.t..c\\
53&ta.ccgga....g.t..atcgg.tc.g.tg.tc.gG..a.g.g....tg....ggt...cg.gt.t..c\\
\end{tabular}

\begin{exercise}
  Identify and describe one recombination event in
  table~\ref{tab.lctseq}. (Don't use any of the events described in
  the answered questions in the back.)
\begin{answer}
There are many possible answers. Here are a few examples.
\begin{inparaenum}[(a)]
\item Sites 1--12 in sequences 41--42.
\item Sites 1--15 in sequence 31.
\item Sites 38-42 in sequences 40--49, excluding sequence 46.
\end{inparaenum}
\end{answer}
\end{exercise}

\begin{exercise}
  Identify and describe another recombination event in
  table~\ref{tab.lctseq}. (Don't use any of the events described in
  the answered questions in the back.)
\begin{answer}
There are many possible answers. Here are a few examples.
\begin{inparaenum}[(a)]
\item Sites 1--12 in sequences 41--42.
\item Sites 1--15 in sequence 31.
\item Sites 38-42 in sequences 40--49, excluding sequence 46.
\end{inparaenum}
\end{answer}
\end{exercise}


