\chapter{Population Structure}
\label{hw.pstruc}

$F_{ST}$ can be expressed in terms of gene diversity:
\begin{equation}
F_{ST} = \frac{H_T - H_S}{H_T}
\label{eq.Fst.H}
\end{equation}
in terms of homozygosity (Gillespie's formula 5.3, p.~132):
\begin{equation}
F_{ST} = \frac{G_S - G_T}{1 - G_T}
\label{eq.Fst.G}
\end{equation}
or in terms of the variance $V$ between allele frequencies of subdivisions:
\begin{equation}
F_{ST} = V/pq
\label{eq.Fst.V}
\end{equation}

\begin{exercise}
Verify algebraically that formulas~\ref{eq.Fst.H} and~\ref{eq.Fst.G}
are equivalent. (Hint: Use the facts that $G_S = 1 - H_S$ and $G_T = 1
- H_T$. This just says that every genotype is either a homozygote or a
heterozygote, so homozygosity and heterozygosity must sum to 1.)
\begin{answer}
$G_S = 1-H_S$, and $G_T = 1-H_T$. Substituting these values into
  equation~\ref{eq.Fst.G} gives
\[
F_{ST} = \frac{1-H_S - 1 + H_T}{1 - 1 + H_T} = \frac{H_T - H_S}{H_T},
\]
which is the same as equation~\ref{eq.Fst.H}.
\end{answer}
\end{exercise}

\begin{exercise}
Verify that formulas~\ref{eq.Fst.H} and~\ref{eq.Fst.V} are equivalent.
(Hint: Use Wahlund's formula, which was explained in the lecture on
population structure. Or consult Gillespie's discussion of $F_{ST}$.)
\begin{answer}
We will work with equation~\ref{eq.Fst.H}, which is a function of
$H_S$ and $H_T$. Let us begin with the formula for $H_S$:
\begin{eqnarray*}
H_S &=& \sum c_i 2p_i(1-p_i)\\
    &=& 2\sum c_i p_i - 2\sum c_i p_i^2\\
    &=& 2p - 2\sum c_i p_i^2
\end{eqnarray*}
The sum here is a weighted average over subpopulations, so think of it
as an expectation, and use the hint mentioned in the text of this
question: $\sum c_i p_i^2 = V + p^2$, where $V$ is the variance of the
$p_i$. This gives
\begin{equation}
H_S = 2p - 2p^2 - 2V = 2p(1-p) - 2V
\label{eq.wahlund}
\end{equation}
By definition, $H_T= 2p(1-p)$. Substitute this, along with
equation~\ref{eq.wahlund}, into equation~\ref{eq.Fst.H}, and you will
get
\begin{eqnarray*}
F_{ST} &=& (H_T - H_S)/H_T\\
      &=& \frac{2p(1-p) - 2p(1-p) + 2V}{2p(1-p)}\\
      &=& V/p(1-p)
\end{eqnarray*}
\end{answer}
\end{exercise}

In 1978, Sewall Wright published a chapter on variation among human
populations.  He included a table of allele frequencies at six loci,
from which I have extracted data on a two biallelic loci.  (I also
left out several of his populations and changed his racial labels to
geographic ones.)  The table below shows the frequencies of the $P^1$
allele at the $P$ locus and the $Fy$ allele at the Duffy locus:
\begin{center}
\begin{tabular}{lcc}
Population         & $P^1$ & $Fy$ \\
\hline
African            & 0.734 & 0.037 \\
European           & 0.496 & 0.436 \\
Australian         & 0.330 & 1.000 \\
Asian              & 0.259 & 0.901 \\
\hline
Average            & 0.455 & 0.593 \\
\end{tabular}
\end{center}
\begin{exercise}
What is $F_{ST}$ at the $P$ locus?
\begin{answer}
The question only asks for $F_{ST}$, but I'm providing more detail:
$H_S = 0.429123$,  $H_T=0.495905$, and $F_{ST}=0.134666$.
\end{answer}
\end{exercise}

\begin{exercise}
What is $F_{ST}$ at the Duffy locus?
\begin{answer}
The question only asks for $F_{ST}$, but I'm providing more detail:
$H_S = 0.185367$,  $H_T=0.482516$, and $F_{ST}=0.615832$.
\end{answer}
\end{exercise}

\begin{exercise}
What is the average of the two $F_{ST}$ estimates?  How does your estimate
compare with the value of 0.12, which has been found in many other studies of
genetic differences between continental human populations?
\begin{answer}
The average is 0.375246. For these two loci, $F_{ST}$ is appreciably
larger than it is for the average locus in Wright's (or anyone else's)
data.
\end{answer}
\end{exercise}

\begin{exercise}
Suppose that
\begin{inparaenum}[(1)]
\item a population is divided into several groups, within which mating
  is random,
\item the frequency of allele $A$ is $\bar p = 1/4$ in the population
  as a whole, and
\item $F_{ST} = 1/3$.
\end{inparaenum}
What is the expected frequency within the population as a whole of
genotypes $AA$, $Aa$, and $aa$?
\begin{answer}
$P_{AA} = 0.125$, $P_{Aa} = 0.25$, and $P_{aa} = 0.625$,
\end{answer}
\end{exercise}

\begin{exercise}
Suppose that
\begin{inparaenum}[(1)]
\item a population is divided into several groups, within which mating
  is random,
\item the frequency of allele $A$ is $\bar p = 1/2$ in the population
  as a whole, and
\item $F_{ST} = 1/10$.
\end{inparaenum}
What is the expected frequency within the population as a whole of
genotypes $AA$, $Aa$, and $aa$?
\begin{answer}
$P_{AA} = 0.275$, $P_{Aa} = 0.45$, and $P_{aa} = 0.275$,
\end{answer}
\end{exercise}

Sewall Wright showed that at equilibrium between migration and genetic
drift,
\[
E[F_{ST}] = \frac{1}{4Nm + 1}
\]
where $N$ is the diploid effective size of each sub-population. The
result is based on Wright's ``Island Model'' of population structure,
which assumes there is an infinite number of sub-populations, and that
in each generation a fraction $m$ of each subpopulation consists of
immigrants sampled from the population as a whole. This model is
unrealistic for several reasons, among them being the infinite number
of sub-populations and the equal rate of exchange between each pair of
sub-populations. In the question below, ignore these discrepancies.

\begin{exercise}
Among continental human populations, estimates of $F_{ST}$ are usually
close to $1/9$.  Assume that this represents an equilibrium between
migration and genetic drift under the Island Model of population
structure.  What does this imply about the number $Nm$ of migrants
between pairs of populations in each generation?
\begin{answer}
It implies that $Nm=2$.
\end{answer}
\end{exercise}

