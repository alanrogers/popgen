\documentclass[12pt]{article}
\usepackage{fullpage}
\begin{document}
\author{Alan R. Rogers}
\title{Counting Mutations on a Tree}
\maketitle

This is supplementary info, intended to clarify the homework problem
about allocating mutations on a tree. We expected students to attack
this problem with arithmetic rather than algebra. Nonetheless, some of
you may find an algebraic approach helpful. 

Here is a fragment of the table of pairwise differences from the
homework:
\begin{center}
\begin{tabular}{lrrrr}
Species & mod & Nea & Den & chi\\
\hline
mod     & --- & 168 & 330 & 1304\\
Nea     &     & --- & 318 & 1289\\
Den     &     &     & --- & 1309\\
\end{tabular}
\end{center}
Your task is to allocate mutations from this table onto the following
tree: 
\begin{verbatim}
mod ----
        |----
Nea  ---     |
             |------
             |      |
Den  --------       |--
                    |
Chi ----------------
\end{verbatim}

Let us first consider a smaller and more abstract problem involving
three arbitrary species, A, B, and C, which have the following
phylogenetic tree:
\begin{verbatim}
   x
A ----
      |----
B ----     |
   y       |
           |
C --------- 
        z
\end{verbatim}
Here, $x$ counts mutations along the branch leading to A, $y$ counts
those along the branch to B, and $z$ counts the number along the
\emph{entire} branch leading to C from the common ancestor of A and
B. In some alignment, the numbers of nucleotide differences between
the three pairs of species are $D_{AB}$, $D_{AC}$, and $D_{BC}$.  

Under the infinite sites model, the number of nucleotide
site differences between two species must equal to the number of
mutations along the two branches that separate them. This implies that 
\begin{eqnarray*}
D_{AB} &=& x+y\\\label{eq.DAB}
D_{AC} &=& x+z\\\label{eq.DAC}
D_{BC} &=& y+z\\\label{eq.DBC}
\end{eqnarray*}
These equations holds exactly under the model of infinite sites but
are approximations under other models of mutation.

This is a system of three equations in three unknowns. Solving it
gives
\begin{eqnarray}
x &=& \Big(D_{AB} + D_{AC} - D_{BC}\Bigr)\Big/2\\
y &=& \Big(D_{AB} - D_{AC} + D_{BC}\Bigr)\Big/2
\end{eqnarray}
For example, let us take A, B, and C as ``mod,'' ``Nea,'' and ``Den.''
Then (using the table above) these formulas become
\begin{eqnarray*}
x &=& \Big(168 + 330 - 318\Bigr)\Big/2=90\\
y &=& \Big(168 - 330 + 318\Bigr)\Big/2=78\\
\end{eqnarray*}
In other words, the model allocates 90 mutations to the branch leading
to modern humans and 78 to the one going to Neanderthals.

We can do exactly the same exercise with any group of three
species. For example, moderns, Denisovans and chimpanzees:
\begin{eqnarray*}
x &=& \Big(330 + 1304 - 1309\Bigr)\Big/2=162.5\\
y &=& \Big(330 - 1304 + 1309\Bigr)\Big/2=167.5\\
\end{eqnarray*}
Had the data been in perfect agreement with the infinite sites model,
these numbers should be integers. As you can see, they are not. The
result allocates 167.5 mutations to the branch leading to Denisovans
and 162.5 to the branch leading to moderns.  This latter branch has
two parts, separated by the common ancestor of humans and
Neanderthals. We already know that 90 mutations occurred on the more
recent portion of this branch. The number on the more ancient portion
must therefore be $167.5 - 90 = 72.5$.

You can use apply this method repeatedly to allocate mutations
throughout the larger tree in the homework assignment. The only
remaining ambiguity will involve the branch separating gorillas from
the other species. The method above cannot help with that problem,
because there is no outgroup.
\end{document}
