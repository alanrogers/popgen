\chapter{Quantitative Characters}
\label{hw.qchar}

You are studying the quantitative genetics of seed weight in quinoa
(\emph{Chenopodium quinoa}), an extremely nutritious non-cereal grain
of the Andes. The seeds have very high protein content, with a good
balance of the essential amino acids. According to Wikipedia, ``quinoa
is being considered a possible crop in NASA's Controlled Ecological
Lifesupport System for long-duration manned spaceflights.'' You grow
four inbred lines under uniform conditions in the same field and weigh
100 randomly chosen seeds from each line.  The table summarizes your
data. You want to know how much heritable variation for seed weight
there is.

\begin{table}
\caption{Quinoa seed weights. Each column refers to a different seed
  weight, each row to the sample of seeds from a different inbred line
  of quinoa. The entries give the number of seeds of each weight
  observed in each line}
\label{tab.quinoa}
{\centering\begin{tabular}{lrrrrr}
\hline
weight: &2~mg&3~mg&4~mg&5~mg&6~mg\\
\hline
\hline
line 1 &  3 & 32 & 38 & 22 & 5\\
line 2 &  1 & 19 & 24 & 43 & 13\\
line 3 &  6 & 23 & 41 & 26 & 4\\
line 4 & 12 & 38 & 35 & 13 & 2\\
\hline
Sum    & 22 &112 &138 &104 &24
\end{tabular}\\}
\end{table}

Several of the exercises below will ask you to estimate the
mean and variance. These are estimated as
\begin{eqnarray}
\bar X &=& S_1/N \label{eq.mean}\\
V &=& \frac{S_2 - S_1^2/N}{N-1}
\label{eq.var}
\end{eqnarray}
where $N$ is the number of observations, $S_1$ is the sum of the
observed values, and $S_2$ is the sum of squares of observed
values. With ordinary (ungrouped) data,
\begin{eqnarray*}
S_1 &=& \sum_{i=1}^N x_i, \quad\hbox{and}\\
S_2 &=& \sum_{i=1}^N x_i^2
\end{eqnarray*}
Here, $x_i$ is the value of the $i$th observation, and the sums run
across all $N$ items in the data set. Our data, however, are grouped,
so we can take a shortcut. Let $x$ represent one of the values that
the data may take. In table~\ref{tab.quinoa}, for example, $x$ takes
the values 2, 3, 4, 5, and 6. The number of observations with value
$x$ is written as $n_x$. For example, in the data for line~1, $n_2=3$,
$n_3=32$, and $n_4=38$. With such data,
\begin{eqnarray*}
N  &=& \sum_x n_x,\\
S_1 &=& \sum_x x n_x, \quad\hbox{and}\\
S_2 &=& \sum_x x^2 n_x
\end{eqnarray*}
Now the sums run across the 5 values that data items may take rather
than the 100 items in each data set. Use whichever method you prefer
in calculating $N$, $S_1$, and $S_2$. Then use equations~\ref{eq.mean}
and~\ref{eq.var} to estimate the mean and variance.
\begin{exercise}
  Calculate the mean and variance of seed weight for line~1.
\begin{answer}
Line~1: $\bar X = 3.94$; $V = 0.865$.
\end{answer}
\end{exercise}

\begin{exercise}
  Calculate the mean and variance of seed weight for line~2.
\begin{answer}
Line~2: $\bar X = 4.48$; $V = 0.959$.
\end{answer}
\end{exercise}

\begin{exercise}
  Calculate the mean and variance of seed weight for line~3.
\begin{answer}
Line~3: $\bar X = 3.99$; $V = 0.899$.
\end{answer}
\end{exercise}

\begin{exercise}
  Calculate the mean and variance of seed weight for line~4.
\begin{answer}
Line~4: $\bar X = 3.55$; $V = 0.876$.
\end{answer}
\end{exercise}

\begin{exercise}
\label{it.pooledvar}
  Calculate the mean and variance of seed weight for all lines taken
  together. By the way, this variance estimates $V_P$, the total
  phenotypic variance. 
\begin{answer}
Pooled: $\bar X = 3.99$; $V_P = 1.002$.
\end{answer}
\end{exercise}

\begin{exercise}
\label{it.VG}
  Calculate the variance of the four line means. 
  Because the sample from each line is pretty large, this variance is
  influenced only slightly by environmental effects. It is almost pure
  genetic variance. We will take it as an estimate of
  $V_G$.
\begin{answer}
Variance between lines: 0.1454.
\end{answer}
\end{exercise}

\begin{exercise}
  \label{it.VE}
  Calculate the mean of the four within-line variances. This is an
  estimate of the variance within lines. Because the lines are inbred,
  there is no genetic variation within them, and the variance you
  calculate here estimates the environmental component of variance,
  $V_E$.
\begin{answer}
Variance w/i lines: 0.8998.
\end{answer}
\end{exercise}

\begin{exercise}
  In exercises~\ref{it.VG} and~\ref{it.VE}, you estimated the genetic
  component of variance, $V_G$, and the environmental component,
  $V_E$. The sum of these should equal the phenotypic variance, $V_P$,
  which you estimated in question~\ref{it.pooledvar}. Does it?
\begin{answer}
  Between plus within: $0.1454 + 0.8998 = 1.04520$. In
  question~\ref{it.pooledvar}, I got $V_P = 1.002$, which is pretty
  close.
\end{answer}
\end{exercise}

\begin{exercise}
\label{ex.H}
What fraction of the phenotypic variance $(V_P)$ is
accounted for by $V_G$?  In other words, what is $V_G/V_P$?
\begin{answer}
According to exercise~\ref{it.VG}, $V_G = 0.1454$. According to
exercise~\ref{it.pooledvar}, $V_P = 1.002$. The ratio of these is $H^2
= V_G/V_P = 0.1451$.
\end{answer}
\end{exercise}

\begin{exercise}
  What kind of heritability estimate did you calculate in
  exercise~\ref{ex.H}? Narrow-sense or broad-sense?
\begin{answer}
Broad-sense heritability.
\end{answer}
\end{exercise}

Encouraged that you have heritable variation for seed weight, you
outcross the plants and apply a selection differential $(S)$ of 1~mg to
the seeds, after harvesting them in bulk from all of the plants in
your field. After four generations, the mean seed weight has increased
by 0.2~mg.

\begin{exercise}
\label{it.h}
What is the mean seed weight now? What is the realized (narrow-sense)
heritability? Why does it differ from your previous estimate
(question~\ref{ex.H})?
\begin{answer}
The realized heritability is $h^2 = 0.2/4 = 0.05$, because $0.2/4$ is
the average response \emph{per generation} to selection. The
broad-sense heritability estimated in exercise~\ref{ex.H} was much
larger---0.1451---suggesting that there is a lot of non-additive
genetic variance.
\end{answer}
\end{exercise}

\begin{exercise}
  Approximately, what is the additive genetic variance $(V_A)$ of seed
  weight in your study population? What is the dominance 
  variance, $V_D$? (Assume there is no other form non-additive genetic
  variance besides the dominance variance). Why are you disappointed?
\begin{answer}
$V_A = h^2 V_P = 0.05 \times 1.002 = 0.0501$. The dominance variance
is $V_D = V_G - V_A = 0.1454 - 0.0501 = 0.0953$. Most of the genetic
variance seems to be dominance variance. This is disappointing because
selection responds only to additive variance.
\end{answer}
\end{exercise}

You decide to test your new estimate of the narrow-sense heritability
by estimating the regression of offspring seed weights on parental
seed weights. This is a lot of work, but you are obsessed. You record
the weights of seeds chosen randomly from the population, then grow
them individually in pots and cross-fertilize them. When their seeds
are fully mature, you weigh samples of seeds from each plant. This
allows you to plot the weights of offspring seeds against the average
weights of their parents (when they were seeds).

\begin{exercise}
If you were to plot offspring values against midparent values and fit
a linear regression to these data, what numerical value would you
expect the slope to have?
\begin{answer}
  The regression of offspring on mid-parent has slope $h^2$. (See
  Gillespie's table~6.2, p.~149.) From exercise~\ref{it.h}, we know
  that $h^2 = 0.05$. The regression line should have approximately
  this slope.
\end{answer}
\end{exercise}

What if you had selected on a whole-plant basis, rather than on an
individual-seed basis? You worked at the level of individual seeds
because the seeds are highly variable (differing by more than a factor
of three, from smallest to largest). But most of that variation is
environmental, so of course the seed-to-seed heritability is low. If
you averaged all the seeds on each plant, then the mean seed weights
of (adult) plants would undoubtedly be more heritable (i.e., show a
much larger value of $h^2$) than the weights of individual seeds. This
suggests that you might make more progress (per generation) in your
quest to increase seed weights, if you selected on the mean seed
weights of whole plants, rather than on the weights of individual
seeds. But maybe not!

\begin{exercise}
  Why might this approach be just as slow as the approach you took?
  Hint: The rate of evolution depends on the absolute amount of
  additive genetic variation for the trait, not just on $h^2$!
\begin{answer}
  The response to selection is $R = h^2 S$, as shown in Gillespie's
  Eqn.~6.11. If you made $h^2$ larger while keeping $S$ the same, the
  response would be larger. But this might be hard to do, because if
  each plant is represented by its average seed weight, differences
  among plants will probably be small. For this reason, it might be
  necessary to reduce $S$ in order to make selection feasible.
\end{answer}
\end{exercise}

In 1903, \citet{Pearson:B-2-357} published the results of what was
then the most extensive study ever of the inheritance of human
physical characteristics.  These data occupy a distinguished position
in the history of population genetics, since they were the basis of
R.A.{} Fisher's \citeyear{Fisher:Relatives} famous demonstration that
Mendelian inheritance could account for variation in continuous
characters.  Pearson and Lee collected data on stature, span (distance
between fingertips of outstretched arms), and cubit (fore-arm length),
from about 1100 families. Some of these data are shown in
table~\ref{tab.pearson}.

\begin{table}
\caption{Data from \citet{Pearson:B-2-357}: the correlation between
  mothers and daughters and the phenotypic variance.} 
\label{tab.pearson}

{\centering\fbox{\begin{tabular}{lrrr}
      &Stature & Span & Cubit\\
Mother-daughter corr. & 0.507 & 0.452 & 0.421\\
Phenotypic var.{} (in$^2$)& 6.26  & 8.27  & 0.784
\end{tabular}}\\}
\end{table}

\begin{exercise}
Use the data of Pearson and Lee to estimate the heritability of stature.
(Hint: use Gillespie's table~6.2.) 
\begin{answer}
  The correlation between parent and offspring is $h^2/2$. Thus, we
  estimate $h^2$ as twice the observed correlation. For stature, this
  gives $h^2 = 1.014$. This cannot be literally correct, because $h^2$
  cannot exceed~1. But all estimates contain error, and this error can
  move the estimate outside the legal range. This estimate implies
  that stature is highly heritable.
\end{answer}
\end{exercise}

\begin{exercise}
Use the data of Pearson and Lee to estimate the heritability of span.
(Hint: use Gillespie's table~6.2.) 
\begin{answer}
  The correlation between parent and offspring is $h^2/2$. Thus, we
  estimate $h^2$ as twice the observed correlation. For span, this
  gives $h^2 = 0.904$.
\end{answer}
\end{exercise}

\begin{exercise}
Use the data of Pearson and Lee to estimate the heritability of cubit.
(Hint: use Gillespie's table~6.2.) 
\begin{answer}
  The correlation between parent and offspring is $h^2/2$. Thus, we
  estimate $h^2$ as twice the observed correlation. For cubit, this
  gives $h^2 = 0.842$.
\end{answer}
\end{exercise}

\begin{exercise}
Over the past 100 years, the stature of young women in most developed
countries has increased about 0.157 inches per decade. This is called the
``secular trend'' in stature. This trend probably reflects changes in
diet and/or public health. But let us consider the possibility that it
was caused by natural selection.  What selection gradient $(\beta)$
does this hypothesis imply? You'll need to convert the change per
decade into change per generation. For this purpose, assume that a
human generation is 28 years.
\begin{answer}
The change per generation is $0.157 \times 28/10 = 0.4396$~in. If this
represents a response to selection, then it should equal $V_A \beta$. The next
step is to convert our estimate that $h^2 \approx 1$ into an estimate of $V_A$.
We know that $h^2 = V_A/V_P$ (Gillespie's Eqn.~6.4), and we know that $V_P =
6.26$ (table~\ref{tab.pearson} of this homework). This implies that $V_A
\approx 6.26$ and that the selection gradient is $\beta \approx 0.4396 / 6.26 =
0.07$. In words, this says that an extra inch of stature implies nearly a 10\%
increase in fitness---\emph{very} strong selection. It seems implausible that
small differences in stature could have such large effects on fitness. It is
more likely that the observed trend reflects changes in the environment.
\end{answer}
\end{exercise}

\begin{exercise}
The average bill depth of Geospiza fortis (Darwin's medium ground
finch) increased by 0.5~mm in one generation (1976 to 1978) in the
population on Isla Daphne Major that has been studied by Peter and
Rosemary Grant and their students and colleagues since the early
1970s. The phenotypic standard deviation of this trait is around 1~mm,
and its heritability has been estimated (from the correlations among
relatives) to be around $h^2 = 0.8$.

Given these numbers, what was the selection gradient $(\beta)$ during
the drought of 1977? (The drought was caused by a severe El Ni{\~ n}o
event that forced the birds to feed on large, hard seeds that they
otherwise wouldn't eat.) 
\begin{answer}
The problem says that $h^2 = 0.8$ and that the phenotypic
standard deviation is $\sqrt{V_P} = 1$. This implies that $V_P=1$. The
problem also says that $h^2=0.8$. This implies that $V_A = h^2 V_P =
0.8$. Finally, the response to selection is $R = V_A \beta$, and we
know that $R=0.5$. Thus, $\beta = 0.5/0.8 = 0.625$.
\end{answer}
\end{exercise}

