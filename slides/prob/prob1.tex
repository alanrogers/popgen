\title{Probability}
\author{Alan R. Rogers}
\date{\today}

%\begin{frame}
%\frametitle{Three systems of vocabulary}
%\begin{tabular}{p{2.2in}ccc}
%                        &   1   & 2 & 3\\
%Position on chromosome  & locus & locus & locus\\
%Protein-coding locus    & gene  & gene & gene\\
%Physical copy of DNA at locus  & gene  & allele &gene copy\\
%One of several variants at a locus& allele & allele & allele\\
%\end{tabular}
%
%\bigskip
%
%1 is classical usage, 2 is Gillespie's, and we prefer 3.
%
%\bigskip
%
%\pause
%\begin{quote}
%\emph{Consistency is the hobgoblin of little minds\\}
%\mbox{}\hfill---Ralph Waldo Emerson
%\end{quote}
%\end{frame}
%
%\begin{frame}
%\frametitle{Illustration of classical usage}
%Those organisms (homozygotes) which received like genes, in any pair
%of corresponding loci, from their two parents, would necessarily hand
%on genes of this kind to all of their offspring alike; whereas those
%(heterozygotes) which received from their two parents genes of
%different kinds\ldots (Fisher, 1930, p.~8)
%\end{frame}
%
%\begin{frame}
%\frametitle{Gene genealogy within a population tree}
%\begin{columns}
%\column{0.6\textwidth}
%\centering\input{figgptree}\\[1ex]
%\column{0.4\textwidth}
%Many gene genealogies are possible.
%
%\bigskip
%
%Probabilities depend on population sizes and separation times.
%
%\bigskip
%
%0, ancestral; 1, derived.
%\end{columns}
%\end{frame}

\frame{\titlepage}

\begin{frame}
\centering\let\put\pictexput
\mbox{\beginpicture
\setcoordinatesystem units <0.15\columnwidth, 2in> point at 0 0
\setplotarea x from 0 to 4, y from 0.0 to 0.7
\axis left label {\lines{Relative\cr frequency\cr of\cr heads}}
  ticks numbered from 0.00 to 0.50 by 0.25 /
\axis bottom label {Number of spins (log scale)} 
%  ticks withvalues 3 10 30 300 1000 3000 10000 / 
%  logged at 3 10 30 300 1000 3000 10000  / /
  ticks withvalues 1 3 10 100 1000 10000 / 
  at 0 0.4771 1 2 3 4 / /
\put {\footnotesize (Kerrich 1946)} [br] <0pt,4pt> at 4 0
\setdots
\putrule from 0 0.5 to 4 0.5
\setsolid
% x axis counts coin tosses on log10 scale
% y is the fraction of heads in x tosses
% Data from Kerrich 1964
\plot  
0.000 0.000
0.301 0.000
0.477 0.000
0.602 0.250
0.699 0.400
0.778 0.500
0.845 0.429
0.903 0.500
0.954 0.444
1.000 0.400
1.041 0.455
1.079 0.500
1.114 0.538
1.146 0.571
1.176 0.600
1.204 0.562
1.230 0.588
1.255 0.556
1.279 0.526
1.301 0.500
1.322 0.524
1.342 0.545
1.362 0.522
1.380 0.542
1.398 0.520
1.415 0.538
1.431 0.556
1.447 0.571
1.462 0.586
1.477 0.567
1.544 0.514
1.602 0.525
1.653 0.489
1.699 0.500
1.740 0.491
1.778 0.483
1.813 0.462
1.845 0.457
1.875 0.453
1.903 0.438
1.929 0.447
1.954 0.444
1.978 0.432
2.000 0.440
2.041 0.436
2.079 0.442
2.114 0.446
2.146 0.464
2.176 0.473
2.204 0.463
2.230 0.465
2.255 0.478
2.279 0.484
2.301 0.490
2.398 0.500
2.477 0.487
2.544 0.494
2.602 0.497
2.653 0.502
2.699 0.510
2.740 0.509
2.778 0.520
2.813 0.525
2.845 0.526
2.875 0.520
2.903 0.516
2.929 0.515
2.954 0.509
2.978 0.509
3.000 0.502
3.041 0.493
3.079 0.497
3.114 0.499
3.146 0.503
3.176 0.504
3.204 0.506
3.230 0.511
3.255 0.510
3.279 0.511
3.301 0.506
3.322 0.509
3.342 0.510
3.362 0.511
3.380 0.508
3.398 0.509
3.415 0.506
3.431 0.500
3.447 0.503
3.462 0.503
3.477 0.503
3.491 0.504
3.505 0.503
3.519 0.505
3.531 0.506
3.544 0.506
3.556 0.507
3.568 0.506
3.580 0.507
3.591 0.507
3.602 0.507
3.613 0.508
3.623 0.508
3.633 0.509
3.643 0.510
3.653 0.510
3.663 0.510
3.672 0.511
3.681 0.510
3.690 0.508
3.699 0.507
3.708 0.505
3.716 0.508
3.724 0.507
3.732 0.505
3.740 0.504
3.748 0.504
3.756 0.504
3.763 0.504
3.771 0.504
3.778 0.501
3.785 0.501
3.792 0.501
3.799 0.500
3.806 0.500
3.813 0.499
3.820 0.499
3.826 0.500
3.833 0.501
3.839 0.501
3.845 0.502
3.851 0.502
3.857 0.503
3.863 0.503
3.869 0.502
3.875 0.504
3.881 0.504
3.886 0.503
3.892 0.504
3.898 0.505
3.903 0.504
3.908 0.504
3.914 0.504
3.919 0.504
3.924 0.503
3.929 0.504
3.934 0.505
3.940 0.505
3.944 0.505
3.949 0.505
3.954 0.504
3.959 0.505
3.964 0.505
3.968 0.505
3.973 0.505
3.978 0.506
3.982 0.506
3.987 0.506
3.991 0.506
3.996 0.507
4.000 0.507
/
\endpicture}
\let\put\latexput
\\
\end{frame}

\begin{frame}
\frametitle{Probability and relative frequency in repeated trials}
\begin{columns}
\column{0.6\textwidth}
{\centering\let\put\pictexput
\mbox{\beginpicture
\setcoordinatesystem units <0.15\columnwidth, 2in> point at 0 0
\setplotarea x from 0 to 4, y from 0.0 to 0.7
\axis left label {\lines{Relative\cr frequency\cr of\cr heads}}
  ticks numbered from 0.00 to 0.50 by 0.25 /
\axis bottom label {Number of spins (log scale)} 
%  ticks withvalues 3 10 30 300 1000 3000 10000 / 
%  logged at 3 10 30 300 1000 3000 10000  / /
  ticks withvalues 1 3 10 100 1000 10000 / 
  at 0 0.4771 1 2 3 4 / /
\put {\footnotesize (Kerrich 1946)} [br] <0pt,4pt> at 4 0
\setdots
\putrule from 0 0.5 to 4 0.5
\setsolid
% x axis counts coin tosses on log10 scale
% y is the fraction of heads in x tosses
% Data from Kerrich 1964
\plot  
0.000 0.000
0.301 0.000
0.477 0.000
0.602 0.250
0.699 0.400
0.778 0.500
0.845 0.429
0.903 0.500
0.954 0.444
1.000 0.400
1.041 0.455
1.079 0.500
1.114 0.538
1.146 0.571
1.176 0.600
1.204 0.562
1.230 0.588
1.255 0.556
1.279 0.526
1.301 0.500
1.322 0.524
1.342 0.545
1.362 0.522
1.380 0.542
1.398 0.520
1.415 0.538
1.431 0.556
1.447 0.571
1.462 0.586
1.477 0.567
1.544 0.514
1.602 0.525
1.653 0.489
1.699 0.500
1.740 0.491
1.778 0.483
1.813 0.462
1.845 0.457
1.875 0.453
1.903 0.438
1.929 0.447
1.954 0.444
1.978 0.432
2.000 0.440
2.041 0.436
2.079 0.442
2.114 0.446
2.146 0.464
2.176 0.473
2.204 0.463
2.230 0.465
2.255 0.478
2.279 0.484
2.301 0.490
2.398 0.500
2.477 0.487
2.544 0.494
2.602 0.497
2.653 0.502
2.699 0.510
2.740 0.509
2.778 0.520
2.813 0.525
2.845 0.526
2.875 0.520
2.903 0.516
2.929 0.515
2.954 0.509
2.978 0.509
3.000 0.502
3.041 0.493
3.079 0.497
3.114 0.499
3.146 0.503
3.176 0.504
3.204 0.506
3.230 0.511
3.255 0.510
3.279 0.511
3.301 0.506
3.322 0.509
3.342 0.510
3.362 0.511
3.380 0.508
3.398 0.509
3.415 0.506
3.431 0.500
3.447 0.503
3.462 0.503
3.477 0.503
3.491 0.504
3.505 0.503
3.519 0.505
3.531 0.506
3.544 0.506
3.556 0.507
3.568 0.506
3.580 0.507
3.591 0.507
3.602 0.507
3.613 0.508
3.623 0.508
3.633 0.509
3.643 0.510
3.653 0.510
3.663 0.510
3.672 0.511
3.681 0.510
3.690 0.508
3.699 0.507
3.708 0.505
3.716 0.508
3.724 0.507
3.732 0.505
3.740 0.504
3.748 0.504
3.756 0.504
3.763 0.504
3.771 0.504
3.778 0.501
3.785 0.501
3.792 0.501
3.799 0.500
3.806 0.500
3.813 0.499
3.820 0.499
3.826 0.500
3.833 0.501
3.839 0.501
3.845 0.502
3.851 0.502
3.857 0.503
3.863 0.503
3.869 0.502
3.875 0.504
3.881 0.504
3.886 0.503
3.892 0.504
3.898 0.505
3.903 0.504
3.908 0.504
3.914 0.504
3.919 0.504
3.924 0.503
3.929 0.504
3.934 0.505
3.940 0.505
3.944 0.505
3.949 0.505
3.954 0.504
3.959 0.505
3.964 0.505
3.968 0.505
3.973 0.505
3.978 0.506
3.982 0.506
3.987 0.506
3.991 0.506
3.996 0.507
4.000 0.507
/
\endpicture}
\let\put\latexput
\\}
\column{0.45\textwidth}
\raggedright
\begin{itemize}
\item rel.\ freq.\ of heads gradually approaches limiting value.
\item Limiting value is the \emph{probability} of heads
\item Need not equal $1/2$.
\item We estimate probabilities from relative frequencies.
\item We never know them exactly.
\end{itemize}
\end{columns}
\end{frame}

\begin{frame}
\frametitle{Kerrich's ``urn'' experiment}

\begin{center}
\huge $(\underline{\circ\;\circ\;\bullet\;\bullet})$
\end{center}

\begin{itemize}
\item Urn contains 4 balls: 2 black and 2 white
\item Mix them up.
\item Draw one at random
\item Draw a second \emph{without} replacing first.
\item Repeat 5000 times.
\end{itemize}
\end{frame}

\begin{frame}
\frametitle{Results from Kerrich's urn experiment}
\begin{center}
\begin{tabular}{lrrr}
First  & \multicolumn{2}{c}{Second ball}\\ \cline{2-3}
ball & Black & White & sum\\ \hline
Black  & 756 &  1689 & 2445 \\
White&1688 &   867 & 2555 \\ \hline
sum  &2444 &  2556 & 5000
\end{tabular}
\end{center}
\begin{itemize}
\item If 1st ball is $B$, 2nd is likely to be $W$
\item And vice versa
\end{itemize}
\end{frame}

\begin{frame}
\frametitle{Model of Kerrich's urn experiment}
Assumption: we are equally likely to draw any ball in urn.
\begin{columns}[t]
\column{0.5\textwidth}
\begin{block}{1st Ball}
\begin{center}
$(\circ\;\circ\;\bullet\;\bullet)$
\end{center}
We are equally likely to draw black or white
\end{block}
%%%%%%%%%%%%%%%%%%%%%%%%%%%%%%%%%%%%%%%%%%%%%%%%%%%%%%%%%%%%%%%%
\column{0.5\textwidth}
\begin{block}{2nd Ball}
\begin{center}
\begin{tabular}{ccc}
First & Remaining & Prob.\\
ball  & balls     & of black\\
\hline
$\bullet$ & $(\circ\; \circ\; \bullet)$ & $1/3$\\
\onslide+<4->{$\circ$} &
 \onslide+<4->{$(\circ\; \bullet\; \bullet)$} &
 \onslide+<4->{$2/3$}\\
\end{tabular}
\end{center}
\onslide+<5->
2nd ball usually black if 1st was white,
and vice versa.
\end{block}
\end{columns}
\end{frame}

\begin{frame}
\begin{center}
%-*-latex-*-
\framebox{\beginpicture
\setcoordinatesystem units <0.94cm, 0.6cm>
\setplotarea x from 0 to 7, y from -3 to 4
\plot 0 0 2 1.7 /
\plot 0 0 2 -1.7 /
\put {$1/2$} [br] <-1pt,1pt> at 1 0.85
\put {$1/2$} [tr] <-1pt,-1pt> at 1 -0.85
\put {red}   [c] at 2.5 2
\put {green} [c] at 2.5 -2
%
\plot 3 2 5 3 /
\plot 3 2 5 1 /
\put {$1/3$} [br] <-1pt,1pt> at 4 2.5
\put {$2/3$} [tr] <-1pt,-1pt> at 4 1.5
\put {red}   [l] <3pt,0pt> at 5 3
\put {green} [l] <3pt,0pt> at 5 1
%
\plot 3 -2  5 -1 /
\plot 3 -2  5 -3 /
\put {$2/3$} [br] <-1pt,1pt> at 4 -1.5
\put {$1/3$} [tr] <-1pt,-1pt> at 4 -2.5
\put {red}   [l] <3pt,0pt> at 5 -1
\put {green} [l] <3pt,0pt> at 5 -3
%
\put {\lines{First\cr ball}}  [b] at 2.5 3.7
\put {\lines{Second\cr ball}} [b] at 5.5 3.7
\put {Event} [b] at 6.8 3.7
\put {$RR$} <0pt,-1pt> at 6.8 3
\put {$RG$} <0pt,0pt> at 6.8 1
\put {$GR$} <0pt,-1pt> at 6.8 -1
\put {$GG$} <0pt,0pt> at 6.8 -3
%
\put {Prob} [b] at 8 3.7
\put {$1/6$} <0pt,-1pt> at 8 3
\put {$1/3$} <0pt,0pt> at 8 1
\put {$1/3$} <0pt,-1pt> at 8 -1
\put {$1/6$} <0pt,0pt> at 8 -3
\endpicture}
\\
\bigskip
Tree diagram for urn model
\end{center}
\end{frame}

\begin{frame}
\frametitle{Kerrich's urn experiment: model versus data}
\begin{center}
\begin{tabular}{ccc}
        &            & Observed\\
        & Theoretical& relative  \\
  Event & probability& frequency \\
\hline
\hline
$BB$    & 0.167& 0.151 \\
$BW$    & 0.333& 0.338 \\
$WB$    & 0.333& 0.338 \\
$WW$    & 0.167& 0.173 \\
\hline
\end{tabular}
\end{center}
Theory and observation are not identical,
but they are close.
\end{frame}

\begin{frame}
\begin{center}
Why do we multiply along branches?
\end{center}
\end{frame}

\begin{frame}
\frametitle{Conditional probability}
\begin{itemize}
\item What is the conditional probability that the 2nd ball is white
  given that the first was black?
\pause
\item $2/3$.

\bigskip

\pause
\item
Called a \emph{conditional probability}
and written
\[
\Pr[\hbox{2nd ball white} | \hbox{1st one black}].
\]
\item  ``$|$'' is  pronounced ``given.''
\end{itemize}
\end{frame}

\begin{frame}
\frametitle{Conditional relative frequencies}
{\centering
\begin{tabular}{lrrr}
First  & \multicolumn{2}{c}{Second ball}\\ \cline{2-3}
ball & Black & White & sum\\ \hline
Black  & \bf 756 &\bf \fbox{1689} & \bf 2445 \\
White&1688 &   867 & 2555 \\ \hline
sum  &2444 &  2556 & 5000
\end{tabular}\\}
\begin{itemize}
\item
On trials where the 1st ball was black, how often was the 2nd white?\\
\pause
\item
A fraction $1689/2445$ of the time, or $\approx 0.69$.\\
\end{itemize}
\pause This is a conditional relative frequency.  If the number of
trials is large, this approximates a conditional probability.
\end{frame}

\begin{frame}
\renewcommand{\tabcolsep}{4pt}
The results of 20,000 throws with two dice (Wolf 1850, cited in Bulmer
1967)
\begin{center}
\small
\renewcommand{\arraystretch}{0.85}
\begin{tabular}{rrrrrrrrr}
    & \multicolumn{6}{c}{White}&  &\\ \cline{2-7}
 Black & 1 & 2 & 3 & 4 & 5 & 6 & $\sum$ &$f$\\ \hline
    1     & 547 & 587 & 500 & 462 & 621 & 690 & 3407 & .170\\
    2     & 609 & 655 & 497 & 535 & 651 & \textbf{684}
                                        & \textbf{3631} & .182\\
    3     & 514 & 540 & 468 & 438 & 587 & 629 & 3176 & .159\\
    4     & 462 & 507 & 414 & 413 & 509 & 611 & 2916 & .146\\
    5     & 551 & 562 & 499 & 506 & 658 & 672 & 3448 & .172\\
    6     & 563 & 598 & 519 & 487 & 609 & 646 & 3422 & .171\\ \hline
$\sum$:     & 3246 & 3449 & 2897 & 2841 & 3635 & 3932 & 20000 & 1.000\\
$f$: & .162 & .172 & .145 & .142 & .182 & .197 & 1.000\\
\end{tabular}
\end{center}
\begin{itemize}
\item<1->What is the conditional frequency of $W6$ given $B2$?
\item<2->$684/3631 \approx 0.188$
\end{itemize}
\end{frame}

\begin{frame}
\frametitle{Product rule for relative frequencies}
How often did Kerrich get $B1$ and $W2$?
\begin{center}
\begin{tabular}{lrrr}
First  & \multicolumn{2}{c}{Second ball}\\ \cline{2-3}
ball & Black & White & sum\\ \hline
Black  & 756 &\bf \fbox{1689} & 2445 \\
White&1688 &   867 & 2555 \\ \hline
sum  &2444 &  2556 & \bf 5000
\end{tabular}
\end{center}
\pause
A fraction $1689/5000$ of the time.
\only<3>{%
\[
\frac{1689}{5000} = \frac{1689}{2445} \times \frac{2445}{5000}
\]
}
\only<4->{%
\[
\overbrace{\frac{1689}{5000}}^{f(B1 \AND W2)}
=
\overbrace{\frac{1689}{2545}}^{f(W2|B1)}
\times
\overbrace{\frac{2445}{5000}}^{f(B1)}
\]
\onslide+<5->{As $N$ increases, relative frequencies $(f)$ become
probabilities.}
}
\end{frame}

\begin{frame}
\frametitle{Product rule}
The probability of $A$ and $B$ is
\[
\Pr[A \AND B] = \Pr[B|A] \Pr[A]
\]
\pause
This is why we multiply along the branches of a tree diagram.
\end{frame}

\begin{frame}
\frametitle{Statistical independence: sampling w/ replacement}
\begin{center}
\let\put\pictexput
\framebox{\beginpicture
\setcoordinatesystem units <0.94cm, 0.6cm>
\setplotarea x from 0 to 7, y from -3 to 4
\plot 0 0 2 1.7 /
\plot 0 0 2 -1.7 /
\put {$1/2$} [br] <-1pt,1pt> at 1 0.85
\put {$1/2$} [tr] <-1pt,-1pt> at 1 -0.85
\put {black}   [c] at 2.5 2
\put {white} [c] at 2.5 -2
%
\plot 3 2 5 3 /
\plot 3 2 5 1 /
\put {$1/2$} [br] <-1pt,1pt> at 4 2.5
\put {$1/2$} [tr] <-1pt,-1pt> at 4 1.5
\put {black}   [l] <3pt,0pt> at 5 3
\put {white} [l] <3pt,0pt> at 5 1
%
\plot 3 -2  5 -1 /
\plot 3 -2  5 -3 /
\put {$1/2$} [br] <-1pt,1pt> at 4 -1.5
\put {$1/2$} [tr] <-1pt,-1pt> at 4 -2.5
\put {black}   [l] <3pt,0pt> at 5 -1
\put {white} [l] <3pt,0pt> at 5 -3
%
\put {\lines{First\cr ball}}  [b] at 2.5 3.7
\put {\lines{Second\cr ball}} [b] at 5.5 3.7
\put {Event} [b] at 6.8 3.7
\put {$BB$} <0pt,-1pt> at 6.8 3
\put {$BW$} <0pt,0pt> at 6.8 1
\put {$WB$} <0pt,-1pt> at 6.8 -1
\put {$WW$} <0pt,0pt> at 6.8 -3
%
\put {Prob} [b] at 8 3.7
\put {$1/4$} <0pt,-1pt> at 8 3
\put {$1/4$} <0pt,0pt> at 8 1
\put {$1/4$} <0pt,-1pt> at 8 -1
\put {$1/4$} <0pt,0pt> at 8 -3
\endpicture}
\let\put\latextexput
\\
\bigskip
$\Pr[W_2 | B_1] = \Pr[W_2 | W_1] = \Pr[W_2] = 1/2$
\end{center}
\end{frame}

\begin{frame}
\frametitle{Sampling with replacement: model versus data}
\begin{center}
\begin{tabular}{ccc}
        &            & Observed\\
        & Theoretical& relative  \\
  Event & probability& frequency \\
\hline
\hline
$BB$    & 0.25& 0.254 \\
$BW$    & 0.25& 0.255 \\
$WB$    & 0.25& 0.252 \\
$WW$    & 0.25& 0.239 \\
\hline
\end{tabular}\\
Data from computer simulation
\end{center}
Theory and observation are not identical,
but they are very close.
\end{frame}

\begin{frame}
\renewcommand{\tabcolsep}{4pt}
\frametitle{Sum rule: $\Pr[\hbox{black 4 or white 5 (or both)}]$}
\begin{center}
\small
\renewcommand{\arraystretch}{0.85}
\begin{tabular}{rrrrrrrr}
    & \multicolumn{6}{c}{White}&\\ \cline{2-7}
 Black & 1 & 2 & 3 & 4 & 5 & 6 & $\sum$\\ \hline
    1     & 547 & 587 & 500 & 462 & \bf 621 & 690 & 3407\\
    2     & 609 & 655 & 497 & 535 & \bf 651 & 684 & 3631\\
    3     & 514 & 540 & 468 & 438 & \bf 587 & 629 & 3176\\
    4     & \bf 462 & \bf 507 & \bf 414 & \bf 413 & \fbox{\bf 509}
                                              & \bf 611 & 2916\\
    5     & 551 & 562 & 499 & 506 & \bf 658 & 672 & 3448\\
    6     & 563 & 598 & 519 & 487 & \bf 609 & 646 & 3422\\ \hline
$\sum$:     & 3246 & 3449 & 2897 & 2841 & 3635 & 3932 & \it 20000\\
\end{tabular}
\end{center}
Relative frequency is the sum of the bold-face values
divided by 20,000.
\pause
\[
f[b4 \OR w5] =
\overbrace{\frac{2916}{20000}}^{f[b4]}
+ \overbrace{\frac{3635}{20000}}^{f[w5]}
- \overbrace{\frac{509}{20000}}^{f[b4 \AND w5]}
\]
\end{frame}

\begin{frame}
\frametitle{Sum rule for probabilities}
\[
\Pr[A \OR B] = \Pr[A] + \Pr[B] - \Pr[A \AND B]
\]
\end{frame}

\begin{frame}
\renewcommand{\tabcolsep}{4pt}
\frametitle{Sum rule again: $\Pr[\hbox{white 3 or white 5}]$}
For mutually exclusive events, there is nothing to subtract.
\begin{center}
\small
\renewcommand{\arraystretch}{0.85}
\begin{tabular}{rrrrrrrr}
    & \multicolumn{6}{c}{White}&\\ \cline{2-7}
 Black & 1 & 2 & 3 & 4 & 5 & 6 & $\sum$\\ \hline
  1  & 547 & 587 & \bf 500 & 462 & \bf 621 & 690 & 3407\\
  2  & 609 & 655 & \bf 497 & 535 & \bf 651 & 684 & 3631\\
  3  & 514 & 540 & \bf 468 & 438 & \bf 587 & 629 & 3176\\
  4  & 462 & 507 & \bf 414 & 413 & \bf 509 & 611 & 2916\\
  5  & 551 & 562 & \bf 499 & 506 & \bf 658 & 672 & 3448\\
  6  & 563 & 598 & \bf 519 & 487 & \bf 609 & 646 & 3422\\ \hline
$\sum$:& 3246 & 3449 & 2897 & 2841 & 3635 & 3932 & \it 20000\\
\end{tabular}\\
\end{center}
What is rel.\ frq.\ of white 3 or white 5?%: 2897/20000 + 3635/20000
\pause
\[
f[w4 \OR w5] =
\overbrace{\frac{2897}{20000}}^{f[w4]}
+ \overbrace{\frac{3635}{20000}}^{f[w5]}
\]
\end{frame}

\begin{frame}
\frametitle{Bayes's rule}
\textbf{Problem:} Our emphasis has been on the probability of an
outcome given a hypothesis. But we often want to know the probability
of the hypothesis, given the outcome.

\bigskip

\textbf{Example:} The probability the patient is sick given a positive
result on some test.

\bigskip

Suppose that 0.1\% of people have some disease. When tested for the
disease 99\% of sick people test positive, but so do 1\% of well
people. What fraction of those with positive results are really sick?
\end{frame}

\begin{frame}
\frametitle{Bayes's rule in terms of counts}
{\centering%-*-latex-*-
\framebox{\beginpicture
\setcoordinatesystem units <0.94cm, 0.6cm>
\setplotarea x from 0 to 6, y from -3 to 3
\put {100,000} [r] <-1pt,1pt> at 0 0
\plot 0 0 1.9 1.7 /
\plot 0 0 1.9 -1.7 /
\put {99,900} [br] <-1pt,1pt> at 1.425 1.25
\put {100} [tr] <-1pt,-1pt> at 1.425 -1.25
\put {healthy}   [c] at 2.5 2
\put {sick} [c] at 2.5 -2
%
\plot 3.1 2 5 3 /
\plot 3.1 2 5 1 /
\put {999} [br] <-1pt,1pt> at 4.525 2.75
\put {98,901} [tr] <-1pt,-1pt> at 4.525 1.25
\put {positive}   [l] <3pt,0pt> at 5 3
\put {negative} [l] <3pt,0pt> at 5 1
%
\plot 3.1 -2  5 -1 /
\plot 3.1 -2  5 -3 /
\put {99} [br] <-1pt,1pt> at 4.525 -1.25
\put {1} [tr] <-1pt,-1pt> at 4.525 -2.75
\put {positive}   [l] <3pt,0pt> at 5 -1
\put {negative} [l] <3pt,0pt> at 5 -3
\endpicture}
\\}

\bigskip
What fraction of those who test positive are really sick?
\pause
\[
\frac{99}{99 + 999} \approx 0.09 \qquad \hbox{Fewer than 1 in 10!}
\]
\end{frame}

\begin{frame}
\frametitle{Bayes's rule in terms of probabilities}
Recall the multiplication law:
\[
\Pr[A\& B] = \Pr[B] \Pr[A|B] = \Pr[A] \Pr[B|A]
\]
Divide through by $\Pr[B]$:
\[
\Pr[A|B] = \frac{\Pr[A] \Pr[B|A]}{\Pr[B]} \quad\hbox{(Bayes's rule)}
\]
Allows us to calculate $\Pr[A|B]$ from $\Pr[B|A]$.
\end{frame}

\begin{frame}
\frametitle{Back to example}
\[
\Pr[A|B] = \frac{\Pr[A] \Pr[B|A]}{\Pr[B]} \quad\hbox{(Bayes's rule)}
\]
$A$: patient is sick. $\Pr[A] = 1/1000$.

\bigskip

$B$: patient tested positive. $\Pr[B] = (999+99)/100000 = 1098/100000$.

\bigskip

Pr[testing positive if sick] is $\Pr[B|A] = 99/100$.

\bigskip

Using Bayes's rule,
\[
\Pr[A|B] = \frac{1/1000 \times 99/100}{1098/100000}
 = \frac{99}{1098} \approx 0.09
\]
This is the same answer we got using counts.
\end{frame}

\begin{frame}
\frametitle{Summary}

Sum rule
\[
\Pr[A \OR B] = \Pr[A] + \Pr[B] - \Pr[A \AND B]\\
\]

\bigskip
Product rule
\[
\Pr[A \AND B] = \Pr[A] \Pr[B|A]
\]

\bigskip
Bayes's rule
\[
\Pr[A|B] = \frac{\Pr[A] \Pr[B|A]}{\Pr[B]}
\]
\end{frame}

\begin{frame}
\frametitle{Problems}
\begin{enumerate}
\item You toss a fair coin twice. What is the probability that the
number of heads is one? 
\item You toss two fair dice, one red and one black.  What is the
probability that you observe either a red 4 or a black 6 (or
both)?
\item Imagine a modified version of Kerrich's urn experiment in which each
trial begins with 3 balls of each color (red and black).  What is the
probability that, in a single trial, both of the balls drawn are red?
\end{enumerate}
\end{frame}
