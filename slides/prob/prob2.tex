\title{Random Variables, Expectations, Variance, and Covariance}
\author{Alan R. Rogers}
\date{\today}
\frame[label=tpage2]{\titlepage}

%\begin{frame}
%\frametitle{Less and more abstract ways to describe the mean.}
%\begin{eqnarray*}
%\hbox{data} &=& 0, 1, 0, 0, 1, 1, 1, 0, 0, 2\\
%\hbox{mean} &=& \frac{0+1+0+0+1+1+1+0+0+2}{10} = \frac{6}{10}\\[1ex]
%\pause
%\hbox{data} &=& x_1, x_2, x_3, x_4, x_5, x_6, x_7, x_8, x_9,x_{10}\\
%\bar X &=& \frac{x_1+x_2+x_3+x_4+x_5+x_6+x_7+x_8+x_9+x_{10}}{10}\\[1ex]
%\pause
%\hbox{data} &=& x_1, x_2, \ldots, x_N\\
%\bar X &=& \frac{1}{N}\sum_{i=1}^N x_i
%\end{eqnarray*}
%\end{frame}

\begin{frame}
\frametitle{Distributions of counts and of relative frequencies}
\[
\hbox{data} = [0, 1, 0, 0, 1, 1, 1, 0, 0, 2]
\]
\begin{columns}
\column<2->{0.5\textwidth}
\centering
\begin{tabular}{ccc}
       &      & Relative\\
Value  & Count&frequency\\
\hline
    0  &   5  & 0.5\\
    1  &   4  & 0.4\\
    2  &   1  & 0.1\\
\cline{2-3}
       &  10  & 1.0
\end{tabular}
\column{0.5\textwidth}
\onslide+<3->{%
\begin{block}{Histogram}
\mbox{\beginpicture
\setcoordinatesystem units <1cm,0.5cm>
\setplotarea x from 0 to 3, y from 0 to 5
\axis bottom invisible ticks withvalues 0 1 2 / at 0.5 1.5 2.5 / /
\axis right ticks withvalues 0 {0.1} {0.4} {0.5} / at 0 1 4 5 / /
\axis left ticks withvalues 0 {1} {4} {5} / at 0 1 4 5 / /
\sethistograms
\plot
0 0
1 5
2 4
3 1
/
\endpicture}
\end{block}}
\end{columns}
\end{frame}

\begin{frame}
\frametitle{The mean $(m)$}
\[
m = \frac{1}{N}\sum_{i=1}^N x_i
\]
where $N$ is sample size and $x_i$ is $i$th data value.

\textbf{Example:} If $x = [3, 2, 2]$, then
\[
m = \frac{1}{3}\times (3 + 2 + 2) = 7/3
\]
\end{frame}

\begin{frame}
\frametitle{Using relative frequencies to calculate the mean}
\[
m = \sum_{x} x f_x
\]
where $x$ is a sample value and $f_x$ is the relative frequency of
that value.

\textbf{Example:} If $x = [3, 2, 2]$, then
\begin{eqnarray*}
f_2&=&2/3\\
f_3&=&1/3\\
m &=& (2 \times 2/3) + (3 \times 1/3) = 7/3
\end{eqnarray*}
\end{frame}

\begin{frame}
\frametitle{Larger example}
\[
\hbox{data} = [0, 1, 0, 0, 1, 1, 1, 0, 0, 2]
\]
\begin{columns}
\column{0.4\textwidth}
\centering
\begin{tabular}{cc}
       &  Relative\\
Value  & frequency\\
\hline
    0  &  0.5\\
    1  &  0.4\\
    2  &  0.1\\
\end{tabular}
\column{0.6\textwidth}
What is the mean?

\pause
\bigskip

\begin{eqnarray*}
m &=& 0\times 0.5 + 1 \times 0.4 + 2 \times 0.1\\
  &=& 0.4 + 0.2 = 0.6
\end{eqnarray*}
\end{columns}
\end{frame}

\begin{frame}
\frametitle{Measures of variation}
\begin{itemize}
\item range of data
\item interquartile range: range of middle half of data
\item variance: average of $(x-m)^2$, where $m$ is the mean
\item square root of variance: the standard deviation
\end{itemize}
In population genetics, the variance is most useful.
\end{frame}

\begin{frame}
\frametitle{Calculating the variance ($v$)}
\[
v = \sum_{x} (x-m)^2 f_x
\]
where $m$ is the mean, $x$ is a sample value, and $f_x$ is the
relative frequency of that value.

\bigskip
What are the mean and variance of this data set: $[3, 2, 2]$?
\end{frame}

\begin{frame}
\frametitle{Calculations}
\begin{columns}
\column{0.5\textwidth}
Frequency distribution:
{\centering\begin{tabular}{cc}
       &  Relative\\
Value  & frequency\\
\hline
    2  &  2/3\\
    3  &  1/3\\
\end{tabular}\\}
\pause
Mean:
\begin{eqnarray*}
m &=& 2 \times \frac{2}{3} + 3\times \frac{1}{3}\\
  &=& 7/3 \approx 2.33
\end{eqnarray*}
\column{0.5\textwidth}
\pause
Variance:
\begin{eqnarray*}
V &=& (2-2.33)^2 \times \frac{2}{3}\\
  && \mbox{} + (3 - 2.33)^2 \times \frac{1}{3}\\
  &=& 0.22
\end{eqnarray*}
\end{columns}
\end{frame}

\begin{frame}
These ideas work not only for relative frequencies but also for
probabilities.
\begin{itemize}
\item<2->
Frequency distributions become probability distributions.
\item<3->
Means become expected values.
\item<4->
Nothing else changes.
\end{itemize}
\end{frame}

\begin{frame}[label=probdist]
\frametitle{Probability distribution}
\begin{itemize}
\item Assigns a probability to every event.
\item When events have numeric values, the probability distribution
  translates one number (the event) into another (the probability).
\item A set of events with associated probabilities is a \emph{random
variable} (r.v.). 
\item Distributions of numerical r.v.s are often described using
mathematical functions.
\end{itemize}
\end{frame}

\begin{frame}
  A \textbf{random variable} is a variable whose values occur with
  particular probabilities.

\small
  (We would need to modify this slightly for variables that vary along a
  continuum, such as height or weight.  But I'm going to ignore that
  distinction here.)
\end{frame}

\begin{frame}
\frametitle{Example 1: a fair coin}
Suppose that $X$ (a random variable) is the number of heads in one toss of
a fair coin.  The \emph{probability distribution} of $X$ is
\begin{center}
\begin{tabular}{cc}
    & Probability\\
$X$ & $(p_X)$\\ \hline
0   & 1/2\\
1   & 1/2\\
\hline
\end{tabular}
\end{center}
Probabilities
\begin{itemize}
\item lie between 0 and 1,
\item sum to 1.
\end{itemize}
\end{frame}

\begin{frame}
\frametitle{Problem}

In the previous slide, $X$ was the number (either 0 or 1) of heads in
one toss of a fair coin. What is the probability distribution of $Y =
X^2$? 
\end{frame}


\begin{frame}
\frametitle{Example 2: a loaded die}
Let $X$ be the number obtained on a roll of the die.  This die is
``loaded,'' so that 1s and 2s are twice as probable as other
values.

\centering
\begin{tabular}{cc}
$X$ & $(p_X)$\\ \hline
1   & 0.250\\
2   & 0.250\\
3   & 0.125\\
4   & 0.125\\
5   & 0.125\\
6   & 0.125\\
\hline
 & 1.0000
\end{tabular}\\
\end{frame}

\begin{frame}
\frametitle{The mean (or expectation) of a random variable}

The mean of $X$ is written $E(X)$ and equals
\[
E(X) = \sum_i p_i x_i
\]
where $x_i$ is the $i$th value that $X$ can take, and $p_i$ is its
probability.   

If $X$ is the number obtained on a roll of our loaded die, then
\begin{eqnarray*}
E[X] &=& 1 \times 0.25 + 2 \times 0.25 + 3 \times 0.125 \\
&&\mbox{}
+ 4 \times 0.125 + 5 \times 0.125 + 6\times 0.125\\
&=& 3
\end{eqnarray*}

\bigskip
The same as an average, except that $p_i$ is a probability rather than
a relative frequency.
\end{frame}

\begin{frame}{Allele frequency as expectation}
\begin{columns}
\column{0.5\textwidth}
\begin{tabular}{ccc} 
       &      & Cond.\\
       & G'type&allele\\
G'type & freq  & freq\\
\hline
$A_1A_1$ & $P_{11}$ & 1\\
$A_1A_2$ & $P_{12}$ & 0.5\\
$A_2A_2$ & $P_{22}$ & 0\\
\end{tabular}
\column{0.5\textwidth}
\textcolor{blue}{Allele frequency}
\begin{eqnarray*}
p_1 &=& 1 \times P_{11}\\
    && \mbox{} + 0.5 \times P_{12}\\
    && \mbox{} + 0 \times P_{22}
\end{eqnarray*}
\end{columns}
\end{frame}

\begin{frame}
\frametitle{The variance}
If $\mu$ is the mean of $X$, then its variance is
\begin{equation}
V[X] = E[(X - \mu)^2]
\label{eq.Vdef}
\end{equation}
For our loaded die, the mean was $\mu = 3$.  The variance is

\begin{eqnarray*}
V[X] &=& (1-3)^2 \times 0.25 \\
&& \mbox{}
+ (2-3)^2 \times 0.25\\
&& \mbox{}
 + (3-3)^2 \times 0.125\\
&& \mbox{}
 + (4-3)^2 \times 0.125\\
&&\mbox{}
 + (5-3)^2 \times 0.125\\
&& \mbox{}
 + (6-3)^2\times 0.125\\
&=& 3
\end{eqnarray*}
\end{frame}

\begin{frame}
\frametitle{A single toss of an unfair coin}
The probability of ``heads'' is an unknown value $p$.

Your winnings: $X=1$ for heads and $X=0$ for tails. What's the
probability distribution of $X$? The mean? The variance?
\end{frame}

\begin{frame}
\frametitle{Properties of expectations}
If $X$ and $Y$ are random variables and $a$ is a constant,
\begin{eqnarray}
E[a] &=& a\label{eq.E[a]}\\
E[aX] &=& aE[X]\label{eq.E[aX]}\\
E[X+Y] &=& E[X] + E[Y]\label{eq.E[X+Y]}
\end{eqnarray}
See JEPr for details.
\end{frame}

\begin{frame}
\frametitle{Using rules of expectations to re-express the variance}
Let $\mu = E[X]$. The variance of $X$ is
\begin{align}
\onslide<1->{%
V &= E[(X-\mu)^2] & \hbox{(by Eqn.~\ref{eq.Vdef})}\notag\\
  &= E[ X^2 - 2\mu X + \mu^2]\notag\\}
\onslide<2->{%
 &= E[X^2] - E[2\mu X] + E[\mu^2] & \hbox{(by Eqn.~\ref{eq.E[X+Y]})}\notag\\
  &= E[X^2] - E[2\mu X] + \mu^2 & \hbox{(by Eqn.~\ref{eq.E[a]})}\notag\\}
\onslide<3->{%
  &= E[X^2] - 2\mu E[X] + \mu^2 & \hbox{(by
    Eqn.~\ref{eq.E[aX]})}\notag\\}
\onslide<4->{%
  &= E[X^2] - 2\mu^2 + \mu^2 & \hbox{(by definition of $\mu$)}\notag\\
  &= E[X^2] - \mu^2\label{eq.V2}}
\end{align}
\end{frame}

\begin{frame}
\frametitle{Variance of our loaded die}
A moment ago, we found for our loaded die that $E[X] = V[X] = 3$. Let
us recalculate this using Eqn.~\ref{eq.V2}. We need

\begin{align*}
E[X^2] &= 1^2 \times 0.25 + 2^2 \times 0.25\\
& \quad + 3^2 \times 0.125 + 4^2 \times 0.125\\
& \quad + 5^2 \times 0.125 + 6^2\times 0.125\\
&= 12
\end{align*}
\[
V = E[X^2] - \mu^2 = 12 - 3^2 = 3
\]
\end{frame}

\begin{frame}
\frametitle{Association between variables}

{\centering\setlength{\headingtoplotskip}{1.5ex}
\mbox{\beginpicture
%%%%%%%%%% Plot 1 %%%%%%%%%%%%%%%%%%%%%%%%%%%%%%%%%%%%%%%%%%%%
  \setcoordinatesystem units <0.2056\columnwidth,0.2056\columnwidth> 
      point at 1.25 0
  \setplotarea x from -0.5 to 0.57,
               y from -0.5 to 0.57
  \axis bottom label {$X$} /
  \axis left label {$Y$} /
  \axis right /
  \axis top /
\plotheading{\lines{Weak\cr positive\cr relationship}}
\multiput {$\bullet$} at 
     0.202     -0.073
     0.281     -0.150
     0.079      0.109
    -0.423     -0.145
     0.018     -0.326
    -0.323      0.144
    -0.072     -0.220
     0.009      0.128
     0.078      0.054
     0.351      0.192
     0.200     -0.015
    -0.119     -0.387
     0.109      0.290
     0.114      0.360
     0.337      0.355
    -0.061     -0.097
     0.072      0.074
     0.202      0.070
    -0.187     -0.239
     0.390      0.504
/
%%%%%%%%%% Plot 2 %%%%%%%%%%%%%%%%%%%%%%%%%%%%%%%%%%%%%%%%%%%%
  \setcoordinatesystem units <0.22\columnwidth,0.22\columnwidth> 
      point at -0.25 0
  \setplotarea x from -0.5 to 0.5,
               y from -0.5 to 0.5
  \axis bottom label {$X$} /
  \axis right label {$Y$} /
  \axis left /
  \axis top /
\plotheading{\lines{Strong\cr negative\cr relationship}}
\multiput {$\bullet$} at 
     0.430     -0.260
     0.372     -0.378
     0.301     -0.145
     0.217     -0.155
     0.185     -0.178
    -0.119     -0.206
    -0.374      0.102
    -0.158      0.055
    -0.203      0.077
    -0.077      0.133
    -0.036      0.310
     0.113     -0.177
     0.145     -0.006
    -0.267      0.141
     0.261     -0.120
    -0.173      0.250
    -0.203     -0.027
    -0.035     -0.068
     0.007      0.270
    -0.015      0.056     
/
\endpicture}
\\[1ex] 
Positive and negative relationships between
variables.\\}
\label{fig.scatplot}
\end{frame}

\begin{frame}
\frametitle{Covariance: a measure of association}

\begin{align*}
C(X,Y) &= \sum_{x,y} (x - E[X])(y-E[Y]) P_{x,y}\\
       &= E\Bigl[(X - E[X])(Y-E[Y])\Bigr]\\
       &= E[XY] - E[X] E[Y]
\end{align*}
When $X$ and $Y$ are independent, $C(X,Y)=0$.
\end{frame}

\begin{frame}
\frametitle{A bivariate probability distribution}
\begin{columns}
\column{0.7\textwidth}
\fbox{\begin{tabular}{cccc}
$X$ & $Y$ & $P_{X,Y}$ & $(X-E[X])(Y-E[Y])$\\
\hline
0 & 0 & 0.4 & +0.25\\
0 & 1 & 0.1 & --0.25\\
1 & 0 & 0.1 & --0.25\\
1 & 1 & 0.4 & +0.25\\
\end{tabular}}\\[1ex]
Note: the $P_{X,Y}$ column lists the probabilities of the $(X,Y)$
pairs. In column~4, $E[X] = E[Y] = 0.5$.\\[0.3\textheight]

\mbox{}
\column{0.3\textwidth}
\mbox{}\\[0.5\textheight]
\mbox{\beginpicture
  \setcoordinatesystem units <0.7\textwidth,0.7\textwidth> 
  \setplotarea x from 0 to 1, y from 0 to 1
  \axis bottom label {$X$} /
  \axis left label {$Y$} /
  \axis right /
  \axis top /
\setdashes
\putrule from 0 0.5 to 1 0.5
\putrule from 0.5 0 to 0.5 1
\setsolid
\multiput {$\bullet$} at 
0 0
0.07 0
0 0.07
0.07 0.07
1 1
0.93 1
1 0.93
0.93 0.93
0 1
1 0
/
\endpicture}
\end{columns}
\end{frame}

\begin{frame}
\frametitle{Numerical value of covariance in previous slide}
\begin{align*}
C(X,Y) &= 0.4 \times 0.25\\
 & \quad - 0.1 \times 0.25\\
 & \quad - 0.1 \times 0.25\\
 & \quad + 0.4 \times 0.25\\
 & = 0.15
\end{align*}
\end{frame}

\begin{frame}
\frametitle{Problem}
In Kerrich's urn experiment, suppose you get \$1 for each red ball and
\$0 for each green one, and let $X$ and $Y$ represent the dollars you
receive on the two draws within a single trial of the experiment.

Write down the probability distribution of $X$ and $Y$ in tabular
form. Your table should have columns for $X$, for $Y$, and for the
joint probability of $X$ and $Y$, i.e. $\Pr[X,Y]$.
\end{frame}

\begin{frame}
This is exactly like Fig.~2 of JEPr, which presents the following
    probability distribution:
\begin{center}
\begin{tabular}{cc}
Event & Prob\\ \hline
RR & 1/6\\
RG & 1/3\\
GR & 1/3\\
GG & 1/6\\
\end{tabular}
\end{center}
where ``R'' and ``G'' stand for ``red'' and ``green''. Now, ``R''
becomes ``1,'' ``G'' becomes``0,'' and the probability distribution
becomes
\begin{center}
\begin{tabular}{ccc}
$X$ & $Y$ & $\Pr(X,Y)$\\ \hline
1 & 1 & 1/6\\
1 & 0 & 1/3\\
0 & 1 & 1/3\\
0 & 0 & 1/6\\
\end{tabular}
\end{center}
\end{frame}
