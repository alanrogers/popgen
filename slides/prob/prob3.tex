\title{Probability Distributions}
\author{Alan R. Rogers}
\date{\today}

\frame{\titlepage}

\begin{frame}
\frametitle{Probability distributions}
A probability distribution is a function.
\begin{description}
\item[Input] event
\item[Output] probability of event
\end{description}
\begin{itemize}
\item So far we have described probability distributions using
tables.
\item When events are numbers, distributions can be expressed as
mathematical functions.
\end{itemize}
\end{frame}

\begin{frame}
\frametitle{The Urn Metaphor}

Imagine two urns: metaphors for a population in two successive
generations.  Urn~1 has 50 balls, some red, some white, representing
parental gene copies.  Urn~2 is empty until urn~1 has ``reproduced''
as follows:

\begin{enumerate}
\item Examine a random ball from urn~1.
\item Put a ball of the same color into urn~2.
\item Replace the ball from urn~1.
\item Repeat until there are 50 balls in urn~2.
\end{enumerate}
The number of red balls in urn~2 is likely to differ from that in urn~1,
because of random sampling.  This metaphor is used as a model of genetic
drift.
\end{frame}

\begin{frame}
\frametitle{Binomial random variable}
In probability theory, the number of red balls in urn~2 is
a \emph{binomial random variable}.
\begin{enumerate}
\item Balls drawn from the urn are statistically independent.
\item Each ball is red with probability $p$, the fraction of red balls
in urn~1.
\end{enumerate}
This distribution has two parameters: $N$, the number of balls put
into urn~2, and $p$, the probability of ``red'' each time a ball is
drawn. 
\end{frame}

\begin{frame}
\frametitle{Probability of HT}
Consider tosses of an unfair coin, for which the probability of
``heads'' is $p$ and that of ``tails'' is $q = 1-p$. Assume that the
tosses are statistically independent.
\begin{tabbing}
ProbabilityXX\=\kill
Experiment \> Toss a coin 2 times.\\
Result     \> HT\\
Probability\> $pq$\\
\end{tabbing}

\bigskip
This is an event of form $\Pr[A \& B]$, where $A$ is the event that
the first toss is H and $B$ is the event that the 2nd is T. By
assumption, $\Pr[A] = p$ and $\Pr[B] = q$. The tosses are
statistically independent, so
\[
\Pr[A \& B] = pq
\]
by the multiplication law of probability.
\end{frame}

\begin{frame}
\frametitle{Probability of HHT}
\begin{tabbing}
ProbabilityXX\=\kill
Experiment \> Toss a coin 3 times.\\
Result     \> HHT\\
Probability\> $p^2q$\\
\end{tabbing}
\end{frame}

\begin{frame}
\frametitle{Probability of 2 heads in 3 tosses}
There are 3 ways to get 2 heads in 3 tosses:\\
{\centering\begin{tabular}{cc}
Event & Probability\\
\hline
THH & $p^2q$\\
HTH & $p^2q$\\
HHT & $p^2q$
\end{tabular}\\}

\bigskip
The probability of 2 heads in 3 tosses is
\begin{align*}
P_2 &= 3 p^2q\\
    &= \binom{3}{2} p^2q
\end{align*}
where $\binom{3}{2}$ is pronounced ``3 choose 2'' and means the number
of ways to choose 2 items out of a collection of 3.
\end{frame}

\begin{frame}
\frametitle{Binomial distribution}
The probability of $x$ heads in $K$ tosses is
\[
P_X = \binom{K}{x} p^x q^{K-x}
\]
\begin{align*}
E[X] &= pK &\hbox{mean}\\
V[X] &= Kpq &\hbox{variance}
\end{align*} 
\end{frame}

\begin{frame}
\frametitle{Poisson distribution}

Consider the lineage that connects me to an ancestor who lived $t$
generations ago.  The expected number of mutations along that lineage
is $\lambda = ut$, where $u$ is the mutation rate per generation. 
The number of mutations is a random variable (r.v.). If the mutation rate is
constant, then the distribution of this r.v.\ is \emph{Poisson}.
\end{frame}

\begin{frame}
\frametitle{Poisson distribution function}
If $X$ is a Poisson-distributed r.v.\ with mean $\lambda$, then $X$
takes value $x$ with probability
\[
P_x = \frac{\lambda^x e^{-\lambda}}{x!}
\]
where $e$ is the base of natural logarithms and $x!$ is ``$x$
factorial,'' or $x \cdot (x-1) \cdot (x-2) \cdots 1$.

\bigskip
Mean equals variance.
\[
E[X] = V[X] = \lambda 
\]

\bigskip
What is $P_0$? (Hint: $0!=1$ and $\lambda^0 = 1$.)

\bigskip
\pause
\[P_0 = e^{-\lambda}\]
\end{frame}

\begin{frame}
\frametitle{Poisson distribution}
{\centering
\mbox{\beginpicture
\setcoordinatesystem units <0.04in, 5in>
\setplotarea x from 0 to 21, y from 0 to 0.21
\axis left shiftedto x=-1 label {$\Pr[x]$}
  ticks  numbered from 0.0 to 0.2 by 0.1 /
\axis bottom label {$x$}
  ticks withvalues 0 10 20 / at 0.5 10.5 20.5 / /
%\shaderectangleson 
\sethistograms
\plot 
% Poisson distribution, lambda=2
0 0
1 .1353352832
2 .2706705664
3 .2706705664
4 .1804470442
5 .09022352214
6 .03608940886
7 .01202980295
8 .003437086558
9 .0008592716393
10 .0001909492532
11 .00003818985064
12 .000006943609207
13 .000001157268201
14 .0000001780412617
15 .00000002543446596
16 .000000003391262128
17 .0000000004239077661
18 .00000000004987150188
19 .000000000005541277987
20 .0000000000005832924196
21 .00000000000005832924196
/
\endpicture}~~\let\put\pictexput
\mbox{\beginpicture
\setcoordinatesystem units <0.04in, 5in>
\setplotarea x from 0 to 21, y from 0 to 0.21
%\axis right shiftedto x=22 
%  ticks  numbered from 0.0 to 0.2 by 0.1 /
\axis bottom label {$x$}
  ticks withvalues 0 10 20 / at 0.5 10.5 20.5 / /
\plotheading{$\lambda=10$}
%\shaderectangleson 
\sethistograms
\plot 
% Poisson distribution, lambda=10
0 0
1 .00004539992976
2 .0004539992976
3 .002269996488
4 .007566654962
5 .01891663740
6 .03783327480
7 .06305545801
8 .09007922571
9 .1125990321
10 .1251100357
11 .1251100357
12 .1137363961
13 .09478033010
14 .07290794623
15 .05207710445
16 .03471806963
17 .02169879352
18 .01276399619
19 .007091108993
20 .003732162628
21 .001866081314
/
\endpicture}
\let\put\latexput
\\[1ex]
Poisson distribution functions\\}
\end{frame}

\begin{frame}
Mutation rates at autosomal nucleotide sites are roughly $10^{-9}$ per
year. Consider a nucleotide in you. If you could trace its ancestry
back across the last $10^9$ years, what is the probability that you
would find no mutations?

\bigskip
\pause
The expected number of mutations is $\lambda = ut$, where $u=10^{-9}$
and $t = 10^9$. Thus, $\lambda=1$. The probability of no mutations is
\[
e^{-1} \approx 0.37
\]
\end{frame}

\begin{frame}[containsverbatim]
\frametitle{Raisin data}
\begin{verbatim}
Date: Aug 28, 2009
N=41, Mean=21.756098, Var=26.239024, Max=33
 1- 3: *
 4- 6: *
 7- 9:  *
10-12: -*
13-15: --*
16-18: -----*-
19-21: ----------  *
22-24: -------*--
25-27: ------- *
28-30: --- *
31-33: *

Key: ---- Poisson distribution w/ mean 21.756098
        * Data
\end{verbatim}
\end{frame}

\begin{frame}[containsverbatim]
\frametitle{Raisin data}
\begin{verbatim}
Date: Sept 6, 2013
N=32, Mean=20.375000, Var=15.080645, Max=34
 1- 3: *
 4- 6: *
 7- 9: *
10-12: *
13-15: --*
16-18: -------  *
19-21: --------*
22-24: -------   *
25-27: --*-
28-30: *
31-33: *

Key: ---- Poisson distribution w/ mean 20.375000
        * Data
\end{verbatim}
\end{frame}

\begin{frame}[containsverbatim]
\frametitle{Raisin data}
\begin{verbatim}
Date: Sept 6, 2017
N=36, Mean=14.111111, Var=13.473016, Max=21
 1- 3: *
 4- 6: *
 7- 9: ---  *
10-12: --------*
13-15: --------*--
16-18: --------  *
19-21: --- *

Key: ---- Poisson distribution w/ mean 14.111111
        * Data
\end{verbatim}
\end{frame}

\begin{frame}
  \frametitle{Homework problem 1.41}
Imagine an urn with $N$ balls, of which 1 is red and the rest are
black. You draw 2 balls from the urn at random \emph{without}
replacement.  Let $X=1$ if the first ball is red and $X=0$
otherwise. Define $Y$ similarly for the second ball.

\bigskip

What is the covariance of $X$ and $Y$?
\end{frame}

\begin{frame}
  \frametitle{Probability tree \& distribution of $(X,Y)$}
  \centering
    %-*-latex-*-
\let\put\pictexput
\framebox{\beginpicture
\setcoordinatesystem units <0.85cm, 0.6cm>
\setplotarea x from 0 to 7.5, y from -3 to 4
\plot 0 0 2 1.7 /
\plot 0 0 2 -1.7 /
\put {$1/N$} [br] <-1pt,1pt> at 1 0.85
\put {$(N-1)/N$} [tr] <-1pt,-1pt> at 1 -0.85
\put {R}   [c] at 2.25 2
\put {G} [c] at 2.25 -2
%
\plot 2.5 2 4.5 3 /
\plot 2.5 2 4.5 1 /
\put {0} [br] <-1pt,1pt> at 4 2.75
\put {1} [tr] <-1pt,-1pt> at 4 1.25
\put {R}   [l] <3pt,0pt> at 4.5 3
\put {G} [l] <3pt,0pt> at 4.5 1
%
\plot 2.5 -2  4.5 -1 /
\plot 2.5 -2  4.5 -3 /
\put {$1/(N-1)$} [br] <-1pt,1pt> at 4 -1.25
\put {$(N-2)/(N-1)$} [tr] <-1pt,-1pt> at 4 -2.75
\put {R}   [l] <3pt,0pt> at 4.5 -1
\put {G} [l] <3pt,0pt> at 4.5 -3
%
\put {\lines{First\cr ball}}  [b] at 2.25 3.7
\put {\lines{Second\cr ball}} [b] at 4.75 3.7
\put {$XY$} [b] at 5.75 3.7
\put {$11$} <0pt,-1pt> at 5.75 3
\put {$10$} <0pt,0pt> at 5.75 1
\put {$01$} <0pt,-1pt> at 5.75 -1
\put {$00$} <0pt,0pt> at 5.75 -3
%
\put {Prob} [b] at 7 3.7
\put {0} <0pt,-1pt> at 7 3
\put {$1/N$} <0pt,0pt> at 7 1
\put {$1/N$} <0pt,-1pt> at 7 -1
\put {$(N-2)/N$} <0pt,0pt> at 7 -3
\endpicture}
\let\put\latexput
\\[3ex]
  Goal: calculate $C(X,Y) = E[XY] - E[X]E[Y]$
\end{frame}

\begin{frame}
  \frametitle{$E[X]$}
  \begin{columns}
    \column{0.3\textwidth}
    \[
    \begin{array}{ccc}
      X & Y & \Pr\\ \hline
      1 & 1 & 0\\
      1 & 0 & 1/N\\
      0 & 1 & 1/N\\
      0 & 0 & (N-2)/N
      \end{array}
    \]
    \column{0.7\textwidth}
    \begin{eqnarray*}
      E[X] &=& 1\times 0\\
      && \mbox{} + 1 \times 1/N \\
      && \mbox{} + 0\times 1/N\\
      && \mbox{} + 0\times (N-2)/N\\
      &=& 1/N
    \end{eqnarray*}
  \end{columns}
\end{frame}

\begin{frame}
  \frametitle{$E[Y]$}
  \begin{columns}
    \column{0.3\textwidth}
    \[
    \begin{array}{ccc}
      X & Y & \Pr\\ \hline
      1 & 1 & 0\\
      1 & 0 & 1/N\\
      0 & 1 & 1/N\\
      0 & 0 & (N-2)/N
      \end{array}
    \]
    \column{0.7\textwidth}
    \begin{eqnarray*}
      E[Y] &=& 1\times 0\\
      && \mbox{} + 0 \times 1/N \\
      && \mbox{} + 1\times 1/N\\
      && \mbox{} + 0\times (N-2)/N\\
      &=& 1/N
    \end{eqnarray*}
  \end{columns}
\end{frame}

\begin{frame}
  \frametitle{$E[XY]$}
  \begin{columns}
    \column{0.3\textwidth}
    \[
    \begin{array}{cccc}
      X & Y & XY & \Pr\\ \hline
      1 & 1 & 1 & 0\\
      1 & 0 & 0 & 1/N\\
      0 & 1 & 0 & 1/N\\
      0 & 0 & 0 & (N-2)/N
      \end{array}
    \]
    \column{0.7\textwidth}
    \begin{eqnarray*}
      E[XY] &=& 1\times 0\\
      && \mbox{} + 0 \times (\hbox{\small the rest})\\
      &=& 0
    \end{eqnarray*}
  \end{columns}

  \vspace{1cm}

  Covariance of $X$ and $Y$:
    \[
    \overbrace{E[XY]}^0 - \overbrace{E[X]}^{1/N}\overbrace{E[Y]}^{1/N}
    = - 1/N^2
    \]
\end{frame}
