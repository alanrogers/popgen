\begin{frame}
\frametitle{Three systems of vocabulary}
\begin{tabular}{p{2.2in}ccc}
                        &   1   & 2 & 3\\
Position on chromosome  & locus & locus & locus\\
Protein-coding locus    & gene  & gene & gene\\
Physical copy of DNA at locus  & gene  & allele &gene copy\\
One of several variants at a locus& allele & allele & allele\\
\end{tabular}

\bigskip

1 is classical usage, 2 is Gillespie's, and we try to keep to 3.
\end{frame}

\begin{frame}
\frametitle{Urns as metaphors for populations}

\begin{center}
\begin{tabular}{ll}
Metaphor & Reality\\
\hline
urn       & population\\
balls     & individual genes at some locus\\
ball drawn from urn & gene in a gamete\\
color of ball & allelic state\\
relative frequency of ``black'' & allele frequency\\
\hline
\end{tabular}
\end{center}
\begin{itemize}
\item To model a neutral locus, we draw balls at random.
\item Which definition of ``gene'' am I using?
\end{itemize}
\end{frame}
