% -*-latex-*-
%\documentclass[handout]{beamer} 
\documentclass[]{beamer} 
\usepackage{etex}
\usepackage{alltt}

\let\latexput\put
\usepackage{pictex}
\let\pictexput\put
\let\put\latexput

\newcommand{\G}{\mathcal{G}}
\newcommand{\Het}{\mathcal{H}}
\newcommand{\mutation}{$\cdot$} 
\newcommand{\OR}{\;\hbox{or}\;}
\newcommand{\AND}{\;\&\;}
\newdimen\offsety
\newdimen\yunit
\newdimen\xunit
\newdimen\thusfar
\newdimen\plotht
\newdimen\plotwd
\newdimen\plotsp
\title{Gene Genealogies}
\author{Alan R. Rogers}
\date{\today}
\begin{document}

\title{Gene Genealogies}
\author{Alan R. Rogers}
\date{\today}
\frame{\titlepage}

\begin{frame}
\begin{columns}
\column{0.57\textwidth}
% -*-latex-*-
%
%%%%%%%%%%%%%% Tree in PicTeX Format %%%%%%%%%%%%%%
%
%Sample size   = 50
%Mutation rate = 1
%     theta         mn        tau          K
%  100.0000     0.0000     8.0000          1
%    0.1000     0.0000        Inf          1
%maxmm=15.000000 maxtau=8.800000 maxtree=8.958047 maxx=15.000000
\newdimen\offsety
\newdimen\yunit
\newdimen\xunit
\newdimen\thusfar
\newdimen\plotht
\newdimen\plotwd
\newdimen\plotsp
\thusfar=0.000000in  % Keeps track of what's above
\plotht=1.400000in   % Height of each plot
\plotwd=4.000000in   % Width of each plot
\plotsp=0.500000in   % Spacing between plots
\begin{figure}
\begin{center}
\mbox{\beginpicture
\def\mutation{\tiny$\bullet$}
\small
\valuestolabelleading=.4\baselineskip
\headingtoplotskip=0.4\baselineskip
%
%%%%%%%%%%%%%% Population size %%%%%%%%%%%%%%
%
\yunit=\plotht
\xunit=\plotwd
\Divide <\xunit> by <15.000000pt> forming <\xunit>
\Divide <\yunit> by <120.000008pt> forming <\yunit>
\advance\thusfar by \plotht
\advance\thusfar by \plotsp
\Divide <\thusfar> by <\yunit> forming <\offsety>
\placevalueinpts of <\offsety> in {\mktree_tmp}
\setcoordinatesystem units <\xunit, \yunit> point at 0 {\mktree_tmp}
\setplotarea x from 0.000000 to 15.000000, y from 0.000000 to 120.000008
\advance\thusfar by 0.030000\plotht
\axis left label {\lines{Population\cr size}} /
\axis bottom shiftedto y=-3.600000
    label {Mutational time before present}
    ticks numbered from 0 to 15 by 5 /
\putrule from 0.000000 100.000000 to 8.000000 100.000000
\putrule from 8.000000 100.000000 to 8.000000 0.100000
\putrule from 8.000000 0.100000 to 15.000000 0.100000
%
%%%%%%%%%%%%%% Gene Genealogy %%%%%%%%%%%%%%
%
\yunit=\plotht
\xunit=\plotwd
\Divide <\xunit> by <15.000000pt> forming <\xunit>
\Divide <\yunit> by <49.000000pt> forming <\yunit>
\advance\thusfar by \plotht
\advance\thusfar by \plotsp
\Divide <\thusfar> by <\yunit> forming <\offsety>
\placevalueinpts of <\offsety> in {\mktree_tmp}
\setcoordinatesystem units <\xunit, \yunit> point at 0 {\mktree_tmp}
\setplotarea x from 0.000000 to 15.000000, y from 0.000000 to 49.000000
\axis left invisible label {\lines{Gene\cr genealogy}} /
\axis bottom invisible
     label {Mutational time before present} /
\putrule from 0.000000 0.000000 to 4.836251 0.000000
\put {\mutation} at 3.452781 0.000000
\put {\mutation} at 1.952839 0.000000
\put {\mutation} at 1.410723 0.000000
\putrule from 0.000000 1.000000 to 4.836251 1.000000
\put {\mutation} at 1.799608 1.000000
\putrule from 4.836251 1.000000 to 4.836251 0.000000
\putrule from 4.836251 0.500000 to 8.000854 0.500000
\put {\mutation} at 7.587026 0.500000
\put {\mutation} at 6.529379 0.500000
\put {\mutation} at 6.834705 0.500000
\putrule from 0.000000 2.000000 to 1.462832 2.000000
\put {\mutation} at 1.328440 2.000000
\put {\mutation} at 0.857991 2.000000
\putrule from 0.000000 3.000000 to 0.044562 3.000000
\putrule from 0.000000 4.000000 to 0.044562 4.000000
\putrule from 0.044562 4.000000 to 0.044562 3.000000
\putrule from 0.044562 3.500000 to 1.462832 3.500000
\put {\mutation} at 0.704485 3.500000
\putrule from 1.462832 3.500000 to 1.462832 2.000000
\putrule from 1.462832 2.750000 to 8.000854 2.750000
\put {\mutation} at 6.041394 2.750000
\put {\mutation} at 1.724101 2.750000
\put {\mutation} at 4.480876 2.750000
\put {\mutation} at 6.547650 2.750000
\put {\mutation} at 4.466406 2.750000
\putrule from 8.000854 2.750000 to 8.000854 0.500000
\putrule from 8.000854 1.625000 to 8.025000 1.625000
\putrule from 0.000000 5.000000 to 8.002358 5.000000
\put {\mutation} at 3.926434 5.000000
\put {\mutation} at 6.075176 5.000000
\put {\mutation} at 6.939764 5.000000
\put {\mutation} at 5.579136 5.000000
\putrule from 0.000000 6.000000 to 8.002358 6.000000
\put {\mutation} at 2.878085 6.000000
\put {\mutation} at 6.922358 6.000000
\put {\mutation} at 4.400364 6.000000
\putrule from 8.002358 6.000000 to 8.002358 5.000000
\putrule from 8.002358 5.500000 to 8.024999 5.500000
\putrule from 8.024999 5.500000 to 8.024999 1.625000
\putrule from 8.024999 3.562500 to 8.050393 3.562500
\putrule from 0.000000 7.000000 to 6.710411 7.000000
\putrule from 0.000000 8.000000 to 6.710411 8.000000
\put {\mutation} at 3.783755 8.000000
\putrule from 6.710411 8.000000 to 6.710411 7.000000
\putrule from 6.710411 7.500000 to 8.050393 7.500000
\put {\mutation} at 7.094459 7.500000
\putrule from 8.050393 7.500000 to 8.050393 3.562500
\putrule from 8.050393 5.531250 to 8.143679 5.531250
\putrule from 0.000000 9.000000 to 8.142600 9.000000
\put {\mutation} at 2.906895 9.000000
\putrule from 0.000000 10.000000 to 3.547134 10.000000
\put {\mutation} at 3.498341 10.000000
\putrule from 0.000000 11.000000 to 0.187752 11.000000
\putrule from 0.000000 12.000000 to 0.187752 12.000000
\putrule from 0.187752 12.000000 to 0.187752 11.000000
\putrule from 0.187752 11.500000 to 0.447217 11.500000
\putrule from 0.000000 13.000000 to 0.447217 13.000000
\putrule from 0.447217 13.000000 to 0.447217 11.500000
\putrule from 0.447217 12.250000 to 3.547134 12.250000
\put {\mutation} at 2.420029 12.250000
\putrule from 3.547134 12.250000 to 3.547134 10.000000
\putrule from 3.547134 11.125000 to 8.054446 11.125000
\put {\mutation} at 4.960887 11.125000
\put {\mutation} at 5.290615 11.125000
\put {\mutation} at 4.422930 11.125000
\put {\mutation} at 5.067940 11.125000
\putrule from 0.000000 14.000000 to 1.629022 14.000000
\putrule from 0.000000 15.000000 to 1.629022 15.000000
\put {\mutation} at 1.344186 15.000000
\putrule from 1.629022 15.000000 to 1.629022 14.000000
\putrule from 1.629022 14.500000 to 4.052386 14.500000
\put {\mutation} at 3.168806 14.500000
\put {\mutation} at 2.324489 14.500000
\put {\mutation} at 1.735080 14.500000
\putrule from 0.000000 16.000000 to 0.052486 16.000000
\putrule from 0.000000 17.000000 to 0.052486 17.000000
\putrule from 0.052486 17.000000 to 0.052486 16.000000
\putrule from 0.052486 16.500000 to 3.664456 16.500000
\put {\mutation} at 0.727443 16.500000
\putrule from 0.000000 18.000000 to 1.215439 18.000000
\putrule from 0.000000 19.000000 to 0.378657 19.000000
\put {\mutation} at 0.309676 19.000000
\putrule from 0.000000 20.000000 to 0.378657 20.000000
\putrule from 0.378657 20.000000 to 0.378657 19.000000
\putrule from 0.378657 19.500000 to 1.215439 19.500000
\putrule from 1.215439 19.500000 to 1.215439 18.000000
\putrule from 1.215439 18.750000 to 3.626270 18.750000
\put {\mutation} at 3.552026 18.750000
\putrule from 0.000000 21.000000 to 0.102288 21.000000
\putrule from 0.000000 22.000000 to 0.102288 22.000000
\putrule from 0.102288 22.000000 to 0.102288 21.000000
\putrule from 0.102288 21.500000 to 1.725883 21.500000
\put {\mutation} at 0.847011 21.500000
\putrule from 0.000000 23.000000 to 1.725883 23.000000
\putrule from 1.725883 23.000000 to 1.725883 21.500000
\putrule from 1.725883 22.250000 to 2.602657 22.250000
\putrule from 0.000000 24.000000 to 2.602657 24.000000
\put {\mutation} at 0.447969 24.000000
\putrule from 2.602657 24.000000 to 2.602657 22.250000
\putrule from 2.602657 23.125000 to 3.626270 23.125000
\put {\mutation} at 2.815562 23.125000
\putrule from 3.626270 23.125000 to 3.626270 18.750000
\putrule from 3.626270 20.937500 to 3.664456 20.937500
\putrule from 3.664456 20.937500 to 3.664456 16.500000
\putrule from 3.664456 18.718750 to 4.052386 18.718750
\putrule from 4.052386 18.718750 to 4.052386 14.500000
\putrule from 4.052386 16.609375 to 8.028200 16.609375
\put {\mutation} at 4.464513 16.609375
\put {\mutation} at 4.202519 16.609375
\putrule from 0.000000 25.000000 to 8.001245 25.000000
\put {\mutation} at 5.238211 25.000000
\put {\mutation} at 0.597891 25.000000
\put {\mutation} at 4.546607 25.000000
\put {\mutation} at 7.491570 25.000000
\putrule from 0.000000 26.000000 to 6.076888 26.000000
\put {\mutation} at 3.655453 26.000000
\put {\mutation} at 5.642963 26.000000
\put {\mutation} at 4.709704 26.000000
\put {\mutation} at 2.365931 26.000000
\put {\mutation} at 1.423714 26.000000
\putrule from 0.000000 27.000000 to 6.076888 27.000000
\put {\mutation} at 1.023258 27.000000
\put {\mutation} at 5.642505 27.000000
\put {\mutation} at 4.191251 27.000000
\putrule from 6.076888 27.000000 to 6.076888 26.000000
\putrule from 6.076888 26.500000 to 8.001245 26.500000
\put {\mutation} at 6.117506 26.500000
\put {\mutation} at 6.742373 26.500000
\putrule from 8.001245 26.500000 to 8.001245 25.000000
\putrule from 8.001245 25.750000 to 8.028199 25.750000
\putrule from 8.028199 25.750000 to 8.028199 16.609375
\putrule from 8.028199 21.179688 to 8.039419 21.179688
\putrule from 0.000000 28.000000 to 3.924574 28.000000
\put {\mutation} at 3.547138 28.000000
\put {\mutation} at 1.944984 28.000000
\put {\mutation} at 0.489562 28.000000
\put {\mutation} at 3.400585 28.000000
\put {\mutation} at 1.640684 28.000000
\putrule from 0.000000 29.000000 to 3.924574 29.000000
\put {\mutation} at 3.178097 29.000000
\put {\mutation} at 0.166115 29.000000
\put {\mutation} at 2.098241 29.000000
\put {\mutation} at 2.261311 29.000000
\putrule from 3.924574 29.000000 to 3.924574 28.000000
\putrule from 3.924574 28.500000 to 8.027688 28.500000
\put {\mutation} at 5.267450 28.500000
\put {\mutation} at 3.949556 28.500000
\put {\mutation} at 4.231240 28.500000
\putrule from 0.000000 30.000000 to 5.551525 30.000000
\put {\mutation} at 5.400300 30.000000
\put {\mutation} at 0.819270 30.000000
\putrule from 0.000000 31.000000 to 0.158670 31.000000
\putrule from 0.000000 32.000000 to 0.158670 32.000000
\putrule from 0.158670 32.000000 to 0.158670 31.000000
\putrule from 0.158670 31.500000 to 2.987517 31.500000
\put {\mutation} at 1.895310 31.500000
\putrule from 0.000000 33.000000 to 2.987517 33.000000
\put {\mutation} at 2.159181 33.000000
\putrule from 2.987517 33.000000 to 2.987517 31.500000
\putrule from 2.987517 32.250000 to 5.551525 32.250000
\put {\mutation} at 4.102941 32.250000
\putrule from 5.551525 32.250000 to 5.551525 30.000000
\putrule from 5.551525 31.125000 to 8.000154 31.125000
\putrule from 0.000000 34.000000 to 1.032043 34.000000
\put {\mutation} at 0.568913 34.000000
\putrule from 0.000000 35.000000 to 1.032043 35.000000
\putrule from 1.032043 35.000000 to 1.032043 34.000000
\putrule from 1.032043 34.500000 to 3.099934 34.500000
\putrule from 0.000000 36.000000 to 0.538050 36.000000
\put {\mutation} at 0.228365 36.000000
\putrule from 0.000000 37.000000 to 0.538050 37.000000
\putrule from 0.538050 37.000000 to 0.538050 36.000000
\putrule from 0.538050 36.500000 to 3.099934 36.500000
\put {\mutation} at 0.649459 36.500000
\put {\mutation} at 0.877839 36.500000
\putrule from 3.099934 36.500000 to 3.099934 34.500000
\putrule from 3.099934 35.500000 to 8.000154 35.500000
\put {\mutation} at 5.197902 35.500000
\put {\mutation} at 3.883441 35.500000
\put {\mutation} at 7.879986 35.500000
\putrule from 8.000154 35.500000 to 8.000154 31.125000
\putrule from 8.000154 33.312500 to 8.009798 33.312500
\putrule from 0.000000 38.000000 to 8.004445 38.000000
\put {\mutation} at 2.441105 38.000000
\put {\mutation} at 5.712032 38.000000
\put {\mutation} at 5.006180 38.000000
\put {\mutation} at 0.229807 38.000000
\putrule from 0.000000 39.000000 to 1.899162 39.000000
\put {\mutation} at 1.401599 39.000000
\putrule from 0.000000 40.000000 to 1.899162 40.000000
\put {\mutation} at 1.448626 40.000000
\putrule from 1.899162 40.000000 to 1.899162 39.000000
\putrule from 1.899162 39.500000 to 4.666477 39.500000
\put {\mutation} at 4.625469 39.500000
\putrule from 0.000000 41.000000 to 4.666476 41.000000
\putrule from 4.666476 41.000000 to 4.666476 39.500000
\putrule from 4.666476 40.250000 to 8.004445 40.250000
\put {\mutation} at 7.990844 40.250000
\putrule from 8.004445 40.250000 to 8.004445 38.000000
\putrule from 8.004445 39.125000 to 8.005302 39.125000
\putrule from 0.000000 42.000000 to 8.003333 42.000000
\put {\mutation} at 1.262157 42.000000
\put {\mutation} at 4.937531 42.000000
\put {\mutation} at 1.317544 42.000000
\put {\mutation} at 2.173934 42.000000
\putrule from 0.000000 43.000000 to 5.748858 43.000000
\put {\mutation} at 1.932616 43.000000
\put {\mutation} at 4.137879 43.000000
\put {\mutation} at 0.070606 43.000000
\putrule from 0.000000 44.000000 to 1.387771 44.000000
\put {\mutation} at 0.905631 44.000000
\put {\mutation} at 1.134355 44.000000
\putrule from 0.000000 45.000000 to 1.387771 45.000000
\put {\mutation} at 0.747416 45.000000
\put {\mutation} at 0.027540 45.000000
\putrule from 1.387771 45.000000 to 1.387771 44.000000
\putrule from 1.387771 44.500000 to 5.324390 44.500000
\putrule from 0.000000 46.000000 to 3.905474 46.000000
\put {\mutation} at 0.372534 46.000000
\putrule from 0.000000 47.000000 to 0.086328 47.000000
\putrule from 0.000000 48.000000 to 0.086328 48.000000
\putrule from 0.086328 48.000000 to 0.086328 47.000000
\putrule from 0.086328 47.500000 to 0.503535 47.500000
\putrule from 0.000000 49.000000 to 0.503535 49.000000
\putrule from 0.503535 49.000000 to 0.503535 47.500000
\putrule from 0.503535 48.250000 to 3.905474 48.250000
\put {\mutation} at 1.573812 48.250000
\put {\mutation} at 1.576257 48.250000
\put {\mutation} at 0.759165 48.250000
\putrule from 3.905474 48.250000 to 3.905474 46.000000
\putrule from 3.905474 47.125000 to 5.324390 47.125000
\putrule from 5.324390 47.125000 to 5.324390 44.500000
\putrule from 5.324390 45.812500 to 5.748859 45.812500
\putrule from 5.748859 45.812500 to 5.748859 43.000000
\putrule from 5.748859 44.406250 to 8.003334 44.406250
\put {\mutation} at 6.363234 44.406250
\put {\mutation} at 7.245713 44.406250
\putrule from 8.003334 44.406250 to 8.003334 42.000000
\putrule from 8.003334 43.203125 to 8.005303 43.203125
\putrule from 8.005303 43.203125 to 8.005303 39.125000
\putrule from 8.005303 41.164062 to 8.009798 41.164062
\putrule from 8.009798 41.164062 to 8.009798 33.312500
\putrule from 8.009798 37.238281 to 8.027688 37.238281
\putrule from 8.027688 37.238281 to 8.027688 28.500000
\putrule from 8.027688 32.869141 to 8.039419 32.869141
\putrule from 8.039419 32.869141 to 8.039419 21.179688
\putrule from 8.039419 27.024414 to 8.054446 27.024414
\putrule from 8.054446 27.024414 to 8.054446 11.125000
\putrule from 8.054446 19.074707 to 8.142601 19.074707
\putrule from 8.142601 19.074707 to 8.142601 9.000000
\putrule from 8.142601 14.037354 to 8.143680 14.037354
\putrule from 8.143680 14.037354 to 8.143680 5.531250
\putrule from 8.143680 9.784302 to 15.000000 9.784302
%
%%%%%%%%%%%%%% Mismatch Distribution %%%%%%%%%%%%%%
%
\yunit=\plotht
\xunit=\plotwd
\Divide <\xunit> by <15.000000pt> forming <\xunit>
\Divide <\yunit> by <0.158367pt> forming <\yunit>
\advance\thusfar by \plotht
\advance\thusfar by \plotsp
\Divide <\thusfar> by <\yunit> forming <\offsety>
\placevalueinpts of <\offsety> in {\mktree_tmp}
\setcoordinatesystem units <\xunit, \yunit> point at 0 {\mktree_tmp}
\setplotarea x from 0.000000 to 15.000000, y from 0.000000 to 0.158367
\axis left label {\lines{Mismatch\cr distribution}} /
\axis bottom
   label {Pairwise differences}
   ticks numbered from 0 to 15 by 5 /
\multiput {$\circ$} at
 0 0.008980 1 0.007347 2 0.017959 3 0.028571 4 0.040816
 5 0.084082 6 0.109388 7 0.136327 8 0.158367 9 0.148571
 10 0.107755 11 0.080816 12 0.043265 13 0.019592 14 0.006531
 15 0.001633
/
\plot
 0 0.008980 1 0.007347 2 0.017959 3 0.028571 4 0.040816
 5 0.084082 6 0.109388 7 0.136327 8 0.158367 9 0.148571
 10 0.107755 11 0.080816 12 0.043265 13 0.019592 14 0.006531
 15 0.001633
/
\plotht=1.100000\plotht
%
%%%%%%%%%%%%%% Simulated site frequency spectrum %%%%%%%%%%%%%%
%
\yunit=\plotht
\xunit=\plotwd
\Divide <\xunit> by <49.000000pt> forming <\xunit>
\Divide <\yunit> by <0.653774pt> forming <\yunit>
\advance\thusfar by \plotht
\advance\thusfar by \plotsp
\Divide <\thusfar> by <\yunit> forming <\offsety>
\placevalueinpts of <\offsety> in {\mktree_tmp}
\setcoordinatesystem units <\xunit, \yunit> point at 0 {\mktree_tmp}
\setplotarea x from 0.000000 to 49.000000, y from 0.000000 to 0.653774
\axis left invisible label {\lines{Site\cr frequency\cr spectrum}} /
\axis bottom
   label {Frequency of mutant allele}
   ticks withvalues 0 1 / at 0 49 / /
\linethickness 1.2pt
\axis top /
\linethickness 0.4pt
\sethistograms
\plot 0 0 1 0.594340 2 0.179245 3 0.113208 4 0.075472
 5 0.000000 6 0.000000 7 0.018868 8 0.000000 9 0.000000
 10 0.000000 11 0.018868 12 0.000000 13 0.000000 14 0.000000
 15 0.000000 16 0.000000 17 0.000000 18 0.000000 19 0.000000
 20 0.000000 21 0.000000 22 0.000000 23 0.000000 24 0.000000
 25 0.000000 26 0.000000 27 0.000000 28 0.000000 29 0.000000
 30 0.000000 31 0.000000 32 0.000000 33 0.000000 34 0.000000
 35 0.000000 36 0.000000 37 0.000000 38 0.000000 39 0.000000
 40 0.000000 41 0.000000 42 0.000000 43 0.000000 44 0.000000
 45 0.000000 46 0.000000 47 0.000000 48 0.000000 49 0.000000
/
\setlinear
%
%%%%%%%%%%%%%% Neutral expectation of site freq spectrum %%%%%%%%%%%%%%
%
\multiput {$\bullet$} at 
     0.500000 0.223254  1.500000 0.111627  2.500000 0.074418  3.500000 0.0558135  4.500000 0.0446508
     5.500000 0.037209  6.500000 0.0318934  7.500000 0.0279067  8.500000 0.024806  9.500000 0.0223254
     10.500000 0.0202958  11.500000 0.0186045  12.500000 0.0171734  13.500000 0.0159467  14.500000 0.0148836
     15.500000 0.0139534  16.500000 0.0131326  17.500000 0.012403  18.500000 0.0117502  19.500000 0.0111627
     20.500000 0.0106311  21.500000 0.0101479  22.500000 0.00970669  23.500000 0.00930224  24.500000 0.00893015
     25.500000 0.00858669  26.500000 0.00826866  27.500000 0.00797335  28.500000 0.00769841  29.500000 0.0074418
     30.500000 0.00720174  31.500000 0.00697668  32.500000 0.00676527  33.500000 0.00656629  34.500000 0.00637868
     35.500000 0.0062015  36.500000 0.00603389  37.500000 0.0058751  38.500000 0.00572446  39.500000 0.00558135
     40.500000 0.00544522  41.500000 0.00531557  42.500000 0.00519195  43.500000 0.00507395  44.500000 0.0049612
     45.500000 0.00485335  46.500000 0.00475008  47.500000 0.00465112  48.500000 0.0045562
/
\endpicture}
\end{center}
mean pairwise difference: 7.783673
\end{figure}

\column{0.43\textwidth}
Gene genealogy: ancestry of a sample of gene copies.

\bigskip

Coalescent event: when lines of descent coalesce at common ancestors.

\bigskip

Shape reflects history of population size.

\bigskip

Coalescent theory connects history to genetics via gene genealogy.
\end{columns}
\end{frame}

\begin{frame}
  \frametitle{Hazards}

  The \emph{hazard} of an event at time $t$ is the conditional
  rate at which it occurs then, given that it did not occur
  earlier.

  \bigskip

  We'll be interested in the hazards of coalescent events in reverse
  time: looking backwards from the present.
\end{frame}  

\begin{frame}
\frametitle{Preliminaries: 2 mathematical tricks}

\begin{enumerate}
\item If the hazard of a coalescent event is $h$ per generation, then
  we wait on average $1/h$ generations until the event happens.
\item There are $k(k-1)/2$ ways to choose 2 items out of $k$.
\end{enumerate}
See \emph{Lecture Notes on Gene Genealogies} (available online) for details.
\end{frame}

\begin{frame}[containsverbatim]
\frametitle{Coalescence time in a sample of two gene copies}
\begin{columns}
\column{0.67\textwidth}
\begin{verbatim}
   X -----------------------|
                            A------
   Y -----------------------|
    
   |--------- t ------------|

\end{verbatim}
\column{0.33\textwidth}
$X$ and $Y$ live now; ancestor $A$ lived $t$ generations ago.
\end{columns}

\bigskip

How can we calculate $E[t]$---the expected ``coalescence time?''
\end{frame}

\begin{frame}[containsverbatim]
\frametitle{Coalescent probability per generation in a sample of 2 gene copies}
\begin{alltt}
              ________Population__________
Parents       0  Z  0  0  0  0  0  0  0  0
                 |
Offspring     0  X  0  Y  0  0  0  0  0  0
\end{alltt}

\medskip A population comprising 10 gene copies ($2N=10$).
Offspring generated by sampling with replacement from parents.

\medskip

$Z$ is the parent of $X$. What is the probability that $Y$ has the
same parent?
\[
1/10 \qquad\hbox{or}\qquad 1/2N
\]
\medskip
$1/2N$ is the probability of a coalescent event in a sample of 2 gene
copies.
\end{frame}

\begin{frame}
  \frametitle{Treating time as a continuum}

  So far, we've treated time in discrete units of 1 generation.

  \medskip

  In continuous time, let $h_2$ represent the hazard of a coalescent
  event in a sample of 2 gene copies. This means that $h_2 \delta$ is
  $\approx$ the probability of an event during a small interval of
  length $\delta$.

  \medskip

  If $2N$ is large, then there will be little change in 1 generation,
  and we can consider this interval to be ``small.'' Let us therefore
  set $\delta=1$.

  \medskip

  Then $h_2 = 1/2N$ is the hazard per generation of a coalescent event
  in a sample of 2 gene copies.
\end{frame}

\begin{frame}
\frametitle{Expected coalescence time in a sample of 2 gene copies}

We just saw that the hazard of a coalescent event is $h = 1/2N$ per
generation.

\bigskip

Recall that if the hazard is $h_2$, the expected waiting time is
$E[t_2] = 1/h_2$.

\bigskip

The expected coalescence time in a sample of 2 gene copies is
\[
E[t_2] = 2N
\]
generations.

\pause
\bigskip
Example: if $N=10,000$, $E[t_2] = 20,000$ generations, or about 60,000
human years.
\end{frame}

\begin{frame}[containsverbatim]
\frametitle{Coalescent hazard in a sample of $K$}
\begin{verbatim}
   W --------|                               
             |------------|                  
   X --------|            |                  
                          |--------------|   
   Y ---------------------|              |---
                                         |   
   Z ------------------------------------|   
   
   |-- t_4 --|---- t_3 ---|------ t_2 ---|
\end{verbatim}

\medskip With 4 gene copies, there are 3 coalescent intervals: $t_4$,
$t_3$, and $t_2$.

\medskip

The intervals are independent, so we can consider them one at a
time.
\end{frame}

\begin{frame}
\frametitle{Expected length of a coalescent interval containing 3
  lines of descent}

With 3 lines of descent ($X$, $Y$, and $Z$), there are 3 pairs of
lines ($XY$, $XZ$, and $YZ$).

\pause
\medskip

For each pair, we already know the coalescent hazard: $h_2 = 1/2N$.

\pause
\medskip

We have 3 pairs, so the coalescent hazard is 3$\times$ as large:
$h_3=3/2N$. (This argument is loose, but the answer is correct.)

\pause
\medskip

Expected length of an interval with 3 lines of descent:
\[
E[t_3] = 1/h_3 = 2N/3
\]
\end{frame}

\begin{frame}
\frametitle{Expected length of a coalescent interval containing $i$
  lines of descent}

With $i$ lines of descent, there are $i(i-1)/2$ pairs of
lines. (See mathematical trick~2 above.)

\pause
\medskip

Coalescent hazard of each pair: $h_2 = 1/2N$.

\pause
\medskip

Hazard within an interval with $i$ lines of descent:
\[
h_i = \frac{i(i-1)}{4N}
\]

\pause
\medskip

Expected length of an interval with $i$ lines of descent:
\[
E[t_i] = 1/h_i = \frac{4N}{i(i-1)}
\]
\end{frame}

\begin{frame}
\frametitle{Coalescent intervals in a sample of size 4}
\def\mystrut{\rule{0pt}{1.2em}}
{\centering
\begin{tabular}{ccc}
         & Coalescent & Expected\\
Interval & hazard     & length\\ \hline
4 & $h_4 = \displaystyle \frac{\mystrut 4 \times 3}{4N} = 6/2N$  & $2N/6$\\
3 & $h_3 = \displaystyle \frac{\mystrut 3 \times 2}{4N} = 3/2N$  & $2N/3$\\
2 & $h_2 = \displaystyle \frac{\mystrut 2 \times 1}{4N} = 1/2N$  & $2N$\\[7pt] 
\hline
\end{tabular}\\}
\end{frame}

\begin{frame}
\frametitle{Expected depth of a gene genealogy}

``Depth'' is the expected time (generations) since the last common
ancestor (LCA).
\begin{center}
\begin{tabular}{cr@{$\;=\;$}l}
Sample & \multicolumn{2}{c}{Mean depth} \\
size   & \multicolumn{2}{c}{of tree} \\ \hline
2 & $1/h_2$ & $2N$\\ 
3 & $1/h_3 + 1/h_2$ & $8N/3$\\
4 & $1/h_4 + 1/h_3 + 1/h_2$ & $3N$\\
5 & $1/h_5 + 1/h_4 + 1/h_3  + 1/h_2$ & $16N/5$\\ \hline
\end{tabular}
\end{center}
In general, mean depth is
\[
4N(1 - 1/K)
\]
where $K$ is the number of gene copies in the sample.
\end{frame}

\begin{frame}
  \frametitle{Deriving the formula}
  $E[t_i] = \frac{4N}{i(i-1)}$, so expected depth is
    \[
    4N\sum_{i=2}^K 1/i(i-1)
    \]
To simplify this sum, convince yourself that
\[
\frac{1}{i(i-1)} = \frac{1}{i-1} - \frac{1}{i}
\]
Then substitute the right side into the sum above and write the sum
out in expanded form. You should end up with $4N(1 - 1/K)$.
\end{frame}  

\begin{frame}
  \frametitle{A simulated genealogy of 50 gene copies}
             {\centering\input{figeq}\\}

             \bigskip 

             Recent coalescent intervals are short; ancient ones are long.

             \bigskip 

             The larger the sample, the shorter will be recent intervals.

             \bigskip 

             Large samples don't help much, because few mutations
             appear on these short intervals.
\end{frame}

\begin{frame}
  \frametitle{Review}
  \begin{itemize}
    \item What is a coalescent event?
    \item What is a hazard?
    \item What is the hazard of a coalescent event in an interval with
      $i$ lines of descent?
    \item What is the expected length of such an interval?
    \item What is the expected depth of a gene genealogy with $K$
      tips?
    \item Why are coalescent intervals longer in large populations?
    \item Why are they shorter in large samples?
  \end{itemize}
\end{frame}

\title{Relating Gene Genealogies to Genetics}
\author{Alan R. Rogers}
\date{\today}
\frame{\titlepage}

\begin{frame}
Gene genealogies are not observable: we can never know the true
genealogy of a sample. To \emph{estimate}, we need a theory that
relates gene genealogies to observable genetic data.

This lecture will:
\begin{enumerate}
\item add mutations to gene genealogies,
\item derive theory for the number, $S$, of segregating sites, and
\item the mean pairwise difference, $\pi$, between sequences.  
\end{enumerate}
\end{frame}

\begin{frame}[containsverbatim]
  \frametitle{A genealogy with 8 (not 9) mutations}
  \begin{columns}
    \column{0.65\textwidth}
\begin{verbatim}

     --x----x---
                |
                |------x-
                |        |
     ------xx---         |
                         |-----x----
                         |
                         |
     ---x-x----------x---

     |---t -----|---t ---|
          3          2
\end{verbatim}
\column{0.35\textwidth}
\raggedleft
We are interested in mutations downstream of the root, because
only these contribute to variation.
\end{columns}
\end{frame}

\begin{frame}[containsverbatim]
  \frametitle{Expected \# of mutations in a sample of size 3}
  \begin{columns}
    \column{0.65\textwidth}
\begin{verbatim}

   --x----x---
              |
              |------x-
              |        |
   ------xx---         |
                       |--------
                       |
                       |
   ---x-x----------x---

   |---t -----|---t ---|
        3          2
\end{verbatim}
\column{0.35\textwidth}
\raggedleft
\[
E[\# mutations|L] = uL
\]
where $u$ is mutation rate and $L$ is total branch length.

\bigskip

\[
L = 3t_3 + 2t_2
\]

\bigskip

Unconditionally, $E[\# mutations] = E[uL] = u E[L]$.
\end{columns}
\end{frame}

\begin{frame}
\frametitle{For a tree with 3 tips}

\[
L = 3t_3 + 2t_2
\]

\bigskip

Recall that $E[t_i] = 4N/i(i-1)$. \pause Therefore,
\begin{eqnarray*}
  E[L] &=& 3E[t_3] + 2E[t_2]\\
  &=& (3 \times 2N/3) + (2 \times 2N)\\
  &=& 6N
\end{eqnarray*}  

\bigskip\pause

Expected number of mutations is $6Nu$ for a tree with 3 tips.
\end{frame}

\begin{frame}
\frametitle{Expected length of tree with $K$ tips}
Expected length of the coalescent interval w/
$i$ lines of descent
\[
E[t_i] = \frac{4N}{i(i-1)}
\]
\pause
Contribution to $E[L]$:
\[
iE[t_i] = \frac{4N}{i-1}
\]
\pause
Total expected length:
\[
E[L] = \sum_{i=2}^K iE[t_i] = 4N \sum_{i=1}^{K-1} \frac{1}{i}
\]
\end{frame}

\begin{frame}
\frametitle{The expected number of mutations}
\begin{eqnarray*}
  E[\hbox{\# of mutations}] &=& u E[L]\\
  &=& 4Nu \sum_{i=1}^{K-1} \frac{1}{i}\\
  &=& \theta \sum_{i=1}^{K-1} \frac{1}{i}\\
  &=& \theta \left\{1 + \frac{1}{2} + \frac{1}{3} + \cdots +
  \frac{1}{K-1}\right\}
\end{eqnarray*}
\end{frame}

\begin{frame}
  \frametitle{Mutation: the model of infinite sites}
  Mutation never strikes the same site twice, so the number of
  nucleotide differences between two sequences equals the number of
  mutations that separate them.

  \bigskip

  This is an approximation that works well with intraspecific data
  sets, because the mutation rate is so low that few sites mutate more
  than once.
\end{frame}

\begin{frame}
  \frametitle{The expected number, $S$, of segregating sites}
  $S$ is the number of segregating (i.e.\ polymorphic) sites in a set
  of sequence data.

  \bigskip

  Under infinite sites, its expectation equals the expected number of
  mutations on the tree:
\[
E[S] = \theta \left\{1 + \frac{1}{2} + \frac{1}{3} + \cdots +
  \frac{1}{K-1}\right\}
\]
\end{frame}

\begin{frame}
  \frametitle{The effect of sample size is modest}
\begin{center}
\begin{tabular}{rr}
$K$ & $\sum_{i=1}^{K-1} 1/i$\\ \hline
2 & 1.00\\
3 & 1.50\\
5 & 2.08\\
10 & 2.82\\
100 & 5.17\\
1000 & 7.48\\
\hline
\end{tabular}
\end{center}

\bigskip

For practical purposes, $E[S]$ is $\theta$ times a number between 2
and 5.
\end{frame}

\begin{frame}
  \frametitle{The mean pairwise difference, $\pi$}

  The mean pairwise difference, $\pi$, is the mean number of
  nucleotide site differences between pairs of sequences in a sample.

  \bigskip

  In other words, it is the number of segregating sites in a sample of
  size 2.

  \bigskip

  Using our formula for $E[S]$,
\[
E[\pi] = \theta \sum_{i=1}^{1} \frac{1}{i} = \theta
\]
\end{frame}

\begin{frame}
  \frametitle{Two ways to estimate $\theta$}
\begin{eqnarray*}
\hat\theta_S &=& \frac{S}{\sum_{i=1}^{K-1} \frac{1}{i}}\\
\hat\theta_{\pi} &=& \pi
\end{eqnarray*}
Here $\hat\theta$ is read ``theta hat.''  The ``hat'' indicates that
these formulas are intended to estimate the parameter $\theta$.

\bigskip\pause

Since these formulas estimate the same parameter, we might expect them
to be similar in real data. Are they?
\end{frame}  

\begin{frame}
  \frametitle{Discrepancy between $\hat\theta_S$ and
    $\pi$}

Mitochondrial sequence data published by Lynn Jorde's lab, describing
77 Asians and 72 Africans:
\begin{center}
\begin{tabular}{lcc}
                            & Asian & African \\ \hline
$S$                         &  82    &   63\\
$\sum_{i=1}^{K-1} 1/i$      & 4.915  & 4.847\\
$\hat\theta_S$ (per sequence) & 16.685 & 12.998\\
$\pi$ (per sequence)        & 6.231  & 9.208\\
\hline
\end{tabular}
\end{center}

\bigskip

Contrary to expectation, $\theta_S$ is much larger than $\pi$. Why?

\end{frame}

\begin{frame}[containsverbatim]
  \frametitle{Thinking about this discrepancy}
  
  $S$ is equally sensitive to mutations anywhere in the gene genealogy.

  $\pi$, the MPD, is less sensitive to singletons than to mutations of
  intermediate frequency. 

\begin{verbatim}
   00000 00001
   12345 67890
S1 AAACT GTCAT
S2 ..... A....
S3 ..... A...C
S4 ..G.. A....
S5 ..G.. A....
S6 ..G.. A....
     ^   ^
     |   |
     |    ------ Contributes 1 X 5 = 5 pairwise diffs
      ---------- Contributes 3 X 3 = 9 pairwise diffs
\end{verbatim}
\end{frame}

\begin{frame}
  \frametitle{What does this imply about Jorde's data?}

  In the data, $\hat\theta_S \gg \pi$.

  \bigskip

  Suggests there are many young mutations (near the tips of the gene
  genealogy), where they affect $\hat\theta_S$ more than $\pi$.

  \bigskip

  As we'll learn later in the course, this implies a history of
  population growth.
\end{frame}  

\end{document}


