% -*-latex-*-
%\documentclass[handout]{beamer}
\documentclass[]{beamer}
\usepackage{etex}

\let\latexput\put
\usepackage{pictex}
\let\pictexput\put
\let\put\latexput

\setbeamercovered{transparent}
\setbeamertemplate{footline}[frame number]
\author{Alan R. Rogers}
\date{\today}
\title{Superarchaic Admixture}
\begin{document}

\frame{\titlepage}

\begin{frame}
\frametitle{Early to middle Pleistocene of Eurasia}

\begin{columns}
\column{0.5\textwidth}
\includegraphics[width=\linewidth]{atapuerca5.png}
\column{0.5\textwidth}
\raggedleft

$\sim$1.8~mya: \emph{Homo erectus} evolves in Africa, spreads into Eurasia

\bigskip

$\sim$550 kya: Late Acheulean appears in Europe.

\bigskip

$\sim$430 kya: large-brained hominins at Sima de los Huesos

\bigskip

Similar fossils and tools occur earlier in Africa.

\bigskip

$\Rightarrow$ African invasion of Europe early in Middle Pleistocene.
\end{columns}

\bigskip

\pause
\raggedleft
What can genetics tell us about this period?
\end{frame}

\begin{frame}
\frametitle{\emph{Legofit}: estimates deep population history in
  subdivided populations}
\begin{itemize}
\item Unaffected by recent inbreeding or changes in population size.
\item Sensitive only to the distant past.
\item Estimates gene flow and the sizes and separation times of
  ancestral populations.
\item New version is orders of magnitude faster.  
\end{itemize}
\end{frame}

\begin{frame}
\frametitle{Population network (now outdated)}
\centering%-*-latex-*-
\let\put\pictexput
\mbox{\beginpicture
\headingtoplotskip=\baselineskip
\setcoordinatesystem units <4mm, 4mm> point at 16 1
\setplotarea x from 1 to 15, y from 1 to 15
\small
\axis bottom invisible ticks length <0pt>
  withvalues {$X$} {$Y$} {$N$} {$D$} /  at 2 7 10 14 / /
%\axis bottom invisible shiftedto y=-0.3 ticks length <0pt>
%  withvalues {$x$} {$y$} {$n$} {$d$} /  at 2 7 10 14 / /
\axis left invisible ticks length <0pt>
  withvalues {$T_{XY}$} {$T_{XYND}$} /
   at 4 11.249724127083585 / /
\axis right invisible ticks length <0pt>
  withvalues {$T_{ND}$}  / at 6 / /
\put {\small Pops:} [r] <0pt, 3pt> at 1 0
%\put {\small Samples:} [rt] <0pt, -4pt> at 1 -0.3
\setdots
\putrule from 1 4 to 4.5 4   % $T_{XY}$
\putrule from 12 6 to 15 6 % $T_{ND}$
%\putrule from 12.75 2 to 15.75 2 % $T_D$
\putrule from 1 11.249724127083585
   to 8.12486206354179 11.249724127083585 % $T_{XYND}$
\setsolid
\setplotsymbol ({\normalsize .})
\plot 1 1 7 13 7 15 /   %X
\plot 3 1 4.5 4 6 1 /   %XY
\plot 8 1 5.5 6 8.12486206354179 11.249724127083585 /    %Y
\plot 9.17572 1 10.1757 6 11.036 7.7206 / %N
\plot 11 1 12 6 /       %N
\plot 13 1 12 6 /            %D
\plot 14.684 1 13.8151 6 12.3389 8.9523 9 13 9 15 / %Dr
\plot 8.12486206354179 11.249724127083585 11.036 7.7206 / % ND l
\arrow <10pt> [.2,.67] from 9.57571931984138 3 to 7 3
%\arrow <0pt> [.2,.67] from 12.75 2 to 11.25 2
%\arrow <10pt> [.2,.67] from 9.32572 2 to 7.5 2
\put {\small $N_{XY}$} at 6 9
\put {\small $N_{ND}$} at 11.2 9
\put {\small $N_N$}   at 10.7 4.3
\put {\small $N_{XYND}$} [b] <0pt,2pt> at 8 15
\put {\small $\alpha$} [b]  <0pt,5pt> at 8.5 3
%\put {\small $\epsilon$} [t]  <0pt,-5pt> at 12 2
\endpicture}
\let\put\latexput
\\[1ex]
$X$, Africa; $Y$, Europe; $N$, Neanderthal; $D$, Denisovan\\
\end{frame}

\begin{frame}
\frametitle{Gene genealogies and nucleotide site patterns}
\begin{columns}
\column{0.6\textwidth}
\centering\input{fighyparch}\\[1ex]
\column{0.4\textwidth}
\raggedleft
Gene genealogy within population network.

\bigskip

Mutation on red branch $\rightarrow$ \emph{site pattern} \textcolor{red}{$d$}.

\bigskip

Blue branch $\rightarrow$ \textcolor{blue}{$xyn$}.

\bigskip

0, ancestral allele; 1, derived (mutant) allele.

\bigskip

Data: frequencies of site patterns across autosomes
\end{columns}
\end{frame}

\begin{frame}
\frametitle{Observed Site Pattern Frequencies (excl. Vindija)}
\begin{columns}
\column{0.6\textwidth}
\includegraphics[width=\linewidth]{xyad-frq.pdf}\\

\medskip

\centering
(fraction of nucleotide sites exhibiting each pattern)\\
\column{0.4\textwidth}
\raggedleft
$x$, Africa; $y$, Europe; $a$, Altai Neanderthal; $d$, Denisovan.

\bigskip

Pattern $xy$ is common because populations $X$ and $Y$ share ancestry.

\bigskip

Ditto $ad$.

\bigskip

Confidence intervals are so small they look like dots.

\bigskip

Goal: infer history from similar data, but including Vindija.
\end{columns}
\end{frame}

\begin{frame}
\frametitle{Estimation}
\begin{enumerate}
\item Maximize composite likelihood, a function of sizes and
  separation times of populations, and rates of gene flow.
\item Old Legofit used simulations to estimate likelihood. New
  algorithm is deterministic.  
\item Uncertainties by moving-blocks bootstrap.
\end{enumerate}
\end{frame}

\begin{frame}
\frametitle{In 2017, we fit model $\alpha$ to the data}
\centering%-*-latex-*-
\let\put\pictexput
\mbox{\beginpicture
\headingtoplotskip=\baselineskip
\setcoordinatesystem units <4mm, 4mm> point at 16 1
\setplotarea x from 1 to 15, y from 1 to 15
\small
\axis bottom invisible ticks length <0pt>
  withvalues {$X$} {$Y$} {$N$} {$D$} /  at 2 7 10 14 / /
%\axis bottom invisible shiftedto y=-0.3 ticks length <0pt>
%  withvalues {$x$} {$y$} {$n$} {$d$} /  at 2 7 10 14 / /
\axis left invisible ticks length <0pt>
  withvalues {$T_{XY}$} {$T_{XYND}$} /
   at 4 11.249724127083585 / /
\axis right invisible ticks length <0pt>
  withvalues {$T_{ND}$}  / at 6 / /
\put {\small Pops:} [r] <0pt, 3pt> at 1 0
%\put {\small Samples:} [rt] <0pt, -4pt> at 1 -0.3
\setdots
\putrule from 1 4 to 4.5 4   % $T_{XY}$
\putrule from 12 6 to 15 6 % $T_{ND}$
%\putrule from 12.75 2 to 15.75 2 % $T_D$
\putrule from 1 11.249724127083585
   to 8.12486206354179 11.249724127083585 % $T_{XYND}$
\setsolid
\setplotsymbol ({\normalsize .})
\plot 1 1 7 13 7 15 /   %X
\plot 3 1 4.5 4 6 1 /   %XY
\plot 8 1 5.5 6 8.12486206354179 11.249724127083585 /    %Y
\plot 9.17572 1 10.1757 6 11.036 7.7206 / %N
\plot 11 1 12 6 /       %N
\plot 13 1 12 6 /            %D
\plot 14.684 1 13.8151 6 12.3389 8.9523 9 13 9 15 / %Dr
\plot 8.12486206354179 11.249724127083585 11.036 7.7206 / % ND l
\arrow <10pt> [.2,.67] from 9.57571931984138 3 to 7 3
%\arrow <0pt> [.2,.67] from 12.75 2 to 11.25 2
%\arrow <10pt> [.2,.67] from 9.32572 2 to 7.5 2
\put {\small $N_{XY}$} at 6 9
\put {\small $N_{ND}$} at 11.2 9
\put {\small $N_N$}   at 10.7 4.3
\put {\small $N_{XYND}$} [b] <0pt,2pt> at 8 15
\put {\small $\alpha$} [b]  <0pt,5pt> at 8.5 3
%\put {\small $\epsilon$} [t]  <0pt,-5pt> at 12 2
\endpicture}
\let\put\latexput
\\[1ex]
$X$, Africa; $Y$, Europe; $N$, Neanderthal; $D$, Denisovan\\
\end{frame}

\begin{frame}
\frametitle{Residual error from model $\alpha$}
\begin{columns}
\column{0.5\textwidth}
\includegraphics[width=\linewidth]{xyvad-a-b2-resid.pdf}
\column{0.5\textwidth}
\raggedleft
Red asterisks: fitted model. Blue circles: bootstrap replicates.

\bigskip

If model fit well, all points would be near 0.

\bigskip

Discrepancies show that something is missing from the model. What?
\end{columns}
\end{frame}

\begin{frame}
  \frametitle{Ideas from the literature}
  \begin{itemize}
    \item[$\beta$] Gene flow from a ``superarchaic'' population into Denisovans
      (Pr{\"u}fer et al 2014)
    \item[$\gamma$] Gene flow from early modern humans into Neanderthals
      (Kuhlwilm et al 2016)
  \end{itemize}

  \bigskip
  
  These improved the fit but were still unsatisfactory.

  \bigskip

  What else is missing?
\end{frame}

\begin{frame}
\frametitle{Think back to what I said about the Middle Pleistocene}

\begin{columns}
\column{0.5\textwidth}
\includegraphics[width=\linewidth]{atapuerca5.png}
\column{0.5\textwidth}
\raggedleft

$\sim$600 kya Eurasia invaded by large-brained hominins, who probably
came from Africa.

\bigskip

Ancestors of Neanderthals and Denisovans: let's call them
``neandersovans.''

\bigskip

But Eurasia had been inhabited since $\sim$2~my ago by ``superarchaics.''

\bigskip

Neandersovans would have met, and maybe interbred with,
superarchaics. Suggests a fourth episode of admixture.
\end{columns}
\end{frame}

\begin{frame}
\centering
\input{figa}~\input{figab}\\[2ex]
\input{figabc}~%-*-latex-*-
\let\put\pictexput
\mbox{\beginpicture
\headingtoplotskip=\baselineskip
\setcoordinatesystem units <2.6mm, 1.8mm>
\setplotarea x from -2 to 17.5, y from 0 to 20
\axis bottom invisible ticks length <0pt>
  withvalues {$X$} {$Y$} {$N$} {$D$} {$S$} /  at 1 6 9 13 17 / /
\put {Model~$\alpha\beta\gamma\delta$} [tr] at 5.5 20
%\axis left ticks andacross numbered from 0 to 20 by 2 /
%\axis top ticks andacross numbered from 0 to 20 by 2 /
\setplotsymbol ({\normalsize .})
\plot 0 0 6 12 6 20 /   %X
\plot 2 0 3.5 3 5 0 /   %XY
\plot 7 0 4.5 5 7.12486206354179 10.249724127083585 /    %Y
\arrow <10pt> [.2, .67] from 13.7 9 to 10.6 9
\put {\small $\delta$} [t]  <0pt,-5pt> at 12.9 9
\put {\small $\gamma$} [b]  <2pt,5pt> at 7.3 6
\arrow <10pt> [.2,.67] from 5 6 to 9.6 6
\plot 8.17572 0 9.1757 5 10.036 6.7206 / %N
\plot 10 0 11 5 /       %N
\plot 12 0 11 5 /            %D
\plot 8 16 14.21226360816738 8.468963009740001 15.684 0 / %Sl
\plot 8 20 8 18.8112 15.8936 9.24189 17.4997 0 / %Sr
\plot 13.684 0 12.8151 5 11.3389 7.9523 8 12 8 16 / %Dr
\plot 7.12486206354179 10.249724127083585 10.036 6.7206 / % ND l
\arrow <10pt> [.2,.67] from 8.57571931984138 2 to 6 2
%\arrow <0pt> [.2,.67] from 11.75 1 to 10.25 1
%\arrow <10pt> [.2,.67] from 8.32572 1 to 6.5 1
\put {\small $\alpha$} [b]  <0pt,5pt> at 7.5 2
%\put {\small $\text{\epsilon}$} [t]  <0pt,-5pt> at 11 1
\put {\small $\beta$} [t]  <2pt,-5pt> at 13.9889 4
\arrow <10pt> [.2,.67] from 14.9889 4 to 12.9889 4
\endpicture}
\let\put\latexput
\\
\end{frame}

\begin{frame}
\begin{columns}
\column{0.5\textwidth}
\includegraphics[width=\linewidth]{xyvad-a-b2-resid.pdf}
\column{0.5\textwidth}
\includegraphics[width=\linewidth]{xyvad-abcd-b2-resid.pdf}
\end{columns}

\bigskip

\centering
\small Key: \textcolor{red}{*}, real data; \textcolor{blue}{$\circ$},
bootstrap replicates.
\end{frame}

\begin{frame}
  \frametitle{Model selection and model averaging}

  Model selection by \textbf{bepe}, the bootstrap estimate of
  predictive error (Efron \& Tibshirani 1993).  Prefer model with
  smallest bepe value. Avoids overfitting. 

\bigskip  

Model averaging by \textbf{booma}, bootstrap model averaging
(Buckland, Burnham, and Augustin, 1997). Weight of $i$th model is
fraction of bootstrap replicates in which that model is
best. Parameter estimates are weighted averages of per-model
estimates. Addresses identifiability problems.
\end{frame}

\begin{frame}
  \frametitle{Evaluating the models}
{\centering
\begin{tabular}{lcc}
Model & bepe & weight\\
\hline
$\alpha$ &$1.16 \times 10^{-6}$ & 0\\
$\alpha\delta$ & $0.87 \times 10^{-6}$ & 0\\
$\alpha\gamma$ & $0.62 \times 10^{-6}$ &  0\\
$\alpha\gamma\delta$ & $0.44 \times 10^{-6}$ & 0\\
$\alpha\beta$ & $0.18 \times 10^{-6}$ &  0\\
$\alpha\beta\gamma$ & $0.17 \times 10^{-6}$ & 0\\
$\alpha\beta\delta$ & $0.15 \times 10^{-6}$ & 0.16\\
$\alpha\beta\gamma\delta$ & $0.13 \times 10^{-6}$ & 0.84\\
\end{tabular}\\}

\bigskip

Reject models with weight zero: their disadvantage is large compared
with variation in repeated sampling.

\bigskip

Strong support for two episodes of superarchaic admixture ($\beta$ and
$\delta$); qualified support for admixture ($\gamma$) from early
moderns into Neanderthals.
\end{frame}

\begin{frame}
\begin{columns}
  \column{0.6\textwidth}
  \textcolor{blue}{\Large Parameter estimates}\\[1ex]
\includegraphics[width=\linewidth]{bmadot.pdf}\\
\column{0.4\textwidth}
\flushright

Superarchaic population separated $\sim$2~mya. It was large---between
20,000 and 50,000---or deeply subdivided.

\bigskip

neandersovan population ($N_{ND}$) was tiny, and split early ($T_{ND} >
700$~kya) to form Neanderthals and Denisovans.

\bigskip

$\sim$3\% admixture into neandersovans from superarchaics.
\end{columns}
\end{frame}

\begin{frame}
\frametitle{Summary}

Superarchaics separated from other hominins $\sim$2~mya. They may
represent the earliest Eurasians. Their population was either large or
deeply subdivided.

\bigskip

$\sim$750~kya, neandersovans separated from an African population,
expanded into Eurasia, endured a bottleneck, interbred with
superarchaics, and then ($\sim$730~kya) split into eastern and western
subpopulations (Denisovans \& Neanderthals).
\end{frame}

\begin{frame}
\frametitle{Acknowledgements}

\textbf{Collaborators:}
Nathan Harris, Alan Achenbach, Kiela Gwin, Mitchell Lokey, Daniel Tabin.

\bigskip

\textbf{Support:}
NSF BCS~1638840; NSF BCS~1945782;
Center for High Performance Computing,  U.\ of Utah.
\end{frame}

\end{document}

\begin{frame}
\frametitle{}

\end{frame}

