% -*-latex-*-
%\documentclass[handout]{beamer}
\documentclass[12pt]{beamer}
\usepackage{pictex}
\begin{document}

\title{Natural Selection}
\author{Alan R. Rogers}
\date{\today}

\frame{\titlepage}

\begin{frame}
  \frametitle{Selection}
  Selection is a bias in survival and/or reproduction: when one allele
  is better than another.

  \bigskip

  Charles Darwin's big idea. (Also Alfred Russell Wallace.)

  \bigskip

  The only evolutionary force capable of producing adaption.

  \bigskip

  Main job: removing harmful mutations.

  \bigskip

  Can produce rapid change in allele frequencies.
\end{frame}

\begin{frame}
  \frametitle{Basic model}
  A locus with two alleles, $A_1$ and $A_2$, with relative
  frequencies in some generation are $p$ and $q \equiv 1 - p$.

  \bigskip

  Genotype $A_{i}A_{j}$ has ``fitness'' $w_{ij}$. Think of this as a
  probability of survival. Selection also works through fertility, but
  the algebra is different.

  \bigskip

  {\centering
\begin{tabular}{lccc}
Genotype & $A_1A_1$ & $A_1A_2$ & $A_2A_2$\\ 
\hline\hline
Freq.\ before selection  & $p^{2}$ & $2pq$ & $q^{2}$\\
Fitness  & $w_{11}$ & $w_{12}$ & $w_{22}$ \\
Freq.\ after selection  & $p^{2}w_{11}/\bar w$
         & $2pq w_{12}/\bar w$
         & $q^{2}w_{22}/\bar w$\\ \hline
\multicolumn{4}{c}{Where $q = 1-p$}\\
\multicolumn{4}{c}{and $\bar w = p^{2}w_{11}+2pqw_{12}+q^{2}w_{22}$}\\
\end{tabular}\\}
\end{frame}

\begin{frame}
After one generation of selection:
\[
p' = \frac{p^{2} w_{11} + pqw_{12}}{\bar w}
\]
where $p'$ is the allele frequency among offspring.

\bigskip

Rate of change:
\begin{eqnarray*}
  \Delta p &=& p' - p\\
  &=& \frac{p q}{\bar w} \bigl[p (w_{11}-w_{12}) + q(w_{12} - w_{22})\bigr]
\end{eqnarray*}
\end{frame}

\begin{frame}
  \frametitle{Implications}

  \[
  \Delta p = \frac{p q}{\bar w} [p (w_{11}-w_{12}) + q(w_{12} -
    w_{22})]
  \]

\begin{enumerate}
\item Response to selection is slow when either allele is rare,
  because then $pq$ is small.
\item Response is slow when a recessive allele is rare. In this case,
  $w_{12} = w_{22}$, and $\Delta p \propto p^2$.
\item Response is slow when a dominant allele is common.
\item A rare allele spreads if its heterozygote is fitter than the common
homozygote, because if $p$ is small, $\Delta p \propto w_{12} - w_{22}$.
\end{enumerate}
\end{frame}

\begin{frame}
  \frametitle{Implications}

  \[
  \Delta p = \frac{p q}{\bar w} [p (w_{11}-w_{12}) + q(w_{12} -
    w_{22})]
  \]
\begin{enumerate}
\setcounter{enumi}{4}
\item If the heterozygote has intermediate fitness, then one allele will
  increase to fixation.
\item If the heterozygote has the highest fitness, then $A_1$ evolves toward
an intermediate equilibrium.
\end{enumerate}
\end{frame}

\begin{frame}
  \frametitle{Wright's equation}
  Sewall Wright noticed that the formula for $\Delta p$, the quantity
  in square brackets is half $d\bar w/dp$:
  \[
  \Delta p = \frac{pq}{2\bar w}\cdot\frac{d \bar w}{d p}
  \]
  This implies that selection pushes a population ``uphill'': in a
  direction that increases mean fitness.

  \bigskip

  This is why selection explains adaptation.
\end{frame}  


\title{Selection plus Mutation}
\author{Alan R. Rogers}
\date{\today}

\frame{\titlepage}


\begin{frame}
  \frametitle{All that matters is \emph{relative} fitness}
The frequency, $p'$, of $A_1$ among offspring is
\[
p' = \frac{p^{2} w_{11} + pqw_{12}}{\bar w}, \quad\hbox{where}
\]
\[
\bar w = p^{2}w_{11}+2pqw_{12}+q^{2}w_{22} \quad\hbox{is mean fitness.}
\]

\bigskip

If we multiply all fitnesses by some constant, $K$, nothing changes,
\[
p' = \frac{p^{2} Kw_{11} + pqKw_{12}}{K\bar w}
\]
because the constant cancels.
\end{frame}

\begin{frame}
  \frametitle{Reformulating fitness}
\[
    \begin{array}{lccc}
      \hbox{Old notation} & w_{11} & w_{12} & w_{22}\\
      \hbox{New notation} & 1 & 1-hs & 1-s
    \end{array}
\]    
  Here, $s$ is the selective disadvantage of $A_2A_2$, and $h$
  measures dominance. $h=1/2$ if effects are additive, $h=1$ if $A_2$
  is recessive, and $h=0$ if $A_2$ is dominant.

\begin{eqnarray*}
  \Delta p &=& \frac{p q}{\bar w}
  \bigl[p (w_{11}-w_{12}) + q(w_{12} - w_{22})\bigr]\\
  &=& \frac{pqs[ph + q(1-h)]}{\bar w}
\end{eqnarray*}
where
\[
\bar w = 1 - 2pqhs - q^2 s
\]
\end{frame}

\begin{frame}
  \frametitle{Mutation and selection}
\[
  \Delta p = \frac{pqs[ph + q(1-h)]}{\bar w} \approx qhs
\]
if $q\approx 0$, $p \approx 1$, and $\bar w \approx 1$.

\end{frame}

\end{document}

