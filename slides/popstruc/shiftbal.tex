\begin{frame}
What role does population structure play in adaptive evolution?
\end{frame}

\begin{frame}
\frametitle{Wright's adaptive landscape}
\includegraphics[width=\textwidth]{adaptive-landscape.pdf}
\end{frame}


\begin{frame}
\frametitle{Wright's adaptive landscape}
\begin{columns}
\column{0.6\textwidth}
\includegraphics[width=\textwidth]{adaptive-landscape.pdf}
\column{0.4\textwidth}
Wright's  (1932) adaptive landscape.  Each $(x,y)$ point represents a
multilocus genotype.

\pause
Contour lines show fitness.  Populations tend to
move uphill.
\end{columns}

\bigskip

\pause
Multiple peaks exist because genes do not act additively.
\end{frame}

\begin{frame}
\frametitle{Wright's shifting balance theory}
\includegraphics[width=\textwidth]{shiftbal.pdf}
\end{frame}

\begin{frame}
\frametitle{Wright's shifting balance theory}
\begin{columns}
\column{0.6\textwidth}
\includegraphics[width=\textwidth]{shiftbal.pdf}
\column{0.4\textwidth}
Adaptive evolution is fastest when the population is divided into
moderately small local groups that exchange migrants.

\pause
Only then can selection choose among co-adapted gene complexes.
\end{columns}
\end{frame}

\begin{frame}
\frametitle{Extinction and recolonization}
\includegraphics[width=\textwidth]{wright-recol.pdf}
\end{frame}

\begin{frame}
\frametitle{Extinction and recolonization}
\begin{columns}
\column{0.6\textwidth}
\includegraphics[width=\textwidth]{wright-recol.pdf}
\column{0.4\textwidth}
Wright's  (1940) conception of local extinction and recolonization by
rare migrants.
Time on horizontal axis.  Territories distinguished vertically.
\pause
Heavily shaded group has passed through 6 bottlenecks.
\end{columns}

\pause
\bigskip
Bottlenecks allow populations to explore the field of gene combinations.
\end{frame}

\begin{frame}
\frametitle{Extinction and recolonization}
\begin{columns}
\column{0.6\textwidth}
\includegraphics[width=\textwidth]{wright-recol.pdf}
\column{0.4\textwidth}
Wright's  (1940) conception of local extinction and recolonization by
rare migrants.
Time on horizontal axis.  Territories distinguished vertically.
Heavily shaded group has passed through 6 bottlenecks.
\end{columns}
\end{frame}

\begin{frame}
\frametitle{Two contrasting views about adaptive evolution}
\begin{description}
\item[R.A. Fisher] Rate $4Nus$ highest in large panmictic populations.
\item[Sewall Wright] Rate highest when population is subdivided in to
  partially isolated local groups.
\end{description}
\pause

We're still arguing about this:
\begin{itemize}
\item Coyne et al. 1997. Perspective: a critique of Wright's shifting
  balance theory of evolution. \emph{Evolution} 51(3):643--671.
\item Wade et al. 1998. Perspective: the theories of Fisher and Wright
  in the context of metapopulations: when nature does many small
  experiments. \emph{Evolution} 52(6):1537--1553.
\end{itemize}
\end{frame}



