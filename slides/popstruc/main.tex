% -*-latex-*-
%\documentclass[pdftex,handout]{beamer}
\documentclass[pdftex,12pt]{beamer}
\usepackage{etex}

\let\latexput\put
\usepackage{pictex}
\let\pictexput\put
\let\put\latexput

\setbeamertemplate{footline}[frame number]
\newcommand{\G}{\mathcal{G}}
\newcommand{\Het}{\mathcal{H}}
\title{Geographic Population Structure}
\author{Alan R. Rogers}
\date{\today}
\setbeamercovered{transparent}
\begin{document}

\frame{\titlepage}

\begin{frame}
\frametitle{Buri's (1956) data}
\includegraphics[height=0.8\textheight]{buridist1.pdf}
\includegraphics[height=0.8\textheight]{buridist2.pdf}
\end{frame}

\begin{frame}
\frametitle{Drift reduces heterozygosity w/i groups while increasing
  variance among groups}
{\centering\let\put\pictexput
\mbox{\beginpicture
%%%%%%%%%%%%%% Top plot
\setcoordinatesystem units <0.07in, 1.92in> point at 19 0
\setplotarea x from 0 to 19, y from 0.0 to 0.52
\axis left shiftedto x=-0.53 label {$H$}
  ticks numbered from 0.00 to 0.50 by  0.25 /
\axis bottom shiftedto y=-0.02 label {Generation} 
  ticks numbered from 0 to 15 by 5 /
\put {Heterozygosity} [lb] at 0 0.01
% Buri 1956. Series I. Observed heterozygosity H
\multiput {$\circ$} at
%Gen. ObsHet
% 0 1.000
 1 0.514
 2 0.464
 3 0.504
 4 0.456
 5 0.448
 6 0.428
 7 0.403
 8 0.402
 9 0.358
10 0.348
11 0.325
12 0.305
13 0.263
14 0.255
15 0.216
16 0.202
17 0.210
18 0.197
19 0.183
/
%%%%%%%%%%%%%% Right plot
\setcoordinatesystem units <0.075in, 5.88in> point at -2 0
\setplotarea x from 0 to 19, y from 0 to 0.17
\axis right shiftedto x=19.53 label {$V$}
  ticks numbered from 0.0 to 0.15 by  0.05 /
\axis bottom shiftedto y=-0.0068  label {Generation} 
  ticks numbered from 0 to 15 by 5 /
%\put {\makebox[0pt]{Drift Experiment, Series I, Buri (1956)}} [c] at -1 0.22
\put {Variance} [rb] at 19 0.01
\multiput {$\circ$} at
% 0 0
 1 0.006
 2 0.026
 3 0.031
 4 0.042
 5 0.050
 6 0.055
 7 0.062
 8 0.072
 9 0.083
10 0.090
11 0.105
12 0.112
13 0.123
14 0.136
15 0.140
16 0.155
17 0.160
18 0.165
19 0.170
/
\endpicture}
\let\put\latexput
\\[2ex]}
\mbox{}\hfill{}Data: Buri (1956)\\           
\end{frame}

\begin{frame}
\frametitle{Gene flow (migration) reduces population differences}
\centering\let\put\pictexput
\mbox{\beginpicture
\valuestolabelleading=.4\baselineskip
\headingtoplotskip=0pt
\setcoordinatesystem units <0.002\linewidth, 0.2\textheight> point at 0 0
\setplotarea x from 0 to 367, y from 0 to 1
\axis left label {$p$} ticks numbered from 0 to 1 by 1 /
\axis bottom 
  ticks unlabeled from 0 to 300 by  100 /
\setplotsymbol ({\normalsize .})
\plotheading{$2Nm=0$: No gene flow}
% m=0
\plot
   1   0.54
   2   0.56
   3   0.57
   4   0.59
   5   0.56
   6   0.60
   7   0.66
   8   0.65
   9   0.70
  10   0.67
  11   0.62
  12   0.77
  13   0.81
  14   0.87
  15   0.86
  16   0.85
  17   0.84
  18   0.83
  19   0.80
  20   0.79
  21   0.81
  22   0.78
  23   0.74
  24   0.75
  25   0.71
  26   0.77
  27   0.77
  28   0.76
  29   0.80
  30   0.77
  31   0.82
  32   0.83
  33   0.89
  34   0.82
  35   0.82
  36   0.86
  37   0.88
  38   0.93
  39   0.97
  40   0.94
  41   0.96
  42   0.96
  43   0.98
  44   0.97
  45   0.98
  46   0.94
  47   0.92
  48   0.97
  49   0.99
  50   1.00
/
\plot
   1   0.50
   2   0.48
   3   0.49
   4   0.44
   5   0.46
   6   0.51
   7   0.50
   8   0.47
   9   0.32
  10   0.35
  11   0.33
  12   0.36
  13   0.37
  14   0.35
  15   0.34
  16   0.36
  17   0.43
  18   0.50
  19   0.43
  20   0.44
  21   0.38
  22   0.35
  23   0.31
  24   0.24
  25   0.22
  26   0.22
  27   0.22
  28   0.17
  29   0.19
  30   0.10
  31   0.08
  32   0.09
  33   0.09
  34   0.05
  35   0.04
  36   0.02
  37   0.02
  38   0.03
  39   0.05
  40   0.06
  41   0.10
  42   0.10
  43   0.10
  44   0.08
  45   0.12
  46   0.09
  47   0.13
  48   0.15
  49   0.14
  50   0.10
/
\setcoordinatesystem units <0.002\linewidth, 0.2\textheight> point at 0 1.8
\setplotarea x from 0 to 367, y from 0 to 1
\axis left label {$p$} ticks numbered from 0 to 1 by 1 /
\axis bottom label {Generation}
  ticks numbered from 0 to 300 by  100  /
\setplotsymbol ({\normalsize .})
\plotheading{$2Nm=10$: strong gene flow}
% m=0.1
\plot
   1   0.57
   2   0.58
   3   0.52
   4   0.50
   5   0.54
   6   0.47
   7   0.46
   8   0.43
   9   0.40
  10   0.34
  11   0.32
  12   0.37
  13   0.47
  14   0.49
  15   0.55
  16   0.54
  17   0.53
  18   0.54
  19   0.50
  20   0.45
  21   0.51
  22   0.52
  23   0.51
  24   0.51
  25   0.48
  26   0.45
  27   0.50
  28   0.41
  29   0.39
  30   0.40
  31   0.40
  32   0.46
  33   0.52
  34   0.49
  35   0.55
  36   0.54
  37   0.48
  38   0.47
  39   0.46
  40   0.43
  41   0.52
  42   0.43
  43   0.45
  44   0.40
  45   0.32
  46   0.31
  47   0.33
  48   0.38
  49   0.35
  50   0.28
  51   0.35
  52   0.35
  53   0.41
  54   0.53
  55   0.53
  56   0.55
  57   0.58
  58   0.51
  59   0.49
  60   0.50
  61   0.58
  62   0.54
  63   0.62
  64   0.69
  65   0.67
  66   0.69
  67   0.63
  68   0.68
  69   0.58
  70   0.47
  71   0.53
  72   0.51
  73   0.44
  74   0.47
  75   0.49
  76   0.49
  77   0.53
  78   0.49
  79   0.51
  80   0.55
  81   0.47
  82   0.46
  83   0.50
  84   0.53
  85   0.47
  86   0.42
  87   0.37
  88   0.35
  89   0.34
  90   0.37
  91   0.35
  92   0.42
  93   0.35
  94   0.32
  95   0.23
  96   0.18
  97   0.18
  98   0.28
  99   0.27
 100   0.24
 101   0.27
 102   0.31
 103   0.29
 104   0.37
 105   0.29
 106   0.32
 107   0.30
 108   0.37
 109   0.40
 110   0.33
 111   0.36
 112   0.36
 113   0.37
 114   0.45
 115   0.46
 116   0.42
 117   0.48
 118   0.49
 119   0.46
 120   0.41
 121   0.48
 122   0.45
 123   0.45
 124   0.55
 125   0.60
 126   0.58
 127   0.57
 128   0.58
 129   0.61
 130   0.57
 131   0.62
 132   0.59
 133   0.61
 134   0.46
 135   0.40
 136   0.43
 137   0.48
 138   0.44
 139   0.47
 140   0.40
 141   0.31
 142   0.34
 143   0.35
 144   0.44
 145   0.40
 146   0.45
 147   0.40
 148   0.46
 149   0.43
 150   0.40
 151   0.44
 152   0.46
 153   0.47
 154   0.47
 155   0.48
 156   0.44
 157   0.37
 158   0.33
 159   0.29
 160   0.36
 161   0.38
 162   0.35
 163   0.33
 164   0.31
 165   0.29
 166   0.27
 167   0.32
 168   0.27
 169   0.29
 170   0.32
 171   0.29
 172   0.26
 173   0.29
 174   0.26
 175   0.21
 176   0.20
 177   0.19
 178   0.19
 179   0.26
 180   0.29
 181   0.24
 182   0.17
 183   0.16
 184   0.13
 185   0.12
 186   0.09
 187   0.15
 188   0.21
 189   0.26
 190   0.23
 191   0.25
 192   0.26
 193   0.33
 194   0.42
 195   0.44
 196   0.49
 197   0.48
 198   0.43
 199   0.33
 200   0.36
 201   0.36
 202   0.39
 203   0.41
 204   0.39
 205   0.35
 206   0.40
 207   0.48
 208   0.57
 209   0.54
 210   0.47
 211   0.43
 212   0.48
 213   0.42
 214   0.48
 215   0.55
 216   0.57
 217   0.53
 218   0.53
 219   0.57
 220   0.53
 221   0.60
 222   0.57
 223   0.48
 224   0.50
 225   0.57
 226   0.61
 227   0.65
 228   0.68
 229   0.71
 230   0.71
 231   0.76
 232   0.75
 233   0.74
 234   0.77
 235   0.71
 236   0.70
 237   0.73
 238   0.78
 239   0.73
 240   0.75
 241   0.72
 242   0.62
 243   0.66
 244   0.64
 245   0.61
 246   0.68
 247   0.73
 248   0.65
 249   0.60
 250   0.54
 251   0.51
 252   0.60
 253   0.60
 254   0.60
 255   0.56
 256   0.56
 257   0.60
 258   0.61
 259   0.59
 260   0.61
 261   0.51
 262   0.59
 263   0.66
 264   0.50
 265   0.51
 266   0.64
 267   0.63
 268   0.59
 269   0.62
 270   0.64
 271   0.62
 272   0.63
 273   0.59
 274   0.65
 275   0.60
 276   0.66
 277   0.64
 278   0.47
 279   0.45
 280   0.50
 281   0.56
 282   0.65
 283   0.54
 284   0.59
 285   0.62
 286   0.68
 287   0.69
 288   0.67
 289   0.74
 290   0.81
 291   0.82
 292   0.70
 293   0.70
 294   0.72
 295   0.83
 296   0.80
 297   0.81
 298   0.79
 299   0.72
 300   0.76
 301   0.72
 302   0.73
 303   0.74
 304   0.74
 305   0.70
 306   0.67
 307   0.67
 308   0.72
 309   0.79
 310   0.81
 311   0.78
 312   0.72
 313   0.73
 314   0.76
 315   0.76
 316   0.77
 317   0.73
 318   0.71
 319   0.79
 320   0.79
 321   0.74
 322   0.69
 323   0.79
 324   0.83
 325   0.81
 326   0.83
 327   0.79
 328   0.77
 329   0.83
 330   0.83
 331   0.82
 332   0.77
 333   0.82
 334   0.81
 335   0.81
 336   0.76
 337   0.68
 338   0.71
 339   0.78
 340   0.80
 341   0.78
 342   0.66
 343   0.71
 344   0.71
 345   0.68
 346   0.74
 347   0.75
 348   0.72
 349   0.74
 350   0.72
 351   0.71
 352   0.74
 353   0.76
 354   0.81
 355   0.84
 356   0.83
 357   0.86
 358   0.87
 359   0.87
 360   0.89
 361   0.89
 362   0.88
 363   0.92
 364   0.97
 365   0.97
 366   0.99
 367   1.00
/
\plot
   1 0.48
   2 0.48
   3 0.53
   4 0.56
   5 0.57
   6 0.63
   7 0.62
   8 0.61
   9 0.56
  10 0.60
  11 0.57
  12 0.55
  13 0.50
  14 0.49
  15 0.50
  16 0.47
  17 0.34
  18 0.33
  19 0.36
  20 0.32
  21 0.31
  22 0.26
  23 0.27
  24 0.23
  25 0.21
  26 0.29
  27 0.24
  28 0.22
  29 0.17
  30 0.23
  31 0.23
  32 0.26
  33 0.24
  34 0.29
  35 0.30
  36 0.36
  37 0.34
  38 0.44
  39 0.42
  40 0.42
  41 0.41
  42 0.32
  43 0.37
  44 0.37
  45 0.34
  46 0.28
  47 0.29
  48 0.33
  49 0.30
  50 0.32
  51 0.32
  52 0.40
  53 0.41
  54 0.31
  55 0.32
  56 0.44
  57 0.46
  58 0.44
  59 0.44
  60 0.44
  61 0.54
  62 0.55
  63 0.58
  64 0.60
  65 0.54
  66 0.53
  67 0.53
  68 0.49
  69 0.54
  70 0.55
  71 0.57
  72 0.58
  73 0.57
  74 0.48
  75 0.52
  76 0.46
  77 0.58
  78 0.61
  79 0.56
  80 0.49
  81 0.44
  82 0.44
  83 0.37
  84 0.38
  85 0.40
  86 0.37
  87 0.44
  88 0.43
  89 0.38
  90 0.46
  91 0.49
  92 0.36
  93 0.23
  94 0.22
  95 0.23
  96 0.18
  97 0.18
  98 0.17
  99 0.21
 100 0.25
 101 0.16
 102 0.21
 103 0.30
 104 0.31
 105 0.31
 106 0.37
 107 0.37
 108 0.32
 109 0.29
 110 0.38
 111 0.37
 112 0.37
 113 0.38
 114 0.47
 115 0.51
 116 0.55
 117 0.57
 118 0.58
 119 0.51
 120 0.48
 121 0.51
 122 0.55
 123 0.62
 124 0.65
 125 0.60
 126 0.62
 127 0.52
 128 0.49
 129 0.53
 130 0.52
 131 0.51
 132 0.48
 133 0.55
 134 0.65
 135 0.64
 136 0.57
 137 0.62
 138 0.55
 139 0.52
 140 0.52
 141 0.50
 142 0.46
 143 0.48
 144 0.49
 145 0.46
 146 0.44
 147 0.40
 148 0.42
 149 0.41
 150 0.48
 151 0.55
 152 0.55
 153 0.49
 154 0.53
 155 0.54
 156 0.46
 157 0.40
 158 0.39
 159 0.40
 160 0.41
 161 0.36
 162 0.31
 163 0.35
 164 0.42
 165 0.37
 166 0.37
 167 0.29
 168 0.28
 169 0.30
 170 0.30
 171 0.36
 172 0.36
 173 0.33
 174 0.32
 175 0.32
 176 0.26
 177 0.26
 178 0.22
 179 0.15
 180 0.16
 181 0.15
 182 0.14
 183 0.17
 184 0.24
 185 0.30
 186 0.30
 187 0.29
 188 0.30
 189 0.31
 190 0.30
 191 0.32
 192 0.25
 193 0.29
 194 0.41
 195 0.51
 196 0.53
 197 0.61
 198 0.67
 199 0.68
 200 0.57
 201 0.54
 202 0.47
 203 0.44
 204 0.48
 205 0.45
 206 0.54
 207 0.57
 208 0.63
 209 0.63
 210 0.64
 211 0.69
 212 0.64
 213 0.68
 214 0.78
 215 0.73
 216 0.65
 217 0.68
 218 0.67
 219 0.62
 220 0.69
 221 0.68
 222 0.67
 223 0.69
 224 0.76
 225 0.78
 226 0.80
 227 0.78
 228 0.75
 229 0.67
 230 0.69
 231 0.57
 232 0.60
 233 0.60
 234 0.58
 235 0.65
 236 0.57
 237 0.56
 238 0.61
 239 0.60
 240 0.57
 241 0.54
 242 0.57
 243 0.55
 244 0.53
 245 0.51
 246 0.46
 247 0.55
 248 0.54
 249 0.52
 250 0.65
 251 0.57
 252 0.51
 253 0.56
 254 0.54
 255 0.54
 256 0.56
 257 0.59
 258 0.57
 259 0.58
 260 0.59
 261 0.62
 262 0.61
 263 0.55
 264 0.59
 265 0.62
 266 0.63
 267 0.64
 268 0.58
 269 0.56
 270 0.53
 271 0.54
 272 0.58
 273 0.54
 274 0.45
 275 0.44
 276 0.38
 277 0.39
 278 0.43
 279 0.42
 280 0.53
 281 0.48
 282 0.48
 283 0.52
 284 0.50
 285 0.49
 286 0.51
 287 0.50
 288 0.54
 289 0.61
 290 0.55
 291 0.52
 292 0.52
 293 0.57
 294 0.59
 295 0.65
 296 0.72
 297 0.69
 298 0.70
 299 0.68
 300 0.62
 301 0.69
 302 0.70
 303 0.67
 304 0.64
 305 0.62
 306 0.70
 307 0.72
 308 0.61
 309 0.47
 310 0.52
 311 0.52
 312 0.55
 313 0.55
 314 0.59
 315 0.61
 316 0.71
 317 0.77
 318 0.75
 319 0.74
 320 0.73
 321 0.69
 322 0.78
 323 0.70
 324 0.66
 325 0.70
 326 0.69
 327 0.75
 328 0.75
 329 0.81
 330 0.78
 331 0.75
 332 0.75
 333 0.81
 334 0.76
 335 0.75
 336 0.73
 337 0.71
 338 0.77
 339 0.73
 340 0.69
 341 0.71
 342 0.77
 343 0.74
 344 0.79
 345 0.83
 346 0.85
 347 0.81
 348 0.79
 349 0.78
 350 0.74
 351 0.80
 352 0.78
 353 0.72
 354 0.80
 355 0.86
 356 0.89
 357 0.88
 358 0.93
 359 0.95
 360 0.93
 361 0.93
 362 0.97
 363 0.95
 364 0.97
 365 0.98
 366 0.98
 367 0.98
/
\endpicture}
\let\put\latexput
\\
\end{frame}

\begin{frame}
\frametitle{PC map of European genetic distances}
\begin{columns}
\column{0.75\linewidth}
\let\put\pictexput
\mbox{\beginpicture
\setcoordinatesystem units <6in,2.70in> point at 0 0
\setplotarea x from -1 to -0.5, y from -0.61 to 0.5
\axis left  / 
\axis bottom / 
\axis top /
\axis right /
%Eigenvector plot from McLellan and Jorde 1984
\put{\footnotesize Mormon}      [tl] <1.4pt,-1.4pt> at -0.88627    0.18768  
\put{\footnotesize US}          [br] <-2pt,2pt>     at -0.96218   -0.15307  
\put{\footnotesize England}     [tl] <1.4pt,-1.4pt> at -0.90383    0.13026  
\put{\footnotesize Denmark}     [t]                 at -0.96385    0.17381  
\put{\footnotesize Germany}     [bl]                at -0.97664   -0.06721  
\put{\footnotesize Sweden}      [l]  <2pt,0pt>      at -0.91044    0.26839  
\put{\footnotesize Finland}     [t]                 at -0.60584    0.33713  
\put{\footnotesize Norway}      [b]                 at -0.91665    0.31107  
\put{\footnotesize Iceland}     [t]                 at -0.70262    0.48162  
\put{\footnotesize France}      [t]                 at -0.91940   -0.34180  
\put{\footnotesize Italy}       [b]  <0pt,2pt>      at -0.62085   -0.60186  
\put{\footnotesize Spain}       [b]  <0pt,2pt>      at -0.68657   -0.58174  
\put{\footnotesize Switzerland} [tl] <1.4pt,-1.4pt> at -0.93749   -0.15389  
\put{\footnotesize Poland}      [br] <-2pt,2pt>     at -0.49885   -0.53355  
\put{\footnotesize Netherlands} [bl] <2pt,0pt>      at -0.92026    0.20310  
%
\put{\footnotesize $\bullet$} at    -0.88627    0.18768  
\put{\footnotesize $\bullet$} at    -0.96218   -0.15307  
\put{\footnotesize $\bullet$} at    -0.90383    0.13026  
\put{\footnotesize $\bullet$} at    -0.96385    0.17381  
\put{\footnotesize $\bullet$} at    -0.97664   -0.06721  
\put{\footnotesize $\bullet$} at    -0.91044    0.26839  
\put{\footnotesize $\bullet$} at    -0.60584    0.33713  
\put{\footnotesize $\bullet$} at    -0.91665    0.31107  
\put{\footnotesize $\bullet$} at    -0.70262    0.48162  
\put{\footnotesize $\bullet$} at    -0.91940   -0.34180  
\put{\footnotesize $\bullet$} at    -0.62085   -0.60186  
\put{\footnotesize $\bullet$} at    -0.68657   -0.58174  
\put{\footnotesize $\bullet$} at    -0.93749   -0.15389  
\put{\footnotesize $\bullet$} at    -0.49885   -0.53355  
\put{\footnotesize $\bullet$} at    -0.92026    0.20310  
\endpicture}
\let\put\latexput

\column{0.25\linewidth}
\small(McLellan et al 1984)
\end{columns}
\end{frame}

\begin{frame}
\frametitle{Genetic and geographic distance}
\centering% -*-latex-*-
\let\put\pictexput
\mbox{\beginpicture
\footnotesize
\valuestolabelleading=.4\baselineskip
\headingtoplotskip=0.5\baselineskip
\setcoordinatesystem units <.000081in, 27.17in> point at 0 0
\setplotarea x from 0 to 3000, y from 0 to 0.036
\axis left shiftedto x=-1000 label {\lines{Genetic\cr Distance}} /
\axis bottom shiftedto y=-0.00184 label {km}
  ticks numbered from 0 to 3000 by  3000 /
\setplotsymbol ({\normalsize .})
\plotheading{Europe}
\multiput {\tiny .} at
 927.7 .0038 547.0 .0049 1282.5 .0047 1857.8 .0188 1112.1 .0042 1801.8
.0154 531.2 .0065 1432.1 .0185 1259.5 .0144 800.4 .0058 1290.0 .0193
342.0 .0054 532.6 .0035 468.1 .0020 974.9 .0120 613.6 .0021 2031.0
.0098 1166.0 .0070 1556.1 .0186 2051.2 .0135 1046.8 .0055 573.4 .0172
660.8 .0030 988.4 .0043 1505.3 .0112 1000.1 .0048 2134.8 .0134 636.8
.0040 1151.3 .0130 1524.4 .0096 555.6 .0041 749.7 .0142 205.2 .0035
582.0 .0116 398.6 .0015 1902.1 .0105 1607.6 .0081 2010.6 .0221 2482.2
.0164 1514.8 .0076 868.8 .0170 1082.1 .0047 905.5 .0133 2280.2 .0173
2140.7 .0178 2383.3 .0235 3026.1 .0293 1982.0 .0174 1114.4 .0207
1630.7 .0151 1510.6 .0095 1540.7 .0081 2128.4 .0222 2366.8 .0146
1555.7 .0074 1154.9 .0209 1016.6 .0035 2321.5 .0189 3222.4 .0355
2781.9 .0247 2583.1 .0172 2604.4 .0269 2004.8 .0117 966.0 .0074 887.8
.0051 418.5 .0030 1299.0 .0129 531.2 .0079 1334.3 .0101 639.5 .0089
1279.2 .0179 1230.0 .0186 1166.1 .0068 2152.3 .0161 1400.4 .0128 996.8
.0168 596.2 .0056 952.5 .0229
/
%Lowess fit
\plot
205.2000 0.002717108 266.7755 0.003260772 328.3510 0.003804436
389.9265 0.004344877 451.5020 0.004881002 513.0775 0.005413444
574.6531 0.005944437 636.2286 0.006476610 697.8041 0.007014341
759.3796 0.007556560 820.9551 0.008115626 882.5306 0.008689553
944.1061 0.009264819 1005.6816 0.009800767 1067.2571 0.010231105
1128.8326 0.010514077 1190.4081 0.010797760 1251.9836 0.011034321
1313.5591 0.011330924 1375.1346 0.011657787 1436.7101 0.012002361
1498.2856 0.012362163 1559.8611 0.012718594 1621.4366 0.013063781
1683.0122 0.013433511 1744.5876 0.013809916 1806.1632 0.014190137
1867.7386 0.014621114 1929.3142 0.015080296 1990.8896 0.015571038
2052.4653 0.016091982 2114.0408 0.016657911 2175.6162 0.017299863
2237.1919 0.018007144 2298.7673 0.018729437 2360.3428 0.019490987
2421.9182 0.020257005 2483.4937 0.021023206 2545.0693 0.021787230
2606.6448 0.022551112 2668.2202 0.023317162 2729.7959 0.024083214
2791.3713 0.024851639 2852.9468 0.025633130 2914.5222 0.026414620
2976.0979 0.027196113 3037.6733 0.027982548 3099.249 0.02879034
3160.824 0.02959814 3222.400 0.03040594
/
%%%%%%%%%%%%%%%%%%%%%%%%%%%%%%%%%%%%%%%%%%%%%%%%%%%%%%%%%%%%%%%%
\setcoordinatesystem units <.000081in, 3.7in> point at -10000 0
\setplotarea x from 0 to 10000, y from 0 to 0.27
\axis left shiftedto x=-1000 /
\axis bottom shiftedto y=-0.0135 label {km}
  ticks numbered from 0 to 9000 by  9000 /
\setplotsymbol ({\normalsize .})
\plotheading{N America}
\multiput {\tiny .} at
2457.699951 0.109755
2285.199951 0.062665
4556.200195 0.048638
1029.400024 0.153648
1966.800049 0.028279
2620.100098 0.073706
811.500000 0.202650
2139.100098 0.029832
2421.800049 0.096454
248.100006 0.012296
5834.299805 0.090920
3377.500000 0.033443
7821.000000 0.068065
5207.700195 0.070685
5422.200195 0.102459
5748.500000 0.246818
3292.699951 0.071468
7727.200195 0.148510
5115.000000 0.107546
5330.399902 0.123097
98.400002 0.036898
3149.899902 0.137060
910.700012 0.015629
5025.100098 0.082039
2418.600098 0.052643
2641.600098 0.059169
2796.399902 0.020311
2702.199951 0.035756
4917.299805 0.093983
2464.300049 0.028187
6985.700195 0.062720
4366.500000 0.066799
4566.399902 0.092693
972.599976 0.005002
908.900024 0.046007
2020.699951 0.016053
3893.399902 0.162661
2061.500000 0.036777
5354.500000 0.135683
2961.100098 0.074725
3208.800049 0.064721
2834.699951 0.047541
2737.500000 0.072780
1195.000000 0.027287
2423.000000 0.043805
3655.500000 0.169422
1806.900024 0.035719
5181.200195 0.111566
2741.100098 0.089373
2987.699951 0.098527
2883.899902 0.017907
2785.600098 0.025628
957.799988 0.016120
2384.399902 0.015119
260.399994 0.030729
363.000000 0.034850
2689.800049 0.045701
1952.300049 0.046153
1004.500000 0.053520
757.500000 0.084098
6050.000000 0.062050
5961.500000 0.167458
3313.500000 0.068561
5153.700195 0.064487
3958.100098 0.094727
3731.800049 0.117881
4340.299805 0.080789
1937.400024 0.037949
6493.100098 0.052627
3898.000000 0.105559
4076.199951 0.112096
1677.599976 0.043764
1620.400024 0.116993
1744.500000 0.044432
716.299988 0.028173
2496.100098 0.057969
2382.300049 0.045840
4603.600098 0.069544
1503.400024 0.094177
3020.500000 0.037152
1765.000000 0.046300
1085.800049 0.054295
1009.200012 0.084096
6120.700195 0.011520
6024.700195 0.043286
3334.000000 0.027540
5340.899902 0.015056
3589.500000 0.062602
3417.899902 0.040311
1200.400024 0.055283
4927.799805 0.054149
6307.799805 0.149555
3871.300049 0.079010
8186.399902 0.121272
5604.000000 0.123560
5830.600098 0.153050
656.400024 0.050734
689.200012 0.087830
3190.000000 0.049957
1585.199951 0.029420
2978.800049 0.098048
3083.300049 0.054662
6496.600098 0.123300
2301.000000 0.058055
6448.000000 0.058908
5233.100098 0.130420
2775.500000 0.027970
7250.700195 0.057321
4632.000000 0.054191
4841.200195 0.074431
608.000000 0.017143
533.200012 0.043927
2238.399902 0.016797
388.399994 0.009689
2450.899902 0.050711
2454.600098 0.019236
5455.200195 0.075560
1104.400024 0.035621
5572.399902 0.017986
1198.599976 0.047630
1622.000000 0.119303
1983.300049 0.022137
2876.399902 0.063611
597.799988 0.040608
814.200012 0.055781
4980.700195 0.012792
4884.700195 0.037274
2197.500000 0.011960
4211.600098 0.009844
2516.800049 0.029696
2322.100098 0.020633
1550.599976 0.063545
3832.199951 0.044405
1140.000000 0.009215
5314.399902 0.052432
4434.799805 0.009289
1947.900024 0.130019
1124.900024 0.029923
4229.600098 0.063973
1993.199951 0.079452
2042.099976 0.090146
4192.799805 0.030347
4120.899902 0.071265
2034.099976 0.027960
3224.399902 0.015049
3177.000000 0.052971
2919.300049 0.019083
2278.899902 0.093582
2559.000000 0.017786
3050.199951 0.044148
4761.100098 0.047372
3587.699951 0.010961
2321.500000 0.026208
1658.000000 0.070272
3319.800049 0.048885
1501.300049 0.007181
1373.400024 0.075363
1272.500000 0.085932
6418.899902 0.089173
6322.700195 0.167659
3634.800049 0.088546
5643.299805 0.084678
3860.800049 0.139269
3696.500000 0.133043
1322.300049 0.047265
5230.000000 0.070439
302.799988 0.063361
6739.000000 0.150298
5872.899902 0.081127
1438.699951 0.077704
3312.199951 0.087936
2993.500000 0.159061
3685.100098 0.035318
2739.399902 0.098673
2230.300049 0.033396
2306.199951 0.042980
6106.000000 0.078497
6007.899902 0.126310
3589.199951 0.038348
5539.200195 0.063555
3285.399902 0.071048
3222.699951 0.080301
2726.899902 0.058719
5332.899902 0.084515
1543.099976 0.080012
6256.000000 0.099637
5661.899902 0.047071
1815.300049 0.052552
4136.399902 0.059838
1547.199951 0.102569
2506.300049 0.093208
3966.500000 0.017148
1654.099976 0.028003
2114.800049 0.042100
2065.300049 0.043214
6834.299805 0.074435
6736.200195 0.146194
4126.299805 0.053084
6144.600098 0.061065
4106.399902 0.086364
3990.000000 0.088098
2166.600098 0.042363
5811.600098 0.047734
1059.199951 0.069520
7073.299805 0.116348
6329.700195 0.054148
1988.599976 0.057461
4100.700195 0.048153
848.599976 0.023286
1095.699951 0.041395
3012.699951 0.136693
5273.899902 0.064379
730.599976 0.045676
3323.800049 0.053725
3135.500000 0.058042
8504.099609 0.142875
8409.099609 0.215704
5708.700195 0.109975
7688.600098 0.129308
5957.299805 0.163454
5800.500000 0.179388
2682.600098 0.064262
7211.299805 0.127674
2395.399902 0.109068
8843.500000 0.192424
7942.899902 0.107273
3531.300049 0.106775
4958.799805 0.123443
2106.699951 0.030720
3084.699951 0.066672
2001.099976 0.025657
2413.000000 0.134088
4870.500000 0.026995
1286.699951 0.091116
3169.399902 0.032889
2926.500000 0.039654
8247.200195 0.045919
8161.100098 0.091377
5531.799805 0.030912
7327.100098 0.038164
6115.500000 0.022654
5904.799805 0.046928
2218.399902 0.057197
6731.500000 0.057721
2674.600098 0.048798
8711.400391 0.091561
7646.000000 0.032347
3608.600098 0.024027
4229.899902 0.043742
2497.199951 0.101756
3935.899902 0.032483
2888.699951 0.051644
1516.900024 0.091943
3628.100098 0.118171
4407.100098 0.031139
2936.899902 0.050136
2937.600098 0.029713
2992.199951 0.043439
6663.000000 0.072554
6565.899902 0.127833
4260.600098 0.043767
6163.299805 0.060264
3828.600098 0.094362
3801.800049 0.093746
3332.100098 0.042445
6004.000000 0.080646
2131.899902 0.059247
6752.299805 0.103796
6256.299805 0.045249
2545.899902 0.047877
4865.500000 0.061871
2054.199951 0.047805
732.299988 0.014895
1343.800049 0.020784
3109.600098 0.026592
4206.799805 0.047491
3760.600098 0.080316
3780.899902 0.053782
3782.300049 0.033919
2836.500000 0.038790
2984.699951 0.065994
5579.100098 0.089510
5483.899902 0.181586
3417.600098 0.081529
5173.700195 0.079366
2769.300049 0.132627
2789.699951 0.138461
3553.500000 0.026887
5117.200195 0.091776
2451.000000 0.066093
5618.700195 0.131077
5220.200195 0.067618
2285.399902 0.068785
4492.100098 0.088021
2523.699951 0.035427
1042.900024 0.040752
2135.800049 0.028050
4107.399902 0.021631
4968.399902 0.065718
1179.900024 0.012499
3100.699951 0.167586
5493.700195 0.047644
1044.599976 0.053121
3606.899902 0.060600
3391.199951 0.056504
8813.200195 0.120658
8721.000000 0.184066
6024.899902 0.085564
7949.100098 0.093287
6397.100098 0.125150
6220.399902 0.116036
2815.300049 0.094946
7417.700195 0.071162
2807.899902 0.108927
9203.200195 0.139403
8231.900391 0.064320
3907.300049 0.085744
5029.600098 0.053971
2545.699951 0.057497
3691.899902 0.046760
2596.899902 0.018218
671.400024 0.036274
1072.099976 0.069179
3765.500000 0.033006
4727.100098 0.044408
5870.000000 0.197447
3415.699951 0.060601
7834.600098 0.150715
5225.600098 0.115657
5443.000000 0.129905
109.599998 0.042021
133.199997 0.058558
2809.800049 0.028270
1042.099976 0.030179
2800.199951 0.053322
2859.100098 0.016427
6079.600098 0.138937
1753.599976 0.062237
6125.399902 0.075152
560.000000 0.052531
664.700012 0.030136
4985.500000 0.044434
4251.899902 0.025310
6422.500000 0.177694
6078.600098 0.063055
6823.100098 0.105756
8513.000000 0.200050
8282.099609 0.060074
6626.200195 0.090833
5533.399902 0.144391
8832.400391 0.117264
774.299988 0.069157
3206.899902 0.020824
1549.500000 0.012541
1506.000000 0.044819
1258.400024 0.064778
6574.200195 0.032344
6486.299805 0.084630
3842.800049 0.035373
5671.100098 0.032917
4467.100098 0.081765
4245.600098 0.065282
529.200012 0.030367
5107.700195 0.045236
1381.300049 0.018576
7025.200195 0.088627
5977.500000 0.030889
2009.900024 0.028625
2712.899902 0.047111
1387.099976 0.016695
2913.500000 0.062649
2164.000000 0.024263
2266.300049 0.048303
1689.199951 0.054213
3439.500000 0.031010
3827.899902 0.032178
2326.399902 0.060304
6605.200195 0.099670
3788.500000 0.103337
1360.099976 0.020078
5915.899902 0.056275
3315.399902 0.063544
3496.600098 0.081311
2107.199951 0.006645
2033.699951 0.039943
1215.900024 0.012761
1146.000000 0.002895
2138.699951 0.037467
1980.099976 0.010504
4040.300049 0.067851
584.400024 0.025289
4343.399902 0.016432
2677.600098 0.042953
1500.699951 0.007849
3250.600098 0.009206
2087.500000 0.011423
4645.600098 0.074508
4784.299805 0.063150
5232.899902 0.051039
6632.000000 0.116862
6192.000000 0.033500
5468.100098 0.058260
4628.899902 0.081051
6850.700195 0.080841
2164.500000 0.031091
4550.399902 0.027247
3263.899902 0.158101
1049.400024 0.061497
5482.700195 0.116985
2963.100098 0.098391
3106.399902 0.126960
2786.800049 0.020366
2720.600098 0.023529
1425.500000 0.025125
1814.199951 0.019708
2563.300049 0.067574
2354.500000 0.028894
3552.100098 0.112397
1132.300049 0.077330
4038.399902 0.024585
3381.800049 0.027149
2189.500000 0.034196
3030.300049 0.025220
1427.599976 0.056002
4333.200195 0.137930
4723.799805 0.102107
5007.899902 0.121961
6210.500000 0.188925
5627.100098 0.076288
5440.399902 0.100199
4743.500000 0.134976
6362.299805 0.168005
2853.199951 0.049366
4037.600098 0.064026
725.200012 0.025993
4386.299805 0.096916
1940.000000 0.019532
6363.799805 0.055690
3747.199951 0.060619
3960.300049 0.079632
1462.199951 0.014200
1371.400024 0.059610
1347.800049 0.018588
715.700012 0.004944
1736.199951 0.045230
1677.500000 0.018849
4591.899902 0.062041
860.900024 0.017997
4681.799805 0.024746
1931.900024 0.029713
890.700012 0.013014
3544.600098 0.016870
2864.699951 0.010702
4982.500000 0.076617
4824.899902 0.058561
5451.399902 0.046496
7053.399902 0.116811
6796.799805 0.035848
5447.700195 0.058912
4466.600098 0.077404
7351.399902 0.072915
1487.800049 0.036234
5117.799805 0.031624
860.099976 0.005250
1579.599976 0.031420
5065.100098 0.127656
2609.300049 0.031152
7060.899902 0.108508
4444.100098 0.075250
4656.100098 0.083087
770.099976 0.024937
683.500000 0.059207
2042.199951 0.015481
372.600006 0.022039
2247.899902 0.013088
2248.000000 0.019875
5279.899902 0.070477
1046.199951 0.036065
5375.100098 0.041665
1303.699951 0.060760
207.500000 0.029745
4236.500000 0.022820
3458.600098 0.035180
5675.100098 0.124082
5454.399902 0.058996
6125.100098 0.078320
7749.899902 0.156565
7477.799805 0.022172
6049.600098 0.081720
5016.899902 0.112896
8047.299805 0.116209
807.200012 0.035141
5804.000000 0.059407
1373.900024 0.019638
2081.699951 0.041098
697.099976 0.021457
4870.799805 0.125387
2413.100098 0.039297
6897.399902 0.078396
4277.700195 0.067666
4484.600098 0.098030
968.200012 0.007125
888.799988 0.026770
1897.300049 0.015585
218.600006 0.007255
2223.500000 0.055987
2196.399902 0.018610
5093.899902 0.080521
820.200012 0.058672
5229.799805 0.012832
1529.599976 0.041091
362.399994 0.011008
4094.699951 0.010143
3237.100098 0.029992
5531.100098 0.099288
5374.000000 0.065358
6007.899902 0.079797
7593.100098 0.135845
7283.600098 0.048694
5987.399902 0.059194
4982.500000 0.085457
7874.200195 0.112280
1017.500000 0.031445
5615.799805 0.038159
1150.000000 0.010151
1853.099976 0.012928
556.599976 0.020360
230.899994 0.034264
4838.500000 0.101352
2381.399902 0.027316
6883.399902 0.081053
4263.299805 0.066627
4467.200195 0.090884
1013.099976 0.004365
938.700012 0.038418
1898.400024 0.009723
149.699997 0.004851
2273.699951 0.034486
2237.300049 0.013609
5067.100098 0.062018
736.500000 0.042737
5226.200195 0.019709
1593.599976 0.042569
405.799988 0.016722
4093.699951 0.011667
3182.199951 0.027049
5528.100098 0.096932
5399.799805 0.057468
6017.500000 0.067660
7582.500000 0.135246
7250.600098 0.035720
6020.200195 0.059571
5026.100098 0.086205
7854.399902 0.110687
1069.800049 0.026220
5587.299805 0.037983
1095.000000 0.006725
1788.599976 0.018557
575.000000 0.013557
309.700012 0.016430
91.400002 0.006379
2227.399902 0.051373
4316.100098 0.066085
516.700012 0.004471
2350.500000 0.097914
2184.600098 0.127032
7494.899902 0.071836
7399.299805 0.163565
4703.100098 0.103798
6697.600098 0.070078
4940.000000 0.152490
4780.600098 0.123636
1869.000000 0.056599
6248.399902 0.058933
1377.300049 0.052734
7824.600098 0.134072
6940.700195 0.073361
2516.100098 0.076685
4122.899902 0.076670
1086.400024 0.014554
2223.899902 0.136516
1137.500000 0.046135
1020.500000 0.071548
1775.500000 0.112135
2432.300049 0.079169
3266.800049 0.054023
1509.300049 0.080703
7501.200195 0.173601
1586.900024 0.022710
5665.500000 0.064760
5289.500000 0.128051
6050.100098 0.064828
6745.299805 0.124339
6594.299805 0.090167
6587.200195 0.091519
4670.200195 0.058087
2219.899902 0.059318
6751.700195 0.065609
4134.200195 0.089601
4330.700195 0.133436
1220.199951 0.027450
1152.900024 0.101901
1816.699951 0.041167
251.100006 0.022648
2321.899902 0.088801
2258.399902 0.065491
4909.799805 0.042389
498.000000 0.054125
5120.399902 0.031954
1820.800049 0.046697
623.200012 0.036094
3996.100098 0.037283
2974.199951 0.051331
5423.000000 0.083220
5368.100098 0.062324
5942.700195 0.072648
7457.799805 0.116094
7078.799805 0.064472
6006.299805 0.048478
5044.700195 0.052959
7709.200195 0.113103
1285.599976 0.061033
5425.899902 0.040230
895.000000 0.033799
1567.699951 0.045984
585.700012 0.036090
548.200012 0.054436
323.399994 0.025296
238.500000 0.023273
6471.299805 0.075402
5107.700195 0.116046
2650.000000 0.035626
7136.500000 0.062347
4516.899902 0.062791
4724.100098 0.085615
738.900024 0.009006
665.400024 0.031009
2132.199951 0.028344
263.799988 0.012470
2398.600098 0.049024
2389.300049 0.023497
5332.500000 0.080432
976.900024 0.051019
5465.799805 0.006871
1329.300049 0.059855
132.399994 0.019127
4329.700195 0.008818
3455.600098 0.035523
5766.899902 0.081896
5584.899902 0.097432
6235.500000 0.078303
7831.399902 0.137937
7520.299805 0.048638
6188.500000 0.082547
5166.000000 0.097297
8113.700195 0.117778
797.099976 0.064835
5854.000000 0.032035
1368.800049 0.010962
2057.300049 0.023546
786.299988 0.019887
167.899994 0.036475
239.500000 0.014326
274.200012 0.018186
6831.600098 0.064490
491.600006 0.051367
4796.899902 0.098674
2350.899902 0.034260
6888.600098 0.058988
4272.100098 0.064307
4467.100098 0.093294
1127.500000 0.011810
1067.599976 0.055210
1959.699951 0.019144
165.000000 0.004184
2446.100098 0.055638
2389.600098 0.024250
5040.299805 0.063942
553.500000 0.034802
5261.500000 0.017108
1747.599976 0.033923
552.700012 0.005934
4138.200195 0.011375
3079.300049 0.015267
5564.200195 0.081886
5509.200195 0.050420
6085.700195 0.058220
7595.899902 0.114181
7203.399902 0.039154
6145.399902 0.044722
5177.899902 0.062320
7842.700195 0.079554
1200.900024 0.028709
5554.899902 0.032616
1012.799988 0.008328
1662.800049 0.027822
712.700012 0.012365
526.200012 0.032363
332.100006 0.006658
241.600006 0.009708
6611.000000 0.070857
143.000000 0.015279
428.700012 0.021357
1966.000000 0.107986
1020.200012 0.037686
4241.500000 0.066476
1955.699951 0.067963
2016.900024 0.098993
4117.200195 0.009571
4044.000000 0.049270
1929.800049 0.021815
3151.000000 0.003490
3071.899902 0.055371
2814.199951 0.019436
2289.199951 0.073095
2493.600098 0.037149
3022.199951 0.018007
4679.799805 0.035494
3511.100098 0.005310
2261.500000 0.011952
105.199997 0.014137
3289.199951 0.093820
4076.800049 0.058000
4068.199951 0.067285
4971.600098 0.130459
4274.799805 0.040445
4807.200195 0.055861
4413.100098 0.074465
5057.600098 0.087765
4174.600098 0.026047
2735.699951 0.038697
2010.400024 0.007397
1365.199951 0.026839
2777.100098 0.011574
3378.500000 0.031314
3158.500000 0.006125
3105.399902 0.009634
4121.100098 0.076587
2900.399902 0.021553
3379.100098 0.018332
3007.899902 0.001157
4870.799805 0.103032
2413.100098 0.027836
6897.399902 0.061070
4277.700195 0.061033
4484.600098 0.088659
968.200012 0.003610
888.799988 0.037438
1897.300049 0.016562
218.600006 0.001453
2223.500000 0.047911
2196.399902 0.014360
5093.899902 0.068077
820.200012 0.037487
5229.799805 0.012344
1529.599976 0.038254
362.399994 0.007674
4094.699951 0.008530
3237.100098 0.017061
5531.100098 0.083158
5374.000000 0.063979
6007.899902 0.061443
7593.100098 0.124069
7283.600098 0.038352
5987.399902 0.058040
4982.500000 0.077755
7874.200195 0.092194
1017.500000 0.031948
5615.799805 0.030332
1150.000000 0.002630
1853.099976 0.020047
556.599976 0.007967
230.899994 0.026742
0.000000 0.004236
91.400002 0.005181
6594.299805 0.068716
323.399994 0.026300
239.500000 0.009367
332.100006 0.004612
3158.500000 0.003108
4571.200195 0.091554
2125.800049 0.017657
6666.600098 0.052620
4051.600098 0.076256
4244.899902 0.090044
1335.000000 0.014303
1269.000000 0.055578
1760.500000 0.013938
362.899994 0.008743
2326.699951 0.047223
2250.000000 0.017590
4815.000000 0.063981
384.899994 0.019772
5047.100098 0.029068
1938.300049 0.044416
740.400024 0.018125
3927.399902 0.021752
2861.699951 0.013676
5349.899902 0.064419
5328.399902 0.062952
5882.600098 0.040857
7375.299805 0.110828
6977.899902 0.048709
5974.600098 0.054898
5029.500000 0.080452
7618.299805 0.076334
1401.900024 0.033660
5329.299805 0.026309
788.299988 0.005316
1452.000000 0.035122
599.099976 0.007836
663.299988 0.026571
435.600006 0.019558
354.000000 0.011035
6392.899902 0.061983
117.699997 0.033140
609.000000 0.025477
225.699997 0.014869
2788.899902 0.016570
435.600006 0.010746
5357.299805 0.190115
2906.399902 0.046707
7318.000000 0.143183
4707.799805 0.105648
4925.000000 0.114305
507.899994 0.028162
411.200012 0.032651
2292.899902 0.024574
649.200012 0.028977
2358.300049 0.033091
2392.100098 0.006528
5563.500000 0.128757
1353.500000 0.057094
5613.500000 0.057972
986.000000 0.067377
283.899994 0.036242
4473.500000 0.037262
3775.100098 0.036191
5911.500000 0.167482
5614.200195 0.093543
6328.799805 0.112179
7998.500000 0.218658
7768.399902 0.056219
6184.500000 0.122125
5119.100098 0.172116
8314.500000 0.147792
518.000000 0.024802
6089.500000 0.086141
1692.300049 0.023923
2401.300049 0.037668
971.799988 0.028377
319.500000 0.016633
549.200012 0.035484
621.799988 0.023477
6988.299805 0.156673
858.099976 0.086931
387.000000 0.038679
813.200012 0.044435
3694.199951 0.037243
549.200012 0.029687
975.000000 0.030598
4340.299805 0.102087
1937.400024 0.033779
6493.100098 0.070274
3898.000000 0.107594
4076.199951 0.107460
1677.599976 0.047375
1620.400024 0.111325
1744.500000 0.038745
716.299988 0.029147
2496.100098 0.050977
2382.300049 0.037060
4603.600098 0.084248
0.000000 0.004096
4927.799805 0.066747
2301.000000 0.054211
1104.400024 0.038445
3832.199951 0.047686
2559.000000 0.014306
5230.000000 0.084989
5332.899902 0.078011
5811.600098 0.047006
7211.299805 0.134225
6731.500000 0.056679
6004.000000 0.081205
5117.200195 0.104558
7417.700195 0.068941
1753.599976 0.047054
5107.700195 0.056267
584.400024 0.024947
1132.300049 0.076263
860.900024 0.017412
1046.199951 0.033831
820.200012 0.059246
736.500000 0.040235
6248.399902 0.079774
498.000000 0.061589
976.900024 0.057872
553.500000 0.037352
2493.600098 0.039449
820.200012 0.038461
384.899994 0.016284
1353.500000 0.047283
4340.299805 0.094024
1937.400024 0.048864
6493.100098 0.026604
3898.000000 0.058296
4076.199951 0.080218
1677.599976 0.060035
1620.400024 0.133587
1744.500000 0.063018
716.299988 0.041946
2496.100098 0.098234
2382.300049 0.084108
4603.600098 0.056561
0.000000 0.036143
4927.799805 0.039874
2301.000000 0.074648
1104.400024 0.027968
3832.199951 0.039543
2559.000000 0.034825
5230.000000 0.048271
5332.899902 0.060510
5811.600098 0.039843
7211.299805 0.067602
6731.500000 0.051248
6004.000000 0.041304
5117.200195 0.033700
7417.700195 0.038728
1753.599976 0.098103
5107.700195 0.031283
584.400024 0.042843
1132.300049 0.093100
860.900024 0.036284
1046.199951 0.073753
820.200012 0.058572
736.500000 0.064477
6248.399902 0.039811
498.000000 0.049760
976.900024 0.054086
553.500000 0.031940
2493.600098 0.036720
820.200012 0.044301
384.899994 0.051431
1353.500000 0.112203
0.000000 0.051411
3939.199951 0.047484
3391.800049 0.031195
4396.799805 0.048897
2930.100098 0.067603
3128.699951 0.092493
4793.899902 0.030236
4700.299805 0.113940
2860.800049 0.053067
4456.700195 0.030488
2033.699951 0.062445
2100.100098 0.060566
3804.100098 0.032578
4477.700195 0.030466
2856.399902 0.039606
4804.100098 0.077463
4469.299805 0.052362
2333.699951 0.043133
4267.200195 0.049985
3004.300049 0.063327
1727.400024 0.091620
2802.100098 0.050519
4802.700195 0.117179
5500.899902 0.050600
2004.099976 0.083093
828.200012 0.076781
5390.299805 0.103189
4742.799805 0.099199
4168.700195 0.033823
4036.199951 0.030894
4259.600098 0.072555
3765.000000 0.021530
4270.100098 0.039800
4256.299805 0.058148
4307.299805 0.037825
3892.399902 0.046930
4349.899902 0.053993
4425.200195 0.037234
4476.899902 0.046520
4176.000000 0.046344
4256.299805 0.035266
4347.399902 0.031334
4348.500000 0.064070
4477.700195 0.038417
4477.700195 0.054518
4071.199951 0.115414
1679.199951 0.018467
6232.000000 0.062726
3644.100098 0.062968
3818.000000 0.078259
1925.500000 0.010994
1863.300049 0.042179
1563.000000 0.013837
954.400024 0.004291
2419.500000 0.040152
2280.800049 0.008862
4336.700195 0.076832
270.399994 0.025165
4682.500000 0.024145
2535.800049 0.037484
1337.400024 0.010120
3597.500000 0.013064
2292.100098 0.008366
4984.100098 0.081652
5137.600098 0.061920
5582.000000 0.050314
6951.899902 0.123574
6461.200195 0.038510
5819.700195 0.060934
4968.100098 0.087685
7151.399902 0.076844
1996.500000 0.027255
4839.299805 0.034017
353.799988 0.002095
867.700012 0.027808
919.299988 0.003202
1253.300049 0.022975
1022.500000 0.013920
947.700012 0.009790
5994.399902 0.072216
715.299988 0.041411
1206.500000 0.015159
798.299988 0.011121
2225.199951 0.009482
1022.500000 0.004791
597.599976 0.005233
1569.400024 0.021149
270.399994 0.021417
270.399994 0.046574
4358.200195 0.032331
3933.800049 0.132565
3696.600098 0.018769
4121.600098 0.033831
2965.699951 0.034981
3137.199951 0.045845
5272.299805 0.033959
5178.299805 0.052790
3246.500000 0.033973
4915.299805 0.031886
2494.800049 0.071866
2542.899902 0.045076
3755.399902 0.074212
4905.899902 0.057495
2707.199951 0.020810
5284.200195 0.083706
4939.100098 0.024639
2387.199951 0.020888
4490.299805 0.037933
2810.199951 0.042550
1390.599976 0.074474
2486.199951 0.037745
4470.600098 0.072160
5283.299805 0.052151
1555.000000 0.049631
379.899994 0.063705
5082.500000 0.060951
5222.100098 0.088884
4067.800049 0.016342
4443.000000 0.023145
4615.799805 0.054055
4217.100098 0.027622
4738.500000 0.061627
4717.899902 0.034609
4766.399902 0.041318
3608.300049 0.044540
4800.200195 0.073615
4891.899902 0.018351
4929.700195 0.036581
4405.200195 0.037215
4717.899902 0.025870
4792.799805 0.030893
4823.600098 0.065917
4905.899902 0.061796
4905.899902 0.040983
480.700012 0.046107
4773.399902 0.023557
1896.699951 0.076299
568.799988 0.010074
4045.199951 0.025024
1518.099976 0.045564
1655.099976 0.053262
3943.300049 0.025258
3859.300049 0.075539
1402.300049 0.027196
3020.600098 0.022114
2452.899902 0.047004
2192.500000 0.040558
2144.600098 0.042987
2459.899902 0.025489
2601.000000 0.021797
4440.100098 0.077645
3339.500000 0.026481
1676.400024 0.018797
803.900024 0.028342
2891.399902 0.029506
3470.699951 0.066954
3601.500000 0.022780
4769.899902 0.070740
4307.799805 0.039429
4202.899902 0.045521
3730.300049 0.054062
4957.899902 0.058983
3982.899902 0.078630
2654.300049 0.010413
1896.500000 0.015190
1451.300049 0.059018
2508.899902 0.017800
3176.100098 0.039466
2977.100098 0.036533
2942.800049 0.028732
3838.600098 0.031447
2773.600098 0.050527
3213.000000 0.020479
2901.199951 0.029092
708.900024 0.032244
2977.100098 0.022352
2675.600098 0.015315
3474.600098 0.058688
2459.899902 0.030333
2459.899902 0.034243
3468.300049 0.020328
2194.100098 0.017620
3706.699951 0.012764
2371.100098 0.149479
2082.399902 0.077994
3433.699951 0.136739
1342.199951 0.070174
1566.500000 0.101399
4609.399902 0.039698
4511.200195 0.067976
1959.699951 0.060720
3950.399902 0.043510
1930.400024 0.057601
1782.500000 0.064252
2300.199951 0.097221
3697.699951 0.106601
1673.400024 0.033851
4860.299805 0.073724
4113.500000 0.061260
752.700012 0.030738
2739.699951 0.092536
1934.400024 0.159636
1642.400024 0.117185
2225.300049 0.142678
4026.899902 0.200899
4291.500000 0.057460
2343.399902 0.122387
1754.199951 0.131332
4478.100098 0.194555
4597.899902 0.103233
2743.699951 0.087005
3143.500000 0.054299
3101.399902 0.039146
3245.300049 0.056089
3907.500000 0.052624
3801.199951 0.044761
3817.399902 0.044666
3009.899902 0.140183
3763.300049 0.075416
4024.000000 0.029825
3905.899902 0.055153
2659.399902 0.051350
3801.199951 0.044075
3714.899902 0.081305
4104.399902 0.071782
3697.699951 0.115727
3697.699951 0.095645
1628.900024 0.063284
3497.000000 0.058456
1765.400024 0.069787
1977.599976 0.069690
3164.300049 0.085784
2249.100098 0.014716
4223.799805 0.073810
2139.300049 0.051548
2375.600098 0.060798
4073.399902 0.020938
3975.399902 0.074908
1763.900024 0.014009
3556.899902 0.017254
1254.599976 0.017187
1193.900024 0.027001
3122.600098 0.041935
3459.399902 0.032037
2482.699951 0.035687
4232.700195 0.066625
3646.600098 0.028114
1572.800049 0.017437
3159.899902 0.029593
2724.100098 0.079190
2034.099976 0.040847
2871.500000 0.041328
4778.700195 0.103754
5124.899902 0.018056
2610.899902 0.046344
1659.599976 0.068131
5266.899902 0.082376
4044.800049 0.040377
3577.800049 0.037091
2970.699951 0.015561
3131.600098 0.050856
2842.399902 0.018939
3439.500000 0.012802
3377.600098 0.029479
3412.399902 0.011552
3774.899902 0.086738
3410.199951 0.035046
3577.600098 0.034002
3546.899902 0.022743
3064.800049 0.025579
3377.600098 0.020586
3387.100098 0.016276
3584.899902 0.036679
3459.399902 0.028549
3459.399902 0.057701
1143.199951 0.028510
3308.600098 0.018311
1487.000000 0.046372
2356.199951 0.020174
837.400024 0.057507
2056.500000 0.138440
889.099976 0.027003
4321.700195 0.054395
1978.199951 0.050994
2056.199951 0.070391
3982.300049 0.018158
3908.300049 0.039811
1796.300049 0.021906
3017.699951 0.011016
2946.800049 0.057218
2690.199951 0.019217
2370.399902 0.085790
2366.399902 0.041060
3054.800049 0.018085
4541.500000 0.049884
3375.600098 0.002797
2248.000000 0.010210
240.699997 0.011699
3327.899902 0.077742
4060.699951 0.058510
4092.899902 0.053747
5052.200195 0.107900
4388.000000 0.039605
4792.500000 0.051229
4362.100098 0.073218
5154.399902 0.064194
4038.600098 0.037765
2829.300049 0.031198
1875.099976 0.008271
1242.599976 0.034792
2636.300049 0.013052
3241.100098 0.039797
3021.899902 0.012598
2969.800049 0.020750
4185.600098 0.067258
2766.800049 0.046738
3243.800049 0.014194
2876.000000 0.009850
142.100006 0.007996
3021.899902 0.007424
2656.000000 0.019296
3556.199951 0.038920
2366.399902 0.042571
2366.399902 0.031397
4099.700195 0.051692
2097.199951 0.008225
4341.899902 0.015489
635.299988 0.023370
2608.100098 0.059252
2979.800049 0.034124
2215.899902 0.167584
901.700012 0.033422
4002.199951 0.111946
1401.400024 0.052102
1631.400024 0.063952
3819.199951 0.029412
3725.100098 0.025522
1023.000000 0.011700
3017.399902 0.035242
1711.000000 0.041577
1460.300049 0.034594
2334.199951 0.083010
2637.100098 0.094961
2323.500000 0.030382
4202.700195 0.076007
3256.100098 0.039047
1200.000000 0.020571
1716.599976 0.068843
2625.800049 0.112703
2796.800049 0.059706
3174.600098 0.086112
4687.500000 0.128527
4546.799805 0.047034
3513.300049 0.061291
2890.899902 0.101154
5004.299805 0.134918
3832.600098 0.057894
2860.899902 0.049359
2058.600098 0.032887
1927.500000 0.023680
2366.000000 0.045425
3062.399902 0.034436
2909.100098 0.021795
2903.399902 0.020464
3685.300049 0.134140
2798.300049 0.059479
3146.300049 0.033200
2939.899902 0.040576
1620.800049 0.042892
2909.100098 0.031253
2727.899902 0.040125
3315.800049 0.041825
2637.100098 0.089645
2637.100098 0.101577
2565.199951 0.076143
2408.800049 0.036876
2830.199951 0.043255
912.700012 0.047680
1181.000000 0.048496
1444.199951 0.032810
1536.599976 0.041689
5131.000000 0.106273
2679.800049 0.020940
7099.399902 0.067286
4486.899902 0.076990
4702.799805 0.089922
721.599976 0.015452
628.099976 0.055556
2074.800049 0.012410
507.000000 0.007808
2204.800049 0.036664
2220.399902 0.011895
5338.200195 0.069355
1174.599976 0.015288
5401.100098 0.032610
1207.099976 0.040420
262.000000 0.013241
4261.399902 0.020010
3555.000000 0.008573
5699.899902 0.085633
5437.200195 0.055904
6130.799805 0.050393
7782.799805 0.129027
7542.299805 0.038541
6020.799805 0.061463
4973.200195 0.089618
8093.100098 0.078047
741.700012 0.022572
5864.000000 0.037212
1476.599976 0.005361
2191.199951 0.035663
746.200012 0.005335
134.699997 0.014766
354.600006 0.020800
439.100006 0.010878
6774.500000 0.079600
676.900024 0.036932
282.000000 0.027732
660.599976 0.013524
3473.000000 0.013342
354.600006 0.011266
790.200012 0.004030
226.600006 0.019354
1174.599976 0.012301
1174.599976 0.048130
4216.100098 0.032971
1374.199951 0.004682
4687.500000 0.039002
3248.100098 0.022360
3908.699951 0.074499
3414.199951 0.014781
3334.600098 0.017706
3097.600098 0.042947
4128.899902 0.143854
3081.800049 0.031604
4935.200195 0.118764
3099.699951 0.058431
3328.000000 0.057784
4070.399902 0.042957
3978.199951 0.081008
2398.699951 0.019373
3792.399902 0.032703
1392.500000 0.008926
1518.300049 0.032485
4060.000000 0.073966
3886.199951 0.048422
3277.699951 0.054771
4062.699951 0.070933
3774.600098 0.035445
2514.699951 0.022714
4081.000000 0.039317
3474.500000 0.126753
2384.399902 0.037939
3411.800049 0.069652
5404.100098 0.137740
5952.299805 0.016860
2735.899902 0.061267
1568.300049 0.094554
5957.600098 0.096527
4015.500000 0.036702
4487.000000 0.070866
3498.100098 0.033011
3824.199951 0.059796
3122.500000 0.034818
3580.199951 0.013364
3585.100098 0.043135
3642.699951 0.026706
4448.500000 0.141214
3707.199951 0.058388
3739.800049 0.051520
3827.399902 0.035178
3981.699951 0.037446
3585.100098 0.038318
3716.699951 0.040030
3637.699951 0.039377
3886.199951 0.042740
3886.199951 0.069741
743.299988 0.060672
3797.300049 0.033664
1222.000000 0.069426
3280.500000 0.045595
1762.000000 0.055344
986.400024 0.011747
3884.100098 0.044996
2373.600098 0.039383
3517.399902 0.028051
1990.199951 0.110814
831.400024 0.021476
4245.600098 0.062816
1876.900024 0.051879
1960.699951 0.074743
3998.600098 0.006703
3922.800049 0.039156
1742.099976 0.011222
3038.600098 0.002547
2878.500000 0.040731
2620.699951 0.012318
2296.699951 0.065011
2399.899902 0.035051
2956.000000 0.015782
4548.200195 0.039861
3391.000000 0.004640
2139.500000 0.006811
299.000000 0.013006
3231.199951 0.083261
3951.800049 0.048246
3991.300049 0.053481
4976.200195 0.115474
4340.799805 0.030835
4683.600098 0.047823
4254.399902 0.071066
5090.600098 0.078924
4052.199951 0.025010
2764.300049 0.030491
1893.099976 0.003047
1286.400024 0.023995
2633.399902 0.007193
3251.199951 0.022191
3034.699951 0.005430
2985.500000 0.005132
4100.299805 0.074926
2787.500000 0.027665
3259.699951 0.013626
2900.300049 0.003843
194.399994 0.002928
3034.699951 0.001498
2678.399902 0.010127
3564.300049 0.026518
2399.899902 0.034351
2399.899902 0.040696
3998.000000 0.037647
2129.899902 0.003913
4236.500000 0.025596
530.700012 0.021713
2500.300049 0.047457
2881.199951 0.016141
109.300003 0.005826
1437.199951 0.028250
3341.199951 0.008327
3792.000000 0.028552
4840.200195 0.104330
2391.399902 0.024984
6924.000000 0.073244
4306.399902 0.081454
4503.100098 0.095406
1072.199951 0.019466
1011.200012 0.062053
1981.599976 0.017489
107.099998 0.007738
2438.699951 0.045212
2388.500000 0.017370
5081.200195 0.077170
610.799988 0.020682
5290.700195 0.037394
1690.199951 0.024142
494.899994 0.022483
4165.100098 0.024479
3130.899902 0.013512
5593.299805 0.089201
5521.200195 0.065622
6107.799805 0.053858
7629.899902 0.133331
7248.000000 0.048663
6153.299805 0.066464
5177.799805 0.094854
7881.399902 0.085972
1144.400024 0.028015
5596.799805 0.044072
1059.699951 0.008886
1716.400024 0.029723
711.299988 0.004935
472.100006 0.025054
287.899994 0.022919
200.300003 0.012993
6642.799805 0.082915
172.399994 0.036282
370.899994 0.030092
57.900002 0.016028
3058.899902 0.016870
287.899994 0.012767
271.399994 0.005215
755.799988 0.028682
610.799988 0.014016
610.799988 0.055611
4471.100098 0.033505
853.299988 0.005231
4926.100098 0.041630
2944.000000 0.025566
3922.699951 0.069918
3551.600098 0.018865
2926.500000 0.022856
2968.100098 0.045980
606.700012 0.005641
3816.300049 0.034817
2949.600098 0.011959
6080.600098 0.149582
3641.899902 0.040063
7973.600098 0.098375
5385.000000 0.080658
5610.200195 0.103044
479.700012 0.015117
490.200012 0.033629
2968.699951 0.014094
1366.000000 0.008476
2801.899902 0.047662
2894.600098 0.009057
6272.000000 0.100252
2082.300049 0.048427
6240.500000 0.031649
230.399994 0.035122
978.000000 0.009967
5104.700195 0.016773
4531.399902 0.016653
6533.000000 0.122886
6085.700195 0.057923
6883.500000 0.081313
8635.400391 0.154787
8485.700195 0.045968
6598.100098 0.065936
5476.200195 0.103287
8987.000000 0.101813
372.600006 0.009819
6800.299805 0.056320
2449.000000 0.010307
3155.300049 0.021198
1702.199951 0.017085
1075.300049 0.027364
1302.300049 0.006552
1368.300049 0.008497
7617.600098 0.114326
1597.800049 0.034719
1107.199951 0.027070
1529.599976 0.007802
4449.899902 0.006122
1302.300049 0.007858
1715.400024 0.016937
756.700012 0.022780
2082.300049 0.042296
2082.300049 0.063469
4668.299805 0.066610
2313.000000 0.010057
5149.000000 0.047546
4210.700195 0.045079
4665.299805 0.062465
4056.500000 0.027473
4311.500000 0.013333
3984.199951 0.032691
976.900024 0.012478
3930.100098 0.033083
4317.899902 0.005613
1471.900024 0.015431
1186.800049 0.169284
2390.500000 0.013474
2275.199951 0.069474
458.899994 0.030938
455.000000 0.025236
5551.200195 0.040883
5456.600098 0.044926
2755.100098 0.020085
4742.799805 0.035927
3146.699951 0.034480
2949.199951 0.032572
1012.400024 0.087252
4307.100098 0.060645
634.099976 0.032632
5912.200195 0.086982
4989.500000 0.027522
630.299988 0.015104
2447.899902 0.038808
929.900024 0.071328
1854.000000 0.058361
1656.199951 0.045240
2953.899902 0.088727
2980.199951 0.033454
2537.199951 0.052023
2573.899902 0.082879
3294.699951 0.067698
5562.200195 0.069738
1414.800049 0.035443
3722.800049 0.025737
3404.500000 0.053005
4099.600098 0.032211
4796.000000 0.041028
4641.100098 0.038065
4632.799805 0.037630
1954.800049 0.088043
4517.000000 0.086891
4878.899902 0.024961
4657.000000 0.042134
2412.699951 0.043220
4641.100098 0.032307
4439.799805 0.034698
5046.500000 0.048183
4307.100098 0.057575
4307.100098 0.058338
2791.699951 0.059233
4058.600098 0.024749
2753.600098 0.012073
1967.599976 0.019286
1332.699951 0.060517
2169.500000 0.032556
2436.899902 0.021902
1733.599976 0.027659
4829.700195 0.036250
3072.800049 0.036658
2335.800049 0.026812
4688.399902 0.040410
5698.600098 0.042240
3961.899902 0.136608
1830.300049 0.023371
5632.600098 0.060065
3116.100098 0.079948
3357.100098 0.079628
2352.000000 0.031895
2253.699951 0.061021
920.599976 0.023639
1865.900024 0.018946
565.099976 0.043792
535.400024 0.016877
4079.199951 0.098098
1939.099976 0.017501
3878.800049 0.041471
2580.899902 0.057820
1920.400024 0.015410
2757.100098 0.023698
2950.899902 0.004526
4165.399902 0.075702
3756.500000 0.070568
4502.200195 0.044614
6272.100098 0.118511
6283.299805 0.046859
4336.600098 0.066364
3311.199951 0.096831
6664.000000 0.057613
2330.500000 0.035329
4603.399902 0.041171
1620.199951 0.010852
2124.300049 0.054656
1172.099976 0.013902
1714.300049 0.034278
1671.500000 0.033262
1717.300049 0.029100
5251.799805 0.071969
1757.699951 0.063776
1858.000000 0.030001
1883.300049 0.023087
2847.100098 0.022875
1671.500000 0.020541
1761.599976 0.011397
1857.699951 0.034142
1939.099976 0.013626
1939.099976 0.042659
2593.800049 0.050267
1878.400024 0.008564
3049.600098 0.028755
2312.699951 0.019377
2281.500000 0.093488
1729.300049 0.028579
2711.800049 0.014003
1748.500000 0.060129
1685.000000 0.009816
1956.400024 0.039463
2661.000000 0.016974
1877.500000 0.014791
2383.899902 0.022792
3372.500000 0.025697
3669.199951 0.150087
2460.800049 0.028261
4745.299805 0.099673
2651.699951 0.047210
2892.100098 0.064029
3746.199951 0.016413
3649.899902 0.034173
1794.199951 0.009718
3343.100098 0.012996
926.799988 0.025752
977.200012 0.013318
3643.899902 0.079666
3357.300049 0.057307
3013.000000 0.026054
3837.399902 0.051662
3377.500000 0.012013
2098.100098 0.007354
3471.300049 0.027154
3248.600098 0.117999
2428.800049 0.041356
3345.000000 0.073005
5283.799805 0.133653
5659.500000 0.021064
2923.500000 0.054587
1843.000000 0.087723
5786.500000 0.097148
3705.600098 0.022246
4106.700195 0.051377
2931.500000 0.013568
3217.399902 0.026279
2645.399902 0.020497
3174.699951 0.016820
3146.899902 0.010585
3194.300049 0.008930
4288.000000 0.118595
3229.399902 0.042273
3325.199951 0.022548
3357.899902 0.013885
3370.899902 0.012442
3146.899902 0.011415
3225.000000 0.026353
3278.699951 0.023567
3357.300049 0.053032
3357.300049 0.060845
1123.000000 0.057487
3245.500000 0.014961
1572.400024 0.040019
2674.399902 0.038465
1371.099976 0.037865
534.599976 0.017127
3270.199951 0.015786
1775.900024 0.018519
3130.300049 0.019233
621.299988 0.015971
3180.000000 0.006881
3353.800049 0.024998
3676.800049 0.008552
2701.800049 0.025607
1477.199951 0.030639
1291.300049 0.099849
3492.500000 0.026425
1064.400024 0.048032
1572.000000 0.047883
1362.800049 0.052023
6779.600098 0.054028
6687.000000 0.092718
3989.399902 0.028583
5929.200195 0.052375
4419.899902 0.064889
4225.799805 0.078196
934.200012 0.038739
5429.000000 0.069692
931.799988 0.036934
7169.500000 0.104839
6202.200195 0.049308
1904.199951 0.034205
3202.500000 0.071878
808.099976 0.036130
2349.500000 0.047486
1489.300049 0.034086
1788.800049 0.047344
1749.199951 0.051872
2803.600098 0.023042
3331.300049 0.039189
2035.599976 0.075637
6797.600098 0.101668
704.799988 0.018072
4852.500000 0.046476
4422.100098 0.071476
5318.799805 0.056784
6015.399902 0.059582
5846.500000 0.049738
5829.799805 0.045128
936.299988 0.069459
5693.200195 0.047932
6085.799805 0.056486
5829.200195 0.047952
3206.399902 0.059878
5846.500000 0.051840
5606.500000 0.042057
6279.799805 0.100594
5429.000000 0.074584
5429.000000 0.059226
3784.100098 0.068009
5167.600098 0.055373
3612.100098 0.040944
2981.500000 0.027892
2551.699951 0.096620
3380.600098 0.034392
3278.699951 0.055963
2968.899902 0.034142
6058.899902 0.053192
4205.700195 0.053184
3198.300049 0.046859
5865.600098 0.059330
6952.399902 0.066339
1277.099976 0.036957
4646.799805 0.059667
3914.600098 0.052272
1424.199951 0.154097
1639.300049 0.022537
3032.800049 0.087048
451.899994 0.054550
697.700012 0.058663
4788.200195 0.024519
4694.399902 0.025622
1992.500000 0.004787
3970.500000 0.023994
2512.199951 0.032960
2290.000000 0.022292
1447.000000 0.082866
3539.399902 0.064940
1388.599976 0.024791
5168.100098 0.064247
4221.200195 0.024453
417.100006 0.011814
1904.599976 0.042776
1690.599976 0.088158
2231.899902 0.055260
2335.500000 0.064239
3722.100098 0.111979
3621.100098 0.041742
2961.800049 0.050791
2672.699951 0.090577
4036.100098 0.102311
4802.299805 0.044110
1955.800049 0.037186
2955.000000 0.019170
2673.600098 0.023729
3332.500000 0.031849
4029.399902 0.029318
3871.100098 0.016253
3861.500000 0.015686
2727.199951 0.107339
3744.300049 0.052006
4109.299805 0.024692
3884.300049 0.027263
1845.400024 0.030487
3871.100098 0.021664
3667.100098 0.023149
4285.299805 0.036218
3539.399902 0.059686
3539.399902 0.076597
2645.800049 0.068331
3293.899902 0.022134
2746.699951 0.030114
1279.599976 0.030018
1018.099976 0.057106
1753.900024 0.023682
1835.099976 0.026058
969.700012 0.005967
4066.699951 0.026745
2732.899902 0.030288
1727.000000 0.018254
3915.699951 0.029738
4951.100098 0.021793
772.700012 0.015858
2666.000000 0.032367
2247.899902 0.014803
2001.400024 0.025071
5308.100098 0.148461
2850.500000 0.035762
7334.399902 0.060612
4715.200195 0.065680
4923.200195 0.083361
545.799988 0.026844
479.799988 0.047279
2325.199951 0.025025
429.100006 0.015746
2539.600098 0.062372
2544.800049 0.025161
5533.000000 0.092196
1142.000000 0.033695
5659.200195 0.023015
1160.000000 0.040035
90.800003 0.005063
4522.100098 0.017176
3647.199951 0.014061
5960.000000 0.088378
5752.700195 0.064212
6419.399902 0.064580
8028.000000 0.121168
7720.600098 0.043933
6346.600098 0.062942
5308.899902 0.085253
8313.299805 0.069778
613.000000 0.042269
6054.500000 0.037637
1562.500000 0.013364
2242.000000 0.034458
977.500000 0.013066
298.299988 0.035163
439.500000 0.021462
471.700012 0.029022
7026.700195 0.076387
674.700012 0.052480
200.600006 0.022343
588.599976 0.015632
3572.100098 0.014007
439.500000 0.015264
790.000000 0.024551
312.299988 0.038550
1142.000000 0.038173
1142.000000 0.027917
4556.200195 0.054860
1383.500000 0.013783
5026.600098 0.025146
3413.199951 0.031999
4203.700195 0.066031
3737.000000 0.043544
3437.399902 0.006339
3341.699951 0.050990
342.799988 0.017410
3859.600098 0.048719
3455.000000 0.013109
531.799988 0.025190
945.299988 0.019306
5074.899902 0.030967
2010.500000 0.016980
3466.100098 0.023403
6284.700195 0.064020
4306.000000 0.034567
2632.100098 0.118498
646.500000 0.030911
4859.899902 0.031970
2380.100098 0.059785
2506.000000 0.073179
3356.199951 0.028692
3283.199951 0.054904
1402.300049 0.028286
2390.199951 0.023886
2596.899902 0.065617
2354.699951 0.040507
2923.199951 0.070799
1741.699951 0.036356
3464.300049 0.013126
3922.300049 0.070851
2750.199951 0.011415
2516.399902 0.015232
838.900024 0.026818
3753.500000 0.047497
4277.000000 0.071331
4463.000000 0.048632
5589.100098 0.087146
4998.299805 0.049707
5005.299805 0.049946
4421.299805 0.062859
5731.100098 0.065301
3414.000000 0.071248
3405.899902 0.015907
1249.500000 0.017648
632.099976 0.047410
2034.099976 0.025565
2619.699951 0.047032
2398.399902 0.026188
2344.500000 0.034766
4681.799805 0.044566
2139.500000 0.050162
2618.199951 0.018961
2248.300049 0.021446
761.099976 0.023660
2398.399902 0.021573
2028.400024 0.026165
2936.300049 0.061806
1741.699951 0.047513
1741.699951 0.023014
4037.000000 0.053228
1471.800049 0.023601
4342.600098 0.014980
863.799988 0.017038
2705.100098 0.073634
2893.899902 0.042410
627.799988 0.012386
1532.199951 0.048520
2716.500000 0.027254
3706.100098 0.056655
658.900024 0.020590
2298.699951 0.038855
3692.699951 0.035440
2831.199951 0.024717
2268.600098 0.021015
3085.000000 0.035014
3802.300049 0.034902
2133.100098 0.029120
2811.000000 0.011801
3860.500000 0.138896
1541.599976 0.017824
6059.399902 0.053216
3502.800049 0.056225
3662.100098 0.063348
2225.300049 0.026154
2166.500000 0.046780
1603.800049 0.013750
1258.900024 0.015125
2576.699951 0.047465
2413.100098 0.019734
4141.000000 0.080063
548.500000 0.025657
4562.100098 0.025969
2843.500000 0.044667
1644.900024 0.007170
3507.800049 0.015278
2028.800049 0.010543
4861.200195 0.070097
5118.200195 0.051687
5496.200195 0.042605
6784.600098 0.098935
6231.100098 0.036776
5816.200195 0.049270
5020.000000 0.077066
6954.899902 0.054848
2299.699951 0.038384
4633.000000 0.028128
439.399994 0.009078
604.799988 0.036831
1204.500000 0.009349
1566.599976 0.027662
1336.000000 0.023004
1259.900024 0.023772
5846.100098 0.070829
1025.300049 0.052949
1515.000000 0.023665
1098.900024 0.017532
1969.099976 0.018070
1336.000000 0.015281
907.700012 0.013226
1881.800049 0.032403
548.500000 0.025210
548.500000 0.032503
4451.799805 0.046503
314.200012 0.008301
4848.500000 0.018257
2014.699951 0.019363
3476.699951 0.074623
3367.699951 0.029490
1847.300049 0.007942
2344.000000 0.039922
1688.199951 0.010040
3929.600098 0.038150
1890.300049 0.011432
1155.300049 0.016322
2622.199951 0.019050
3931.699951 0.018544
2059.399902 0.008184
3355.500000 0.021373
4996.000000 0.045150
3179.300049 0.022353
1687.099976 0.004700
1233.300049 0.011448
4071.199951 0.123526
1679.199951 0.025524
6232.000000 0.044570
3644.100098 0.059949
3818.000000 0.074759
1925.500000 0.025979
1863.300049 0.063022
1563.000000 0.029907
954.400024 0.013890
2419.500000 0.060105
2280.800049 0.024650
4336.700195 0.083226
270.399994 0.023310
4682.500000 0.028202
2535.800049 0.049888
1337.400024 0.007463
3597.500000 0.018780
2292.100098 0.004969
4984.100098 0.068308
5137.600098 0.063739
5582.000000 0.042942
6951.899902 0.104519
6461.200195 0.042462
5819.700195 0.056043
4968.100098 0.073317
7151.399902 0.050128
1996.500000 0.043730
4839.299805 0.031953
353.799988 0.010016
867.700012 0.050422
919.299988 0.009196
1253.300049 0.041495
1022.500000 0.026219
947.700012 0.029121
5994.399902 0.054957
715.299988 0.052661
1206.500000 0.021990
798.299988 0.014825
2225.199951 0.013342
1022.500000 0.013379
597.599976 0.016834
1569.400024 0.043829
270.399994 0.025158
270.399994 0.023082
4358.200195 0.041537
0.000000 0.008077
4773.399902 0.018829
2194.100098 0.018842
3497.000000 0.075211
3308.600098 0.034612
2097.199951 0.004431
2408.800049 0.062521
1374.199951 0.014415
3797.300049 0.047974
2129.899902 0.011336
853.299988 0.018823
2313.000000 0.022076
4058.600098 0.027463
1878.400024 0.007345
3245.500000 0.027495
5167.600098 0.063525
3293.899902 0.039715
1383.500000 0.007378
1471.800049 0.014532
314.200012 0.006911
5917.799805 0.123828
3462.000000 0.055721
7892.100098 0.096353
5281.299805 0.087981
5497.500000 0.123421
98.400002 0.012981
169.300003 0.028337
2867.000000 0.023983
1068.400024 0.024279
2869.500000 0.066883
2926.899902 0.036626
6130.299805 0.076189
1775.000000 0.081362
6185.899902 0.016600
560.900024 0.065978
699.099976 0.032765
5046.000000 0.024833
4286.100098 0.065300
6483.399902 0.113259
6147.000000 0.088822
6888.200195 0.108499
8572.200195 0.161386
8330.400391 0.067614
6695.899902 0.082484
5603.700195 0.107182
8887.700195 0.164391
70.500000 0.059037
6655.299805 0.043990
2199.699951 0.026750
2882.300049 0.015650
1539.500000 0.040788
852.799988 0.037336
1056.900024 0.011961
1104.800049 0.014782
7561.299805 0.107461
1315.000000 0.034272
830.799988 0.024896
1224.400024 0.027918
4210.000000 0.027968
1056.900024 0.021040
1430.300049 0.035284
574.099976 0.043148
1775.000000 0.087481
1775.000000 0.090446
4813.200195 0.065048
2022.000000 0.035978
5292.500000 0.054004
4028.500000 0.052142
4663.000000 0.049572
4113.299805 0.042108
4074.600098 0.039336
3889.899902 0.018118
794.799988 0.037098
4085.899902 0.060740
4089.899902 0.024102
1168.800049 0.045384
395.700012 0.027613
5620.500000 0.057809
2397.600098 0.063935
3775.600098 0.026865
6853.299805 0.050080
4859.399902 0.022776
640.400024 0.043249
3449.000000 0.040146
2322.600098 0.043725
2022.000000 0.058611
1466.199951 0.066094
3393.000000 0.055164
1248.500000 0.003731
1426.699951 0.083907
1268.400024 0.098129
6589.200195 0.082570
6494.299805 0.164738
3793.399902 0.088591
5778.200195 0.075093
4114.100098 0.136213
3935.600098 0.125261
1108.500000 0.051940
5324.500000 0.058397
536.200012 0.055950
6940.899902 0.136950
6027.899902 0.069025
1628.199951 0.070485
3257.800049 0.074818
361.299988 0.003716
1903.000000 0.105439
1098.199951 0.028724
1915.400024 0.039394
2151.800049 0.096554
2378.100098 0.050965
2885.100098 0.035790
2284.800049 0.054925
6599.200195 0.165112
1069.300049 0.017510
4741.500000 0.066711
4374.399902 0.131551
5138.100098 0.070033
5834.500000 0.118355
5678.899902 0.090727
5669.700195 0.092122
924.000000 0.009294
5550.200195 0.072808
5916.899902 0.074139
5689.500000 0.069301
3246.500000 0.080046
5678.899902 0.074444
5470.700195 0.061063
6084.000000 0.163061
5324.500000 0.075916
5324.500000 0.033017
3355.600098 0.063243
5070.799805 0.075603
3170.100098 0.043354
2923.199951 0.028998
2206.500000 0.151993
3018.899902 0.077866
3300.699951 0.067465
2772.100098 0.119273
5867.799805 0.078554
3805.000000 0.121282
3210.500000 0.075659
5721.899902 0.083901
6729.600098 0.111766
1038.500000 0.073852
4384.100098 0.066209
3549.199951 0.111384
447.500000 0.041484
1808.199951 0.089782
6113.100098 0.074975
3773.399902 0.037380
4925.500000 0.063286
5070.799805 0.056909
6657.899902 0.109482
4093.100098 0.102876
1672.900024 0.020113
6229.000000 0.028074
3628.899902 0.063563
3810.000000 0.075710
1844.900024 0.028523
1777.800049 0.065717
1478.900024 0.023602
872.900024 0.017326
2291.600098 0.070507
2160.300049 0.035854
4349.299805 0.066904
274.899994 0.023039
4654.700195 0.024412
2441.100098 0.046918
1246.800049 0.014570
3557.500000 0.022755
2353.000000 0.015347
4957.000000 0.040359
5064.399902 0.062186
5536.700195 0.029922
6945.500000 0.077138
6492.200195 0.054245
5739.500000 0.041473
4868.500000 0.058902
7161.200195 0.048811
1910.400024 0.057091
4857.299805 0.015113
313.500000 0.012421
942.799988 0.044645
775.700012 0.012660
1146.699951 0.047146
916.200012 0.026933
847.599976 0.028993
5978.799805 0.040447
625.000000 0.041554
1114.599976 0.027165
726.400024 0.017961
2281.500000 0.021623
916.200012 0.017785
510.000000 0.010211
1465.400024 0.057053
274.899994 0.025385
274.899994 0.026120
4245.799805 0.041593
144.899994 0.012722
4666.399902 0.015830
2204.800049 0.012669
3427.300049 0.090486
3209.199951 0.034652
2149.899902 0.013238
2362.199951 0.052004
1262.800049 0.015807
3674.000000 0.057321
2175.699951 0.016870
777.900024 0.017115
2215.800049 0.029190
4035.199951 0.028430
1745.599976 0.012014
3129.600098 0.037897
5165.399902 0.036241
3266.600098 0.029971
1299.599976 0.014076
1522.199951 0.010698
440.000000 0.006968
144.899994 0.009356
1940.000000 0.049026
5054.799805 0.034762
4566.500000 0.074842
2120.899902 0.048563
6661.399902 0.064553
4046.300049 0.072472
4239.700195 0.115912
1338.400024 0.006609
1272.099976 0.058480
1754.900024 0.033892
366.500000 0.007941
2321.800049 0.069166
2244.899902 0.038744
4810.100098 0.054816
384.299988 0.051411
5041.700195 0.010347
1940.900024 0.038832
743.200012 0.023048
3921.899902 0.017965
2858.000000 0.041148
5344.500000 0.089483
5322.799805 0.082020
5877.100098 0.082679
7370.000000 0.139463
6973.299805 0.054036
5969.100098 0.070136
5024.299805 0.075133
7613.299805 0.127087
1405.000000 0.058145
5324.500000 0.035292
784.000000 0.016525
1448.699951 0.021732
594.700012 0.019428
664.700012 0.039855
436.600006 0.010652
355.600006 0.012187
6387.600098 0.066942
120.199997 0.015403
611.700012 0.015552
230.399994 0.010769
2785.000000 0.010371
436.600006 0.008243
5.600000 0.026391
976.900024 0.054127
384.299988 0.060087
384.299988 0.049289
4342.299805 0.033266
595.099976 0.021841
4787.700195 0.043498
2670.800049 0.034915
3709.300049 0.032794
3381.699951 0.031800
2652.100098 0.024870
2722.500000 0.042286
791.200012 0.028171
3711.899902 0.054008
2674.300049 0.013097
275.700012 0.026767
1717.900024 0.023687
4434.399902 0.057918
1756.699951 0.048664
3219.899902 0.025491
5601.299805 0.059414
3661.800049 0.038951
793.299988 0.031681
2024.500000 0.034675
905.599976 0.037174
595.099976 0.033037
1433.599976 0.018914
5465.399902 0.079364
506.600006 0.034133
2492.699951 0.086231
682.799988 0.015642
4351.600098 0.057580
1740.099976 0.026899
1962.300049 0.048437
3469.800049 0.019183
3376.399902 0.065971
679.299988 0.013791
2659.199951 0.018203
1545.199951 0.039366
1285.300049 0.041427
2640.699951 0.026319
2284.699951 0.050137
2681.800049 0.018096
3868.899902 0.059961
2901.899902 0.022456
1558.099976 0.012370
1686.199951 0.042242
2984.199951 0.065239
3108.000000 0.029542
3527.500000 0.040744
5041.100098 0.078001
4858.799805 0.020157
3814.199951 0.027723
3118.100098 0.041117
5347.000000 0.079173
3486.000000 0.057667
3169.699951 0.021103
1709.400024 0.019860
1638.099976 0.038413
2012.900024 0.020984
2709.899902 0.023541
2553.100098 0.021134
2546.100098 0.014006
4041.500000 0.074492
2439.600098 0.025418
2790.800049 0.028301
2581.300049 0.019124
1583.199951 0.023066
2553.100098 0.018381
2369.500000 0.025141
2968.399902 0.051842
2284.699951 0.053379
2284.699951 0.043327
2709.000000 0.030800
2061.500000 0.024219
3019.399902 0.032192
916.700012 0.022096
1470.800049 0.042418
1566.199951 0.012019
1475.000000 0.028056
358.700012 0.020391
2748.500000 0.025162
2418.199951 0.025137
1389.699951 0.014718
2609.399902 0.029503
3648.000000 0.030640
2088.100098 0.029440
1482.599976 0.044186
1800.099976 0.015956
3311.699951 0.022196
1319.599976 0.020409
2986.899902 0.034221
1328.400024 0.031342
2013.000000 0.026931
2061.500000 0.035712
3542.199951 0.026453
3126.199951 0.066209
2009.900024 0.030368
2364.000000 0.020445
2602.500000 0.079147
936.599976 0.020281
4868.500000 0.036485
2481.699951 0.063523
2579.300049 0.075814
3544.000000 0.028111
3476.500000 0.093579
1723.199951 0.030467
2571.699951 0.015924
2917.699951 0.048068
2675.199951 0.034284
2917.399902 0.052325
1888.599976 0.008446
3566.000000 0.034818
4132.399902 0.058786
2944.500000 0.016450
2685.899902 0.023984
686.400024 0.008301
3846.100098 0.053323
4476.700195 0.054327
4589.799805 0.027606
5599.100098 0.088631
4914.799805 0.032116
5208.200195 0.047404
4677.000000 0.057359
5698.299805 0.045265
3608.899902 0.049319
3374.399902 0.026625
1457.699951 0.012234
758.400024 0.066195
2282.199951 0.010087
2829.500000 0.031491
2602.899902 0.036614
2541.600098 0.027349
4726.500000 0.045465
2324.100098 0.039434
2812.100098 0.033241
2421.100098 0.017370
642.599976 0.019184
2602.899902 0.019401
2209.100098 0.012955
3148.600098 0.051679
1888.599976 0.011425
1888.599976 0.020635
4324.000000 0.025548
1625.800049 0.013330
4614.799805 0.032679
1009.700012 0.012789
2942.100098 0.084348
3182.500000 0.018632
547.099976 0.019098
1788.000000 0.069627
2934.199951 0.011022
4014.000000 0.036325
626.299988 0.016496
2474.500000 0.016887
3904.600098 0.031478
2939.500000 0.037869
2584.800049 0.010760
3392.699951 0.033773
3823.800049 0.049852
2285.800049 0.045473
2998.699951 0.021482
320.799988 0.021991
1350.199951 0.014935
1625.800049 0.009059
3639.100098 0.063863
3834.600098 0.043269
1699.199951 0.012934
2205.699951 0.034502
1617.400024 0.028453
4340.299805 0.080806
1937.400024 0.012798
6493.100098 0.042313
3898.000000 0.051867
4076.199951 0.070213
1677.599976 0.013867
1620.400024 0.065218
1744.500000 0.014915
716.299988 0.006668
2496.100098 0.043784
2382.300049 0.024049
4603.600098 0.043854
0.000000 0.017261
4927.799805 0.021856
2301.000000 0.041133
1104.400024 0.012070
3832.199951 0.015595
2559.000000 0.012899
5230.000000 0.057751
5332.899902 0.045831
5811.600098 0.031926
7211.299805 0.090395
6731.500000 0.029244
6004.000000 0.041705
5117.200195 0.057244
7417.700195 0.060528
1753.599976 0.040102
5107.700195 0.019456
584.400024 0.005082
1132.300049 0.037899
860.900024 0.002767
1046.199951 0.020654
820.200012 0.020113
736.500000 0.011962
6248.399902 0.053410
498.000000 0.027899
976.900024 0.022603
553.500000 0.011061
2493.600098 0.012584
820.200012 0.008800
384.899994 0.005453
1353.500000 0.035386
0.000000 0.018062
0.000000 0.030909
4477.700195 0.019489
270.399994 0.005914
4905.899902 0.025038
2459.899902 0.011904
3697.699951 0.063931
3459.399902 0.012716
2366.399902 0.014629
2637.100098 0.040588
1174.599976 0.005299
3886.199951 0.032499
2399.899902 0.007166
610.799988 0.008180
2082.300049 0.019658
4307.100098 0.030476
1939.099976 0.015152
3357.300049 0.020754
5429.000000 0.038918
3539.399902 0.027505
1142.000000 0.016340
1741.699951 0.020923
548.500000 0.009527
270.399994 0.011496
1775.000000 0.036489
5324.500000 0.053160
274.899994 0.009594
384.299988 0.019701
2284.699951 0.013315
1888.599976 0.006602
2008.000000 0.141906
524.200012 0.034522
4192.399902 0.077157
1693.300049 0.069324
1820.000000 0.088146
3855.800049 0.022846
3774.199951 0.044376
1416.900024 0.012313
2918.800049 0.013534
2519.000000 0.053866
2259.600098 0.020697
2274.000000 0.085529
2333.300049 0.046819
2778.300049 0.029545
4371.200195 0.042593
3249.300049 0.006769
1864.800049 0.016378
675.799988 0.017339
3066.500000 0.095631
3658.600098 0.041260
3784.800049 0.061041
4919.399902 0.112293
4410.100098 0.039695
4390.700195 0.041288
3905.800049 0.074513
5089.200195 0.074160
3900.000000 0.018273
2774.399902 0.041463
1782.199951 0.012830
1292.099976 0.030552
2440.000000 0.021231
3093.000000 0.032889
2887.600098 0.009478
2848.399902 0.014011
3999.600098 0.098244
2669.600098 0.028782
3120.800049 0.033677
2793.300049 0.006787
573.799988 0.008857
2887.600098 0.012170
2567.800049 0.017798
3397.300049 0.039818
2333.300049 0.043166
2333.300049 0.046768
3624.000000 0.072423
2064.899902 0.015552
3874.000000 0.041522
188.500000 0.039886
2155.399902 0.077779
2502.300049 0.027926
477.600006 0.012796
1059.699951 0.032714
3171.300049 0.015592
3409.500000 0.034865
382.500000 0.008368
2838.100098 0.021319
4141.100098 0.005421
2145.899902 0.038597
2336.600098 0.022458
2797.600098 0.012577
3130.699951 0.045087
1466.900024 0.019171
3320.000000 0.016886
687.400024 0.024919
1869.199951 0.014756
2064.899902 0.021127
3943.199951 0.030447
3088.100098 0.084951
2085.300049 0.020613
2563.199951 0.029151
1010.700012 0.025497
821.099976 0.026937
2333.300049 0.017877
2333.300049 0.017877
/
%Lowess fit
\plot
   0.0000 0.02278214
 187.8204 0.02396958
 375.6408 0.02513357
 563.4612 0.02627533
 751.2817 0.02739847
 939.1021 0.02850713
1126.9225 0.02960487
1314.7429 0.03069406
1502.5634 0.03177626
1690.3838 0.03285039
1878.2042 0.03389936
2066.0247 0.03481849
2253.8450 0.03560085
2441.6655 0.03641997
2629.4858 0.03739190
2817.3064 0.03853390
3005.1267 0.03978521
3192.9470 0.04116089
3380.7676 0.04266053
3568.5879 0.04431327
3756.4084 0.04611041
3944.2288 0.04800185
4132.0493 0.04994082
4319.8696 0.05181951
4507.6899 0.05368338
4695.5103 0.05556004
4883.3311 0.05744581
5071.1514 0.05935243
5258.9717 0.06134224
5446.7920 0.06342643
5634.6128 0.06556284
5822.4331 0.06774928
6010.2534 0.06997193
6198.0737 0.07221062
6385.8940 0.07445294
6573.7148 0.07669318
6761.5352 0.07892762
6949.3555 0.08115731
7137.1758 0.08338400
7324.9966 0.08560968
7512.8169 0.08783670
7700.6372 0.09007004
7888.4575 0.09231168
8076.2778 0.09456414
8264.0986 0.09682812
8451.9189 0.09910294
8639.7393 0.10139073
8827.56 0.1036905
9015.38 0.1060052
9203.20 0.1083300
/
%%%%%%%%%%%%%%%%%%%%%%%%%%%%%%%%%%%%%%%%%%%%%%%%%%%%%%%%%%%%%%%%
\setcoordinatesystem units <.000081in, 3.7in> point at -25000 0
\setplotarea x from 0 to 9000, y from 0 to 0.27
\axis left shiftedto x=-1000 /
\axis bottom shiftedto y=-0.0135 label {km}
  ticks numbered from 0 to 9000 by  9000 /
\setplotsymbol ({\normalsize .})
\plotheading{S America}
\multiput {\tiny .} at
2403.8 0.03969540
6193.2 0.02468286
4669.0 0.06013543
1545.0 0.08817769
3304.6 0.07366254
5754.2 0.16273360
3516.3 0.01787892
2256.3 0.03262910
2687.0 0.01520983
3338.0 0.09258271
3976.7 0.03959357
3762.9 0.10460615
2967.1 0.06779571
3019.3 0.20913988
1802.6 0.07811122
2671.4 0.03183961
1459.3 0.01939718
3585.4 0.02868734
2791.9 0.08962997
919.1 0.01638507
2314.3 0.11434284
2677.6 0.04997996
1366.3 0.01678889
3611.7 0.07182947
2855.6 0.05575359
967.6 0.04346949
2414.9 0.15381346
106.4 0.02451130
3727.5 0.04125007
3544.0 0.08907418
3096.1 0.06101233
2804.0 0.17813417
1668.4 0.05710007
252.6 0.01745622
2105.7 0.11013102
2208.4 0.13774779
2323.7 0.07052176
452.7 0.04950652
4367.1 0.09749918
3019.6 0.12671953
1853.5 0.06355779
3311.6 0.08961204
1016.1 0.05466018
928.1 0.10113047
3091.7 0.09781795
2323.7 0.01728305
452.7 0.07435615
4367.1 0.017790477
3019.6 0.139150634
1853.5 0.020638462
3311.6 0.063223463
1016.1 0.033365240
928.1 0.086673246
3091.7 0.073464922
0.0 0.079151153
2247.5 0.029341868
825.1 0.047530186
5481.2 0.037374254
3482.9 0.131674847
3069.8 0.030046756
4485.4 0.058007217
2230.6 0.042725681
2147.1 0.094473042
4254.1 0.044882107
1221.6 0.048722742
1221.6 0.036511986
2018.0 0.080499283
1393.3 0.113892235
4200.2 0.129156356
2220.7 0.106365913
1513.2 0.099150248
2510.1 0.128216252
659.9 0.088169744
685.2 0.153419736
2270.0 0.149862999
969.4 0.052605212
969.4 0.075748988
2002.4 0.062035943
2938.9 0.073353215
2631.2 0.066485623
3425.1 0.080965093
2339.5 0.195231683
1154.2 0.070120795
1178.9 0.068315261
1237.3 0.079404088
1343.6 0.145208625
937.6 0.059046733
2178.8 0.036194710
2178.8 0.083482749
3320.7 0.016549644
1332.4 0.083696035
3130.5 0.021213473
2549.0 0.007682872
3150.6 0.036239666
2638.0 0.078654992
843.1 0.012194117
1214.7 0.078637742
1100.9 0.011980227
1202.9 0.023820598
1005.9 0.065825983
2098.4 0.039015155
2098.4 0.040196433
3284.9 0.032452472
1357.6 0.095409805
322.1 0.057064651
2792.5 0.052615925
1746.8 0.096141687
3414.3 0.028532954
2748.2 0.245719600
727.4 0.044396376
2027.7 0.056302652
288.0 0.060265481
387.2 0.144833398
1821.9 0.067369651
1302.0 0.068468004
1302.0 0.025116958
2512.1 0.035335618
786.0 0.110244089
975.6 0.046986913
816.1 0.059691607
3288.6 0.034074747
2373.2 0.003590119
2916.2 0.049130646
2958.0 0.074588352
443.2 0.020576969
1467.8 0.096189312
922.2 0.014428682
1008.4 0.024393123
1298.4 0.078197042
1936.2 0.035986710
1936.2 0.058376229
3152.8 0.034417891
1365.1 0.095074507
712.8 0.052466388
399.9 0.003281175
645.3 0.075020607
3288.6 0.057629188
2373.2 0.031556238
2916.2 0.073522217
2958.0 0.114865600
443.2 0.053456349
1467.8 0.115207917
922.2 0.023674335
1008.4 0.067215834
1298.4 0.127752225
1936.2 0.020909531
1936.2 0.059826866
3152.8 0.036915238
1365.1 0.048247037
712.8 0.043092724
399.9 0.029659974
645.3 0.065116623
0.0 0.024882776
3130.5 0.035950374
2549.0 0.012282491
3150.6 0.061362362
2638.0 0.063401371
843.1 0.029198822
1214.7 0.110909892
1100.9 0.012739562
1202.9 0.031412584
1005.9 0.104626526
2098.4 0.029955894
2098.4 0.051241056
3284.9 0.034276415
1357.6 0.059311147
322.1 0.056644082
0.0 0.009961712
816.1 0.078206880
399.9 0.007540625
399.9 0.010815983
2236.4 0.022227591
679.3 0.043742239
5340.4 0.040340508
3421.0 0.077031467
2922.6 0.023639476
4346.7 0.091546424
2085.5 0.045929642
2001.3 0.052685495
4116.7 0.056455616
1075.1 0.113134018
1075.1 0.049030316
147.2 0.031441731
1871.9 0.116500953
3184.8 0.084641955
3143.9 0.033334804
2367.7 0.092442121
3007.7 0.039028506
3007.7 0.08477933
3143.9 0.04692462
1108.2 0.03812542
2040.2 0.06178656
6486.7 0.01326664
2644.9 0.12067787
3832.9 0.01493071
4723.2 0.12333316
2915.2 0.02140254
2879.0 0.05049884
4470.9 0.10637003
2182.4 0.11536793
2182.4 0.02843881
1523.9 0.06551227
2365.9 0.13617789
3580.9 0.12847443
3689.1 0.03962633
3126.0 0.06501969
3730.9 0.05168411
3730.9 0.08337294
3689.1 0.05880274
1581.2 0.04459068
945.7 0.03406469
3092.7 0.03345650
6175.8 0.04992980
700.0 0.14176715
3615.0 0.03354990
3596.1 0.03669166
2919.6 0.05436179
2960.4 0.07957302
3365.1 0.02345793
2895.5 0.03785143
2895.5 0.06057065
3103.9 0.02658408
2279.0 0.11667834
2760.3 0.02738564
3025.2 0.02080208
2946.9 0.04676700
3288.0 0.02530965
3288.0 0.05559939
3025.2 0.04329962
3068.6 0.05277028
2048.7 0.08304480
1473.7 0.03807380
2088.4 0.04941615
6655.9 0.05725412
3012.7 0.15691344
4037.9 0.048925665
5027.4 0.041063981
3119.5 0.066930491
3072.9 0.109078847
4775.9 0.030140262
2301.3 0.053382516
2301.3 0.060463402
1447.7 0.008784494
2613.5 0.090881140
3870.8 0.011269792
3958.3 0.039708981
3348.9 0.046820765
3970.7 0.040813676
3970.7 0.051695767
3958.3 0.048899477
1535.8 0.044223099
368.2 0.101635059
2410.2 0.015832156
840.7 0.020879591
1944.4 0.068189580
6263.6 0.022864198
2355.4 0.108832013
3592.6 0.017113348
4421.2 0.084121661
2680.2 0.034068198
2652.1 0.082803004
4168.7 0.073113344
2023.8 0.089037533
2023.8 0.010341098
1552.5 0.024232200
2103.8 0.067875594
3287.3 0.075761882
3407.1 0.041042370
2876.4 0.043796382
3467.2 0.052543697
3467.2 0.061751691
3407.1 0.045569717
1581.1 0.024744803
314.9 0.027345394
1786.3 0.065605496
673.6 0.051929178
3566.2 0.093560112
3141.5 0.046356249
2939.6 0.133995633
2832.1 0.056882176
1218.0 0.091000028
628.4 0.240803784
1688.3 0.049025495
1788.0 0.035115721
452.7 0.228933425
2691.6 0.128575119
2691.6 0.120956341
3877.7 0.112763949
1929.7 0.124593651
638.1 0.166115644
593.8 0.062894764
1401.0 0.198156033
848.0 0.054982603
848.0 0.062650200
593.8 0.037202873
3737.2 0.079515882
4216.8 0.103609459
3322.6 0.145580527
4502.3 0.137584261
3924.6 0.097945039
2152.7 0.044260776
1790.4 0.004168550
4044.2 0.060968217
2055.1 0.069091402
1391.7 0.030726903
2131.9 0.111051914
742.7 0.024566900
818.0 0.013535570
1887.3 0.087850835
1355.9 0.060162782
1355.9 0.082995783
2404.6 0.062424398
404.6 0.136275314
954.2 0.085923506
1033.2 0.010391214
709.9 0.109458376
1136.4 0.006454415
1136.4 0.050296233
1033.2 0.022594759
2275.4 0.046954106
2656.2 0.056875236
2238.4 0.036995425
2930.8 0.065075571
2374.2 0.077286504
1574.3 0.060464197
2158.8 0.026057632
514.9 0.022732763
5183.4 0.045004546
3276.9 0.073555017
2732.8 0.017319740
4142.6 0.066654274
1886.1 0.035304063
1803.6 0.054525434
3912.4 0.038397879
880.7 0.043300827
880.7 0.052039487
345.2 0.018270994
1669.0 0.080202941
2980.5 0.038484721
2940.4 0.015198922
2167.1 0.068102175
2808.4 0.014755584
2808.4 0.046767781
2940.4 0.024666275
204.3 0.022421718
1605.4 0.054904700
2956.4 0.018269722
1606.6 0.023855275
1564.4 0.035750082
3533.5 0.092106129
2072.9 0.024657489
3286.5 0.047410871
2819.2 0.021102441
3091.4 0.065286789
2663.8 0.084804174
1025.5 0.026501632
943.8 0.114889027
1372.8 0.020246969
1474.7 0.048174695
734.6 0.101706557
2368.1 0.024162260
2368.1 0.052563629
3548.9 0.038701089
1600.1 0.073303446
351.0 0.052425545
271.9 0.011246099
1087.8 0.066191821
602.8 0.010757713
602.8 0.018205386
271.9 0.006924736
3408.8 0.059485003
3901.3 0.063674546
3105.5 0.037387337
4181.4 0.051561895
3612.6 0.050273498
330.6 0.061386896
1250.8 0.02926269
3205.0 0.02609046
308.7 0.00845773
2267.5 0.01726893
5898.0 0.03060394
1376.1 0.06801298
3227.4 0.01657275
3671.8 0.07184218
2399.9 0.01801607
2412.7 0.02537093
3422.1 0.06142064
2135.7 0.06943373
2135.7 0.03307827
2220.0 0.02711865
1741.8 0.08801920
2630.3 0.06955695
2823.8 0.01102442
2507.1 0.07088635
2988.8 0.01614801
2988.8 0.04264575
2823.8 0.01865415
2188.4 0.01257500
1285.7 0.03595099
884.8 0.03602874
1653.2 0.03730172
985.0 0.02596508
3257.5 0.05307008
1854.4 0.02316347
2084.8 0.01891690
2978.1 0.03331470
839.8 0.03024007
1740.3 0.03235215
6016.7 0.04378357
2277.4 0.11210863
3345.9 0.03039463
4203.0 0.05104747
2433.2 0.02854224
2405.3 0.06965683
3950.8 0.05812489
1792.2 0.01294080
1792.2 0.03999345
1431.3 0.02556765
1859.7 0.05335296
3059.9 0.03257221
3171.4 0.02040317
2630.4 0.04030696
3224.1 0.02250467
3224.1 0.020611318
3171.4 0.023079829
1439.4 0.066975543
521.1 0.061892924
1761.8 0.021773439
840.8 0.033373376
247.0 0.048235425
3696.0 0.112683474
2138.2 0.039320820
1396.5 0.022284282
3381.2 0.025642409
904.5 0.033747439
5087.2 0.042451406
3640.8 0.012794643
1107.6 0.062950609
4720.2 0.052606601
1579.8 0.033922124
2176.1 0.146065412
2483.3 0.020948129
2514.2 0.007764941
2236.6 0.121126193
3307.0 0.096186499
3307.0 0.074505848
4463.6 0.065646892
3092.8 0.130130864
2380.8 0.113257970
2086.8 0.019802880
2306.9 0.127545407
1821.2 0.017353284
1821.2 0.055240375
2086.8 0.021966470
4319.2 0.030025176
5393.1 0.048126179
5101.7 0.070332034
5575.5 0.079753454
5163.4 0.058093471
1977.7 0.022514812
2941.1 0.014854429
4150.4 0.037747078
2067.4 0.037240928
4793.4 0.015136392
4916.4 0.065412964
3382.9 0.042356448
2552.7 0.031027601
2849.2 0.039818444
2965.3 0.106773865
562.8 0.031810199
1288.8 0.111831878
1097.8 0.007160828
1186.7 0.035697757
1127.2 0.125912555
2113.5 0.046329666
2113.5 0.040967553
3327.6 0.062523570
1503.4 0.086007112
654.2 0.092710197
334.0 0.022809970
816.1 0.063725506
183.2 0.026557864
183.2 0.024567423
334.0 0.022627838
3182.9 0.080133261
3867.2 0.033336851
3331.9 0.064198705
4115.9 0.087037827
3597.5 0.054925558
674.7 0.071371514
1240.9 0.035200070
2982.8 0.053275675
464.1 0.031090895
3079.7 0.035353938
3356.3 0.023755779
1770.5 0.042973821
2018.0 0.020278660
1393.3 0.012991673
4200.2 0.037165786
2220.7 0.066066478
1513.2 0.013225554
2510.1 0.074639376
659.9 0.025538778
685.2 0.030192195
2270.0 0.046742613
969.4 0.059103258
969.4 0.051463095
2002.4 0.027999384
0.0 0.103179947
1332.4 0.058440859
1357.6 0.008743686
786.0 0.077636161
1365.1 0.009141686
1365.1 0.051232737
1357.6 0.022470532
1871.9 0.015189369
2365.9 0.041952482
2279.0 0.022581845
2613.5 0.033756698
2103.8 0.038014484
1929.7 0.073081386
404.6 0.011785298
1669.0 0.005168100
1600.1 0.028644970
1741.8 0.009029302
1859.7 0.030064171
3092.8 0.019344189
1503.4 0.044531334
3130.5 0.122207919
2549.0 0.119699699
3150.6 0.132306215
2638.0 0.238603800
843.1 0.113074503
1214.7 0.102795162
1100.9 0.135946840
1202.9 0.221350927
1005.9 0.081531170
2098.4 0.058759026
2098.4 0.130120817
3284.9 0.037679861
1357.6 0.096727742
322.1 0.010870457
0.0 0.105072728
816.1 0.080220632
399.9 0.097044150
399.9 0.081977491
0.0 0.099400646
3143.9 0.123422471
3689.1 0.187616196
3025.2 0.057770780
3958.3 0.033423651
3407.1 0.111595512
593.8 0.236051965
1033.2 0.141290889
2940.4 0.065157224
271.9 0.089264384
2823.8 0.120673901
3171.4 0.065777353
2086.8 0.176976406
334.0 0.154128885
1357.6 0.098685630
2668.1 0.116492975
2256.1 0.129746473
3584.6 0.118402752
2267.3 0.268139627
1078.3 0.110641428
1556.7 0.081871882
901.5 0.141811929
1007.1 0.232971714
1318.1 0.064870532
1804.6 0.067077903
1804.6 0.121084971
2939.0 0.038299081
953.5 0.112257484
381.7 0.012258832
463.3 0.108940284
673.3 0.064830322
694.2 0.105805746
694.2 0.095062859
463.3 0.113693371
2803.4 0.125633098
3240.4 0.181175326
2599.7 0.052159296
3519.1 0.031405972
2953.8 0.110100439
985.7 0.268501476
588.9 0.149269692
2599.1 0.069460696
662.3 0.101445278
2361.9 0.123598369
2720.8 0.065920145
2502.3 0.191176295
735.4 0.157372102
953.5 0.102199249
463.3 0.003465521
751.6 0.036423024
2870.7 0.054828661
6068.8 0.047556073
812.8 0.161639249
3474.4 0.041475943
3565.9 0.035393804
2747.9 0.058952344
2783.5 0.114257398
3328.5 0.030401424
2682.0 0.034961773
2682.0 0.047119420
2879.3 0.007365561
2099.1 0.072797687
2672.0 0.008194080
2922.5 0.037413826
2791.3 0.028843845
3165.0 0.040909710
3165.0 0.043076398
2922.5 0.046283512
2843.1 0.055499854
1859.8 0.088191456
225.8 0.013325588
2224.9 0.005715636
1586.7 0.046341132
3254.9 0.154057414
2087.5 0.068216010
2730.6 0.023658353
3021.0 0.044980241
661.9 0.042357566
1548.8 0.018463680
4984.7 0.091547162
3219.8 0.069676039
2099.1 0.037320003
2922.5 0.026907967
2484.6 0.022687370
1879.0 0.020882962
972.7 0.077863799
5624.8 0.047647903
3162.9 0.132864465
3112.4 0.048786518
4400.6 0.049640597
2227.3 0.072842264
2156.6 0.108241463
4161.0 0.044659889
1271.7 0.110065258
1271.7 0.037823383
375.9 0.021916718
1890.9 0.086443848
3223.4 0.063421884
3225.4 0.059395708
2495.1 0.065472036
3140.3 0.069316867
3140.3 0.083909789
3225.4 0.067733086
413.9 0.023582236
1167.3 0.079801564
2752.0 0.056654765
1138.2 0.026702019
1178.0 0.025223099
3811.0 0.123721710
2279.9 0.092982470
470.9 0.043410895
3480.4 0.084188670
1867.2 0.02845229
1056.7 0.06380727
4576.0 0.07595805
3307.2 0.09989291
1890.9 0.04472016
3225.4 0.09653529
2843.9 0.09178141
2528.8 0.03648902
444.4 0.01086285
2246.7 0.05111150
6342.6 0.03812839
1988.9 0.09586818
3655.8 0.03431330
4294.4 0.05754669
2768.0 0.03113695
2756.6 0.04975970
4042.7 0.08232745
2252.5 0.07257664
2252.5 0.02007112
1951.2 0.05953036
2142.9 0.07835408
3206.6 0.10758170
3365.5 0.03230960
2928.8 0.06424322
3477.2 0.04768714
3477.2 0.05505056
3365.5 0.04110757
1964.5 0.05749821
668.2 0.04406704
1380.6 0.06110315
1031.0 0.07376429
431.7 0.03819194
3842.9 0.09059768
2349.1 0.05649552
1922.0 0.05530745
3546.3 0.05582739
671.1 0.02323156
526.3 0.03486410
5235.5 0.05396184
3589.9 0.02760985
2142.9 0.04529975
3365.5 0.17233406
2903.2 0.16678325
1194.2 0.06467233
1575.4 0.05095578
379.2 0.01583551
2148.9 0.05958844
5814.1 0.03429323
1446.3 0.15333060
3138.0 0.04395242
3635.5 0.03832499
2299.9 0.05242624
2309.8 0.09072986
3384.7 0.04830395
2017.5 0.08617506
2017.5 0.02785793
2114.1 0.01990763
1643.9 0.08936683
2574.3 0.05138057
2757.6 0.04327694
2415.0 0.04133950
2910.1 0.05461746
2910.1 0.05806761
2757.6 0.05305668
2079.1 0.03581199
1247.4 0.07026928
990.1 0.04159528
1611.8 0.02153602
938.7 0.03037804
3205.5 0.11505885
1773.9 0.07747052
1971.8 0.04601447
2920.0 0.06799588
119.1 0.02271005
832.4 0.04472032
4708.2 0.06903055
3006.0 0.06865082
1643.9 0.04365580
2757.6 0.09546956
2294.6 0.08800409
765.1 0.02687033
1765.4 0.00935120
671.2 0.03372129
841.6 0.03183378
1739.1 0.03748702
6016.5 0.03856848
2279.2 0.08662154
3345.8 0.01492283
4203.8 0.06815535
2433.1 0.04082795
2405.1 0.06308165
3951.6 0.03696555
1791.4 0.05081919
1791.4 0.05247608
1429.6 0.031550327
1859.9 0.097240499
3060.5 0.055771539
3171.8 0.022414871
2630.4 0.066292442
3224.3 0.025385921
3224.3 0.066257638
3171.8 0.041009946
1437.8 0.033833972
520.5 0.043838918
1763.6 0.023107536
839.7 0.042687732
247.1 0.041633460
3696.6 0.127190849
2138.6 0.031494133
1394.9 0.006723113
3381.7 0.038376194
906.3 0.031318732
1.8 0.024857312
4916.3 0.054682744
3356.6 0.053209686
1859.9 0.011704303
3171.8 0.081158997
2721.3 0.079946090
1550.6 0.033015853
1054.9 0.062686774
527.9 0.059451539
834.1 0.064506832
2205.8 0.026252707
556.8 0.017503595
5224.4 0.064047042
3343.8 0.041291569
2791.7 0.031644852
4214.4 0.104629301
1952.1 0.028051900
1868.2 0.029725744
3984.9 0.085157765
942.5 0.063836518
942.5 0.058958054
279.1 0.032568761
1744.1 0.071446127
3053.8 0.070457259
3010.8 0.018225295
2234.2 0.105217613
2874.3 0.016020745
2874.3 0.038300666
3010.8 0.012485004
133.5 0.018538926
1616.5 0.062346181
3014.5 0.049184911
1600.0 0.044122070
1590.2 0.039513883
3604.2 0.034917363
2148.2 0.025460116
76.3 0.018805709
3276.0 0.028608535
2139.9 0.008395422
1431.0 0.042340357
4197.3 0.013815444
3049.4 0.049909090
1744.1 0.013081616
3010.8 0.111516707
2672.4 0.125571064
2788.7 0.052614139
462.2 0.041072244
1957.2 0.045068851
2027.9 0.043163864
1429.4 0.038475072
3168.4 0.114765665
2787.9 0.060696520
3226.0 0.129596189
2529.2 0.089085435
1105.1 0.069002057
985.1 0.194822216
1358.2 0.061459581
1462.6 0.072956675
756.6 0.170480820
2335.5 0.047237888
2335.5 0.125941706
3500.6 0.123440692
1530.3 0.127517382
229.9 0.133331400
284.6 0.052601908
1079.6 0.147911714
668.0 0.048228349
668.0 0.072374274
284.6 0.051923556
3362.3 0.137630815
3805.0 0.099910717
2973.4 0.089998818
4091.2 0.146325528
3513.1 0.128494552
411.8 0.120941638
1165.7 0.050716639
3158.1 0.068540304
134.6 0.037914367
2859.8 0.099295735
3284.2 0.060166835
2200.7 0.085435061
554.9 0.055293359
1530.3 0.071898640
284.6 0.168838927
579.0 0.185342671
2891.0 0.120583809
3418.4 0.203920692
3436.1 0.110490899
2804.2 0.183503640
3284.8 0.058774191
3230.2 0.088548280
1880.9 0.025320394
658.8 0.059353027
5266.3 0.046682261
3015.5 0.102562882
2738.0 0.031225860
4046.1 0.060135383
1853.1 0.060057839
1781.8 0.096603604
3809.2 0.034875916
903.1 0.076352735
903.1 0.043024728
496.3 0.009191184
1541.9 0.071134728
2872.0 0.040547211
2861.8 0.042074738
2122.5 0.062712345
2767.8 0.046525438
2767.8 0.068416242
2861.8 0.048667627
407.5 0.015429416
1384.7 0.068266401
2680.4 0.037297004
1434.2 0.018433340
1312.5 0.021103438
3450.2 0.118007284
1938.3 0.069066105
278.8 0.015663620
3119.7 0.055481759
1806.8 0.026185613
1131.1 0.044406278
4210.1 0.065056223
2935.8 0.088150087
1541.9 0.02459358
2861.8 0.05855566
2491.1 0.05934615
2454.5 0.02189778
374.8 0.01339510
1654.3 0.06399840
1694.3 0.02679209
1129.7 0.02755705
334.3 0.02866368
3061.9 0.13891072
1771.2 0.03675426
737.5 0.04803069
5302.3 0.07471264
2913.2 0.09005804
2747.8 0.05035441
4012.4 0.06508955
1851.6 0.04172899
1785.3 0.07367151
3773.1 0.08904069
935.3 0.01777867
935.3 0.04811945
582.3 0.05505633
1503.1 0.03941990
2835.5 0.06914564
2837.7 0.03429073
2114.8 0.07056787
2759.3 0.03933223
2759.3 0.02861529
2837.7 0.02885452
507.8 0.09205146
1303.2 0.08385669
2571.5 0.04844149
1375.1 0.06510302
1216.0 0.06433926
3422.8 0.10634688
1893.9 0.05538337
388.9 0.04459703
3092.1 0.03489631
1697.1 0.04640388
1028.1 0.01124618
4238.2 0.07702701
2924.3 0.02849089
1503.1 0.05136897
2837.7 0.10930113
2455.8 0.11326086
2345.7 0.04706182
388.4 0.08289413
1549.4 0.025683207
1584.9 0.063330094
1026.7 0.049324542
443.1 0.050433205
3030.0 0.059803451
110.1 0.070142349
1831.8 0.014724636
576.6 0.053063786
4933.1 0.038479798
2769.9 0.070355854
2353.3 0.021644660
3627.3 0.076028293
1452.8 0.042239620
1388.4 0.067098822
3390.8 0.055425846
586.6 0.087015822
586.6 0.029602217
884.2 0.021165500
1125.2 0.066602294
2453.8 0.071919744
2444.2 0.035253203
1715.2 0.073480653
2359.8 0.042505447
2359.8 0.066471931
2444.2 0.039384338
763.4 0.008105132
1595.8 0.043751073
2522.3 0.053763947
1724.2 0.040231704
1444.6 0.010442523
3031.8 0.084506211
1524.2 0.059630867
569.5 0.020797460
2701.2 0.051597794
1690.9 0.014542010
1219.6 0.048233826
3859.3 0.043474261
2525.2 0.067739613
1125.2 0.021479357
2444.2 0.103100521
2072.6 0.105872373
2299.2 0.042468174
783.0 0.015409293
1702.8 0.041145233
1572.3 0.028517496
1218.6 0.030626682
645.6 0.018371615
2643.1 0.125846626
418.8 0.007883734
399.6 0.062354306
3286.5 0.039431770
2819.2 0.033092544
3091.4 0.044248283
2663.8 0.125164344
1025.5 0.024004027
943.8 0.079816967
1372.8 0.029959349
1474.7 0.044677493
734.6 0.077886722
2368.1 0.050959210
2368.1 0.047892834
3548.9 0.071430427
1600.1 0.140520502
351.0 0.092101138
271.9 0.015280576
1087.8 0.058079831
602.8 0.026170887
602.8 0.059423981
271.9 0.040028793
3408.8 0.076856782
3901.3 0.047497092
3105.5 0.031769085
4181.4 0.075779144
3612.6 0.069683761
330.6 0.117294880
1250.8 0.028413408
3205.0 0.044210607
0.0 0.029383819
2978.1 0.041953420
3381.2 0.032448673
2067.4 0.054881315
464.1 0.028280244
1600.1 0.035813287
271.9 0.150753300
662.3 0.144833445
3021.0 0.062768508
3480.4 0.104561432
3546.3 0.038166821
2920.0 0.075190070
3381.7 0.038642282
3276.0 0.064171848
134.6 0.048683576
3119.7 0.086868485
3092.1 0.041236576
2701.2 0.07674159
3288.6 0.05373622
2373.2 0.02731204
2916.2 0.06792289
2958.0 0.08512737
443.2 0.04770435
1467.8 0.15232584
922.2 0.01080862
1008.4 0.03277575
1298.4 0.16251252
1936.2 0.05849262
1936.2 0.05701613
3152.8 0.06210265
1365.1 0.07421981
712.8 0.09572081
399.9 0.02858852
645.3 0.09392408
0.0 0.02662289
0.0 0.01494157
399.9 0.01145701
3007.7 0.07256462
3730.9 0.05671926
3288.0 0.08550211
3970.7 0.08715531
3467.2 0.05672203
848.0 0.02507915
1136.4 0.04138760
2808.4 0.05996465
602.8 0.02580511
2988.8 0.03234139
3224.1 0.04406401
1821.2 0.02698225
183.2 0.01573185
1365.1 0.05014093
399.9 0.15673979
694.2 0.17302056
3165.0 0.08322694
3140.3 0.09231280
3477.2 0.04259652
2910.1 0.06783593
3224.3 0.07922210
2874.3 0.03122720
668.0 0.07641754
2767.8 0.08569530
2759.3 0.04470667
2359.8 0.06306293
602.8 0.05783750
1450.2 0.04528895
1575.2 0.04627871
6146.2 0.02437693
2907.0 0.18954963
3546.3 0.03170551
4628.5 0.05514323
2631.1 0.03568311
2578.9 0.08870088
4379.6 0.05527673
1786.0 0.04224215
1786.0 0.04436722
995.3 0.03105743
2162.3 0.11555186
3457.1 0.03448047
3518.5 0.03063983
2871.1 0.01973390
3502.9 0.03545757
3502.9 0.04346054
3518.5 0.04995325
1060.2 0.07630672
530.3 0.05278735
2386.6 0.02161606
517.3 0.03624137
631.4 0.05733536
4079.0 0.15630640
2505.4 0.05144478
1105.8 0.03797275
3752.9 0.04572695
1535.5 0.04820966
631.0 0.01657789
5070.9 0.08319193
3654.8 0.03505562
2162.3 0.04129564
3518.5 0.07278104
3093.4 0.06285615
2176.7 0.02166076
648.5 0.07563709
1062.7 0.05772262
1460.8 0.04746789
629.3 0.03438799
1107.5 0.07406325
3671.1 0.09300810
918.5 0.05911075
857.9 0.05024783
1213.1 0.06998732
3752.9 0.03306911
3502.9 0.06827312
200.5 0.051678823
2356.8 0.084863270
6289.4 0.052481730
1744.2 0.187852442
3605.1 0.050494757
4139.3 0.055892146
2739.5 0.080137831
2738.0 0.150167872
3889.0 0.037611121
2315.5 0.068989427
2315.5 0.053380658
2134.7 0.007229693
2095.7 0.085486305
3080.5 0.015492412
3258.9 0.061706267
2878.2 0.034666185
3397.9 0.065779355
3397.9 0.069375060
3258.9 0.070440290
2135.2 0.051523252
917.1 0.094537811
1131.7 0.034507061
1280.0 0.011628627
666.0 0.039738317
3712.3 0.178072500
2262.2 0.100864215
2073.8 0.034048200
3425.4 0.067389210
471.9 0.057347241
705.9 0.044991063
5181.9 0.112125655
3500.1 0.103826791
2095.7 0.051965241
3258.9 0.024517687
2795.6 0.019831720
945.7 0.008883675
1761.3 0.030918611
249.1 0.094185768
506.9 0.033412019
707.7 0.044651445
2115.5 0.066923985
3310.4 0.163705845
1799.2 0.015197575
1691.0 0.085070487
1794.4 0.039250553
3425.4 0.099369713
3397.9 0.111054849
1289.8 0.04122370
900.8 0.03496884
3003.4 0.06409419
6090.7 0.08295008
689.1 0.14755626
3522.8 0.06861007
3532.0 0.02436479
2823.4 0.08008299
2863.9 0.11718428
3298.8 0.04279559
2801.8 0.03760527
2801.8 0.05883325
3028.5 0.03380689
2182.3 0.06701081
2678.9 0.03478053
2940.4 0.04802326
2852.4 0.05742664
3198.9 0.05585132
3198.9 0.05143359
2940.4 0.05291997
2990.4 0.07924829
2008.4 0.12856701
96.8 0.02529781
2372.4 0.02124237
1738.2 0.07202196
3247.0 0.15124091
2144.3 0.08062286
2874.9 0.04578859
3025.2 0.05751209
813.7 0.04947298
1702.3 0.02774092
5014.3 0.10349538
3245.1 0.08135017
2182.3 0.05635156
2940.4 0.06387672
2512.0 0.06051365
153.5 0.01876836
2679.7 0.04330263
1341.4 0.04802006
914.8 0.02921850
1704.1 0.06333392
2933.9 0.05788013
2893.6 0.13829043
2599.7 0.04524590
2491.2 0.03153519
2434.4 0.05925330
3025.2 0.06747982
3198.9 0.08883276
2330.0 0.05294916
1092.5 0.04394815
3286.5 0.06461992
2819.2 0.02969208
3091.4 0.07946930
2663.8 0.14226654
1025.5 0.05122001
943.8 0.08600359
1372.8 0.04556244
1474.7 0.08340562
734.6 0.07991105
2368.1 0.01106930
2368.1 0.07969793
3548.9 0.03530779
1600.1 0.08934881
351.0 0.02131692
271.9 0.02436747
1087.8 0.05793424
602.8 0.02029738
602.8 0.02142968
271.9 0.02501378
3408.8 0.08792115
3901.3 0.10414427
3105.5 0.01871915
4181.4 0.03188445
3612.6 0.08447651
330.6 0.11547887
1250.8 0.04074540
3205.0 0.03198269
0.0 0.01725303
2978.1 0.05379065
3381.2 0.01796538
2067.4 0.07309214
464.1 0.04924766
1600.1 0.04108568
271.9 0.04883840
662.3 0.05504431
3021.0 0.02427216
3480.4 0.09457884
3546.3 0.07975132
2920.0 0.06862655
3381.7 0.04423619
3276.0 0.05266870
134.6 0.05935258
3119.7 0.06401847
3092.1 0.03896322
2701.2 0.08026898
0.0 0.04061641
602.8 0.05559811
3752.9 0.02791703
3425.4 0.05251068
3025.2 0.03769843
908.2 0.02957210
1498.3 0.05079290
5605.3 0.06028622
2105.6 0.11476371
2928.2 0.04447822
3794.8 0.08286896
2019.9 0.06406008
1997.3 0.07307809
3543.2 0.05985179
1461.9 0.11667190
1461.9 0.05793270
1403.4 0.03333838
1435.4 0.13052572
2643.1 0.07388259
2747.8 0.04073698
2209.5 0.08856743
2799.2 0.04831467
2799.2 0.08406460
2747.8 0.05312833
1367.2 0.01416081
941.4 0.08238246
1702.1 0.04931103
1233.0 0.03173109
668.3 0.04113337
3277.6 0.08618158
1714.9 0.06390676
1266.1 0.03933311
2960.6 0.06291936
821.4 0.02071102
425.3 0.07937239
4502.5 0.04421855
2931.0 0.10207200
1435.4 0.03182281
2747.8 0.11577444
2299.3 0.11725053
1476.9 0.05147381
1062.2 0.01855762
793.5 0.06565501
712.1 0.02304651
425.6 0.06477576
1319.4 0.02753462
2865.9 0.17805476
987.5 0.02453435
877.4 0.10281664
931.8 0.02445714
2960.6 0.08561190
2799.2 0.08053775
893.7 0.08592096
862.8 0.05022751
1625.4 0.05866106
2960.6 0.08131941
839.8 0.01620225
1740.3 0.05012399
6016.7 0.02155798
2277.4 0.12036365
3345.9 0.01804115
4203.0 0.04869674
2433.2 0.04841799
2405.3 0.07627173
3950.8 0.02395218
1792.2 0.09038275
1792.2 0.03381058
1431.3 0.01649041
1859.7 0.11142664
3059.9 0.05344263
3171.4 0.03204100
2630.4 0.04994100
3224.1 0.04089862
3224.1 0.07919812
3171.4 0.05443279
1439.4 0.01059849
521.1 0.04266558
1761.8 0.02839388
840.8 0.02319449
247.0 0.02102445
3696.0 0.12542618
2138.2 0.05357134
1396.5 0.01710946
3381.2 0.05974769
904.5 0.01867817
0.0 0.04466965
4916.4 0.05446765
3356.3 0.07387867
1859.7 0.01686272
3171.4 0.08596009
2720.8 0.07870757
1548.8 0.02645157
1056.7 0.01708756
526.3 0.051466187
832.4 0.022035552
1.8 0.021629647
1431.0 0.035662395
3284.2 0.138175616
1131.1 0.009132063
1028.1 0.076734756
1219.6 0.012978527
3381.2 0.061272092
3224.1 0.085877326
631.0 0.041225145
705.9 0.023992599
1702.3 0.053900167
3381.2 0.067760498
425.3 0.021969623
1365.3 0.022874851
1994.7 0.067987825
6536.0 0.034047982
2897.4 0.167207313
3909.9 0.042871715
4888.0 0.047667810
2991.0 0.069322637
2946.2 0.093793290
4636.5 0.036713823
2189.6 0.125897962
2189.6 0.042549449
1391.6 0.031562000
2478.1 0.147187428
3731.9 0.070972489
3821.1 0.049977665
3217.4 0.056899232
3837.1 0.063830774
3837.1 0.098259884
3821.1 0.076105617
1469.2 0.023655600
259.2 0.069251921
2307.2 0.042977501
139.5 0.028620531
547.1 0.038311452
4363.7 0.136971014
2792.8 0.078948304
1524.9 0.048074141
4043.2 0.086274225
1531.4 0.027430299
703.1 0.072361036
5452.4 0.069781509
3981.1 0.098397861
2478.1 0.04056627
3821.1 0.11624424
3380.9 0.10427897
2116.8 0.04104784
1062.5 0.01317386
926.9 0.05596758
1485.5 0.01077573
702.0 0.06283426
1524.0 0.05290646
3952.6 0.20663822
1336.9 0.02776431
1271.4 0.10387460
1608.5 0.03135646
4043.2 0.08056463
3837.1 0.10093206
420.4 0.06065198
1175.6 0.03962920
2265.9 0.05406359
4043.2 0.09068772
1093.5 0.01280321
703.1 0.01291955
/
%Lowess fit
\plot
   0.0000 0.04182983
 135.8347 0.04269947
 271.6694 0.04354620
 407.5041 0.04437308
 543.3387 0.04518818
 679.1735 0.04600149
 815.0081 0.04682454
 950.8428 0.04766412
1086.6775 0.04852242
1222.5122 0.04939071
1358.3469 0.05023585
1494.1815 0.05098154
1630.0162 0.05161383
1765.8510 0.05225700
1901.6855 0.05290677
2037.5203 0.05360913
2173.3550 0.05435274
2309.1897 0.05513250
2445.0244 0.05594724
2580.8591 0.05673512
2716.6938 0.05746653
2852.5283 0.05811325
2988.3630 0.05871053
3124.1978 0.05930601
3260.0325 0.05987519
3395.8672 0.06037228
3531.7019 0.06078522
3667.5366 0.06110360
3803.3711 0.06134486
3939.2058 0.06144255
4075.0405 0.06130348
4210.8755 0.06090686
4346.7100 0.06036695
4482.5444 0.05979092
4618.3794 0.05923154
4754.2139 0.05870369
4890.0488 0.05820807
5025.8833 0.05773950
5161.7183 0.05729052
5297.5527 0.05685356
5433.3877 0.05642115
5569.2222 0.05598700
5705.0566 0.05554455
5840.8916 0.05509169
5976.7261 0.05462041
6112.5610 0.05413264
6248.3955 0.05362155
6384.230 0.05308750
6520.065 0.05253641
6655.900 0.05196465
/
%%%%%%%%%%%%%%%%%%%%%%%%%%%%%%%%%%%%%
\setcoordinatesystem units <.000081in, 0.05556in> point at 0 32
\setplotarea x from 0 to 37000, y from 0 to 18
\axis left shiftedto x=-1000 label {\lines{Genetic\cr Distance}} /
\axis bottom shiftedto y=-0.9 label {Geographic Distance (km)}
  ticks numbered from 0 to 35000 by  5000 /
\plotheading{World}
%Eurasia
\multiput {\tiny .} at
  1448.9   3.1
   847.5   3.8
  1558.5   2.7
  2508.5   7.9
  3649.4   7.5
  3235.4   3.7
  8123.4  14.7
  8247.6  14.5
  7309.8  12.3
 10211.5  11.0
  7603.8  13.1
  7728.0  13.3
  6790.2  10.7
  9691.9   9.8
   556.6   1.8
  7727.0  13.0
  7851.2  13.1
  6913.4  10.6
  9815.1   9.8
  2193.0   2.4
  1847.4   0.5
  8888.3  13.3
  9012.4  13.3
  8074.7  10.8
 10976.3   9.7
  3407.7   2.5
  3190.7   0.5
  1451.4   0.2
  6976.0  13.5
  7100.1  13.0
  6162.4  10.6
  9064.0   9.8
  3068.4   2.5
  2582.3   0.8
  1244.4   0.6
  1999.8   0.7
  6168.7  13.2
  6292.9  13.6
  5355.1  10.4
  8256.8  10.1
  4780.9   3.6
  4391.1   2.0
  5088.3   1.8
  6363.0   2.2
  4545.7   1.6
  9220.3  12.7
  9344.5  13.5
  8406.7  10.8
 11308.4   9.9
  7117.5   3.0
  6899.4   1.5
  7927.4   1.6
  9212.5   2.0
  7550.7   1.5
  3073.6   1.6
 15827.7  14.5
 15951.9  14.9
 15014.1  11.7
 17915.8  10.8
 13300.1   6.1
 13097.3   5.4
 14338.3   5.7
 15742.8   6.1
 14037.9   5.5
  9795.4   5.0
  7423.9   4.0
 14493.1  14.3
 14617.3  14.3
 13679.5  11.5
 16581.2  10.5
 11965.6   6.4
 11762.7   5.4
 13003.7   5.7
 14408.2   6.1
 12703.3   5.5
  8460.8   5.2
  6413.1   4.0
  1340.4   0.6
 10586.5  14.8
 10710.7  15.3
  9772.9  12.1
 12674.6  11.5
  8058.9   6.3
  7856.1   6.1
  9097.1   6.6
 10501.6   7.0
  8796.7   6.0
  4554.2   5.0
  4382.8   4.5
  5468.4   1.1
  4151.1   1.2
 11528.1  13.6
 11652.2  14.1
 10714.5  11.2
 13616.1  10.4
  9000.5   5.9
  8797.7   4.8
 10038.6   5.2
 11443.1   5.5
  9738.2   5.1
  5495.7   4.6
  5324.4   3.3
  4484.4   1.2
  3190.7   0.7
  1999.8   1.5
 13399.7  15.0
 13523.9  15.5
 12586.1  12.5
 15487.8  11.7
 10872.1   7.8
 10669.3   6.8
 11910.3   6.8
 13080.1   7.3
 11609.8   6.5
  7367.4   6.0
  4586.3   5.0
  2923.9   2.3
  1882.3   2.1
  2883.4   2.4
  2940.7   1.6
 11670.2  15.3
 11794.4  16.1
 10856.6  13.4
 13758.3  12.6
  9065.6   8.3
  8961.9   7.3
 10019.2   7.3
 11171.3   8.1
  9820.8   7.0
  5619.4   6.1
  2677.5   4.7
  4746.4   3.0
  3735.6   2.7
  2244.3   3.1
  3563.9   2.3
  1908.8   1.3
 14008.0  14.4
 14132.2  15.0
 13194.4  12.3
 16096.1  11.8
 11403.4   7.0
 11299.7   6.6
 12357.0   6.8
 13509.1   7.4
 12158.6   6.4
  7957.2   5.9
  5015.3   4.7
  2980.6   2.6
  2508.5   2.9
  3986.2   3.1
  4248.0   2.4
  1356.6   1.8
  2337.8   1.8
 13732.6  15.3
 13856.8  15.1
 12919.0  12.7
 15820.7  11.9
 11128.0   7.1
 11024.3   6.4
 12081.5   6.4
 13233.7   7.0
 11883.2   6.1
  7681.8   5.6
  4739.9   4.2
  4884.5   3.1
  4399.3   3.0
  4248.0   3.2
  5361.0   2.7
  2801.8   2.1
  2062.4   1.8
  1999.8   1.6
/
%Pacific vs previous
\multiput {$\times$} at 
  16594.9  17.0
  16719.1  17.6
  15781.3  15.7
  18683.0  14.4
  13990.4  10.4
  13886.1  10.8
  14943.9  12.1
  16096.1  12.2
  14745.5  11.7
  10544.1  12.2
   7602.3  10.4
   6287.0   6.2
   6351.1   5.5
   6886.5   6.8
   7817.3   5.9
   5223.4   7.6
   4924.8   8.1
   3984.1   8.2
   2967.7   8.5
  16745.7  17.7
  16869.9  17.4
  15932.1  15.7
  18833.8  15.0
  14141.1   9.7
  14037.5  10.2
  15094.7  10.8
  16246.9  11.3
  14896.3  10.6
  10694.9  10.8
   7753.1   9.1
   4796.3   5.7
   5119.4   5.9
   6837.0   6.1
   7135.8   6.2
   4555.5   6.8
   5075.6   7.4
   3272.2   6.5
   3421.2   5.8
   1961.7   4.4
  18798.6  16.7
  18922.8  16.9
  17985.0  14.8
  20886.7  14.8
  16271.0   9.6
  16068.2   9.2
  17309.2   9.0
  18546.3   9.7
  17008.8   8.9
  12766.3   8.8
  10052.5   7.2
   4391.1   3.8
   5392.4   4.2
   8709.6   4.8
   8080.3   4.2
   6029.3   3.5
   7375.0   3.8
   5071.3   2.3
   6156.0   2.8
   4969.0   7.9
   3165.1   5.3
  20695.4  16.3
  20819.5  16.6
  19881.8  14.6
  22783.4  14.5
  18167.8  10.0
  17964.9   8.7
  19205.9   8.9
  20610.4   9.6
  18905.5   8.9
  14663.0   8.2
  14491.7   7.1
   8837.1   3.5
   9609.9   3.4
  12115.9   4.0
  10948.8   2.9
  10463.0   2.1
  11818.4   2.7
   9879.9   3.2
  10916.2   3.5
   8961.9   6.0
   7676.0   5.6
   5466.2   2.9
/
%NAmerindian
\multiput {$\circ$} at 
 29286.7  15.4
 29410.9  15.8
 28473.1  12.7
 31374.8  11.2
 26759.1   7.1
 26556.3   6.6
 27797.3   7.4
 29201.8   7.6
 27496.8   6.9
 23254.4   7.3
 20747.9   6.6
 14464.0   4.2
 15536.0   4.1
 19061.9   4.5
 18140.5   4.4
 16886.5   6.3
 18070.4   7.5
 16622.5   6.2
 17600.4   7.9
 17918.0   8.0
 16884.5  10.0
 14990.0   9.3
 17567.1   7.8
/
%SAmerindian
\multiput {$\circ$} at 
 34575.9  15.9
 34700.0  15.8
 33762.2  13.3
 36663.9  12.6
 32048.3   8.0
 31845.4   7.3
 33086.4   8.0
 34490.9   7.9
 32786.0   7.5
 28543.5   9.3
 26037.1   7.2
 19753.2   5.5
 20825.2   5.5
 24351.1   6.0
 23429.6   6.2
 22175.7   7.2
 23359.6   8.9
 21911.7   7.0
 22889.6   8.5
 23207.1   8.9
 22173.7  10.8
 20279.1   8.9
 22856.2   8.4
  7081.4   3.6
/
%Eskimo
\multiput {$\circ$} at 
 25636.8  15.8
 25761.0  15.8
 24823.2  13.2
 27724.9  11.2
 23109.2   5.9
 22906.4   6.2
 24147.4   7.1
 25551.9   7.3
 23847.0   6.7
 19604.5   6.9
 17098.0   5.9
 10814.1   4.1
 11886.2   4.2
 15412.0   4.4
 14490.6   4.5
 13236.6   5.9
 14420.5   7.2
 12972.6   6.0
 13950.6   6.9
 14268.1   6.8
 13234.7   7.2
 11340.1   8.3
 13917.2   6.7
  4988.8   2.3
 11455.2   5.6
/
%%Lowess-smoothed data (All data)
%\plot
%556.600 1.295033 1293.484 1.864975 2030.367 2.430611 2767.251
%2.990693 3504.134 3.544281 4241.018 4.090004 4977.902 4.626326
%5714.785 5.152969 6451.669 5.671564 7188.553 6.185345 7925.436
%6.659620 8662.319 7.056992 9399.203 7.428753 10136.087 7.767186
%10872.970 8.091806 11609.854 8.395116 12346.737 8.670144 13083.621
%8.912766 13820.505 9.124352 14557.388 9.329449 15294.271 9.486019
%16031.155 9.554003 16768.039 9.573334 17504.922 9.559166 18241.807
%9.539662 18978.689 9.513832 19715.574 9.512110 20452.457 9.527680
%21189.340 9.542263 21926.225 9.554329 22663.107 9.581008 23399.992
%9.621318 24136.875 9.670933 24873.758 9.722916 25610.643 9.774335
%26347.525 9.823871 27084.410 9.871354 27821.293 9.917323 28558.176
%9.961836 29295.061 10.005780 30031.943 10.049917 30768.828 10.094101
%31505.711 10.138444 32242.594 10.183835 32979.480 10.229980 33716.363
%10.277512 34453.246 10.326772 35190.13 10.37944 35927.02 10.43325
%36663.90 10.48707
%/
%Lowess-smoothed data (excludes N Am, S Am, and Eskimo)
\plot
556.600 0.9547186 1010.208 1.3449914 1463.816 1.7359731 1917.425
2.1275859 2371.033 2.5194554 2824.641 2.9114504 3278.249 3.3032751
3731.857 3.6945174 4185.465 4.0846567 4639.074 4.4729004 5092.682
4.8588977 5546.290 5.2428999 5999.898 5.6254377 6453.506 5.9969845
6907.115 6.3378859 7360.723 6.6279635 7814.331 6.9210911 8267.939
7.1903429 8721.547 7.4377022 9175.155 7.6701727 9628.764 7.8841267
10082.371 8.1004686 10535.979 8.3218336 10989.588 8.5660076 11443.196
8.8341551 11896.805 9.1241970 12350.412 9.3971615 12804.021 9.6791201
13257.629 9.9639473 13711.237 10.2414436 14164.845 10.5157604
14618.453 10.7918758 15072.062 11.0742970 15525.670 11.3545933
15979.278 11.6350317 16432.887 11.9158306 16886.494 12.1968861
17340.102 12.4778271 17793.711 12.7578545 18247.318 13.0368099
18700.928 13.3142643 19154.535 13.5900526 19608.143 13.8638945
20061.752 14.1366091 20515.359 14.4078979 20968.969 14.6773338
21422.576 14.9438915 21876.18 15.21045 22329.79 15.47701 22783.40
15.74357
/
\endpicture}
\let\put\latexput
\\
\small $\times$ Australia and Pacific; $\circ$, Americas.\\
\end{frame}

\begin{frame}
\frametitle{Wahlund principle: subdivision reduces heterozygosity}
\begin{centering}
\begin{tabular}{lc|ccc}
  & $A_1$ & $A_1A_1$ & $A_1A_2$ & $A_2A_2$\\
\hline
Pop 1          &  4/16& 1/16 & 6/16 & 9/16\\
Pop 2          & 12/16& 9/16 & 6/16 & 1/16\\
Species        &  8/16& 5/16 & \textbf{6/16} & 5/16\\
Hardy-Weinberg &  8/16& 4/16 & \textbf{8/16} & 4/16\\
\hline
\multicolumn{5}{c}{\only<2->{Amount of reduction: 2/16}}
\end{tabular}
\end{centering}

\bigskip

\onslide<3->
\begin{centering}
\begin{tabular}{lcc}
      & $p_i - \bar p$ & $(p_i - \bar p)^2$\\
\hline
Pop 1 &      --4/16   &  1/16 \\
Pop 2 &      4/16     &  1/16 \\
\hline
\multicolumn{2}{l}{Variance:}&1/16\\
\end{tabular}\\

\onslide<4->
\bigskip
\[
\Het_S = \Het_T - 2V \qquad (\hbox{Wahlund 1928})
\]
\end{centering}
\end{frame}

\begin{frame}
\begin{itemize}
\item Wahlund showed that heterozygosity is reduced by group
  differences in allele frequencies.
\item Buri's experiment illustrates that
\begin{itemize}
\item Drift reduces heterozygosity
\item Drift increases group differences
\end{itemize}
\item We need a theory to connect these facts.
\item Let us build one on top of what we have already.
\end{itemize}
\end{frame}

\begin{frame}
\frametitle{What we already know about heterozygosity}
\[
\begin{array}{rcll}
E[\Het_t | p_t] &=& 2 p_t q_t & \hbox{(Hardy-Weinberg)}\\
E[\Het_t | p_0] &=& 2p_0q_0 \left(1 -\frac{1}{2N}\right)^t & \hbox{(Ch.~2)}\\
\end{array}
\]
How can these both be true?

\pause
It must be true that
\[
E[p_t q_t | p_0] = p_0q_0 \left(1 -\frac{1}{2N}\right)^t
\]

\pause
On the other hand, it is also true that
\[
\begin{array}{rcll}
E[p_t q_t | p_0] &=& p_0 q_0 -V_t & \hbox{(Wahlund)}
\end{array}
\]
where $V_t$ is the variance of $p_t$ about $p_0$.
\end{frame}

\begin{frame}
Rearranging gives the variance among groups
\[
V = p_0q_0\left[ 1 - \left(1 - \frac{1}{2N}\right)^t\right]
\]

\pause
We usually normalize this expression by dividing both sides by $p_0q_0$.  The
result is called $F_{ST}$:
\[
F_{ST} = \frac{V}{p_0 q_0} = 1 - \left(1 - \frac{1}{2N}\right)^t
\]

In data analysis, we take $p_0 \approx \bar p$, the current population mean.
\end{frame}

\begin{frame}
  \frametitle{Observed $F_{ST}$ in Buri's experiments}
  \begin{center}
    \mbox{\let\put\pictexput\sf%
\beginpicture
%%%%%%%%%%%%%% left plot
\setcoordinatesystem units <0.075in, 7.35in> point at 19 0
\setplotarea x from 0 to 19, y from 0 to 0.17
\axis left label {$F_{ST}$}
  ticks numbered from 0.0 to 0.15 by  0.05 /
\axis bottom   label {Generation} 
  ticks numbered from 0 to 15 by 5 /
\put {Series I} [rb] at 19 0.01
%data
\multiput {$\bullet$} at
%Gen.      FstI
   0  0.0000000
   1  0.0060000
   2  0.0260000
   3  0.0310000
   4  0.0420000
   5  0.0500000
   6  0.0550000
   7  0.0620000
   8  0.0720000
   9  0.0830000
  10  0.0900000
  11  0.1050000
  12  0.1120000
  13  0.1230000
  14  0.1360000
  15  0.1400000
  16  0.1550000
  17  0.1600000
  18  0.1650000
  19  0.1700000
/
%%%%%%%%%%%%%% right plot
\setcoordinatesystem units <0.075in, 7.35in> point at -1 0
\setplotarea x from 0 to 19, y from 0 to 0.17
\axis right label {$F_{ST}$}
  ticks numbered from 0.0 to 0.15 by  0.05 /
\axis bottom   label {Generation} 
  ticks numbered from 0 to 15 by 5 /
\put {Series II} [rb] at 19 0.01
%data
\multiput {$\bullet$} at
%Gen.     FstII
   0  0.0000000
   1  0.0090000
   2  0.0180000
   3  0.0240000
   4  0.0300000
   5  0.0450000
   6  0.0520000
   7  0.0600000
   8  0.0700000
   9  0.0720000
  10  0.0760000
  11  0.0840000
  12  0.0940000
  13  0.1010000
  14  0.1050000
  15  0.1170000
  16  0.1220000
  17  0.1330000
  18  0.1390000
  19  0.1440000
/
\endpicture\let\put\latexput}


  \end{center}
\end{frame}

\begin{frame}
  \frametitle{Observed and expected $F_{ST}$ in Buri's experiments}
\begin{center}
  \let\put\pictexput
\mbox{%
\sf
\beginpicture
%%%%%%%%%%%%%% left plot
\setcoordinatesystem units <0.075in, 7.35in> point at 19 0
%\setcoordinatesystem units <0.151in, 14.7in> point at 19 0
\setplotarea x from 0 to 19, y from 0 to 0.17
\axis left label {$F_{ST}$}
  ticks numbered from 0.0 to 0.15 by  0.05 /
\axis bottom   label {Generation} 
  ticks numbered from 0 to 15 by 5 /
\put {Series I} [rb] at 19 0.01
\put {\begin{tabular}{cl}
$\circ$ & theory\\
$\bullet$ & data
\end{tabular}} [lt] at 1 0.15
\multiput {$\circ$} at
%Gen.     EFstI
   0  0.0000000
   1  0.0138889
   2  0.0270062
   3  0.0393947
   4  0.0510950
   5  0.0621453
   6  0.0725817
   7  0.0824382
   8  0.0917472
   9  0.1005390
  10  0.1088424
  11  0.1166845
  12  0.1240909
  13  0.1310859
  14  0.1376922
  15  0.1439315
  16  0.1498242
  17  0.1553896
  18  0.1606457
  19  0.1656098
/
\multiput {$\bullet$} at
%Gen.      FstI
   0  0.0000000
   1  0.0060000
   2  0.0260000
   3  0.0310000
   4  0.0420000
   5  0.0500000
   6  0.0550000
   7  0.0620000
   8  0.0720000
   9  0.0830000
  10  0.0900000
  11  0.1050000
  12  0.1120000
  13  0.1230000
  14  0.1360000
  15  0.1400000
  16  0.1550000
  17  0.1600000
  18  0.1650000
  19  0.1700000
/
%%%%%%%%%%%%%% right plot
\setcoordinatesystem units <0.075in, 7.35in> point at -1 0
%\setcoordinatesystem units <0.15in, 14.7in> point at -1 0
\setplotarea x from 0 to 19, y from 0 to 0.17
\axis right label {$F_{ST}$}
  ticks numbered from 0.0 to 0.15 by  0.05 /
\axis bottom   label {Generation} 
  ticks numbered from 0 to 15 by 5 /
\put {Series II} [rb] at 19 0.01
%theory
\multiput {$\circ$} at
%Gen.  EFstII
   0  0.00000
   1  0.01087
   2  0.02127
   3  0.03121
   4  0.04072
   5  0.04982
   6  0.05853
   7  0.06685
   8  0.07481
   9  0.08243
  10  0.08972
  11  0.09669
  12  0.10335
  13  0.10973
  14  0.11583
  15  0.12166
  16  0.12724
  17  0.13258
  18  0.13768
  19  0.14257
/
\multiput {$\bullet$} at
%Gen.     FstII
   0  0.0000000
   1  0.0090000
   2  0.0180000
   3  0.0240000
   4  0.0300000
   5  0.0450000
   6  0.0520000
   7  0.0600000
   8  0.0700000
   9  0.0720000
  10  0.0760000
  11  0.0840000
  12  0.0940000
  13  0.1010000
  14  0.1050000
  15  0.1170000
  16  0.1220000
  17  0.1330000
  18  0.1390000
  19  0.1440000
/
\endpicture}
\let\put\latexput

\end{center}  
\end{frame}

\begin{frame}
\frametitle{Application to differences among human races}

Theory assumes \emph{no} migration.  Seems unlikely, but let's see
where it leads.

\bigskip

For major human populations, $F_{ST} \approx 0.1$. Modern humans have
been in Europe and E Asia for 50,000 years (2000 generations).  Plug
this into
\[
F_{ST} = 1 - \left(1 - \frac{1}{2N_e}\right)^t
\]
and solve for $N_e$.

\bigskip

Answer: $N_e \approx 10,000$: agrees with genetic diversity
within populations.
\end{frame}

\begin{frame}
\frametitle{The role of migration}
\begin{itemize}
\item Drift increases group differences.
\item Migration $(m)$ reduces them
\item Eventually, an equilibrium is reached.
\end{itemize}
\[
F_{ST} = \frac{1}{4N_e m + 1}
\]
Depends only on $N_e m$.
Small if $N_e m > 1$.

\bigskip
\pause
Human value ($\approx 1/9$) implies $N_e m \approx 2$.
\end{frame}

\begin{frame}
\frametitle{Summary}
\begin{itemize}
\item Drift increases variance among groups and reduces that within them.
\item Wahlund's principle: $\Het_S = \Het_T - 2V$
\item $F_{ST}$ measures between-group variance relative to total
  variance.
\item If major human populations have been isolated for 50,000~y, then
  $N_e \approx 10,000$.
\item If they are at migration-drift equilibrium, then pairs of
  populations exchange $\sim$2 migrants per generation.
\end{itemize}
\end{frame}

\end{document}

