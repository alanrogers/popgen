% -*-latex-*-
\documentclass[pdftex,12pt]{beamer}
%\documentclass[pdftex,handout]{beamer}
\usepackage{etex}
\setbeamertemplate{footline}[frame number]
\title{Population Structure with Genomic Data}
\author{Alan R. Rogers}
\date{\today}
\setbeamercovered{transparent}
\setbeamertemplate{footline}[frame number]
\begin{document}

\frame{\titlepage}

\begin{frame}
\frametitle{Shared blocks of identity by descent (IBD)}

\begin{itemize}
\item With molecular data, we can identify shared IBD blocks.
\item Close relatives tend to share long blocks.
\item Distant relatives share short blocks.
\end{itemize}
\end{frame}

\begin{frame}
\frametitle{IBD sharing between my mother and daughter}
\begin{columns}
\column{0.5\textwidth}
\includegraphics[height=0.8\textheight]{NalaNana-1-11.png}
\column{0.5\textwidth}
\includegraphics[height=0.8\textheight]{NalaNana-12-X.png}
\end{columns}
\end{frame}

\begin{frame}
\frametitle{IBD sharing with French-speaking Swiss and UK}
\begin{columns}
\column{0.3\textwidth}
\includegraphics[height=0.8\textheight]{RalphCoop-ital-scat.png}
\column{0.7\textwidth}
Each dot an individual. Color shows individual's origin. {\small (FR,
  France; IT, Italy; EL, Greece; TR, Turkey, CY, Cyprus)\\}

\bigskip

Those who share ancestry with Swiss also share ancestry with UK.

\bigskip

Presumably reflects immigration from ancestors of Swiss and British.

\bigskip

\mbox{}\hfill {\small Ralph and Coop (2013)\\}
\end{columns}
\end{frame}

\begin{frame}
\frametitle{Sharing with Irish and Germans}
\begin{columns}
\column{0.3\textwidth}
\includegraphics[height=0.8\textheight]{RalphCoop-uk-scat.png}
\column{0.7\textwidth}
Each dot an individual. Pink, Germans; Green, British; Blue, Irish.

\bigskip

Brits with lots of German ancestry have little Irish ancestry.

\bigskip

British population is a mixture of Celts and Germans.

\bigskip

If we focused on shorter IBD blocks, we'd be studying a different time
scale, and the pattern might be different.
\end{columns}
\end{frame}

\begin{frame}
\frametitle{Geographic decay of recent relatedness}
{\centering\includegraphics[width=\linewidth]{coop-ggdist.png}\\}
Genetic similarity versus geographic distance. Small dots are pairs of
individuals. \small E, Eastern Europe; W, Western
Europe; N, Northern Europe; I, Italy \& Iberia; TC, Turkey
\& Cyprus.  \hfill\footnotesize(Ralph \& Coop 2013)\\
%white, black, red, green, blue, cyan, magenta, yellow.
\end{frame}

\begin{frame}
\begin{columns}
\column{0.7\textwidth}
\includegraphics[height=\textheight]{Leslie-map.png}
\column{0.3\textwidth}
\textcolor{blue}{Genetic map of Britain}

\bigskip

Colors indicate genetically similar groups.
\end{columns}
\end{frame}

\begin{frame}
\begin{columns}
\column{0.5\textwidth}
\includegraphics[width=\linewidth]{Leslie-map-crop.png}
\column{0.5\textwidth}
\includegraphics[width=\linewidth]{Leslie-map-9600BC.png}
\end{columns}
\end{frame}

\begin{frame}
\begin{columns}
\column{0.5\textwidth}
\includegraphics[width=\linewidth]{Leslie-map-crop.png}
\column{0.5\textwidth}
\includegraphics[width=\linewidth]{Leslie-map-43AD.png}
\end{columns}
\end{frame}

\begin{frame}
\begin{columns}
\column{0.5\textwidth}
\includegraphics[width=\linewidth]{Leslie-map-crop.png}
\column{0.5\textwidth}
\includegraphics[width=\linewidth]{Leslie-map-600AD.png}
\end{columns}
\end{frame}

\begin{frame}
\begin{columns}
\column{0.5\textwidth}
\includegraphics[width=\linewidth]{Leslie-map-crop.png}
\column{0.5\textwidth}
\includegraphics[width=\linewidth]{Leslie-map-800AD.png}
\end{columns}
\end{frame}

\begin{frame}
\frametitle{Summary}
\begin{itemize}
\item By distinguishing between long and short IBD blocks, we can
  examine geographic population structure on different time scales.
\end{itemize}
\end{frame}

\end{document}

