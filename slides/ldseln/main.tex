% -*-latex-*-
%\documentclass[handout]{beamer}
\documentclass[12pt]{beamer}
\usepackage{etex}

\let\latexput\put
\usepackage{pictex}
\let\pictexput\put
\let\put\latexput

\usepackage{color}
\setbeamercovered{transparent}
%\usepackage{pst-plot,color,pstricks}
%\setlength{\topmargin}{-1.7in}
\begin{document}

\title{Why LD Helps Us Find Selective Sweeps}
\author{Alan R. Rogers}
\date{\today}
\frame{\titlepage}

\begin{frame}
\frametitle{LD helps us detect ongoing selective sweeps}

This is curious, because we have known for over 50 years that
selection at a single locus does not cause LD (Felsenstein 1965).

\bigskip
\pause

This lecture will explain why selective sweeps are \emph{associated}
with LD even though they don't cause it.

\bigskip

We begin with a graphical argument that shows why selection at a
single locus doesn't cause LD.
\end{frame}

\begin{frame} 
\frametitle{Linkage equilibrium $\Longleftrightarrow$ shaded
  fractions equal} 
\centering
\begin{columns}
\column{0.7\textwidth}
\let\put\pictexput
\mbox{\beginpicture
\setcoordinatesystem units <2cm,2cm>
\setplotarea x from 0 to 3.8, y from 0.0 to 2.4
\put {\makebox(0,0)[c]{\large $A$-gametes}} at 1 2.2
\put {\large $b$} at 1 1.5
\put {\large $B$} at 1 0.5
\circulararc 360 degrees from 2.000000 1.000000 center at 1.000000 1.000000
\plot 1.000000 1.000000 0.292893 0.292893 /
\plot 1.000000 1.000000 1.707107 0.292893 /
\setshadesymbol <z,z,z,z> ({.})
\setshadegrid span <3.5pt> point at 1.000000 1.000000
\vshade
0.292893 0.292893 0.292893
0.500000 0.133975 0.500000
0.741181 0.034074 0.741181
1.000000 0.000000 1.000000
1.258819 0.034074 0.741181
1.500000 0.133975 0.500000
1.707107 0.292893 0.292893
/
\put {\makebox(0,0)[c]{\large $a$-gametes}} at 3 2
\put {\large $b$} at 3 1.4
\put {\large $B$} at 3 0.6
\circulararc 360 degrees from 3.800000 1.000000 center at 3.000000 1.000000
\plot 3.000000 1.000000 2.434315 0.434315 /
\plot 3.000000 1.000000 3.565685 0.434315 /
\setshadesymbol <z,z,z,z> ({\large .})
\setshadegrid span <3.5pt> point at 3.000000 1.000000
\vshade
2.434315 0.434315 0.434315
2.600000 0.307180 0.600000
2.792945 0.227259 0.792945
3.000000 0.200000 1.000000
3.207055 0.227259 0.792945
3.400000 0.307180 0.600000
3.565685 0.434315 0.434315
/
\endpicture}
\let\put\latexput

\column{0.35\textwidth}
\raggedright
LE: Neither locus predicts other

\bigskip

Here, $B$ is equally common among $a$-gametes and $A$-gametes.
\end{columns}
\end{frame}

\begin{frame} 
\frametitle{Suppose allele $A$ is favored; $B/b$ are neutral}
\begin{columns}
\column{0.7\textwidth}
\let\put\pictexput
\mbox{\beginpicture
\setcoordinatesystem units <2cm,2cm>
\setplotarea x from 0 to 3.8, y from 0.0 to 2.4
\put {\makebox(0,0)[c]{\large $A$-gametes}} at 1 2.2
\put {\large $b$} at 1 1.5
\put {\large $B$} at 1 0.5
\circulararc 360 degrees from 2.000000 1.000000 center at 1.000000 1.000000
\plot 1.000000 1.000000 0.292893 0.292893 /
\plot 1.000000 1.000000 1.707107 0.292893 /
\setshadesymbol <z,z,z,z> ({.})
\setshadegrid span <3.5pt> point at 1.000000 1.000000
\vshade
0.292893 0.292893 0.292893
0.500000 0.133975 0.500000
0.741181 0.034074 0.741181
1.000000 0.000000 1.000000
1.258819 0.034074 0.741181
1.500000 0.133975 0.500000
1.707107 0.292893 0.292893
/
\put {\makebox(0,0)[c]{\large $a$-gametes}} at 3 2
\put {\large $b$} at 3 1.4
\put {\large $B$} at 3 0.6
\circulararc 360 degrees from 3.800000 1.000000 center at 3.000000 1.000000
\plot 3.000000 1.000000 2.434315 0.434315 /
\plot 3.000000 1.000000 3.565685 0.434315 /
\setshadesymbol <z,z,z,z> ({\large .})
\setshadegrid span <3.5pt> point at 3.000000 1.000000
\vshade
2.434315 0.434315 0.434315
2.600000 0.307180 0.600000
2.792945 0.227259 0.792945
3.000000 0.200000 1.000000
3.207055 0.227259 0.792945
3.400000 0.307180 0.600000
3.565685 0.434315 0.434315
/
\endpicture}
\let\put\latexput

\column{0.35\textwidth}
\raggedright
Selection inflates left circle.

\medskip

Shrinks right one.

\medskip

Shaded wedges unchanged.

\medskip

System remains in LE.
\end{columns}
\end{frame}

\begin{frame}
  \frametitle{If selection doesn't \emph{cause} LD, why does LD help
    us detect selection?}

$D$, the conventional measure of LD, doesn't help us here, because it
  \emph{is} affected by selection---but not in a way that is helpful.

  \bigskip

To explain this puzzle, I introduce another measure, which does not
change in response to selection at a single locus.  
\end{frame}

\begin{frame}
\frametitle{Conditional allele frequencies}
{\centering
\begin{tabular}{lcccc}
Gamete type & $AB$ & $Ab$ & $aB$ & $ab$\\
Frequency   & $x_1$& $x_2$& $x_3$& $x_4$
\end{tabular}\\}

\vskip 1cm

\begin{columns}
\column{0.5\textwidth}
\fbox{\begin{minipage}{\linewidth}
Freq of $B$ among $A$-gametes
\[
p_{B|A} = \frac{x_1}{x_1+x_2}
\]
The size of one pie slice.
\end{minipage}}
\column{0.5\textwidth}
\fbox{\begin{minipage}{\linewidth}
Freq of $B$ among $a$-gametes
\[
p_{B|a} = \frac{x_3}{x_3+x_4}
\]
The size of the other slice.
\end{minipage}}
\end{columns}

\vskip 1cm

\centering
Neither is affected by selection on $A$/$a$.\\
\end{frame}

\begin{frame}
\frametitle{The Nei-Li measure of linkage disequilibrium}
\[
d = p_{B|A} - p_{B|a}
\]
\mbox{} \hfill\footnotesize (Nei \& Li, 1980)

\bigskip
Difference in frequency of $B$ between $A$- and $a$-gametes.

\bigskip\pause

Selection on $A/a$ affects neither $p_{B|A}$, $p_{B|a}$, nor $d$.
\end{frame}

\begin{frame}
\frametitle{Calculating $D$ and $d$}
\begin{columns}
\column{0.3\textwidth}
\centering
\begin{tabular}{ccc}
       & \multicolumn{2}{c}{Locus}\\
Gamete &  1 & 2\\ \hline
 1&A&B\\
 2&A&B\\
 3&A&B\\
 4&A&B\\
 5&A&B\\
 6&A&b\\
 7&a&B\\
 8&a&B\\
 9&a&b\\
10&a&b\\
\end{tabular}
\column{0.7\textwidth}
\centering
\begin{tabular}{c|cc|c}
 \multicolumn{1}{c}{} & A & \multicolumn{1}{c}{a}\\
\cline{2-3}
B & 5 & 2 & 7\\
b & 1 & 2 & 3\\
\cline{2-3}
\multicolumn{1}{c}{} 
  & 6 & \multicolumn{1}{c}{4} & 10
\end{tabular}\\
\[
D = \frac{5}{10} \cdot \frac{2}{10} - \frac{1}{10}\cdot\frac{2}{10}
= \frac{2}{25}
\]
\[
d = p_{B|A} - p_{B|a} = \frac{5}{6} - \frac{1}{2} = \frac{1}{3}
\]
%\pause
%\begin{eqnarray*}
%d &=& D\big/\bigl(p_A(1-p_A)\bigr)
% = 
%\frac{2}{25} \bigg / \left(\frac{6}{10}\cdot \frac{4}{10}\right)\\
% &=& 1/3
%\end{eqnarray*}
\end{columns}
\end{frame}

\begin{frame}
\frametitle{$D$ depends on heterozygosity at locus~$A$.}
\begin{eqnarray*}
d &=& \frac{x_1}{x_1+x_2} - \frac{x_3}{x_3+x_4}\\
 &=& \frac{D}{p_A (1-p_A)}\\[2ex]
\pause 
D &=& d p_A (1-p_A)
\end{eqnarray*}
Selection at locus $A$ affects $p_A(1-p_A)$ and therefore $D$.
\end{frame}

\begin{frame}[containsverbatim]
  \frametitle{Mutation generated LD}
A tiny population before and after mutation:  
\begin{verbatim}
__a_____B__                 __a_____B__

__a_____B__  --Mutation-->  __A_____B__

__a_____b__                 __a_____b__

__a_____b__                 __a_____b__  
\end{verbatim}
Mutation occurs on a $B$ gamete. After mutation, $\Pr[B|A] = 1$ but
$\Pr[B|a] = 1/3$, so mutation has created LD.
\end{frame}  

\begin{frame}
\frametitle{LD of a new mutant}
When allele $A$ first arises by mutation, it will be rare, so
$p_A(1-p_A) \approx 0$ and $D \approx 0$.

\bigskip

But $d$ may be large: it is either $1-p_B$ or $-p_B$, depending on
whether the mutant gamete is $AB$ or $Ab$.

\bigskip

The value of $d$ stays large if selection outruns recombination and
drift.

\bigskip

\emph{This is why LD helps us detect selection.}
\end{frame}

\begin{frame}
\frametitle{Selective sweep of $A$, with $s=0.02$, $c=0.001$, and
  $2N=50,000$} 
\includegraphics[width=0.9\textwidth]{bigsweep.png}
\end{frame}

\begin{frame}
\frametitle{Neutral drift to fixation, $c=0.001$, and
  $2N=50,000$} 
\includegraphics[width=0.9\textwidth]{bigs0.png}
\end{frame}

\begin{frame}
\frametitle{Selective sweep in small population ($s=0.02$, $c=0.001$, 
  $2N=5000$)} 
\includegraphics[width=0.8\textwidth]{smallsweep.png}
\end{frame}

\begin{frame}
\frametitle{Summary}
\begin{itemize}
\item LD decays gradually in response to recombination.
\item This is more obvious for $d$ than for $D$, because $d$ is
  insensitive to $p_A$.
\item Advantageous alleles increase rapidly: there is little time for
  LD to decay.
\item Neutral alleles increase slowly; plenty of time for LD to decay.
\end{itemize}
\end{frame}

\end{document}

