% -*-latex-*-
\documentclass[]{beamer} 
%\documentclass[handout]{beamer} 
\usepackage{etex}
\usepackage{hyperref}
\hypersetup{
    colorlinks=true,
    linkcolor=blue,
    filecolor=magenta,      
    urlcolor=cyan,
    }
\urlstyle{same}

\let\latexput\put
\usepackage{pictex}
\let\pictexput\put
\let\put\latexput

\newdimen\thusfar
\newdimen\plotht
\newdimen\plotwd
\newdimen\plotsp
\newdimen\xunit
\newdimen\yunit
\newdimen\offsety
\title{The Site Frequency Spectrum}
\author{Alan R. Rogers}
\date{\today}
\begin{document}

\frame{\titlepage}

\begin{frame}
\frametitle{A site frequency spectrum}
\begin{columns}
\column{0.5\textwidth}
%-*-latex-*-
\let\put\pictexput
\mbox{%
\beginpicture
\headingtoplotskip=0mm
\valuestolabelleading=0.2\baselineskip
\setcoordinatesystem units <0.07in, 0.003in>
\setplotarea x from 0 to 20, y from 0 to 436
\axis bottom 
  label {Derived allele copies}
  ticks withvalues 1 5 10 15 20 / at 1 5 10 15 20 / /
\axis left label {\lines{Number\cr of\cr sites}} /
%\plotheading{Y chromosome}
%\put{$\begin{array}{rcl}
%      K &=& 718\\ % number of sequences in sample
%      S &=& 20\\  % number of segregating sites
%       \end{array}$} [tr] at 20 436
%Observed site frequency spectrum
\sethistograms
\plot
 0 0
 1 122
 2 85
 3 28
 4 27
 5 5
 6 1
 7 1
 8 436
 9 1
10 10
11 6
12 8
13 5
14 1
15 1
16 34
17 1
18 10
19 1
20 10
/                                              
\endpicture}
\let\put\latexput

\column{0.5\textwidth}
Y chromosome data from Underhill et al.

\bigskip

$K=718$ DNA sequences

\end{columns}
\end{frame}

\begin{frame}
  \frametitle{Calling ancestral and derived alleles}
  {\centering\input{figancder}\\[2ex]}

  \begin{itemize}
  \item Two hypotheses about which allele is ancestral.
  \item ``C'' requires 2 mutations; ``T'' requires 1.
  \item Because mutations are rare, ``T'' is more likely.
    \item When the in-group is polymorphic, the ancestral allele is
      usually the one present in the out-group.
  \end{itemize}
\end{frame}

\begin{frame}[containsverbatim]
\frametitle{Unfolded site frequency spectrum}
\begin{columns}
\column{0.4\textwidth}
\begin{verbatim}
       123456
Human1 AATAGC
Human2 ..AC..
Human3 .TACT.
Human4 ..ACT.
-------------
Chimp  AAAATC
\end{verbatim}

\vfill

\mbox{}\\
\column{0.6\textwidth}
1: fixed.

2: T derived; singleton.

3: T derived; singleton.

4: C derived; tripleton.

5: G derived; doubleton.

6: fixed.

\hrulefill\\

\begin{tabular}{lr}
Singletons & 2\\
Doubletons & 1\\
Tripletons & 1
\end{tabular}
\end{columns}
\end{frame}

\begin{frame}[containsverbatim]
\frametitle{A site's position in spectrum depends on position in gene
  tree}
\begin{columns}
\column{0.5\textwidth}
\begin{verbatim}
----A------
           |
           |----B---
           |        |
-----------         |
                    |-----
                    |
                    |
------C-------------
\end{verbatim}
\column{0.5\textwidth}
\raggedleft
Mutations A and C are singletons; B is a doubleton

\bigskip

Most recent interval: singletons only

\bigskip

2nd most recent: singletons and doubletons

\bigskip

3rd most recent: singletons, doubletons, and tripletons
\end{columns}
\end{frame}

\begin{frame}[containsverbatim]
\frametitle{A tree with 2 leaves has only singletons}
\begin{verbatim}

     -------------------
                        |
                        |
                        |-----------
                        |
                        |
     -------------------

    |- 2N generations --| 

\end{verbatim}
We expect $4Nu=\theta$ mutations, all singletons.
\end{frame}

\begin{frame}
\frametitle{3 leaves $\Rightarrow$ $\theta$ singletons and $\theta/2$
doubletons}
{\centering\mbox{\beginpicture
\def\mutation{\tiny$\bullet$}
\linethickness=0.75pt
\setplotsymbol ({\small .})
\valuestolabelleading=\baselineskip
\tickstovaluesleading=0.5\baselineskip
%
%%%%%%%%%%%%%% Tree in PicTeX format %%%%%%%%%%%%%%
%
\setcoordinatesystem units <0.5\linewidth, 10ex>
\setplotarea x from 0.000000 to 1.463000, y from 0.000000 to 2.000000
\axis left invisible ticks length <0pt> withvalues  {C}  {B}  {A}  /
at  0  1  2  / /
\axis bottom shiftedto y=-0.2 label {Generations} 
  ticks withvalues 0 {$2N/3$} {$2N\bigl(\frac{1}{3} + 1\bigr)$} / 
  at 0 0.33 1.33 / /
\plot 0.000000 0.000000 1.330000 0.000000 /
\plot 0.000000 1.000000 0.330000 1.000000 /
\plot 0.000000 2.000000 0.330000 2.000000 /
\plot 0.330000 2.000000 0.330000 1.000000 /
\plot 0.330000 1.500000 1.330000 1.500000 /
\plot 1.330000 1.500000 1.330000 0.000000 /
\plot 1.330000 0.750000 1.463000 0.750000 /
\endpicture}
\\[2.5ex]}

Expect $\theta$ singletons to arise during oldest interval.

\bigskip

Half of these become doubletons at coalescent event: we end up with
$\theta/2$ doubletons.

\bigskip

New singletons in recent interval: $\frac{2N}{3}
\times 3 \times u = 2Nu = \theta/2$. This gain exactly compensates for
the singletons that became doubletons.
\end{frame}

\begin{frame}
\frametitle{Expected spectrum in population of constant size}
\begin{center}
  \begin{tabular}{cl}
Sample & Expected spectrum\\
size   & (singletons, doubletons, $\ldots$)\\ \hline
2 & $\theta$\\
3 & $\theta, \quad \theta/2$\\
4 & $\theta, \quad \theta/2, \quad \theta/3$\\
5 & $\theta, \quad \theta/2, \quad \theta/3, \quad \theta/4$\\
\multicolumn{2}{c}{Etcetera}
\end{tabular}
\end{center}
It is remarkable that as we increase sample size, the number of
mutants in each category doesn't change.  We merely add a new category 
at the right side of the spectrum.

\bigskip

Details: \href{https://arxiv.org/abs/2103.00335}{Rogers \& Wooding 2021}
\end{frame}

\begin{frame}
\frametitle{How to estimate $\theta$?}
To use this formula with data, we need an estimate of $\theta$.

The sum of the observed spectrum is the number, $S$, of segregating
sites. 

This is true of the theoretical spectrum only if you use
$\hat\theta_S$ to estimate $\theta$.
\end{frame}

\begin{frame}
\centering%-*-latex-*-
\let\put\pictexput
\mbox{%
\beginpicture
\headingtoplotskip=0mm
\valuestolabelleading=0.2\baselineskip
%%%%%%%%%%%%%%%% row 1 col 1 %%%%%%%%%%%%%%%%
%point at 2.125in 0in
\setcoordinatesystem units <2.5in, 1.51in> point at 0.85 0
\setplotarea x from 0 to 0.5, y from 0 to 0.83
\axis left label {\lines{Frac.\cr of\cr sites}}  /
\axis bottom ticks numbered from 0.0 to 0.5 by 0.5 /
\plotheading{MtDNA$^{a,b}$}
\put{$\begin{array}{rcl}
      K &=& 636\\ % number of sequences in sample
      S &=& 226\\  % number of segregating sites
       \end{array}$} [tr] at 0.5 0.83
%Expected site frequency spectrum
\multiput {\footnotesize$\bullet$} at
0.013 0.475 0.038 0.104 0.062 0.062 0.088 0.045 0.113 0.036
0.138 0.030 0.163 0.026 0.188 0.023 0.213 0.020 0.237 0.019
0.263 0.016 0.288 0.016 0.312 0.015 0.338 0.014 0.363 0.013
0.388 0.013 0.413 0.012 0.438 0.012 0.463 0.011 0.488 0.011
/
%Observed site frequency spectrum
\sethistograms
\plot
0.000 0.000 0.025 0.827 0.050 0.075 0.075 0.044 0.100 0.013
0.125 0.013 0.150 0.004 0.175 0.004 0.200 0.004 0.225 0.000
0.250 0.000 0.275 0.000 0.300 0.004 0.325 0.000 0.350 0.004
0.375 0.000 0.400 0.004 0.425 0.000 0.450 0.000 0.475 0.000
0.500 0.000
/
%%%%%%%%%%%%%%%% row 1 col 2 %%%%%%%%%%%%%%%%
% point at 0.625in 0in
\setcoordinatesystem units <1.25in, 1.47in> point at 0.5 0
\setplotarea x from 0 to 1, y from 0 to 0.85
\axis bottom ticks numbered from 0 to 1 by 1 /
\plotheading{Y$^{b,c}$}
\put{$\begin{array}{rcl}
      K &=& 718\\ % number of sequences in sample
      S &=& 20\\  % number of segregating sites
       \end{array}$} [tr] at 1 0.85
%Expected site frequency spectrum
\multiput {\footnotesize$\bullet$} at
0.033 0.349
0.100 0.161
0.167 0.100
0.233 0.071
0.300 0.055
0.367 0.045
0.433 0.038
0.500 0.032
0.567 0.029
0.633 0.026
0.700 0.024
0.767 0.022
0.833 0.020
0.900 0.017
0.967 0.010                      
/
%Observed site frequency spectrum
\sethistograms
\plot
0.000 0.000
0.067 0.850
0.133 0.050
0.200 0.050
0.267 0.000
0.333 0.000
0.400 0.000
0.467 0.000
0.533 0.000
0.600 0.000
0.667 0.000
0.733 0.000
0.800 0.000
0.867 0.000
0.933 0.000
1.000 0.000
/                                              
%%%%%%%%%%%%%%%% row 2 col 1 %%%%%%%%%%%%%%%%
% point at 2.125in 1.85in
\setcoordinatesystem units <2.5in, 2.272in> point at 0.85 0.814
\setplotarea x from 0 to 0.5, y from 0 to 0.55
\axis left label {\lines{Frac.\cr of\cr sites}} /
\axis bottom ticks numbered from 0.0 to 0.5 by 0.5 /
\plotheading{$\beta$-globin$^e$}
\put{$\begin{array}{rcl}
      K &=& 253\\ % number of sequences in sample
      S &=& 33\\  % number of segregating sites
       \end{array}$} [tr] at 0.5 0.55
%Expected site frequency spectrum
\multiput {\footnotesize$\bullet$} at
0.025 0.516
0.075 0.125
0.125 0.071
0.175 0.057
0.225 0.046
0.275 0.036
0.325 0.034
0.375 0.031
0.425 0.026
0.475 0.026           
/
%Observed site frequency spectrum
\sethistograms
\plot
0.000 0.000
0.050 0.545
0.100 0.061
0.150 0.000
0.200 0.030
0.250 0.000
0.300 0.030
0.350 0.182
0.400 0.061
0.450 0.030
0.500 0.061
/                 
%%%%%%%%%%%%%%%% row 2 col 2 %%%%%%%%%%%%%%%%
% point at 0.625in 1.85in
\setcoordinatesystem units <2.5in, 3.125in> point at 0.25 0.592
\setplotarea x from 0 to 0.5, y from 0 to 0.4
\axis bottom ticks numbered from 0.0 to 0.5 by  0.5 /
\plotheading{Misc X$^f$}
\put{$\begin{array}{rcl}
      K &=& 10\\ % number of sequences in sample
      S &=& 20\\  % number of segregating sites
       \end{array}$} [tr] at 0.5 0.4
%Expected site frequency spectrum
\multiput {\footnotesize$\bullet$} at
0.050 0.393
0.150 0.216
0.250 0.157
0.350 0.128
0.450 0.071
/
%Observed site frequency spectrum
\sethistograms
\plot
0.000 0.000
0.100 0.400
0.200 0.200
0.300 0.100
0.400 0.250
0.500 0.050
/                                       
\endpicture}
\let\put\latexput
\\
\end{frame}

\begin{frame}
  \begin{columns}
    \column{0.6\textwidth}
    \includegraphics[width=\linewidth]{Wooding.png}
    \column{0.4\textwidth}
    \raggedleft
    \textcolor{blue}{When population size varies}
  \end{columns}
\end{frame}

\begin{frame}
\frametitle{Summary}
\begin{itemize}
\item Under neutrality and constant population size, $\theta/i$ is the
  expected number of sites at which the derived allele is present in
  $i$ copies.
\item This does not depend on sample size.
\item Some human loci conform to this model; many do not.
\item Departures imply something interesting: selection or changes in
  population size.
\end{itemize}
\end{frame}

\end{document}
