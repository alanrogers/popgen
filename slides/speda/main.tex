% -*-latex-*-
%\documentclass[handout]{beamer}
\documentclass[]{beamer}
\usepackage{etex}
\usepackage{graphicx}

\let\latexput\put
\usepackage{pictex}
\let\pictexput\put
\let\put\latexput

\setbeamercovered{transparent}
\setbeamertemplate{footline}[frame number]
\author{Alan R. Rogers}
\date{\today}
\title{Using Genetic Data to Build Intuition about Population History}
\begin{document}

\frame{\titlepage}

\begin{frame}
\frametitle{Notation}
{\centering
  \input{fignotation}\hspace{5mm}%-*-latex-*-
\let\put\pictexput
\mbox{\beginpicture
\headingtoplotskip=\baselineskip
\setcoordinatesystem units <0.35cm, 0.35cm> point at 16 1
\setplotarea x from 1 to 15, y from 1 to 15
\axis bottom invisible ticks length <0pt>
  withvalues {$X$} {$Y$} {$N$} {$C$} /  at 2 6 10 14 / /
\axis bottom invisible shiftedto y=-0.5 ticks length <0pt>
  withvalues {1} {1} {0} {0} /  at 2 6 10 14 / /
\setsolid
\setplotsymbol ({\normalsize .})
\plot 1 1 7 13 7 15 /
\plot 3 1 4 3 5 1 / 
\plot 7 1 5 5 6 7 9 1 /
\plot 11 1 7 9 8 11 13 1 /
\plot 15 1 9 13 9 15 /
\setplotsymbol ({\textcolor{red}{\footnotesize .}})
\plot 2 1 5 7 /
\plot 6 1 4.25 4.5 4.5 6 /
\setplotsymbol ({\textcolor{blue}{\footnotesize .}})
\plot 5 7 8 13 8 15 /
\plot 10 1 6.5 8 6.5 9 7 11 /
\plot 14 1 8 13 /
\put {$\bullet$} at 5 7
\endpicture}
\let\put\latexput
\\}

\bigskip

$xy \succ yn$ will mean that $xy$ is more common than $yn$.
$xy \sim yn$ will mean that the two are about equal in frequency.
\end{frame}

\begin{frame}
\frametitle{Calling ancestral alleles}
{\centering\input{figancder}\\}

\bigskip

When one allele is present in the outgroup, but two are present in the
ingroup, the outgroup allele is likely to be ancestral.
\end{frame}

\begin{frame}
  \frametitle{Observed site pattern frequencies}
  \begin{columns}
    \column{0.5\textwidth}
  \includegraphics[width=\linewidth]{xynd-frq.pdf}
  \column{0.5\textwidth}
  \raggedleft

  $X$ is Africa; $Y$, Europe; $N$, Neanderthal, $D$, Denisovan.\\

  \bigskip
  
  ``Dots'' in circles are 95\% confidence intervals.\\
  
  \bigskip

  $x \succ y$

  \bigskip

  \textit{d} $\succ$ all other singletons.\\

  \bigskip

  \textit{xy}, \textit{nd} $\succ$ other doubletons.\\

  \bigskip

  $xy \succ nd$\\
  \end{columns}

  \bigskip

\flushright  
  \textit{yn} $\succ$ other rare doubletons;
  \textit{xyn} $\succ$ other tripletons

\end{frame}

\begin{frame}
  \frametitle{Why are $xy$, $nd$ $\succ$ other doubletons?}
  \begin{center}
    \input{fig2fork}
  \end{center}
$(X,Y)$ and $(N,D)$ are pairs of closely related populations. Close
relatives share ancestors, and mutations in these ancestors generate
\textit{xy} and \textit{nd}.           
\end{frame}

\begin{frame}
\frametitle{Why is $xy \succ nd$?}
\begin{center}
  \input{fig2fork}
\end{center}
Either $N$ and $D$ split earlier than $X$ and $Y$, or \textit{ND} was
larger than \textit{XY}.
\end{frame}

\begin{frame}
  \frametitle{Where do counterintuitive site patterns come from?}
  \begin{columns}
    \column{0.3\textwidth}
    %-*-latex-*-
\let\put\pictexput
\mbox{\beginpicture
\headingtoplotskip=\baselineskip
\setcoordinatesystem units <0.35cm, 0.35cm> point at 16 1
\setplotarea x from 1 to 15, y from 1 to 15
\axis bottom invisible ticks length <0pt>
  withvalues {$X$} {$Y$} {$N$} {$C$} /  at 2 6 10 14 / /
\axis bottom invisible shiftedto y=-0.5 ticks length <0pt>
  withvalues {1} {1} {0} {0} /  at 2 6 10 14 / /
\setsolid
\setplotsymbol ({\normalsize .})
\plot 1 1 7 13 7 15 /
\plot 3 1 4 3 5 1 / 
\plot 7 1 5 5 6 7 9 1 /
\plot 11 1 7 9 8 11 13 1 /
\plot 15 1 9 13 9 15 /
\setplotsymbol ({\textcolor{red}{\footnotesize .}})
\plot 2 1 5 7 /
\plot 6 1 4.25 4.5 4.5 6 /
\setplotsymbol ({\textcolor{blue}{\footnotesize .}})
\plot 5 7 8 13 8 15 /
\plot 10 1 6.5 8 6.5 9 7 11 /
\plot 14 1 8 13 /
\put {$\bullet$} at 5 7
\endpicture}
\let\put\latexput

    \column{0.7\textwidth}
    \raggedleft
    It's easy to understand where $xy$ comes from.

    \bigskip

    But why do $xn$ and $yn$ exist?

    \bigskip

    One answer is \emph{incomplete lineage sorting}\ldots
    \end{columns}
\end{frame}  

\begin{frame}
\frametitle{Incomplete lineage sorting}
\centering
%-*-latex-*-
\let\put\pictexput
\mbox{\beginpicture
\headingtoplotskip=\baselineskip
\setcoordinatesystem units <0.35cm, 0.35cm> point at 16 1
\setplotarea x from 1 to 15, y from 1 to 15
\axis bottom invisible ticks length <0pt>
  withvalues {$X$} {$Y$} {$N$} {$C$} /  at 2 6 10 14 / /
\axis bottom invisible shiftedto y=-0.5 ticks length <0pt>
  withvalues {1} {0} {1} {0} /  at 2 6 10 14 / /
\setdots
\axis left invisible ticks andacross unlabeled at 7 / /
%\axis top ticks andacross numbered from 1 to 15 by 1 /
\plotheading{Pattern $xn$}
\setsolid
\setplotsymbol ({\normalsize .})
\plot 1 1 7 13 7 15 /
\plot 3 1 4 3 5 1 / 
\plot 7 1 5 5 6 7 9 1 /
\plot 11 1 7 9 8 11 13 1 /
\plot 15 1 9 13 9 15 /
% gene tree...
\setplotsymbol ({\textcolor{red}{\footnotesize .}})
\plot 2 1 6.5 10 /
\plot 10 1 6 9 /
\setplotsymbol ({\textcolor{blue}{\footnotesize .}})
\plot 6.5 10 8 13 8 15 /
\plot 6 1 4.25 4.5 7 10 7 11 /
\plot 14 1 8 13 /
\put {$\bullet$} at 6.5 10
%%%%%%%%%%%%% Right plot %%%%%%%%%%%%%%%%%%%%%%%%%%%%%%%%%%%%
\setcoordinatesystem units <0.35cm, 0.35cm> point at 0 1
\setplotarea x from 1 to 15, y from 1 to 15
\axis bottom invisible ticks length <0pt>
  withvalues {$X$} {$Y$} {$N$} {$C$} /  at 2 6 10 14 / /
\axis bottom invisible shiftedto y=-0.5 ticks length <0pt>
  withvalues {0} {1} {1} {0} /  at 2 6 10 14 / /
\setdots
\axis right invisible ticks andacross unlabeled at 7 / /
\setsolid
\plotheading{Pattern $yn$}
\setplotsymbol ({\normalsize .})
\plot 1 1 7 13 7 15 /
\plot 3 1 4 3 5 1 / 
\plot 7 1 5 5 6 7 9 1 /
\plot 11 1 7 9 8 11 13 1 /
\plot 15 1 9 13 9 15 /
% gene tree...
\setplotsymbol ({\textcolor{blue}{\footnotesize .}})
\plot 2 1 8 13 8 15 /
\setplotsymbol ({\textcolor{red}{\footnotesize .}})
\plot 6 1 4.25 4.5 6.5 9 /
\plot 10 1 6.5 8 6.5 9  6.8 9.9 /
\setplotsymbol ({\textcolor{blue}{\footnotesize .}})
\plot 6.8 9.9 7.5 12 /
\plot 14 1 8 13 /
\put {$\bullet$} at 6.8 9.9
\endpicture}
\let\put\latexput
\\
\end{frame}

\begin{frame}
  \frametitle{Two puzzles}
  \begin{columns}
    \column{0.5\textwidth}
  \includegraphics[width=\linewidth]{xynd-frq.pdf}
  \column{0.5\textwidth}
  \raggedleft
  Incomplete lineage sorting predicts that $yn \sim xn$, and $yd \sim xd$.
  
  \bigskip

  It's true that $yd \sim xd$.

  \bigskip

  However, $yn \succ xn$: why?

  \bigskip

  Also, $x \succ y$: why?
  \end{columns}
\end{frame}

\begin{frame}
\frametitle{$N\rightarrow Y$ admixture inflates \textit{yn} and $x$}
\centering
\input{figmix}\\
\end{frame}

\begin{frame}
  \frametitle{Evidence for Neanderthal admixture into Eurasians}
  \begin{columns}
    \column{0.5\textwidth}
  \includegraphics[width=\linewidth]{xynd-frq.pdf}
  \column{0.5\textwidth}
  \raggedleft
   $yn \succ xn$ and $x \succ y$

  \bigskip

  This suggest admixture from Neanderthals into Eurasians.
  \end{columns}
\end{frame}

\begin{frame}
  \frametitle{Another puzzle}
  \begin{columns}
    \column{0.5\textwidth}
  \includegraphics[width=\linewidth]{xynd-frq.pdf}
  \column{0.5\textwidth}
  \raggedleft
  Note that $d$ $\succ$ other singletons and that $xyn$ $\succ$ other
  tripletons.

  \bigskip

  Why should this be?
  \end{columns}
\end{frame}

\begin{frame}
\frametitle{$S\rightarrow D$ gene flow inflates \textit{d} and
  \textit{xyn}}
\centering
\input{fighyparch}\\
\end{frame}

\begin{frame}
  \frametitle{Observed site pattern frequencies}
  \begin{columns}
    \column{0.5\textwidth}
  \includegraphics[width=\linewidth]{xynd-frq.pdf}
  \column{0.5\textwidth}
  \raggedleft
  High frequencies of $d$ and $xyn$ suggest $S \rightarrow D$ admixture.\\
  \end{columns}
\end{frame}

\begin{frame}
  \frametitle{Conclusions}
  We have used no formal model; we have tested no hypotheses. All
  we've done is to look at the data. Yet this informal analysis has
  been productive. It suggests
  \begin{enumerate}
  \item modern Europeans and Africans are closely related,
  \item so were Neanderthals and Denisovans,
    \item the separation between Europeans and
Africans was more recent than that between Neanderthals and
Denisovans,
\item Neanderthals contributed DNA to Europeans, and
\item superarchaics contributed DNA to Denisovans.
  \end{enumerate}

\end{frame}
\end{document}

