% -*-latex-*-
\begin{center}
\mbox{%
\beginpicture
\headingtoplotskip=0.5\baselineskip
\setcoordinatesystem units <.1in, .1in> point at 0 0
\setplotarea x from 0 to 12, y from -0.4 to 12
\axis left label {\small\lines{Quantiles\cr of $\hat\tau$}} shiftedto x=-1
   ticks numbered from 0 to 12 by 3 /
\axis bottom shiftedto y=-1.4
  label {$\tau$} ticks numbered from 0 to 12 by 3 /
\multiput {$\bullet$} at 0 0 2 2 4 4 6 6 8 8 10 10 12 12 /
\setdots
%x=tau y=quantile 0.025 for tau hat
\plot 0 -0.470163 2 1.264931 4 2.621103 6 4.041647 8 5.189375 10
6.512056 12 7.781337 /
\setdashes
%x=tau y=quantile 0.25 for tau hat
\plot 0 0.093195 2 1.899364 4 3.533029 6 5.129342 8 6.560104 10
7.886912 12 9.24927 /
\setsolid
%x=tau y=quantile 0.5 for tau hat
\plot 0 0.583741 2 2.356760 4 4.016899 6 5.727125 8 7.253701 10
8.664867 12 10.022979 /
\setdashes
%x=tau y=quantile 0.75 for tau hat
\plot 0 0.941562 2 2.807729 4 4.520144 6 6.302821 8 7.922965 10
9.332856 12 10.799049 /
\setdots
%x=tau y=quantile 0.975 for tau hat
\plot 0 1.625900 2 3.622672 4 5.643163 6 7.436202 8 9.148493 10
10.937300 12 12.206018 /
\endpicture%
}
\end{center}
%
\footnotesize At least 1000 data sets were simulated at each of
several values of $\tau$, and each was used to estimate the model's
parameters.  The bold dots indicate points at which $\hat\tau=\tau$.
The solid line is the median, the dashed lines enclose the central
50\% of the distribution, and the dotted lines the central 95\%.  Each
simulated data set was generated using the coalescent algorithm with
$\theta_0 = 1$, $\theta_1 = 100$, and $N=147$.








