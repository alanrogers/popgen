\begin{frame}
\frametitle{Population tree}
\centering%-*-latex-*-
\let\put\pictexput
\mbox{\beginpicture
\headingtoplotskip=\baselineskip
\setcoordinatesystem units <4mm, 4mm> point at 16 1
\setplotarea x from 1 to 15, y from 1 to 15
\small
\axis bottom invisible ticks length <0pt>
  withvalues {$X$} {$Y$} {$N$} {$D$} /  at 2 7 10 14 / /
%\axis bottom invisible shiftedto y=-0.3 ticks length <0pt>
%  withvalues {$x$} {$y$} {$n$} {$d$} /  at 2 7 10 14 / /
\axis left invisible ticks length <0pt>
  withvalues {$T_{XY}$} {$T_{XYND}$} /
   at 4 11.249724127083585 / /
\axis right invisible ticks length <0pt>
  withvalues {$T_{ND}$}  / at 6 / /
\put {\small Pops:} [r] <0pt, 3pt> at 1 0
%\put {\small Samples:} [rt] <0pt, -4pt> at 1 -0.3
\setdots
\putrule from 1 4 to 4.5 4   % $T_{XY}$
\putrule from 12 6 to 15 6 % $T_{ND}$
%\putrule from 12.75 2 to 15.75 2 % $T_D$
\putrule from 1 11.249724127083585
   to 8.12486206354179 11.249724127083585 % $T_{XYND}$
\setsolid
\setplotsymbol ({\normalsize .})
\plot 1 1 7 13 7 15 /   %X
\plot 3 1 4.5 4 6 1 /   %XY
\plot 8 1 5.5 6 8.12486206354179 11.249724127083585 /    %Y
\plot 9.17572 1 10.1757 6 11.036 7.7206 / %N
\plot 11 1 12 6 /       %N
\plot 13 1 12 6 /            %D
\plot 14.684 1 13.8151 6 12.3389 8.9523 9 13 9 15 / %Dr
\plot 8.12486206354179 11.249724127083585 11.036 7.7206 / % ND l
\arrow <10pt> [.2,.67] from 9.57571931984138 3 to 7 3
%\arrow <0pt> [.2,.67] from 12.75 2 to 11.25 2
%\arrow <10pt> [.2,.67] from 9.32572 2 to 7.5 2
\put {\small $N_{XY}$} at 6 9
\put {\small $N_{ND}$} at 11.2 9
\put {\small $N_N$}   at 10.7 4.3
\put {\small $N_{XYND}$} [b] <0pt,2pt> at 8 15
\put {\small $\alpha$} [b]  <0pt,5pt> at 8.5 3
%\put {\small $\epsilon$} [t]  <0pt,-5pt> at 12 2
\endpicture}
\let\put\latexput
\\[1ex]
$X$, Africa; $Y$, Europe; $N$, Neanderthal; $D$, Denisovan\\
\end{frame}

\begin{frame}
\frametitle{Gene genealogies and nucleotide site patterns}
\begin{columns}
\column{0.6\textwidth}
\centering\input{../pophist/figgptree}\\[1ex]
\column{0.4\textwidth}
\raggedleft
Gene genealogy within population tree.

\bigskip

Mutation on red branch $\rightarrow$ \emph{site pattern} \textcolor{red}{$yn$}.

\bigskip

Blue branch $\rightarrow$ \textcolor{blue}{$ynd$}.

\bigskip

0, ancestral; 1, derived.
\end{columns}
\end{frame}

\begin{frame}
\frametitle{Observed Site Pattern Frequencies}
\begin{columns}
\column{0.6\textwidth}
\includegraphics[width=\linewidth]{../pophist/patfrq.pdf}\\

\medskip

\centering
(fraction of nucleotide sites exhibiting each pattern)\\
\column{0.4\textwidth}
\raggedleft
$X$, Africa; $Y$, Europe; $N$, Neanderthal; $D$, Denisovan.

\bigskip

$xy$ is common because $X$ and $Y$ share ancestry.

\bigskip

Ditto $nd$.

\bigskip

Goal: infer history from these data
\end{columns}
\end{frame}

\begin{frame}{The mystery of the 4000-year-old Denisovan}
     We argued in 2017 for an early separation of Neanderthals and
     Denisivans and a bottleneck among their ancestors. Mafessoni and
     Pr{\"u}fer showed that results are different if one includes
     singleton site patterns. However, the with-singleton anaysis also
     implied an implausible 4~kya date for the Denisovan fossil. How
     can this be explained?
\end{frame}

\begin{frame}{Clues: an excess of site patterns $d$ and $xyn$}
     {\centering\includegraphics[width=\linewidth]{../pophist/singleton.pdf}\\
       \includegraphics[width=\linewidth]{../pophist/triplet.pdf}\\} Suggests
     hyperarchaic admixture into Denisovans (Pr\"{u}fer et al.\ 2014)
     or early modern admixture into Neanderthals (Kuhlwilm et al.\
     2016).
\end{frame}


\begin{frame}
\frametitle{Two ways to inflate $d$ and $xyn$.}
{\centering\input{../pophist/fighyparch}~%auto-ignore
%-*-latex-*-
\mbox{\beginpicture
\headingtoplotskip=0.2\baselineskip
\setcoordinatesystem units <2mm, 1.8mm>
\setplotarea x from -2 to 17.5, y from 0 to 20
\color{black}
\plotheading{$XY{\rightarrow}N$}
\axis bottom invisible ticks length <0pt>
  withvalues {$X$} {$Y$} {$N$} {$D$} {$H$} /  at 1 6 9 13 17 / /
\color{red}
\axis bottom invisible shiftedto y=-2.5 ticks length <0pt>
  withvalues {\llap{$d$:}} {0} {0} {0} {1} /
  at 0 1 6 9 13 / /
\color{cyan}
\axis bottom invisible shiftedto y=-5 ticks length <0pt>
  withvalues {\llap{$xyn$:}} {1} {1} {1} {0} /
  at 0 1 6 9 13 / /
\color{black} % I issue this command twice, because otherwise,
\color{black} % the rest of the document is cyan.
%\axis left ticks andacross numbered from 0 to 20 by 2 /
%\axis top ticks andacross numbered from 0 to 16 by 2 /
\setplotsymbol ({\normalsize .})
\plot 0 0 6 12 6 20 /   %X
\plot 2 0 3.5 3 5 0 /   %XY
\plot 7 0 4.5 5 7.12486206354179 10.249724127083585 /    %Y
\plot 8.17572 0 9.1757 5 10.036 6.7206 / %N
\plot 10 0 11 5 /       %N
\plot 12 0 11 5 /            %Dl
\plot 8 16 14.21226360816738 8.468963009740001 15.684 0 / %Hl
\plot 8 20 8 18.8112 15.8936 9.24189 17.4997 0 / %Hr
\plot 13.684 0 12.8151 5 11.3389 7.9523 8 12 8 16 / %Dr
\plot 7.12486206354179 10.249724127083585 10.036 6.7206 / % ND l
%\arrow <10pt> [.2,.67] from 8.57571931984138 2 to 6 2
%\arrow <0pt> [.2,.67] from 11.75 1 to 10.25 1
%\arrow <10pt> [.2,.67] from 8.32572 1 to 6.5 1
%\put {\small $m_N$} [b]  <0pt,5pt> at 7.5 2
%\put {\small $m_D$} [t]  <0pt,-5pt> at 11 1
\put {\small $m_{XY}$} [b]  <0pt,5pt> at 7.3 6
\arrow <10pt> [.2,.67] from 5 6 to 9.6 6
%\put {\small $m_H$} [b]  <0pt,5pt> at 13.9889 4
%\arrow <10pt> [.2,.67] from 14.9889 4 to 12.9889 4
\setplotsymbol ({\footnotesize .})
\setdashes
\plot 1 0 5 8 /         % x, bottom of xy
\setplotsymbol ({\textcolor{cyan}{\footnotesize .}})
\setsolid
\plot 5 8 7 12 7 13 /  % top of xy
\setplotsymbol ({\footnotesize .})
\setdashes
\plot 7 13 7 20 /             % top of xynd
\plot 6 0 4 4 4 6 / %y
\setplotsymbol ({\textcolor{red}{\footnotesize .}})
\setsolid
\plot
   12.649562539059012 0.5
   11.946500312472102 4
   10.581012916966557 7.465564584832791
   7.31003 11.4308
   7 13
/ % d and nd
\setplotsymbol ({\footnotesize .})
\setdashes
\plot
   9.187899999999999 0.5
   10.287899999999999 6
   5 6
   5 8
   / % n
\setsolid
\endpicture}
\\}
\textcolor{red}{Red} branch mutations generate site pattern $d$;
\textcolor{cyan}{Blue} generates $xyn$; 0, ancestral allele; 1,
derived; $X$, Africa; $Y$, Eurasia; $N$, Neanderthal; $D$,
Denisovan; $H$, Hyperarchaic; $XY$, population ancestral to $X$
and $Y$.
\end{frame}

\begin{frame}
\begin{raggedleft}
\includegraphics[width=0.8\linewidth]{../pophist/french-ma-mdot.pdf}\\
\includegraphics[width=0.83\linewidth]{../pophist/french-ma-tdot.pdf}\\
\includegraphics[width=0.83\linewidth]{../pophist/french-ma-ndot.pdf}\\
\end{raggedleft}
\end{frame}

\begin{frame}
\frametitle{Implications of previous slide}
   \begin{tabular}{lp{0.8\linewidth}}
   $m_H$, $T_{XYNDH}$ & \raggedright Substantial
     $H{\rightarrow}D$ admixture and a hyperarchaic separation time of
     $\sim$1.6~mya.\tabularnewline
   $m_{XY}$ & \raggedright $XY{\rightarrow}N$ admixture.\tabularnewline
   $T_{ND}$ & \raggedright Early separation of Neanderthals and Denisovans:
   $\sim$747~kya.\tabularnewline
   $2N_{ND}$ & \raggedright Narrow bottleneck in Neanderthal-Denisovan
   ancestors.\tabularnewline
   $2N_{AV}$, $2N_N$ & \raggedright Effective Neanderthal population was large
     early ($2N_{AV}\approx 21$~k) but small later ($2N_N\approx
     5$~k).\tabularnewline
   $T_V$, $T_A$, $T_D$ & \raggedright Vindija, Altai, and Denisovan fossil
     ages are $\sim$70~ky, $\sim$150~ky, and $\sim$100~ky.\tabularnewline
   \end{tabular}
\end{frame}

\begin{frame}
\frametitle{Summary (part 1)}
\begin{itemize}
\item History of population size affects depth of gene trees, genetic
  variation, and length of MRCA segments.
\item We can use these facts to infer the history of population size.
\item Human population has varied in size over past 3~my.
\item Bottleneck during last ice age, ending 20~kya.
\item African bottleneck was shorter and shallower.
\item Eurasian/African split 150~kya.
\item European/Asian split 20~kya.
\end{itemize}
\end{frame}

\begin{frame}
\frametitle{Summary (part 2)}

\textbf{Current consensus}
\begin{itemize}
\item Neanderthals and Denisovans separated $\sim$450 kya,
\item then declined to tiny population sizes ($<$1000 individuals).
\end{itemize}

\bigskip

\textbf{Our view}
\begin{itemize}
\item Archaics separated from moderns 750~kya,
\item then endured a bottleneck of $\sim$5~ky.
\item Neanderthals \& Denisovan separated shortly thereafter.
\item Neanderthal population was large early \& small later.
\end{itemize}
\end{frame}
