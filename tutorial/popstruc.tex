\documentclass[twocolumn]{article}
\usepackage{pictex,fullpage}
\usepackage[numbers,sort&compress]{natbib}
%\setcounter{secnumdepth}{0}
\begin{document}
\title{Geographic Population Structure}
\author{Alan R. Rogers}
\maketitle

In Buri's \citep{Buri:E-10-367} drift experiment, heterozygosity ($H$)
declined.  At the same time, the variance ($V$) among populations
increased.  We already have a model describing the first of these
phenomena.  Here, we consider the second---the variance among
populations---assuming throughout that genetic drift is the only
evolutionary force at work.

\begin{figure*}
{\centering\let\put\pictexput
\mbox{\beginpicture
%%%%%%%%%%%%%% Top plot
\setcoordinatesystem units <0.07in, 1.92in> point at 19 0
\setplotarea x from 0 to 19, y from 0.0 to 0.52
\axis left shiftedto x=-0.53 label {$H$}
  ticks numbered from 0.00 to 0.50 by  0.25 /
\axis bottom shiftedto y=-0.02 label {Generation} 
  ticks numbered from 0 to 15 by 5 /
\put {Heterozygosity} [lb] at 0 0.01
% Buri 1956. Series I. Observed heterozygosity H
\multiput {$\circ$} at
%Gen. ObsHet
% 0 1.000
 1 0.514
 2 0.464
 3 0.504
 4 0.456
 5 0.448
 6 0.428
 7 0.403
 8 0.402
 9 0.358
10 0.348
11 0.325
12 0.305
13 0.263
14 0.255
15 0.216
16 0.202
17 0.210
18 0.197
19 0.183
/
%%%%%%%%%%%%%% Right plot
\setcoordinatesystem units <0.075in, 5.88in> point at -2 0
\setplotarea x from 0 to 19, y from 0 to 0.17
\axis right shiftedto x=19.53 label {$V$}
  ticks numbered from 0.0 to 0.15 by  0.05 /
\axis bottom shiftedto y=-0.0068  label {Generation} 
  ticks numbered from 0 to 15 by 5 /
%\put {\makebox[0pt]{Drift Experiment, Series I, Buri (1956)}} [c] at -1 0.22
\put {Variance} [rb] at 19 0.01
\multiput {$\circ$} at
% 0 0
 1 0.006
 2 0.026
 3 0.031
 4 0.042
 5 0.050
 6 0.055
 7 0.062
 8 0.072
 9 0.083
10 0.090
11 0.105
12 0.112
13 0.123
14 0.136
15 0.140
16 0.155
17 0.160
18 0.165
19 0.170
/
\endpicture}
\let\put\latexput
\\}
\caption{In Buri's \citep{Buri:E-10-367} drift experiment,
  heterozygosity ($H$) declined.  At the same time, the variance ($V$)
  among populations increased.  Data are from Buri's series~I and are
  tabulated in Table~\ref{tab.buri}.}
  \label{fig.buri}
\end{figure*}

First a few terms.  In the initial generation, the allele frequency
was $p_0$.  We will treat this as a constant, not a random variable.
In generation $t$, the corresponding quantity $p_t$ is a random
variable, the randomness having been introduced by the process of
genetic drift.  For convenience, we set $q_t = 1-p_t$.  The
heterozygosity in generation $t$ (also a random variable) is $H_t =
2p_t q_t$.  We are interested in the evolutionary change that has
happened since generation~0.

\section{Genetic drift does not change the expected allele 
frequency} 

Let us begin with the expected value of $p_1$, the allele frequency in
the generation~1.  According to the urn model, the number of copies of
allele $A$ in each subpopulation is a Binomial random variable with
mean $Np_0$.  The expected allele frequency is thus $Np_0/N = p_0$.
This demonstrates a remarkable fact: the expected allele frequency is
unchanged by genetic drift.  This is true not only of the first
generation, but of each succeeding generation.  No matter how many
generations are involved, the allele frequency of each subpopulation
is a random variable whose expected value is $p_0$.  In symbols,
\begin{equation}
E[p_t] = p_0.
\label{eq.Ept}
\end{equation}

\section{The Wahlund Principle and $F_{ST}$}
In contrast to $p_t$, the variance $V_t$ has an expected value that
does change with time.  To model the variance, we begin with its
definition:
\[
V_t = E[p_t^2] - p_0^2
\]
This is just the standard definition of the variance, modified
slightly in light of Eqn.~\ref{eq.Ept}.  Note that, in view of
this definition,
\begin{equation}
E[p_t^2] = V_t + p_0^2
\label{eq.Ep2}
\end{equation}
We'll need this fact in a minute.

Let us turn now to the heterozygosity.  Its expected value in
generation $t$ is
\begin{eqnarray}
E[2p_t q_t] &=& 2E[p_t] - 2E[p_t^2]\nonumber\\
 &=& 2p_0 - 2(V_t + p_0^2)\qquad 
\hbox{using Eqns.~\ref{eq.Ept}--\ref{eq.Ep2}}\nonumber\\
 &=& 2p_0q_0 - 2V_t
\label{eq.wahlund}
\end{eqnarray}
This is another important fact: average heterozygosity in
generation~$t$ will be smaller than that in generation~0.  The amount
of the reduction is \emph{exactly twice} the variance of group allele
frequencies about their expected value $p_0$.  This is known as
Wahlund's principle \citep{Wahlund:75}.  It shows that there is a close
and necessary connection between the decline of heterozygosity (shown
on the left in Fig.~\ref{fig.buri}) and the increase in variance
(shown on the right).  In effect, genetic drift converts
heterozygosity into variance among groups.

It is often useful to express the absolute reduction in heterozygosity
$(2V_t)$ as a proportion of the original heterozygosity $(2p_0q_0)$.
This proportional reduction is
\begin{equation}
F_{ST} = \frac{2V_t}{2p_0q_0} = \frac{V_t}{p_0q_0} 
\label{eq.fst}
\end{equation}
The notation $F_{ST}$ was introduced by Sewall Wright
\citep{Wright:AE-15-323} and is now conventional within population
genetics.  Gillespie defines $F_{ST}$ using different notation (his
Eqn.~5.3), which was introduced by \citet{Nei:PNA-70-3321}.  Although
the two definitions look different, they are interchangeable.  In view
of Eqn.~\ref{eq.fst}, we can re-express Eqn.~\ref{eq.wahlund} as
\begin{equation}
E[2p_t q_t] = 2p_0q_0(1-F_{ST})
\label{eq.hfst}
\end{equation}
Now $E[2p_t q_t]$ is the expected heterozygosity in generation $t$,
and $2p_0q_0$ is that in generation 0. Thus, Eqn.~\ref{eq.hfst} says
that $F_{ST}$ is the proportional reduction in heterozygosity caused
by genetic drift. There are corresponding increases in expected
homozygosity. The expected frequencies of the three genotypes at a
biallelic locus are:
\[
\begin{array}{lc}
\hbox{Genotype} & \hbox{Frequency}\\
\hline
AA & E[p_t^2] = p_0^2 + p_0q_0F_{ST}\\
Aa & E[2p_t q_t] = 2p_0q_0(1-F_{ST})\\
aa & E[q_t^2] = q_0^2 + p_0q_0F_{ST}\\
\end{array}
\]
These formulas are identical to those for pedigree inbreeding, because
genetic drift is a form of inbreeding.

\section{Model of completely isolated subpopulations}
\label{sec.nomig}

In Eqn.~\ref{eq.hfst}, we express the expected heterozygosity
$(2p_tq_t)$ in generation~$t$ as a function of that in generation~0.
Early in the semester, we derived a similar result:
\begin{equation}
E[2p_t q_t] = 2p_0q_0\bigl(1-1/2N\bigr)^t
\label{eq.hetdecay}
\end{equation}
To refresh your memory, see Gillespie's Eqn.~2.3.  These equations
look very different, yet both are correct.  They provide a simple way
to derive the rule by which $F_{ST}$ changes with time.  Set
Eqns.~\ref{eq.hfst} and~\ref{eq.hetdecay} equal to each other, and
solve for $F_{ST}$.  You will discover that
\begin{eqnarray}
F_{ST} &=& 1 - \bigl(1-1/2N\bigr)^t\nonumber\\
  &\approx& 1 - e^{-t/2N}
\label{eq.fst.noneq}
\end{eqnarray}
The last line above uses the approximation that $e^x \approx 1+x$ if
$x$ is near zero.  Eqn.~\ref{eq.fst.noneq} applies when the
populations are totally separated.  It shows that $F_{ST}$ increases
according to a very simple rule, increasing gradually toward its
maximal value, 1.0.

\section{Migration in addition to drift}

If there is migration between populations, we cannot use any of the
results in section~\ref{sec.nomig}.  This case demands a different
theory, which is discussed by Gillespie.  Under the ``island-model''
of population structure, $F_{ST}$ converges toward an equilibrium at
which
\begin{equation}
F_{ST} = \frac{1}{4Nm+1}
\label{eq.fst.eq}
\end{equation}
as shown on Gillespie's p.~136.

We have two equations for $F_{ST}$.  Eqn.~\ref{eq.fst.eq} refers to
equilibrium under the island model, and Eqn.~\ref{eq.fst.noneq} to the
case of totally isolated populations.  In the exercises below, we will
consider human data under these two extreme cases.

\section*{Exercises}
\begin{enumerate}
\item
Plot the $V_t$ data in Table~\ref{tab.buri}, to make a graph like that
on the right side of Fig.~\ref{fig.buri}.

\item
Combine Eqns.~\ref{eq.fst} and~\ref{eq.fst.noneq} to obtain a
formula for the variance, $V_t$.  Plot this formula as a line in your
graph, assuming as Buri did that $2N = 18$.  How well do Buri's
variance data fit the model?

\begin{table}[htb]
  \caption{Data from series I of Buri's \citep{Buri:E-10-367} drift
    experiment, as plotted in Fig.~\ref{fig.buri}. Key: $t$, generation;
    $H_t$, mean heterozygosity $(2pq)$ within subpopulations; $V_t$,
    variance among group allele frequencies.}
\label{tab.buri}
\centering
\begin{tabular}{rrr|rrr}
 $t$ & $H_t$ & $V_t$ & $t$ & $H_t$ & $V_t$ \\ \hline
 0 & 1.000 & 0.000 & 10 & 0.348 & 0.090\\
 1 & 0.514 & 0.006 & 11 & 0.325 & 0.105\\
 2 & 0.464 & 0.026 & 12 & 0.305 & 0.112\\
 3 & 0.504 & 0.031 & 13 & 0.263 & 0.123\\
 4 & 0.456 & 0.042 & 14 & 0.255 & 0.136\\
 5 & 0.448 & 0.050 & 15 & 0.216 & 0.140\\
 6 & 0.428 & 0.055 & 16 & 0.202 & 0.155\\
 7 & 0.403 & 0.062 & 17 & 0.210 & 0.160\\
 8 & 0.402 & 0.072 & 18 & 0.197 & 0.165\\
 9 & 0.358 & 0.083 & 19 & 0.183 & 0.170\\
\hline
\end{tabular}
\end{table}

\item In the continental human populations, $F_{ST} \approx 1/9$.  Use
  this value to estimate $Nm$ (under the equilibrium model).

\item Now assume that the human continental populations have been
  totally isolated, and use the observed $F_{ST}$ to estimate $t/2N$. 
  Then convert this into an estimate of the time in years since the
  human populations separated.  (Assume that $N=10,000$ and that
  generations are 25 years long.) 
\end{enumerate}

\bibliographystyle{plainnat}
\bibliography{defs,gs,arr,molrec}

\end{document}
