\documentclass[11pt]{article}
\usepackage[letterpaper,margin=1in]{geometry}
\usepackage{mathtools}
\usepackage{hyperref}
\usepackage[backend=biber,
  natbib=true,
  style=ext-numeric-comp,
  sorting=nyt]{biblatex}
\DeclareFieldFormat % Titles in sentence case
  [article,inbook,incollection,inproceedings,patent,thesis,unpublished]
  {titlecase:title}{\MakeSentenceCase*{#1}}
\renewbibmacro{in:}{ % Don't use "In:" before journal names.  
  \ifentrytype{article}{}{\printtext{\bibstring{in}\intitlepunct}}}  
\addbibresource{defs.bib}
\addbibresource{arr.bib}
\addbibresource{popecol.bib}
\addbibresource{gs.bib}
\addbibresource{arrpubs.bib}
\addbibresource{sexdimorph.bib}
\begin{document}
\title{Response to Natural Selection on a Quantitative Character}
\author{Alan R. Rogers}
\maketitle

\section{Introduction}
\label{sec.intro}

\citet{Lande:E-37-1210} rewrote the classical ``breeder's equation''
\citep[Eqn.~11.2]{Falconer:IQG-81} of quantitative genetics in a form
that is more useful within evolutionary biology. Rather than assuming
that selection acts by truncation, as the classical theory did, they
allow for a continuum of fitness values. Their treatment was
multivariate; this note provides a univariate summary.

\section{Truncation selection}
\label{sec.truncation}

The classical theory deals with the case in which a breeder wishes to
increase the value of some character, $z$. Individuals are selected
for breeding only if their character value exceeds some
threshold. Thus, the mean character value of parents exceeds that of
the general population, and the difference, $S$, is called the
\emph{selection differential}.

If the character is heritable, then its mean value among offspring
will also exceed that of the parental generation before
selection. This excess, written either as $R$ or as $\Delta z$, is
called the \emph{response to selection} and obeys what is called the
``breeder's equation'' \citep[Eqn.~11.2]{Falconer:IQG-81}:
\begin{equation}
  \Delta z = h^2 S
\label{eq.breeder}  
\end{equation}
where $h^2$ is called the heritability and equals the ratio,
$V_A/V_P$, of additive genetic variance to phenotypic variance.
The response tends to be smaller than the selection differential,
because characters are seldom perfectly heritable.

This equation is useful not only to breeders of livestock, but also to
ecologists. The two groups, however, use it in different ways.  The
breeder asks such questions as ``how large is $S$ likely to be if I
select the upper 20\% of my population for breeding?'' This question
would be of little interest to an ecologist, because natural selection
does not work by truncation---fitness is more likely to vary
continuously as a function of character value. To deal with natural
selection, we need a different way to calculate the selection
differential.

\section{Response to univariate selection}
\label{sec.gradient}
Let $f(z)$ represent the probability density function of a
quantitative character. In other words, $f(z) dz$ is approximately
the probability that a random individual has a value in the interval
between $z$ and $z+dz$, where $dz$ is some small number. The
population mean is
\[
\bar z = \int z f(z) dz
\]
where the integral runs over all possible character values, and I am
using an overbar to represent the expected value.

An individual with character value $z$ survives to reproduce with
probability $W(z)$, which is called the \emph{absolute fitness} of
that individual. Under truncation selection, $W(z) = 1$ for the
individuals who are selected and equals 0 for the others. In a natural
population, however, $W$ may take a wide range of values. The mean
fitness in the population is
\begin{equation}
  \bar W = \int W(z) f(z) dz
  \label{eq.barW}
\end{equation}
The density of $z$ among selected parents must be proportional to
$W(z) f(z)$, because $f(z)$ is the density of individuals with
character value $z$, and $W(z)$ is the fraction of those that survive
to become parents. However, $W(z)f(z)$ is not a proper density,
because its integral isn't 1; it's $\bar W$ (Eqn.~\ref{eq.barW}).  The
density among selected parents is $w(z) f(z)$, where $w(z) = W(z)/\bar
W$ is the \emph{relative fitness} of individuals with character value
$z$. Note that mean relative fitness equals unity:
\begin{equation}
  \bar w =   \int w(z) f(z) dz = \frac{\int W(z) f(z) dz}{\bar W} = 1
\label{eq.barw}  
\end{equation}  

The mean character value, $\bar z^*$, of selected parents is 
\[
\bar z^* = \int z w(z) f(z) dz
\]
which equals $E[zw]$, the expected product of $z$ and
$w(z)$. The selection differential is
\begin{eqnarray}
  S &=& \bar z^* - \bar z\nonumber\\
   &=& E[z w] - \bar z\nonumber\\
  &=& E[z w] - \bar z \bar w\nonumber\\
  &=& C_{wz}
\label{eq.price}  
\end{eqnarray}
where the third line uses the fact that $\bar w = 1$
(Eqn.~\ref{eq.barw}). Here, $C_{wz}$ is the covariance between
character value and relative fitness. This relationship between
selection and covariance was described by \citet{Price:N-227-520}.

Equation~\ref{eq.price} expresses the selection differential in
general terms. It is consistent with but is not limited to trucation
selection. We can use this formulation no matter how fitness varies
among individuals. Let us now plug it into the breeder's
equation. Eqn.~\ref{eq.breeder} becomes
\begin{eqnarray}
  \Delta z &=& h^2 C_{wz}\nonumber\\
  &=&  V_A C_{wz}/V_P\nonumber\\
  &=& V_A \beta_{wz}
  \label{eq.landearnold}
\end{eqnarray}
where $\beta_{wz} = C_{wz}/V_P$ is the regression of relative fitness
on character value. (To convince yourself of this, look up the formula
for linear regression, and remember that $V_P$ is the variance of
$z$.)  Equation~\ref{eq.landearnold} implies that we can study
selection in nature by doing a linear regression of fitness (perhaps
measured as the number of surviving offspring) against character
value.

If the relative fitness function, $w(z)$, is a straight line,
$\beta_{wz}$ should approximate the slope of that line. It is more
likely however that $w(z)$ has curvature, perhaps rising to a peak and
then declining. In such circumstances, \citet{Lande:E-37-1210} show
that $\beta_{wz}$ measures the average derivative of $w(z)$:
\[
\beta_{wz} = \int w'(z) f(z) dz
\]
where $w'$ is the derivative of $w$.

\section{Response to multivariate selection}

When there are $K$ quantitative characters, $z$ becomes a vector,
\[
z = \begin{pmatrix}
  z_1\\
  z_2\\
  \vdots\\
  z_K
  \end{pmatrix}
\]
and Eqn.~\ref{eq.landearnold} becomes
\begin{equation}
  \Delta z = G P^{-1} s = G \beta
  \label{eq.multivar}
\end{equation}
where $G$ is the additive genetic covariance matrix, $P$ is the
phenotypic covariance matrix, $s$ is a vector of selection
differentials, and $\beta = P^{-1} s$ is a vector of partial
regression coefficients of relative fitness on each of the $K$
characters \citep[p.~1212]{Lande:E-37-1210}. This vector is called the
\emph{selection gradient} and describes the direction in which
selection is pushing. The response to selection may however go in a
somewhat different direction, if that other direction has more
additive variance.

Lande used this framework to explore a variety of issues within
evolutionary ecology. One study dealt with allometry between brain
size and body size \citep{Lande:E-33-402}. Another considered male and
female versions of the same character as separate characters in order
to build an evolutionary theory of sexual dimorphism
\citep{Lande:E-34-292, Rogers:E-46-226}. In another, he
treated as separate characters the same character in different
environments. This yielded an evolutionary theory of phenotypic
plasticity \citep{Via:E-39-505}. In another, he treated as separate
characters the same character at different ages. This yielded an
evolutionary theory of life histories \citep{Lande:E-63-607}.  All in
all, Eqn.~\ref{eq.multivar} has been remarkably productive within
evolutionary ecology.

\printbibliography
\end{document}
