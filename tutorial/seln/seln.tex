% -*-latex-*-
\documentclass[11pt]{article}
\usepackage[letterpaper,margin=1in]{geometry}
\usepackage{verbatim,moreverb,exercise}
\usepackage{pictex}
\usepackage[numbers]{natbib}
\usepackage{graphicx}
\usepackage[raggedright]{sidecap}
\exercisehead{$\star$\,\footnotesize EXERCISE \normalsize\bf}{}
\newtheorem{result}{Result}
\begin{document}
\title{Not Quite Enough Selection (Or Maybe a Bit Too Much)}
\author{Alan R. Rogers}
\maketitle

\section{Introduction}
This note summarizes basic results from the theory of natural
selection. It develops the conventional algebraic model and uses this
to derive a series of qualitative conclusions, which are stated below
as ``Results.''

\section{Basic model}
Consider a locus with two alleles $A_1$ and $A_2$, whose relative
frequencies in some generation are $p$ and $q = 1 - p$,
respectively.  Let $w_{ij}$ denote the probability that an individual
of genotype $A_{i}A_{j}$ will survive to reproduce.  This probability
is called the fitness of genotype $A_{i}A_{j}$.  The genotypic
fitnesses and frequencies are shown in table~\ref{tab.fitness}.
\begin{table}
\caption{Genotypic fitnesses and frequencies\label{tab.fitness}}
\begin{center}
\begin{tabular}{lccc}
Genotype & $A_1A_1$ & $A_1A_2$ & $A_2A_2$\\ 
\hline\hline
Freq.\ before selection  & $p^{2}$ & $2pq$ & $q^{2}$\\
Fitness  & $w_{11}$ & $w_{12}$ & $w_{22}$ \\
Freq.\ after selection  & $p^{2}w_{11}/\bar w$
         & $2pq w_{12}/\bar w$
         & $q^{2}w_{22}/\bar w$\\ \hline
\multicolumn{4}{c}{Where $q = 1-p$}\\
\multicolumn{4}{c}{and $\bar w = p^{2}w_{11}+2pqw_{12}+q^{2}w_{22}$}\\
\end{tabular}
\end{center}
\end{table}%
The frequencies before selection are in Hardy-Weinberg equilibrium.
The contribution of each genotype to the gene pool of the next
generation is proportional to the product of its frequency and
fitness.  Dividing by $\bar w$, the mean fitness, ensures that the
genotypic frequencies after selection sum to unity.

The frequency of allele $A_1$ is $p = P_{11} + \frac{1}{2}
P_{12}$, where $P_{11}$ and $P_{12}$ are the frequencies of
$A_1A_1$ and $A_1A_2$.  
\begin{exercise}
Show that $p = P_{11} + \frac{1}{2}P_{12}$.
\answer
The population contains $NP_{11}$ individuals of genotype $A_1A_1$ and
$NP_{12}$ of genotype $A_1A_2$.  Each of the former carries two copies
of $A_1$ and each of the latter carries one copy.  The total number of
copies of $A_1$ is thus $2NP_{11} + NP_{12}$.  Dividing by $2N$ (the total
number of genes) gives the relative frequency of $A_1$ within the
population: $p = P_{11} + \frac{1}{2}P_{12}$.
\end{exercise}
Applying this to the last row of table~\ref{tab.fitness} gives the
frequency of $A_1$ after selection.
\begin{equation}
p' = \frac{p^{2} w_{11} + pqw_{12}}{\bar w}
\label{eq.seln.recur}
\end{equation}
Here, $p$ is the allele frequency among the parents and $p'$ is that
among the children.  Equation~\ref{eq.seln.recur} tells us $p$'s value
in one generation as a function of its value in the previous
generation.  Such equations are called \emph{recurrence equations}.
They provide recipes for getting from one generation to the next.

\section{The rate of change}
\label{sec.deltap}
How much change occurs between successive generations?  To find out,
we calculate $\Delta p$ (pronounced ``delta pee''), which equals $p' -
p$.  $\Delta p$ is positive when $p$ is increasing, and negative when
$p$ is decreasing.  When there is no tendency to change, the system is
said to be at \emph{equilibrium}, and $\Delta p = 0$.  Subtracting $p$
from equation~\ref{eq.seln.recur} gives (after some algebra)
\begin{equation}
  \Delta p = \frac{p q}{\bar w}
  \bigl[p (w_{11}-w_{12}) + q(w_{12} - w_{22})\bigr]
\label{eq.dpseln}
\end{equation}
This equation provides a great deal of intuition concerning the dynamics of
natural selection, which I summarize in the paragraphs that follow.

\begin{exercise}
  Derive Eqn.~\ref{eq.dpseln} from Eqn.~\ref{eq.seln.recur}.
  \answer
  \begin{eqnarray*}
\Delta p &=& \frac{p^{2} w_{11} + pqw_{12}}{\bar w} - p\\
   &=& \frac{p^{2} w_{11} + pqw_{12} - p^3 w_{11} - 2p^2q w_{12} -
  pq^2 w_{22}}{\bar w}\\
   &=& \frac{p^{2}q w_{11} + pq(q-p)w_{12} - pq^2 w_{22}}{\bar w}\\
   &=& \frac{pq}{\bar w}[pw_{11} + (q-p)w_{12} - q w_{22}]\\
   &=& \frac{pq}{\bar w}[p(w_{11}-w_{12}) + q(w_{12} - w_{22})]\\
  \end{eqnarray*}    
\end{exercise}  

\begin{result}
Response to selection is slow when either allele is rare.
\end{result}
This is because $\Delta p$ is proportional to $pq$ and is therefore
low when $p$ is small or when $q$ is small.

\begin{result}
  Selection is especially slow when a recessive allele is rare.
  \label{res.slowrecess}
\end{result}
If $A_1$ is recessive, $w_{12} = w_{22}$, and Eqn.~\ref{eq.dpseln}
becomes $\Delta p = p^2 q(w_{11} - w_{12})/\bar w$. When $A_1$ is
rare, $p^2$ is very small, so response to selection is very slow.

\begin{result}
  Selection is especially slow when a dominant allele is common.
\end{result}
If $A_1$ is dominant and common, then $A_2$ is recessive and rare, so
Result~\ref{res.slowrecess} implies that $\Delta p \approx 0$.

\begin{result}
A rare allele spreads if its heterozygote is fitter than the common
homozygote.
\label{res.invade}
\end{result}
This result is used widely, because all mutations are initially rare,
and we would like to know which ones will spread. It holds because
when $A_1$ is rare, $p$ is small, the expression in square brackets is
approximately $w_{12} - w_{22}$.  Allele $A_1$ will increase in
frequency if this quantity is positive.

\begin{result}
If the heterozygote has intermediate fitness, then one allele will
increase to fixation.
\end{result}
To see why, note that if $w_{11} > w_{12} > w_{22}$, then the quantity
in square brackets in Eqn.~\ref{eq.dpseln} is always positive, and
$A_1$ will increase to fixation. On the other hand, $A_2$ will fix if
$w_{11} < w_{12} < w_{22}$.

\begin{result}
If the heterozygote has the highest fitness, then $A_1$ evolves toward
an intermediate equilibrium.
\label{res.overdom}
\end{result}
This is the case of ``fitness overdominance'' or ``heterozygote
superiority.'' In this case, $w_{11} - w_{12} < 0$ and $w_{12} -
w_{22} > 0$, so $\Delta p$ is positive when $p$ is small but negative
when $p$ is large.

\begin{SCfigure}
\centering % -*-latex-*-
\mbox{\beginpicture
\small
\setcoordinatesystem units <2in, 10in> point at 0 0
\setplotarea x from 0 to 1, y from -0.028 to 0.08
\axis left label {$\Delta p$}
  ticks numbered from -0.02 to 0.08 by 0.02 /
\axis bottom 
  ticks withvalues 0 {$\hat p$} 1 / at 0 {0.6363636364} 1 / /
\setdots
\putrule from 0 0 to 1 0
\setsolid  
\plot 
0 0 .05000000000 .03993157380 .1000000000 .06405307600
.1500000000 .07705450438 .2000000000 .08205128205
.2500000000 .08121019108 .3000000000 .07610186092
.3500000000 .06791044776 .4000000000 .05756457565
.4500000000 .04582298487 .5000000000 .03333333333
.5500000000 .02067487360 .6000000000 .008391608392
0.6363636364 0 
.6500000000 -.002979698756 .7000000000 -.01288343558
.7500000000 -.02071823204 .8000000000 -.02580645161
.8500000000 -.02735676786 .9000000000 -.02441534144
.9500000000 -.01579898771 1.000000000 0
/
\setcoordinatesystem units <2in, 2in>  point at 0 1.5
\setplotarea x from 0 to 1, y from 0.7 to 1.2
\axis left label {$\bar w$}
  ticks numbered from 0.7 to 1.1 by 0.2 /
\axis bottom label {$p$}
  ticks withvalues 0 {$\hat p$} 1 / at 0 {0.6363636364} 1 / /
\plot 
0 .7 .05000000000 .7672500000 .1000000000 .8290000000
.1500000000 .8852500000 .2000000000 .9360000000
.2500000000 .9812500000 .3000000000 1.021000000
.3500000000 1.055250000 .4000000000 1.084000000
.4500000000 1.107250000 .5000000000 1.125000000
.5500000000 1.137250000 .6000000000 1.144000000
.6500000000 1.145250000 .7000000000 1.141000000
.7500000000 1.131250000 .8000000000 1.116000000
.8500000000 1.095250000 .9000000000 1.069000000
.9500000000 1.037250000 1.000000000 1.000000000
/
\endpicture}

\caption{Model of heterozygote advantage. Upper panel: the rate,
  $\Delta p$, of change in $p$. Equilibria occur where $\Delta
  p=0$. The equilibria at $p=0$ and $p=1$ are unstable. That at $p
  \approx 0.64$ is stable. Lower panel: mean fitness, $\bar w$, as a
  function of $p$. The equilibrium value, $\hat p$, maximizes $\bar w$
  and coincides the point at which $\Delta p = 0$.  }
\label{fig.dpseln}
\end{SCfigure}

\begin{SCfigure}
% -*-latex-*-
\mbox{\beginpicture
\small
\setcoordinatesystem units <0.1in, 1.1in>
\setplotarea x from 0 to 23, y from 0 to 1
\axis left label {$p(t)$} ticks
  withvalues 0.0 0.2 0.4 {$\hat p$} 0.8 1.0 / at 0.0 0.2 0.4 0.6363636364
  0.8 1.0 / /
\axis bottom label {$t$ (Generations)}
  ticks numbered from 0 to 20 by 5 /
\setdots
\putrule from 0 0.6363636364 to 22 0.6363636364
\setsolid  
\plot 
0 0.1 1 .1640530760 2 .2432133616 3 .3248181678 4 .3971767598 5
.4553692119 6 .4998761659 7 .5332408700 8 .5581434063 9 .5767780168 10
.5907937712 11 .6013945473 12 .6094541224 13 .6094541224 14
.6156092326 15 .6203275030 16 .6239553742 17 .6267516688 18
.6289111975 19 .6305815324 20 .6318750562 21 .6328777273 22
.6336555245 23 .6342592332 /
\endpicture}

\caption{Time path of allele frequency using the same model as
  Fig.~\ref{fig.dpseln}. The dotted line indicates the equilibrium
  value, $\hat p \approx 0.64$.}
\label{fig.timepath}
\end{SCfigure}

\begin{result}
  When the heterozygote has highest fitness, the allele that is most
  common at the stable equilibrium is the one whose homozygote has
  highest fitness.
\end{result}
To derive this result, set to zero the quantity in square brackets in
Eqn.~\ref{eq.dpseln} and rearrange to obtain
\[
\frac{\hat p}{\hat q} = \frac{w_{12} - w_{22}}{w_{12} - w_{11}}
\]
where $\hat p$ and $\hat q$ are the equilibrium values of $p$ and $q$.
On the right side of this equation, the numerator and denominator are
both positive, because we are concerned with the case of heterozygote
superiority. If $w_{11} > w_{22}$, then the denominator will be
smaller than the numerator, and this implies that $\hat p$ (the frequency
of $A_1$) exceeds $\hat q$ (that of $A_2$). For example, suppose that
$w_{22}=0.1$, $w_{11}=0.2$, and $w_{12}=0.3$. Then
\[
\frac{\hat p}{\hat q} = \frac{0.3 - 0.1}{0.3 - 0.2} = 2
\]
and $\hat p$ is twice as large as $\hat q$.

\begin{result}
If the heterozygote has lowest fitness, then there are two stable
equilibria, one at which $A_1$ is fixed and another at which it is
lost.
\end{result}
In this case, $w_{11} - w_{12} > 0$ and $w_{12} - w_{22} < 0$, so
$\Delta p$ is negative when $p$ is small but positive when $p$ is
large.

\section{The same result in terms of genic fitnesses}
\label{sec.genicfitness}
We can also write equation~\ref{eq.dpseln} as
\begin{equation}
\Delta p = \frac{p q(w_1 - w_2)}{\bar w}
\label{eq.dpseln.genic}
\end{equation}
where $w_1$ and $w_2$ are the \emph{genic} or \emph{marginal}
fitnesses of $A_1$ and $A_2$.  The genic fitness of allele $A_i$ is
simply the average fitness of copies of $A_i$.

In diploid organisms, genes occur in pairs.  Each copy of the $A_1$
allele is paired with some other gene.  What is the probability that
this second gene is also a copy of $A_1$?  If mating is at random,
then the second gene is (in effect) drawn at random from among all the
genes in the population.  Since a fraction $p$ of these are copies of
$A_1$, $p$ is also the probability that the second gene is an $A_1$.
In this case, the original copy of $A_1$ resides in an $A_1A_1$
homozygote with fitness $w_{11}$.  On the other hand, the original
$A_1$ might (with probability $q$) be paired with a copy of $A_2$.  If
so, then the initial $A_1$ resides in a heterozygote with fitness
$w_{12}$.  The mean fitness of allele $A_1$ is therefore
\[
w_1 = p w_{11} + q w_{12}
\]
A similar argument applies if the initial gene is a copy of $A_2$ and
leads to
\[
w_2 = p w_{12} + q w_{22}
\]
Since $pq/\bar w$ is always positive, equation~\ref{eq.dpseln.genic}
implies that
\begin{result}
  Selection favors the allele whose genic fitness is highest.
\end{result}  

\begin{exercise}
\label{ex.selection.7}
Verify that $w_1-w_2$ is equal to $p (w_{11}-w_{12}) +
     q(w_{12} - w_{22})$.
\answer
The calculation is very simple:
\begin{eqnarray*}
w_1 - w_2 &=& p w_{11} + q w_{12} - q w_{22} - p w_{12}\\
 &=& p(w_{11} - w_{12}) + q(w_{12} - w_{22})
\end{eqnarray*}
\end{exercise}

%\begin{exercise}
%\label{ex.selection.8}
%Suppose that there are three genotypes, $A_1A_1$, $A_1A_2$, and $A_2
%A_2$, with fitnesses 1, 1.5, and 0.5.  Calculate the two genic
%fitnesses under the assumption that $p=0.1$.  Which allele is being
%favored by selection?  
%\answer
%First enter the formulas for $w_1$ and $w_2$, along with the three
%genotypic fitnesses.  Then calculate
%\begin{verbatim}
%> subs(p=0.1, w1 - w2);
%    .85
%\end{verbatim}
%Since the answer is positive, allele~1 is favored.
%\end{exercise}

%\begin{exercise}
%\label{ex.selection.9}
%Repeat this calculation assuming that $p=0.9$.  Which allele is being
%favored by selection?
%\answer
%\begin{verbatim}
%> subs(p=0.1, w1 - w2);
%    -.35
%\end{verbatim}
%Now the difference is negative, so allele~2 is favored.
%\end{exercise}

%\begin{exercise}
%\label{ex.selection.10}
%Equation~\ref{eq.dpseln.genic} also tells us that equilibrium will
%occur when $w_1 = w_2$.  Use this principle to find the equilibrium value of
%$p$ from the genotypic fitnesses above.
%\answer
%\begin{verbatim}
%> solve(w1=w2, p);
%    .6666666667
%\end{verbatim}
%\end{exercise}

\section{Wright's equation}
We saw above that Eqn.~\ref{eq.dpseln} is rich in insight about the
behavior of natural selection. Most of that insight came from studying
the quantity in square brackets. Sewall Wright \cite{Wright:TGF-69-II}
noticed that this quantity is equal to $\frac{1}{2}\frac{d \bar
  w}{dp}$, where $d \bar w/ dp$ is the derivative (or slope) of the
curve that describes $\bar w$ as a function of $p$. Thus,
\begin{equation}
\Delta p = \frac{pq}{2\bar w}\cdot\frac{d \bar w}{d p}
\label{eq.dpseln.wright}
\end{equation}
This equation is equivalent to~\ref{eq.dpseln}, but provides a
different kind of insight.  Notice that $pq/2\bar w$ is always
positive when $p$ is neither zero nor unity.  This means that $p$ will
increase when $d\bar w/d p > 0$ and will decrease when this derivative
is negative.  Selection pushes allele frequency in a direction that
increases $\bar w$, the mean fitness.  It pushes the population
``uphill'' on the function $\bar w(p)$.  Consequently,
\begin{result}
  Stable equilibria occur at peaks in the graph of mean fitness
  against allele frequency.
\end{result}
The lower panel of Fig.~\ref{fig.dpseln} illustrates this principle.

\begin{exercise}
\label{ex.selection.11}
Verify that equations~\ref{eq.dpseln} and~\ref{eq.dpseln.wright} are
equivalent.
\answer
Mean fitness is
\begin{eqnarray*}
\bar w &=& p^2w_{11}+2pqw_{12}+q^2w_{22}\\
   &=& p^2w_{11}+2(p - p^2)w_{12}+(1 - 2p + p^2)w_{22}
\end{eqnarray*}
Taking the derivative,
\begin{eqnarray*}
  \frac{d\bar w}{dp} &=& 2pw_{11}
  + (2 - 4p)w_{12}
  + (- 2 + 2p)w_{22}\\[0.5\baselineskip]
\frac{1}{2}\frac{d\bar w}{dp} &=& pw_{11}
  + (1 - 2p)w_{12}
  - (1 - p)w_{22}\\
  &=& pw_{11}  + (q - p)w_{12} - qw_{22}\\
  &=& p(w_{11} - w_{12})  + q(w_{12} - w_{22})
\end{eqnarray*}
which equals the quantity in square brackets in Eqn.~\ref{eq.dpseln}.
\end{exercise}

\begin{exercise}
\label{ex.selection.12}
Define $w_{11}=1$, $w_{12}=1.1$, and $w_{22}=0.8$.  Find the
equilibria in this system by graphing $\Delta p$ as a function of $p$.
\answer
There are unstable equilibria at 0 and 1, and a stable equilibrium at
3/4.   
\end{exercise}

\section{The time required for a selective sweep}
\label{sec.sweeptime}
How long does it take for an allele to increase from one value to
another, under the influence of natural selection? This question has
been addressed by \citet{Haldane:CE-32} and several later authors
\citetext{\citealp[sec.~5.3]{Crow:IPG-70};
  \citealp[sec.~3.1.iii.a]{Charlesworth:EEG-10}}. The model is
simplest if there is no dominance, so that alleles contribute
additively to fitness. Let's suppose that genotypes $A_1A_1$,
$A_1A_2$, and $A_2A_2$ have relative fitnesses $w_{11} = 1+s$, $w_{12}
= 1+s/2$, and $w_{22} = 1$. The left panel of Fig.~\ref{fig.plogit}
shows the time path of allele $A_1$ over the range $0.01 \leq p \leq
0.99$. As the graph shows, this selective sweep takes more than 1500
generations.

\begin{figure}
  {\centering
    \includegraphics[width=0.9\textwidth]{plogit-0.01-0.01-0.99.pdf}\\}
  \caption{Time path of $p$, the frequency of an advantageous allele,
    $A_1$, over the range $0.01 \leq p \leq 0.99$. The model assumes
    that the fitness of the favored homozygote ($A_1A_1$) exceeds that
    of the unfavored one ($A_2A_2$) by $s=0.01$ and that there is no
    dominance. Calculations used Eqn.~\ref{eq.seln.recur}.}
  \label{fig.plogit}
\end{figure}

The graph on the right shows the time path of the transformed
variable,
\[
x = \ln \frac{p}{1-p}
\]
This is called the ``logit transform'' and is often used in
statistics.  As you can see, the graph of $x$ is a straight line, or
very nearly so. Appendix~\ref{sec.slope} shows that the slope of this
line is approximately $s/2$. Consequently, it is easy to write $x$ as
a function of $t$:
\[
x(t) = x_0 + ts/2
\]
where $x_0$ is the initial value of $x$.  Solving for $t$ gives
\citetext{\citealp[Eqn.~5.3.13]{Crow:IPG-70};
  \citealp[Eqn.~3.3]{Charlesworth:EEG-10}}
\begin{eqnarray}
t &=& \frac{2}{s}(x(t) - x_0)\nonumber\\
&=& \frac{2}{s}\ln \frac{p(t)q_0}{q(t)p_0}
\label{eq.t}
\end{eqnarray}
where $p_0$ and $q_0$ are the initial values of $p$ and $q = 1-p$.
This result makes it easy to figure out how long it would take for
selection to raise the frequency of an allele from one value to
another. It is an approximation for weak selection, but it works
remarkably well for values of $s$ up to about 0.1.

Let us use it to ask about the time required for a selective
sweep. Equation~\ref{eq.t} is deterministic and consequently does not
inform us about the initial generations, when the advantageous allele
is very rare. Neither does it inform us about the final generations,
when the alternate allele is rare. Those dynamics are stochastic and
are not captured by our deterministic model. We therefore ignore the
beginning and end of the selective sweep and ask how long it takes for
the advantageous allele to increase from 1\% to 99\%. For this case,
Eqn.~\ref{eq.t} becomes $t \approx 18/s$.
\begin{result}
  It takes roughly $18/s$ generations for an advantageous allele to
  increase in frequency from 0.01 to 0.99, if there is no dominance.
\end{result}  
For example, in Fig.~\ref{fig.plogit}, $s=1/100$, so the selective
sweep lasts about 1800 generations.  If generations are 25~years in
length, this corresponds to about 45 thousand years. If there is
dominance, the time required for a selective sweep is longer
\citetext{\citealp[p.~193]{Crow:IPG-70};
  \citealp[Box~3.2]{Charlesworth:EEG-10}}.

\begin{exercise}
  \label{ex.t}
  How many generations are required for selection to increase the
  frequency of an allele from 1\% to 99\%, if the coefficient of
  selection is $s=0.001$, and there is no dominance?
  \answer
  18,380 generations
\end{exercise}

\begin{exercise}
  \label{ex.t2}
  Repeat exercise~\ref{ex.t}, but this time figure out the time
  required for the allele to increase in frequency from 25\% to 75\%.
  \answer
  4394 generations
\end{exercise}

\section{Summary}
This note introduces the conventional algebraic model of natural
selection. Equilibria occur where $\Delta p=0$. There are always
equilibria at $p=0$ and $p=1$. There may also be an intermediate
equilibrium at which $0 < p < 1$. Stable equilibria occur at peaks in
the graph of mean fitness against $p$.

When the heterozygote has intermediate fitness, selection favors the
allele whose homozygote has the higher fitness. A rare allele will
spread if the fitness of its heterozygote is higher than that of the
common homozygote. When the heterozygote's fitness is higher than that
of either homozygote, the system evolves toward a stable equilibrium
at which $p$ is greater than zero but less than unity. When the
heterozygote has lowest fitness, the system evolves away from an
unstable intermediate equilibrium and toward either $p=0$ and $p=1$.

When gene effects are additive, the selective sweep of an advantageous
allele takes roughly $18/s$ generations.

\section*{Acknowledgements} I thank Jon Seger for comments and hold
him responsible for the title.

\appendix

\section{The slope of the straight line in Fig.~\ref{fig.plogit}}
\label{sec.slope}
In sec.~\ref{sec.sweeptime}, we assumed that genotypes $A_1A_1$,
$A_1A_2$, and $A_2A_2$ had relative fitnesses $w_{11} = 1+s$, $w_{12}
= 1+s/2$, and $w_{22} = 1$. With these definitions,
Eqn.~\ref{eq.dpseln} becomes $\Delta p = pqs/2\bar w$, which is
approximately $pqs/2$ if selection is weak. Under weak selection, the
change in each generation will be small, and we can treat $p$ as a
continuous function of time, and $\Delta p$ as a derivative:
\[
  dp/dt = pqs/2
\]
As we saw above in Fig.~\ref{fig.plogit}, the time path is much
simpler if we transform $p$ into
\[
x = \log \frac{p}{q} = \log p - \log q
\]
where $q = 1 - p$. The derivative of $x$ is
\begin{eqnarray*}
  dx/dt &=& \frac{1}{p} \frac{dp}{dt} + \frac{1}{q}\frac{dp}{dt}
  =\frac{pqs}{2}\left(\frac{1}{p} + \frac{1}{q}\right)
  =\frac{pqs}{2}\left(\frac{q}{pq} + \frac{p}{pq}\right)\\
  &=& s/2
\end{eqnarray*}  
Because $s$ is a constant, this result says that the graph of $x$
against $t$ is a straight line with slope $s/2$.


\section{Answers to Exercises}
\listanswers

\bibliographystyle{plainnat}
\bibliography{defs,arr,math,molrec}

\end{document}
