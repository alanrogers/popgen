% -*-latex-*-
\documentclass[11pt]{article}
\usepackage[margin=1in,letterpaper]{geometry}
\usepackage{hyperref}
\usepackage[backend=biber,
  natbib=true,
  style=ext-numeric-comp,
  sorting=nyt]{biblatex}
\DeclareFieldFormat % Titles in sentence case
  [article,inbook,incollection,inproceedings,patent,thesis,unpublished]
  {titlecase:title}{\MakeSentenceCase*{#1}}
\renewbibmacro{in:}{ % Don't use "In:" before journal names.  
  \ifentrytype{article}{}{\printtext{\bibstring{in}\intitlepunct}}}  
\addbibresource{defs.bib}
\addbibresource{arr.bib}
\addbibresource{molrec.bib}
\begin{document}
\title{The Confusing Vocabulary of Population Genetics}
\author{Alan R. Rogers}
\maketitle

The modern literature of population genetics is confusing because two
systems of vocabulary are in use, and these involve inconsistent
definitions of the terms \emph{gene} and \emph{allele}. To avoid
confusion, Jon Seger and I have used a third system in our course on
Human Evolutionary Genetics.  These systems are summarized in
Table~\ref{tab.vocab}. System~1 is the one used in the classical
literature. System~2 began to appear in the 1980s and is common
today. System~3 is the one Jon and I use in our class.

\begin{table}
  \caption{Three systems of vocabulary}
  \label{tab.vocab}
{\centering\begin{tabular}{p{2.2in}ccc}
                        &   1   & 2 & 3\\
Position on chromosome  & locus & locus & locus\\
Protein-coding locus    & gene  & gene & gene\\
Physical copy of DNA at locus  & gene  & allele &gene copy\\
One of several variants at a locus& allele & allele & allele\\
  \end{tabular}\\}
\end{table}

As an example of the classical vocabulary in use, I offer Thomas Hunt
\citet[p.~25]{Morgan:TG-26}, who wrote that
\begin{quote}
the characters of the individual are referrable to paired elements
(genes) in the germinal material.
\end{quote}
For Morgan, the genotype at each locus consisted of a \emph{pair} of
genes. He thought of genes as physical objects, which are
transmitted from parent to
offspring. R.A.\ \citet[p.~8]{Fisher:GTN-30} made this even clearer a
few years later.
\begin{quote}
Those organisms (homozygotes) which received like genes, in any pair
of corresponding loci\footnote{The plural of ``locus'' is ``loci,''
pronounced ``low sigh.''}, from their two parents, would necessarily
hand on genes of this kind to all of their offspring alike; whereas
those (heterozygotes) which received from their two parents genes of
different kinds\ldots
\end{quote}

In the 1960s, people began using the term ``gene'' in a new way.
Before that, a ``gene'' was the unknown hereditary something
responsible for some difference between individuals---for example,
round versus wrinkled in Mendel's peas.  But in the 1960s and 1970s,
``gene'' came to mean something more specific: the DNA that encodes a
particular protein \citep{Watson:MBG-65}.

This was unfortunate for at least two reasons. First, we now know that
there are ``genes'' in the classical sense that do not encode
protein. For example, some are ``binding sites,'' which allow an
enzyme to attach to the DNA. Others encode RNA that is never
translated into protein. These are genes in the classical sense
because a mutation in one can have phenotypic consequences. But they
do not encode protein and are therefore not genes in the modern sense.

Second, people began to be uncomfortable with sentences that would
previously have been unremarkable. For example, ``at each locus we
inherit one gene from our mother and another from our father.'' To
geneticists trained in molecular biology but not in population
genetics, this must have seemed wrong. At a given locus, the DNA we
get from Mom and Dad represent \emph{two copies of a single gene},
because both copies encode the same protein. It therefore seemed wrong
to talk about getting two genes, one from Mom and one from Dad.  This
left us with no word for the physical thing that one inherits from Mom
or from Dad at a particular locus. To fill this gap, the word
``allele'' was pressed into service. In this new system of vocabulary,
``allele'' means two different things, as shown in
Table~\ref{tab.vocab}.

Here is how John \citet[p.~6]{Gillespie:PGC-04} summarized the second
system of vocabulary:
\begin{quote}
  Here we will use \emph{locus} to refer to the place on a chromosome
  where an \emph{allele} resides. An \emph{allele} is just a bit of
  DNA at that place. A locus is a template for an allele. An allele is
  an instantiation of a locus. A locus is not a tangible thing;
  rather, it is a map describing where to find a tangible thing, an
  allele, on a chromosome. (Some books use \emph{gene} as a synonym
  for our \emph{allele}. However, \emph{gene} has been used in so many
  different contexts that it is not very useful for our purposes.)
  With this convention, a diploid individual may be said to have two
  alleles at a particular autosomal locus, one from its mother and the
  other from its father.
\end{quote}

Vocabulary systems~1 and~2 both use a single word to mean two
different things. In system~1, that word is ``gene;'' in system~2, it
is ``allele.'' Unfortunately, the ambiguity inherent in system~2 is
confusing more often than is that in system~1. For example, here is
the same sentence in each of the three systems:
\begin{enumerate}
\item If the \emph{genes} you inherit from Mom and Dad are
  different alleles, then you are a heterozygote.
\item If the \emph{alleles} you inherit from Mom
  and Dad are different alleles, then you are a heterozygote.
\item If the \emph{gene copies} you inherit from Mom and Dad are
  different alleles, then you are a heterozygote.
\end{enumerate}
Our own vocabulary (system~3) isn't perfect either. We often find
ourselves referring to a ``copy of a gene copy'' for example. All in
all, I prefer system~1 to either alternative.
  
\printbibliography
  
\end{document}
