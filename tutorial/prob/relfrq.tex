\subsection{Relative frequency}
\label{sec.relfrq}

There is a close relationship between the concepts of probability and
relative frequency.  Since relative frequency is more familiar, we
cover it first.  As an example, consider a toy data set consisting of 10
items, which represent genotypes in a sample of diploid individuals:
$[GG, CG, GG, GG, CG, CG, CG, GG, GG, CC]$.  We can describe these
data concisely by counting the number of copies of each value in the
list.  The resulting counts are tabulated in panel~a of
Fig~\ref{fig.toydat}.  That table also shows the \emph{relative
frequency} of each value in the data.  The relative frequency is
simply the count as a proportion of sample size.  For example, the
value ``$CG$'' appears 4 times in the data, and there are 10 items in
all.  Thus, the relative frequency of ``$CG$'' is $0.4$.  Confusingly,
the word ``frequency'' is used interchangeably for counts and relative
frequencies.  You have to catch the meaning from context.

The bottom row of the table in Fig.~\ref{fig.toydat} shows the sums of
the two distributions.  The counts sum to the sample size, and the
relative frequencies sum to unity.

\begin{exercise}
Make a table showing the relative frequency distribution of the
following data: $[A, B, B, A, A, A, C, A, B, C]$.  Verify that it sums
to unity.
\answer
The frequency distribution is:
\begin{center}
\begin{tabular}{cc}
        & Relative\\
Value   &frequency\\
\hline 
 $A$   & 0.5\\
 $B$   & 0.3\\
 $C$   & 0.2\\
\cline{2-2}
        & 1.0
\end{tabular}
\end{center}
\end{exercise}

\begin{figure}
\begin{boxedminipage}{\columnwidth}
\centering
\subfloat[Frequency distribution table]
{\begin{tabular}{ccc}
       &           & Relative\\
Value  & Count     &frequency\\
\hline 
 $GG$  &   5  & 0.5\\
 $CG$  &   4  & 0.4\\
 $CC$  &   1  & 0.1\\
\cline{2-3}
       &  10  & 1.0
\end{tabular}}\\
\subfloat[Histogram]{\mbox{\beginpicture
\setcoordinatesystem units <0.4in, 0.2in> 
\setplotarea x from 0 to 3, y from 0 to 5
\axis left shiftedto x=-0.1 label {\lines{Count}} 
  ticks numbered from 0 to 4 by 2 /
\axis right shiftedto x=3.1 label {\lines{Relative\cr frequency}} 
  ticks withvalues 0.0 0.2 0.4 / at 0 2  4 / /
\axis bottom
   label {Value}
%   ticks numbered from 0 to 2 by 1 /
   ticks length <0pt> withvalues {$GG$} {$CG$} {$CC$} / at 0.5 1.5 2.5 / /
\sethistograms
\plot 0 0 1 5 2 4  3 1 
/
\endpicture}
}
\end{boxedminipage}
\caption{Toy data set summarized using (a)~a table of relative
  frequencies and (b)~a histogram.}
\label{fig.toydat}
\end{figure}

Distributions of counts and relative frequencies are often shown using
graphs rather than tables.  In Fig.~\ref{fig.toydat}, panel~b uses a
histogram to display the data from panel~a.

\begin{exercise}
Make a histogram showing the relative frequency distribution that you
tabulated in the preceding exercise.
\answer
The histogram is:
\begin{center}
\mbox{\beginpicture
\setcoordinatesystem units <0.4in, 2in> 
\setplotarea x from 0 to 3, y from 0 to 0.5
\axis left shiftedto x=-0.1 label {\lines{Relative\cr frequency}} 
  ticks numbered from 0.00 to 0.50 by 0.25 /
\axis bottom
   label {Value}
%   ticks numbered from 0 to 2 by 1 /
   ticks length <0pt> withvalues {$A$} {$B$} {$C$} / at 0.5 1.5 2.5 / /
\sethistograms
\plot 0 0 1 0.5 2 0.3  3 0.2 
/
\endpicture}
\end{center}
\end{exercise}

This section has talked about three ideas: relative frequencies,
frequency distributions, and histograms.  Make sure you understand
them, for they will all reappear (slightly modified) throughout the
pages that follow.  Probability, for example, is just a special kind
of relative frequency.  It is the subject of the remainder of this
chapter.
