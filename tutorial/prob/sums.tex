\section{Sums and sigma notation}
\label{sec.sigma}
You learned in grade school to calculate sums such as $10 + 12 + 10 +
8 = 40$.  To generalize this calculation, suppose we have 4 arbitrary
numbers, $x_1$, $x_2$, $x_3$, and $x_4$.  Their sum is
\[
x_1 + x_2 + x_3 + x_4
\]
This sum can also be written using ``sigma notation'' as
\[
\sum_{i=1}^4 x_i
\]
The ``$\Sigma$'' symbol is a Greek sigma and indicates summation.  The
subscript ``$i=1$'' indicates that the sum begins with $x_1$, and the
superscript ``4'' indicates that the sum ends with $x_4$.

More generally, if the number of numbers is an unknown value, $N$,
then their sum is
\[
\sum_{i=1}^N x_i
=
x_1 + x_2 + \cdots + x_N
\]
Sometimes sums are written without limits, as in \[\sum_i x_i.\]
This means the sum over all terms, however many there may be.  
When sums are written within the text of a paragraph, the limits look
like subscripts and superscripts, as in $\sum_{i=1}^N x_i$.
