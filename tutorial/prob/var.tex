\section{Variance}

There are many ways to measure variation, of which the \emph{variance}
is the most common.  The variance is the average squared difference
from the mean.  For example, suppose we are dealing with data
consisting of the numbers 10, 12, 10, and 8.  The mean is 10, so the
variance is
\begin{displaymath}
V = ((10 - 10)^2 + (12-10)^2 + (10-10)^2 + (8-10)^2)/4 = 2
\end{displaymath}
There is a variety of ways to represent the variance, including
\begin{eqnarray}
V &=& N^{-1} \sum_i (x_i - \bar x)^2\nonumber\\
  &=& \sum_x (x-\bar x)^2 p_x\nonumber\\
  &=& \sum_x x^2 p_x - \bar x^2
\label{eq.variance.def}
\end{eqnarray}

\begin{exercise}
Verify that the formulas Equation~\ref{eq.variance.def} are
equivalent.
\end{exercise}

\begin{exercise}
What are the mean and variance of the numbers 3, 9, 15, and 8?
\answer
The mean and variance are 8.75 and 18.1875.
\end{exercise}

If $X$ is a random variable (rather than data), its variance is
\begin{equation}
V[X] = E\left[ (X - E[X])^2 \right]\label{eq.V1}
\end{equation}
Note the similarity between this expression and
Equation~\ref{eq.variance.def}.  The variance can also be written in
either of the following ways:
\begin{eqnarray}
V[X] &=& E\bigl[ X^2 \bigr] - E[X]^2\label{eq.V2}\\
     &=& E\bigl[ X(X - E[X]) \bigr]\label{eq.V3}
\end{eqnarray}
These expressions hold irrespective of whether the random variable is
continuous or discrete.

\begin{exercise}
Prove that if $a$ is a constant and $X$ a random variable, then $V[aX]
= a^2 V[X]$.
\answer
First, $E[aX] = aE[X]$ by equation~\ref{eq.E[aX]}.  Next,
\begin{eqnarray*}
V[aX] &=& E[(aX)^2] - E[aX]^2\\
 &=& E[a^2 X^2] - a^2 E[X]^2\\
 &=& a^2\left( E[X^2] - E[X]^2\right)\\
 &=& a^2 V[X]
\end{eqnarray*}
\end{exercise}
