\section{Factorials and binomial coefficients}
\label{sec.factorial}

The \emph{factorial} of $x$ is written $x!$ and equals
\[
x! = x \cdot (x-1) \cdot (x-1) \cdots 1
\]
It is pronounced ``$x$ factorial.''  For example, $3! = 3\cdot 2 \cdot
1 = 6$.  As a special case, $0!$ is defined to equal 1.  Factorials
arise in problems that involve rearrangements of the items in a list.
For example, the letters ``ABC'' can be arranged in six different
orders: (1)~ABC, (2)~ACB, (3)~BAC, (4)~BCA, (5)~CAB, and (6)~CBA.
These rearrangements are called \emph{permutations}.  More generally,
suppose we have a string of $x$ letters.  How many permutations does
it have?  There are $x$ ways to choose the first letter.  Having
chosen the first, there are then $x-1$ ways to choose the second,
$x-2$ ways to choose the third, and so on.  There is only one way to
choose the last, for by then all the other letters have been chosen.
Thus, the number of permutations of $x$ items is $x!$.

A \emph{binomial coefficient} is
written $\binom{N}{x}$ and pronounced ``$N$ choose x.''  It equals
\[
\binom{N}{x} = \frac{N!}{x!(N-x)!}
\]
and can be interpreted as the number of ways of choosing $x$ items out
of a list of $N$.  For example, consider the number of pairs of
letters in the string ABC.  According to the formula, there should be
$\binom{3}{2} = 3!/(2!\cdot 1!) = 3$ pairs.  We get the same answer by
listing the pairs: AB, AC, and BC.
