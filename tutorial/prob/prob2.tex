\section{Random Variables and Expectations}
\label{sec.distrv}

It is time to introduce some vocabulary.  For any experiment, the set
of possible outcomes is called the \emph{sample space}.  For example,
there were four possible outcomes ($RR$, $RG$, $GR$, and $GG$) in
Kerrich's urn experiment.  These constitute the sample space of that
experiment.  There is nothing new here except the term itself.

We will also need the idea of a \emph{random variable}.  A variable
$X$ is called a \emph{random variable} if the values it takes are
numbers and it takes each value with a specified probability.  For
example, we might define $X$ as the result of one roll of a die.  The
sample space of $X$ is \{1, 2, 3, 4, 5, 6\}.  If the die is fair, it
takes each of these values with probability $1/6$.  Another random
variable is $Y=X^2$.  The sample space of $Y$ is \{1, 4, 9, 16, 25,
36\} and again it takes each value with probability $1/6$.  $X$ and
$Y$ are distinct random variables even though the underlying
experiment---rolling a die---is the same in both cases.

\begin{exercise}
  Consider the experiment of throwing two dice, one red and one white.
  What is the sample space?  If the dice are fair, what are the
  associated probabilities?  Do not enumerate the entire sample space;
  just describe it.
\answer
The sample space consists of all possible $(X,Y)$ pairs, where $X$ is
the number on the red die and $Y$ the number on the white.  $X$ and
$Y$ are both integers between 1 and 6, so there are 36 possible
outcomes.  If the dice are fair, then each outcome has probability
$1/36$.
\end{exercise}

\begin{exercise}
What is the sample space of $X$, the number of heads in two spins of a
fair coin?  (Hint: the events $HT$ and $TH$ both yield the same number
of heads.)
\answer
The sample space is $\{0, 1, 2\}$.
\end{exercise}

\subsection{Averages and expectations}
\label{sec.expectation}

There are two ways to calculate an average, one using relative
frequencies and the other using the method you learned in grade
school.  Take for example the following toy data set: $[0, 0, 0, 1, 1,
2, 2, 2]$.  There are 8 numbers here, and their sum is also 8, so
their average is 1.  Let us now repeat this calculation using the
relative frequencies of these data, which are shown in
Table~\ref{tab.toynum}.

\begin{table}
\caption{Frequency distribution of toy data.  $f_i$ is the relative
  frequency of value $i$.}
\label{tab.toynum}

\centering
\begin{tabular}{cc}
       &  Relative\\
Value  & frequency\\
\hline
    0  &  $f_0=3/8$\\
    1  &  $f_1=2/8$\\
    2  &  $f_2=3/8$
\end{tabular}
\end{table}

In ``sigma notation,'' the average (or mean) is\footnote{If you are
  unfamiliar with the ``$\Sigma$'' symbol, see
  appendix~\ref{sec.sigma}.}
\begin{equation}
\hbox{mean} = \sum_{x} x f_x
\label{eq.mean}
\end{equation}
where $x$ is a sample value and $f_x$ is the relative frequency of
that value.  Using the relative frequencies from
Table~\ref{tab.toynum}, this gives $(0 \times 3/8) + (1 \times 2/8) +
(2 \times 3/8) = 1$, just as we calculated using the grade-school
method.  If the data set is large, relative frequencies often make the
problem easier.

\begin{exercise}
Calculate the mean of the numbers 1, 1, and 3 using both methods.
\answer
The mean  is $5/3$, since there are 3 numbers that sum to 5.  Using
relative frequencies, the problem becomes
$1 \times  (2/3) + 3\times (1/3) = 2/3 + 3/3 = 5/3$.
\end{exercise}

As the sample size grows large, the relative frequencies in
Eqn.~\ref{eq.mean} get closer and closer to the corresponding
probabilities.  As this happens, the mean converges toward what is
called the \emph{expected value} of the corresponding random variable.
The expected value of $X$ is written ``$E[X]$'' and is calculated just
as you calculate a mean:
\begin{equation}
E[X] = \sum_x x \Pr[X=x]
\label{eq.expectation}
\end{equation}
Note the similarity between Eqns.~\ref{eq.mean}
and~\ref{eq.expectation}.  Relative frequencies $(f_x)$ have been
replaced by probabilities $(\Pr[X=x])$; the two formulas are otherwise
the same.

\paragraph{Example}  If you spin a coin twice, the number $X$ of heads
must equal either 0, 1, or 2.  If the coin is fair, then these events
have probabilities $1/4$, $1/2$, and $1/4$, respectively.  The
expectation of $X$ is
\[
E[X] = (0\times 1/4) + (1\times 1/2) + (2\times 1/4) = 1
\]
\begin{exercise}
What is the expected value of $X^2$?
\answer
\[
E[X^2] = (0\times 1/4) + (1\times 1/2) + (4\times 1/4) = 1.5
\]
\end{exercise}

\begin{exercise}
What is the expected value of $X+X^2$?
\answer
\begin{eqnarray*}
E[X + X^2]&=& ((0+0)\times 1/4) + ((1+1)\times 1/2)\\
 &&\mbox{} + ((2+4)\times 1/4)\\
& =& 2.5
\end{eqnarray*}
Compare this answer to that of the preceding exercise, and you will
see that $E[X + X^2] = E[X] + E[X^2]$.
\end{exercise}

Expected values have several properties that make them easy to
manipulate.  If $X$ and $Y$ are random variables and $a$ is a
constant, then
\begin{eqnarray}
E[a] &=& a\label{eq.E[a]}\\
E[aX] &=& aE[X]\label{eq.E[aX]}\\
E[X + Y] &=& E[X] + E[Y]\label{eq.E[X+Y]}
\end{eqnarray}
If $X$ and $Y$ are statistically independent, it is also true that
\begin{equation}
E[XY] = E[X]E[Y]\label{eq.E[XY]}
\end{equation}
These are really properties of averages.  They apply to expectations
because expectations are a kind of average.  Rather than proving them,
I will illustrate them using averages.

\paragraph{The average of a constant}
The average of 4, 4, and 4 is 4.  This is why $E[a]=a$ when $a$ is a
constant.

\paragraph{The average of $aX$}  Start with the numbers 1, 3, and 5.  The
sum of these numbers is 9, and their average is 3.  Now multiply each
number by a constant $a$.  The average of the resulting numbers is
\[
(a + 3a + 5a)/3 = a\times(1+3+5)/3 = 3a
\]
which is $a$ times the original average.  This illustrates that $E[aX]
= aE[X]$.

\paragraph{The average of $X+Y$}  Consider the following table
\[
\begin{array}{rccc}
&X & Y & X+Y\\
\cline{2-4}\cline{2-4}
&0 & 2 &   2\\
&2 & 3 &   5\\
&7 & 7 &  14\\
\cline{2-4}  
\hbox{sum}& 9 &12 &  21\\
\hbox{average}& 3 &4 &  7\\
\end{array}
\]
The average of $X$ is 3 and that of $Y$ is 4.  The sum of these is 7,
which is also the average of $X+Y$.  This illustrates that
$E[X+Y] = E[X] + E[Y]$.

\begin{exercise}
What is $E[3]$?
\answer
$E[3]=3$
\end{exercise}

\begin{exercise}
If $E[X]=5$, then what is $E[2X]$?
\answer
$E[2X] = 10$
\end{exercise}

\begin{exercise}
If $E[X]=5$ and $E[Y]=6$, then what is $E[2X + 3Y]$?
\answer
$E[2X + 3Y] = 10 + 18 = 28$
\end{exercise}

\begin{exercise}
What is $E[aX + bY^2]$, assuming that $a$ and $b$ are constant and the
values of $E[X]$ and $E[Y^2]$ are unknown?
\answer
$E[aX + bY^2] = aE[X] + bE[Y^2]$
\end{exercise}

\begin{exercise}
Prove that $E[(X + Y)^2]=E[X^2] + 2E[X]E[Y] + E[Y^2]$
if $X$ and $Y$ are statistically independent. Hint:
Begin by expanding $(X+Y)^2 = X^2 + 2XY + Y^2$. Then use
Eqns.~\ref{eq.E[X+Y]} and~\ref{eq.E[XY]}.
\answer
First expand the squared term:
\[
E[(X + Y)^2] = E[X^2 + 2XY + Y^2]
\]
Next, use Eqn.~\ref{eq.E[X+Y]} to turn the expectation of a sum into a
sum of expectations:
\[
E[X^2 + 2XY + Y^2] = E[X^2] + E[2XY] + E[Y^2]
\]
Finally, re-express the middle term using
Eqns.~\ref{eq.E[aX]} and~\ref{eq.E[XY]}:
\[
E[X^2] + E[2XY] + E[Y^2] = E[X^2] + 2E[X]E[Y] + E[Y^2]
\]
\end{exercise}

\subsection{Variance}
\label{sec.variance}

We are often interested in quantities that vary.  There are several
ways to measure variation, of which the most important is the
\emph{variance}.  We can measure variance either in a data set or in a
random variable.  The procedures are similar so let's begin with data.

The variance is the \emph{average squared difference from the mean}.
Take for example the numbers 10, 12, 10, and 8.  Their mean is 10, so
their variance is $V = ((10 - 10)^2 + (12-10)^2 + (10-10)^2 +
(8-10)^2)/4 = 2$. There are several ways to write the variance,
including
\begin{eqnarray}
V &=& N^{-1} \sum_i (x_i - \bar x)^2\label{eq.variance.a}\\
  &=& \sum_x (x-\bar x)^2 f_x\label{eq.variance.b}\\
  &=& \sum_x x^2 f_x - \bar x^2\label{eq.variance.c}
\end{eqnarray}
Here, $\bar x$ is the average of the $x_i$ and $N^{-1}$ means $1/N$.

\begin{exercise}
Verify that the formulas Eqn.~\ref{eq.variance.a}--\ref{eq.variance.c}
are equivalent, using the numbers 10, 12, 10, and 8.
\answer
%
The text used Eqn~\ref{eq.variance.a} to calculate that $m=10$ and
$V=2$.  For the other versions, we need relative frequencies: $f_8=1/4$,
$f_{10}=1/2$, and $f_{12}=1/4$.  Eqn.~\ref{eq.variance.b} gives
\begin{eqnarray*}
V&=&(1/4)(8-10)^2 + (1/2)(10-10)^2 \\
 && \mbox{} + (1/4)(12-10)^2\\
 &=& 1 + 0 + 1 = 2
\end{eqnarray*}
For the Eqn.~\ref{eq.variance.c}, we need
\begin{eqnarray*}
\sum x^2 f_x &=& 8^2/4 +10^2/2 + 12^2/4\\
 &=& 16 + 50 + 36 = 102
\end{eqnarray*}
We also need $m^2 = 10^2 = 100$.  Subtracting gives $V = 102 - 100 =
2$.
\end{exercise}

\begin{exercise}
What are the mean and variance of the numbers 3, 9, 15, and 8?
\answer
The mean and variance are 8.75 and 18.1875.
\end{exercise}

If $X$ is a random variable (rather than data), its variance is
\begin{equation}
V[X] = E\left[ (X - E[X])^2 \right]\label{eq.V1}
\end{equation}
Note the similarity between this expression and
Eqn.~\ref{eq.variance.b}.  The variance can also be written in either
of the following ways:
\begin{eqnarray}
V[X] &=& E\bigl[ X^2 \bigr] - E[X]^2\label{eq.V2}\\
     &=& E\bigl[ X(X - E[X]) \bigr]\label{eq.V3}
\end{eqnarray}

\begin{exercise}
Suppose that the random variable $X$ takes the values 0, 1, and 2 with
probabilities 1/3, 1/2, and 1/6.  What are the mean and variance of
$X$?
\answer
The mean is
\[
E[X] = 0 \times 1/3 + 1\times 1/2 + 2 \times 1/6 = 0.833.
\]
The variance is
\begin{eqnarray*}
V[X] &=& (0-0.833)^2 \times 1/3\\
&&\mbox{} + (1-0.833)^2\times 1/2\\
&&\mbox{} + (2-0.833)^2 \times 1/6\\
&=& 0.472.
\end{eqnarray*}
\end{exercise}

\begin{exercise}
In a previous exercise, you verified that
Eqns.~\ref{eq.variance.b}--\ref{eq.variance.c} were equivalent, using
as data the numbers 8, 10, 10, and 12.  This illustrates that
Eqns.~\ref{eq.V1} and~\ref{eq.V2} are equivalent, since they are the
same formulas in a different notation.  Now use the same method and
data to verify the equivalence of Eqn.~\ref{eq.V3}.
%
\answer
%
In the data, the relative frequencies are $f_8=1/4$, $f_{10} = 2/4$,
and $f_{12} = 1/4$.  The mean is $10$.  According to Eqn.~\ref{eq.V3},
the variance is
\begin{eqnarray*}
V &=& (1/4)\times 8\times(8-10)\\
 &&\mbox{} + (1/2)\times 10\times (10-10)\\
 &&\mbox{} + (1/4)\times 12\times (12-10)\\
 &=& (1/4)\times (-16)\\
 &&\mbox{} + (1/2)\times 0\\
 &&\mbox{} + (1/4)\times 24\\
 &=& -4 + 6 = 2\\
\end{eqnarray*}
\end{exercise}

\begin{exercise}
Prove that if $a$ is a constant and $X$ a random variable, then $V[aX]
= a^2 V[X]$.
\answer
First, $E[aX] = aE[X]$ by equation~\ref{eq.E[aX]}.  Next,
\begin{eqnarray*}
V[aX] &=& E[(aX - aE[x])^2]\\
 &=& E[a^2(X - E[x])^2]\\
 &=& a^2E[(X - E[x])^2] \qquad \hbox{using (\ref{eq.E[aX]}) again}\\
 &=& a^2 V[X]
\end{eqnarray*}
\end{exercise}

\subsection{Covariance}
\label{sec.covariance}

In addition to variation, we are often interested in the
\emph{relationship} between variables.  Fig.~\ref{fig.scatplot}
illustrates this idea.  The scatterplot on the left illustrates a
positive relationship: one in which $Y$ tends to increase when $X$
increases.  On the right we see the opposite case: $Y$ decreases as
$X$ increases, so the relationship is negative.  The two relationships
differ not only in direction but also in strength.  The one on the
right is the stronger of the two.

\begin{figure}
{\centering\setlength{\headingtoplotskip}{1.5ex}
\mbox{\beginpicture
%%%%%%%%%% Plot 1 %%%%%%%%%%%%%%%%%%%%%%%%%%%%%%%%%%%%%%%%%%%%
  \setcoordinatesystem units <0.2056\columnwidth,0.2056\columnwidth> 
      point at 1.25 0
  \setplotarea x from -0.5 to 0.57,
               y from -0.5 to 0.57
  \axis bottom label {$X$} /
  \axis left label {$Y$} /
  \axis right /
  \axis top /
\plotheading{\lines{Weak\cr positive\cr relationship}}
\multiput {$\bullet$} at 
     0.202     -0.073
     0.281     -0.150
     0.079      0.109
    -0.423     -0.145
     0.018     -0.326
    -0.323      0.144
    -0.072     -0.220
     0.009      0.128
     0.078      0.054
     0.351      0.192
     0.200     -0.015
    -0.119     -0.387
     0.109      0.290
     0.114      0.360
     0.337      0.355
    -0.061     -0.097
     0.072      0.074
     0.202      0.070
    -0.187     -0.239
     0.390      0.504
/
%%%%%%%%%% Plot 2 %%%%%%%%%%%%%%%%%%%%%%%%%%%%%%%%%%%%%%%%%%%%
  \setcoordinatesystem units <0.22\columnwidth,0.22\columnwidth> 
      point at -0.25 0
  \setplotarea x from -0.5 to 0.5,
               y from -0.5 to 0.5
  \axis bottom label {$X$} /
  \axis right label {$Y$} /
  \axis left /
  \axis top /
\plotheading{\lines{Strong\cr negative\cr relationship}}
\multiput {$\bullet$} at 
     0.430     -0.260
     0.372     -0.378
     0.301     -0.145
     0.217     -0.155
     0.185     -0.178
    -0.119     -0.206
    -0.374      0.102
    -0.158      0.055
    -0.203      0.077
    -0.077      0.133
    -0.036      0.310
     0.113     -0.177
     0.145     -0.006
    -0.267      0.141
     0.261     -0.120
    -0.173      0.250
    -0.203     -0.027
    -0.035     -0.068
     0.007      0.270
    -0.015      0.056     
/
\endpicture}
\\}
\caption{Examples of positive and negative relationship
  between variables}
\label{fig.scatplot}
\end{figure}

These two ideas---strength and direction of relationships---come up
all the time, and we need ways to measure them.  Several statistics
have been invented for this purpose, but most are based on the same
idea: if the relationship is positive, the $X$ and $Y$ will often be
on the same side of their respective means, so $(X-E[X])(Y-E[Y])$ is
positive for most $(X,Y)$ pairs.  For a negative relationship, this
product tends to be negative.  We can measure a relationship by
\begin{equation}
C[X,Y] = E\big[ (X-E[X])(Y-E[Y]) \big]
\label{eq.C1}
\end{equation}
which is called the \emph{covariance}.  It is positive for positive
relationships but negative for negative ones.  It is far from zero for
strong relationships but near zero for weak ones.  Thus, it measures
both the strength and direction of relationships.

The covariance, like the variance, can be written in several different
ways:
\begin{eqnarray}
C[X,Y] &=& E[XY] - E[X] E[Y] \label{eq.C2}\\
     &=& E[ X(Y - E[Y]) ]\label{eq.C3}
\end{eqnarray}
In calculations, it is often most convenient to use (\ref{eq.C2}).

\begin{table}
\caption{A bivariate probability distribution}
\label{tab.bivdist}

{\centering\begin{tabular}{cccc}
$X$ & $Y$ & $P_{XY}$ & $(X-E[X])(Y-E[Y])$\\
\hline
0 & 0 & 0.4 & +0.25\\
0 & 1 & 0.1 & --0.25\\
1 & 0 & 0.1 & --0.25\\
1 & 1 & 0.4 & +0.25\\
\end{tabular}\\
\begin{minipage}{0.9\columnwidth}
Here, $P_{XY}$ is the probability of the pair $(X,Y)$, and $E[X] =
E[Y] = 0.5$.
\end{minipage}\\}
\end{table}

To get familiar with covariances, consider the two random variables in
Table~\ref{tab.bivdist}.  The probabilities imply that we should see
lots of $(X,Y)$ pairs like $(0,0)$ or $(1,1)$ but few like $(0,1)$ or
$(1,0)$.  For the most part then, $X$ and $Y$ will vary in the same
direction, and their relationship should be positive.

The first step is to calculate $E[X]$ and $E[Y]$.  I leave the details
to you, but you should find that both equal 0.5.  Next, calculate
$(X-0.5)(Y-0.5)$ for each $(X,Y)$ pair.  These values appear in the
right column of the table.  Finally, we take the expectation by
multiplying column~3 by column~4 and summing the results.  (If this
seems mysterious, consult Eqn.~\ref{eq.expectation}.)  Here is the
calculation in detail:
\begin{eqnarray*}
C[X, Y] &=& \sum_{x,y} P_{xy} (x-0.5)(y - 0.5) \\
   &=& (0.4 \times 0.25) - (0.1\times 0.25) \\
   && \mbox{} - (0.1\times 0.25) + (0.4 \times 0.25)\\
   &=& 0.15
\end{eqnarray*}
The covariance is positive, just as expected.

\begin{exercise}
  In Table~\ref{tab.bivdist}, suppose that the $P_{XY}$ values were
  0.3, 0.2, 0.2, and 0.3.  First use your intuition to figure out what
  this would do to the relationship between $X$ and $Y$.  Is it still
  positive?  Does it become stronger or weaker?  Then calculate
  $C[X,Y]$ to check your intuition.
%
  \answer With the new probabilities, we expect fewer $(0,0)$ and
  $(1,1)$ pairs but more $(0,1)$ and $(1,0)$.  There is still a
  tendency for $X$ and $Y$ to vary in the same direction, but that
  tendency is weaker.  In other words, the relationship is weaker but
  still positive.  The covariance turns out to be $C[X,Y] = 0.05$.
\end{exercise}

\begin{exercise}
Construct a bivariate probability distribution in which the
relationship between $X$ and $Y$ is negative, and use it to calculate
$C[X,Y]$.
\answer
One way is to replace the $P_{XY}$ values in Table~\ref{tab.bivdist}
with 0.1, 0.4, 0.4, and 0.1.  This yields $C[X,Y] = -0.15$.
\end{exercise}

\begin{exercise}
  So far we have been dealing with the covariance between random
  variables.  To deal with data, we need formulas analogous to
  Eqns,~\ref{eq.variance.a}--\ref{eq.variance.c}.  Write these
  formulas down and use them to estimate the covariance of $X$ and $Y$
  in the following collection of $(X,Y)$ values: (0,0), (0,0), (1,0),
  (0,1), (1,1), (1,1).
\answer
The corresponding formulas are
\begin{eqnarray}
C[X,Y] &=& N^{-1} \sum_i (x_i - \bar x)(y_i - \bar y)\label{eq.cov.a}\\
  &=& \sum_{x,y} (x-\bar x)(y-\bar y) f_{xy}\label{eq.cov.b}\\
  &=& \sum_{x,y} xy f_{xy} - \bar x\bar y\label{eq.cov.c}
\end{eqnarray}
where $f_{xy}$ is the frequency of $(X,Y)$ pairs in the data for
which $X=x$ and $Y=y$.  Using any of these formulas, the given data
imply that $C[X,Y] = 0.0833$.
\end{exercise}

You may have been wondering what the magnitude of the covariance
really means.  If the covariance is 0.25, is the relationship a strong
one or a weak one?  It is impossible to say.  The problem is that the
magnitude of $C[X,Y]$ depends not only on the strength of the
relationship but also on the units of measurement. For example, the
value of $C[X,Y]$ would change if we decided to measure $X$ and $Y$ in
millimeters rather than centimeters. To avoid such effects, data
analysts often normalize their covariances to obtain what is called a
\emph{correlation coefficient}.  This topic is not covered here.

\section{Probability distributions}

By now you are familiar with distributions of relative frequencies,
and you know that as $N$ grows large, relative frequencies converge to
probabilities.  It is thus easy to see that a frequency distribution
will converge to a distribution of probabilities.
Section~\ref{sec.urnmodel} above described a model of Kerrich's urn
experiment, which led to the probabilities in the right column of
Fig.~\ref{fig.tree}.  These constitute a \emph{probability
  distribution}.

In that case, it was easy to list the events in the sample space and
their probabilities.  This is the simplest and most obvious way to
describe a probability distribution.  There is also another approach,
which involves thinking of probability distributions as functions.  A
function, you may remember, is a translation rule.  The function $f(x)
= x^2$ for example would translate 2 into 4, or 3 into 9.  Similarly,
the probability distribution from the urn model in Fig.~\ref{fig.tree}
translates the event $RR$ into the probability $1/4$, and $RG$ into
$1/2$.  It is thus a function too. Every probability distribution is a
function that translates events into probabilities.

\begin{exercise}
What is the probability distribution of the number $X$ of heads
in two spins of a fair coin?  (You wrote down the sample space in a
previous exercise.)
\answer
The sample space is $\{0, 1, 2\}$; the corresponding probabilities
are $1/4$, $1/2$, and $1/4$.
\end{exercise}

In the case of random variables, we are translating numbers into
numbers, and the function can often be expressed in mathematical
form. This chapter will cover several probability functions that are
widely used in science.  These fall into two categories:
\emph{discrete} and \emph{continuous}.  The random variables that we
have discussed thus far are discrete, but it will be easier to define
continuous ones first.

A continuous random variable is one whose sample space is a continuum,
such as space or time.  The remarkable thing about a continuum is that
between any two points there is an infinity of other points.  In
biology, continuous random variables are used to model such things as
lifespan and body size.  Section~\ref{sec.continuousfun} will describe
the methods to describe them.

If a sample space is not continuous, it is discrete.  This does not
make it finite.  For example, there is an infinity of integers, but
they are not continuous.  There is no integer, for example, between 2
and 3.  Discrete random variables are used to describe things that you
can count: the number of heads in $N$ spins of a coin, and so forth.
The methods used to describe these random variables are described
below in section~\ref{sec.discretefun}.


\subsection{Discrete random variables}
\label{sec.discretefun}

For discrete random variables, the \emph{probability distribution
  function}, $P_x$, gives the probability associated with each point
$x$ in the sample space.  In genetics and many other parts of biology,
the discrete distributions used most often are the binomial and the
Poisson.

\subsubsection{The binomial distribution}

Kerrich's coin experiment is a good example of a binomial experiment.
There are $N$ independent trials (in Kerrich's case, $N=10,000$), and
in each trial we observe some event with probability $p$.  (In
Kerrich's case, the event was ``heads'' and $p$ was apparently close
to $1/2$.)  The number $x$ of events is a binomial random variable.
Its distribution function is
\begin{equation}
P_x = \binom{N}{x} p^x (1-p)^{N-x}
\label{eq.binomial}
\end{equation}
Here, $\binom{N}{x}$ is pronounced ``$N$ choose $x$'' and represents
the number of ways of obtaining $x$ heads and $N-x$ tails.  For
example, there are are two ways ($HT$ and $TH$) of obtaining one head
in two spins, so $\binom{2}{1} = 2$.  The expression $p^x(1-p)^{N-x}$
is the probability of obtaining any given sequence of $x$ heads and
$N-x$ tails.  The binomial distribution is illustrated in
Fig.~\ref{fig.binom} for $N=20$ and two values of $p$.  The mean and
variance of the binomial distribution are $E[X] = Np$ and $V[X] =
Np(1-p)$.

% -*-latex-*-
\begin{figure}
\centering
\subfloat[$N=20, p=0.2$]{\mbox{\beginpicture
\setcoordinatesystem units <0.04in, 5in>
\setplotarea x from 0 to 21, y from 0 to 0.21
\axis left shiftedto x=-1 label {$\Pr[x]$}
  ticks  numbered from 0.0 to 0.2 by 0.1 /
\axis bottom label {$x$}
  ticks withvalues 0 10 20 / at 0.5 10.5 20.5 / /
%\shaderectangleson 
\sethistograms
\plot 
% Binomial distribution: N=20, p=0.2
 0 0
 1 0.01152921505
 2 0.05764607524
 3 0.1369094287
 4 0.2053641430
 5 0.2181994019
 6 0.1745595216
 7 0.1090997010
 8 0.05454985049
 9 0.02216087676
10 0.007386958920
11 0.002031413703
12 0.0004616849325
13 0.00008656592484
14 0.00001331783459
15 0.000001664729324
16 0.0000001664729324
17 0.00000001300569784
18 0.0000000007650410496
19 0.00000000003187671040
20 0.00000000000083886080
21 0.000000000000010485760
/
\endpicture}}~~%
\subfloat[$N=20, p=0.5$]{\mbox{\beginpicture
\setcoordinatesystem units <0.04in, 5in>
%\setcoordinatesystem units <0.12in, 15in>
\setplotarea x from 0 to 21, y from 0 to 0.21
% Add a little horizontal space to keep the
% label on one line.
\axis left invisible label \hspace{3pt} /
%  ticks  numbered from 0.0 to 0.2 by 0.1 /
\axis bottom label {$x$}
  ticks withvalues 0 10 20 / at 0.5 10.5 20.5 / /
%\shaderectangleson 
\sethistograms
\plot 
% Binomial distribution: N=20, p=0.5
0 0
1 .0000009536743164
2 .00001907348633
3 .0001811981201
4 .001087188721
5 .004620552062
6 .01478576660
7 .03696441650
8 .07392883301
9 .1201343536
10 .1601791382
11 .1761970520
12 .1601791382
13 .1201343536
14 .07392883301
15 .03696441650
16 .01478576660
17 .004620552062
18 .001087188721
19 .0001811981201
20 .00001907348633
21 .0000009536743164
/
\endpicture}}
\caption{Binomial distribution functions.}
\label{fig.binom}
\end{figure}



The form of the binomial distribution function is not hard to
understand.  To see why, consider the probability of the following
outcome:
\[
\overbrace{
\overbrace{H\;H\;\ldots H}^{\hbox{$x$ tosses}}
\;
\overbrace{T\;T\;\ldots T}^{\hbox{$N-x$ tosses}}
}^{\hbox{$N$ tosses}}
\]
Here, the coin has been tossed $N$ times, producing heads on the first
$x$ tosses and tails on the remaining $N-x$.  Since each heads is an
event of probability $p$ and each tails is an event of probability
$1-p$, the probability of the outcome observed on this sequence of
tosses is $p^x (1-p)^{N-x}$.  But this is not the only outcome that
would yield $x$ heads in $N$ tosses.  No matter what order the heads
and tails appear in, if there are $x$ heads in $N$ tosses we have
observed an event of probability $p^x (1-p)^{N-x}$.  If we don't know
the order in which the heads and tails appear, we have to sum across
all the ways in which $x$ heads and $N-x$ tails can be re-ordered.
This sum accounts for the term $\binom{N}{x}$ in
equation~\ref{eq.binomial}.

\paragraph{Example} Population geneticists use the binomial
distribution to model the random component of evolutionary
change---genetic drift.  Suppose that in the parental generation there
are $N$ diploid individuals.  At each diploid locus, the population
contains $2N$ genes, of which a fraction $p$ are copies of allele
$A_1$ and $1-p$ are copies of $A_2$.  The model assumes that each of
the $2N$ genes in the offspring generation is (in effect) drawn at
random with replacement from an urn with $2Np$ copies of allele $A_1$
and $2N(1-p)$ copies of $A_2$.  The number of copies of $A_1$ among
the offspring is binomial with mean $2Np$ and variance $2Np(1-p)$.

\begin{exercise}
Suppose that a population contains 1000 diploid individuals and that
the relative frequency of $A_1$ is $1/1000$.  If this population
produces 1000 offspring, what is the probability (under the binomial
model) that allele $A_1$ will not be represented among the offspring?
\answer
The allele disappears with probability
\[
P_0 = \binom{2000}{0} p^0 (1-p)^{2000} \approx 0.135
\]
In this calculation, there is no need to calculate $\binom{2000}{0}$;
it must equal 1 because there is only 1 way to choose 0 of something.
In addition, $p^0 = 1$ because \emph{anything} raised to the zeroth
power equals 1.  The only part that needs calculating is
$(1-p)^{2000}$.
\end{exercise}

\subsubsection{The Bernoulli distribution}

The binomial distribution has an important special case: that in which
we observe just one trial.  For example, suppose we toss a coin a
single time, and let $x=1$ if the result is ``heads'' or~0 if the
result is ``tails.''  The distribution function has just two values:
\begin{equation}
P_x = \cases{p & \hbox{if $x=1$}\cr
             1-p & \hbox{if $x=0$}\cr}
\label{eq.bernoulli}
\end{equation}
This is just a special case of Eqn.~\ref{eq.binomial}.

\begin{exercise}
Prove Eqn.~\ref{eq.bernoulli} by substituting $N=1$ into
Eqn.~\ref{eq.binomial}.
\answer
For the Bernoulli distribution, there is only one event.
Consequently, $N=1$ and Eqn.~\ref{eq.binomial} becomes
\[
P_x = \frac{1!}{1! \times 0!} p^x (1-p)^{1-x}
\]
Recall that $0! = 1! = 1$.  (See appendix~\ref{sec.factorial} for
details.)  Consequently, $1!/(1! \times 0!) = 1$ and drops out of the
equation.  If $x=0$, then $p^x = p^0=1$, and this term drops out.
Eqn.~\ref{eq.binomial} becomes $P_0 = 1-p$.  On the other hand, if
$x=1$, then $(1-p)^{1-x} = (1-p)^0 = 1$ and drops out.  We are then
left with $P_1 = p$.
\end{exercise}

It is also easy to derive the mean and variance of the Bernoulli
distribution, simply by setting $N=1$ in the corresponding formulas
for the Binomial: $E[X] = p$, and $V[X] = p(1-p)$.

\begin{exercise}
Consider the experiment of tossing a fair coin a single time, and
recording $X=1$ if the result is ``heads,'' or $X=0$ if ``tails.''
What are the mean and variance of this random variable?
\answer
$E[X] = 1/2$ and $V[X] = 1/4$.
\end{exercise}

\begin{exercise}
Consider the experiment of drawing a copy of a single gene at random
from some population, and scoring $X=1$ if the result is allele $A_1$,
or $X=0$ otherwise.  If allele $A_1$ has frequency $p$, then what are
the mean and variance of random variable $X$?
\answer
Since $A_1$ has frequency $p$, that is also the probability that we
have drawn a copy of this allele.  It follows that $E[X] = p$ and
$V[X] = p(1-p)$.
\end{exercise}

\subsubsection{The Poisson distribution}

This distribution comes up a lot when we are interested in counts.
How many prey items will a forager encounter during one hour?  How
many gamma rays will strike the tube of a Geiger counter in one
minute?  How many rain drops will strike your roof during one second,
in the middle of a storm?  How many mutations occurred along the
lineage that connects you to your mother's mother's mother's
mother's\ldots mother, who lived 10,000 years ago?  In each case, if
the events in question are independent and occur at a constant rate,
then the random variable is Poisson.

% Data from von Bortkiewicz's (1898) 
\mbox{\beginpicture
\setcoordinatesystem units <0.2in, 0.028cm>
\setplotarea x from 0 to 6, y from 0 to 109
\axis left shiftedto x=-0.25 
  ticks withvalues 0.00 0.25 0.50 / at 0 50 100 / /
\axis bottom label {Deaths per corps per year}
  ticks withvalues 0 2 4 / at 0.5 2.5 4.5 / /
% Observed
\multiput {$\bullet$} at
0.5  109
1.5  65
2.5  22
3.5   3
4.5   1
5.5   0
/
\sethistograms
% Poisson
\plot
   0  0
   1  108.7 
   2  66.3 
   3  20.2
   4  4.1
   5  0.6
   6  0.1
/
\endpicture}


At the end of the 19th century, Ladislaus von Bortkiewicz fit the
Poisson to some peculiar data involving deaths in the Prussian Army.
In those days, the army's supply train involved wagons pulled by
mules.  Mules have tempers and are seldom eager to pull wagons.  Every
now and then, a soldier was killed by the kick of a mule.  Military
records list the number of soldiers killed this way in each year
within each army corps.  These data, as tabulated by von Bortkiewicz,
are shown as bullets ($\bullet$) in Fig.~\ref{fig.kick}.  The
histogram shows the corresponding Poisson distribution function.

The distribution has one parameter, the mean $(\lambda)$.  The
probability that $x$ events occur is
\begin{equation}
P_x = \frac{\lambda^x e^{-\lambda}}{x!}
\label{eq.poisson}
\end{equation}
Here, $e\approx 2.718$ is the base of natural logarithms; $x!$ is
pronounced ``$x$ factorial'' and represents the number of ways of
rearranging $x$ items.  (See appendix~\ref{sec.factorial} for
details.)  The variance of the Poisson is the same as the mean:
$E[X] = V[X] = \lambda$.

% -*-latex-*-
\begin{figure}
\centering
\subfloat[$\lambda=2$]{\mbox{\beginpicture
\setcoordinatesystem units <0.04in, 5in>
\setplotarea x from 0 to 21, y from 0 to 0.21
\axis left shiftedto x=-1 label {$\Pr[x]$}
  ticks  numbered from 0.0 to 0.2 by 0.1 /
\axis bottom label {$x$}
  ticks withvalues 0 10 20 / at 0.5 10.5 20.5 / /
%\shaderectangleson 
\sethistograms
\plot 
% Poisson distribution, lambda=2
0 0
1 .1353352832
2 .2706705664
3 .2706705664
4 .1804470442
5 .09022352214
6 .03608940886
7 .01202980295
8 .003437086558
9 .0008592716393
10 .0001909492532
11 .00003818985064
12 .000006943609207
13 .000001157268201
14 .0000001780412617
15 .00000002543446596
16 .000000003391262128
17 .0000000004239077661
18 .00000000004987150188
19 .000000000005541277987
20 .0000000000005832924196
21 .00000000000005832924196
/
\endpicture}}~~\subfloat[$\lambda=10$]{\let\put\pictexput
\mbox{\beginpicture
\setcoordinatesystem units <0.04in, 5in>
\setplotarea x from 0 to 21, y from 0 to 0.21
%\axis right shiftedto x=22 
%  ticks  numbered from 0.0 to 0.2 by 0.1 /
\axis bottom label {$x$}
  ticks withvalues 0 10 20 / at 0.5 10.5 20.5 / /
\plotheading{$\lambda=10$}
%\shaderectangleson 
\sethistograms
\plot 
% Poisson distribution, lambda=10
0 0
1 .00004539992976
2 .0004539992976
3 .002269996488
4 .007566654962
5 .01891663740
6 .03783327480
7 .06305545801
8 .09007922571
9 .1125990321
10 .1251100357
11 .1251100357
12 .1137363961
13 .09478033010
14 .07290794623
15 .05207710445
16 .03471806963
17 .02169879352
18 .01276399619
19 .007091108993
20 .003732162628
21 .001866081314
/
\endpicture}
\let\put\latexput
}
\caption{Poisson distribution functions.}
\label{fig.poisson}
\end{figure}


The shape of the Poisson varies in response to the parameter
$\lambda$, as shown in Fig.~\ref{fig.poisson}.  When $\lambda$ is
small (as shown in the left-hand graph), the distribution function is
asymmetric, with a high left shoulder.  When $\lambda$ is large (as
shown in the right-hand graph), the function becomes symmetrical.  For
large $\lambda$, the Poisson is very nearly identical to the normal
distribution (discussed below).

Because $P_x$ depends in such a simple fashion on the mean, $\lambda$,
the Poisson is among the easiest distributions to fit to data.  For
example, in von Bortkiewicz's data there are on average 0.61 mule-kick
deaths per corps per year.  To fit these data to the Poisson, we
simply set $\lambda = 0.61$.  That is all there is to it.  With
$\lambda$ known we can calculate numerical probabilities.  For
example, the probability of a single death is $P_1 = \lambda
e^{-\lambda} = 0.331$.  In other words, we expect a death in any given
corp 1 year in 3.  There were 200 corps-years in von Bortkiewicz's
data, so the number of these with a single mule-kick death should have
been $0.331\times 200 = 66.2$.  There were 65 in the real data.

\begin{exercise}
Use the same procedure to calculate the expected number of corps-years
with 2 mule-kick deaths.  Compare this expected value to the real
value, 22.
%
\answer For $x=2$ and $\lambda = 0.61$, the Poisson formula gives $P_2
= 0.101$.  We therefore expect to see about $0.101\times 200 = 20.2$
corps-years with 2 mule-kick deaths.  This is close to the real number
of 22.
\end{exercise}


\paragraph{Example} Consider the lineage that connects me to an
ancestor who lived $t$ generations ago.  The expected number of
mutations along that lineage is $\lambda = ut$, where $u$ is the
mutation rate.  The probability that $x$ mutations occurred is given
by the Poisson distribution function.  If $u=10^{-3}$ and $t=2000$
generations, then $\lambda=2$.  The probability that 1 mutation
occurred is $\lambda e^{-\lambda} = 0.271$.

\begin{exercise}
What is the probability that no mutation occurred?
\answer
The probability of no mutation is
$P_0 = e^{-2} \approx 0.135$.
Note that this answer is identical to that of the preceding exercise.
This illustrates the fact that, when $N$ is large and $p$ is small,
the Poisson is a good approximation to the binomial.
\end{exercise}

\begin{exercise}
What is the probability that at least one mutation occurred?
\answer
The probability of ``at least one mutation'' is $\Pr[X>0]$.
There are two ways to think about this problem. The hard way sums
across all non-zero entries of the Poisson distribution:
\[
\Pr[X>0] = \sum_{x=1}^\infty \frac{\lambda^x e^{-\lambda}}{x!}
\]
As I said, that is the hard way and is not recommended. The easy
solution proceeds from the observation that all probability
distributions sum to~1. For a non-negative random variable such as the
Poisson, this implies that $\Pr[X=0] + \Pr[X>0] = 1$. Thus,
\[
\Pr[X>0] = 1 - \Pr[X=0] = 1 -e^{-\lambda}
\]
In the current question, $\lambda=2$, so $\Pr[X>0] \approx 0.865$.
\end{exercise}

\subsection{Continuous random variables}
\label{sec.continuousfun}

Board games often come with a device for generating random numbers.
One type consists of a flat piece of cardboard to which a needle is
attached.  You spin the needle, and it ends up pointing in a random
direction.  In the real world, these devices probably have
irregularities that make the needle more likely to land in some
positions than others.  But let's ignore that.  In our hypothetical
world, the needle is equally likely to point in any direction.  What
is the probability that it stops exactly 87.729543328 degrees
clockwise of where it started?

This is a trick question.  The problem is that there is a continuum of
possible outcomes, all equally probable.  The probability of any
particular outcome, such as 87.729543328, is zero.  Why?  With an
infinity of equally probable outcomes, the probability of each must be
something like $1/\infty$.

It makes more sense to talk about the probability that the random
variable will lie within some range of values.  Let us define a
function $f(x)$ such that\footnote{The symbol ``$\int$'' is from
  calculus, which you will need from here on.}
\[
\int_a^b f(x) dx
\]
is the probability that the random variable lies between $a$ and $b$.
Here, $f(x)$ is called the \emph{probability density function (pdf)}.
Loosely speaking, $f(x) dx$ is the probability that the random
variable lies within the small range from $x$ to $x + dx$.

In biology, the most widely used continuous distributions are the
uniform, the exponential, and the normal.

\subsubsection{The uniform distribution}
\label{sec.uniform}

A \emph{uniform distribution} (which we encountered above on
p.~\pageref{pg.simuni}) is equally likely to take any value between
two constants $a$ and $b$ but never takes values outside this range.
Thus, its density function is very simple: $f(x) = 1/(b-a)$, as shown
in Fig.~\ref{fig.uniform}.  The mean and variance are $E[X] = (a+b)/2$
and $V[X] = (b-a)^2/12$.  An important special case is the
\emph{standard uniform distribution}, for which $a=0$, $b=1$, and
$f(x) = 1$.

% -*-latex-*-
\begin{figure}
\centering
\mbox{\beginpicture
\setcoordinatesystem units <1.2in,1in>
  \setplotarea x from -0.2 to 1.2, y from 0 to 1
  \axis bottom shiftedto y=-0.05 label {$x$} 
    ticks withvalues $a$ $b$ / at 0 1 / /
  \axis left shiftedto x=-0.25 label {$f(x)$} 
    ticks withvalues 0 $1/(b-a)$ / at 0 1 / /
% lambda
\plot  
-0.2 0
0 0
0 1
1 1
1 0
1.2 0
/
\endpicture}
\caption{Uniform density function}
\label{fig.uniform}
\end{figure}


\begin{exercise}
  Make a graph of the standard uniform distribution function, and
  shade the area that corresponds to the range from 0.2 to 0.3.  What
  is the area of this shaded region?  What is the probability that a
  standard uniform r.v.\ will lie between these values?
%
\answer
The graph is:
%
% -*-latex-*-
\begin{center}
\mbox{\beginpicture
\setcoordinatesystem units <1.2in,1in>
  \setplotarea x from -0.2 to 1.2, y from 0 to 1
  \axis bottom shiftedto y=-0.05 label {$x$} ticks numbered from 0 to 1 by 1 /
  \axis left shiftedto x=-0.25 label {$f(x)$} ticks numbered from 0 to 1 by 1 /
% lambda
\plot  
-0.2 0
0 0
0 1
1 1
1 0
1.2 0
/
\shaderectangleson
\putrectangle corners at 0.2 0 and 0.3 1
\endpicture}
\end{center}

%
The area of the shaded rectangle is $(0.3-0.2) \times 1 = 0.1$.  This
is also the probability that $0.2 < X < 0.3$.
\end{exercise}

\begin{exercise}
Solve the same problem using calculus.
\answer
For any continuous r.v., the probability that $X$ lies between two
values $a$ and $b$ is $\int_a^b f(x) dx$.  In this problem, $a=0.2$,
$b=0.3$, and $f(x)=1$.  The integral is thus equal to 0.1.
\end{exercise}

\subsubsection{The exponential distribution}
\label{sec.exponential}

We are often interested in the waiting time until some event.  If
these events happen at a constant rate (or hazard) $h$, then the
waiting time has an exponential distribution.  The density function is
\begin{equation}
f(x) = h e^{-hx}
\label{eq.exponential}
\end{equation}
for values of $x$ between 0 and $\infty$.  As shown in
Fig.~\ref{fig.exponential}, the density is greatest at $x=0$ and
declines smoothly with increasing $x$.  The rate of decline increases
with $h$.  The mean of the exponential is $E[X] = 1/h$ and its
variance is $V[x] = 1/h^2$. For this distribution, the standard
deviation (the square root of the variance) is equal to the mean.

% -*-latex-*-
\begin{figure}
\centering
\mbox{\beginpicture
\setcoordinatesystem units <0.6in,0.5in>
  \setplotarea x from 0 to 2, y from 0 to 2
  \axis bottom label {$x$} ticks numbered from 0 to 2 by 1 /
  \axis left label {$f(x)$} ticks numbered from 0 to 2 by 1 /
% h=1
\plot  
0.000 1.0000
0.050 0.9512
0.100 0.9048
0.150 0.8607
0.200 0.8187
0.250 0.7788
0.300 0.7408
0.350 0.7047
0.400 0.6703
0.450 0.6376
0.500 0.6065
0.550 0.5769
0.600 0.5488
0.650 0.5220
0.700 0.4966
0.750 0.4724
0.800 0.4493
0.850 0.4274
0.900 0.4066
0.950 0.3867
1.000 0.3679
1.050 0.3499
1.100 0.3329
1.150 0.3166
1.200 0.3012
1.250 0.2865
1.300 0.2725
1.350 0.2592
1.400 0.2466
1.450 0.2346
1.500 0.2231
1.550 0.2122
1.600 0.2019
1.650 0.1920
1.700 0.1827
1.750 0.1738
1.800 0.1653
1.850 0.1572
1.900 0.1496
1.950 0.1423
/
\setdashes
% h=2
\plot
0.000 2.0000
0.050 1.8097
0.100 1.6375
0.150 1.4816
0.200 1.3406
0.250 1.2131
0.300 1.0976
0.350 0.9932
0.400 0.8987
0.450 0.8131
0.500 0.7358
0.550 0.6657
0.600 0.6024
0.650 0.5451
0.700 0.4932
0.750 0.4463
0.800 0.4038
0.850 0.3654
0.900 0.3306
0.950 0.2991
1.000 0.2707
1.050 0.2449
1.100 0.2216
1.150 0.2005
1.200 0.1814
1.250 0.1642
1.300 0.1485
1.350 0.1344
1.400 0.1216
1.450 0.1100
1.500 0.0996
1.550 0.0901
1.600 0.0815
1.650 0.0738
1.700 0.0667
1.750 0.0604
1.800 0.0546
1.850 0.0494
1.900 0.0447
1.950 0.0405
/
\endpicture}
\caption{Exponential density functions with $h=1$ (solid line) and
  $h=2$ (dashed).}
\label{fig.exponential}
\end{figure}


\begin{exercise}
What is the probability that $X < 1$?
\answer
%
An exponential variable is $<1$ with probability
$\int_0^1 h e^{-hx} dx = 1-e^{-h}$.
\end{exercise}

\begin{exercise}
  In Europe, the crude death rate (including individuals of all ages)
  is close to $0.01$ deaths per individual per year.  If this rate
  were constant throughout life, what would be the average lifespan?
  \answer 100~years, because the mean is $1/h$, and the problem says
  that $h=0.01$.
\end{exercise}

\begin{exercise}
The mutation rate in autosomal DNA is roughly $10^{-9}$ per nucleotide
site per year.  If we follow a single copy of a single nucleotide
forward across the generations, how long must we wait on average until
it mutates?  What is the variance of this number?
\answer
The mean is $10^9$ years; the variance is $10^{18}$.
\end{exercise}

\subsubsection{The normal distribution}

The density function of the normal distribution is the familiar
bell-shaped curve, two examples of which are shown in
Fig.~\ref{fig.normal}.  The normal distribution has two parameters,
the mean $\mu$ and the variance $\sigma^2$.  As Fig.~\ref{fig.normal}
illustrates, $\mu$ controls the location of the center (peak) of the
distribution and $\sigma$ controls its width.  The density function is
\begin{equation}
f(x) = \frac{1}{\sqrt{2\pi \sigma^2}}
\exp\left(-\frac{(x-\mu)^2}{2\sigma^2}\right)
\label{eq.normal}
\end{equation}
% -*-latex-*-
\begin{figure}
\centering
\mbox{\beginpicture
\setcoordinatesystem units <0.6in,1.4in>
  \setplotarea x from -2 to 2, y from 0 to 0.8
  \axis bottom label {$x$} ticks numbered from -2 to 2 by 1 /
  \axis left label {$f(x)$} /
% Normal(0,1)
\plot  
-2.000 0.0540
-1.900 0.0656
-1.800 0.0790
-1.700 0.0940
-1.600 0.1109
-1.500 0.1295
-1.400 0.1497
-1.300 0.1714
-1.200 0.1942
-1.100 0.2179
-1.000 0.2420
-0.900 0.2661
-0.800 0.2897
-0.700 0.3123
-0.600 0.3332
-0.500 0.3521
-0.400 0.3683
-0.300 0.3814
-0.200 0.3910
-0.100 0.3970
0.000 0.3989
0.100 0.3970
0.200 0.3910
0.300 0.3814
0.400 0.3683
0.500 0.3521
0.600 0.3332
0.700 0.3123
0.800 0.2897
0.900 0.2661
1.000 0.2420
1.100 0.2179
1.200 0.1942
1.300 0.1714
1.400 0.1497
1.500 0.1295
1.600 0.1109
1.700 0.0940
1.800 0.0790
1.900 0.0656
/
\setdashes
% Normal(1,0.5)
\plot
-2.000 0.0000
-1.900 0.0000
-1.800 0.0000
-1.700 0.0000
-1.600 0.0000
-1.500 0.0000
-1.400 0.0000
-1.300 0.0000
-1.200 0.0000
-1.100 0.0001
-1.000 0.0003
-0.900 0.0006
-0.800 0.0012
-0.700 0.0025
-0.600 0.0048
-0.500 0.0089
-0.400 0.0158
-0.300 0.0272
-0.200 0.0448
-0.100 0.0709
0.000 0.1080
0.100 0.1579
0.200 0.2218
0.300 0.2995
0.400 0.3884
0.500 0.4839
0.600 0.5794
0.700 0.6664
0.800 0.7365
0.900 0.7821
1.000 0.7979
1.100 0.7821
1.200 0.7365
1.300 0.6664
1.400 0.5794
1.500 0.4839
1.600 0.3884
1.700 0.2995
1.800 0.2218
1.900 0.1579
/
\endpicture}
\caption{Normal density functions with $\mu=0$, $\sigma=1$ (solid line) and
  $\mu=1$, $\sigma=1/2$ (dashed).}
\label{fig.normal}
\end{figure}


The normal distribution is widely used in statistics.  There are
several reasons, but chief among them is this: any variable that is
the sum of many other random variables tends to look normal.  The
larger the number of variables, the more normal their sum will look.
We have already passed over two special cases of this.  A binomial
r.v.\ is a sum of $N$ smaller r.v.s, one for each toss of the coin.
If you look closely at the binomial distributions in
Fig.~\ref{fig.binom}, you will see that they closely resemble the
normal.  The Poisson is also approximately normal if $\lambda$ is
large, as you can see from panel~b of Fig.~\ref{fig.poisson}.  To
understand why, recall that the Poisson describes the number of events
that occur in a fixed interval of time.  But we can think of this as
the sum of the numbers of events that occur in a series of
sub-intervals.  Thus, the Poisson is also a sum.

In addition to these technical concerns, there is a very practical
reason for wide interest in the normal distribution: many of the
variables studied in biology, agriculture, and medicine seem to be
approximately normal.  Why should the same pattern appear so often and
in so many different contexts?  The answer to this question is very
likely the same business about sums that we discussed just above.
Many of the variables we study are affected by a multitude of causes.
Many genetic loci, for example, contribute to variation in human
stature.  To the extent that these loci act additively, stature is a
sum.  Stature, of course, is affected by environmental causes as well
as genetic ones.  To the extent that these environmental factors act
additively, they contribute to the sum.  In this sense, many of the
variables we study are sums of a sort, and it makes sense that their
distributions should look normal.


