\section{Probability}
\label{ch.prob}

Probability theory is about things that cannot be predicted with
certainty.  It contributes to science in two ways.  First, there is
uncertainty in the phenomena we study.  Even when we know the genes of
the parents, we cannot predict with certainty those of their
offspring.  Neither can we predict with certainty the path of a
molecule through a gas, or how long anyone will live.  All such
phenomena need theories with probabilistic components.  Probability
also contributes to science in a second way.  We often study
populations of things that are too large to examine in their entirety.
Instead we study incomplete samples, which reflect only imperfectly
the properties of the larger whole.  Thus, probability theory also
underlies statistics, the science of sampling.

There are two versions of probability theory.  The first studies
\emph{statistical probability}, the relative frequency with which an
event occurs in the long run.  The second studies \emph{subjective (or
  Bayesian) probability}, which measures one's degree of belief.
Bayesian probability has recently become important in statistics.  The
focus here is on statistical probability, which has long played a
central role in many disciplines.

This primer is short, but you should not expect to read it quickly.
Quantitative material takes time to digest.  Read it with pencil in
hand and do the exercises as you go.  But do not spend too much time
on any one exercise.  If you get stuck, consult the answers in the
back.  Read these answers in any case, because they contain some of
the material I am trying to teach.

\subsection{Probability}

All of us have an intuitive understanding of statements such as ``the
chances are three in five.''  This is exactly what is meant by the
statement ``the (statistical) probability is $3/5$.''  In a large
number of repeated trials we would expect the event in question to
occur about $3/5$ of the time.  In other words, its \emph{relative
  frequency} should be about $3/5$.  The relative frequency of an
event is simply the number of trials on which it occurs expressed as a
fraction of the total.  It is a cumbersome term for a simple idea, and
people often shorten it to ``frequency.''  Unfortunately, \emph{that}
word is also used for the raw count of events.  For example, if we
spin a coin 100 times and observe 30 heads, then the relative
frequency of heads is 0.3.  This is also referred to simply as the
frequency of heads.  In other contexts, however, the ``frequency of
heads'' might refer to 30, the number of heads.  You have to catch
meaning from context.

The larger the number of trials, the smaller will be the difference
between the relative frequency and the probability.  My favorite
illustration of this idea comes from World War II.  Just before the
war, the mathematician John Kerrich \cite{Kerrich:EIT-46} was visiting
Denmark.  Two days before his scheduled departure, the German army
overran Denmark, and Kerrich was interned for the duration of the war.
He must have had a lot of spare time, for he undertook several lengthy
experiments on probability.  In one of them, he spun a coin 10,000
times and kept a record of the outcomes.

\let\put\pictexput
\mbox{\beginpicture
\setcoordinatesystem units <0.15\columnwidth, 2in> point at 0 0
\setplotarea x from 0 to 4, y from 0.0 to 0.7
\axis left label {\lines{Relative\cr frequency\cr of\cr heads}}
  ticks numbered from 0.00 to 0.50 by 0.25 /
\axis bottom label {Number of spins (log scale)} 
%  ticks withvalues 3 10 30 300 1000 3000 10000 / 
%  logged at 3 10 30 300 1000 3000 10000  / /
  ticks withvalues 1 3 10 100 1000 10000 / 
  at 0 0.4771 1 2 3 4 / /
\put {\footnotesize (Kerrich 1946)} [br] <0pt,4pt> at 4 0
\setdots
\putrule from 0 0.5 to 4 0.5
\setsolid
% x axis counts coin tosses on log10 scale
% y is the fraction of heads in x tosses
% Data from Kerrich 1964
\plot  
0.000 0.000
0.301 0.000
0.477 0.000
0.602 0.250
0.699 0.400
0.778 0.500
0.845 0.429
0.903 0.500
0.954 0.444
1.000 0.400
1.041 0.455
1.079 0.500
1.114 0.538
1.146 0.571
1.176 0.600
1.204 0.562
1.230 0.588
1.255 0.556
1.279 0.526
1.301 0.500
1.322 0.524
1.342 0.545
1.362 0.522
1.380 0.542
1.398 0.520
1.415 0.538
1.431 0.556
1.447 0.571
1.462 0.586
1.477 0.567
1.544 0.514
1.602 0.525
1.653 0.489
1.699 0.500
1.740 0.491
1.778 0.483
1.813 0.462
1.845 0.457
1.875 0.453
1.903 0.438
1.929 0.447
1.954 0.444
1.978 0.432
2.000 0.440
2.041 0.436
2.079 0.442
2.114 0.446
2.146 0.464
2.176 0.473
2.204 0.463
2.230 0.465
2.255 0.478
2.279 0.484
2.301 0.490
2.398 0.500
2.477 0.487
2.544 0.494
2.602 0.497
2.653 0.502
2.699 0.510
2.740 0.509
2.778 0.520
2.813 0.525
2.845 0.526
2.875 0.520
2.903 0.516
2.929 0.515
2.954 0.509
2.978 0.509
3.000 0.502
3.041 0.493
3.079 0.497
3.114 0.499
3.146 0.503
3.176 0.504
3.204 0.506
3.230 0.511
3.255 0.510
3.279 0.511
3.301 0.506
3.322 0.509
3.342 0.510
3.362 0.511
3.380 0.508
3.398 0.509
3.415 0.506
3.431 0.500
3.447 0.503
3.462 0.503
3.477 0.503
3.491 0.504
3.505 0.503
3.519 0.505
3.531 0.506
3.544 0.506
3.556 0.507
3.568 0.506
3.580 0.507
3.591 0.507
3.602 0.507
3.613 0.508
3.623 0.508
3.633 0.509
3.643 0.510
3.653 0.510
3.663 0.510
3.672 0.511
3.681 0.510
3.690 0.508
3.699 0.507
3.708 0.505
3.716 0.508
3.724 0.507
3.732 0.505
3.740 0.504
3.748 0.504
3.756 0.504
3.763 0.504
3.771 0.504
3.778 0.501
3.785 0.501
3.792 0.501
3.799 0.500
3.806 0.500
3.813 0.499
3.820 0.499
3.826 0.500
3.833 0.501
3.839 0.501
3.845 0.502
3.851 0.502
3.857 0.503
3.863 0.503
3.869 0.502
3.875 0.504
3.881 0.504
3.886 0.503
3.892 0.504
3.898 0.505
3.903 0.504
3.908 0.504
3.914 0.504
3.919 0.504
3.924 0.503
3.929 0.504
3.934 0.505
3.940 0.505
3.944 0.505
3.949 0.505
3.954 0.504
3.959 0.505
3.964 0.505
3.968 0.505
3.973 0.505
3.978 0.506
3.982 0.506
3.987 0.506
3.991 0.506
3.996 0.507
4.000 0.507
/
\endpicture}
\let\put\latexput


The resulting data are shown in Figure~\ref{fig.coin}.  There, the
vertical axis shows the relative frequency of heads.  Its value
bounced around at first but got closer and closer to $1/2$ as the
number of spins increased.  When we say that the probability of heads
on a single spin is $1/2$, this is what we have in mind.  There is
nothing special about the number $1/2$.  Had Kerrich's coin been bent,
it might have tended to favor one side.  Even so, in 10,000 tosses it
would have converged toward some particular value.  This value is
called the \emph{probability} of ``heads.''

A coin that is equally likely to fall on either side is said to be
``fair,'' but nothing guarantees that coins must be fair.  When we say
that one is fair, we are stating a hypothesis, not a fact.  One can
estimate a probability, as Kerrich did by spinning his coin.  But
estimates always have error, and the exact probability is never known.
We often make judgments about probabilities even without this sort of
evidence.  I have never spun any of the coins in my pocket, yet I am
confident that all of them are very nearly fair.  Why?  I'm not sure.
Perhaps because people have told me so, and perhaps because coins are
nearly (although not quite) symmetrical.  Neither of these prove that
my coins are exactly fair.  When we assume that probabilities have
particular values, we are building a model.  The fit of that model to
the real world is always a matter to be tested against data.

If Kerrich were doing his experiment today, he might have automated
the process.  Most computer languages provide a way to generate
``uniform\label{pg.simuni} random numbers,'' which we'll discuss more
fully on p.~\pageref{sec.uniform}.  The gist is that these generators
deliver numbers that are (in theory) equally likely to fall anywhere
in the interval from 0 to 1.  Hence, the value falls between 0 and
$1/2$ with probability $1/2$.  This makes it easy to simulate the spin
of a fair coin.  Simply get one number from the random number
generator.  Interpret it as heads if less than $1/2$ and as tails if
greater.

\begin{exercise}
Describe a method for simulating the spin of an unfair coin, for which
heads has probability 0.3.
\answer
There are many correct answers.  Here are two: (1)~Interpret numbers
as heads if less than 0.3 but as tails if greater.  (2)~Interpret
numbers as heads if between 0.2 and 0.5 but as tails otherwise.
\end{exercise}

\begin{exercise}
If you know how to write computer programs, write one that replicates
Kerrich's experiment.
\answer
\begin{verbatim}
# Python program that simulates 10000
# spins of a fair coin

from random import random

for i in range(10000):
    u = random()
    if u < 0.5:
        print('heads')
    else:
        print('tails')
\end{verbatim}
\end{exercise}


\subsection{An urn experiment}

Texts on probability theory frequently discuss experiments that
involve drawing balls out of urns.  During his captivity, Kerrich did
his own version of an urn experiment.  Instead of an urn, he used a
box and four ping-pong balls---two red and two green.  The experiment
consisted of 5000 identical trials, each of which began with all four
balls in the box.  Kerrich's assistant then shook the box, looked
away, and drew out first one ball and then (without replacing the
first) another.  They wrote down the colors of the two balls.  On each
trial, there were four possible outcomes.  Kerrich counted the trials
in which each occurred, with results as shown in Table~\ref{tab.urn}.

\begin{table}
\caption{The results of 5000 repetitions of Kerrich's
  \cite{Kerrich:EIT-46} urn experiment}
\label{tab.urn}
\centering
\begin{tabular}{lrrr}
First  & \multicolumn{2}{c}{Second ball}\\ \cline{2-3}
ball & Red & Green & sum\\ \hline
Red  & 756 &  1689 & 2445 \\
Green&1688 &   867 & 2555 \\ \hline
sum  &2444 &  2556 & 5000
\end{tabular}
\end{table}

\subsection{A model}
\label{sec.urnmodel}

There is a clear pattern in these data: the second ball was usually
red if the first was green and usually green if the first was red.
What features of the experiment might account for this pattern?

To answer such a question, we need to build a model.  This involves
making assumptions that seem plausible and then using these to
calculate the probabilities of the various events in the data.
Finally, we compare the predicted values to those in the data.

I built my own model using the tree diagram in Fig.~\ref{fig.tree}.
Each trial begins at the root and progresses through the tree to one
of the tips.  For example, if two red balls are drawn, we take the
upper path at each node.  There are labels at the tips of the
branches, which correspond to events: $RR$ for red--red, $RG$ for
red--green, and so forth.

\begin{figure}[b]
{\centering%-*-latex-*-
\framebox{\beginpicture
\setcoordinatesystem units <0.94cm, 0.6cm>
\setplotarea x from 0 to 7, y from -3 to 4
\plot 0 0 2 1.7 /
\plot 0 0 2 -1.7 /
\put {$1/2$} [br] <-1pt,1pt> at 1 0.85
\put {$1/2$} [tr] <-1pt,-1pt> at 1 -0.85
\put {red}   [c] at 2.5 2
\put {green} [c] at 2.5 -2
%
\plot 3 2 5 3 /
\plot 3 2 5 1 /
\put {$1/3$} [br] <-1pt,1pt> at 4 2.5
\put {$2/3$} [tr] <-1pt,-1pt> at 4 1.5
\put {red}   [l] <3pt,0pt> at 5 3
\put {green} [l] <3pt,0pt> at 5 1
%
\plot 3 -2  5 -1 /
\plot 3 -2  5 -3 /
\put {$2/3$} [br] <-1pt,1pt> at 4 -1.5
\put {$1/3$} [tr] <-1pt,-1pt> at 4 -2.5
\put {red}   [l] <3pt,0pt> at 5 -1
\put {green} [l] <3pt,0pt> at 5 -3
%
\put {\lines{First\cr ball}}  [b] at 2.5 3.7
\put {\lines{Second\cr ball}} [b] at 5.5 3.7
\put {Event} [b] at 6.8 3.7
\put {$RR$} <0pt,-1pt> at 6.8 3
\put {$RG$} <0pt,0pt> at 6.8 1
\put {$GR$} <0pt,-1pt> at 6.8 -1
\put {$GG$} <0pt,0pt> at 6.8 -3
%
\put {Prob} [b] at 8 3.7
\put {$1/6$} <0pt,-1pt> at 8 3
\put {$1/3$} <0pt,0pt> at 8 1
\put {$1/3$} <0pt,-1pt> at 8 -1
\put {$1/6$} <0pt,0pt> at 8 -3
\endpicture}
\\}
\caption{Tree showing the calculation of probabilities in urn model.}
\label{fig.tree}
\end{figure}

The numbers in the figure are all probabilities.  To explain them, I
begin at the root of the tree, which corresponds to the beginning of a
trial.  At that point, the box contains two red balls and two green
ones.  If each ball is equally likely to be chosen, then the first
ball is equally likely to be red or green.  Thus, the two paths from
the root each have probability $1/2$.  So far, the model is consistent
with the data, since the first ball was red on $2445/5000$---very nearly
$1/2$---of the trials.

Suppose now that the first ball was red and you are about to draw a
second ball.  Three balls are left: two green and one red.  If these
are still equally likely to be chosen, then the probability of drawing
a second red ball is $1/3$ and that of drawing a green ball is $2/3$.
If the first ball was green then the same argument applies, except
that now the red ball has probability $2/3$ and the green ball $1/3$.
This accounts for the probabilities of the remaining branches.

We can now use the tree to calculate the probabilities of the events
($RR$, $RG$, $GR$, and $GG$) associated with the branch tips.  The
trick is to start at the root and trace a path through the tree,
multiplying together the probabilities of the branches in the path.
Take for example event $RR$.  Its probability is $1/2$ (the
probability that the first ball is red) times $1/3$ (probability that
the second is red given that the first one was).  The result, $1/6$,
is shown in the right column of Fig.~\ref{fig.tree} along with the
probabilities of the other three events.

\begin{table}
\caption{Theoretical probabilities and observed relative frequencies
  of events in Kerrich's urn experiment.}
\label{tab.urn.obsexp}
\centering
\begin{tabular}{ccc}
        &      & Rel.  \\
  Event & Prob.& freq. \\
\hline
\hline
$RR$    & 0.167& 0.151 \\
$RG$    & 0.333& 0.338 \\
$GR$    & 0.333& 0.338 \\
$GG$    & 0.167& 0.173 \\
\hline
\end{tabular}
\end{table}

These probabilities also appear in Table~\ref{tab.urn.obsexp}, where
they are compared with the relative frequencies in Kerrich's data.  I
have re-expressed everything as a decimal fraction.  For example, the
theoretical probability of $RR$ is $1/6\approx 0.167$ and its observed
relative frequency is $756/5000 \approx 0.151$.  The relative
frequencies do not quite equal the probabilities, but the differences
are small.  A careful statistical analysis would probably conclude
that this model is consistent with the data.  (We can't know without
doing such an analysis, and that is beyond my scope here.)

\subsection{Conditional and joint probability: the multiplication law}
\label{sec.condjoint}

In the tree diagram, the probabilities involving the second ball
deserve comment.  They are called \emph{conditional probabilities}
because their values depend on (i.e.\ are conditioned by) the color of
the first ball.  The conditional probability of $B$ given $A$ is
written $\Pr[B|A]$.  In multiplying along paths within the tree, we
were using something called the \emph{law of multiplication of
  probabilities}:
\begin{equation}
\Pr[A \And B] = \Pr[A] \Pr[B|A]
\label{eq.mlaw}
\end{equation}
Here, ``$\Pr[A\And B]$'' is called the \emph{joint probability} of
events $A$ and $B$; it is the probability that $A$ and $B$ both
happened on the same trial.  For example, suppose $A$ is the event
that the first ball is red and $B$ the event that the second is green.
Using the tree diagram, we calculated $\Pr[A\And B]$ by multiplying
along the relevant path.  In terms of the multiplication law, $\Pr[A]
= 1/2$, $\Pr[B|A] = 2/3$, and $\Pr[A\And B] = 1/2 \times 2/3 = 1/3$.

The multiplication law is important because it explains \emph{why} we
multiply along paths within the tree.  To me, this seems intuitive and
obvious.  But some things that are obvious are also wrong, so
Box~\ref{box.mlaw} explains why it works.

\begin{Box}[tbp]
  We are interested in the probability of event ``$A\And B$.'' The
  relative frequency of this event is
  $n(A \And  B)/N$, where $n(A \And  B)$ is the number of trials on
  which this event occurred and $N$ is the total number of trials.
  Multiply by $n(A)/n(A)=1$ (which changes nothing) to obtain
\[
\frac{n(A \And  B)}{N} = \frac{n(A)}{N} \cdot \frac{n(A\And B)}{n(A)}
\]
As the number of trials grows large, relative frequencies become
closer and closer to the corresponding probabilities.  Thus, the left
side approaches $\Pr[A \And B]$, the ``joint probability'' of ``$A
\And B$.''  Meanwhile, on the right $n(A)/N$ approaches $\Pr[A]$.  But
what are we to make of the remaining fraction, $n(A\And B)/n(A)$?  It
is the relative frequency of $B$ among trials in which $A$ occurred.
As $N$ grows large, this ratio converges to the \emph{conditional
  probability of $B$ given $A$}, or $\Pr(B|A)$.  Thus, in the equation
above all three ratios converge to probabilities as $N$ grows large,
and the equation itself converges to the multiplication law
(Eqn.~\ref{eq.mlaw}).
\caption{Deriving the multiplication law}
\label{box.mlaw}
\end{Box}

\subsection{The addition law}

We turn now to events of form ``$A \Or B$'' (or both).  The
probabilities of such events are especially easy when the two events
are \emph{mutually exclusive}, i.e.\ when they could not have happened
on a single trial.  Take for example the event that the second ball is
green.  This happens in either of two cases: $RG$ and $GG$.  The
probability of this event is the probability of ``$RG \Or GG$,'' which
(according to the tree diagram) equals $1/3 + 1/6 =
1/2$.\label{pg.Pr_green} To get this answer, we simply summed the
probabilities of $RG$ and $GG$.

\begin{exercise}
  In Kerrich's urn data, show that the relative frequency of ``$RG \Or
  GG$'' equals the sum of the frequencies of $RG$ and $GG$.
%
\answer
%
The relative frequency of ``$RG \Or GG$'' is $2556/5000$, that of $RG$
is $1689/5000$, and that of $GG$ is $867/5000$.  The sum of the last
two is $1689/5000 + 867/5000 = 2556/5000$.
\end{exercise}

The calculation is a little harder when the two events are \emph{not}
mutually exclusive.  To see why, consider the event that either the
first ball is red or the second is green (or both).  This is also of
form ``$A \Or B$,'' but if we try summing $\Pr[\hbox{1st ball red}]$
and $\Pr[\hbox{2nd ball green}]$ we get $1/2 + 1/2 = 1$.  This
\emph{can't} be right.

\begin{exercise}
  In Kerrich's urn data, calculate the relative frequencies of events
  $A$, $B$, and ``$A\Or B$.''  Show that the sum of the first two does
  not equal the third.
%
\answer
The relative frequency of the event $(A)$ that the first ball is red
equals $2445/5000$; that of the event $(B)$ that the second is green
equals $2556/5000$; that of event ``$A\Or B$'' is $(756+1689+867)/5000
= 3312/5000$.  The sum of the first two relative frequencies is
$2445/5000 + 2556/5000 = 5001/5000$, which is much larger than the
relative frequency of ``$A\Or B$.''
\end{exercise}

To see what went wrong, let us look under the hood.  The first ball is
red in either of two cases: $RR$ and $RG$.  Similarly, the second is
green in cases $RG$ and $GG$.  Thus, our incorrect calculation can be
expanded as:
\[
\overbrace{\Pr[RR] + \Pr[RG]}^{\hbox{$\Pr[A]$}}
+ \overbrace{\Pr[RG] + \Pr[GG]}^{\hbox{$\Pr[B]$}}
\]
We have (incorrectly) summed $\Pr[RG]$ twice.  To fix this, we must
subtract $\Pr[RG]$.  This illustrates the \emph{law of addition of
probabilities}:
\begin{equation}
\Pr[A \Or B] = \Pr[A] + \Pr[B] - \Pr[A \And B]
\label{eq.alaw}
\end{equation}
When the two events are mutually exclusive (as in the preceding
example), the probability of ``$A \And B$'' is zero, and there is no
need to subtract it off.

\begin{exercise}
Use the addition law to calculate the probability that the first ball
is red or the second green.
\answer
The first ball is red with probability $\Pr[A] = 1/2$, and the second
is green with probability $\Pr[B] = 1/2$.  The probability that both
events happened is $\Pr[A\And B] = 1/3$.  All this is from
Fig.~\ref{fig.tree}.  Using these values, the addition law gives
$\Pr[A\Or B] = 1/2 + 1/2 - 1/3 = 2/3$.
\end{exercise}

\subsection{Statistical independence}

If one event does not influence another, this fact should be reflected
in their probabilities.  Two events $A$ and $B$ are said to be
\emph{statistically independent} if $\Pr[B | A] = \Pr[B]$.

For an example in which this condition is not met, we return to
Kerrich's urn experiment.  The probabilities of red and green in the
second ball depend on the color of the first ball.  Thus, intuition
suggests that the balls are not independent.

\begin{exercise}
Define $A$ as the event that the first ball is is red and $B$ as the
event that the second is green.  Use the tree diagram in
Fig.~\ref{fig.tree} to calculate $\Pr[B]$ and $\Pr[B|A]$ and thus to
decide whether the balls in the model are statistically independent.
\answer As shown on p.~\pageref{pg.Pr_green}, the tree diagram implies
that the unconditional probability of $B$ is $\Pr[B] = 1/2$.  On the
other hand, the conditional probability is $\Pr[B|A] = 2/3$.  These
probabilities are not equal, so $A$ and $B$ are not independent.
\end{exercise}

\begin{exercise}
Use Kerrich's data to estimate the same probabilities.
\answer
According to Table~\ref{tab.urn}, $\Pr[B]$ is estimated by $2556/5000
\approx 0.51$, and $\Pr[B|A]$ by $1689/2445 \approx 0.69$.  The two
numbers are only estimates, so we cannot conclude that the
probabilities differ merely because the estimates do.  However, the
difference between these estimates is large, and so is the sample.
Even without a careful statistical analysis, these results suggest
that the two balls were not statistically independent.
\end{exercise}

\begin{table}
\renewcommand{\tabcolsep}{3pt}
\begin{center}
\renewcommand{\arraystretch}{0.95}
\begin{tabular}{rrrrrrrr}
    & \multicolumn{6}{c}{White}&\\ \cline{2-7}
 Red & 1 & 2 & 3 & 4 & 5 & 6 & sum\\
\hline
\hline
    1     & 547 & 587 & 500 & 462 & 621 & 690 & 3407\\
    2     & 609 & 655 & 497 & 535 & 651 & 684 & 3631\\
    3     & 514 & 540 & 468 & 438 & 587 & 629 & 3176\\
    4     & 462 & 507 & 414 & 413 & 509 & 611 & 2916\\
    5     & 551 & 562 & 499 & 506 & 658 & 672 & 3448\\
    6     & 563 & 598 & 519 & 487 & 609 & 646 & 3422\\ \hline
sum:     & 3246 & 3449 & 2897 & 2841 & 3635 & 3932 & 20000
\end{tabular}
\end{center}
\caption{The results of 20,000 throws with two dice (Wolf 1850, cited
  in \cite{Bulmer:67})}
\label{tab.dice}
\end{table}

Dice give us the other sort of example.  In 1850, the astronomer
Rudolf Wolf described the results of 20,000 throws of two dice, one
red and one white.  The results are shown in Table~\ref{tab.dice}.  We
can use these data to ask whether the two dice were independent.
Consider the event ``red 4'' of a 4 on the red die.  The unconditional
frequency of this event was $2916/20000 \approx 0.15$.  Conditional on
a 5 on the white die, the frequency of ``red 4'' was $509/3635 \approx
0.14$.  These two numbers are nearly the same, as they ought to be if
the red and white dice are independent.

\begin{exercise}
  Use Wolf's data to estimate (a)~the
  unconditional probability of ``red 2,'' and (b)~the conditional
  probability of ``red 2'' given ``white 4.''  Use the results to
  comment on whether Wolf's dice were statistically independent.
\answer
  The relative frequency of ``red 2'' is $3631/20,000 \approx 0.18$.
  This estimates the unconditional probability of rolling ``2'' with the
  red.  If we restrict attention to those trials on which the white die
  rolled ``4,'' the relative frequency of ``red 2'' is $535/2841 \approx
  0.19$.  This estimates the conditional probability of ``red 2'' given
  ``white 4.''  The numbers are pretty nearly equal, as they should be
  if the dice are statistically independent.
\end{exercise}

If these dice were fair, each of the row and column sums in
Table~\ref{tab.dice} should be close to $20000/6$, or 3333.  Instead,
both dice show an excess of 2s and 5s and a deficit of 3s and 4s.  In
addition, the white die shows an excess of 6s.  Michael Bulmer
\cite{Bulmer:67} discusses several plausible causes: the dice may not
be cubes, their corners may be rounded unevenly, or the process of
cutting pips (dots) into their faces may have altered their centers of
gravity.  Whatever the explanation, these data show that the model of
a ``fair die'' is only an approximation.

\begin{exercise}
Suppose that Kerrich had placed the first ball back into the box and
then shaken it again before drawing the second.  Draw a decision tree
to represent this experiment and use it to calculate the probabilities
of $RR$, $RG$, $GR$, and $GG$.  Use the tree to show that the two
balls are independent.  (By the way, this new version of the
experiment involves what is called \emph{sampling with replacement}.)
\answer Under sampling with replacement, the decision tree is as shown
below.
%\begin{figure}[tbhp]
\begin{center}
\input{figwithrep}
\end{center}
%\caption{Tree diagram for sampling with replacement}
%\label{fig.withrep}
%\end{figure}
Using this decision tree, we can test for statistical independence as
follows: the second ball is green with unconditional probability
$\Pr[RG\Or GG] = 1/2$.  If the first ball is red, the second is green
with conditional probability $1/2$.  These numbers are the same, so
the two balls are statistically independent.
\end{exercise}

\subsection{Bayes's rule}
\label{sec.bayes}

Suppose your doctor orders a test to see whether you have some
hypothetical disease. Experiments have shown that the test was
positive in 99\% of patients who have the disease.  But it was also
positive in 1\% of patients who \emph{didn't}.  These, of course, are
the wrong numbers. You want the probability that \emph{you have the
  disease} given that your test was positive. To figure this out, you
or your doctor will need to use what is called Bayes's rule.

\begin{figure}
{\centering%-*-latex-*-
\framebox{\beginpicture
\setcoordinatesystem units <0.94cm, 0.6cm>
\setplotarea x from 0 to 6, y from -3 to 3
\put {100,000} [r] <-1pt,1pt> at 0 0
\plot 0 0 1.9 1.7 /
\plot 0 0 1.9 -1.7 /
\put {99,900} [br] <-1pt,1pt> at 1.425 1.25
\put {100} [tr] <-1pt,-1pt> at 1.425 -1.25
\put {healthy}   [c] at 2.5 2
\put {sick} [c] at 2.5 -2
%
\plot 3.1 2 5 3 /
\plot 3.1 2 5 1 /
\put {999} [br] <-1pt,1pt> at 4.525 2.75
\put {98,901} [tr] <-1pt,-1pt> at 4.525 1.25
\put {positive}   [l] <3pt,0pt> at 5 3
\put {negative} [l] <3pt,0pt> at 5 1
%
\plot 3.1 -2  5 -1 /
\plot 3.1 -2  5 -3 /
\put {99} [br] <-1pt,1pt> at 4.525 -1.25
\put {1} [tr] <-1pt,-1pt> at 4.525 -2.75
\put {positive}   [l] <3pt,0pt> at 5 -1
\put {negative} [l] <3pt,0pt> at 5 -3
\endpicture}
\\}
\caption{Bayes's rule example.}
\label{fig.bayes}
\end{figure}

It is easier to understand the principles involved if we think in
terms of counts rather than probabilities.  Consider the data in
Fig.~\ref{fig.bayes}. This tells us that in 100,000 people of your sex
and age, 99,900 will be healthy and 100 will be sick, with some
hypothetical disease. We get positive test results from nearly all
(99/100, or 99\%) of the sick individuals, but also from a small
fraction (999/99,900 or 1\%) of the healthy ones. Of those who test
positive, the sick fraction is
\[
\frac{99}{99 + 999} \approx 0.09
\]
Fewer than 1/10 of those who test positive are really sick!

Let us now rephrase this result in terms of probabilities.  According
to the multiplication law (Eqn.~\ref{eq.mlaw})
\[
\Pr[A\& B] = \Pr[B] \Pr[A|B] = \Pr[A] \Pr[B|A]
\]
Divide through by $\Pr[B]$ to get
\begin{equation}
\Pr[A|B] = \frac{\Pr[A] \Pr[B|A]}{\Pr[B]}
\label{eq.bayes}
\end{equation}
This is called \emph{Bayes's Rule}.  In the context of our example,
$B$ is the event that an individual got positive test, and $A$ is the
event that the individual is really sick.

To illustrate Bayes's Rule in action, let us revisit
Fig.~\ref{fig.bayes}. We use the figure to calculate relative
frequencies and then interpret these as probabilities. If $A$ is the
event that you are sick and $B$ the event that your test was positive,
then $\Pr[A] = 100/100,000 = 1/1000$, $\Pr[B] = (999+99)/100,000 =
1098/100,000$, and $\Pr[B|A] = 99/100$. Eqn.~\ref{eq.bayes} gives
\begin{eqnarray*}
\Pr[A|B] &=& \frac{1/1000 \times 99/100}{1098/100,000}\\
 &=& \frac{99}{1098} \approx 0.09
\end{eqnarray*}
This is the same answer we got above using counts.
