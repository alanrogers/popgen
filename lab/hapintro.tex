\chapter{Using HapMap}
\label{ch.hapintro}
Several large databases of human genetic variation have become
available during the past decade, and they have revolutionized the
field.  This project will introduce you to one of them, the HapMap.  The
HapMap project has genotyped millions of single-nucleotide
polymorphisms (SNPs) throughout the human genome within samples from
several human populations.  You will learn how to download and
manipulate these data, and will then use them to explore the
relationship between observed and expected heterozygosity in HapMap
SNPs. 

\section{Choosing a population and chromosome at random}
\label{sec.randpopchrom}
The original HapMap provided data on the following populations:
\begin{center}
\begin{tabular}{cl}
Label & Sample\\
\hline
\texttt{YRI} & 90~Yorubans from Nigeria\\
\texttt{CEU} & 90~Utahns of western and northern European ancestry\\
\texttt{CHB} & 45 Han Chinese from Beijing\\
\texttt{JPT} & 44 Japanese from Tokyo\\
\verb|JPT+CHB| & combined Chinese and Japanese samples
\end{tabular}
\label{pg.poptab}
\end{center}
Although recent releases of HapMap have added additional populations,
we will focus on these.  The data from these populations are made
available as plain text files, one for each chromosome within each
population.  You will download one of these files onto your computer,
but first you must figure out which one.  

For this project, you will choose a chromosome and population at
random.  To do so, cut and paste the following program into the Python
interpreter, and run it once. \verbatiminput{ranpopchrom.py} It will
tell you the number of your chromosome, and the name of your
population.  This program uses two new
functions---\texttt{choice}\label{pg.choice-intro} and
\texttt{randrange}---from Python's \texttt{random} module.  We'll be
using these at several points in this project, so you should
experiment with them until you understand how they work.  They are
described in section~\ref{sec.randommodule},
page~\pageref{sec.randommodule}.

\section{Downloading data}
Now that you know your chromosome and population, you are ready to
visit the HapMap web site.  This is a task that you'll need to do for
several projects, so we've put the instructions into a self-contained
appendix, which you'll find on page~\pageref{ch.gettingdata}.  
Please download two data files: the one for your own chromosome and
population, and also the one for chromosome~22, population
\texttt{JPT}. 

From here on, we'll assume that you have already downloaded the
appropriate data file for this project.  We recommend that you put all
relevant files into the \texttt{Documents} folder.  Before proceeding,
make sure that the data file and \texttt{pgen.py} are both in the same
folder (or directory), and that Python can execute the command
\verb|from pgen import *| without generating an error.

\section{Using the \texttt{pgen} module to work with HapMap data}
\label{sec.pgenhapmapintro}
The \texttt{pgen} module provides several tools for use with HapMap
data files.  These are described in section~\ref{sec.pgenhapmap}
(page~\pageref{sec.pgenhapmap}).  Here, we merely illustrate their
use.  At the top of your program, you will need the line:
\begin{leftindent}
\begin{verbatim}
from pgen import *
\end{verbatim}
\end{leftindent}
The portion of the program that uses HapMap data should begin like
this: 
\begin{leftindent}
\begin{verbatim}
pop = "JPT"
chromosome = 22
hds = hapmap_dataset(hapmap_fname(chromosome, pop))
\end{verbatim}
\end{leftindent}
Here, the call to \texttt{hapmap\_fname} tries to find the name of an
appropriate HapMap data set---one for chromosome~22 and population
\texttt{JPT}.  If it succeeds, it hands that name to
\texttt{hapmap\_dataset}, which reads the file and stores it in a
useful form (more on this in a moment).  In the last line, the
assignment statement creates a variable called \texttt{hds}, which
points to this data structure.  We use the name \texttt{hds} to help
us remember that it refers to an object of type
``\texttt{hapmap\_dataset},'' but you can use any name you please.

But what exactly is an object of type ``\texttt{hapmap\_dataset}?''
Well, it is a lot like a Python list.  Like a list, it has a length:
\begin{leftindent}
\begin{verbatim}
>>> len(hds)
11235
\end{verbatim}
\end{leftindent}
And like a list, it also has a sequence of data values that we can
access like this:
\begin{leftindent}
\begin{verbatim}
>>> snp = hds[3]
\end{verbatim}
\end{leftindent}
Now \texttt{snp} refers to the data value at position~3 within the
data set.  This data value is an object of another specially-created
type, called ``\texttt{hapmap\_snp}.''

An object of type \texttt{hapmap\_snp} has a variety of types of data.
Having defined \texttt{snp}, we can type
\begin{leftindent}
\begin{verbatim}
>>> snp.chromosome
'22'
>>> snp.position
14603201
>>> snp.id
'rs1041770'
\end{verbatim}
\end{leftindent}
These commands report
\begin{inparaenum}[(1)]
\item the chromosome on which this SNP lies,
\item the position of the SNP in base pairs, measured from the end of
  the chromosome, and
\item the ``rs-number,'' a name that uniquely identifies this SNP.
\end{inparaenum}
You might think that the rs-number was redundant, once we know the
SNP's position.  However, the position values change a little with
each new build of the data base, whereas the rs-numbers are invariant
from build to build.\footnote{Occasionally, rs-numbers do change.  This
  happens when it is discovered that two numbers actually refer to the
  same SNP.  Then the higher number is retired, and lower number
  becomes the official designation.}  Thus, the rs-number is essential
if you want to find a SNP that is discussed in the literature.

\texttt{hapmap\_snp} also contains all the genetic data for the SNP.
For example,
\begin{leftindent}
\begin{verbatim}
>>> snp.alleles
['G', 'T']
\end{verbatim}
\end{leftindent}
returns a list containing the two alleles that are present at this
locus.  In the original data file, the genotype of each individual
would have been represented as GG, GT, or TT.  In
\texttt{hapmap\_snp}, these genotypes are recoded as numbers:
\begin{leftindent}
\begin{verbatim}
>>> snp.gtype
[0, 0, 0, 0, 0, 0, 0, 0, 0, 0, 0, 0, 0, 0, 0, 0, 0, 0, 0, 0, 0, 0, 0,
 1, 1, 0, 0, 0, 0, 0, 0, 0, 0, 0, 0, 0, 1, 0, 0, 0, 0, 0, 0, 0, 0]
\end{verbatim}
\end{leftindent}
Each integer counts the number of copies of \texttt{alleles[0]} in one
individual.  In these data, allele~0 is G, so the integers 0, 1, and
2\label{pg.recoding} correspond to genotypes TT, TG, and GG.  Since
the genotypes are now integers, they have a mean and a variance, which
you can access like this:
\begin{leftindent}
\begin{verbatim}
>>> snp.mean
0.066666666666666666
>>> snp.variance
0.06222222222222222
\end{verbatim}
\label{pg.variance}
\end{leftindent}
Objects of type \texttt{hapmap\_snp} behave like lists.  The command
\texttt{snp[3]} returns the genotypic value at position~3,
\texttt{len(snp)} and \texttt{snp.sampleSize} both return the number
of genotypes in the data, and \texttt{sum(snp)} returns the sum of the
genotypic values.

In these examples, \texttt{snp} is the name of a variable that points
to an object of type \texttt{hapmap\_snp}.  But we could have used any
other name just as well.  For example,
\begin{leftindent}
\begin{verbatim}
>>> another_snp = hds[23]
>>> another_snp.alleles
['A', 'C']
\end{verbatim}
\end{leftindent}

\subsection{Some examples}
This section shows how \texttt{pgen}'s HapMap facilities can be used
to solve problems.  Read them carefully, and try them out in Python's
interactive interpreter.  They will all be useful later on.

\paragraph{Working with \texttt{gtype}}
In a miniature sample of 8 genotypes, \texttt{gtype} might look like
this: 
\begin{leftindent}
\begin{verbatim}
>>> snp.gtype
[1, 0, 0, 1, 1, 2, 0, 1]
\end{verbatim}
\end{leftindent}
Each number counts the copies of allele~0 in an individual genotype.
The sum of these values---in this case 6---is the number of copies of
this allele in the data set.  Since each genotype has two genes, the
total number of genes is $2\times 8 = 16$.  The frequency of allele~0
is therefore $p = 6/16$.  On the other hand, the average of the data
values is $\bar x = 6/8$.  The allele frequency is exactly half the
mean and is thus amazingly easy to calculate:
\begin{leftindent}
\begin{verbatim}
>>> p = 0.5*snp.mean
\end{verbatim}
\end{leftindent}
Just remember that this is the frequency of allele~0---the one that is
listed first in \texttt{snp.alleles}.  

\paragraph{Counting genes and genotypes}\label{pg.countgene}
As explained above, the number of copies of allele~0 is just the sum
of genotypic values:
\begin{leftindent}
\begin{verbatim}
>>> n0 = sum(snp)
\end{verbatim}
\end{leftindent}
What about the number of heterozygotes?  The variable
\texttt{snp.gtype} is a Python list.  As such, it has a built-in
method called \texttt{count} that will count the number of copies of
any given value within the list.  In our data, every heterozygote is
represented by the integer ``1.''  We can count the number of
heterozygotes with a single command:
\begin{leftindent}
\begin{verbatim}
>>> snp.gtype.count(1)
\end{verbatim}
\end{leftindent}
Here, \texttt{count} is a built-in method that would work on any
Python list. To calculate the frequency of heterozygotes:
\begin{leftindent}
\begin{verbatim}
>>> snp.gtype.count(1)/snp.sampleSize
\end{verbatim}
\end{leftindent}

\paragraph{Looping over SNPs}\label{pg.snploop} There are several ways
to loop over the SNPs in a \texttt{hapmap\_dataset}.  If you want to
include \emph{all} the SNPs, the code is very simple:
\begin{leftindent}
\begin{verbatim}
>>> for snp in hds:
...     print(snp.mean)
\end{verbatim}
\end{leftindent}
This works because objects of type \texttt{hapmap\_snp} behave like
lists.  Today's project involves making a graph.  If you tried to
graph all the SNPs, the graph would be incomprehensible.
Consequently, we will look only at a limited sample.  The easy way to
do this is as follows:
\begin{leftindent}
\begin{verbatim}
>>> for i in range(50):
...     snp = choice(hds)
...     print(snp.mean)
\end{verbatim}
\end{leftindent}
This loop includes 50 SNPs chosen at random from among all those in
the data.  The code makes use of the \texttt{choice} function, which
we introduced on p.~\pageref{pg.choice-intro} and discuss more fully
on p.~\pageref{pg.choice}.

\subsection*{Exercise}
Create a \texttt{hapmap\_dataset} for chromosome~22 of population
\texttt{JPT}.  From this dataset, select the SNP whose index is 83,
i.e.\ \texttt{hds[83]}.  Use this SNP to answer the following
questions:
\begin{inparaenum}[(1)]
\item What is the nucleotide state (A, T, G, or C) of allele~0?
\item How many copies of this allele are present?
\item What is its relative frequency?
\item What is the expected frequency of heterozygotes at
  Hardy-Weinberg equilibrium?
\item What is the observed frequency of hetetozygotes?
\item What is the position (in base pairs) of this SNP on the chromosome?
\end{inparaenum}
