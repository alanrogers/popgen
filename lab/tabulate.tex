\chapter{Tabulating Frequency Distributions}
\label{ch.frqdst}
\section{Introduction}
To see the pattern in data, we often need to summarize it in a compact
form. Tabulations are one such summary. They tell us how many
observations are in each of several categories. This appendix
describes two methods for tabulating data in Python, one for discrete
data and one for continuous data.

\section{Tabulating discrete data}
\label{sec.tabdiscrete}
Consider the following data set, which represents genotypes at a
single nucleotide site:
\begin{verbatim}
>>> gtype = ['AA', 'AA', 'AT', 'TA', 'TA', 'AA', 'TA']
\end{verbatim}
There are 7 observations in this data set, but all of them fall into 3
categories. We can use Python's \texttt{set} function to extract these
three categories: 
\begin{verbatim}
>>> for i in set(gtype):
...     print(i)
... 
AA
AT
TA
\end{verbatim}

In these data, we have both \texttt{AT} and \texttt{TA}. The
distinction between these ordinarily doesn't matter in genetics. To
convert all these into a common format, we can sort the characters
within each genotype as follows:
\begin{verbatim}
>>> gtype = ["".join(sorted(i)) for i in gtype]
\end{verbatim}
This uses a list comprehension (indicated by the square brackets) to
create a new list. Within the list comprehension, the builtin function
\texttt{sorted} turns each genotype into a sorted list of individual
characters. Then \verb|"".join| converts each list into a character
string. In the end, we have a data set with 7 observations but only
two genotypes, \texttt{AA} and \texttt{AT}:
\begin{verbatim}
>>> for i in set(gtype):
...     print(i)
... 
AA
AT
\end{verbatim}

To count the occurrences of each genotype in the data, we make use of
the fact that \texttt{gtype} is defined as a Python list, and lists
have a builtin method called \texttt{count}, which counts the
occurrences of its argument in the list. For example,
\begin{verbatim}
>>> gtype.count('AA')
3
\end{verbatim}
Here's a snippet of code that counts the occurrences of each genotype
in the data:
\begin{verbatim}
>>> h = {}
>>> for i in set(gtype):
...     h[i] = gtype.count(i)
... 
>>> print(h)
{'AA': 3, 'AT': 4}
\end{verbatim}
This defines a Python dictionary called \texttt{h}, which is initially
empty. Then we iterate across the genotypes, using the \texttt{count}
method to add observations to the dictionary. Finally, we print the
dictionary. The output says that the data contain 3 copies of
\texttt{AA}, and 4 of \texttt{AT}.

This is helpful, but the output is hardly pretty. It would be better
to print the output as a table, with genotypes in one column and
counts in another. Here's how:
\begin{verbatim}
>>> for key in sorted(h.keys()):
...     print(key, h[key])
... 
AA 3
AT 4
\end{verbatim}
Here, I've used the \texttt{keys} method that is built into Python
dictionaries to get a list of genotypes. I then hand this list to
\texttt{sorted}, which sorts the list of genotypes. We then iterate
across these genotypes, printing the genotype and the corresponding
count in two columns.

Suppose you had a list of genotypes, but you wanted to tabulate
alleles rather than genotypes. The trick is to begin by concatenating
all the genotypes into a single character string:
\begin{verbatim}
>>> dnaseq = ''.join(gtype)
>>> for i in set(dnaseq):
...     print(i, dnaseq.count(i))
... 
A 10
T 4
\end{verbatim}

\section{Tabulating continuous data}
\label{sec.tabcontinuous}

When working with continuous data, the number of different values may
be enormous. There is thus little point in counting the occurrences of
each value.  We need a different approach.  With continuous variables,
it makes more sense to group the data into bins.  This is tedious, so
we have written a Python module that does the heavy lifting.  It is in
a file called \texttt{pgen.py}, which you can download from the class
web site.  Put it in the same directory (or folder) as your other
Python programs.

Here is a program that tabulates continuous data into 5 bins:
\begin{listing}{1}
from pgen import Tabulator  # read module Tabulator

data = [0.39760402926232336, 0.38844722349063998, 0.15828823308128848,
        0.21675307512013373, 0.67759054634129579, 0.63336008432108437,
        0.7791838758913473, 0.11329205659594056, 0.088616376501101851,
        0.27797955173023731]

# Construct a Tabulator named tab.  The range [0,1] is divided into 5 
# bins, and the Tabulator counts the number of values within each
# bin. It also counts the values that fall above or below the range [0,1].

tab = Tabulator(low=0.0, high=1.0, nBins=5) 

# Each pass through the for loop examines one piece of data
for x in data:
    tab += x     # Add 1 to the count in the relevant bin

print(tab)       # print the tabulated data
\end{listing}
