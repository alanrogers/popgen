\chapter{Heterozygosity of HapMap SNPs}
\label{ch.haphet}
\setcounter{cenumEnumi}{0}
Heterozygosity is perhaps the most fundamental measure of genetic
variation.  In this project, we'll study the heterozygosity of SNPs within
the HapMap data base.  For each SNP, we'll calculate not only the
observed heterozygosity, but also the value expected at Hardy-Weinberg
equilibrium.  We'll be interested to see how well the two numbers
agree. 

The exercise will involve modifying the incomplete code below.  You
can download it from the lab website, where it is called
\texttt{haphetinc.py}.
\begin{leftindent}
\listinginput[5]{1}{haphetinc.py}
\end{leftindent}
The guts of this little program is the loop in lines~8--14.  Each time
through the loop, we choose a SNP at random from the data set.  Then
we set the observed (\texttt{ohet}) and expected (\texttt{ehet})
heterozygosities, and print these values.  Only, as you can see, the
present code just sets these values to 0.

\section*{Exercise}
\begin{cenum}
\item Change lines~11-12 so that they calculate the observed and
  expected heterozygosities of each SNP.
\end{cenum}

So far, your program merely lists the expected and observed
heterozygosity values.  It is hard to get a good sense of these
values, just by staring at the numbers.  You'll appreciate the pattern
better after making a scatter plot.  Before you can do this, you must
first get the data values into lists.

\section{Storing values in a list} 
\label{sec.initlist}
As you calculate values of \texttt{ohet} and \texttt{ehet}, you will
need to store them in a list. Before you can do so, the list must
first exist.  Thus, you will need to create an empty list before the
main loop in your program.  For example, you might use a command like
\verb|obs_vals = 3*[None]| to create a list with room for three
observed values.  Here, the keyword ``\texttt{None}'' is Python's way
of representing an unknown value.  In subsequent commands, you replace
these with real values.  Here is how it works:
\begin{leftindent}
\begin{verbatim}
>>> obs_val = 2*[None]
>>> obs_val
[None, None]
>>> obs_val[0] = 3.0
>>> obs_val[1] = 111
>>> obs_val
[3.0, 111]
\end{verbatim}
\end{leftindent}
Below, you will create two lists (one for \texttt{ohet} and one for
\texttt{ehet}) to store the values you calculate within the loop in
the program above. Your lists should be defined \emph{before} the loop
and should be long enough to hold all the values you will calculate
within the loop.

It is also possible to start with an empty list:
\begin{leftindent}
\begin{verbatim}
>>> obs_val = []
>>> obs_val
[]
\end{verbatim}
\end{leftindent}
Then you can add values using the \texttt{append} method, which works
with any list:
\begin{leftindent}
\begin{verbatim}
>>> obs_val.append(3.0)
>>> obs_val.append(111)
>>> obs_val
[3.0, 111]
\end{verbatim}
\end{leftindent}
Either way, you end up with the same answer. The first method is
faster with large jobs, but the second is easier.  Feel free to use
either approach.

\section{Making a scatter plot}
\label{sec.scatplot}
There are several Python packages for making high-quality graphics.
If you are working on your own machine at home, you may want to
download either \texttt{matplotlib} or \texttt{gnuplot.py}.
Unfortunately, none of these packages is available in the Mac Lab at
Marriott Library.  As a partial solution to this problem, we have
implemented a crude graphics function within \texttt{pgen.py}.  It
constructs a scatter plot out of ordinary text characters---a style of
graphics that has not been common since the 1970s.

The function \texttt{charplot} is described on
page~\pageref{pg.charplot}.  Here is a listing that illustrates its
use:
\begin{leftindent}
\begin{verbatim}
from pgen import charplot

x = [0, 1, 2, 3,  4,  5,  6,  7,  8,  9]
y = [0, 1, 4, 9, 16, 25, 36, 49, 64, 81]

charplot(x, y)
\end{verbatim}
\end{leftindent}
When this program runs, it makes a scatter plot of the $(x_i, y_i)$
values, with $x$ on the horizontal axis and $y$ on the vertical.  Try
it yourself.  The function accepts several optional arguments, which
you can use if you wish to control the number of tick marks or the
dimensions of the plot.  See page~\pageref{pg.charplot} for details.

\section*{Exercise}
\begin{cenum}
\item Instead of printing the heterozygosity values, put them into two
  lists: one for the expected values and the other for the observed
  values.  At the end of the loop, make a scatter plot, with expected
  heterozygosity on the horizontal axis and the observed value on the
  vertical axis.
\item Use your program to produce a graph.  Experiment to find the
  best number of SNPs.  With a ruler and pencil, draw a line on the
  graph representing the function $y(x) = x$.  (To help you draw this
  line, you may want to append a dummy pair of data values such as
  $(0.7, 0.7)$ before calling \texttt{charplot}.)  If observed values
  always equaled expected ones, all points would fall on this line.
  Write a short paragraph describing the pattern you see.  Is there
  good agreement between observed and expected values, or do you see a
  discrepancy? Does variation about the expected value increase with the
  expected value, or does it decrease or stay the same? Why might this
  be? 
\end{cenum}

\paragraph*{What to hand in} Your answers to step~1, the final version
of your program, the graph, and a paragraph discussing the results.


