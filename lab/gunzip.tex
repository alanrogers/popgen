\section{Uncompress the data files}
At this point, the file will be on the hard disk of your own computer.
Your next task is to find it.  Check the folders ``Desktop,'' and
``Documents.''  On Linux, check your home directory.  Once you find
the file, you will need to uncompress it.  HapMap delivers it in a
compressed format in order to speed the download process.  (That is
the significance of the ``.gz'' in the file name.)  If the file shows
up as an icon on your Desktop or file browser, try clicking on the
icon.  This will probably bring up a dialog box that will allow you to
uncompress the file.

On the Mac, be careful \emph{not} to click on this file repeatedly.
If you do, each click will generate an additional copy of the
uncompressed file.  These additional files will have names that end in
``\texttt{-2},'' ``\texttt{-3},'' and so on.  You will need to delete
these extraneous files before you can proceed.

If you are unable to uncompress the data file by clicking on it, here
is an alternative method:
\begin{description}
\item[Mac] Open a command-line interpreter, which is called a
  ``terminal'' on the Mac operating system.  Then use the \texttt{ls}
  and \texttt{cd}   commands to navigate to the directory where the
  .gz file resides, and type \texttt{gunzip filename.gz} at the prompt.

\item[Linux] The same, except that the command-line interpreter is
  sometimes called a shell window.

\item[Windows] The \texttt{gunzip} facility does not come with
  Windows, but you can download it for free from \url{www.gzip.org}.
  After that, the procedure is the same as for Mac or Linux, except
  that the command-line interpreter is called a command window, and
  you use \texttt{dir} instead of \texttt{ls}.  If you prefer a
  point-and-click interface, look into the commercial program WinZip.
\end{description}
If you decide to download \emph{all} the HapMap data files, you can
uncompress them all with a single command:
\begin{leftindent}
\begin{verbatim}
gunzip *.gz
\end{verbatim}
\end{leftindent}
This is \emph{much} faster than pointing and clicking on the files one
at a time.
