\chapter{Downloading HapMap data}
\label{ch.gettingdata}
\section{Interacting with the HapMap ftp server}
The HapMap data are available via an ftp site at NCBI. There is a link
to that site under the ``lab'' page of the course website. Or to go
directly there, point your web browser at
\begin{center}
  \small
  \url{https://ftp.ncbi.nlm.nih.gov/hapmap/genotypes/2008-10_phaseII/fwd_strand/non-redundant}
\end{center}  
This will bring you to a list of files with names like
\url{genotypes_chr22_JPT_r24_nr.b36_fwd.txt.gz} In these file names,
the string ``\texttt{chr22\_JPT}'' means ``chromosome~22'' in
population ``\texttt{JPT}'' (Japan).  Scroll down the list to the file
you need.  Then right-click and select ``\texttt{Save link as}'' in
order to download the file onto your own computer.

The file will end with ``\texttt{.gz},'' which indicates that it has
been compressed using the \texttt{gnuzip} program. You don't need to
decompress it, because \texttt{pgen.py} can read files that are
compressed in this way.

\section{The dummy data file}
\label{sec.gettingdummy}

As you are debugging your programs, you must execute it after each
edit.  This can be a slow process unless the program executes very
fast.  Unfortunately, even the smaller HapMap data files are pretty
large, and parsing them repeatedly can slow down the process of
programming. 

To solve this problem, we have created a dummy HapMap data file called
\url{genotypes_chr99_CEU_r23a_nr.b36_fwd.txt}.  It contains only a
small number of SNPs, and your program can parse it almost instantly.
We encourage you to use this file rather than any real data until you
get your code running.  To do so, place the file in the same folder
with your other programs and specify
\begin{leftindent}
\begin{verbatim}
chromosome = 99
pop = 'CEU'
\end{verbatim}
\end{leftindent}
The dummy data file is under \texttt{Data} on the lab page of the
class web site.

\section{Put the data files into the folder where you plan to use them}
Once the data file is on your machine, you need to figure out where
(in which folder) to put it.  If you work on your own machine, it's
probably best to create a separate folder.  On the Marriott Macs, it
makes more sense just to work on the desktop.  Whatever you decide,
make sure that the folder exists and that it contains your HapMap data
file along with a copy of \texttt{pgen.py}.  Then launch Idle, get
into the interactive window, and try importing \texttt{pgen}.  To do
so, just type ``\texttt{from pgen import *}''.  If this works, nothing
will print.  If you get an error message, then Idle was unable to find
the file \texttt{pgen.py}.  To fix this problem, see
section~\ref{sec.cwd} on page~\pageref{sec.cwd}.

\section{Downloading all the HapMap files}
On your home machine, you may want to download \emph{all} of the
HapMap files, for your own personal use.  You would not want to do
this by pointing and clicking on each file, one after the other.  It
is much easier to use the program \texttt{ftp}, and connect to
\url{ftp.hapmap.org}.  If you want to do this, we'll be happy to
show you how.

