\chapter{Using Simulation to Test a Statistical Hypothesis}
\label{ch.stattest}
\setcounter{cenumEnumi}{0}
In Project~\ref{ch.coal1} you used a coalescent simulation to evaluate
a hypothesis about $2N$ and $u$, two parameters that play central
roles within population genetics.  Your evaluation was subjective
because we did not ask for any formal statistical test.  In this
project, you'll do it right.  But first, you need to learn some basic
principles.  We begin with two definitions.

\section{A statistic and its sampling distribution}

All data have noise.  They are noisy because of measurement error,
because of sampling error, and because of unpredictable factors
involved in the phenomena we study.  This is why we need statistical
methods in the first place.  Things that we calculate from data are
called \emph{statistics}.  Familiar examples include the mean and
variance of a sample.  Less familiar examples include estimates of the
site frequency spectrum and the mismatch distribution.  When a gene
genealogy is estimated from data, even that is a statistic.  Because
they are based on noisy data, statistics are noisy too.  In this
context, ``noise'' refers to random variation.  Scientists deal with
such randomness using probability theory, and you should be familiar
with that before reading further.  (See \emph{Just Enough
  Probability}.)

Scientific hypotheses are often tested by comparing a statistic to its
sampling distribution.  To get this idea across, we re-visit a simple
problem: that of tossing a coin.  As you learned in \emph{Just Enough
  Probability}, John Kerrich tossed a coin 10,000 times and observed
5070 heads.  This number is a statistic.  How could we use it to test
a hypothesis?

Suppose we suspected that Kerrich's coin was slightly unfair, that the
probability $(p)$ of heads was 0.45 on each toss.  To test this idea,
we would need to know whether the value that Kerrich observed is
unusual in similar experiments for which $p$ really does equal 0.45.
Suppose that we somehow managed to manufacture such a coin and then
used it in 100,000 repetitions of Kerrich's experiment.  Suppose in
addition that in none of these was the number of heads as large as
Kerrich's value, 5070.  Then Kerrich's result would clearly rank as
unusual, and this might lead us to doubt the original hypothesis that
$p=0.45$.  This is the reasoning that usually underlies tests of
statistical hypotheses.  The difficulty is that usually (as here)
these repetitions are impossible to carry out.  There is no way to
manufacture a coin for which $p$ is exactly 0.45.  Even if that were
possible, who would volunteer to toss that coin $100,000 \times
10,000 = \hbox{1 billion}$ times?

Nonetheless, let's pretend that we had really done 100,000 repetitions
of Kerrich's experiment under conditions that guarantee that $p=0.45$.
We could calculate the number of heads from each repetition and then
summarize these values using a \emph{frequency distribution}.  Since
the number of repetitions was so large, this frequency distribution
would approximate a special sort of probability distribution called a
\emph{sampling distribution}.

The sampling distribution of a statistic is its probability
distribution, but only in a restricted sense: it is the probability
distribution implied by a particular hypothesis.  Different hypotheses
imply different sampling distributions, even for the same statistic.
In our example, we assumed that $p=0.45$.  A different hypothesis
would imply a different sampling distribution.

In short, \emph{a sampling distribution is the probability
  distribution of a statistic, as implied by a particular hypothesis}.

\section{Using sampling distributions to test hypotheses}
Let us postpone the question of how one gets a sampling distribution.
For the moment, our focus is on using one to test a hypothesis.  It is
sometimes useful to distinguish between a random variable and the
particular values it may take.  In this section, we use upper case for
the former and lower case for the latter.  Thus, $X$ represents a
random variable and has a probability distribution, but $x$ is just a
number.

% -*-latex-*-
\begin{figure}
\begin{center}
\mbox{\beginpicture
\setcoordinatesystem units <3in, 0.3in> point at 0 0
\setplotarea x from 0 to 1, y from 0 to 10
\axis left /
\axis right /
\axis top /
\axis bottom label {$\hat p$} ticks numbered from 0.00 to 1.00 by 0.25 /
\plot
0.6 0.00003716798785
0.6200000000 0.0003995935275
0.6400000000 0.003345755643
0.6600000000 0.02181706737
0.6800000000 0.1107962103
0.7000000000 0.4382075122
0.7200000000 1.349774162
0.7400000000 3.237939891
0.7600000000 6.049268110
0.7800000000 8.801633164
0.8000000000 9.973557006
0.8200000000 8.801633164
0.8400000000 6.049268110
0.8600000000 3.237939891
0.8800000000 1.349774162
0.9000000000 0.4382075122
0.9200000000 0.1107962103
0.9400000000 0.02181706737
0.9600000000 0.003345755643
0.9800000000 0.0003995935275
01.000000000 0.00003716798785
/
\multiput {$\bullet$} at
0.60000000 0
0.61000000 0.001
0.62000000 0.001
0.63000000 0.001
0.64000000 0.008
0.65000000 0.019
0.66000000 0.038
0.67000000 0.085
0.68000000 0.173
0.69000000 0.241
0.70000000 0.564
0.71000000 0.888
0.72000000 1.382
0.73000000 2.154
0.74000000 3.235
0.75000000 4.402
0.76000000 5.703
0.77000000 7.283
0.78000000 8.548
0.79000000 9.638
0.80000000 9.778
0.81000000 9.755
0.82000000 9.159
0.83000000 7.713
0.84000000 6.418
0.85000000 4.776
0.86000000 3.289
0.87000000 2.189
0.88000000 1.266
0.89000000 0.702
0.90000000 0.354
0.91000000 0.155
0.92000000 0.053
0.93000000 0.02
0.94000000 0.006
0.95000000 0.003
0.96000000 0
/
\endpicture}
\end{center}
\caption{Normal approximation to sampling distribution}
\small The solid line shows the normal approximation to the sampling
distribution of $\hat p$ in the case where $p=0.8$ and $N=100$.  The
bullets are copied from the the corresponding curve in
figure~\ref{fig.phat}, and show the true sampling distribution of
$\hat p$ as estimated by computer simulation.
\label{fig.norm}
\end{figure}


Figure~\ref{fig.norm} shows two ways to graph the sampling
distribution of a hypothetical statistic, $X$.  In the left panel, the
vertical axis shows the \emph{probability density function}, $f$.
Areas under the curve correspond to probabilities in the sampling
distribution of $X$.  The shape of this particular density function
may look familiar, but that has no relevance here.  Sampling
distributions can have any shape.

In the density function in the left panel, the two ``tails'' of the
distribution are shaded. These tails comprise only a small fraction of
the total area under the curve. For this reason, the observed value of
our statistic is unlikely to fall within either tail. If it does, then
either (a)~we have observed an unusual chance event, or (b)~our
hypothesis and its implied sampling distribution are incorrect.  When
this happens, we say that the hypothesis has been rejected.

This does not mean it is wrong, for correct hypotheses may be rejected
too.  The probability of such an error is called $\alpha$ and is equal
to the shaded area of the graph.  To minimize these errors, we usually
choose small values of $\alpha$, such as 0.05 or 0.01.  In
figure~\ref{fig.norm}, we have set $\alpha$ to a very large value
(0.2) so that the shading is easy to see.

The right panel of Figure~\ref{fig.norm} shows another way to graph
the same sampling distribution.  The vertical axis shows $F(x)$, the
probability of observing a value less than or equal to $x$.  This is
called the \emph{cumulative distribution function}.  For example, in
the figure's left panel, a fraction 0.1 of the area under the curve
lies to the left of $x_0$ and a fraction 0.9 lies to the left of
$x_1$.  Consequently, the right panel shows that $F(x_0) = 0.1$ and
that $F(x_1)=0.9$.

We explained above how $f$ can be used to test a statistical
hypothesis: if the observed value falls within the shaded tails, then
we reject.  This idea can be re-expressed in terms of $F$ because $F$
and $f$ contain the same information.  In terms of $F$, we reject if
\begin{equation}
F(x\obs) \leq \alpha/2, \qquad \mbox{or} \qquad
F(x\obs) \geq 1 -\alpha/2,
\label{eq.stattest}
\end{equation}
where $x\obs$ is the observed value of the statistic, and $\alpha$ is
the level of significance.\footnote{Consider first the left shaded
  tail.  This tail contains half the shaded area and thus has
  probability $\alpha/2$.  This means that $F(x_0) = \alpha/2$.  If
  $x\obs$ falls within this tail, then it must be true that $F(x\obs)
  \leq \alpha/2$.  We can reject any hypothesis for which this condition
  is true.  Applying the same reasoning to the right shaded tail, we
  reject if $F(x\obs) \geq 1-\alpha/2$.}  This is just another way of
saying that the observed value fell into one of the tails. 

In testing a statistical hypothesis, one can work either with the
density function $(f)$ or with the cumulative distribution function
$(F)$, but the latter approach is often easier.  We don't need to know
the whole sampling distribution---just a single value, $F(x\obs)$.  As
we shall see, this value is easy to estimate by computer simulation.

\section{Sampling distributions from computer simulations} 
Long ago there was only one way to figure out the sampling
distribution implied by a hypothesis---it required a sharp pencil and
a mathematical turn of mind.  That approach doesn't always work
however, even for gifted mathematicians.  Fortunately there is now
another way: we can estimate sampling distributions from computer
simulations.

In an informal way, you have done this already.  In
Project~\ref{ch.coal1}, you generated several simulated values of a
genetic statistic $(S)$ under a particular hypothesis about $u$ and
$2N$.  You then compared these simulated values to the one $(S\obs =
82)$ that Jorde obtained from real data.  Near the end of that
assignment, we asked: ``How many of the simulated values are smaller
than 82?  How many are larger?''  These are questions about
$F(S\obs)$.  We then asked: ``Does the observed value represent an
unusual outcome, or is it pretty typical of the simulated values?''
In answering that question, you used the same logic that we explained
above.  The approach is more formal this time, but the idea is the
same.

To see how this works in detail, let us return to our hypothesis about
Kerrich's coin.  Our statistic $(x\obs)$ is the number of heads in
10,000 tosses, which equalled 5070 in Kerrich's data.  We are
interested in $F(x\obs)$ under the hypothesis that $p=0.45$.  The
conditions of Kerrich's experiment imply that the number of heads is
drawn from a binomial distribution.  This distribution has two
parameters: the number $(N)$ of trials and the probability $(p)$ of
``heads'' on each trial.  Our hypothesis assumes that $p=0.45$, and we
can take $N=10,000$, because that is how many times Kerrich tossed the
coin.  We need to calculate $F(5070)$ from a binomial distribution
with these parameter values.

\begin{table}
\begin{center}
\begin{tabular}{cccc}
Simulation & $x$ & Simulation & $x$ \\
\hline
 1 & 4439 &  6 & 4508\\
 2 & 4474 &  7 & 4521\\
 3 & 4484 &  8 & 4527\\
 4 & 4489 &  9 & 4558\\
 5 & 4495 & 10 & 4566
\end{tabular}
\end{center}
\caption{The number $(x)$ of heads in ten draws from the binomial
  distribution with $N=10,000$ and $p=0.45$.}
\label{tab.cointest}
\end{table}

We could do this calculation by summing across the binomial
distribution function, but we'll do it here by computer simulation.
The procedure is simple: over and over, we simulate Kerrich's
experiment under the conditions of our hypothesis.  Each simulation
generates a value of $x$, the number of heads.
Table~\ref{tab.cointest} shows the result of 10 such simulations.  In
every one, the simulated value of $x$ is smaller than $x\obs = 5070$,
the value that Kerrich observed.  Already we begin to suspect that our
hypothesis is inconsistent with the data.  Let us relate this to the
probability, $F(x\obs)$, discussed above.

As you know, probabilities are estimated by relative frequencies.  We
can estimate $F(x\obs)$ as the relative frequency of observations such
that $x \leq x\obs$.  Every one of the 10 observations in
Table~\ref{tab.cointest} satisfies this condition.  Thus, we estimate
$F(x\obs)$ as $10/10 = 1$.  Our sample is small, so this may not be a
very good estimate.  But let us take it at face value for the moment.
To test a hypothesis, we must first specify $\alpha$, so let us set
$\alpha=0.05$.  With this significance level,
equation~\ref{eq.stattest} says that we can reject if $F(x\obs)
\leq 0.025$ or $\geq 0.975$.  Our estimate is $F(x\obs) = 1$, which is
clearly greater than 0.975.  Thus, we reject the hypothesis---or we
would do so if we trusted our estimate of $F(x\obs)$.

A convincing answer will require many more simulations, a potentially
tedious project.  Here is a program called \texttt{cointest.py}, which
removes the tedium: \listinginput[5]{1}{cointest.py} In this code, each
execution of line~10 repeats Kerrich's experiment, using the function
\texttt{bnldev}.  (This function is described in
Project~\ref{ch.drift} and documented more fully on
p.~\pageref{pg.bnldev} of appendix~\ref{sec.randnum}.)  Lines~11--12
count the number of such repetitions for which the outcome is less
than or equal to the observed value 5070, and line~14 converts the
resulting count into a fraction.  Download this code from the class
web site and run it.  You should get something that looks like this:
\begin{verbatim}
F[5070] =  1.000 for hypothesis: p=0.45
\end{verbatim}
The critical piece here is the value 1.000.  This is the fraction of
simulations for which the simulated value was less than or equal to
Kerrich's value, 5070.  It estimates $F(x\obs)$.  Even in this larger
sample, every simulation resulted in fewer heads than Kerrich saw.  If
our hypothesis about $p$ is correct, then Kerrich saw something
remarkable.  This is probably not what happened.  It seems more likely
that our hypothesis about $p$ is way off the mark.

This estimate of $F(x\obs)$ is a lot more reliable than the previous
one, because it is based on 100 simulations rather than 10.  It would
be more accurate still if we had done 10,000.  We did not start with
10,000, because it's wise to keep the number small until you get the
code running.  Try increasing \texttt{nreps} to 10,000.  Does it makes
a difference?

\subsection*{Exercise}
\begin{cenum}
\item Revise your coalescent code from Project~\ref{ch.coal1} so that
  it tests the same hypothesis ($u=0.001$ and $2N=5000$) that we
  considered there, using the same data ($S\obs = 82$, $K=77$).  To do
  this, use \texttt{cointest.py} as your model.  Simply replace
  line~10 with the simulation that you wrote in
  Project~\ref{ch.coal1}.  Like \texttt{cointest.py}, your program
  should produce a single line of output.  Show us the program and its
  output, and write a sentence or two saying whether and why the
  hypothesis can be rejected.
\end{cenum}
The tricky parts of this exercise are getting the indentation right
and making sure that each variable is initialized where it needs to
be.  

Before proceeding, make sure you understand what this program does.
In principle, it is exactly what you did in Project~\ref{ch.coal1}.
The program does the same simulation many times and allows you to
tell whether the observed value, $S\obs = 82$, is unusually large,
unusually small, or typical, under a particular evolutionary
hypothesis.

\section{Confidence intervals}
In the preceding section, we tested a single hypothesis about the
value of $p$.  It would be more interesting to examine a range of $p$
values in order to see which can be rejected and which cannot.  Here
is a modified program called \verb|coin_ci.py|, which does this:
\label{pg.coinci}\listinginput[5]{1}{coin_ci.py} Notice how little this
differs from \texttt{cointest.py}.  Line~6 has been re-written, and
lines~7--16 have been shifted to the right.  We also bumped
\texttt{nreps} up to 1000.  Apart from these minor changes, the two
programs are identical.  These changes create an additional loop,
which does a separate test for each of the values of $p$ listed on
line~6.  Make sure you understand why the values of \texttt{xobs} and
\texttt{nreps} are set outside this loop, and why that of \texttt{F}
is set inside it.

Download this program and get it running.  Your output should look
similar to this:
\begin{verbatim}
F[5070] =  1.000 for hypothesis: p=0.490
F[5070] =  0.995 for hypothesis: p=0.495
F[5070] =  0.921 for hypothesis: p=0.500
F[5070] =  0.675 for hypothesis: p=0.505
F[5070] =  0.259 for hypothesis: p=0.510
F[5070] =  0.067 for hypothesis: p=0.515
F[5070] =  0.004 for hypothesis: p=0.520
F[5070] =  0.001 for hypothesis: p=0.525
\end{verbatim}
Each line of this output tests a different hypothesis.  We can reject
the first two (at $\alpha=0.05$) because $F(x\obs) \geq 0.975$.  We
can reject the last two because $F(x\obs) \leq 0.025$.  The hypotheses
we \emph{cannot} reject include all values within the range $0.5 \leq
p \leq 0.515$.  These values---the ones we cannot reject---are called
the \emph{confidence interval} of $p$.

There is a little ambiguity here.  We can reject $p=0.495$ but not
$p=0.5$, so the lower boundary of the confidence interval must lie
somewhere between these values.  Exactly where, we cannot say.  There
is a similar ambiguity involving the upper boundary.  It is safe
however to say that the confidence interval is enclosed by the
slightly larger interval $0.495 < p < 0.52$.  The limits (0.495 and
0.52) of this larger interval are clearly \emph{outside} the
confidence interval, for we were able to reject them.  This is why we
use the strict inequality symbol $(<)$.  To be conservative, it is
good practice to use this larger interval when reporting a confidence
interval.

\subsection*{Exercise}
\begin{cenum}
\item Revise your coalescent code using \verb|coin_ci.py| as a model,
  and estimate the confidence interval of $2N$.  Show us your code,
  the output, and a brief explanation.  Your explanation does not have
  to be long.  Just explain what you have learned about the parameter
  $2N$.  Which hypotheses about $2N$ can we reject, and which are
  still on the table?
\end{cenum}

At this point, we hope you can see (a)~that testing a statistical
hypothesis just amounts to asking whether your observation is unusual,
and (b)~that a confidence interval is just a summary of hypothesis
tests.  The methods of statistics are complex, but the underlying
ideas are simple.

\section{$p$-hacking}
This lab has introduced the classical approach to testing a
statistical hypothesis. Although such tests play an important role in
science, they can also mislead. To see how, we encourage you to spend
some time with this website:
\begin{verbatim}
https://fivethirtyeight.com/features/science-isnt-broken
\end{verbatim}
The interactive tool there will enable you to prove either that
Democrats are good for the economy or that Republicans are, depending
on your own preconceptions.

That bit is discouraging, but keep reading. The second part of the
article describes a study in which 29 teams of scientists each used
identical data in an effort to answer the same question: do referees
give more red cards to dark-skinned soccer players than to
light-skinned ones? The teams of scientists used different methods and
got different answers. Yet if you focus on the results that were
statistically significant, you'll find that they tell a consistent
story.
