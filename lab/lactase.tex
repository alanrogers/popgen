\chapter{Linkage Disequilibrium Near the Human Lactase Gene}
\label{ch.lactase}
\setcounter{cenumEnumi}{0}
Most mammals are able to digest milk only during infancy.  Later in
life, their bodies stop producing the enzyme \emph{lactase} and are
thus unable to digest milk sugar (\emph{lactose}).  Some humans
however continue to make lactase throughout life, a condition called
\emph{lactase persistance}.  (It's opposite is \emph{lactose
  intolerance}.) This condition is common in European (and some
African) populations and seems to result from a single point-mutation
in the promoter region of the lactase gene.  Todd Bersaglieri and his
colleagues (\emph{Am.\ J.\ Hum.\ Genet.}, 74:1111--1120, 2004) argue
that the European mutant arose a few thousand years ago and has been
strongly favored by natural selection.  Their evidence for this comes
from the amount of linkage disequilibrium (LD) in this genomic region.
In this project, you'll see for yourself.  You will estimate LD in the
region around the putative advantageous mutation and then compare this
estimate to the LD on chromosome~2 as a whole.

\section{An incomplete program}
As usual, we provide code that does much of the work.  You will find
this code on the class web site in a file called
\texttt{lactaseinc.py}.  Here is what it looks like:
%
\listinginput[3]{1}{lactaseinc.py}
%
Lines~4--7 define several parameters.  The parameter \texttt{reach}
determines determines the size of the window within which you will
estimate LD.  In this initial version of the program, the window
reaches from 5000 bases to the left of the focal SNP to 5000 bases to
the right.  (That is why it is called ``\texttt{reach}.'')  The
chromosome number is set to 99, so that we can use the dummy data file
during debugging.  (See page~\pageref{sec.gettingdummy}.)  In the end,
you'll want to set \texttt{chromosome = 2} in order to analyze the
real data.  Line~7 identifies the SNP that is thought to be under
selection.  This SNP was identified in the Bersaglieri paper mentioned
above.  It's label is ``rs4988235,'' and in our data it lies at
position specified on line~7 of the code above.  (This number may
change in future versions of HapMap, but the label should remain the
same.)

In the HapMap data for this target SNP, one of the genotypes has a
missing value. Our pgen software eliminates SNPs with missing values,
so this site will not appear in our data. For this reason, the call to
\verb|hds.find_position(focal_position)| will not find the target
SNP. Instead, it will find the nearest SNP. That's OK, because there
is so much LD in this region that SNPs in LD with the target SNP will
also be in LD with this nearest SNP.

Lines~9--11 define the stub of a function that is supposed to
calculate $r^2$---except that it doesn't in this incomplete code.
Replace this function definition with your own version from
Project~\ref{ch.rsq}.  The function \verb|window_rsq| is new.  It takes
three arguments, \verb|hds| (the data set), \texttt{mid} (an index
into that data set), and \texttt{reach} (discussed above).  The
function compares a central SNP (\verb|hds[mid]|) to each SNP in the
region around it.  If no other SNPs are found in this region, the
function returns \texttt{None}.  Otherwise it returns the average
value of $r^2$ between the SNPs in this region and the central
SNP. The calls to \verb|find_position| will seem mysterious, since you
have not yet seen this function used.  If you are curious, see
page~\pageref{pg.findposition}.

In line~35, we are done defining things and ready to do real work.  In
rapid succession, we define the dataset (\texttt{hds}), find the index
(\texttt{focndx}) of the ``focal SNP'' (the one that is supposedly
under selection), and calculate the mean value (\verb|rsq_mean_obs|)
of $r^2$ within the region around this focal SNP.

\subsection*{Exercise}
\begin{cenum}
\item Download the relevant files for the current project.  You will
  need:
\begin{inparaenum}
\item \texttt{pgen.py},
\item the incomplete Python program \texttt{lactaseinc.py},
\item the HapMap data file for chromosome~2 in the European population
  (\texttt{CEU}), and 
\item the dummy data file for ``chromosome 99.''
\end{inparaenum}
For instructions on downloading HapMap data files, see
appendix~\ref{ch.gettingdata}.  For instructions on getting the dummy
data file, see page~\pageref{sec.gettingdummy}.
\item Replace lines~10--11 with your own working copy of the
  \texttt{get\_rsq} function.  You'll need the \texttt{cov} function
  too. You won't need the \texttt{var} function, because variances
  have already been calculated for you (see
  pages~\pageref{pg.variance} and~\pageref{pg.snp.variance}).
\end{cenum}
At this point, your program should calculate the mean value of $r^2$
in the region surrounding the putatitive selected SNP.
\begin{cenum}
\item Add code to the end of the program that calculates 
\begin{enumerate}
\item the mean $r^2$ value within 100 randomly selected
  regions on chromosome~2,
\item the fraction of these regions whose mean $r^2$ is greater than
  or equal to the ``observed value,'' i.e.\ the value for the region
  surrounding the putative selected SNP.
\end{enumerate}
\end{cenum}
Hint: To calculate $r^2$ in randomly selected regions, choose random
SNPs as we did in Project~\ref{ch.rsq}, and then hand each one to function
\verb|window_rsq|.  To calculate the fraction of these values that is
greater than or equal to the observed value, proceed as we did in
Project~\ref{ch.stattest}.
\begin{cenum}
\item Now it is time to analyze real data, so set \texttt{chromosome}
  equal to 2.  Then run the program with \texttt{reach} set to each of
  the following values: 20,000, 200,000, and 1,000,000.
  These runs will take longer, so be patient.
\item Write a paragraph discussing the results.  Is the LD around the
  lactase gene unusual on chromosome~2? If so, how far does the region
  of unusual LD seem to reach? The answer to this question should be
  based on the tail probabilities you calculated for the three values
  of \texttt{reach}.
\end{cenum}

