\chapter{Using Tabulator to Tabulate Data}
\label{ch.tabulator}
Tabulator is a python class for tabulating numerical data.  Before
using it, you must get a copy of the file ``pgen.py,'' which defines
the Tabulator module.

\begin{verbatim}
from pgen import Tabulator

# A dummy data set consisting of numbers between 0.0 and 1.0.

data = [0.33236342968786442, 0.73055958580640823, 0.34100448055922677, 
        0.93062544008569004, 0.57670398113446419, 0.51266858597089471]

tab = Tabulator()     # construct a new tabulator
for x in data:        # tabulate the data
    tab += x

print(tab)            # print the tabulation
\end{verbatim}
The output looks like this:
\begin{verbatim}
Range                     Count     Freq
[     -inf ...   0.0000]      0   0.0000
[   0.0000 ...   0.2000]      0   0.0000
[   0.2000 ...   0.4000]      2   0.3333
[   0.4000 ...   0.6000]      2   0.3333
[   0.6000 ...   0.8000]      1   0.1667
[   0.8000 ...   1.0000]      1   0.1667
[   1.0000 ...      inf]      0   0.0000
Total                         6   1.0000
\end{verbatim}

By default, Tabulator groups observations into five bins, which span
the range between 0 and 1. There are two additional bins for
observations less than 0 and greater than 1. You can change these
defaults in the call to the Tabulator constructor. For example, we
could define \textt{tab} as follows
\begin{verbatim}
tab = Tabulator(low=0.2, high=0.8, nBins=4)
\end{verbatim}
to tabulate the range $[0.2, 0.8]$ into 4 bins. Using the data above,
we would end up with
\begin{verbatim}
Range                     Count     Freq
[     -inf ...   0.2000]      0   0.0000
[   0.2000 ...   0.3500]      2   0.3333
[   0.3500 ...   0.5000]      0   0.0000
[   0.5000 ...   0.6500]      2   0.3333
[   0.6500 ...   0.8000]      1   0.1667
[   0.8000 ...      inf]      1   0.1667
Total                         6   1.0000
\end{verbatim}
