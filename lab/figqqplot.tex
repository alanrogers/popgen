% -*-latex-*-
\begin{figure}
\begin{center}
\mbox{\beginpicture
  \headingtoplotskip=0.5\baselineskip
  \valuestolabelleading=0.4\baselineskip
%%%%%%%%%%%%%%%%% Left Plot %%%%%%%%%%%%%%%%%%%%%%%%%%%%%%%%%%%%%%%%%%%
\setcoordinatesystem units <0.3\textwidth,0.3\textwidth> point at 1 0
  \setplotarea x from 0 to 1, y from 0 to 1
  \axis bottom label {Quantiles of $X$} 
   ticks withvalues 0 {2000} {4000} {6000} {8000} {10000} / 
   at 0 0.2 0.4 0.6 0.8 1.0 / /
  \axis left label {\stack{Quantiles\cr of\cr $Y$}} 
   ticks withvalues 0 {2000} {4000} {6000} {8000} {10000} / 
   at 0 0.2 0.4 0.6 0.8 1.0 / /
  \plotheading{Distributions differ}
% Estimates of two different cumulative distributions
% X is binomial(10,0.45), Y is binomial(10,0.55)
\multiput {$\bullet$} at
  0.00   0.00
  0.02   0.00
  0.10   0.03
  0.26   0.10
  0.50   0.26
  0.73   0.50
  0.89   0.74
  0.97   0.90
  1.00   0.98
  1.00   1.00
  1.00   1.00
/
%
\setdots
\plot 0 0 1 1 /
\setsolid
%%%%%%%%%%%%%%%%% Right Plot %%%%%%%%%%%%%%%%%%%%%%%%%%%%%%%%%%%%%%%%%%%
\setcoordinatesystem units <0.3\textwidth,0.3\textwidth> point at -0.2 0
  \setplotarea x from 0 to 1, y from 0 to 1
  \axis bottom label {Quantiles of $X$} 
   ticks withvalues 0 {2000} {4000} {6000} {8000} {10000} / 
   at 0 0.2 0.4 0.6 0.8 1.0 / /
  \axis right 
   ticks withvalues 0 {2000} {4000} {6000} {8000} {10000} / 
   at 0 0.2 0.4 0.6 0.8 1.0 / /
  \plotheading{Distributions identical}
% Two estimates of same cumulative distribution, binomial(10,0.4)
\multiput {$\bullet$} at
  0.01   0.01
  0.05   0.05
  0.17   0.17
  0.39   0.38
  0.63   0.63
  0.83   0.84
  0.95   0.94
  0.99   0.99
  1.00   1.00
  1.00   1.00
  1.00   1.00
/
\setdots
\plot 0 0 1 1 /
\setsolid
\endpicture}
\end{center}
\caption{Using quantile-quantile plots to compare distributions.  The
  left side compares two slightly different binomial distributions,
  with $(N,p)=(10,0.45)$ in one case and $(10,0.55)$ in the other.
  The right side compares two independent binomial samples, both with
  $(N,p)=(10,0.4)$.  In all cases, 10000 binomial random variates were
  drawn.}
\label{fig.qqplot}
\end{figure}
