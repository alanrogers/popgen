\chapter{Using Python}
In this lab, you will need to know about
\begin{itemize}
\item how to use a text editor
\item genetic concepts: ``genotype,'' ``allele,'' ``locus,''
  ``heterozygosity''  
\item Python concepts: list, string, for, if, assignment
  statements, print, float 
\end{itemize}
Python's on-line documentation is so good that you don't need to buy a
book.\footnote{If you really want one, condsider \emph{Practical
    Python}, by Magnus Lie Hetland, or \emph{Learning Python}, by
  Mark Lutz and David Ascher.}  You will find everything you need at
\url{http://docs.python.org}.  Point your browser there and follow the
link to ``Tutorial.''  There is also a ``Library Reference''
that you will use all the time after you know the basics.

During the first few weeks of the course, you should read
sections~1--5 and 7 of the tutorial.  For now, just read
sections~1--3.  If you run into an unfamiliar concept, use Google to
look it up.  For example, try typing the following into Google's
search window:
\begin{verbatim}
string site:docs.python.org/tut
\end{verbatim}

As you read the tutorial, be sure to experiment by using the
interpreter interactively.  The procedure for starting the interpreter
differs on Windows, Mac, and linux.  On Mac OS10 and linux, you open a
window to the command interpreter (or shell) and type
\verb|python|.  On Windows, you should be able to do the same from the
DOS prompt.  

Use a text editor to create a file named ``heterozygosity.py,''
  which consists of the following lines:
\newpage
\begin{verbatim}
data = ['AA', 'AA', 'AT', 'TA', 'TA', 'AA', 'TA']

heterozygosity = 0
for genotype in data:
    if genotype[0] != genotype[1]:
        heterozygosity += 1
heterozygosity /= float(len(data))
print("Heterozygosity:", heterozygosity, "N:", len(data))
\end{verbatim}
The first line defines a variable called ``data,'' which contains a
dummy data set: a list of genotypes.  You will soon encounter real
data in exactly this format.  Each entry in the list represents the
genotype of an individual at some hypothetical nucleotide site.

These lines of code are a Python program that calculates the
heterozygosity of the individuals in the data.  (The term
``heterozygosity'' is defined in the text.)

Once you have got the program typed in, the next step is to run it.
At the command prompt, type ``python heterozygosity.py''.  If all goes
well, you should see
\begin{verbatim}
Heterozygosity: 0.571428571429 N: 7
\end{verbatim}
This says that about 57\% of the sample is heterozygous, and there are
7 individuals in the sample.

Next, figure out how the program works.  Read it carefully.  Look up
everything you don't understand in the Tutorial.

\subsection*{Exercise}
\begin{itemize}
\item Use your text editor to put comments into the code,
explaining what each line does.
\item Change the program to incorporate the following data:
\begin{verbatim}
data = ['TA', 'AT', 'TA', TA', 'TA', 'TA', 'AA', 'TA']
\end{verbatim}
How does the program behave now?
\end{itemize}
