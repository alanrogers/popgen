\chapter{The Site Frequency Spectrum of HapMap SNPs}
\label{ch.hapspec}
\setcounter{cenumEnumi}{0}
\section{Introduction}
This project will use the HapMap data, which you first encountered in
Project~\ref{ch.hapintro}.  (Please do that project before this one.)
The current project concerns the \emph{site frequency spectrum}, which
was discussed in lecture and in chapters~\ref{L-ch.descrip}
and~\ref{L-ch.spectrum} of \emph{Lecture Notes on Gene Genealogies}.
As those chapters explained, the spectrum has a very simple shape---at
least in expectation---under the conditions of the standard neutral
model.  In this project you will estimate the site frequency spectrum
from an entire human chromosome and compare this empirical
distribution to its theoretical counterpart.  The goal is to see how
well the HapMap SNPs conform to the assumptions of the standard
neutral model.

As explained in \emph{Lecture Notes on Gene Genealogies}, the site
frequency spectrum may be either ``unfolded'' or ``folded.'' In an
unfolded spectrum, the $i$th entry of the spectrum is the number of
sites at which the derived allele is present in $i$ copies, where $i
\in \{1,2,\ldots,K-1\}$, and $K$ is the haploid sample size (twice the
number of diploid individuals). In this project, we will work instead
with the folded spectrum, which counts the number of copies of the
``minor'' allele---the rarer of the two alleles at the locus.

\section{The project}
\label{sec.proj}
On the course website, you will find an incomplete program called 
\texttt{hapspecinc.py}, which looks like this:
\listinginput[3]{1}{hapspecinc.py} 
Lines~1--8 use methods that we've seen before to create \texttt{hds},
an object of type \verb|hapmap_dataset|. Beginning on line~14, we'll
be iterating across all the nucleotide sites in this data set. But
first we need to set up an array to hold the site frequency spectrum.

That is the purpose of line 11, which defines a list called
\texttt{ospec}. In that line, \texttt{K//2} divides $K$ by 2 and then
truncates the fractional part of the answer, so that we end with an
integer. This value is of interest, because the number of copies of
the minor allele cannot exceed \texttt{K//2}, no matter whether $K$ is
odd or even. For example, if $K=5$, \texttt{K//2} is 2. If $K=4$,
\texttt{K//2} is also 2. In either case, 2 is the largest possible
number of copies of the minor allele. Thus, the number of copies of
the minor allele is no larger than \texttt{K//2}, no matter whether
$K$ is odd or even.

Line~11 of the code above creates a list of length \texttt{K//2 +
  1}. What is the point of adding 1? I did this because the initial
index of a Python array is 0 rather than 1. I am allocating a list
that is 1 element larger than I really need, so that I can use
\texttt{ospec[1]} for sites with 1 copy of the minor allele,
\texttt{ospec[2]} for sites with 2 copies, and so on. I will simply
ignore \texttt{ospec[0]}. In the end, \texttt{ospec[i]} will refer to
  the number of sites at which the minor allele is present in
  \texttt{i} copies. 

Lines 14--19 define a loop which runs across all sites in the
chromosome. The first step is to set \texttt{x} equal to the number of
copies of the minor allele. This should be done by replacing line~16
with code that sets \texttt{x} equal to the number of copies of the
minor allele. Then there are two \texttt{assert} statements, which
make sure that \texttt{x} has a sensible value. If either of the
conditions defined in these statements is false, the program will
abort with an error message. Finally, line~19 adds one SNP to the
relevent entry of \texttt{ospec}.

Beginning in line~21, the indentation indicates that we are out of the
loop. There are two remaining steps that you need to add. First create
a list called \texttt{espec}, whose $i$th entry is the expected number
of sites at which the minor allele is present in $i$ copies. Then use
the \texttt{oehist} method (a part of \texttt{pgen.py}) to compare the
observed and expected spectra.

There are thus three changes that you need to make to the code, and
we'll discuss those in the sections that follow.

\subsection{Calculating the frequency of the minor allele}
As you know, each object of type \texttt{hapmap\_snp} describes one
HapMap SNP, including the genotypes of all individuals sampled.  On
page~\pageref{pg.countgene}, we explained how to count the copies of
allele~0.  Here we extend that code to count the copies of the
\emph{minor} allele---the rarer of the two. Consider the following
snippet:
\begin{leftindent}
\begin{verbatim}
x = sum(snp)     # copies of allele 0
if x > K//2:
    x = K - x
\end{verbatim}
\end{leftindent}
Here, \texttt{snp} is an object of type \texttt{hapmap\_snp}.  The
\texttt{sum} statement (explained on p.~\pageref{pg.countgene}) counts
copies of allele~0, and the last two lines convert this into a count
of the minor allele.  The code works because the number of copies of
the minor allele cannot exceed $K//2$.

\subsection{The expectation of the folded spectrum}
Our goal is to compare the observed spectrum with its expected value
under the standard model of selective neutrality, random mating, and
constant population size. Let us refer to the $i$th entry of the
folded spectrum as $s_i$.  This is is the number of sites at which the
minor allele is present in $i$ copies within the data.  There is a
well-known formula for the expected value of $s_i$ under the standard
neutral model.  (See sections~\ref{L-sec.thspec-nleaves}
and~\ref{L-sec.foldedspectrum} of \emph{Lecture Notes on Gene
  Genealogies}.)
\[
E[s_i] = \cases{\theta / i              & \hbox{if $i = K-i$}\cr
                \theta/i + \theta/(K-i) & \hbox{otherwise} \cr}
\]
where $\theta = 4Nu$, $2N$ is the number of gene copies in the
population, and $u$ is the mutation rate.  To calculate this value,
you need an estimate of $\theta$.  In the lecture notes, we suggested
basing this estimate on $S$, the number of segregating sites.  The
appropriate estimate is
\[
\hat\theta = S/a \qquad\hbox{where}\qquad
a = \sum_{i=1}^{K-1} 1/i
\]
and $K$ is the number of gene copies in the sample.

Let us implement these formulas in Python.  There are two easy ways to
calculate the number of segregating sites in your data.  Suppose that
your observed spectrum is a Python list named \texttt{ospec}.  Then
\verb|S = sum(ospec}|, or alternatively, \verb|S = len(hds)|.  These
two approaches should give the same answer.  (Check this, just to make
sure that your code is correct.) There are also several ways to
calculate $a$.  Here is one:
\begin{leftindent}
\begin{verbatim}
a = sum([1/i for i in range(1,K)])
\end{verbatim}
\end{leftindent}
Given $S$ and $a$, it is a simple matter to calculate $\hat\theta$ and
$E[s_i]$, using the formulas above.  Here is one way to do it:
\begin{leftindent}
\begin{verbatim}
espec = (K//2+1)*[0]
for i in range(1,len(espec)):
    if i == K-i:
        espec[i] = theta/i
    else:    
        espec[i] = theta/i + theta/(K-i)
\end{verbatim}
\end{leftindent}
This generates a list called \texttt{espec}, such that
\texttt{espec[i]} is the expected value of \texttt{ospec[i]} under the
standard neutral model.

\subsection{Using \texttt{oehist} to compare observed and expected spectra}
Finally, you will want to compare the observed and expected values
contained in \texttt{ospec} and \texttt{espec}.  \texttt{pgen.py}
contains a function called \texttt{oehist} (for observed-expected
histogram), which does this for you.  To use \texttt{oehist}, add the
following line to the end of your program:
\begin{leftindent}
\begin{verbatim}
oehist(ospec, espec)
\end{verbatim}
\end{leftindent}
By default, this prints a line for the 0th position of the two lists,
which isn't used.  The easy solution is to ignore this line of
output. If you want to supress it, the \texttt{oehist} command should
look like:
\begin{leftindent}
\begin{verbatim}
oehist(ospec[1:], espec[1:], range(1,K//2+1))
\end{verbatim}
\end{leftindent}
The optional third argument of oehist specifies the X-axis values that
correspond to the entries of the lists provided in the first two
arguments. 

\section*{Exercise}
\begin{cenum}
\item Modify the code above in order to calculate the observed folded
  spectrum, \texttt{ospec}.
\item Add code to calculate \texttt{espec}, the expected folded
  spectrum under the standard neutral model.
\item Add code that uses \texttt{oehist} to compare \texttt{ospec} and
  \texttt{espec}. 
\item Write a short paragraph commenting on the observed and expected
  spectra.  Do they seem approximately the same, or do they differ?
  If they differ, summarize how they differ.  What might account for
  the discrepancies?
\end{cenum}
As you contemplate the last item in the exercise above, bear in mind
that our data (provided by HapMap) are not a random sample of the
genome. When nucleotide sites were chosen for inclusion in HapMap,
preference was given to sites with high heterozygosity.

\paragraph*{What to turn in} The program incorporating steps~1--3, and
a short paragraph responding to step~4.
