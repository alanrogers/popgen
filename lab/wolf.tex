\chapter{Were Wolf's dice fair? An approach using the variance.}
\label{ch.wolf}
This project expands on the one in section 3.5 of JEPy. It makes use
of material that is explained in sec.~4 of JEPy, which you should read.

\begin{table}
\renewcommand{\tabcolsep}{3pt}
\begin{center}
\renewcommand{\arraystretch}{0.95}
\begin{tabular}{rrrrrrrr} 
    & \multicolumn{6}{c}{White}&\\ \cline{2-7}
 Red & 1 & 2 & 3 & 4 & 5 & 6 & sum\\ 
\hline
\hline
    1     & 547 & 587 & 500 & 462 & 621 & 690 & 3407\\
    2     & 609 & 655 & 497 & 535 & 651 & 684 & 3631\\
    3     & 514 & 540 & 468 & 438 & 587 & 629 & 3176\\
    4     & 462 & 507 & 414 & 413 & 509 & 611 & 2916\\
    5     & 551 & 562 & 499 & 506 & 658 & 672 & 3448\\
    6     & 563 & 598 & 519 & 487 & 609 & 646 & 3422\\ \hline
sum:     & 3246 & 3449 & 2897 & 2841 & 3635 & 3932 & 20000
\end{tabular}
\end{center}
\caption{The results of 20,000 throws with two dice (Wolf 1850, cited
  in JEPr)}
\label{tab.dice}
\end{table}

We are interested in the results from Wolf's experiment, which is
described in JEPr. Wolf tossed a pair of dice 20,000 times and
obtained the results shown in table~\ref{tab.dice}.  If Wolf's dice
were fair, each face of the die should appear on about $1/6$ of the
20,000 rolls, or 3333 times. There is no reason to expect Wolf's data
to agree exactly with this expectation, because after all, dice are
random. But several of Wolf's marginal counts do look suspiciously far
from 3333. The question is, are these discrepancies to larger than the
ought to be if Wolf's dice were fair?

In a previous lab (JEPy section~3.5), you addressed this problem by
tabulating counts. Here, we take a different approach. Consider what
we might expect of an unfair die. With such a die, some faces appear
frequently and others rarely. There should therefore be large
differences between the counts for different faces. We can measure
this effect by calculating the variance of the counts for the 6 faces
of each die.

For example, the counts for the white die are 3246, 3449, 2897, 2841,
3635, and 3932.  The variance of these numbers is 150531.56. If Wolf's
white die were fair, we should expect to see comparable values fairly
often in experiments with fair dice. But if Wolf's die were unfair, his
variance should turn out to be unusually large.

To study this question, you will modify the Python program shown
below.  This code is available on the class web site, where it is
called \texttt{wolfinc.py}.  To download it, find it on the website,
right-click, and select ``save as.'' Then run it to see what it does.

Rather than downloading from the website, you can also cut and paste
from the text below. But there is danger here. When you cut and paste,
you may end up with a file in which characters have been replaced by
other characters that look similar. The document will look the same
but may not execute.  Cut-and-paste usually works, but it is not
reliable. To be safe, we recommend downloading files from the website.

\begin{leftindent}
\listinginput[5]{1}{wolfinc.py}
\end{leftindent}
When you run this program, you should get 12 lines of output. The
first two report the variances of Wolf's two dice. The next 10 show
the results of 10 simulated tosses of a fair die.

The first line of this program makes the ``random'' function (a member
of the ``random'' module) available to our program. Lines~3--12 define
a function that calculates the variance of a list of
numbers. Sec.~\ref{Pr-sec.variance} of JEPr explains variance, and
sec.~4.4 of JEPy describes Python functions. Lines~14--15 define two
Python lists, which contain the counts from both of Wolf's
dice. Lines~17-18 calculate the variances of these lists, using the
``var'' function defined above. Line~23 defines a variable called
``\texttt{nreps},'' which is short for ``number of repetitions.'' This
is the number of times we will repeat Wolf's experiment. Initially,
\texttt{nreps} is only 10, because we don't want to drown in
output. You will eventually want to make this number larger, in order
to get accurate answers. 

Lines~25--33 define a loop, which is executed
repeatedly---\texttt{nreps} times---once for each repetition of Wolf's
experiment. The loop is controlled by the first line, which uses
\texttt{for} and \texttt{range}.  These are explained in JEPy. Within
the loop, line~26 defines and initializes a list called
\texttt{count}, which will keep track of the number of times each face
of the die appears. Lines~28--30 do Wolf's experiment, and line~32
calculates the variance. Finally, line~33 prints a line of
output. This line is there only so that the loop will generate output
when you first run the program. You'll want to remove this line when
you get your modified code working.

\subsubsection*{Exercise}
\begin{enumerate}
\item Add a variable called \verb|tail_r| to the program. This
  variable should initially equal zero. Each time through the loop, if
  \texttt{v} is greater than or equal to \texttt{Vr}, then add 1 to
  \verb|tail_r|. Thus, \verb|tail_r| will count the number of
  repetitions in which the simulated variance is at least as large as
  the observed one. Before you modify the code, think carefully. Where
  should \verb|tail_r| be defined and initialized? Inside the inner
  loop? Before the inner loop but inside the outer one? Or before both
  loops?
\item Add another variable, \verb|tail_w|, to keep track of how often
  the simulated variance is at least as large as \texttt{Vw}.
\item At the end of the program, and outside of the \texttt{for} loop,
  print two lines of output, one telling us what \emph{fraction} of
  the time the simulated variance exceeded \texttt{Vr} and the other
  doing the same for \texttt{Vw}. The print statements should include
  text that explains what is being printed.
\item Once this works, remove the \texttt{REMOVE ME} line, and
  increase \texttt{nreps} first to 100 (make sure that works) and then
  to 1000. If you go much over 1000, your program will take a long
  time to run.
\item Use the results to figure out whether either of Wolf's dice were
  unfair.
\end{enumerate}
Hand in your computer program, its output, and a paragraph explaining
your conclusions.


  
