\section{Smoothing and graphing data}
In this lab, you will be comparing LD between tens of thousands of
pairs of loci.  Without some way of simplifying the output, you would
drown in data.  It will help to graph the data rather than just
staring at lists of numbers.  Below, you will use a function called
\verb|charplot| (available in the \texttt{pgen.py} module) to plot LD
against distance along the chromosome.  This will help, but it is not
enough---the noise in the LD estimates would still obscure the
pattern.  We can get rid of much of this noise by \emph{smoothing} the
data.  This is the purpose of yet another function within
\texttt{pgen.py}, called \texttt{expmovav}.

We'll get to \texttt{expmovav} in a moment, but let's begin with
\texttt{charplot}.  Here is a listing that shows how to use it:
\begin{verbatim}
from pgen import charplot

x = [0, 1, 2, 3, 4, 5, 6, 7, 8, 9]
y = [0, 1, 4, 9, 16, 25, 36, 49, 64, 81]

charplot(x, y)
\end{verbatim}
When this program runs, it makes a scatter plot of the $(x_i, y_i)$
values, with $x$ on the horizontal axis and $y$ on the vertical.  Try
this yourself.

%\begin{figure}
%\tiny
%\centering
%\begin{minipage}{6cm}
%\begin{verbatim}
% 80 +------------+-----------+------------+------*
%    |                                            |
%    |                                            |
%    |                                            |
% 60 +                                       *    +
%    |                                            |
%    |                                            |
%    |                                  *         |
%    |                                            |
% 40 +                             *              +
%    |                                            |
%    |                                            |
%    |                        *                   |
% 20 +                                            +
%    |                   *                        |
%    |              *                             |
%    |         *                                  |
%  0 *----*-------+-----------+------------+------+
%   0.0          2.5         5.0          7.5
%\end{verbatim}
%\end{minipage}
%\caption{Example \texttt{charplot} graphics}
%\label{fig.charplot}
%\end{figure}

%\begin{figure}
%\tiny
%\centering
%\begin{minipage}{6cm}
%\begin{verbatim}
%>>> from pgen import expmovav, charplot
%>>> charplot(x,y)
% 2.0 +---------*+*--------+----------+---------+-+
%     |           *                               |
%     |        *    *                             |
% 1.5 +    ***** *                                +
%     |       *  *   * * *                        |
%     |* *       * *    * *                       |
% 1.0 +  ** *   *   *****                         +
%     *** *  *       **     **                   *|
%     |   *                   *                 * *
% 0.5 +                  *  **  *              ** +
%     | *                 **                  ** **
%     *                        **                 |
% 0.0 +                          *  *       ***   +
%     |                       *    * *     *      |
%     |                        * ***   ** *   *   |
%-0.5 +                         * * *     *       +
%     |                               ** * **     |
%     |                                ***        |
%-1.0 +----------+---------+----------*---------+-+
%    0.0        1.5       3.0        4.5       6.0
%\end{verbatim}
%\end{minipage}~~~~~~~~\begin{minipage}{6cm}
%\begin{verbatim}
%>>> smooth = expmovav(y)
%>>> charplot(x,smooth)
% 1.5 +--------**+*--------+----------+---------+-+
%     |      *****                                |
%     |    ***     **                             |
%     |    **       **                            |
%     |   *          ****                         |
% 1.0 +  **             **                        +
%     |**                **                       |
%     |**                 *                       |
%     *                    ***                    |
% 0.5 +                     **                  **+
%     *                       *                ****
%     |                                        *  |
%     |                       ***             *   |
% 0.0 +                        *             **   +
%     |                         **          **    |
%     |                          ****      **     |
%     |                             **    **      |
%-0.5 +                               *** *       +
%     +----------+---------+----------****------+--
%    0.0        1.5       3.0        4.5       6.0
%\end{verbatim}
%\end{minipage}
%\caption{Using \texttt{expmovav} and \texttt{charplot} to smooth and
%  plot data}
%\label{fig.charplot}
%\end{figure}

In that example, the data were very smooth and the pattern easy to
see.  Let us construct a noisier data set.  The following snippet of
Python code will build two Python lists, \texttt{x} and \texttt{y}.
\begin{listing}{1}
>>> from math import sin, pi
>>> from random import random
>>> from pgen import charplot, expmovav
>>> x = [0.02*pi*i for i in range(0,101)]
>>> y = [sin(xx) + 2*random() for xx in x]
\end{listing}
\texttt{x} ranges from 0 to $2\pi$.  Each value within \texttt{y}
contains the sum of $\sin x$ and some random noise.  If you didn't
follow the steps in this listing, don't worry.  Just type it in, and
then use \texttt{charplot} as shown below to plot \texttt{y} against
\texttt{x}:\newline
\begin{listing}{6}
>>> charplot(x,y)
\end{listing}
\begin{minipage}{\columnwidth}
\begin{listing}{7}
 3 +--------*-+----------+----------+----------+-+
   |       *  **   *                             |
   |    ***  *   ** *                            |
 2 +   *    *  **  **    *                       +
   ** * **           **                          |
   | *            *    ** *                  *   |
   * *   *   ** *   *   *  *  **                 *
 1 +* *   **     *         * *  ***           * *+
   |   *              *  ** *             ** * * |
   |                   *      * **  ***          |
 0 +                      *    **      * *  *  * *
   |                        **     *     * *    *|
   |                              * *** * * * *  |
   |                               *             |
-1 +----------+----------+----------+*--*------+-+
  0.0        1.5        3.0        4.5        6.0
\end{listing}
\end{minipage}\\[7pt]
You can see the wave of the sine curve, but only barely.  Let us use
\texttt{expmovav} to create a smoothed version of \texttt{y} and then
plot that.  The next listing shows how:
\begin{listing}{26}
>>> ysmooth = expmovav(y)
>>> charplot(x,ysmooth)
     +--------*-**--------+----------+---------+--
 2.0 +       * ******                            +
     |   ** *  ** * ***                          |
     |   * **      *   *                         |
 1.5 +  **             ****                      +
     |***               **                       |
 1.0 *                    **                     +
     *                     **                    |
     |                     * *****           *   |
 0.5 +                       ******          *****
     |                            *         ******
 0.0 +                             **  *  ***    +
     |                             ********      |
     |                               ** **       |
-0.5 +----------+---------+----------+---------+-+
    0.0        1.5       3.0        4.5       6.0
\end{listing}
Now the sine wave is much easier to see.  Experiment with
\texttt{charplot} and \texttt{expmovav} until you understand how they
work.
